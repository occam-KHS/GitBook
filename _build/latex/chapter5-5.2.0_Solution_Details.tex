%% Generated by Sphinx.
\def\sphinxdocclass{jupyterBook}
\documentclass[letterpaper,10pt,english]{jupyterBook}
\ifdefined\pdfpxdimen
   \let\sphinxpxdimen\pdfpxdimen\else\newdimen\sphinxpxdimen
\fi \sphinxpxdimen=.75bp\relax
\ifdefined\pdfimageresolution
    \pdfimageresolution= \numexpr \dimexpr1in\relax/\sphinxpxdimen\relax
\fi
%% let collapsible pdf bookmarks panel have high depth per default
\PassOptionsToPackage{bookmarksdepth=5}{hyperref}
%% turn off hyperref patch of \index as sphinx.xdy xindy module takes care of
%% suitable \hyperpage mark-up, working around hyperref-xindy incompatibility
\PassOptionsToPackage{hyperindex=false}{hyperref}
%% memoir class requires extra handling
\makeatletter\@ifclassloaded{memoir}
{\ifdefined\memhyperindexfalse\memhyperindexfalse\fi}{}\makeatother

\PassOptionsToPackage{warn}{textcomp}

\catcode`^^^^00a0\active\protected\def^^^^00a0{\leavevmode\nobreak\ }
\usepackage{cmap}
\usepackage{fontspec}
\defaultfontfeatures[\rmfamily,\sffamily,\ttfamily]{}
\usepackage{amsmath,amssymb,amstext}
\usepackage{polyglossia}
\setmainlanguage{english}



\setmainfont{FreeSerif}[
  Extension      = .otf,
  UprightFont    = *,
  ItalicFont     = *Italic,
  BoldFont       = *Bold,
  BoldItalicFont = *BoldItalic
]
\setsansfont{FreeSans}[
  Extension      = .otf,
  UprightFont    = *,
  ItalicFont     = *Oblique,
  BoldFont       = *Bold,
  BoldItalicFont = *BoldOblique,
]
\setmonofont{FreeMono}[
  Extension      = .otf,
  UprightFont    = *,
  ItalicFont     = *Oblique,
  BoldFont       = *Bold,
  BoldItalicFont = *BoldOblique,
]



\usepackage[Bjarne]{fncychap}
\usepackage[,numfigreset=1,mathnumfig]{sphinx}

\fvset{fontsize=\small}
\usepackage{geometry}


% Include hyperref last.
\usepackage{hyperref}
% Fix anchor placement for figures with captions.
\usepackage{hypcap}% it must be loaded after hyperref.
% Set up styles of URL: it should be placed after hyperref.
\urlstyle{same}


\usepackage{sphinxmessages}



        % Start of preamble defined in sphinx-jupyterbook-latex %
         \usepackage[Latin,Greek]{ucharclasses}
        \usepackage{unicode-math}
        % fixing title of the toc
        \addto\captionsenglish{\renewcommand{\contentsname}{Contents}}
        \hypersetup{
            pdfencoding=auto,
            psdextra
        }
        % End of preamble defined in sphinx-jupyterbook-latex %
        

\title{해결책 개발}
\date{Jul 02, 2022}
\release{}
\author{KHS}
\newcommand{\sphinxlogo}{\vbox{}}
\renewcommand{\releasename}{}
\makeindex
\begin{document}

\pagestyle{empty}
\sphinxmaketitle
\pagestyle{plain}
\sphinxtableofcontents
\pagestyle{normal}
\phantomsection\label{\detokenize{chapter5/5.2.0_Solution_Details::doc}}


\sphinxAtStartPar
가설 검증을 통하여 각 가설이 유의미한지 알아보았습니다. 가설 검정과정에서 얻은 지식을 이용하여  종목을 찾고, 매매 수익을 실현하고 싶습니다. 하지만 유미의한 가설들을 동시에 활용하여 종목을 뽑아내기는 쉬운 일이 아닙니다.  단지 2 개의 가설을 동시에 고려하는 것도 어렵습니다.  이것을 가능하게 해 주는 것이 예측 모델인데요. 가설들이 입력변수가 되는 예측 모델을 만들어 보겠습니다.

\sphinxAtStartPar
매수 후 5 영업일 이내 주가 상승 여부를 알 수 있는 예측력이 좋은 모델이 개발되면, 매일 모델을 실행시켜 종목 추천을 모델로 부터 받을 것입니다. 정규장이 3시 30에 종료되므로 3시 30분에 모델을 돌리고 종목 추천을 받겠습니다. 그리고 익일 정해진 지정가에 매수를 하겠습니다. 매수와 매도는 자동매매로 구현해볼 계획입니다.

\begin{sphinxuseclass}{cell}\begin{sphinxVerbatimInput}

\begin{sphinxuseclass}{cell_input}
\begin{sphinxVerbatim}[commandchars=\\\{\}]
\PYG{k+kn}{import} \PYG{n+nn}{FinanceDataReader} \PYG{k}{as} \PYG{n+nn}{fdr}
\PYG{o}{\PYGZpc{}}\PYG{k}{matplotlib} inline
\PYG{k+kn}{import} \PYG{n+nn}{matplotlib}\PYG{n+nn}{.}\PYG{n+nn}{pyplot} \PYG{k}{as} \PYG{n+nn}{plt}
\PYG{k+kn}{import} \PYG{n+nn}{pandas} \PYG{k}{as} \PYG{n+nn}{pd}
\PYG{k+kn}{import} \PYG{n+nn}{numpy} \PYG{k}{as} \PYG{n+nn}{np}

\PYG{n}{pd}\PYG{o}{.}\PYG{n}{options}\PYG{o}{.}\PYG{n}{display}\PYG{o}{.}\PYG{n}{float\PYGZus{}format} \PYG{o}{=} \PYG{l+s+s1}{\PYGZsq{}}\PYG{l+s+si}{\PYGZob{}:,.3f\PYGZcb{}}\PYG{l+s+s1}{\PYGZsq{}}\PYG{o}{.}\PYG{n}{format}
\end{sphinxVerbatim}

\end{sphinxuseclass}\end{sphinxVerbatimInput}

\end{sphinxuseclass}

\part{피처 엔지니어링}
\label{\detokenize{chapter5/5.2.1_Feature_Engineering:id1}}\label{\detokenize{chapter5/5.2.1_Feature_Engineering::doc}}
\sphinxAtStartPar
가설 검정에서 만들었던 모든 피쳐(변수)를 정리해 보겠습니다. 이제 예측 모델링을 위한 데이터가 준비되었습니다. 예측모델링에 활용한 데이터의 기간은 2021년 1월 5일부터 2022년 3월 24일까지입니다.

\begin{sphinxuseclass}{cell}\begin{sphinxVerbatimInput}

\begin{sphinxuseclass}{cell_input}
\begin{sphinxVerbatim}[commandchars=\\\{\}]
\PYG{n}{mdl\PYGZus{}data} \PYG{o}{=} \PYG{n}{pd}\PYG{o}{.}\PYG{n}{read\PYGZus{}pickle}\PYG{p}{(}\PYG{l+s+s1}{\PYGZsq{}}\PYG{l+s+s1}{mdl\PYGZus{}data.pkl}\PYG{l+s+s1}{\PYGZsq{}}\PYG{p}{)} \PYG{c+c1}{\PYGZsh{} 수익률 결과값이 있는 데이터}
\PYG{n}{mdl\PYGZus{}data}\PYG{o}{.}\PYG{n}{head}\PYG{p}{(}\PYG{p}{)}
\PYG{n+nb}{print}\PYG{p}{(}\PYG{n}{mdl\PYGZus{}data}\PYG{o}{.}\PYG{n}{index}\PYG{o}{.}\PYG{n}{min}\PYG{p}{(}\PYG{p}{)}\PYG{p}{,} \PYG{n}{mdl\PYGZus{}data}\PYG{o}{.}\PYG{n}{index}\PYG{o}{.}\PYG{n}{max}\PYG{p}{(}\PYG{p}{)}\PYG{p}{)}
\end{sphinxVerbatim}

\end{sphinxuseclass}\end{sphinxVerbatimInput}
\begin{sphinxVerbatimOutput}

\begin{sphinxuseclass}{cell_output}
\begin{sphinxVerbatim}[commandchars=\\\{\}]
2021\PYGZhy{}01\PYGZhy{}05 2022\PYGZhy{}03\PYGZhy{}24
\end{sphinxVerbatim}

\end{sphinxuseclass}\end{sphinxVerbatimOutput}

\end{sphinxuseclass}
\sphinxAtStartPar
 가설검정에서 만들었던 모든 피쳐를 정리합니다. 단, \sphinxstyleemphasis{“5일 이동평균선이 종가보다 위에 있다”} 는 유의미하지 않았으므로 제외입니다. 결과를 feature\_all 이라는 데이터프레임에 저장합니다.

\begin{sphinxuseclass}{cell}\begin{sphinxVerbatimInput}

\begin{sphinxuseclass}{cell_input}
\begin{sphinxVerbatim}[commandchars=\\\{\}]
\PYG{n}{kosdaq\PYGZus{}list} \PYG{o}{=} \PYG{n}{pd}\PYG{o}{.}\PYG{n}{read\PYGZus{}pickle}\PYG{p}{(}\PYG{l+s+s1}{\PYGZsq{}}\PYG{l+s+s1}{kosdaq\PYGZus{}list.pkl}\PYG{l+s+s1}{\PYGZsq{}}\PYG{p}{)}

\PYG{n}{feature\PYGZus{}all} \PYG{o}{=} \PYG{n}{pd}\PYG{o}{.}\PYG{n}{DataFrame}\PYG{p}{(}\PYG{p}{)}

\PYG{k}{for} \PYG{n}{code}\PYG{p}{,} \PYG{n}{sector} \PYG{o+ow}{in} \PYG{n+nb}{zip}\PYG{p}{(}\PYG{n}{kosdaq\PYGZus{}list}\PYG{p}{[}\PYG{l+s+s1}{\PYGZsq{}}\PYG{l+s+s1}{code}\PYG{l+s+s1}{\PYGZsq{}}\PYG{p}{]}\PYG{p}{,} \PYG{n}{kosdaq\PYGZus{}list}\PYG{p}{[}\PYG{l+s+s1}{\PYGZsq{}}\PYG{l+s+s1}{sector}\PYG{l+s+s1}{\PYGZsq{}}\PYG{p}{]}\PYG{p}{)}\PYG{p}{:}

    \PYG{n}{data} \PYG{o}{=} \PYG{n}{mdl\PYGZus{}data}\PYG{p}{[}\PYG{n}{mdl\PYGZus{}data}\PYG{p}{[}\PYG{l+s+s1}{\PYGZsq{}}\PYG{l+s+s1}{code}\PYG{l+s+s1}{\PYGZsq{}}\PYG{p}{]}\PYG{o}{==}\PYG{n}{code}\PYG{p}{]}\PYG{o}{.}\PYG{n}{sort\PYGZus{}index}\PYG{p}{(}\PYG{p}{)}\PYG{o}{.}\PYG{n}{copy}\PYG{p}{(}\PYG{p}{)}
    
    
    \PYG{c+c1}{\PYGZsh{} 가격변동성이 크고, 거래량이 몰린 종목이 주가가 상승한다}
    \PYG{n}{data}\PYG{p}{[}\PYG{l+s+s1}{\PYGZsq{}}\PYG{l+s+s1}{price\PYGZus{}mean}\PYG{l+s+s1}{\PYGZsq{}}\PYG{p}{]} \PYG{o}{=} \PYG{n}{data}\PYG{p}{[}\PYG{l+s+s1}{\PYGZsq{}}\PYG{l+s+s1}{close}\PYG{l+s+s1}{\PYGZsq{}}\PYG{p}{]}\PYG{o}{.}\PYG{n}{rolling}\PYG{p}{(}\PYG{l+m+mi}{20}\PYG{p}{)}\PYG{o}{.}\PYG{n}{mean}\PYG{p}{(}\PYG{p}{)}
    \PYG{n}{data}\PYG{p}{[}\PYG{l+s+s1}{\PYGZsq{}}\PYG{l+s+s1}{price\PYGZus{}std}\PYG{l+s+s1}{\PYGZsq{}}\PYG{p}{]} \PYG{o}{=} \PYG{n}{data}\PYG{p}{[}\PYG{l+s+s1}{\PYGZsq{}}\PYG{l+s+s1}{close}\PYG{l+s+s1}{\PYGZsq{}}\PYG{p}{]}\PYG{o}{.}\PYG{n}{rolling}\PYG{p}{(}\PYG{l+m+mi}{20}\PYG{p}{)}\PYG{o}{.}\PYG{n}{std}\PYG{p}{(}\PYG{n}{ddof}\PYG{o}{=}\PYG{l+m+mi}{0}\PYG{p}{)}
    \PYG{n}{data}\PYG{p}{[}\PYG{l+s+s1}{\PYGZsq{}}\PYG{l+s+s1}{price\PYGZus{}z}\PYG{l+s+s1}{\PYGZsq{}}\PYG{p}{]} \PYG{o}{=} \PYG{p}{(}\PYG{n}{data}\PYG{p}{[}\PYG{l+s+s1}{\PYGZsq{}}\PYG{l+s+s1}{close}\PYG{l+s+s1}{\PYGZsq{}}\PYG{p}{]} \PYG{o}{\PYGZhy{}} \PYG{n}{data}\PYG{p}{[}\PYG{l+s+s1}{\PYGZsq{}}\PYG{l+s+s1}{price\PYGZus{}mean}\PYG{l+s+s1}{\PYGZsq{}}\PYG{p}{]}\PYG{p}{)}\PYG{o}{/}\PYG{n}{data}\PYG{p}{[}\PYG{l+s+s1}{\PYGZsq{}}\PYG{l+s+s1}{price\PYGZus{}std}\PYG{l+s+s1}{\PYGZsq{}}\PYG{p}{]}    
    \PYG{n}{data}\PYG{p}{[}\PYG{l+s+s1}{\PYGZsq{}}\PYG{l+s+s1}{volume\PYGZus{}mean}\PYG{l+s+s1}{\PYGZsq{}}\PYG{p}{]} \PYG{o}{=} \PYG{n}{data}\PYG{p}{[}\PYG{l+s+s1}{\PYGZsq{}}\PYG{l+s+s1}{volume}\PYG{l+s+s1}{\PYGZsq{}}\PYG{p}{]}\PYG{o}{.}\PYG{n}{rolling}\PYG{p}{(}\PYG{l+m+mi}{20}\PYG{p}{)}\PYG{o}{.}\PYG{n}{mean}\PYG{p}{(}\PYG{p}{)}
    \PYG{n}{data}\PYG{p}{[}\PYG{l+s+s1}{\PYGZsq{}}\PYG{l+s+s1}{volume\PYGZus{}std}\PYG{l+s+s1}{\PYGZsq{}}\PYG{p}{]} \PYG{o}{=} \PYG{n}{data}\PYG{p}{[}\PYG{l+s+s1}{\PYGZsq{}}\PYG{l+s+s1}{volume}\PYG{l+s+s1}{\PYGZsq{}}\PYG{p}{]}\PYG{o}{.}\PYG{n}{rolling}\PYG{p}{(}\PYG{l+m+mi}{20}\PYG{p}{)}\PYG{o}{.}\PYG{n}{std}\PYG{p}{(}\PYG{n}{ddof}\PYG{o}{=}\PYG{l+m+mi}{0}\PYG{p}{)}
    \PYG{n}{data}\PYG{p}{[}\PYG{l+s+s1}{\PYGZsq{}}\PYG{l+s+s1}{volume\PYGZus{}z}\PYG{l+s+s1}{\PYGZsq{}}\PYG{p}{]} \PYG{o}{=} \PYG{p}{(}\PYG{n}{data}\PYG{p}{[}\PYG{l+s+s1}{\PYGZsq{}}\PYG{l+s+s1}{volume}\PYG{l+s+s1}{\PYGZsq{}}\PYG{p}{]} \PYG{o}{\PYGZhy{}} \PYG{n}{data}\PYG{p}{[}\PYG{l+s+s1}{\PYGZsq{}}\PYG{l+s+s1}{volume\PYGZus{}mean}\PYG{l+s+s1}{\PYGZsq{}}\PYG{p}{]}\PYG{p}{)}\PYG{o}{/}\PYG{n}{data}\PYG{p}{[}\PYG{l+s+s1}{\PYGZsq{}}\PYG{l+s+s1}{volume\PYGZus{}std}\PYG{l+s+s1}{\PYGZsq{}}\PYG{p}{]}      
    
    
    \PYG{c+c1}{\PYGZsh{} 위꼬리가 긴 양봉이 자주 발생한다.}
    \PYG{n}{data}\PYG{p}{[}\PYG{l+s+s1}{\PYGZsq{}}\PYG{l+s+s1}{positive\PYGZus{}candle}\PYG{l+s+s1}{\PYGZsq{}}\PYG{p}{]} \PYG{o}{=} \PYG{p}{(}\PYG{n}{data}\PYG{p}{[}\PYG{l+s+s1}{\PYGZsq{}}\PYG{l+s+s1}{close}\PYG{l+s+s1}{\PYGZsq{}}\PYG{p}{]} \PYG{o}{\PYGZgt{}} \PYG{n}{data}\PYG{p}{[}\PYG{l+s+s1}{\PYGZsq{}}\PYG{l+s+s1}{open}\PYG{l+s+s1}{\PYGZsq{}}\PYG{p}{]}\PYG{p}{)}\PYG{o}{.}\PYG{n}{astype}\PYG{p}{(}\PYG{n+nb}{int}\PYG{p}{)} \PYG{c+c1}{\PYGZsh{} 양봉}
    \PYG{n}{data}\PYG{p}{[}\PYG{l+s+s1}{\PYGZsq{}}\PYG{l+s+s1}{high/close}\PYG{l+s+s1}{\PYGZsq{}}\PYG{p}{]} \PYG{o}{=} \PYG{p}{(}\PYG{n}{data}\PYG{p}{[}\PYG{l+s+s1}{\PYGZsq{}}\PYG{l+s+s1}{positive\PYGZus{}candle}\PYG{l+s+s1}{\PYGZsq{}}\PYG{p}{]}\PYG{o}{==}\PYG{l+m+mi}{1}\PYG{p}{)}\PYG{o}{*}\PYG{p}{(}\PYG{n}{data}\PYG{p}{[}\PYG{l+s+s1}{\PYGZsq{}}\PYG{l+s+s1}{high}\PYG{l+s+s1}{\PYGZsq{}}\PYG{p}{]}\PYG{o}{/}\PYG{n}{data}\PYG{p}{[}\PYG{l+s+s1}{\PYGZsq{}}\PYG{l+s+s1}{close}\PYG{l+s+s1}{\PYGZsq{}}\PYG{p}{]} \PYG{o}{\PYGZgt{}} \PYG{l+m+mf}{1.1}\PYG{p}{)}\PYG{o}{.}\PYG{n}{astype}\PYG{p}{(}\PYG{n+nb}{int}\PYG{p}{)} \PYG{c+c1}{\PYGZsh{} 양봉이면서 고가가 종가보다 높게 위치}
    \PYG{n}{data}\PYG{p}{[}\PYG{l+s+s1}{\PYGZsq{}}\PYG{l+s+s1}{num\PYGZus{}high/close}\PYG{l+s+s1}{\PYGZsq{}}\PYG{p}{]} \PYG{o}{=}  \PYG{n}{data}\PYG{p}{[}\PYG{l+s+s1}{\PYGZsq{}}\PYG{l+s+s1}{high/close}\PYG{l+s+s1}{\PYGZsq{}}\PYG{p}{]}\PYG{o}{.}\PYG{n}{rolling}\PYG{p}{(}\PYG{l+m+mi}{20}\PYG{p}{)}\PYG{o}{.}\PYG{n}{sum}\PYG{p}{(}\PYG{p}{)}
    \PYG{n}{data}\PYG{p}{[}\PYG{l+s+s1}{\PYGZsq{}}\PYG{l+s+s1}{long\PYGZus{}candle}\PYG{l+s+s1}{\PYGZsq{}}\PYG{p}{]} \PYG{o}{=} \PYG{p}{(}\PYG{n}{data}\PYG{p}{[}\PYG{l+s+s1}{\PYGZsq{}}\PYG{l+s+s1}{positive\PYGZus{}candle}\PYG{l+s+s1}{\PYGZsq{}}\PYG{p}{]}\PYG{o}{==}\PYG{l+m+mi}{1}\PYG{p}{)}\PYG{o}{*}\PYG{p}{(}\PYG{n}{data}\PYG{p}{[}\PYG{l+s+s1}{\PYGZsq{}}\PYG{l+s+s1}{high}\PYG{l+s+s1}{\PYGZsq{}}\PYG{p}{]}\PYG{o}{==}\PYG{n}{data}\PYG{p}{[}\PYG{l+s+s1}{\PYGZsq{}}\PYG{l+s+s1}{close}\PYG{l+s+s1}{\PYGZsq{}}\PYG{p}{]}\PYG{p}{)}\PYG{o}{*}\PYGZbs{}
    \PYG{p}{(}\PYG{n}{data}\PYG{p}{[}\PYG{l+s+s1}{\PYGZsq{}}\PYG{l+s+s1}{low}\PYG{l+s+s1}{\PYGZsq{}}\PYG{p}{]}\PYG{o}{==}\PYG{n}{data}\PYG{p}{[}\PYG{l+s+s1}{\PYGZsq{}}\PYG{l+s+s1}{open}\PYG{l+s+s1}{\PYGZsq{}}\PYG{p}{]}\PYG{p}{)}\PYG{o}{*}\PYG{p}{(}\PYG{n}{data}\PYG{p}{[}\PYG{l+s+s1}{\PYGZsq{}}\PYG{l+s+s1}{close}\PYG{l+s+s1}{\PYGZsq{}}\PYG{p}{]}\PYG{o}{/}\PYG{n}{data}\PYG{p}{[}\PYG{l+s+s1}{\PYGZsq{}}\PYG{l+s+s1}{open}\PYG{l+s+s1}{\PYGZsq{}}\PYG{p}{]} \PYG{o}{\PYGZgt{}} \PYG{l+m+mf}{1.2}\PYG{p}{)}\PYG{o}{.}\PYG{n}{astype}\PYG{p}{(}\PYG{n+nb}{int}\PYG{p}{)} \PYG{c+c1}{\PYGZsh{} 장대 양봉을 데이터로 표현}
    \PYG{n}{data}\PYG{p}{[}\PYG{l+s+s1}{\PYGZsq{}}\PYG{l+s+s1}{num\PYGZus{}long}\PYG{l+s+s1}{\PYGZsq{}}\PYG{p}{]} \PYG{o}{=}  \PYG{n}{data}\PYG{p}{[}\PYG{l+s+s1}{\PYGZsq{}}\PYG{l+s+s1}{long\PYGZus{}candle}\PYG{l+s+s1}{\PYGZsq{}}\PYG{p}{]}\PYG{o}{.}\PYG{n}{rolling}\PYG{p}{(}\PYG{l+m+mi}{60}\PYG{p}{)}\PYG{o}{.}\PYG{n}{sum}\PYG{p}{(}\PYG{p}{)} \PYG{c+c1}{\PYGZsh{} 지난 20 일 동안 장대양봉의 갯 수}
    
    
     \PYG{c+c1}{\PYGZsh{} 거래량이 종좀 터지며 매집의 흔적을 보인다   }
    \PYG{n}{data}\PYG{p}{[}\PYG{l+s+s1}{\PYGZsq{}}\PYG{l+s+s1}{volume\PYGZus{}mean}\PYG{l+s+s1}{\PYGZsq{}}\PYG{p}{]} \PYG{o}{=} \PYG{n}{data}\PYG{p}{[}\PYG{l+s+s1}{\PYGZsq{}}\PYG{l+s+s1}{volume}\PYG{l+s+s1}{\PYGZsq{}}\PYG{p}{]}\PYG{o}{.}\PYG{n}{rolling}\PYG{p}{(}\PYG{l+m+mi}{60}\PYG{p}{)}\PYG{o}{.}\PYG{n}{mean}\PYG{p}{(}\PYG{p}{)}
    \PYG{n}{data}\PYG{p}{[}\PYG{l+s+s1}{\PYGZsq{}}\PYG{l+s+s1}{volume\PYGZus{}std}\PYG{l+s+s1}{\PYGZsq{}}\PYG{p}{]} \PYG{o}{=} \PYG{n}{data}\PYG{p}{[}\PYG{l+s+s1}{\PYGZsq{}}\PYG{l+s+s1}{volume}\PYG{l+s+s1}{\PYGZsq{}}\PYG{p}{]}\PYG{o}{.}\PYG{n}{rolling}\PYG{p}{(}\PYG{l+m+mi}{60}\PYG{p}{)}\PYG{o}{.}\PYG{n}{std}\PYG{p}{(}\PYG{p}{)}
    \PYG{n}{data}\PYG{p}{[}\PYG{l+s+s1}{\PYGZsq{}}\PYG{l+s+s1}{volume\PYGZus{}z}\PYG{l+s+s1}{\PYGZsq{}}\PYG{p}{]} \PYG{o}{=} \PYG{p}{(}\PYG{n}{data}\PYG{p}{[}\PYG{l+s+s1}{\PYGZsq{}}\PYG{l+s+s1}{volume}\PYG{l+s+s1}{\PYGZsq{}}\PYG{p}{]} \PYG{o}{\PYGZhy{}} \PYG{n}{data}\PYG{p}{[}\PYG{l+s+s1}{\PYGZsq{}}\PYG{l+s+s1}{volume\PYGZus{}mean}\PYG{l+s+s1}{\PYGZsq{}}\PYG{p}{]}\PYG{p}{)}\PYG{o}{/}\PYG{n}{data}\PYG{p}{[}\PYG{l+s+s1}{\PYGZsq{}}\PYG{l+s+s1}{volume\PYGZus{}std}\PYG{l+s+s1}{\PYGZsq{}}\PYG{p}{]} \PYG{c+c1}{\PYGZsh{} 거래량은 종목과 주가에 따라 다르기 떄문에 표준화한 값이 필요함}
    \PYG{n}{data}\PYG{p}{[}\PYG{l+s+s1}{\PYGZsq{}}\PYG{l+s+s1}{z\PYGZgt{}1.96}\PYG{l+s+s1}{\PYGZsq{}}\PYG{p}{]} \PYG{o}{=} \PYG{p}{(}\PYG{n}{data}\PYG{p}{[}\PYG{l+s+s1}{\PYGZsq{}}\PYG{l+s+s1}{close}\PYG{l+s+s1}{\PYGZsq{}}\PYG{p}{]} \PYG{o}{\PYGZgt{}} \PYG{n}{data}\PYG{p}{[}\PYG{l+s+s1}{\PYGZsq{}}\PYG{l+s+s1}{open}\PYG{l+s+s1}{\PYGZsq{}}\PYG{p}{]}\PYG{p}{)}\PYG{o}{*}\PYG{p}{(}\PYG{n}{data}\PYG{p}{[}\PYG{l+s+s1}{\PYGZsq{}}\PYG{l+s+s1}{volume\PYGZus{}z}\PYG{l+s+s1}{\PYGZsq{}}\PYG{p}{]} \PYG{o}{\PYGZgt{}} \PYG{l+m+mf}{1.65}\PYG{p}{)}\PYG{o}{.}\PYG{n}{astype}\PYG{p}{(}\PYG{n+nb}{int}\PYG{p}{)} \PYG{c+c1}{\PYGZsh{} 양봉이면서 거래량이 90\PYGZpc{}신뢰구간을 벗어난 날}
    \PYG{n}{data}\PYG{p}{[}\PYG{l+s+s1}{\PYGZsq{}}\PYG{l+s+s1}{num\PYGZus{}z\PYGZgt{}1.96}\PYG{l+s+s1}{\PYGZsq{}}\PYG{p}{]} \PYG{o}{=}  \PYG{n}{data}\PYG{p}{[}\PYG{l+s+s1}{\PYGZsq{}}\PYG{l+s+s1}{z\PYGZgt{}1.96}\PYG{l+s+s1}{\PYGZsq{}}\PYG{p}{]}\PYG{o}{.}\PYG{n}{rolling}\PYG{p}{(}\PYG{l+m+mi}{60}\PYG{p}{)}\PYG{o}{.}\PYG{n}{sum}\PYG{p}{(}\PYG{p}{)}  \PYG{c+c1}{\PYGZsh{} 양봉이면서 거래량이 90\PYGZpc{}신뢰구간을 벗어난 날을 카운트}
    
    \PYG{c+c1}{\PYGZsh{} 주가지수보다 더 좋은 수익율을 보여준다}
    \PYG{n}{data}\PYG{p}{[}\PYG{l+s+s1}{\PYGZsq{}}\PYG{l+s+s1}{num\PYGZus{}win\PYGZus{}market}\PYG{l+s+s1}{\PYGZsq{}}\PYG{p}{]} \PYG{o}{=} \PYG{n}{data}\PYG{p}{[}\PYG{l+s+s1}{\PYGZsq{}}\PYG{l+s+s1}{win\PYGZus{}market}\PYG{l+s+s1}{\PYGZsq{}}\PYG{p}{]}\PYG{o}{.}\PYG{n}{rolling}\PYG{p}{(}\PYG{l+m+mi}{60}\PYG{p}{)}\PYG{o}{.}\PYG{n}{sum}\PYG{p}{(}\PYG{p}{)} \PYG{c+c1}{\PYGZsh{} 주가지수 수익율이 1 보다 작을 때, 종목 수익율이 1 보다 큰 날 수}
    \PYG{n}{data}\PYG{p}{[}\PYG{l+s+s1}{\PYGZsq{}}\PYG{l+s+s1}{pct\PYGZus{}win\PYGZus{}market}\PYG{l+s+s1}{\PYGZsq{}}\PYG{p}{]} \PYG{o}{=} \PYG{p}{(}\PYG{n}{data}\PYG{p}{[}\PYG{l+s+s1}{\PYGZsq{}}\PYG{l+s+s1}{return}\PYG{l+s+s1}{\PYGZsq{}}\PYG{p}{]}\PYG{o}{/}\PYG{n}{data}\PYG{p}{[}\PYG{l+s+s1}{\PYGZsq{}}\PYG{l+s+s1}{kosdaq\PYGZus{}return}\PYG{l+s+s1}{\PYGZsq{}}\PYG{p}{]}\PYG{p}{)}\PYG{o}{.}\PYG{n}{rolling}\PYG{p}{(}\PYG{l+m+mi}{60}\PYG{p}{)}\PYG{o}{.}\PYG{n}{mean}\PYG{p}{(}\PYG{p}{)} \PYG{c+c1}{\PYGZsh{} 주가지수 수익율 대비 종목 수익율}
    
    
    \PYG{c+c1}{\PYGZsh{} 동종업체 수익률보다 더 좋은 수익율을 보여준다.           }
    \PYG{n}{data}\PYG{p}{[}\PYG{l+s+s1}{\PYGZsq{}}\PYG{l+s+s1}{return\PYGZus{}mean}\PYG{l+s+s1}{\PYGZsq{}}\PYG{p}{]} \PYG{o}{=} \PYG{n}{data}\PYG{p}{[}\PYG{l+s+s1}{\PYGZsq{}}\PYG{l+s+s1}{return}\PYG{l+s+s1}{\PYGZsq{}}\PYG{p}{]}\PYG{o}{.}\PYG{n}{rolling}\PYG{p}{(}\PYG{l+m+mi}{60}\PYG{p}{)}\PYG{o}{.}\PYG{n}{mean}\PYG{p}{(}\PYG{p}{)} \PYG{c+c1}{\PYGZsh{} 종목별 최근 60 일 수익율의 평균}
    \PYG{n}{data}\PYG{p}{[}\PYG{l+s+s1}{\PYGZsq{}}\PYG{l+s+s1}{sector}\PYG{l+s+s1}{\PYGZsq{}}\PYG{p}{]} \PYG{o}{=} \PYG{n}{sector} 
       
    \PYG{n}{data}\PYG{p}{[}\PYG{l+s+s1}{\PYGZsq{}}\PYG{l+s+s1}{max\PYGZus{}close}\PYG{l+s+s1}{\PYGZsq{}}\PYG{p}{]}  \PYG{o}{=} \PYG{n}{data}\PYG{p}{[}\PYG{p}{[}\PYG{l+s+s1}{\PYGZsq{}}\PYG{l+s+s1}{close\PYGZus{}r1}\PYG{l+s+s1}{\PYGZsq{}}\PYG{p}{,}\PYG{l+s+s1}{\PYGZsq{}}\PYG{l+s+s1}{close\PYGZus{}r2}\PYG{l+s+s1}{\PYGZsq{}}\PYG{p}{,}\PYG{l+s+s1}{\PYGZsq{}}\PYG{l+s+s1}{close\PYGZus{}r3}\PYG{l+s+s1}{\PYGZsq{}}\PYG{p}{,}\PYG{l+s+s1}{\PYGZsq{}}\PYG{l+s+s1}{close\PYGZus{}r4}\PYG{l+s+s1}{\PYGZsq{}}\PYG{p}{,}\PYG{l+s+s1}{\PYGZsq{}}\PYG{l+s+s1}{close\PYGZus{}r5}\PYG{l+s+s1}{\PYGZsq{}}\PYG{p}{]}\PYG{p}{]}\PYG{o}{.}\PYG{n}{max}\PYG{p}{(}\PYG{n}{axis}\PYG{o}{=}\PYG{l+m+mi}{1}\PYG{p}{)} \PYG{c+c1}{\PYGZsh{} 5 영업일 종가 수익율 중 최고 값   }
    \PYG{n}{data}\PYG{p}{[}\PYG{l+s+s1}{\PYGZsq{}}\PYG{l+s+s1}{mean\PYGZus{}close}\PYG{l+s+s1}{\PYGZsq{}}\PYG{p}{]}  \PYG{o}{=} \PYG{n}{data}\PYG{p}{[}\PYG{p}{[}\PYG{l+s+s1}{\PYGZsq{}}\PYG{l+s+s1}{close\PYGZus{}r1}\PYG{l+s+s1}{\PYGZsq{}}\PYG{p}{,}\PYG{l+s+s1}{\PYGZsq{}}\PYG{l+s+s1}{close\PYGZus{}r2}\PYG{l+s+s1}{\PYGZsq{}}\PYG{p}{,}\PYG{l+s+s1}{\PYGZsq{}}\PYG{l+s+s1}{close\PYGZus{}r3}\PYG{l+s+s1}{\PYGZsq{}}\PYG{p}{,}\PYG{l+s+s1}{\PYGZsq{}}\PYG{l+s+s1}{close\PYGZus{}r4}\PYG{l+s+s1}{\PYGZsq{}}\PYG{p}{,}\PYG{l+s+s1}{\PYGZsq{}}\PYG{l+s+s1}{close\PYGZus{}r5}\PYG{l+s+s1}{\PYGZsq{}}\PYG{p}{]}\PYG{p}{]}\PYG{o}{.}\PYG{n}{mean}\PYG{p}{(}\PYG{n}{axis}\PYG{o}{=}\PYG{l+m+mi}{1}\PYG{p}{)} \PYG{c+c1}{\PYGZsh{} 5 영업일 종가 수익율 중 최고 값   }
    \PYG{n}{data}\PYG{p}{[}\PYG{l+s+s1}{\PYGZsq{}}\PYG{l+s+s1}{min\PYGZus{}close}\PYG{l+s+s1}{\PYGZsq{}}\PYG{p}{]}  \PYG{o}{=} \PYG{n}{data}\PYG{p}{[}\PYG{p}{[}\PYG{l+s+s1}{\PYGZsq{}}\PYG{l+s+s1}{close\PYGZus{}r1}\PYG{l+s+s1}{\PYGZsq{}}\PYG{p}{,}\PYG{l+s+s1}{\PYGZsq{}}\PYG{l+s+s1}{close\PYGZus{}r2}\PYG{l+s+s1}{\PYGZsq{}}\PYG{p}{,}\PYG{l+s+s1}{\PYGZsq{}}\PYG{l+s+s1}{close\PYGZus{}r3}\PYG{l+s+s1}{\PYGZsq{}}\PYG{p}{,}\PYG{l+s+s1}{\PYGZsq{}}\PYG{l+s+s1}{close\PYGZus{}r4}\PYG{l+s+s1}{\PYGZsq{}}\PYG{p}{,}\PYG{l+s+s1}{\PYGZsq{}}\PYG{l+s+s1}{close\PYGZus{}r5}\PYG{l+s+s1}{\PYGZsq{}}\PYG{p}{]}\PYG{p}{]}\PYG{o}{.}\PYG{n}{min}\PYG{p}{(}\PYG{n}{axis}\PYG{o}{=}\PYG{l+m+mi}{1}\PYG{p}{)} \PYG{c+c1}{\PYGZsh{} 5 영업일 종가 수익율 중 최저 값   }

    \PYG{n}{data} \PYG{o}{=} \PYG{n}{data}\PYG{p}{[}\PYG{p}{(}\PYG{n}{data}\PYG{p}{[}\PYG{l+s+s1}{\PYGZsq{}}\PYG{l+s+s1}{price\PYGZus{}std}\PYG{l+s+s1}{\PYGZsq{}}\PYG{p}{]}\PYG{o}{!=}\PYG{l+m+mi}{0}\PYG{p}{)} \PYG{o}{\PYGZam{}} \PYG{p}{(}\PYG{n}{data}\PYG{p}{[}\PYG{l+s+s1}{\PYGZsq{}}\PYG{l+s+s1}{volume\PYGZus{}std}\PYG{l+s+s1}{\PYGZsq{}}\PYG{p}{]}\PYG{o}{!=}\PYG{l+m+mi}{0}\PYG{p}{)}\PYG{p}{]} 
    
    \PYG{n}{feature\PYGZus{}all} \PYG{o}{=} \PYG{n}{pd}\PYG{o}{.}\PYG{n}{concat}\PYG{p}{(}\PYG{p}{[}\PYG{n}{data}\PYG{p}{,} \PYG{n}{feature\PYGZus{}all}\PYG{p}{]}\PYG{p}{,} \PYG{n}{axis}\PYG{o}{=}\PYG{l+m+mi}{0}\PYG{p}{)}
    

\PYG{n}{feature\PYGZus{}all}\PYG{p}{[}\PYG{l+s+s1}{\PYGZsq{}}\PYG{l+s+s1}{sector\PYGZus{}return}\PYG{l+s+s1}{\PYGZsq{}}\PYG{p}{]} \PYG{o}{=} \PYG{n}{feature\PYGZus{}all}\PYG{o}{.}\PYG{n}{groupby}\PYG{p}{(}\PYG{p}{[}\PYG{l+s+s1}{\PYGZsq{}}\PYG{l+s+s1}{sector}\PYG{l+s+s1}{\PYGZsq{}}\PYG{p}{,} \PYG{n}{feature\PYGZus{}all}\PYG{o}{.}\PYG{n}{index}\PYG{p}{]}\PYG{p}{)}\PYG{p}{[}\PYG{l+s+s1}{\PYGZsq{}}\PYG{l+s+s1}{return}\PYG{l+s+s1}{\PYGZsq{}}\PYG{p}{]}\PYG{o}{.}\PYG{n}{transform}\PYG{p}{(}\PYG{k}{lambda} \PYG{n}{x}\PYG{p}{:} \PYG{n}{x}\PYG{o}{.}\PYG{n}{mean}\PYG{p}{(}\PYG{p}{)}\PYG{p}{)} \PYG{c+c1}{\PYGZsh{} 섹터의 평균 수익율 계산}
\PYG{n}{feature\PYGZus{}all}\PYG{p}{[}\PYG{l+s+s1}{\PYGZsq{}}\PYG{l+s+s1}{return over sector}\PYG{l+s+s1}{\PYGZsq{}}\PYG{p}{]} \PYG{o}{=} \PYG{p}{(}\PYG{n}{feature\PYGZus{}all}\PYG{p}{[}\PYG{l+s+s1}{\PYGZsq{}}\PYG{l+s+s1}{return}\PYG{l+s+s1}{\PYGZsq{}}\PYG{p}{]}\PYG{o}{/}\PYG{n}{feature\PYGZus{}all}\PYG{p}{[}\PYG{l+s+s1}{\PYGZsq{}}\PYG{l+s+s1}{sector\PYGZus{}return}\PYG{l+s+s1}{\PYGZsq{}}\PYG{p}{]}\PYG{p}{)} \PYG{c+c1}{\PYGZsh{} 섹터 평균 수익률 대비 종목 수익률 계산}
\PYG{n}{feature\PYGZus{}all}\PYG{o}{.}\PYG{n}{dropna}\PYG{p}{(}\PYG{n}{inplace}\PYG{o}{=}\PYG{k+kc}{True}\PYG{p}{)} \PYG{c+c1}{\PYGZsh{} Missing 값 있는 행 모두 제거}


\PYG{c+c1}{\PYGZsh{} 최종 피처 및 수익률 데이터만으로 구성}
\PYG{n}{feature\PYGZus{}all} \PYG{o}{=} \PYG{n}{feature\PYGZus{}all}\PYG{p}{[}\PYG{p}{[}\PYG{l+s+s1}{\PYGZsq{}}\PYG{l+s+s1}{code}\PYG{l+s+s1}{\PYGZsq{}}\PYG{p}{,} \PYG{l+s+s1}{\PYGZsq{}}\PYG{l+s+s1}{sector}\PYG{l+s+s1}{\PYGZsq{}}\PYG{p}{,}\PYG{l+s+s1}{\PYGZsq{}}\PYG{l+s+s1}{return}\PYG{l+s+s1}{\PYGZsq{}}\PYG{p}{,}\PYG{l+s+s1}{\PYGZsq{}}\PYG{l+s+s1}{kosdaq\PYGZus{}return}\PYG{l+s+s1}{\PYGZsq{}}\PYG{p}{,}\PYG{l+s+s1}{\PYGZsq{}}\PYG{l+s+s1}{price\PYGZus{}z}\PYG{l+s+s1}{\PYGZsq{}}\PYG{p}{,}\PYG{l+s+s1}{\PYGZsq{}}\PYG{l+s+s1}{volume\PYGZus{}z}\PYG{l+s+s1}{\PYGZsq{}}\PYG{p}{,}\PYG{l+s+s1}{\PYGZsq{}}\PYG{l+s+s1}{num\PYGZus{}high/close}\PYG{l+s+s1}{\PYGZsq{}}\PYG{p}{,}\PYG{l+s+s1}{\PYGZsq{}}\PYG{l+s+s1}{num\PYGZus{}long}\PYG{l+s+s1}{\PYGZsq{}}\PYG{p}{,}\PYG{l+s+s1}{\PYGZsq{}}\PYG{l+s+s1}{num\PYGZus{}z\PYGZgt{}1.96}\PYG{l+s+s1}{\PYGZsq{}}\PYG{p}{,}\PYG{l+s+s1}{\PYGZsq{}}\PYG{l+s+s1}{num\PYGZus{}win\PYGZus{}market}\PYG{l+s+s1}{\PYGZsq{}}\PYG{p}{,}\PYG{l+s+s1}{\PYGZsq{}}\PYG{l+s+s1}{pct\PYGZus{}win\PYGZus{}market}\PYG{l+s+s1}{\PYGZsq{}}\PYG{p}{,}\PYG{l+s+s1}{\PYGZsq{}}\PYG{l+s+s1}{return over sector}\PYG{l+s+s1}{\PYGZsq{}}\PYG{p}{,}\PYG{l+s+s1}{\PYGZsq{}}\PYG{l+s+s1}{max\PYGZus{}close}\PYG{l+s+s1}{\PYGZsq{}}\PYG{p}{,}\PYG{l+s+s1}{\PYGZsq{}}\PYG{l+s+s1}{mean\PYGZus{}close}\PYG{l+s+s1}{\PYGZsq{}}\PYG{p}{,}\PYG{l+s+s1}{\PYGZsq{}}\PYG{l+s+s1}{min\PYGZus{}close}\PYG{l+s+s1}{\PYGZsq{}}\PYG{p}{]}\PYG{p}{]}
\PYG{n}{feature\PYGZus{}all}\PYG{o}{.}\PYG{n}{to\PYGZus{}pickle}\PYG{p}{(}\PYG{l+s+s1}{\PYGZsq{}}\PYG{l+s+s1}{feature\PYGZus{}all.pkl}\PYG{l+s+s1}{\PYGZsq{}}\PYG{p}{)}  
\end{sphinxVerbatim}

\end{sphinxuseclass}\end{sphinxVerbatimInput}

\end{sphinxuseclass}
\sphinxAtStartPar
 이제 모델링을 위한 데이터 준비가 끝났습니다. 간단한 프로파일을 뽑아봅니다. 평균과 표준편차 값을 보고, 피처들이 제대로 생성되었는 지 확인합니다. 그리고 price\_z 와 volum\_z 는 같이 분석했을 때 유의미했다는 사실을 기억하면 좋겠습니다.

\begin{sphinxuseclass}{cell}\begin{sphinxVerbatimInput}

\begin{sphinxuseclass}{cell_input}
\begin{sphinxVerbatim}[commandchars=\\\{\}]
\PYG{n}{feature\PYGZus{}all} \PYG{o}{=} \PYG{n}{pd}\PYG{o}{.}\PYG{n}{read\PYGZus{}pickle}\PYG{p}{(}\PYG{l+s+s1}{\PYGZsq{}}\PYG{l+s+s1}{feature\PYGZus{}all.pkl}\PYG{l+s+s1}{\PYGZsq{}}\PYG{p}{)} 
\PYG{n}{feature\PYGZus{}all}\PYG{o}{.}\PYG{n}{describe}\PYG{p}{(}\PYG{n}{percentiles}\PYG{o}{=}\PYG{p}{[}\PYG{l+m+mf}{0.05}\PYG{p}{,} \PYG{l+m+mf}{0.1}\PYG{p}{,} \PYG{l+m+mf}{0.9}\PYG{p}{,} \PYG{l+m+mf}{0.95}\PYG{p}{]}\PYG{p}{)}\PYG{o}{.}\PYG{n}{style}\PYG{o}{.}\PYG{n}{set\PYGZus{}table\PYGZus{}attributes}\PYG{p}{(}\PYG{l+s+s1}{\PYGZsq{}}\PYG{l+s+s1}{style=}\PYG{l+s+s1}{\PYGZdq{}}\PYG{l+s+s1}{font\PYGZhy{}size: 12px}\PYG{l+s+s1}{\PYGZdq{}}\PYG{l+s+s1}{\PYGZsq{}}\PYG{p}{)}\PYG{o}{.}\PYG{n}{format}\PYG{p}{(}\PYG{n}{precision}\PYG{o}{=}\PYG{l+m+mi}{3}\PYG{p}{)}
\end{sphinxVerbatim}

\end{sphinxuseclass}\end{sphinxVerbatimInput}
\begin{sphinxVerbatimOutput}

\begin{sphinxuseclass}{cell_output}
\begin{sphinxVerbatim}[commandchars=\\\{\}]
\PYGZlt{}pandas.io.formats.style.Styler at 0x19b5bda8130\PYGZgt{}
\end{sphinxVerbatim}

\end{sphinxuseclass}\end{sphinxVerbatimOutput}

\end{sphinxuseclass}
\begin{sphinxuseclass}{cell}\begin{sphinxVerbatimInput}

\begin{sphinxuseclass}{cell_input}
\begin{sphinxVerbatim}[commandchars=\\\{\}]
\PYG{k+kn}{import} \PYG{n+nn}{FinanceDataReader} \PYG{k}{as} \PYG{n+nn}{fdr}
\PYG{o}{\PYGZpc{}}\PYG{k}{matplotlib} inline
\PYG{k+kn}{import} \PYG{n+nn}{matplotlib}\PYG{n+nn}{.}\PYG{n+nn}{pyplot} \PYG{k}{as} \PYG{n+nn}{plt}
\PYG{k+kn}{import} \PYG{n+nn}{pandas} \PYG{k}{as} \PYG{n+nn}{pd}
\PYG{k+kn}{import} \PYG{n+nn}{numpy} \PYG{k}{as} \PYG{n+nn}{np}
\PYG{k+kn}{import} \PYG{n+nn}{warnings}
\PYG{n}{warnings}\PYG{o}{.}\PYG{n}{filterwarnings}\PYG{p}{(}\PYG{l+s+s1}{\PYGZsq{}}\PYG{l+s+s1}{ignore}\PYG{l+s+s1}{\PYGZsq{}}\PYG{p}{)}
\PYG{n}{pd}\PYG{o}{.}\PYG{n}{options}\PYG{o}{.}\PYG{n}{display}\PYG{o}{.}\PYG{n}{float\PYGZus{}format} \PYG{o}{=} \PYG{l+s+s1}{\PYGZsq{}}\PYG{l+s+si}{\PYGZob{}:,.3f\PYGZcb{}}\PYG{l+s+s1}{\PYGZsq{}}\PYG{o}{.}\PYG{n}{format}
\end{sphinxVerbatim}

\end{sphinxuseclass}\end{sphinxVerbatimInput}

\end{sphinxuseclass}

\part{모델링 라이브러리 소개}
\label{\detokenize{chapter5/5.2.2_Modeling_Library:id1}}\label{\detokenize{chapter5/5.2.2_Modeling_Library::doc}}
\sphinxAtStartPar
모델은 가설을 활용하여 타겟변수를 예측하는 알고리즘을 만드는 것이라고 생각하시면 될 것 같습니다. 파이썬에서는 모델링을 위하여 여러개의 라이브러리(패키지)를 제공하고 있습니다. 대표적인 모델개발 라이브러리는 Statsmodels, Scikit\sphinxhyphen{}Learn, Keras 등이 있습니다. 책에서 종목 추천으로 사용할 모델은 일반화 가법 모형(Generalized Additive Model) 입니다. 일반화가법모형은 Statsmodels 에서도 구현할 수 있으나, pyGAM 패키지를 사용하면 더 편리합니다.

\sphinxAtStartPar
Statsmodels 는 전통적인 통계모델에 특화되어 있고, Scikit\sphinxhyphen{}Learn 는 머신러닝모델, Keras 는 딥러닝 모델을 개발할 때 활용할 수 있습니다. 라이브러리는 모델을 만드는 패키지가 들어가 있으므로 호출해서 사용하면 됩니다. Statsmodels 은 주로 일반화 선형모형을 구현할 때 주로 사용합니다. 통계 선형모형의 장점은 해석이 가능한 모델을 만들 수 있다는 것입니다. 예를 들면 변수 X 가 1 단위 증가하면 타겟변수는 얼마나 증가하느냐? 등의 해석을 할 수 있습니다..

\sphinxAtStartPar
Scikit\sphinxhyphen{}Learn 은 주로 머신러닝 모델을 만들 때 활용하는 라이브러리입니다. 머신러닝 모델은 각 피쳐의 해석보다는 예측력을 최우선으로 합니다. 특히 트리(Tree) 기반 모델은 변수간의 상호작용을 고려하므로 입력 변수사이에 상호작용이 많을 때 효과가 좋습니다. 머신러닝 모델 중에는 앙상블 모델이 인기인데요. 앙상블도 Bias 를 줄이는데 집중하는 Boosting  모델(예 Ada Boost) 계열과 Variance 를 줄이는데 집중하는 Bagging 모델(예 Random Forest) 계열로 나눌 수 있습니다. Bias 랑 Variance 는 하나를 내리면 하나는 올라가는 특징이 있습니다. 두 명의 양궁선수가 있습니다. 한 명은 과녁 중앙에 골고루 퍼지게 활을 쏘는 능력이 있고, 한 명은 일단 처음 쏜 화살에 근처에 집중에서 쏘는 능력이 있다고 하면 누구를 선택하시겠습니까? 첫 번째 양궁선수는 과녁근방에 골고루 쏘는 분이므로 큰 점수는 못 얻어도 항상 기본점수 이상은 획득하는 안전함이 있습니다. 즉 Variance 가 낮음에 해당합니다. 두 번째 양궁선수는  일단 첫 화살에 중앙에 명중하면, 나머지도 10점을 얻을 수 있습니다. 하지만, 처음 화살이 빗나가면 나머지도 다 빗나갑니다. 따라서 첫 화살이 중요합니다. Bias 가 낮기 때문에 overfitting (과대적합) 을 주의해야 합니다. 운이 안 좋아 중앙에서 먼거리에 첫 화살이 명중했다면, 나머지 화살도 그  근방으로 가므로, 우리가 원하는 해답이 아닌 곳으로 모델학습이 이루어지게 되는 것입니다.

\sphinxAtStartPar
Keras 는 딥러닝을 위한 라이브러리입니다. 데이터 수가 많지 않고,  피쳐의 디멘젼이 5 개(시종고저, 거래량) 라면 데이터 복잡성도 높지 않습니다. 구현하고자 하는 예측모델은 딥러닝이 적절하지는 않아보입니다. 굳이 일봉 데이터가 요약된 피쳐로 뉴럴네트워크 모델을 구현한다면  Multi\sphinxhyphen{}Layer Perceptron (다층 퍼셉트론) 모델을 생각해 볼 수 있습니다. MLP 는 비선형관계를 표현하기 위해서 Activation Function (활성화함수) 를 이용하고,  Activation 함수에서 나온 값을 다시 다음 층의 입력변수로 넣는 형태입니다. 이렇게 함으로써 변수간의 상호작용과 비선형관계를 동시에 표현할 수 있습니다. 사실, 활성화 함수가 Sigmoid  함수인 뉴럴네트워크 모델은 Logistic Regression.모델을 가로 세로층으로 중첩한 것과 동일한 구조가 됩니다. 즉, Logistic Regression 의 확장형으로도 생각할 수 있습니다.  뉴럴네트워크 계통의 모델은 Loss Function (손실함수) 를 만들고 Loss Function 를 최소화하는 네트워크의 가중치를 찾도록 훈련합니다. 많이 쓰는 훈련방식은 오류 역전파(BackPropagation) 입니다. 이런 식의 접근 법은 과대적합이 항상 문제가 됩니다. 따라서 과대적합을 피하기 위해 다양한 기법이 개발 되고 있습니다.

\sphinxAtStartPar
이번절에서는 Statsmodel 과 Scikit\sphinxhyphen{}Learn 라이브러리가 모델 개발에 어떻게 활용되는지 경험해 보는 시간입니다.

\sphinxAtStartPar
모델링을 위해 준비한 데이터를 읽습니다. 그리고 모델의 오버피팅을 최소화하기 위하여 타겟변수를 0 과 1 로 치환합니다. 예를 들어, 5\% 익절의 데이터 표현은 \sphinxhyphen{} ‘max\_close’ 가 5\% 이상일 때 1, 아니면 0 이 됩니다. ‘max\_close’ 가 1 인 비율을 보니, 약 24\% 입니다. 10000 개 샘플을 뽑아 예측모델을 만들고 나머지로 데이터로 테스트(혹은 백테스팅)를 하겠습니다.

\begin{sphinxuseclass}{cell}\begin{sphinxVerbatimInput}

\begin{sphinxuseclass}{cell_input}
\begin{sphinxVerbatim}[commandchars=\\\{\}]
\PYG{n}{feature\PYGZus{}all} \PYG{o}{=} \PYG{n}{pd}\PYG{o}{.}\PYG{n}{read\PYGZus{}pickle}\PYG{p}{(}\PYG{l+s+s1}{\PYGZsq{}}\PYG{l+s+s1}{feature\PYGZus{}all.pkl}\PYG{l+s+s1}{\PYGZsq{}}\PYG{p}{)} 
\PYG{n}{feature\PYGZus{}all}\PYG{p}{[}\PYG{l+s+s1}{\PYGZsq{}}\PYG{l+s+s1}{target}\PYG{l+s+s1}{\PYGZsq{}}\PYG{p}{]} \PYG{o}{=} \PYG{n}{np}\PYG{o}{.}\PYG{n}{where}\PYG{p}{(}\PYG{n}{feature\PYGZus{}all}\PYG{p}{[}\PYG{l+s+s1}{\PYGZsq{}}\PYG{l+s+s1}{max\PYGZus{}close}\PYG{l+s+s1}{\PYGZsq{}}\PYG{p}{]}\PYG{o}{\PYGZgt{}}\PYG{o}{=} \PYG{l+m+mf}{1.05}\PYG{p}{,} \PYG{l+m+mi}{1}\PYG{p}{,} \PYG{l+m+mi}{0}\PYG{p}{)}
\PYG{n}{target} \PYG{o}{=} \PYG{n}{feature\PYGZus{}all}\PYG{p}{[}\PYG{l+s+s1}{\PYGZsq{}}\PYG{l+s+s1}{target}\PYG{l+s+s1}{\PYGZsq{}}\PYG{p}{]}\PYG{o}{.}\PYG{n}{mean}\PYG{p}{(}\PYG{p}{)}
\PYG{n+nb}{print}\PYG{p}{(}\PYG{l+s+sa}{f}\PYG{l+s+s1}{\PYGZsq{}}\PYG{l+s+s1}{\PYGZpc{} of target:}\PYG{l+s+si}{\PYGZob{}}\PYG{n}{target}\PYG{l+s+si}{:}\PYG{l+s+s1}{ 5.1\PYGZpc{}}\PYG{l+s+si}{\PYGZcb{}}\PYG{l+s+s1}{\PYGZsq{}}\PYG{p}{)}

\PYG{n}{mdl\PYGZus{}all} \PYG{o}{=} \PYG{n}{feature\PYGZus{}all}\PYG{o}{.}\PYG{n}{set\PYGZus{}index}\PYG{p}{(}\PYG{p}{[}\PYG{n}{feature\PYGZus{}all}\PYG{o}{.}\PYG{n}{index}\PYG{p}{,}\PYG{l+s+s1}{\PYGZsq{}}\PYG{l+s+s1}{code}\PYG{l+s+s1}{\PYGZsq{}}\PYG{p}{]}\PYG{p}{)}

\PYG{n}{train} \PYG{o}{=} \PYG{n}{mdl\PYGZus{}all}\PYG{o}{.}\PYG{n}{sample}\PYG{p}{(}\PYG{l+m+mi}{10000}\PYG{p}{,} \PYG{n}{random\PYGZus{}state}\PYG{o}{=}\PYG{l+m+mi}{124}\PYG{p}{)}
\PYG{n}{test} \PYG{o}{=} \PYG{n}{mdl\PYGZus{}all}\PYG{o}{.}\PYG{n}{loc}\PYG{p}{[}\PYG{o}{\PYGZti{}}\PYG{n}{mdl\PYGZus{}all}\PYG{o}{.}\PYG{n}{index}\PYG{o}{.}\PYG{n}{isin}\PYG{p}{(}\PYG{n}{train}\PYG{o}{.}\PYG{n}{index}\PYG{p}{)}\PYG{p}{]}
\end{sphinxVerbatim}

\end{sphinxuseclass}\end{sphinxVerbatimInput}
\begin{sphinxVerbatimOutput}

\begin{sphinxuseclass}{cell_output}
\begin{sphinxVerbatim}[commandchars=\\\{\}]
\PYGZpc{} of target: 24.3\PYGZpc{}
\end{sphinxVerbatim}

\end{sphinxuseclass}\end{sphinxVerbatimOutput}

\end{sphinxuseclass}

\part{Statsmodels \sphinxhyphen{} Logistic Regression}
\label{\detokenize{chapter5/5.2.2_Modeling_Library:statsmodels-logistic-regression}}
\sphinxAtStartPar
아래 코드는 Statsmodels 라이브러리에 대한 이해가 목적입니다. Statsmodels 는 전통적인 통계모델을 구현하는데 주로 활용하는데요. 통계모델의 장점은 변수의 해석이 가능하다는 것입니다. 아래 코드는 랜덤해게 뽑은 5천개의 샘플로 모델을 만들고, 나머지 데이터로 모델 성능을 테스트하는 과정입니다. 모델 개발은 여기서부터 시작입니다.  결과를 보면 P Value(P>|z|) 가 0.01(유의수준) 보다 큰 변수가 많은데요. P Value(P>|z|) 가 유의수준보다 크다는 이야기는 coefficient 가 0 일 가능성이 높다는 말이고, Coefficient 가 0 이라는 말은 예측에 도움을 안 준다는 말입니다. 이런 변수들은 적절한 변형을 통하여 유의미하게 만들거나 제거해야 합니다. 가장 대표적인 방법이 Binning 입니다. 이 절은 라이브러리를 소개하는 것이 목적이라, 모델 완성을 위한 나머지 과정은 생략하도록 하겠습니다. 제가 통계모델의 장점으로 해석을 언급했는데요. 아직 모델이 완성되지 않았지만, 변수 ‘volume\_z’ 를 해석해 보도록 하겠습니다. ‘volume\_z’ 는 과거 20일대비 당일 거래량이 얼마나 많은 지를 의미하는 변수입니다. ‘volume\_z’ 가 1 증가하면 log(odds) 는 그 변수의 계수 0.1765 만큼 증가하게 됩니다. 즉, odds 는 exp(0.1765) 증가하게 됩니다. 풀어서 이야기하면, 전일 20일 대비 당일 거래량 표준화 값 z 가  1 증가할 때마다, 5\% 로 익절할 odds(=p/1\sphinxhyphen{}p)는 exp(0.1765) 증가한다고 말할 수 있습니다.

\sphinxAtStartPar
모델을 완성하기까지 필요한 나머지 절차는 아래와 같습니다.
\begin{enumerate}
\sphinxsetlistlabels{\arabic}{enumi}{enumii}{}{.}%
\item {} 
\sphinxAtStartPar
각 설명변수와 타겟변수와 관계를 분석합니다 (변수간에 상호작용 강한 지 체크)

\item {} 
\sphinxAtStartPar
선형적인 관계가 없는 변수는 binning 등을 통해 문제를 해결합니다. 혹은 제곱근, 제곱, 로그 등의 변형으로 선형적으로 만들 수 도 있습니다.

\item {} 
\sphinxAtStartPar
다중 공선성이 의심되는 변수는 제거하거나 새로운 변수로 대체합니다. (다중 공선성이 높은 모델은 변수의 해석이 부정확함)

\item {} 
\sphinxAtStartPar
테스트 데이터셋과 예측성능을 비교합니다 (오버피팅 여부 확인).

\item {} 
\sphinxAtStartPar
변수를 해석하고 예측값을 만듭니다.

\end{enumerate}

\begin{sphinxuseclass}{cell}\begin{sphinxVerbatimInput}

\begin{sphinxuseclass}{cell_input}
\begin{sphinxVerbatim}[commandchars=\\\{\}]
\PYG{k+kn}{import} \PYG{n+nn}{statsmodels}\PYG{n+nn}{.}\PYG{n+nn}{api} \PYG{k}{as} \PYG{n+nn}{sm}

\PYG{n}{feature\PYGZus{}list} \PYG{o}{=} \PYG{p}{[}\PYG{l+s+s1}{\PYGZsq{}}\PYG{l+s+s1}{price\PYGZus{}z}\PYG{l+s+s1}{\PYGZsq{}}\PYG{p}{,} \PYG{l+s+s1}{\PYGZsq{}}\PYG{l+s+s1}{volume\PYGZus{}z}\PYG{l+s+s1}{\PYGZsq{}}\PYG{p}{,} \PYG{l+s+s1}{\PYGZsq{}}\PYG{l+s+s1}{num\PYGZus{}high/close}\PYG{l+s+s1}{\PYGZsq{}}\PYG{p}{,} \PYG{l+s+s1}{\PYGZsq{}}\PYG{l+s+s1}{num\PYGZus{}long}\PYG{l+s+s1}{\PYGZsq{}}\PYG{p}{,} \PYG{l+s+s1}{\PYGZsq{}}\PYG{l+s+s1}{num\PYGZus{}z\PYGZgt{}1.96}\PYG{l+s+s1}{\PYGZsq{}}\PYG{p}{,} \PYG{l+s+s1}{\PYGZsq{}}\PYG{l+s+s1}{num\PYGZus{}win\PYGZus{}market}\PYG{l+s+s1}{\PYGZsq{}}\PYG{p}{,} \PYG{l+s+s1}{\PYGZsq{}}\PYG{l+s+s1}{pct\PYGZus{}win\PYGZus{}market}\PYG{l+s+s1}{\PYGZsq{}}\PYG{p}{,} \PYG{l+s+s1}{\PYGZsq{}}\PYG{l+s+s1}{return over sector}\PYG{l+s+s1}{\PYGZsq{}}\PYG{p}{]}

\PYG{n}{X} \PYG{o}{=} \PYG{n}{train}\PYG{p}{[}\PYG{n}{feature\PYGZus{}list}\PYG{p}{]}
\PYG{n}{y} \PYG{o}{=} \PYG{n}{train}\PYG{p}{[}\PYG{l+s+s1}{\PYGZsq{}}\PYG{l+s+s1}{target}\PYG{l+s+s1}{\PYGZsq{}}\PYG{p}{]}

\PYG{n}{X} \PYG{o}{=} \PYG{n}{sm}\PYG{o}{.}\PYG{n}{add\PYGZus{}constant}\PYG{p}{(}\PYG{n}{X}\PYG{p}{)}
\PYG{n}{model} \PYG{o}{=} \PYG{n}{sm}\PYG{o}{.}\PYG{n}{Logit}\PYG{p}{(}\PYG{n}{y}\PYG{p}{,} \PYG{n}{X}\PYG{p}{)}
\PYG{n}{results} \PYG{o}{=} \PYG{n}{model}\PYG{o}{.}\PYG{n}{fit}\PYG{p}{(}\PYG{p}{)}
\PYG{n+nb}{print}\PYG{p}{(}\PYG{n}{results}\PYG{o}{.}\PYG{n}{summary}\PYG{p}{(}\PYG{p}{)}\PYG{p}{)}
\PYG{n}{yhat} \PYG{o}{=} \PYG{n}{results}\PYG{o}{.}\PYG{n}{predict}\PYG{p}{(}\PYG{n}{X}\PYG{p}{)}
\PYG{n}{yhat} \PYG{o}{=} \PYG{n}{pd}\PYG{o}{.}\PYG{n}{Series}\PYG{p}{(}\PYG{n}{yhat}\PYG{p}{,} \PYG{n}{name}\PYG{o}{=}\PYG{l+s+s1}{\PYGZsq{}}\PYG{l+s+s1}{yhat}\PYG{l+s+s1}{\PYGZsq{}}\PYG{p}{)}

\PYG{n}{X\PYGZus{}test} \PYG{o}{=} \PYG{n}{test}\PYG{p}{[}\PYG{n}{feature\PYGZus{}list}\PYG{p}{]}
\PYG{n}{y\PYGZus{}test} \PYG{o}{=} \PYG{n}{test}\PYG{p}{[}\PYG{l+s+s1}{\PYGZsq{}}\PYG{l+s+s1}{target}\PYG{l+s+s1}{\PYGZsq{}}\PYG{p}{]}
\PYG{n}{X\PYGZus{}test} \PYG{o}{=} \PYG{n}{sm}\PYG{o}{.}\PYG{n}{add\PYGZus{}constant}\PYG{p}{(}\PYG{n}{X\PYGZus{}test}\PYG{p}{)}
\PYG{n}{yhat\PYGZus{}test} \PYG{o}{=} \PYG{n}{results}\PYG{o}{.}\PYG{n}{predict}\PYG{p}{(}\PYG{n}{X\PYGZus{}test}\PYG{p}{)}
\PYG{n}{yhat\PYGZus{}test} \PYG{o}{=} \PYG{n}{pd}\PYG{o}{.}\PYG{n}{Series}\PYG{p}{(}\PYG{n}{yhat\PYGZus{}test}\PYG{p}{,} \PYG{n}{name}\PYG{o}{=}\PYG{l+s+s1}{\PYGZsq{}}\PYG{l+s+s1}{yhat}\PYG{l+s+s1}{\PYGZsq{}}\PYG{p}{)}
\end{sphinxVerbatim}

\end{sphinxuseclass}\end{sphinxVerbatimInput}
\begin{sphinxVerbatimOutput}

\begin{sphinxuseclass}{cell_output}
\begin{sphinxVerbatim}[commandchars=\\\{\}]
Optimization terminated successfully.
         Current function value: 0.549822
         Iterations 6
                           Logit Regression Results                           
==============================================================================
Dep. Variable:                 target   No. Observations:                10000
Model:                          Logit   Df Residuals:                     9991
Method:                           MLE   Df Model:                            8
Date:                Sun, 26 Jun 2022   Pseudo R\PYGZhy{}squ.:                 0.01089
Time:                        05:30:08   Log\PYGZhy{}Likelihood:                \PYGZhy{}5498.2
converged:                       True   LL\PYGZhy{}Null:                       \PYGZhy{}5558.7
Covariance Type:            nonrobust   LLR p\PYGZhy{}value:                 2.047e\PYGZhy{}22
======================================================================================
                         coef    std err          z      P\PYGZgt{}|z|      [0.025      0.975]
\PYGZhy{}\PYGZhy{}\PYGZhy{}\PYGZhy{}\PYGZhy{}\PYGZhy{}\PYGZhy{}\PYGZhy{}\PYGZhy{}\PYGZhy{}\PYGZhy{}\PYGZhy{}\PYGZhy{}\PYGZhy{}\PYGZhy{}\PYGZhy{}\PYGZhy{}\PYGZhy{}\PYGZhy{}\PYGZhy{}\PYGZhy{}\PYGZhy{}\PYGZhy{}\PYGZhy{}\PYGZhy{}\PYGZhy{}\PYGZhy{}\PYGZhy{}\PYGZhy{}\PYGZhy{}\PYGZhy{}\PYGZhy{}\PYGZhy{}\PYGZhy{}\PYGZhy{}\PYGZhy{}\PYGZhy{}\PYGZhy{}\PYGZhy{}\PYGZhy{}\PYGZhy{}\PYGZhy{}\PYGZhy{}\PYGZhy{}\PYGZhy{}\PYGZhy{}\PYGZhy{}\PYGZhy{}\PYGZhy{}\PYGZhy{}\PYGZhy{}\PYGZhy{}\PYGZhy{}\PYGZhy{}\PYGZhy{}\PYGZhy{}\PYGZhy{}\PYGZhy{}\PYGZhy{}\PYGZhy{}\PYGZhy{}\PYGZhy{}\PYGZhy{}\PYGZhy{}\PYGZhy{}\PYGZhy{}\PYGZhy{}\PYGZhy{}\PYGZhy{}\PYGZhy{}\PYGZhy{}\PYGZhy{}\PYGZhy{}\PYGZhy{}\PYGZhy{}\PYGZhy{}\PYGZhy{}\PYGZhy{}\PYGZhy{}\PYGZhy{}\PYGZhy{}\PYGZhy{}\PYGZhy{}\PYGZhy{}\PYGZhy{}\PYGZhy{}
const                \PYGZhy{}24.3868      6.812     \PYGZhy{}3.580      0.000     \PYGZhy{}37.739     \PYGZhy{}11.035
price\PYGZus{}z               \PYGZhy{}0.1266      0.021     \PYGZhy{}6.156      0.000      \PYGZhy{}0.167      \PYGZhy{}0.086
volume\PYGZus{}z               0.1491      0.024      6.255      0.000       0.102       0.196
num\PYGZus{}high/close         0.1396      0.058      2.411      0.016       0.026       0.253
num\PYGZus{}long              \PYGZhy{}0.1293      0.172     \PYGZhy{}0.752      0.452      \PYGZhy{}0.466       0.208
num\PYGZus{}z\PYGZgt{}1.96             0.0257      0.013      1.982      0.047       0.000       0.051
num\PYGZus{}win\PYGZus{}market         0.0251      0.009      2.722      0.006       0.007       0.043
pct\PYGZus{}win\PYGZus{}market        23.8495      6.797      3.509      0.000      10.527      37.172
return over sector    \PYGZhy{}0.8839      0.835     \PYGZhy{}1.058      0.290      \PYGZhy{}2.521       0.753
======================================================================================
\end{sphinxVerbatim}

\end{sphinxuseclass}\end{sphinxVerbatimOutput}

\end{sphinxuseclass}
\sphinxAtStartPar
 개발 데이터가 아니라, 테스트데이터에서도 좋은 성능을 보이는지 확인해봅니다. 쉽게 확인하는 방법은 Decile 분석입니다. 예측값의 변별력을 알기 위해서 정렬된 예측값을 10 개 구간으로 나누고, 각 구간에서 ‘target’의 평균값을 찍어봅니다. 파란색이 개발데이터, 주황색이 테스트 데이터입니다. 모델이 예측력이 좋다면, 예측값의 십분위 수가 증가하면 5\%로 익절할 확률도 같이 증가하는 형태를 보이게 됩니다. 아래 결과에서 완성되지 않은 모델이지만 단조증가하는 좋은 흐름을 보여주고 있습니다. 테스트 결과에서 제 1 십분위수(첫 번째 구간) 에서 종목을 선택한다면 19.7\% 로 익절할 확률이 있지만, 제 10 분위수(마지막 구간)에서 종목을 선택한다면 34.9\% 로 익절할 확률이 생깁니다.

\begin{sphinxuseclass}{cell}\begin{sphinxVerbatimInput}

\begin{sphinxuseclass}{cell_input}
\begin{sphinxVerbatim}[commandchars=\\\{\}]
\PYG{k}{def} \PYG{n+nf}{perf}\PYG{p}{(}\PYG{n}{y}\PYG{p}{,} \PYG{n}{yhat}\PYG{p}{)}\PYG{p}{:} \PYG{c+c1}{\PYGZsh{} Decile 분석 함수}
    \PYG{n}{combined} \PYG{o}{=} \PYG{n}{pd}\PYG{o}{.}\PYG{n}{concat}\PYG{p}{(}\PYG{p}{[}\PYG{n}{y}\PYG{p}{,} \PYG{n}{yhat}\PYG{p}{]}\PYG{p}{,} \PYG{n}{axis}\PYG{o}{=}\PYG{l+m+mi}{1}\PYG{p}{)}
    \PYG{n}{ranks} \PYG{o}{=} \PYG{n}{pd}\PYG{o}{.}\PYG{n}{qcut}\PYG{p}{(}\PYG{n}{combined}\PYG{p}{[}\PYG{l+s+s1}{\PYGZsq{}}\PYG{l+s+s1}{yhat}\PYG{l+s+s1}{\PYGZsq{}}\PYG{p}{]}\PYG{p}{,} \PYG{n}{q}\PYG{o}{=}\PYG{l+m+mi}{10}\PYG{p}{)}
    \PYG{n+nb}{print}\PYG{p}{(}\PYG{n}{combined}\PYG{o}{.}\PYG{n}{groupby}\PYG{p}{(}\PYG{n}{ranks}\PYG{p}{)}\PYG{p}{[}\PYG{l+s+s1}{\PYGZsq{}}\PYG{l+s+s1}{target}\PYG{l+s+s1}{\PYGZsq{}}\PYG{p}{]}\PYG{o}{.}\PYG{n}{agg}\PYG{p}{(}\PYG{p}{[}\PYG{l+s+s1}{\PYGZsq{}}\PYG{l+s+s1}{count}\PYG{l+s+s1}{\PYGZsq{}}\PYG{p}{,}\PYG{l+s+s1}{\PYGZsq{}}\PYG{l+s+s1}{mean}\PYG{l+s+s1}{\PYGZsq{}}\PYG{p}{]}\PYG{p}{)}\PYG{p}{)}
    \PYG{n}{combined}\PYG{o}{.}\PYG{n}{groupby}\PYG{p}{(}\PYG{n}{ranks}\PYG{p}{)}\PYG{p}{[}\PYG{l+s+s1}{\PYGZsq{}}\PYG{l+s+s1}{target}\PYG{l+s+s1}{\PYGZsq{}}\PYG{p}{]}\PYG{o}{.}\PYG{n}{mean}\PYG{p}{(}\PYG{p}{)}\PYG{o}{.}\PYG{n}{plot}\PYG{p}{(}\PYG{p}{)}

\PYG{n}{perf}\PYG{p}{(}\PYG{n}{y}\PYG{p}{,} \PYG{n}{yhat}\PYG{p}{)}
\PYG{n}{perf}\PYG{p}{(}\PYG{n}{y\PYGZus{}test}\PYG{p}{,} \PYG{n}{yhat\PYGZus{}test}\PYG{p}{)}
\end{sphinxVerbatim}

\end{sphinxuseclass}\end{sphinxVerbatimInput}
\begin{sphinxVerbatimOutput}

\begin{sphinxuseclass}{cell_output}
\begin{sphinxVerbatim}[commandchars=\\\{\}]
                count  mean
yhat                       
(0.138, 0.192]   1000 0.204
(0.192, 0.206]   1000 0.173
(0.206, 0.218]   1000 0.236
(0.218, 0.227]   1000 0.225
(0.227, 0.237]   1000 0.208
(0.237, 0.248]   1000 0.228
(0.248, 0.259]   1000 0.240
(0.259, 0.275]   1000 0.266
(0.275, 0.302]   1000 0.303
(0.302, 0.637]   1000 0.359
                              count  mean
yhat                                     
(0.11699999999999999, 0.192]  31931 0.195
(0.192, 0.205]                31931 0.198
(0.205, 0.216]                31930 0.212
(0.216, 0.226]                31931 0.221
(0.226, 0.236]                31931 0.225
(0.236, 0.247]                31930 0.230
(0.247, 0.259]                31931 0.248
(0.259, 0.275]                31930 0.260
(0.275, 0.302]                31931 0.292
(0.302, 0.728]                31931 0.349
\end{sphinxVerbatim}

\noindent\sphinxincludegraphics{{5.2.2_Modeling_Library_7_1}.png}

\end{sphinxuseclass}\end{sphinxVerbatimOutput}

\end{sphinxuseclass}

\part{SK\sphinxhyphen{}Learn \sphinxhyphen{} Logistic Regression}
\label{\detokenize{chapter5/5.2.2_Modeling_Library:sk-learn-logistic-regression}}
\sphinxAtStartPar
Scikit\sphinxhyphen{}Learn 에서도 Logistic Regression 을 지원합니다. 하지만 계수를 추정하는 방식이 Statsmodels 과는 다른데요. Scikit\sphinxhyphen{}Learn Logistic Regression 은 loss 함수를 만들고, 과대적합을 해결하기 위해 penalty term (L1/L2) 도 추가합니다. 이런 방식으로 Penalty Term 이 있는 Loss 함수를 최소화하는 방식을 계수를 찾을 때는 입력 피처의 스케일이 동일해야 의미가 있습니다. 아래 코드에서 입력 피쳐를 Scaling 하는 부분이 반드시 들어가야 합니다. 아래 Test 데이터의 결과가 Train 데이터보다 모델성능의 차이가 크지는 않습니다. 즉 overfitting(과대적합)이 심하지는 않아 보입니다.

\sphinxAtStartPar
SK\sphinxhyphen{}Learn 으로 만드는 모델은 절차가 아래와 같습니다.
\begin{enumerate}
\sphinxsetlistlabels{\arabic}{enumi}{enumii}{}{.}%
\item {} 
\sphinxAtStartPar
입력 피처 스케일링 (Loss 함수 + Penalty Term 로 훈련하는 모델만 해당)

\item {} 
\sphinxAtStartPar
하이퍼파라미터 튜닝 (성능을 최고로 만드는 하이퍼파라미터을 찾기)

\item {} 
\sphinxAtStartPar
테스트 데이터셋과 예측성능을 비교합니다 (오버피팅 여부 확인)

\item {} 
\sphinxAtStartPar
변수의 중요도 파악과 이해

\item {} 
\sphinxAtStartPar
예측값 만들기

\end{enumerate}

\begin{sphinxuseclass}{cell}\begin{sphinxVerbatimInput}

\begin{sphinxuseclass}{cell_input}
\begin{sphinxVerbatim}[commandchars=\\\{\}]
\PYG{k+kn}{from} \PYG{n+nn}{sklearn}\PYG{n+nn}{.}\PYG{n+nn}{linear\PYGZus{}model} \PYG{k+kn}{import} \PYG{n}{LogisticRegression}
\PYG{k+kn}{from} \PYG{n+nn}{sklearn}\PYG{n+nn}{.}\PYG{n+nn}{preprocessing} \PYG{k+kn}{import} \PYG{n}{StandardScaler}

\PYG{n}{feature\PYGZus{}list} \PYG{o}{=} \PYG{p}{[}\PYG{l+s+s1}{\PYGZsq{}}\PYG{l+s+s1}{price\PYGZus{}z}\PYG{l+s+s1}{\PYGZsq{}}\PYG{p}{,} \PYG{l+s+s1}{\PYGZsq{}}\PYG{l+s+s1}{volume\PYGZus{}z}\PYG{l+s+s1}{\PYGZsq{}}\PYG{p}{,} \PYG{l+s+s1}{\PYGZsq{}}\PYG{l+s+s1}{num\PYGZus{}high/close}\PYG{l+s+s1}{\PYGZsq{}}\PYG{p}{,} \PYG{l+s+s1}{\PYGZsq{}}\PYG{l+s+s1}{num\PYGZus{}long}\PYG{l+s+s1}{\PYGZsq{}}\PYG{p}{,} \PYG{l+s+s1}{\PYGZsq{}}\PYG{l+s+s1}{num\PYGZus{}z\PYGZgt{}1.96}\PYG{l+s+s1}{\PYGZsq{}}\PYG{p}{,} \PYG{l+s+s1}{\PYGZsq{}}\PYG{l+s+s1}{num\PYGZus{}win\PYGZus{}market}\PYG{l+s+s1}{\PYGZsq{}}\PYG{p}{,} \PYG{l+s+s1}{\PYGZsq{}}\PYG{l+s+s1}{pct\PYGZus{}win\PYGZus{}market}\PYG{l+s+s1}{\PYGZsq{}}\PYG{p}{,} \PYG{l+s+s1}{\PYGZsq{}}\PYG{l+s+s1}{return over sector}\PYG{l+s+s1}{\PYGZsq{}}\PYG{p}{]}

\PYG{n}{X} \PYG{o}{=} \PYG{n}{train}\PYG{p}{[}\PYG{n}{feature\PYGZus{}list}\PYG{p}{]}
\PYG{n}{y} \PYG{o}{=} \PYG{n}{train}\PYG{p}{[}\PYG{l+s+s1}{\PYGZsq{}}\PYG{l+s+s1}{target}\PYG{l+s+s1}{\PYGZsq{}}\PYG{p}{]}

\PYG{n}{X\PYGZus{}test} \PYG{o}{=} \PYG{n}{test}\PYG{p}{[}\PYG{n}{feature\PYGZus{}list}\PYG{p}{]}
\PYG{n}{y\PYGZus{}test} \PYG{o}{=} \PYG{n}{test}\PYG{p}{[}\PYG{l+s+s1}{\PYGZsq{}}\PYG{l+s+s1}{target}\PYG{l+s+s1}{\PYGZsq{}}\PYG{p}{]}

\PYG{c+c1}{\PYGZsh{} 입력 피처 표준화}
\PYG{n}{scaler} \PYG{o}{=} \PYG{n}{StandardScaler}\PYG{p}{(}\PYG{p}{)} \PYG{c+c1}{\PYGZsh{} 평균이 0 이고 편차가 1 가 되도록 피처 표준화}
\PYG{n}{scaler}\PYG{o}{.}\PYG{n}{fit\PYGZus{}transform}\PYG{p}{(}\PYG{n}{X}\PYG{p}{)}
\PYG{n}{scaler}\PYG{o}{.}\PYG{n}{transform}\PYG{p}{(}\PYG{n}{X\PYGZus{}test}\PYG{p}{)}

\PYG{n}{lr} \PYG{o}{=} \PYG{n}{LogisticRegression}\PYG{p}{(}\PYG{n}{fit\PYGZus{}intercept}\PYG{o}{=}\PYG{k+kc}{True}\PYG{p}{,} \PYG{n}{C}\PYG{o}{=}\PYG{l+m+mi}{1}\PYG{p}{)} \PYG{c+c1}{\PYGZsh{} 정해진 하이퍼파라미터를 가진 객체를 생성, C 값은 다중 공선성을 제거하기 위한 페널티의 가중치이며 디폴트는 L2(Ridge) 페널티}
\PYG{n}{lr}\PYG{o}{.}\PYG{n}{fit}\PYG{p}{(}\PYG{n}{X}\PYG{p}{,} \PYG{n}{y}\PYG{p}{)}
\PYG{n+nb}{print}\PYG{p}{(}\PYG{n}{lr}\PYG{o}{.}\PYG{n}{coef\PYGZus{}}\PYG{p}{)}
\PYG{n}{yhat} \PYG{o}{=} \PYG{n}{lr}\PYG{o}{.}\PYG{n}{predict\PYGZus{}proba}\PYG{p}{(}\PYG{n}{X}\PYG{p}{)}\PYG{p}{[}\PYG{p}{:}\PYG{p}{,}\PYG{l+m+mi}{1}\PYG{p}{]}
\PYG{n}{yhat\PYGZus{}test} \PYG{o}{=} \PYG{n}{lr}\PYG{o}{.}\PYG{n}{predict\PYGZus{}proba}\PYG{p}{(}\PYG{n}{X\PYGZus{}test}\PYG{p}{)}\PYG{p}{[}\PYG{p}{:}\PYG{p}{,}\PYG{l+m+mi}{1}\PYG{p}{]}

\PYG{n}{yhat} \PYG{o}{=} \PYG{n}{pd}\PYG{o}{.}\PYG{n}{Series}\PYG{p}{(}\PYG{n}{yhat}\PYG{p}{,} \PYG{n}{name}\PYG{o}{=}\PYG{l+s+s1}{\PYGZsq{}}\PYG{l+s+s1}{yhat}\PYG{l+s+s1}{\PYGZsq{}}\PYG{p}{,} \PYG{n}{index}\PYG{o}{=}\PYG{n}{y}\PYG{o}{.}\PYG{n}{index}\PYG{p}{)} 
\PYG{n}{yhat\PYGZus{}test} \PYG{o}{=} \PYG{n}{pd}\PYG{o}{.}\PYG{n}{Series}\PYG{p}{(}\PYG{n}{yhat\PYGZus{}test}\PYG{p}{,} \PYG{n}{name}\PYG{o}{=}\PYG{l+s+s1}{\PYGZsq{}}\PYG{l+s+s1}{yhat}\PYG{l+s+s1}{\PYGZsq{}}\PYG{p}{,} \PYG{n}{index}\PYG{o}{=}\PYG{n}{y\PYGZus{}test}\PYG{o}{.}\PYG{n}{index}\PYG{p}{)} 
\end{sphinxVerbatim}

\end{sphinxuseclass}\end{sphinxVerbatimInput}
\begin{sphinxVerbatimOutput}

\begin{sphinxuseclass}{cell_output}
\begin{sphinxVerbatim}[commandchars=\\\{\}]
[[\PYGZhy{}0.10997133  0.14023527  0.15290205 \PYGZhy{}0.03634754  0.04334864  0.03748485
   0.51234537 \PYGZhy{}0.51150234]]
\end{sphinxVerbatim}

\end{sphinxuseclass}\end{sphinxVerbatimOutput}

\end{sphinxuseclass}
\begin{sphinxuseclass}{cell}\begin{sphinxVerbatimInput}

\begin{sphinxuseclass}{cell_input}
\begin{sphinxVerbatim}[commandchars=\\\{\}]
\PYG{n}{perf}\PYG{p}{(}\PYG{n}{y}\PYG{p}{,} \PYG{n}{yhat}\PYG{p}{)}
\PYG{n}{perf}\PYG{p}{(}\PYG{n}{y\PYGZus{}test}\PYG{p}{,} \PYG{n}{yhat\PYGZus{}test}\PYG{p}{)}
\PYG{n}{plt}\PYG{o}{.}\PYG{n}{show}\PYG{p}{(}\PYG{p}{)}
\end{sphinxVerbatim}

\end{sphinxuseclass}\end{sphinxVerbatimInput}
\begin{sphinxVerbatimOutput}

\begin{sphinxuseclass}{cell_output}
\begin{sphinxVerbatim}[commandchars=\\\{\}]
                count  mean
yhat                       
(0.142, 0.194]   1000 0.175
(0.194, 0.208]   1000 0.201
(0.208, 0.219]   1000 0.224
(0.219, 0.228]   1000 0.236
(0.228, 0.238]   1000 0.217
(0.238, 0.248]   1000 0.234
(0.248, 0.26]    1000 0.270
(0.26, 0.275]    1000 0.249
(0.275, 0.3]     1000 0.294
(0.3, 0.632]     1000 0.342
                count  mean
yhat                       
(0.126, 0.194]  31931 0.187
(0.194, 0.207]  31931 0.207
(0.207, 0.218]  31930 0.217
(0.218, 0.228]  31931 0.218
(0.228, 0.237]  31931 0.226
(0.237, 0.247]  31930 0.235
(0.247, 0.259]  31931 0.254
(0.259, 0.274]  31930 0.258
(0.274, 0.299]  31931 0.289
(0.299, 0.652]  31931 0.338
\end{sphinxVerbatim}

\noindent\sphinxincludegraphics{{5.2.2_Modeling_Library_10_1}.png}

\end{sphinxuseclass}\end{sphinxVerbatimOutput}

\end{sphinxuseclass}

\part{SK\sphinxhyphen{}learn \sphinxhyphen{} Random Forerst}
\label{\detokenize{chapter5/5.2.2_Modeling_Library:sk-learn-random-forerst}}
\sphinxAtStartPar
Random Forest 는 Scikit\sphinxhyphen{}Learn 에서 인기있는 모델인데요. Decision\sphinxhyphen{}Tree(의사결정나무)의 문제점을 보완하고자 나온 개념입니다. 모델을 훈련시키기 위한 loss 함수와 Penalty term 이 없기 때문에 피쳐 스케일링이 불필요해서 쉽게 모델을 만들어 볼 수 있습니다. 보통 모델의 최소성능을 파악하기 위해 먼저 만들어보는 모델입니다.
sklearn 의 ensemble(앙상블) 모델군에서 RandomForestClassifier 을 불러옵니다. 그 다음 정해진 하이퍼파라미터(hyperparameter)를 가진 객체를 하나 생성합니다. 여기서 어떤 하이퍼파라미터로 객체를 생성하는가에 따라 모델의 성능이 결정되므로, 하이퍼파라미터 튜닝이라 절차가 필요합니다. 보통 최적의 하이퍼파라미터는 Grid Search 로 찾습니다. 그리고 fit 를 이용해서 데이터를 적용하면 됩니다. 예측값 생성은 predict 함수나 predict\_proba 함수로 할 수 있는데요. predict 함수는 0/1 값을 리턴하고, predict\_proba 함수는 ‘0 일 확률’/’1 일 확률’을 리턴합니다. 각 피처의 중요도를 그래프로 파악해보겠습니다. 이전 모델들에 비해 예측성능이 좋습니다.

\begin{sphinxuseclass}{cell}\begin{sphinxVerbatimInput}

\begin{sphinxuseclass}{cell_input}
\begin{sphinxVerbatim}[commandchars=\\\{\}]
\PYG{k+kn}{from} \PYG{n+nn}{sklearn}\PYG{n+nn}{.}\PYG{n+nn}{ensemble} \PYG{k+kn}{import} \PYG{n}{RandomForestClassifier}
\PYG{n}{rf} \PYG{o}{=} \PYG{n}{RandomForestClassifier}\PYG{p}{(}\PYG{n}{max\PYGZus{}depth}\PYG{o}{=}\PYG{l+m+mi}{4}\PYG{p}{,} \PYG{n}{min\PYGZus{}samples\PYGZus{}leaf}\PYG{o}{=}\PYG{l+m+mi}{30}\PYG{p}{)} \PYG{c+c1}{\PYGZsh{} 정해진 하이퍼파라미터를 가진 객체를 생성}

\PYG{n}{feature\PYGZus{}list} \PYG{o}{=} \PYG{p}{[}\PYG{l+s+s1}{\PYGZsq{}}\PYG{l+s+s1}{price\PYGZus{}z}\PYG{l+s+s1}{\PYGZsq{}}\PYG{p}{,} \PYG{l+s+s1}{\PYGZsq{}}\PYG{l+s+s1}{volume\PYGZus{}z}\PYG{l+s+s1}{\PYGZsq{}}\PYG{p}{,} \PYG{l+s+s1}{\PYGZsq{}}\PYG{l+s+s1}{num\PYGZus{}high/close}\PYG{l+s+s1}{\PYGZsq{}}\PYG{p}{,} \PYG{l+s+s1}{\PYGZsq{}}\PYG{l+s+s1}{num\PYGZus{}long}\PYG{l+s+s1}{\PYGZsq{}}\PYG{p}{,} \PYG{l+s+s1}{\PYGZsq{}}\PYG{l+s+s1}{num\PYGZus{}z\PYGZgt{}1.96}\PYG{l+s+s1}{\PYGZsq{}}\PYG{p}{,} \PYG{l+s+s1}{\PYGZsq{}}\PYG{l+s+s1}{num\PYGZus{}win\PYGZus{}market}\PYG{l+s+s1}{\PYGZsq{}}\PYG{p}{,} \PYG{l+s+s1}{\PYGZsq{}}\PYG{l+s+s1}{pct\PYGZus{}win\PYGZus{}market}\PYG{l+s+s1}{\PYGZsq{}}\PYG{p}{,} \PYG{l+s+s1}{\PYGZsq{}}\PYG{l+s+s1}{return over sector}\PYG{l+s+s1}{\PYGZsq{}}\PYG{p}{]}

\PYG{n}{X} \PYG{o}{=} \PYG{n}{train}\PYG{p}{[}\PYG{n}{feature\PYGZus{}list}\PYG{p}{]}
\PYG{n}{y} \PYG{o}{=} \PYG{n}{train}\PYG{p}{[}\PYG{l+s+s1}{\PYGZsq{}}\PYG{l+s+s1}{target}\PYG{l+s+s1}{\PYGZsq{}}\PYG{p}{]}
\PYG{n}{rf}\PYG{o}{.}\PYG{n}{fit}\PYG{p}{(}\PYG{n}{X}\PYG{p}{,} \PYG{n}{y}\PYG{p}{)}
\PYG{n}{yhat} \PYG{o}{=} \PYG{n}{rf}\PYG{o}{.}\PYG{n}{predict\PYGZus{}proba}\PYG{p}{(}\PYG{n}{X}\PYG{p}{)}\PYG{p}{[}\PYG{p}{:}\PYG{p}{,}\PYG{l+m+mi}{1}\PYG{p}{]} \PYG{c+c1}{\PYGZsh{} 첫번째 열은 0일 확률, 두번째 열은 1 일 확률 \PYGZhy{}\PYGZgt{} 1 일 확률을 저장}
\PYG{n}{yhat} \PYG{o}{=} \PYG{n}{pd}\PYG{o}{.}\PYG{n}{Series}\PYG{p}{(}\PYG{n}{yhat}\PYG{p}{,} \PYG{n}{name}\PYG{o}{=}\PYG{l+s+s1}{\PYGZsq{}}\PYG{l+s+s1}{yhat}\PYG{l+s+s1}{\PYGZsq{}}\PYG{p}{,} \PYG{n}{index}\PYG{o}{=}\PYG{n}{y}\PYG{o}{.}\PYG{n}{index}\PYG{p}{)} 

\PYG{n}{X\PYGZus{}test} \PYG{o}{=} \PYG{n}{test}\PYG{p}{[}\PYG{n}{feature\PYGZus{}list}\PYG{p}{]}
\PYG{n}{y\PYGZus{}test} \PYG{o}{=} \PYG{n}{test}\PYG{p}{[}\PYG{l+s+s1}{\PYGZsq{}}\PYG{l+s+s1}{target}\PYG{l+s+s1}{\PYGZsq{}}\PYG{p}{]}
\PYG{n}{yhat\PYGZus{}test} \PYG{o}{=} \PYG{n}{rf}\PYG{o}{.}\PYG{n}{predict\PYGZus{}proba}\PYG{p}{(}\PYG{n}{X\PYGZus{}test}\PYG{p}{)}\PYG{p}{[}\PYG{p}{:}\PYG{p}{,}\PYG{l+m+mi}{1}\PYG{p}{]} \PYG{c+c1}{\PYGZsh{} 첫번째 열은 0일 확률, 두번째 열은 1 일 확률 \PYGZhy{}\PYGZgt{} 1 일 확률을 저장}
\PYG{n}{yhat\PYGZus{}test} \PYG{o}{=} \PYG{n}{pd}\PYG{o}{.}\PYG{n}{Series}\PYG{p}{(}\PYG{n}{yhat\PYGZus{}test}\PYG{p}{,} \PYG{n}{name}\PYG{o}{=}\PYG{l+s+s1}{\PYGZsq{}}\PYG{l+s+s1}{yhat}\PYG{l+s+s1}{\PYGZsq{}}\PYG{p}{,} \PYG{n}{index}\PYG{o}{=}\PYG{n}{y\PYGZus{}test}\PYG{o}{.}\PYG{n}{index}\PYG{p}{)} 

\PYG{n}{importances} \PYG{o}{=} \PYG{n}{rf}\PYG{o}{.}\PYG{n}{feature\PYGZus{}importances\PYGZus{}}
\PYG{n}{sorted\PYGZus{}indices} \PYG{o}{=} \PYG{n}{np}\PYG{o}{.}\PYG{n}{argsort}\PYG{p}{(}\PYG{n}{importances}\PYG{p}{)}\PYG{p}{[}\PYG{p}{:}\PYG{p}{:}\PYG{o}{\PYGZhy{}}\PYG{l+m+mi}{1}\PYG{p}{]}

\PYG{k+kn}{import} \PYG{n+nn}{matplotlib}\PYG{n+nn}{.}\PYG{n+nn}{pyplot} \PYG{k}{as} \PYG{n+nn}{plt}
 
\PYG{n}{plt}\PYG{o}{.}\PYG{n}{title}\PYG{p}{(}\PYG{l+s+s1}{\PYGZsq{}}\PYG{l+s+s1}{Feature Importance}\PYG{l+s+s1}{\PYGZsq{}}\PYG{p}{)}
\PYG{n}{plt}\PYG{o}{.}\PYG{n}{bar}\PYG{p}{(}\PYG{n+nb}{range}\PYG{p}{(}\PYG{n}{X}\PYG{o}{.}\PYG{n}{shape}\PYG{p}{[}\PYG{l+m+mi}{1}\PYG{p}{]}\PYG{p}{)}\PYG{p}{,} \PYG{n}{importances}\PYG{p}{[}\PYG{n}{sorted\PYGZus{}indices}\PYG{p}{]}\PYG{p}{,} \PYG{n}{align}\PYG{o}{=}\PYG{l+s+s1}{\PYGZsq{}}\PYG{l+s+s1}{center}\PYG{l+s+s1}{\PYGZsq{}}\PYG{p}{)}
\PYG{n}{plt}\PYG{o}{.}\PYG{n}{xticks}\PYG{p}{(}\PYG{n+nb}{range}\PYG{p}{(}\PYG{n}{X}\PYG{o}{.}\PYG{n}{shape}\PYG{p}{[}\PYG{l+m+mi}{1}\PYG{p}{]}\PYG{p}{)}\PYG{p}{,} \PYG{n}{X}\PYG{o}{.}\PYG{n}{columns}\PYG{p}{[}\PYG{n}{sorted\PYGZus{}indices}\PYG{p}{]}\PYG{p}{,} \PYG{n}{rotation}\PYG{o}{=}\PYG{l+m+mi}{90}\PYG{p}{)}
\PYG{n}{plt}\PYG{o}{.}\PYG{n}{show}\PYG{p}{(}\PYG{p}{)}
\end{sphinxVerbatim}

\end{sphinxuseclass}\end{sphinxVerbatimInput}
\begin{sphinxVerbatimOutput}

\begin{sphinxuseclass}{cell_output}
\noindent\sphinxincludegraphics{{5.2.2_Modeling_Library_12_0}.png}

\end{sphinxuseclass}\end{sphinxVerbatimOutput}

\end{sphinxuseclass}
\begin{sphinxuseclass}{cell}\begin{sphinxVerbatimInput}

\begin{sphinxuseclass}{cell_input}
\begin{sphinxVerbatim}[commandchars=\\\{\}]
\PYG{n}{perf}\PYG{p}{(}\PYG{n}{y}\PYG{p}{,} \PYG{n}{yhat}\PYG{p}{)}
\PYG{n}{perf}\PYG{p}{(}\PYG{n}{y\PYGZus{}test}\PYG{p}{,} \PYG{n}{yhat\PYGZus{}test}\PYG{p}{)}
\PYG{n}{plt}\PYG{o}{.}\PYG{n}{show}\PYG{p}{(}\PYG{p}{)}
\end{sphinxVerbatim}

\end{sphinxuseclass}\end{sphinxVerbatimInput}
\begin{sphinxVerbatimOutput}

\begin{sphinxuseclass}{cell_output}
\begin{sphinxVerbatim}[commandchars=\\\{\}]
                count  mean
yhat                       
(0.167, 0.203]   1000 0.131
(0.203, 0.212]   1000 0.152
(0.212, 0.218]   1000 0.182
(0.218, 0.226]   1000 0.209
(0.226, 0.233]   1000 0.221
(0.233, 0.242]   1000 0.241
(0.242, 0.255]   1000 0.263
(0.255, 0.274]   1000 0.295
(0.274, 0.304]   1000 0.313
(0.304, 0.469]   1000 0.435
                count  mean
yhat                       
(0.164, 0.203]  31932 0.155
(0.203, 0.212]  31930 0.180
(0.212, 0.218]  31930 0.187
(0.218, 0.225]  31931 0.204
(0.225, 0.233]  31931 0.225
(0.233, 0.241]  31931 0.238
(0.241, 0.254]  31930 0.268
(0.254, 0.274]  31930 0.287
(0.274, 0.304]  31931 0.308
(0.304, 0.487]  31931 0.378
\end{sphinxVerbatim}

\noindent\sphinxincludegraphics{{5.2.2_Modeling_Library_13_1}.png}

\end{sphinxuseclass}\end{sphinxVerbatimOutput}

\end{sphinxuseclass}
\begin{sphinxuseclass}{cell}\begin{sphinxVerbatimInput}

\begin{sphinxuseclass}{cell_input}
\begin{sphinxVerbatim}[commandchars=\\\{\}]
\PYG{k+kn}{import} \PYG{n+nn}{FinanceDataReader} \PYG{k}{as} \PYG{n+nn}{fdr}
\PYG{o}{\PYGZpc{}}\PYG{k}{matplotlib} inline
\PYG{k+kn}{import} \PYG{n+nn}{matplotlib}\PYG{n+nn}{.}\PYG{n+nn}{pyplot} \PYG{k}{as} \PYG{n+nn}{plt}
\PYG{k+kn}{import} \PYG{n+nn}{pandas} \PYG{k}{as} \PYG{n+nn}{pd}
\PYG{k+kn}{import} \PYG{n+nn}{numpy} \PYG{k}{as} \PYG{n+nn}{np}
\PYG{k+kn}{import} \PYG{n+nn}{warnings}
\PYG{k+kn}{import} \PYG{n+nn}{pickle}
\PYG{n}{warnings}\PYG{o}{.}\PYG{n}{filterwarnings}\PYG{p}{(}\PYG{l+s+s1}{\PYGZsq{}}\PYG{l+s+s1}{ignore}\PYG{l+s+s1}{\PYGZsq{}}\PYG{p}{)}
\PYG{n}{pd}\PYG{o}{.}\PYG{n}{options}\PYG{o}{.}\PYG{n}{display}\PYG{o}{.}\PYG{n}{float\PYGZus{}format} \PYG{o}{=} \PYG{l+s+s1}{\PYGZsq{}}\PYG{l+s+si}{\PYGZob{}:,.3f\PYGZcb{}}\PYG{l+s+s1}{\PYGZsq{}}\PYG{o}{.}\PYG{n}{format}
\end{sphinxVerbatim}

\end{sphinxuseclass}\end{sphinxVerbatimInput}

\end{sphinxuseclass}

\part{종목 선정 모델 개발}
\label{\detokenize{chapter5/5.2.3_GAM:id1}}\label{\detokenize{chapter5/5.2.3_GAM::doc}}
\sphinxAtStartPar
선형모델에 대한 중요한 가정과 설명은 다음 절에서 추가로 설명드리겠으나,  수익률에 따라 단조 증가나 감소의 형태를 보이지 않는 피쳐(설명변수)는 변형을 해야 선형모형에서 더 유의미하게 사용될 수 있습니다.  주로 Binning (오름차순으로 정렬 후, 여러개 구간으로 분리) 을 통하여 이런 비선형적인 관계를 선형적으로 변경합니다. 2차 함수나 로그함수 등을 이용해 선형적으로 변경할 수 도 있습니다. 우리는 앞서 수익율과 피쳐사이에 선형적인 관계를 가지지 않는 가설(예: 섹터의 평균 수익률 대비 종목 수익률)들 이 있었습니다. 이런 피처들에 대하여 Binning 없이 적합할 수 있는 모델이 일반화가법모형(Generalized Additive Model) 입니다. 또한 가설 검정에서는 5 영업일 동안의 최대 수익률을 예측변수로 이용했으나, 모델의 overfitting (과대적합) 문제를 최소화하기 위하여, 예측값을 이진값(0/1)으로 치환한 후, 로지스틱 일반화가법모형(Logistic Generalized Additive Model) 을 구현합니다. 로지스틱 회귀모형은 \(log(odds) = a0 + a1*x1 + a2*x2 …\)  으로 표현할 수 있는데요. 여기서 X 를 여러개의 spline 로 함수로 만든 후, 다시 합하여 X 와 \(log(odds)\) 의  비선형적관계를 표현할 수 있도록 한 것이  Logistic GAM 입니다.  이 모델의 구현은 Statsmodels 에서 가능합니다만, pyGAM 패키지는 자동으로 하이퍼파라미터를 찾는 기능이 있어 편리합니다. GAM 을 선택한 다른 이유는 피처사이에 상호작용이 크지 않을 것이라는 가정이 있습니다. 무엇보다도 좋은 점은 모델이 왜 이 종목을 선택했는지에 대한 해석이 가능합니다. 향후, 모델의 예측력이 저하되는 경우 어떤 피처가 원인인지도 파악이 가능합니다.

\sphinxAtStartPar
단순히, 스코어가 높은 모든 종목을 매수하는 것이 아니라,  오늘의 종가 수익률과 주가를 고려하여 기본적인 필터링을 합니다. 분석결과 종가 수익률은 높고, 최근 20일 대비 가격이 낮은 종목을 매수하면 리스크가 적은 것으로 판단됩니다.

\sphinxAtStartPar
이번절에서는 책에서 종목선정을 위해 사용할 GAM 모델을 개발하겠습니다. 아나콘다 프롬프트에서 conda install \sphinxhyphen{}c conda\sphinxhyphen{}forge pygam 로 설치를 해 줍니다. 관련 링크 \sphinxurl{https://anaconda.org/conda-forge/pygam}

\sphinxAtStartPar
모델링을 위해 준비한 데이터를 읽습니다. 그리고 모델의 오버피팅을 최소화하기 위하여 타겟변수를 0 과 1 로 치환합니다. 5\% 익절은 다음과 같이 데이터로 표현할 수 있습니다. \sphinxhyphen{} ‘max\_close’ 가 5\% 이상일 때 1, 아니면 0. 파이썬 코드는 아래와 같습니다.

\begin{sphinxVerbatim}[commandchars=\\\{\}]
\PYG{n}{np}\PYG{o}{.}\PYG{n}{where}\PYG{p}{(}\PYG{n}{feature\PYGZus{}all}\PYG{p}{[}\PYG{l+s+s1}{\PYGZsq{}}\PYG{l+s+s1}{max\PYGZus{}close}\PYG{l+s+s1}{\PYGZsq{}}\PYG{p}{]}\PYG{o}{\PYGZgt{}}\PYG{o}{=} \PYG{l+m+mf}{1.05}\PYG{p}{,} \PYG{l+m+mi}{1}\PYG{p}{,} \PYG{l+m+mi}{0}\PYG{p}{)}
\end{sphinxVerbatim}

\sphinxAtStartPar
타겟 변수 \sphinxhyphen{} ‘target’ 값이 1 인 비율을 보니, 약 24\% 입니다. 타겟변수의 비율이 너무 적으면 모델 트레이닝이 어렵습니다.

\begin{sphinxuseclass}{cell}\begin{sphinxVerbatimInput}

\begin{sphinxuseclass}{cell_input}
\begin{sphinxVerbatim}[commandchars=\\\{\}]
\PYG{n}{feature\PYGZus{}all} \PYG{o}{=} \PYG{n}{pd}\PYG{o}{.}\PYG{n}{read\PYGZus{}pickle}\PYG{p}{(}\PYG{l+s+s1}{\PYGZsq{}}\PYG{l+s+s1}{feature\PYGZus{}all.pkl}\PYG{l+s+s1}{\PYGZsq{}}\PYG{p}{)} 
\PYG{n}{feature\PYGZus{}all}\PYG{p}{[}\PYG{l+s+s1}{\PYGZsq{}}\PYG{l+s+s1}{target}\PYG{l+s+s1}{\PYGZsq{}}\PYG{p}{]} \PYG{o}{=} \PYG{n}{np}\PYG{o}{.}\PYG{n}{where}\PYG{p}{(}\PYG{n}{feature\PYGZus{}all}\PYG{p}{[}\PYG{l+s+s1}{\PYGZsq{}}\PYG{l+s+s1}{max\PYGZus{}close}\PYG{l+s+s1}{\PYGZsq{}}\PYG{p}{]}\PYG{o}{\PYGZgt{}}\PYG{o}{=} \PYG{l+m+mf}{1.05}\PYG{p}{,} \PYG{l+m+mi}{1}\PYG{p}{,} \PYG{l+m+mi}{0}\PYG{p}{)}
\PYG{n}{target} \PYG{o}{=} \PYG{n}{feature\PYGZus{}all}\PYG{p}{[}\PYG{l+s+s1}{\PYGZsq{}}\PYG{l+s+s1}{target}\PYG{l+s+s1}{\PYGZsq{}}\PYG{p}{]}\PYG{o}{.}\PYG{n}{mean}\PYG{p}{(}\PYG{p}{)}
\PYG{n+nb}{print}\PYG{p}{(}\PYG{l+s+sa}{f}\PYG{l+s+s1}{\PYGZsq{}}\PYG{l+s+s1}{\PYGZpc{} of target:}\PYG{l+s+si}{\PYGZob{}}\PYG{n}{target}\PYG{l+s+si}{:}\PYG{l+s+s1}{ 5.1\PYGZpc{}}\PYG{l+s+si}{\PYGZcb{}}\PYG{l+s+s1}{\PYGZsq{}}\PYG{p}{)}
\end{sphinxVerbatim}

\end{sphinxuseclass}\end{sphinxVerbatimInput}
\begin{sphinxVerbatimOutput}

\begin{sphinxuseclass}{cell_output}
\begin{sphinxVerbatim}[commandchars=\\\{\}]
\PYGZpc{} of target: 24.3\PYGZpc{}
\end{sphinxVerbatim}

\end{sphinxuseclass}\end{sphinxVerbatimOutput}

\end{sphinxuseclass}
\sphinxAtStartPar
 날짜와 종목은 모델의 입력피처가 아닙니다. 편의를 위해 제거하거나 인덱스로 처리합니다. 모델 트레이닝 용도로 10,000 개 샘플을 뽑아 예측모델을 만들고, 나머지 데이터는 테스트(혹은 백테스팅)를 하겠습니다.

\begin{sphinxuseclass}{cell}\begin{sphinxVerbatimInput}

\begin{sphinxuseclass}{cell_input}
\begin{sphinxVerbatim}[commandchars=\\\{\}]
\PYG{n}{mdl\PYGZus{}all} \PYG{o}{=} \PYG{n}{feature\PYGZus{}all}\PYG{o}{.}\PYG{n}{set\PYGZus{}index}\PYG{p}{(}\PYG{p}{[}\PYG{n}{feature\PYGZus{}all}\PYG{o}{.}\PYG{n}{index}\PYG{p}{,}\PYG{l+s+s1}{\PYGZsq{}}\PYG{l+s+s1}{code}\PYG{l+s+s1}{\PYGZsq{}}\PYG{p}{]}\PYG{p}{)}

\PYG{n}{train} \PYG{o}{=} \PYG{n}{mdl\PYGZus{}all}\PYG{o}{.}\PYG{n}{sample}\PYG{p}{(}\PYG{l+m+mi}{10000}\PYG{p}{,} \PYG{n}{random\PYGZus{}state}\PYG{o}{=}\PYG{l+m+mi}{124}\PYG{p}{)}
\PYG{n}{test} \PYG{o}{=} \PYG{n}{mdl\PYGZus{}all}\PYG{o}{.}\PYG{n}{loc}\PYG{p}{[}\PYG{o}{\PYGZti{}}\PYG{n}{mdl\PYGZus{}all}\PYG{o}{.}\PYG{n}{index}\PYG{o}{.}\PYG{n}{isin}\PYG{p}{(}\PYG{n}{train}\PYG{o}{.}\PYG{n}{index}\PYG{p}{)}\PYG{p}{]}
\PYG{n+nb}{print}\PYG{p}{(}\PYG{n+nb}{len}\PYG{p}{(}\PYG{n}{train}\PYG{p}{)}\PYG{p}{,} \PYG{n+nb}{len}\PYG{p}{(}\PYG{n}{test}\PYG{p}{)}\PYG{p}{)}
\end{sphinxVerbatim}

\end{sphinxuseclass}\end{sphinxVerbatimInput}
\begin{sphinxVerbatimOutput}

\begin{sphinxuseclass}{cell_output}
\begin{sphinxVerbatim}[commandchars=\\\{\}]
10000 319307
\end{sphinxVerbatim}

\end{sphinxuseclass}\end{sphinxVerbatimOutput}

\end{sphinxuseclass}
\sphinxAtStartPar
입력 피처의 갯수와 데이터타입을 확인합니다.

\begin{sphinxuseclass}{cell}\begin{sphinxVerbatimInput}

\begin{sphinxuseclass}{cell_input}
\begin{sphinxVerbatim}[commandchars=\\\{\}]
\PYG{n}{train}\PYG{o}{.}\PYG{n}{info}\PYG{p}{(}\PYG{p}{)}
\end{sphinxVerbatim}

\end{sphinxuseclass}\end{sphinxVerbatimInput}
\begin{sphinxVerbatimOutput}

\begin{sphinxuseclass}{cell_output}
\begin{sphinxVerbatim}[commandchars=\\\{\}]
\PYGZlt{}class \PYGZsq{}pandas.core.frame.DataFrame\PYGZsq{}\PYGZgt{}
MultiIndex: 10000 entries, (\PYGZsq{}2021\PYGZhy{}10\PYGZhy{}22\PYGZsq{}, \PYGZsq{}312610\PYGZsq{}) to (\PYGZsq{}2021\PYGZhy{}04\PYGZhy{}28\PYGZsq{}, \PYGZsq{}011320\PYGZsq{})
Data columns (total 15 columns):
 \PYGZsh{}   Column              Non\PYGZhy{}Null Count  Dtype  
\PYGZhy{}\PYGZhy{}\PYGZhy{}  \PYGZhy{}\PYGZhy{}\PYGZhy{}\PYGZhy{}\PYGZhy{}\PYGZhy{}              \PYGZhy{}\PYGZhy{}\PYGZhy{}\PYGZhy{}\PYGZhy{}\PYGZhy{}\PYGZhy{}\PYGZhy{}\PYGZhy{}\PYGZhy{}\PYGZhy{}\PYGZhy{}\PYGZhy{}\PYGZhy{}  \PYGZhy{}\PYGZhy{}\PYGZhy{}\PYGZhy{}\PYGZhy{}  
 0   sector              10000 non\PYGZhy{}null  object 
 1   return              10000 non\PYGZhy{}null  float64
 2   kosdaq\PYGZus{}return       10000 non\PYGZhy{}null  float64
 3   price\PYGZus{}z             10000 non\PYGZhy{}null  float64
 4   volume\PYGZus{}z            10000 non\PYGZhy{}null  float64
 5   num\PYGZus{}high/close      10000 non\PYGZhy{}null  float64
 6   num\PYGZus{}long            10000 non\PYGZhy{}null  float64
 7   num\PYGZus{}z\PYGZgt{}1.96          10000 non\PYGZhy{}null  float64
 8   num\PYGZus{}win\PYGZus{}market      10000 non\PYGZhy{}null  float64
 9   pct\PYGZus{}win\PYGZus{}market      10000 non\PYGZhy{}null  float64
 10  return over sector  10000 non\PYGZhy{}null  float64
 11  max\PYGZus{}close           10000 non\PYGZhy{}null  float64
 12  mean\PYGZus{}close          10000 non\PYGZhy{}null  float64
 13  min\PYGZus{}close           10000 non\PYGZhy{}null  float64
 14  target              10000 non\PYGZhy{}null  int32  
dtypes: float64(13), int32(1), object(1)
memory usage: 1.2+ MB
\end{sphinxVerbatim}

\end{sphinxuseclass}\end{sphinxVerbatimOutput}

\end{sphinxuseclass}
\sphinxAtStartPar
 각 변수별로 다른 ‘lambda’ (Wiggliness Penalty Weight) 을 적용해서 grid Search 를 합니다. spline 수는 20 이 default 값입니다. spline 수는 고정하고 lambda의 최적 조합을 찾거나, lambda 를 고정하고, spline 수의 최적 조합을 찾는 것이 현실적이고, 두 하이퍼파라미터를 동시에 조합하여 grid Search 하는 것은 시간이 많이 걸립니다. 다양한 시도를 통하여 더 좋은 모델을 구현할 수 있겠으나, 이 책에서는 grid search 로 변수별 최적의 lambda 를 찾는 것으로 모델을 완성합니다. P value 가 크게 나타나는 입력변수는 제거하는 것이 좋겠습니다.

\begin{sphinxuseclass}{cell}\begin{sphinxVerbatimInput}

\begin{sphinxuseclass}{cell_input}
\begin{sphinxVerbatim}[commandchars=\\\{\}]
\PYG{k+kn}{from} \PYG{n+nn}{pygam} \PYG{k+kn}{import} \PYG{n}{LogisticGAM}\PYG{p}{,} \PYG{n}{s}\PYG{p}{,} \PYG{n}{f}\PYG{p}{,} \PYG{n}{te}\PYG{p}{,} \PYG{n}{l}
\PYG{k+kn}{from} \PYG{n+nn}{sklearn}\PYG{n+nn}{.}\PYG{n+nn}{metrics} \PYG{k+kn}{import} \PYG{n}{accuracy\PYGZus{}score}
\PYG{k+kn}{from} \PYG{n+nn}{sklearn}\PYG{n+nn}{.}\PYG{n+nn}{metrics} \PYG{k+kn}{import} \PYG{n}{log\PYGZus{}loss}

\PYG{n}{feature\PYGZus{}list} \PYG{o}{=} \PYG{p}{[}\PYG{l+s+s1}{\PYGZsq{}}\PYG{l+s+s1}{price\PYGZus{}z}\PYG{l+s+s1}{\PYGZsq{}}\PYG{p}{,}\PYG{l+s+s1}{\PYGZsq{}}\PYG{l+s+s1}{volume\PYGZus{}z}\PYG{l+s+s1}{\PYGZsq{}}\PYG{p}{,}\PYG{l+s+s1}{\PYGZsq{}}\PYG{l+s+s1}{num\PYGZus{}high/close}\PYG{l+s+s1}{\PYGZsq{}}\PYG{p}{,}\PYG{l+s+s1}{\PYGZsq{}}\PYG{l+s+s1}{num\PYGZus{}win\PYGZus{}market}\PYG{l+s+s1}{\PYGZsq{}}\PYG{p}{,}\PYG{l+s+s1}{\PYGZsq{}}\PYG{l+s+s1}{pct\PYGZus{}win\PYGZus{}market}\PYG{l+s+s1}{\PYGZsq{}}\PYG{p}{,}\PYG{l+s+s1}{\PYGZsq{}}\PYG{l+s+s1}{return over sector}\PYG{l+s+s1}{\PYGZsq{}}\PYG{p}{]}
\PYG{n}{X} \PYG{o}{=} \PYG{n}{train}\PYG{p}{[}\PYG{n}{feature\PYGZus{}list}\PYG{p}{]}
\PYG{n}{y} \PYG{o}{=} \PYG{n}{train}\PYG{p}{[}\PYG{l+s+s1}{\PYGZsq{}}\PYG{l+s+s1}{target}\PYG{l+s+s1}{\PYGZsq{}}\PYG{p}{]}
\PYG{n}{X\PYGZus{}test} \PYG{o}{=} \PYG{n}{test}\PYG{p}{[}\PYG{n}{feature\PYGZus{}list}\PYG{p}{]}
\PYG{n}{y\PYGZus{}test} \PYG{o}{=} \PYG{n}{test}\PYG{p}{[}\PYG{l+s+s1}{\PYGZsq{}}\PYG{l+s+s1}{target}\PYG{l+s+s1}{\PYGZsq{}}\PYG{p}{]}

\PYG{c+c1}{\PYGZsh{} 하이퍼파라미터 설정 N 개의 변수면 (M x N) 개의 리스트로 생성함으로써 변수별로 다른 하이퍼파라미터 테스트 가능. }
\PYG{c+c1}{\PYGZsh{} M 개만 1D 리스트를 만들면 동일한 lambda 른 모든 변수에 적용함.}
\PYG{n}{lam\PYGZus{}list} \PYG{o}{=} \PYG{p}{[}\PYG{n}{np}\PYG{o}{.}\PYG{n}{logspace}\PYG{p}{(}\PYG{l+m+mi}{0}\PYG{p}{,} \PYG{l+m+mi}{3}\PYG{p}{,} \PYG{l+m+mi}{2}\PYG{p}{)}\PYG{p}{]}\PYG{o}{*}\PYG{l+m+mi}{8}
   
\PYG{n}{gam} \PYG{o}{=} \PYG{n}{LogisticGAM}\PYG{p}{(}\PYG{n}{te}\PYG{p}{(}\PYG{l+m+mi}{0}\PYG{p}{,} \PYG{l+m+mi}{1}\PYG{p}{,} \PYG{n}{n\PYGZus{}splines}\PYG{o}{=}\PYG{l+m+mi}{5}\PYG{p}{)} \PYG{o}{+} \PYG{n}{s}\PYG{p}{(}\PYG{l+m+mi}{1}\PYG{p}{)} \PYG{o}{+} \PYG{n}{s}\PYG{p}{(}\PYG{l+m+mi}{2}\PYG{p}{)} \PYG{o}{+} \PYG{n}{s}\PYG{p}{(}\PYG{l+m+mi}{3}\PYG{p}{)} \PYG{o}{+} \PYG{n}{s}\PYG{p}{(}\PYG{l+m+mi}{4}\PYG{p}{)} \PYG{o}{+} \PYG{n}{te}\PYG{p}{(}\PYG{l+m+mi}{4}\PYG{p}{,} \PYG{l+m+mi}{5}\PYG{p}{,} \PYG{n}{n\PYGZus{}splines}\PYG{o}{=}\PYG{l+m+mi}{5}\PYG{p}{)}\PYG{p}{)}\PYG{o}{.}\PYG{n}{gridsearch}\PYG{p}{(}\PYG{n}{X}\PYG{o}{.}\PYG{n}{to\PYGZus{}numpy}\PYG{p}{(}\PYG{p}{)}\PYG{p}{,} \PYG{n}{y}\PYG{o}{.}\PYG{n}{to\PYGZus{}numpy}\PYG{p}{(}\PYG{p}{)}\PYG{p}{,} \PYG{n}{lam}\PYG{o}{=}\PYG{n}{lam\PYGZus{}list}\PYG{p}{)} 

\PYG{n+nb}{print}\PYG{p}{(}\PYG{n}{gam}\PYG{o}{.}\PYG{n}{summary}\PYG{p}{(}\PYG{p}{)}\PYG{p}{)}
\PYG{n+nb}{print}\PYG{p}{(}\PYG{n}{gam}\PYG{o}{.}\PYG{n}{accuracy}\PYG{p}{(}\PYG{n}{X\PYGZus{}test}\PYG{p}{,} \PYG{n}{y\PYGZus{}test}\PYG{p}{)}\PYG{p}{)}
\end{sphinxVerbatim}

\end{sphinxuseclass}\end{sphinxVerbatimInput}
\begin{sphinxVerbatimOutput}

\begin{sphinxuseclass}{cell_output}
\begin{sphinxVerbatim}[commandchars=\\\{\}]
100\PYGZpc{} (256 of 256) |\PYGZsh{}\PYGZsh{}\PYGZsh{}\PYGZsh{}\PYGZsh{}\PYGZsh{}\PYGZsh{}\PYGZsh{}\PYGZsh{}\PYGZsh{}\PYGZsh{}\PYGZsh{}\PYGZsh{}\PYGZsh{}\PYGZsh{}\PYGZsh{}\PYGZsh{}\PYGZsh{}\PYGZsh{}\PYGZsh{}\PYGZsh{}\PYGZsh{}| Elapsed Time: 0:03:32 Time:  0:03:32
\end{sphinxVerbatim}

\begin{sphinxVerbatim}[commandchars=\\\{\}]
LogisticGAM                                                                                               
=============================================== ==========================================================
Distribution:                      BinomialDist Effective DoF:                                     21.7127
Link Function:                        LogitLink Log Likelihood:                                 \PYGZhy{}5473.5193
Number of Samples:                        10000 AIC:                                            10990.4639
                                                AICc:                                           10990.5719
                                                UBRE:                                               3.1008
                                                Scale:                                                 1.0
                                                Pseudo R\PYGZhy{}Squared:                                   0.0197
==========================================================================================================
Feature Function                  Lambda               Rank         EDoF         P \PYGZgt{} x        Sig. Code   
================================= ==================== ============ ============ ============ ============
te(0, 1)                          [1. 1.]              25           6.2          1.11e\PYGZhy{}15     ***         
s(1)                              [1000.]              20           0.1          1.39e\PYGZhy{}04     ***         
s(2)                              [1000.]              20           1.8          1.65e\PYGZhy{}01                 
s(3)                              [1000.]              20           2.7          4.35e\PYGZhy{}02     *           
s(4)                              [1.]                 20           8.9          1.11e\PYGZhy{}11     ***         
te(4, 5)                          [1. 1.]              25           2.0          1.80e\PYGZhy{}01                 
intercept                                              1            0.0          2.07e\PYGZhy{}01                 
==========================================================================================================
Significance codes:  0 \PYGZsq{}***\PYGZsq{} 0.001 \PYGZsq{}**\PYGZsq{} 0.01 \PYGZsq{}*\PYGZsq{} 0.05 \PYGZsq{}.\PYGZsq{} 0.1 \PYGZsq{} \PYGZsq{} 1

WARNING: Fitting splines and a linear function to a feature introduces a model identifiability problem
         which can cause p\PYGZhy{}values to appear significant when they are not.

WARNING: p\PYGZhy{}values calculated in this manner behave correctly for un\PYGZhy{}penalized models or models with
         known smoothing parameters, but when smoothing parameters have been estimated, the p\PYGZhy{}values
         are typically lower than they should be, meaning that the tests reject the null too readily.
None
0.7571130530379218
\end{sphinxVerbatim}

\end{sphinxuseclass}\end{sphinxVerbatimOutput}

\end{sphinxuseclass}
\begin{sphinxuseclass}{cell}\begin{sphinxVerbatimInput}

\begin{sphinxuseclass}{cell_input}
\begin{sphinxVerbatim}[commandchars=\\\{\}]
\PYG{k}{for} \PYG{n}{i}\PYG{p}{,} \PYG{n}{term} \PYG{o+ow}{in} \PYG{n+nb}{enumerate}\PYG{p}{(}\PYG{n}{gam}\PYG{o}{.}\PYG{n}{terms}\PYG{p}{)}\PYG{p}{:}
    \PYG{n+nb}{print}\PYG{p}{(}\PYG{n}{i}\PYG{p}{,} \PYG{n}{term}\PYG{p}{)}
\end{sphinxVerbatim}

\end{sphinxuseclass}\end{sphinxVerbatimInput}
\begin{sphinxVerbatimOutput}

\begin{sphinxuseclass}{cell_output}
\begin{sphinxVerbatim}[commandchars=\\\{\}]
0 tensor\PYGZus{}term
1 spline\PYGZus{}term
2 spline\PYGZus{}term
3 spline\PYGZus{}term
4 spline\PYGZus{}term
5 tensor\PYGZus{}term
6 intercept\PYGZus{}term
\end{sphinxVerbatim}

\end{sphinxuseclass}\end{sphinxVerbatimOutput}

\end{sphinxuseclass}
\begin{sphinxuseclass}{cell}\begin{sphinxVerbatimInput}

\begin{sphinxuseclass}{cell_input}
\begin{sphinxVerbatim}[commandchars=\\\{\}]
\PYG{n}{feature\PYGZus{}list} \PYG{o}{=}  \PYG{p}{[}\PYG{l+s+s1}{\PYGZsq{}}\PYG{l+s+s1}{price\PYGZus{}z}\PYG{l+s+s1}{\PYGZsq{}}\PYG{p}{,}\PYG{l+s+s1}{\PYGZsq{}}\PYG{l+s+s1}{volume\PYGZus{}z}\PYG{l+s+s1}{\PYGZsq{}}\PYG{p}{,}\PYG{l+s+s1}{\PYGZsq{}}\PYG{l+s+s1}{num\PYGZus{}high/close}\PYG{l+s+s1}{\PYGZsq{}}\PYG{p}{,}\PYG{l+s+s1}{\PYGZsq{}}\PYG{l+s+s1}{num\PYGZus{}win\PYGZus{}market}\PYG{l+s+s1}{\PYGZsq{}}\PYG{p}{,}\PYG{l+s+s1}{\PYGZsq{}}\PYG{l+s+s1}{pct\PYGZus{}win\PYGZus{}market}\PYG{l+s+s1}{\PYGZsq{}}\PYG{p}{,}\PYG{l+s+s1}{\PYGZsq{}}\PYG{l+s+s1}{return over sector}\PYG{l+s+s1}{\PYGZsq{}}\PYG{p}{]}
\PYG{k}{for} \PYG{n}{i}\PYG{p}{,} \PYG{n}{term} \PYG{o+ow}{in} \PYG{n+nb}{enumerate}\PYG{p}{(}\PYG{n}{gam}\PYG{o}{.}\PYG{n}{terms}\PYG{p}{)}\PYG{p}{:}
    
    \PYG{k}{if} \PYG{n}{i}\PYG{o}{\PYGZgt{}}\PYG{o}{=}\PYG{l+m+mi}{1} \PYG{o+ow}{and} \PYG{n}{i}\PYG{o}{\PYGZlt{}}\PYG{o}{=}\PYG{l+m+mi}{4}\PYG{p}{:}

        \PYG{n}{XX} \PYG{o}{=} \PYG{n}{gam}\PYG{o}{.}\PYG{n}{generate\PYGZus{}X\PYGZus{}grid}\PYG{p}{(}\PYG{n}{term}\PYG{o}{=}\PYG{n}{i}\PYG{p}{)}
        \PYG{n}{pdep}\PYG{p}{,} \PYG{n}{confi} \PYG{o}{=} \PYG{n}{gam}\PYG{o}{.}\PYG{n}{partial\PYGZus{}dependence}\PYG{p}{(}\PYG{n}{term}\PYG{o}{=}\PYG{n}{i}\PYG{p}{,} \PYG{n}{X}\PYG{o}{=}\PYG{n}{XX}\PYG{p}{,} \PYG{n}{width}\PYG{o}{=}\PYG{l+m+mf}{0.95}\PYG{p}{)}

        \PYG{n}{plt}\PYG{o}{.}\PYG{n}{figure}\PYG{p}{(}\PYG{p}{)}
        \PYG{n}{plt}\PYG{o}{.}\PYG{n}{plot}\PYG{p}{(}\PYG{n}{XX}\PYG{p}{[}\PYG{p}{:}\PYG{p}{,} \PYG{n}{term}\PYG{o}{.}\PYG{n}{feature}\PYG{p}{]}\PYG{p}{,} \PYG{n}{pdep}\PYG{p}{)}
        \PYG{n}{plt}\PYG{o}{.}\PYG{n}{plot}\PYG{p}{(}\PYG{n}{XX}\PYG{p}{[}\PYG{p}{:}\PYG{p}{,} \PYG{n}{term}\PYG{o}{.}\PYG{n}{feature}\PYG{p}{]}\PYG{p}{,} \PYG{n}{confi}\PYG{p}{,} \PYG{n}{c}\PYG{o}{=}\PYG{l+s+s1}{\PYGZsq{}}\PYG{l+s+s1}{r}\PYG{l+s+s1}{\PYGZsq{}}\PYG{p}{,} \PYG{n}{ls}\PYG{o}{=}\PYG{l+s+s1}{\PYGZsq{}}\PYG{l+s+s1}{\PYGZhy{}\PYGZhy{}}\PYG{l+s+s1}{\PYGZsq{}}\PYG{p}{)}
        \PYG{n}{plt}\PYG{o}{.}\PYG{n}{title}\PYG{p}{(}\PYG{n}{feature\PYGZus{}list}\PYG{p}{[}\PYG{n}{i}\PYG{p}{]}\PYG{p}{)}
        \PYG{n}{plt}\PYG{o}{.}\PYG{n}{show}\PYG{p}{(}\PYG{p}{)}
\end{sphinxVerbatim}

\end{sphinxuseclass}\end{sphinxVerbatimInput}
\begin{sphinxVerbatimOutput}

\begin{sphinxuseclass}{cell_output}
\noindent\sphinxincludegraphics{{5.2.3_GAM_11_0}.png}

\noindent\sphinxincludegraphics{{5.2.3_GAM_11_1}.png}

\noindent\sphinxincludegraphics{{5.2.3_GAM_11_2}.png}

\noindent\sphinxincludegraphics{{5.2.3_GAM_11_3}.png}

\end{sphinxuseclass}\end{sphinxVerbatimOutput}

\end{sphinxuseclass}
\sphinxAtStartPar
 완성된 모델을 pickle 로 binary 파일로 저장합니다.

\begin{sphinxuseclass}{cell}\begin{sphinxVerbatimInput}

\begin{sphinxuseclass}{cell_input}
\begin{sphinxVerbatim}[commandchars=\\\{\}]
\PYG{k+kn}{import} \PYG{n+nn}{pickle}
\PYG{k}{with} \PYG{n+nb}{open}\PYG{p}{(}\PYG{l+s+s2}{\PYGZdq{}}\PYG{l+s+s2}{gam.pkl}\PYG{l+s+s2}{\PYGZdq{}}\PYG{p}{,} \PYG{l+s+s2}{\PYGZdq{}}\PYG{l+s+s2}{wb}\PYG{l+s+s2}{\PYGZdq{}}\PYG{p}{)} \PYG{k}{as} \PYG{n}{file}\PYG{p}{:}
    \PYG{n}{pickle}\PYG{o}{.}\PYG{n}{dump}\PYG{p}{(}\PYG{n}{gam}\PYG{p}{,} \PYG{n}{file}\PYG{p}{)}    
\end{sphinxVerbatim}

\end{sphinxuseclass}\end{sphinxVerbatimInput}

\end{sphinxuseclass}
\begin{sphinxuseclass}{cell}\begin{sphinxVerbatimInput}

\begin{sphinxuseclass}{cell_input}
\begin{sphinxVerbatim}[commandchars=\\\{\}]
\PYG{k}{with} \PYG{n+nb}{open}\PYG{p}{(}\PYG{l+s+s2}{\PYGZdq{}}\PYG{l+s+s2}{gam.pkl}\PYG{l+s+s2}{\PYGZdq{}}\PYG{p}{,} \PYG{l+s+s2}{\PYGZdq{}}\PYG{l+s+s2}{rb}\PYG{l+s+s2}{\PYGZdq{}}\PYG{p}{)} \PYG{k}{as} \PYG{n}{file}\PYG{p}{:}
    \PYG{n}{gam} \PYG{o}{=} \PYG{n}{pickle}\PYG{o}{.}\PYG{n}{load}\PYG{p}{(}\PYG{n}{file}\PYG{p}{)} 
\end{sphinxVerbatim}

\end{sphinxuseclass}\end{sphinxVerbatimInput}

\end{sphinxuseclass}
\begin{sphinxuseclass}{cell}\begin{sphinxVerbatimInput}

\begin{sphinxuseclass}{cell_input}
\begin{sphinxVerbatim}[commandchars=\\\{\}]
\PYG{n+nb}{print}\PYG{p}{(}\PYG{n}{gam}\PYG{o}{.}\PYG{n}{get\PYGZus{}params}\PYG{p}{(}\PYG{p}{)}\PYG{p}{)}
\PYG{n+nb}{print}\PYG{p}{(}\PYG{n}{gam}\PYG{o}{.}\PYG{n}{coef\PYGZus{}}\PYG{o}{.}\PYG{n}{shape}\PYG{p}{)}
\end{sphinxVerbatim}

\end{sphinxuseclass}\end{sphinxVerbatimInput}
\begin{sphinxVerbatimOutput}

\begin{sphinxuseclass}{cell_output}
\begin{sphinxVerbatim}[commandchars=\\\{\}]
\PYGZob{}\PYGZsq{}max\PYGZus{}iter\PYGZsq{}: 100, \PYGZsq{}tol\PYGZsq{}: 0.0001, \PYGZsq{}callbacks\PYGZsq{}: [Deviance(), Diffs(), Accuracy()], \PYGZsq{}verbose\PYGZsq{}: False, \PYGZsq{}terms\PYGZsq{}: te(0, 1) + s(1) + s(2) + s(3) + s(4) + te(4, 5) + intercept, \PYGZsq{}fit\PYGZus{}intercept\PYGZsq{}: True\PYGZcb{}
(131,)
\end{sphinxVerbatim}

\end{sphinxuseclass}\end{sphinxVerbatimOutput}

\end{sphinxuseclass}
\begin{sphinxuseclass}{cell}\begin{sphinxVerbatimInput}

\begin{sphinxuseclass}{cell_input}
\begin{sphinxVerbatim}[commandchars=\\\{\}]
\PYG{k}{for} \PYG{n}{i} \PYG{o+ow}{in} \PYG{n+nb}{range}\PYG{p}{(}\PYG{l+m+mi}{6}\PYG{p}{)}\PYG{p}{:}
    \PYG{n+nb}{print}\PYG{p}{(}\PYG{l+s+sa}{f}\PYG{l+s+s1}{\PYGZsq{}}\PYG{l+s+si}{\PYGZob{}}\PYG{n}{i}\PYG{l+s+si}{\PYGZcb{}}\PYG{l+s+s1}{: }\PYG{l+s+si}{\PYGZob{}}\PYG{n}{gam}\PYG{o}{.}\PYG{n}{\PYGZus{}compute\PYGZus{}p\PYGZus{}value}\PYG{p}{(}\PYG{n}{i}\PYG{p}{)}\PYG{l+s+si}{:}\PYG{l+s+s1}{ 5.3f}\PYG{l+s+si}{\PYGZcb{}}\PYG{l+s+s1}{ }\PYG{l+s+si}{\PYGZob{}}\PYG{n}{gam}\PYG{o}{.}\PYG{n}{generate\PYGZus{}X\PYGZus{}grid}\PYG{p}{(}\PYG{n}{term}\PYG{o}{=}\PYG{n}{i}\PYG{p}{)}\PYG{o}{.}\PYG{n}{shape}\PYG{l+s+si}{\PYGZcb{}}\PYG{l+s+s1}{\PYGZsq{}}\PYG{p}{)}
\end{sphinxVerbatim}

\end{sphinxuseclass}\end{sphinxVerbatimInput}
\begin{sphinxVerbatimOutput}

\begin{sphinxuseclass}{cell_output}
\begin{sphinxVerbatim}[commandchars=\\\{\}]
0:  0.000 (10000, 6)
1:  0.000 (100, 6)
2:  0.165 (100, 6)
3:  0.043 (100, 6)
4:  0.000 (100, 6)
5:  0.180 (10000, 6)
\end{sphinxVerbatim}

\end{sphinxuseclass}\end{sphinxVerbatimOutput}

\end{sphinxuseclass}
\sphinxAtStartPar
 간단하게 십분위수 분석을 하고, 성능을 평가합니다. 안정적인 모델을 만들었습니다. 이론적으로는 마지막 Decile(제 10 십분위 수)에서 랜덤하게 종목을 골라 동일한 금액으로 매수를 한다면, 5 영업일이내 5\% 익절할 확률이 37\% 가 됩니다.  100\% 만족스럽지는 않지만, 생성된 GAM 모델을 이용하여 종목 추천을 받도록 하겠습니다.

\begin{sphinxuseclass}{cell}\begin{sphinxVerbatimInput}

\begin{sphinxuseclass}{cell_input}
\begin{sphinxVerbatim}[commandchars=\\\{\}]
\PYG{n}{feature\PYGZus{}list} \PYG{o}{=} \PYG{p}{[}\PYG{l+s+s1}{\PYGZsq{}}\PYG{l+s+s1}{price\PYGZus{}z}\PYG{l+s+s1}{\PYGZsq{}}\PYG{p}{,}\PYG{l+s+s1}{\PYGZsq{}}\PYG{l+s+s1}{volume\PYGZus{}z}\PYG{l+s+s1}{\PYGZsq{}}\PYG{p}{,}\PYG{l+s+s1}{\PYGZsq{}}\PYG{l+s+s1}{num\PYGZus{}high/close}\PYG{l+s+s1}{\PYGZsq{}}\PYG{p}{,}\PYG{l+s+s1}{\PYGZsq{}}\PYG{l+s+s1}{num\PYGZus{}win\PYGZus{}market}\PYG{l+s+s1}{\PYGZsq{}}\PYG{p}{,}\PYG{l+s+s1}{\PYGZsq{}}\PYG{l+s+s1}{pct\PYGZus{}win\PYGZus{}market}\PYG{l+s+s1}{\PYGZsq{}}\PYG{p}{,}\PYG{l+s+s1}{\PYGZsq{}}\PYG{l+s+s1}{return over sector}\PYG{l+s+s1}{\PYGZsq{}}\PYG{p}{]}
\PYG{n}{X} \PYG{o}{=} \PYG{n}{train}\PYG{p}{[}\PYG{n}{feature\PYGZus{}list}\PYG{p}{]}
\PYG{n}{y} \PYG{o}{=} \PYG{n}{train}\PYG{p}{[}\PYG{l+s+s1}{\PYGZsq{}}\PYG{l+s+s1}{target}\PYG{l+s+s1}{\PYGZsq{}}\PYG{p}{]}
\PYG{n}{X\PYGZus{}test} \PYG{o}{=} \PYG{n}{test}\PYG{p}{[}\PYG{n}{feature\PYGZus{}list}\PYG{p}{]}
\PYG{n}{y\PYGZus{}test} \PYG{o}{=} \PYG{n}{test}\PYG{p}{[}\PYG{l+s+s1}{\PYGZsq{}}\PYG{l+s+s1}{target}\PYG{l+s+s1}{\PYGZsq{}}\PYG{p}{]}

\PYG{n}{yhat} \PYG{o}{=} \PYG{n}{gam}\PYG{o}{.}\PYG{n}{predict\PYGZus{}proba}\PYG{p}{(}\PYG{n}{X}\PYG{o}{.}\PYG{n}{to\PYGZus{}numpy}\PYG{p}{(}\PYG{p}{)}\PYG{p}{)}
\PYG{n}{yhat} \PYG{o}{=} \PYG{n}{pd}\PYG{o}{.}\PYG{n}{Series}\PYG{p}{(}\PYG{n}{yhat}\PYG{p}{,} \PYG{n}{name}\PYG{o}{=}\PYG{l+s+s1}{\PYGZsq{}}\PYG{l+s+s1}{yhat}\PYG{l+s+s1}{\PYGZsq{}}\PYG{p}{,} \PYG{n}{index}\PYG{o}{=}\PYG{n}{y}\PYG{o}{.}\PYG{n}{index}\PYG{p}{)}

\PYG{n}{yhat\PYGZus{}test} \PYG{o}{=} \PYG{n}{gam}\PYG{o}{.}\PYG{n}{predict\PYGZus{}proba}\PYG{p}{(}\PYG{n}{X\PYGZus{}test}\PYG{o}{.}\PYG{n}{to\PYGZus{}numpy}\PYG{p}{(}\PYG{p}{)}\PYG{p}{)}
\PYG{n}{yhat\PYGZus{}test} \PYG{o}{=} \PYG{n}{pd}\PYG{o}{.}\PYG{n}{Series}\PYG{p}{(}\PYG{n}{yhat\PYGZus{}test}\PYG{p}{,} \PYG{n}{name}\PYG{o}{=}\PYG{l+s+s1}{\PYGZsq{}}\PYG{l+s+s1}{yhat}\PYG{l+s+s1}{\PYGZsq{}}\PYG{p}{,} \PYG{n}{index}\PYG{o}{=}\PYG{n}{y\PYGZus{}test}\PYG{o}{.}\PYG{n}{index}\PYG{p}{)}
\end{sphinxVerbatim}

\end{sphinxuseclass}\end{sphinxVerbatimInput}

\end{sphinxuseclass}
\begin{sphinxuseclass}{cell}\begin{sphinxVerbatimInput}

\begin{sphinxuseclass}{cell_input}
\begin{sphinxVerbatim}[commandchars=\\\{\}]
\PYG{k}{def} \PYG{n+nf}{perf}\PYG{p}{(}\PYG{n}{y}\PYG{p}{,} \PYG{n}{yhat}\PYG{p}{)}\PYG{p}{:} \PYG{c+c1}{\PYGZsh{} Decile 분석 함수}
    \PYG{n}{combined} \PYG{o}{=} \PYG{n}{pd}\PYG{o}{.}\PYG{n}{concat}\PYG{p}{(}\PYG{p}{[}\PYG{n}{y}\PYG{p}{,} \PYG{n}{yhat}\PYG{p}{]}\PYG{p}{,} \PYG{n}{axis}\PYG{o}{=}\PYG{l+m+mi}{1}\PYG{p}{)}
    \PYG{n}{ranks} \PYG{o}{=} \PYG{n}{pd}\PYG{o}{.}\PYG{n}{qcut}\PYG{p}{(}\PYG{n}{combined}\PYG{p}{[}\PYG{l+s+s1}{\PYGZsq{}}\PYG{l+s+s1}{yhat}\PYG{l+s+s1}{\PYGZsq{}}\PYG{p}{]}\PYG{p}{,} \PYG{n}{q}\PYG{o}{=}\PYG{l+m+mi}{10}\PYG{p}{)}
    \PYG{n+nb}{print}\PYG{p}{(}\PYG{n}{combined}\PYG{o}{.}\PYG{n}{groupby}\PYG{p}{(}\PYG{n}{ranks}\PYG{p}{)}\PYG{p}{[}\PYG{l+s+s1}{\PYGZsq{}}\PYG{l+s+s1}{target}\PYG{l+s+s1}{\PYGZsq{}}\PYG{p}{]}\PYG{o}{.}\PYG{n}{agg}\PYG{p}{(}\PYG{p}{[}\PYG{l+s+s1}{\PYGZsq{}}\PYG{l+s+s1}{count}\PYG{l+s+s1}{\PYGZsq{}}\PYG{p}{,}\PYG{l+s+s1}{\PYGZsq{}}\PYG{l+s+s1}{mean}\PYG{l+s+s1}{\PYGZsq{}}\PYG{p}{]}\PYG{p}{)}\PYG{p}{)}
    \PYG{n}{combined}\PYG{o}{.}\PYG{n}{groupby}\PYG{p}{(}\PYG{n}{ranks}\PYG{p}{)}\PYG{p}{[}\PYG{l+s+s1}{\PYGZsq{}}\PYG{l+s+s1}{target}\PYG{l+s+s1}{\PYGZsq{}}\PYG{p}{]}\PYG{o}{.}\PYG{n}{mean}\PYG{p}{(}\PYG{p}{)}\PYG{o}{.}\PYG{n}{plot}\PYG{p}{(}\PYG{n}{figsize}\PYG{o}{=}\PYG{p}{(}\PYG{l+m+mi}{8}\PYG{p}{,}\PYG{l+m+mi}{5}\PYG{p}{)}\PYG{p}{)}

\PYG{n}{perf}\PYG{p}{(}\PYG{n}{y}\PYG{p}{,} \PYG{n}{yhat}\PYG{p}{)}
\PYG{n}{perf}\PYG{p}{(}\PYG{n}{y\PYGZus{}test}\PYG{p}{,} \PYG{n}{yhat\PYGZus{}test}\PYG{p}{)}
\end{sphinxVerbatim}

\end{sphinxuseclass}\end{sphinxVerbatimInput}
\begin{sphinxVerbatimOutput}

\begin{sphinxuseclass}{cell_output}
\begin{sphinxVerbatim}[commandchars=\\\{\}]
                count  mean
yhat                       
(0.121, 0.184]   1000 0.144
(0.184, 0.198]   1000 0.183
(0.198, 0.21]    1000 0.195
(0.21, 0.222]    1000 0.193
(0.222, 0.235]   1000 0.229
(0.235, 0.251]   1000 0.231
(0.251, 0.268]   1000 0.294
(0.268, 0.291]   1000 0.281
(0.291, 0.326]   1000 0.310
(0.326, 0.688]   1000 0.382
                 count  mean
yhat                        
(0.0342, 0.184]  31931 0.156
(0.184, 0.198]   31931 0.182
(0.198, 0.209]   31930 0.192
(0.209, 0.221]   31931 0.203
(0.221, 0.234]   31931 0.224
(0.234, 0.249]   31930 0.238
(0.249, 0.267]   31931 0.259
(0.267, 0.292]   31930 0.283
(0.292, 0.327]   31931 0.320
(0.327, 0.783]   31931 0.373
\end{sphinxVerbatim}

\noindent\sphinxincludegraphics{{5.2.3_GAM_19_1}.png}

\end{sphinxuseclass}\end{sphinxVerbatimOutput}

\end{sphinxuseclass}

\part{Basic Filtering}
\label{\detokenize{chapter5/5.2.3_GAM:basic-filtering}}
\sphinxAtStartPar
단순히 스코어가 높다고 무조건 매수했다가 큰 낙폭으로 손해를 볼 수도 있기 때문에 기본적인 필터링이 필요합니다. 오늘 종가 수익률과 가격 변동성으로 기본적인 필터를 만들어 보겠습니다.

\begin{sphinxuseclass}{cell}\begin{sphinxVerbatimInput}

\begin{sphinxuseclass}{cell_input}
\begin{sphinxVerbatim}[commandchars=\\\{\}]
\PYG{n}{test}\PYG{p}{[}\PYG{l+s+s1}{\PYGZsq{}}\PYG{l+s+s1}{yhat}\PYG{l+s+s1}{\PYGZsq{}}\PYG{p}{]} \PYG{o}{=} \PYG{n}{yhat\PYGZus{}test}
\PYG{n}{test}\PYG{p}{[}\PYG{l+s+s1}{\PYGZsq{}}\PYG{l+s+s1}{yhat\PYGZus{}rank}\PYG{l+s+s1}{\PYGZsq{}}\PYG{p}{]} \PYG{o}{=} \PYG{n}{pd}\PYG{o}{.}\PYG{n}{qcut}\PYG{p}{(}\PYG{n}{test}\PYG{p}{[}\PYG{l+s+s1}{\PYGZsq{}}\PYG{l+s+s1}{yhat}\PYG{l+s+s1}{\PYGZsq{}}\PYG{p}{]}\PYG{p}{,} \PYG{n}{q}\PYG{o}{=}\PYG{l+m+mi}{10}\PYG{p}{)}
\PYG{n}{test}\PYG{o}{.}\PYG{n}{groupby}\PYG{p}{(}\PYG{l+s+s1}{\PYGZsq{}}\PYG{l+s+s1}{yhat\PYGZus{}rank}\PYG{l+s+s1}{\PYGZsq{}}\PYG{p}{)}\PYG{p}{[}\PYG{l+s+s1}{\PYGZsq{}}\PYG{l+s+s1}{target}\PYG{l+s+s1}{\PYGZsq{}}\PYG{p}{]}\PYG{o}{.}\PYG{n}{mean}\PYG{p}{(}\PYG{p}{)}
\end{sphinxVerbatim}

\end{sphinxuseclass}\end{sphinxVerbatimInput}
\begin{sphinxVerbatimOutput}

\begin{sphinxuseclass}{cell_output}
\begin{sphinxVerbatim}[commandchars=\\\{\}]
yhat\PYGZus{}rank
(0.0342, 0.184]   0.156
(0.184, 0.198]    0.182
(0.198, 0.209]    0.192
(0.209, 0.221]    0.203
(0.221, 0.234]    0.224
(0.234, 0.249]    0.238
(0.249, 0.267]    0.259
(0.267, 0.292]    0.283
(0.292, 0.327]    0.320
(0.327, 0.783]    0.373
Name: target, dtype: float64
\end{sphinxVerbatim}

\end{sphinxuseclass}\end{sphinxVerbatimOutput}

\end{sphinxuseclass}
\sphinxAtStartPar
 종목선정은 상위 스코어 구간에서 할 것이므로 상위 구간에서 대하여 수익률 및 표준화 가격 구간으로 분리해서 미래 수익률을 보겠습니다. 표준화된 가격이 낮고 당일 수익율이 높은 경우 미래 수익률이 높을 것으로 예상됩니다.

\begin{sphinxuseclass}{cell}\begin{sphinxVerbatimInput}

\begin{sphinxuseclass}{cell_input}
\begin{sphinxVerbatim}[commandchars=\\\{\}]
\PYG{n}{tops} \PYG{o}{=} \PYG{n}{test}\PYG{p}{[}\PYG{n}{test}\PYG{p}{[}\PYG{l+s+s1}{\PYGZsq{}}\PYG{l+s+s1}{yhat}\PYG{l+s+s1}{\PYGZsq{}}\PYG{p}{]} \PYG{o}{\PYGZgt{}} \PYG{l+m+mf}{0.3}\PYG{p}{]}\PYG{o}{.}\PYG{n}{copy}\PYG{p}{(}\PYG{p}{)}

\PYG{n}{tops}\PYG{p}{[}\PYG{l+s+s1}{\PYGZsq{}}\PYG{l+s+s1}{return\PYGZus{}rank}\PYG{l+s+s1}{\PYGZsq{}}\PYG{p}{]}  \PYG{o}{=} \PYG{n}{pd}\PYG{o}{.}\PYG{n}{qcut}\PYG{p}{(}\PYG{n}{tops}\PYG{p}{[}\PYG{l+s+s1}{\PYGZsq{}}\PYG{l+s+s1}{return}\PYG{l+s+s1}{\PYGZsq{}}\PYG{p}{]}\PYG{p}{,} \PYG{n}{q}\PYG{o}{=}\PYG{l+m+mi}{5}\PYG{p}{)} \PYG{c+c1}{\PYGZsh{} 종가 수익률}
\PYG{n}{tops}\PYG{p}{[}\PYG{l+s+s1}{\PYGZsq{}}\PYG{l+s+s1}{price\PYGZus{}rank}\PYG{l+s+s1}{\PYGZsq{}}\PYG{p}{]}  \PYG{o}{=} \PYG{n}{pd}\PYG{o}{.}\PYG{n}{qcut}\PYG{p}{(}\PYG{n}{tops}\PYG{p}{[}\PYG{l+s+s1}{\PYGZsq{}}\PYG{l+s+s1}{price\PYGZus{}z}\PYG{l+s+s1}{\PYGZsq{}}\PYG{p}{]}\PYG{p}{,} \PYG{n}{q}\PYG{o}{=}\PYG{l+m+mi}{5}\PYG{p}{)} \PYG{c+c1}{\PYGZsh{} 가격 변동성}
\PYG{n}{tops}\PYG{o}{.}\PYG{n}{groupby}\PYG{p}{(}\PYG{p}{[}\PYG{l+s+s1}{\PYGZsq{}}\PYG{l+s+s1}{return\PYGZus{}rank}\PYG{l+s+s1}{\PYGZsq{}}\PYG{p}{,}\PYG{l+s+s1}{\PYGZsq{}}\PYG{l+s+s1}{price\PYGZus{}rank}\PYG{l+s+s1}{\PYGZsq{}}\PYG{p}{]}\PYG{p}{)}\PYG{p}{[}\PYG{l+s+s1}{\PYGZsq{}}\PYG{l+s+s1}{target}\PYG{l+s+s1}{\PYGZsq{}}\PYG{p}{]}\PYG{o}{.}\PYG{n}{mean}\PYG{p}{(}\PYG{p}{)}\PYG{o}{.}\PYG{n}{unstack}\PYG{p}{(}\PYG{p}{)}\PYG{o}{.}\PYG{n}{style}\PYG{o}{.}\PYG{n}{set\PYGZus{}table\PYGZus{}attributes}\PYG{p}{(}\PYG{l+s+s1}{\PYGZsq{}}\PYG{l+s+s1}{style=}\PYG{l+s+s1}{\PYGZdq{}}\PYG{l+s+s1}{font\PYGZhy{}size: 12px}\PYG{l+s+s1}{\PYGZdq{}}\PYG{l+s+s1}{\PYGZsq{}}\PYG{p}{)}
\end{sphinxVerbatim}

\end{sphinxuseclass}\end{sphinxVerbatimInput}
\begin{sphinxVerbatimOutput}

\begin{sphinxuseclass}{cell_output}
\begin{sphinxVerbatim}[commandchars=\\\{\}]
\PYGZlt{}pandas.io.formats.style.Styler at 0x1d8d6e270a0\PYGZgt{}
\end{sphinxVerbatim}

\end{sphinxuseclass}\end{sphinxVerbatimOutput}

\end{sphinxuseclass}
\sphinxAtStartPar
 참고로 groupby 로 데이터를 요약하는 방법은 직관적이나, 각 행과 열의 총계는 보여주지 않는다는 단점이 있습니다. 총계가 보고 싶을 때는 pivot\_table 에서 ‘margins=True’ 를 인수로 넣어주면 총계를 볼 수 있습니다.

\begin{sphinxuseclass}{cell}\begin{sphinxVerbatimInput}

\begin{sphinxuseclass}{cell_input}
\begin{sphinxVerbatim}[commandchars=\\\{\}]
\PYG{n}{pd}\PYG{o}{.}\PYG{n}{pivot\PYGZus{}table}\PYG{p}{(}\PYG{n}{data} \PYG{o}{=} \PYG{n}{tops}\PYG{p}{,} \PYG{n}{index} \PYG{o}{=} \PYG{l+s+s1}{\PYGZsq{}}\PYG{l+s+s1}{return\PYGZus{}rank}\PYG{l+s+s1}{\PYGZsq{}}\PYG{p}{,} \PYG{n}{columns} \PYG{o}{=} \PYG{l+s+s1}{\PYGZsq{}}\PYG{l+s+s1}{price\PYGZus{}rank}\PYG{l+s+s1}{\PYGZsq{}}\PYG{p}{,} \PYG{n}{values} \PYG{o}{=} \PYG{l+s+s1}{\PYGZsq{}}\PYG{l+s+s1}{target}\PYG{l+s+s1}{\PYGZsq{}}\PYG{p}{,} \PYG{n}{aggfunc}\PYG{o}{=}\PYG{l+s+s1}{\PYGZsq{}}\PYG{l+s+s1}{mean}\PYG{l+s+s1}{\PYGZsq{}}\PYG{p}{,} \PYG{n}{margins}\PYG{o}{=}\PYG{k+kc}{True}\PYG{p}{)}\PYG{o}{.}\PYG{n}{style}\PYG{o}{.}\PYG{n}{set\PYGZus{}table\PYGZus{}attributes}\PYG{p}{(}\PYG{l+s+s1}{\PYGZsq{}}\PYG{l+s+s1}{style=}\PYG{l+s+s1}{\PYGZdq{}}\PYG{l+s+s1}{font\PYGZhy{}size: 12px}\PYG{l+s+s1}{\PYGZdq{}}\PYG{l+s+s1}{\PYGZsq{}}\PYG{p}{)}
\end{sphinxVerbatim}

\end{sphinxuseclass}\end{sphinxVerbatimInput}
\begin{sphinxVerbatimOutput}

\begin{sphinxuseclass}{cell_output}
\begin{sphinxVerbatim}[commandchars=\\\{\}]
\PYGZlt{}pandas.io.formats.style.Styler at 0x1d8d6e44af0\PYGZgt{}
\end{sphinxVerbatim}

\end{sphinxuseclass}\end{sphinxVerbatimOutput}

\end{sphinxuseclass}
\sphinxAtStartPar
 최저 수익률(리스크)도 조사합니다. 당일 수익률 높고, 표준화 된 주가가 낮은 좌하단 부분의 리스크가 낮습니다.

\begin{sphinxuseclass}{cell}\begin{sphinxVerbatimInput}

\begin{sphinxuseclass}{cell_input}
\begin{sphinxVerbatim}[commandchars=\\\{\}]
\PYG{n}{pd}\PYG{o}{.}\PYG{n}{pivot\PYGZus{}table}\PYG{p}{(}\PYG{n}{data} \PYG{o}{=} \PYG{n}{tops}\PYG{p}{,} \PYG{n}{index} \PYG{o}{=} \PYG{l+s+s1}{\PYGZsq{}}\PYG{l+s+s1}{return\PYGZus{}rank}\PYG{l+s+s1}{\PYGZsq{}}\PYG{p}{,} \PYG{n}{columns} \PYG{o}{=} \PYG{l+s+s1}{\PYGZsq{}}\PYG{l+s+s1}{price\PYGZus{}rank}\PYG{l+s+s1}{\PYGZsq{}}\PYG{p}{,} \PYG{n}{values} \PYG{o}{=} \PYG{l+s+s1}{\PYGZsq{}}\PYG{l+s+s1}{min\PYGZus{}close}\PYG{l+s+s1}{\PYGZsq{}}\PYG{p}{,} \PYG{n}{aggfunc}\PYG{o}{=}\PYG{l+s+s1}{\PYGZsq{}}\PYG{l+s+s1}{mean}\PYG{l+s+s1}{\PYGZsq{}}\PYG{p}{,} \PYG{n}{margins}\PYG{o}{=}\PYG{k+kc}{True}\PYG{p}{)}\PYG{o}{.}\PYG{n}{style}\PYG{o}{.}\PYG{n}{set\PYGZus{}table\PYGZus{}attributes}\PYG{p}{(}\PYG{l+s+s1}{\PYGZsq{}}\PYG{l+s+s1}{style=}\PYG{l+s+s1}{\PYGZdq{}}\PYG{l+s+s1}{font\PYGZhy{}size: 12px}\PYG{l+s+s1}{\PYGZdq{}}\PYG{l+s+s1}{\PYGZsq{}}\PYG{p}{)}
\end{sphinxVerbatim}

\end{sphinxuseclass}\end{sphinxVerbatimInput}
\begin{sphinxVerbatimOutput}

\begin{sphinxuseclass}{cell_output}
\begin{sphinxVerbatim}[commandchars=\\\{\}]
\PYGZlt{}pandas.io.formats.style.Styler at 0x1d8d432a310\PYGZgt{}
\end{sphinxVerbatim}

\end{sphinxuseclass}\end{sphinxVerbatimOutput}

\end{sphinxuseclass}
\sphinxAtStartPar
 위 결과를 종합하면 당일 종가 수익률은 높고, 최근 20일 대비 가격이 낮은 종목을 매수하면 리스크가 적을 것으로 판단됩니다. ‘return’ 은 1.03 보다 크고, ‘price\_z’ 는 0 보다 작은 종목만을 고르겠습니다.

\begin{sphinxuseclass}{cell}\begin{sphinxVerbatimInput}

\begin{sphinxuseclass}{cell_input}
\begin{sphinxVerbatim}[commandchars=\\\{\}]
\PYG{n}{tops}\PYG{p}{[} \PYG{p}{(}\PYG{n}{tops}\PYG{p}{[}\PYG{l+s+s1}{\PYGZsq{}}\PYG{l+s+s1}{return}\PYG{l+s+s1}{\PYGZsq{}}\PYG{p}{]} \PYG{o}{\PYGZgt{}} \PYG{l+m+mf}{1.03}\PYG{p}{)} \PYG{o}{\PYGZam{}} \PYG{p}{(}\PYG{n}{tops}\PYG{p}{[}\PYG{l+s+s1}{\PYGZsq{}}\PYG{l+s+s1}{price\PYGZus{}z}\PYG{l+s+s1}{\PYGZsq{}}\PYG{p}{]} \PYG{o}{\PYGZlt{}} \PYG{l+m+mi}{0}\PYG{p}{)}\PYG{p}{]}\PYG{p}{[}\PYG{p}{[}\PYG{l+s+s1}{\PYGZsq{}}\PYG{l+s+s1}{return}\PYG{l+s+s1}{\PYGZsq{}}\PYG{p}{,}\PYG{l+s+s1}{\PYGZsq{}}\PYG{l+s+s1}{price\PYGZus{}z}\PYG{l+s+s1}{\PYGZsq{}}\PYG{p}{]}\PYG{p}{]}\PYG{o}{.}\PYG{n}{head}\PYG{p}{(}\PYG{p}{)}\PYG{o}{.}\PYG{n}{style}\PYG{o}{.}\PYG{n}{set\PYGZus{}table\PYGZus{}attributes}\PYG{p}{(}\PYG{l+s+s1}{\PYGZsq{}}\PYG{l+s+s1}{style=}\PYG{l+s+s1}{\PYGZdq{}}\PYG{l+s+s1}{font\PYGZhy{}size: 12px}\PYG{l+s+s1}{\PYGZdq{}}\PYG{l+s+s1}{\PYGZsq{}}\PYG{p}{)}
\end{sphinxVerbatim}

\end{sphinxuseclass}\end{sphinxVerbatimInput}
\begin{sphinxVerbatimOutput}

\begin{sphinxuseclass}{cell_output}
\begin{sphinxVerbatim}[commandchars=\\\{\}]
\PYGZlt{}pandas.io.formats.style.Styler at 0x1d8d7617a00\PYGZgt{}
\end{sphinxVerbatim}

\end{sphinxuseclass}\end{sphinxVerbatimOutput}

\end{sphinxuseclass}

\part{선형모델 가정에 대한 이해}
\label{\detokenize{chapter5/5.2.4_LM_Assumptions:id1}}\label{\detokenize{chapter5/5.2.4_LM_Assumptions::doc}}
\sphinxAtStartPar
왜 매수결정을 했는지에 대한 이유를 구체적으로 설명하기 유리한 모델은 Linear Model 입니다. 그 중 다변량 회귀모델 (Multivariate Linear Regression) 은 데이터분석을 배울 때, 가장 기초적으로 다루는 예측모델입니다. 예측하고자 하는 종속변수 Y (레이블 혹은 타겟 변수) 가 연속형이고, 이것을 설명 혹은 예측하는 독립변수 X (입력피쳐 혹은 입력변수) 들의 선형조합 Z 로 Fitting 을 하는 것인데, 충분한 이해없이 사용하면, 잘못된 결론을 내기 쉽습니다. 다변량 회귀분석 모델이 의미가 있을려면, 데이터가 상당히 강한 Assumptions 를 만족해야 합니다. 중요한 4 가지는 다음과 같습니다.
\begin{enumerate}
\sphinxsetlistlabels{\arabic}{enumi}{enumii}{}{.}%
\item {} 
\sphinxAtStartPar
Normality \sphinxhyphen{} 에러(실제값 \sphinxhyphen{} 예측값)가 정규분포를 따라야한다. 사실 이건 Y 가 정규분포를 따라야 한다는 것과 크게 다르지 않습니다.

\item {} 
\sphinxAtStartPar
Weak Heteroscedasticity \sphinxhyphen{} 에러가 등분산성을 만족해야 한다. 즉 에러의 분산이 예측 값의 크기에 따라서 크게 변화하지 않아야 한다.

\item {} 
\sphinxAtStartPar
Linearity \sphinxhyphen{} 선형성. 이것은 추정된 베타값이 X 값의 크기에 따라서 변화하지 않아야 한다. 예를 들어, 소득을 추정하는데, 카드 사용량이 변수라면 카드 월 사용량이 백만원일 때 추정된 계수(coefficient) 가 50 이라면, 카드 사용량이 천 만원일때도 베타 계수가 50이여야 한다는 말입니다.

\item {} 
\sphinxAtStartPar
Weak Multicollinearity \sphinxhyphen{} (다중 공선성)이 크지 않아야 한다. 쉽게 이야기 하면 어떤 여러개의 X 가 Y 를 설명하는데 있어서 X 들이 같은 방향으로 움직이면 안 된다고 이해하면 될 것 같습니다. 다중 공선성이 큰 경우는 계수 값이 정확하지 않아서, 계수에 대한 해석이 불가능하게 됩니다. 아주 심한 경우는 다른 변수의 영향으로 양의 계수가 음의 계수로 바뀌게 됩니다.

\end{enumerate}

\sphinxAtStartPar
위 가정 1 번과 2 번을 만족하지 않아도 Regression 을 할 수 있게 일반화 한 것이 일반화 회귀모형(Generalized Linear Model) 입니다. GLM 에서는 Y 가 갯수(count), 비율(proportion), 이진(0과 1) 등 같이 연속형 변수가 아니고 정규 분포를 따르지 않아도 선형모델을 구현할 수 있습니다. 물론 등분산성을 만족하지 않아도 됩니다. 대신에 Y 에 대한 명확한 분포 설정과 Y 에 대한 Link Function 필요합니다. 가장 많이 쓰이는 것이 Log Link 입니다. 이 부분을 쉽게 이해하기 위해서는 이렇게 생각하면 됩니다. X 의 선형조합 Z 는 음수의 값도 갖게 되는데, 비율이나, 갯수는 항상 양수입니다. 따라서 Y 에 Log 를 씌워서 음수를 갖게 할 수 있습니다. 반대로 EXP( a0 + a1\sphinxstyleemphasis{x1 + a2}x2 …) 로 항상 양수인 Y 를 Fitting 한다고 보시면 될 것 같습니다. 많이 다루는 로지스틱 회귀 모델은 Log(odds) 를 X 의 선형조합으로 Fitting 을 하는 일반화 선형모형의 한 예로 볼 수 있습니다. 데이터상으로는 Y 가 이항분포(Bernoulli 분포 혹은 0 과 1) 이므로 Link Function 가 Logit Link 즉, log(p/1\sphinxhyphen{}p) 로 하는 일반화 선형모형과 동일한 의미가 됩니다.  Y 가 0 과 1 이므로 이것을 가장 잘 근사하게 따라갈 수 있는 변형은 Logit Link 인 것입니다. logit Link 를 풀면 Y = exp(z) / 1 + exp(z) 가 됩니다. 즉 Y 를 설명하기에 좋은 형태로 변경이 되는 것입니다. Y 가  개 수(count) 인 경우는 포아송 회귀분석 (Poisson regression) 입니다. 주어진 시간 혹은 범위에서 뽑은 count 샘플은 포아송 분포를 따라간다는 것이 알려져 있습니다. 예를 들면 인구 만명당 암 발생 환자 수 등이 예가 될 것 같습니다. 포아송 분포의 평균과 분산은 같습니다. 즉 평균이 증가하면 분산이 증가하는 분포입니다. 따라서 등분산성을 만족하지 않아도 Y 를 fitting 할 수 있습니다. 이 경우, Link 는 log 입니다. 즉, X 의 선형조합인 Z 에 Exponential 를 씌워서 양수가 되도록 합니다. Y 가 비율(Proportion) 인 경우도 있습니다. 그럼 비율은 어떤 분포일지 궁금합니다. 비율은 항상 0 과 1 사이 양수이므로 Link 함수는 log link 를 쓰면 될 것 같습니다. 일반적으로 비율은 분자의 특성에 따라 분포가 바뀔 수 있습니다. 위에 예시한 인구 만명 당 암환자의 비율은 GLM 으로 Fitting (Y \textasciitilde{} Normal 분포, Log link) 할 수 있습니다. 하지만 더 Fitting 을 잘 하려면 분자를 Y 로 하고 분모인 인구 수를 exposure 요인으로 처리하는 것입니다. 이 경우 당연히 Y 는 포아송분포가 됩니다.  log(암 환수/인구 수) = Z(X 선형조합)  형태의 모델을 (암 환자수) = exp(Z)*(인구수) 이렇게 변경하는 것과 동일합니다. 그럼 여기서 인구수가 exposure 가 되고, 인구수를 고려하여 Z 에 계수값을 추정하게 됩니다. Proportion 을 Y 로  fitting 하는 것보다 훨씬 좋은 결과가 나옵니다.

\sphinxAtStartPar
마지막으로 3 번째 가정이 선형성을 만족하지 않아도 쓸 수 있는 Linear Model 이 있습니다.  Generalized Additive Model (GAM) 인데, 이경우는 spline 함수를 이용하여 각 X 를 곡선으로 만들어 Y 와 fitting 합니다. 예를 들어 Y 가 그랜저를 살 확률이고, X 가 소득이라고 할 때, 소득이 증가함에 따라 그랜저를 살 확률은 증가하다가 어느 순간 다시 감소할 것 입니다. 그럼 2 차원 곡선이 되는데요. 이런 경우도 소득을 spline 함수(곡선형태)로 만들면 Y 를 잘 Fitting 할 수 있습니다.

\sphinxAtStartPar
마지막 4 번째 가정은 선형모형의 구조상 피할 수 가 없습니다. 공선성을 일으키는 입력 변수를 빼거나, 주성분등으로 공선성을 완전히 제거해야 합니다. 기본적으로 Linear 모델이라는 것은 X 의 합으로 연결이 되어 있습니다. 따라서 Fitting 된 모델에서 X1 이 1 증가할 때,  X2 도 1 증가하는 구조라면, X1 와 X2 의 계수의 추정은 해석하기 어렵게 됩니다. 하지만, 이런 구조이기 때문에 잘 fitting 된 선형모델에서는 X 변화에 따른 Y 의 변화를 이해할 수 있는 장점으로 작용합니다. 요즘 관심을 받고 있는 해석가능한 모델이 되는 것입니다.







\renewcommand{\indexname}{Index}
\printindex
\end{document}