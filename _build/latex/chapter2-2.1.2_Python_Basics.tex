%% Generated by Sphinx.
\def\sphinxdocclass{jupyterBook}
\documentclass[letterpaper,10pt,english]{jupyterBook}
\ifdefined\pdfpxdimen
   \let\sphinxpxdimen\pdfpxdimen\else\newdimen\sphinxpxdimen
\fi \sphinxpxdimen=.75bp\relax
\ifdefined\pdfimageresolution
    \pdfimageresolution= \numexpr \dimexpr1in\relax/\sphinxpxdimen\relax
\fi
%% let collapsible pdf bookmarks panel have high depth per default
\PassOptionsToPackage{bookmarksdepth=5}{hyperref}
%% turn off hyperref patch of \index as sphinx.xdy xindy module takes care of
%% suitable \hyperpage mark-up, working around hyperref-xindy incompatibility
\PassOptionsToPackage{hyperindex=false}{hyperref}
%% memoir class requires extra handling
\makeatletter\@ifclassloaded{memoir}
{\ifdefined\memhyperindexfalse\memhyperindexfalse\fi}{}\makeatother

\PassOptionsToPackage{warn}{textcomp}

\catcode`^^^^00a0\active\protected\def^^^^00a0{\leavevmode\nobreak\ }
\usepackage{cmap}
\usepackage{fontspec}
\defaultfontfeatures[\rmfamily,\sffamily,\ttfamily]{}
\usepackage{amsmath,amssymb,amstext}
\usepackage{polyglossia}
\setmainlanguage{english}



\setmainfont{FreeSerif}[
  Extension      = .otf,
  UprightFont    = *,
  ItalicFont     = *Italic,
  BoldFont       = *Bold,
  BoldItalicFont = *BoldItalic
]
\setsansfont{FreeSans}[
  Extension      = .otf,
  UprightFont    = *,
  ItalicFont     = *Oblique,
  BoldFont       = *Bold,
  BoldItalicFont = *BoldOblique,
]
\setmonofont{FreeMono}[
  Extension      = .otf,
  UprightFont    = *,
  ItalicFont     = *Oblique,
  BoldFont       = *Bold,
  BoldItalicFont = *BoldOblique,
]



\usepackage[Bjarne]{fncychap}
\usepackage[,numfigreset=1,mathnumfig]{sphinx}

\fvset{fontsize=\small}
\usepackage{geometry}


% Include hyperref last.
\usepackage{hyperref}
% Fix anchor placement for figures with captions.
\usepackage{hypcap}% it must be loaded after hyperref.
% Set up styles of URL: it should be placed after hyperref.
\urlstyle{same}


\usepackage{sphinxmessages}



        % Start of preamble defined in sphinx-jupyterbook-latex %
         \usepackage[Latin,Greek]{ucharclasses}
        \usepackage{unicode-math}
        % fixing title of the toc
        \addto\captionsenglish{\renewcommand{\contentsname}{Contents}}
        \hypersetup{
            pdfencoding=auto,
            psdextra
        }
        % End of preamble defined in sphinx-jupyterbook-latex %
        

\title{Index}
\date{Jul 02, 2022}
\release{}
\author{KHS}
\newcommand{\sphinxlogo}{\vbox{}}
\renewcommand{\releasename}{}
\makeindex
\begin{document}

\pagestyle{empty}
\sphinxmaketitle
\pagestyle{plain}
\sphinxtableofcontents
\pagestyle{normal}
\phantomsection\label{\detokenize{chapter2/2.1.2_Python_Basics::doc}}


\sphinxAtStartPar
데이터 처리에 중요한 역활을 하는 Index 에 대하여 알아보겠습니다. Index 는 우리말로 색인이라고 할 수 있을 것 같은데요. 색인은 무엇을 빨리 찾기 위해 순서대로 정리되어 있는 목록입니다. Index 는 색인처럼 어떤 값을 빨리 찾을 때도 필요하지만, 두 데이터를 어떤 값을 기준으로 결합하는데도 유용하게 쓰입니다. Index 는 Series 와 DataFrame 에 주로 활용됩니다. ss2 는 바로 이전 장에서 만든 Series 입니다. 출력을 해 보면 맨 왼쪽에 0 \textasciitilde{} 4 까지 값이 보이는데요. 이게 Index 입니다. 특별하게 지정하지 않으면 숫자 0 부터서 순서대로 들어가게 됩니다. 다음은 알파벳 Index 를 넣어서 ss3 를 생성하고 출력 해보겠습니다. 맨 왼쪽 index 값이 숫자가 아니라 알파벳으로 바뀌었습니다.

\begin{sphinxuseclass}{cell}\begin{sphinxVerbatimInput}

\begin{sphinxuseclass}{cell_input}
\begin{sphinxVerbatim}[commandchars=\\\{\}]
\PYG{k+kn}{import} \PYG{n+nn}{pandas} \PYG{k}{as} \PYG{n+nn}{pd}

\PYG{n}{ss1} \PYG{o}{=} \PYG{p}{[}\PYG{l+m+mi}{11}\PYG{p}{,}\PYG{l+m+mi}{12}\PYG{p}{,}\PYG{l+m+mi}{13}\PYG{p}{,}\PYG{l+m+mi}{14}\PYG{p}{,}\PYG{l+m+mi}{15}\PYG{p}{]}
\PYG{n}{ss2} \PYG{o}{=} \PYG{n}{pd}\PYG{o}{.}\PYG{n}{Series}\PYG{p}{(}\PYG{n}{ss1}\PYG{p}{)}
\PYG{n+nb}{print}\PYG{p}{(}\PYG{n}{ss2}\PYG{p}{)}

\PYG{n}{ss3} \PYG{o}{=} \PYG{n}{pd}\PYG{o}{.}\PYG{n}{Series}\PYG{p}{(}\PYG{n}{ss1}\PYG{p}{,} \PYG{n}{index}\PYG{o}{=}\PYG{p}{[}\PYG{l+s+s1}{\PYGZsq{}}\PYG{l+s+s1}{a}\PYG{l+s+s1}{\PYGZsq{}}\PYG{p}{,} \PYG{l+s+s1}{\PYGZsq{}}\PYG{l+s+s1}{b}\PYG{l+s+s1}{\PYGZsq{}}\PYG{p}{,} \PYG{l+s+s1}{\PYGZsq{}}\PYG{l+s+s1}{c}\PYG{l+s+s1}{\PYGZsq{}}\PYG{p}{,} \PYG{l+s+s1}{\PYGZsq{}}\PYG{l+s+s1}{d}\PYG{l+s+s1}{\PYGZsq{}}\PYG{p}{,} \PYG{l+s+s1}{\PYGZsq{}}\PYG{l+s+s1}{e}\PYG{l+s+s1}{\PYGZsq{}}\PYG{p}{]}\PYG{p}{)}
\PYG{n+nb}{print}\PYG{p}{(}\PYG{n}{ss3}\PYG{p}{)}
\end{sphinxVerbatim}

\end{sphinxuseclass}\end{sphinxVerbatimInput}
\begin{sphinxVerbatimOutput}

\begin{sphinxuseclass}{cell_output}
\begin{sphinxVerbatim}[commandchars=\\\{\}]
0    11
1    12
2    13
3    14
4    15
dtype: int64
a    11
b    12
c    13
d    14
e    15
dtype: int64
\end{sphinxVerbatim}

\end{sphinxuseclass}\end{sphinxVerbatimOutput}

\end{sphinxuseclass}

\part{Index 활용}
\label{\detokenize{chapter2/2.1.2_Python_Basics:id1}}
\sphinxAtStartPar
Index 의 본연의 기능은 찾기입니다. ss3.loc{[}인덱스값{]} 를 이용하여 원하는 값을 찾을 수 있습니다. 인덱스 ‘c’ 에 해당하는 값은 13입니다. ss3.loc{[}‘c’{]} 를 하면 13이 출력됩니다. 만약, 인덱스 ‘a’ 와 ‘c’ 를 다 찾고 싶으면 {[}‘a’, ‘c’{]} 와 같이 List 로 넣어주면 됩니다. loc 를 하지 않아도 같은 결과를 얻으시겠지만, loc 를 넣으면 ‘a’,’c’ 를 column 이 아니라 index 에서 찾는다는 것을 명확하게 해 줍니다.

\begin{sphinxuseclass}{cell}\begin{sphinxVerbatimInput}

\begin{sphinxuseclass}{cell_input}
\begin{sphinxVerbatim}[commandchars=\\\{\}]
\PYG{n+nb}{print}\PYG{p}{(}\PYG{n}{ss3}\PYG{o}{.}\PYG{n}{loc}\PYG{p}{[}\PYG{l+s+s1}{\PYGZsq{}}\PYG{l+s+s1}{a}\PYG{l+s+s1}{\PYGZsq{}}\PYG{p}{]}\PYG{p}{,} \PYG{n}{ss3}\PYG{p}{[}\PYG{l+s+s1}{\PYGZsq{}}\PYG{l+s+s1}{c}\PYG{l+s+s1}{\PYGZsq{}}\PYG{p}{]}\PYG{p}{)}
\PYG{n+nb}{print}\PYG{p}{(}\PYG{n}{ss3}\PYG{o}{.}\PYG{n}{loc}\PYG{p}{[}\PYG{p}{[}\PYG{l+s+s1}{\PYGZsq{}}\PYG{l+s+s1}{a}\PYG{l+s+s1}{\PYGZsq{}}\PYG{p}{,}\PYG{l+s+s1}{\PYGZsq{}}\PYG{l+s+s1}{c}\PYG{l+s+s1}{\PYGZsq{}}\PYG{p}{]}\PYG{p}{]}\PYG{p}{)}
\end{sphinxVerbatim}

\end{sphinxuseclass}\end{sphinxVerbatimInput}
\begin{sphinxVerbatimOutput}

\begin{sphinxuseclass}{cell_output}
\begin{sphinxVerbatim}[commandchars=\\\{\}]
11 13
a    11
c    13
dtype: int64
\end{sphinxVerbatim}

\end{sphinxuseclass}\end{sphinxVerbatimOutput}

\end{sphinxuseclass}
\sphinxAtStartPar
 DataFrame 에서도 동일하게 활용가능합니다. 먼저 df1 이라는 DataFrame 을 생성하고 출력합니다. Default Index 인 숫자 0 \textasciitilde{} 4 로 되어 있음을 확인할 수 있습니다. 이제 원하는 인덱스 s1 \textasciitilde{} s5 를 할당하고 df2 에 저장합니다. 출력 결과를 보니 df2 의 인덱스가 바뀌었습니다.

\sphinxAtStartPar
이번에는 원하는 값을 찾아보겠습니다. df2 의 index 가 ‘s3’ 인 c1 컬럼값을 알고 싶다면 df2.loc{[}‘s3’{]}{[}‘c1’{]} 이라고 하면 됩니다. 만약, c1 과 c2 둘다 출력하고 싶으면  df2.loc{[}‘s3’{]}{[}{[}‘c1’,’c2’{]}{]} 형태로 리스트로 입력합니다. 실수로 df2.loc{[}‘s3’{]}{[}‘c1’,’c2’{]} 로 입력을 하면 Pandas 패키지는 ‘c1,’c2’ 가 하나의 column 이름이라고 착각하게 되어 에러가 발생합니다.

\begin{sphinxuseclass}{cell}\begin{sphinxVerbatimInput}

\begin{sphinxuseclass}{cell_input}
\begin{sphinxVerbatim}[commandchars=\\\{\}]
\PYG{c+c1}{\PYGZsh{} DataFrame 생성}
\PYG{n}{c1\PYGZus{}list} \PYG{o}{=} \PYG{p}{[}\PYG{l+m+mi}{11}\PYG{p}{,}\PYG{l+m+mi}{12}\PYG{p}{,}\PYG{l+m+mi}{13}\PYG{p}{,}\PYG{l+m+mi}{14}\PYG{p}{,}\PYG{l+m+mi}{15}\PYG{p}{]}
\PYG{n}{c2\PYGZus{}list} \PYG{o}{=} \PYG{p}{[}\PYG{l+s+s1}{\PYGZsq{}}\PYG{l+s+s1}{a}\PYG{l+s+s1}{\PYGZsq{}}\PYG{p}{,}\PYG{l+s+s1}{\PYGZsq{}}\PYG{l+s+s1}{b}\PYG{l+s+s1}{\PYGZsq{}}\PYG{p}{,}\PYG{l+s+s1}{\PYGZsq{}}\PYG{l+s+s1}{c}\PYG{l+s+s1}{\PYGZsq{}}\PYG{p}{,}\PYG{l+s+s1}{\PYGZsq{}}\PYG{l+s+s1}{d}\PYG{l+s+s1}{\PYGZsq{}}\PYG{p}{,}\PYG{l+s+s1}{\PYGZsq{}}\PYG{l+s+s1}{e}\PYG{l+s+s1}{\PYGZsq{}}\PYG{p}{]}
\PYG{n}{df1} \PYG{o}{=} \PYG{n}{pd}\PYG{o}{.}\PYG{n}{DataFrame}\PYG{p}{(}\PYG{p}{\PYGZob{}}\PYG{l+s+s1}{\PYGZsq{}}\PYG{l+s+s1}{c1}\PYG{l+s+s1}{\PYGZsq{}}\PYG{p}{:} \PYG{n}{c1\PYGZus{}list}\PYG{p}{,} \PYG{l+s+s1}{\PYGZsq{}}\PYG{l+s+s1}{c2}\PYG{l+s+s1}{\PYGZsq{}}\PYG{p}{:} \PYG{n}{c2\PYGZus{}list}\PYG{p}{\PYGZcb{}}\PYG{p}{)}
\PYG{n+nb}{print}\PYG{p}{(}\PYG{n}{df1}\PYG{p}{)}

\PYG{n+nb}{print}\PYG{p}{(}\PYG{l+s+s1}{\PYGZsq{}}\PYG{l+s+se}{\PYGZbs{}n}\PYG{l+s+s1}{\PYGZsq{}}\PYG{p}{)}
\PYG{n}{df2} \PYG{o}{=} \PYG{n}{pd}\PYG{o}{.}\PYG{n}{DataFrame}\PYG{p}{(}\PYG{p}{\PYGZob{}}\PYG{l+s+s1}{\PYGZsq{}}\PYG{l+s+s1}{c1}\PYG{l+s+s1}{\PYGZsq{}}\PYG{p}{:} \PYG{n}{c1\PYGZus{}list}\PYG{p}{,} \PYG{l+s+s1}{\PYGZsq{}}\PYG{l+s+s1}{c2}\PYG{l+s+s1}{\PYGZsq{}}\PYG{p}{:} \PYG{n}{c2\PYGZus{}list}\PYG{p}{\PYGZcb{}}\PYG{p}{,} \PYG{n}{index}\PYG{o}{=}\PYG{p}{[}\PYG{l+s+s1}{\PYGZsq{}}\PYG{l+s+s1}{s1}\PYG{l+s+s1}{\PYGZsq{}}\PYG{p}{,}\PYG{l+s+s1}{\PYGZsq{}}\PYG{l+s+s1}{s2}\PYG{l+s+s1}{\PYGZsq{}}\PYG{p}{,}\PYG{l+s+s1}{\PYGZsq{}}\PYG{l+s+s1}{s3}\PYG{l+s+s1}{\PYGZsq{}}\PYG{p}{,}\PYG{l+s+s1}{\PYGZsq{}}\PYG{l+s+s1}{s4}\PYG{l+s+s1}{\PYGZsq{}}\PYG{p}{,}\PYG{l+s+s1}{\PYGZsq{}}\PYG{l+s+s1}{s4}\PYG{l+s+s1}{\PYGZsq{}}\PYG{p}{]}\PYG{p}{)}
\PYG{n+nb}{print}\PYG{p}{(}\PYG{n}{df2}\PYG{p}{)}

\PYG{n+nb}{print}\PYG{p}{(}\PYG{l+s+s1}{\PYGZsq{}}\PYG{l+s+se}{\PYGZbs{}n}\PYG{l+s+s1}{\PYGZsq{}}\PYG{p}{)}
\PYG{n+nb}{print}\PYG{p}{(}\PYG{n}{df2}\PYG{o}{.}\PYG{n}{loc}\PYG{p}{[}\PYG{l+s+s1}{\PYGZsq{}}\PYG{l+s+s1}{s3}\PYG{l+s+s1}{\PYGZsq{}}\PYG{p}{]}\PYG{p}{[}\PYG{l+s+s1}{\PYGZsq{}}\PYG{l+s+s1}{c1}\PYG{l+s+s1}{\PYGZsq{}}\PYG{p}{]}\PYG{p}{)} \PYG{c+c1}{\PYGZsh{} 13 출력}
\PYG{n+nb}{print}\PYG{p}{(}\PYG{n}{df2}\PYG{o}{.}\PYG{n}{loc}\PYG{p}{[}\PYG{l+s+s1}{\PYGZsq{}}\PYG{l+s+s1}{s3}\PYG{l+s+s1}{\PYGZsq{}}\PYG{p}{]}\PYG{p}{[}\PYG{p}{[}\PYG{l+s+s1}{\PYGZsq{}}\PYG{l+s+s1}{c1}\PYG{l+s+s1}{\PYGZsq{}}\PYG{p}{,}\PYG{l+s+s1}{\PYGZsq{}}\PYG{l+s+s1}{c2}\PYG{l+s+s1}{\PYGZsq{}}\PYG{p}{]}\PYG{p}{]}\PYG{p}{)} \PYG{c+c1}{\PYGZsh{} 13 과 c 출력}
\end{sphinxVerbatim}

\end{sphinxuseclass}\end{sphinxVerbatimInput}
\begin{sphinxVerbatimOutput}

\begin{sphinxuseclass}{cell_output}
\begin{sphinxVerbatim}[commandchars=\\\{\}]
   c1 c2
0  11  a
1  12  b
2  13  c
3  14  d
4  15  e


    c1 c2
s1  11  a
s2  12  b
s3  13  c
s4  14  d
s4  15  e


13
c1    13
c2     c
Name: s3, dtype: object
\end{sphinxVerbatim}

\end{sphinxuseclass}\end{sphinxVerbatimOutput}

\end{sphinxuseclass}

\part{Index 생성 및 추출}
\label{\detokenize{chapter2/2.1.2_Python_Basics:id2}}
\sphinxAtStartPar
set\_index 메소드로 기존의 column 을 index 로 만들 수 있습니다. set\_index(‘c2’) 처리 후, df2 를 출력하시면 df1 의 ‘c2’ 컬럼이 index 로 되어 있음을 확인할 수 있습니다.
이제 df2 의 index 값을 변경해 보겠습니다. 아래와 같이 DataFrame 의 Index를 호출하여 원하는 Index 로 교체도 가능합니다. 참고로 아래 df2 는 column 하나지만 현재 Series 가 아닌 DataFrame 입니다.

\begin{sphinxuseclass}{cell}\begin{sphinxVerbatimInput}

\begin{sphinxuseclass}{cell_input}
\begin{sphinxVerbatim}[commandchars=\\\{\}]
\PYG{n}{c1\PYGZus{}list} \PYG{o}{=} \PYG{p}{[}\PYG{l+m+mi}{11}\PYG{p}{,}\PYG{l+m+mi}{12}\PYG{p}{,}\PYG{l+m+mi}{13}\PYG{p}{,}\PYG{l+m+mi}{14}\PYG{p}{,}\PYG{l+m+mi}{15}\PYG{p}{]}
\PYG{n}{c2\PYGZus{}list} \PYG{o}{=} \PYG{p}{[}\PYG{l+s+s1}{\PYGZsq{}}\PYG{l+s+s1}{a}\PYG{l+s+s1}{\PYGZsq{}}\PYG{p}{,}\PYG{l+s+s1}{\PYGZsq{}}\PYG{l+s+s1}{b}\PYG{l+s+s1}{\PYGZsq{}}\PYG{p}{,}\PYG{l+s+s1}{\PYGZsq{}}\PYG{l+s+s1}{c}\PYG{l+s+s1}{\PYGZsq{}}\PYG{p}{,}\PYG{l+s+s1}{\PYGZsq{}}\PYG{l+s+s1}{d}\PYG{l+s+s1}{\PYGZsq{}}\PYG{p}{,}\PYG{l+s+s1}{\PYGZsq{}}\PYG{l+s+s1}{e}\PYG{l+s+s1}{\PYGZsq{}}\PYG{p}{]}
\PYG{n}{df1} \PYG{o}{=} \PYG{n}{pd}\PYG{o}{.}\PYG{n}{DataFrame}\PYG{p}{(}\PYG{p}{\PYGZob{}}\PYG{l+s+s1}{\PYGZsq{}}\PYG{l+s+s1}{c1}\PYG{l+s+s1}{\PYGZsq{}}\PYG{p}{:} \PYG{n}{c1\PYGZus{}list}\PYG{p}{,} \PYG{l+s+s1}{\PYGZsq{}}\PYG{l+s+s1}{c2}\PYG{l+s+s1}{\PYGZsq{}}\PYG{p}{:} \PYG{n}{c2\PYGZus{}list}\PYG{p}{\PYGZcb{}}\PYG{p}{)}
\PYG{n+nb}{print}\PYG{p}{(}\PYG{n}{df1}\PYG{p}{)}        

\PYG{n}{df2} \PYG{o}{=} \PYG{n}{df1}\PYG{o}{.}\PYG{n}{set\PYGZus{}index}\PYG{p}{(}\PYG{l+s+s1}{\PYGZsq{}}\PYG{l+s+s1}{c2}\PYG{l+s+s1}{\PYGZsq{}}\PYG{p}{)} \PYG{c+c1}{\PYGZsh{} c1 를 index 로 변경}
\PYG{n+nb}{print}\PYG{p}{(}\PYG{n}{df2}\PYG{p}{)}

\PYG{n+nb}{print}\PYG{p}{(}\PYG{l+s+s1}{\PYGZsq{}}\PYG{l+s+se}{\PYGZbs{}n}\PYG{l+s+s1}{\PYGZsq{}}\PYG{p}{)}
\PYG{n}{df2}\PYG{o}{.}\PYG{n}{index} \PYG{o}{=} \PYG{p}{[}\PYG{l+s+s1}{\PYGZsq{}}\PYG{l+s+s1}{ss1}\PYG{l+s+s1}{\PYGZsq{}}\PYG{p}{,} \PYG{l+s+s1}{\PYGZsq{}}\PYG{l+s+s1}{ss2}\PYG{l+s+s1}{\PYGZsq{}}\PYG{p}{,} \PYG{l+s+s1}{\PYGZsq{}}\PYG{l+s+s1}{ss3}\PYG{l+s+s1}{\PYGZsq{}}\PYG{p}{,} \PYG{l+s+s1}{\PYGZsq{}}\PYG{l+s+s1}{ss4}\PYG{l+s+s1}{\PYGZsq{}}\PYG{p}{,} \PYG{l+s+s1}{\PYGZsq{}}\PYG{l+s+s1}{ss5}\PYG{l+s+s1}{\PYGZsq{}}\PYG{p}{]}
\PYG{n+nb}{print}\PYG{p}{(}\PYG{n}{df2}\PYG{p}{,} \PYG{n+nb}{type}\PYG{p}{(}\PYG{n}{df2}\PYG{p}{)}\PYG{p}{)}
\end{sphinxVerbatim}

\end{sphinxuseclass}\end{sphinxVerbatimInput}
\begin{sphinxVerbatimOutput}

\begin{sphinxuseclass}{cell_output}
\begin{sphinxVerbatim}[commandchars=\\\{\}]
   c1 c2
0  11  a
1  12  b
2  13  c
3  14  d
4  15  e
    c1
c2    
a   11
b   12
c   13
d   14
e   15


     c1
ss1  11
ss2  12
ss3  13
ss4  14
ss5  15 \PYGZlt{}class \PYGZsq{}pandas.core.frame.DataFrame\PYGZsq{}\PYGZgt{}
\end{sphinxVerbatim}

\end{sphinxuseclass}\end{sphinxVerbatimOutput}

\end{sphinxuseclass}
\sphinxAtStartPar
 항상 두 데이터셋을 index 로 병합할 때는 index 에 중복이 있는지 확인을 할 필요가 있습니다. index 가 중복 여부를 체크하는 인수는 verify\_integriry 입니다. 아래는 중복이 있는 경우 에러를 발생시킵니다.

\begin{sphinxuseclass}{cell}\begin{sphinxVerbatimInput}

\begin{sphinxuseclass}{cell_input}
\begin{sphinxVerbatim}[commandchars=\\\{\}]
\PYG{n}{c1\PYGZus{}list} \PYG{o}{=} \PYG{p}{[}\PYG{l+m+mi}{11}\PYG{p}{,}\PYG{l+m+mi}{12}\PYG{p}{,}\PYG{l+m+mi}{13}\PYG{p}{,}\PYG{l+m+mi}{14}\PYG{p}{,}\PYG{l+m+mi}{15}\PYG{p}{]}
\PYG{n}{c2\PYGZus{}list} \PYG{o}{=} \PYG{p}{[}\PYG{l+s+s1}{\PYGZsq{}}\PYG{l+s+s1}{a}\PYG{l+s+s1}{\PYGZsq{}}\PYG{p}{,}\PYG{l+s+s1}{\PYGZsq{}}\PYG{l+s+s1}{a}\PYG{l+s+s1}{\PYGZsq{}}\PYG{p}{,}\PYG{l+s+s1}{\PYGZsq{}}\PYG{l+s+s1}{b}\PYG{l+s+s1}{\PYGZsq{}}\PYG{p}{,}\PYG{l+s+s1}{\PYGZsq{}}\PYG{l+s+s1}{c}\PYG{l+s+s1}{\PYGZsq{}}\PYG{p}{,}\PYG{l+s+s1}{\PYGZsq{}}\PYG{l+s+s1}{d}\PYG{l+s+s1}{\PYGZsq{}}\PYG{p}{]} \PYG{c+c1}{\PYGZsh{} 값에 중복이 있음}
\PYG{n}{df} \PYG{o}{=} \PYG{n}{pd}\PYG{o}{.}\PYG{n}{DataFrame}\PYG{p}{(}\PYG{p}{\PYGZob{}}\PYG{l+s+s1}{\PYGZsq{}}\PYG{l+s+s1}{c1}\PYG{l+s+s1}{\PYGZsq{}}\PYG{p}{:} \PYG{n}{c1\PYGZus{}list}\PYG{p}{,} \PYG{l+s+s1}{\PYGZsq{}}\PYG{l+s+s1}{c2}\PYG{l+s+s1}{\PYGZsq{}}\PYG{p}{:} \PYG{n}{c2\PYGZus{}list}\PYG{p}{\PYGZcb{}}\PYG{p}{)}
\PYG{n}{df}\PYG{o}{.}\PYG{n}{set\PYGZus{}index}\PYG{p}{(}\PYG{l+s+s1}{\PYGZsq{}}\PYG{l+s+s1}{c2}\PYG{l+s+s1}{\PYGZsq{}}\PYG{p}{,} \PYG{n}{verify\PYGZus{}integrity}\PYG{o}{=}\PYG{k+kc}{True}\PYG{p}{)} \PYG{c+c1}{\PYGZsh{} index 중복여부를 체크}
\end{sphinxVerbatim}

\end{sphinxuseclass}\end{sphinxVerbatimInput}
\begin{sphinxVerbatimOutput}

\begin{sphinxuseclass}{cell_output}
\begin{sphinxVerbatim}[commandchars=\\\{\}]
\PYG{g+gt}{\PYGZhy{}\PYGZhy{}\PYGZhy{}\PYGZhy{}\PYGZhy{}\PYGZhy{}\PYGZhy{}\PYGZhy{}\PYGZhy{}\PYGZhy{}\PYGZhy{}\PYGZhy{}\PYGZhy{}\PYGZhy{}\PYGZhy{}\PYGZhy{}\PYGZhy{}\PYGZhy{}\PYGZhy{}\PYGZhy{}\PYGZhy{}\PYGZhy{}\PYGZhy{}\PYGZhy{}\PYGZhy{}\PYGZhy{}\PYGZhy{}\PYGZhy{}\PYGZhy{}\PYGZhy{}\PYGZhy{}\PYGZhy{}\PYGZhy{}\PYGZhy{}\PYGZhy{}\PYGZhy{}\PYGZhy{}\PYGZhy{}\PYGZhy{}\PYGZhy{}\PYGZhy{}\PYGZhy{}\PYGZhy{}\PYGZhy{}\PYGZhy{}\PYGZhy{}\PYGZhy{}\PYGZhy{}\PYGZhy{}\PYGZhy{}\PYGZhy{}\PYGZhy{}\PYGZhy{}\PYGZhy{}\PYGZhy{}\PYGZhy{}\PYGZhy{}\PYGZhy{}\PYGZhy{}\PYGZhy{}\PYGZhy{}\PYGZhy{}\PYGZhy{}\PYGZhy{}\PYGZhy{}\PYGZhy{}\PYGZhy{}\PYGZhy{}\PYGZhy{}\PYGZhy{}\PYGZhy{}\PYGZhy{}\PYGZhy{}\PYGZhy{}\PYGZhy{}}
\PYG{n+ne}{ValueError}\PYG{g+gWhitespace}{                                }Traceback (most recent call last)
\PYG{o}{\PYGZti{}}\PYGZbs{}\PYG{n}{AppData}\PYGZbs{}\PYG{n}{Local}\PYGZbs{}\PYG{n}{Temp}\PYGZbs{}\PYG{n}{ipykernel\PYGZus{}3484}\PYGZbs{}\PYG{l+m+mf}{2078573242.}\PYG{n}{py} \PYG{o+ow}{in} \PYG{o}{\PYGZlt{}}\PYG{n}{module}\PYG{o}{\PYGZgt{}}
\PYG{g+gWhitespace}{      }\PYG{l+m+mi}{2} \PYG{n}{c2\PYGZus{}list} \PYG{o}{=} \PYG{p}{[}\PYG{l+s+s1}{\PYGZsq{}}\PYG{l+s+s1}{a}\PYG{l+s+s1}{\PYGZsq{}}\PYG{p}{,}\PYG{l+s+s1}{\PYGZsq{}}\PYG{l+s+s1}{a}\PYG{l+s+s1}{\PYGZsq{}}\PYG{p}{,}\PYG{l+s+s1}{\PYGZsq{}}\PYG{l+s+s1}{b}\PYG{l+s+s1}{\PYGZsq{}}\PYG{p}{,}\PYG{l+s+s1}{\PYGZsq{}}\PYG{l+s+s1}{c}\PYG{l+s+s1}{\PYGZsq{}}\PYG{p}{,}\PYG{l+s+s1}{\PYGZsq{}}\PYG{l+s+s1}{d}\PYG{l+s+s1}{\PYGZsq{}}\PYG{p}{]} \PYG{c+c1}{\PYGZsh{} 값에 중복이 있음}
\PYG{g+gWhitespace}{      }\PYG{l+m+mi}{3} \PYG{n}{df} \PYG{o}{=} \PYG{n}{pd}\PYG{o}{.}\PYG{n}{DataFrame}\PYG{p}{(}\PYG{p}{\PYGZob{}}\PYG{l+s+s1}{\PYGZsq{}}\PYG{l+s+s1}{c1}\PYG{l+s+s1}{\PYGZsq{}}\PYG{p}{:} \PYG{n}{c1\PYGZus{}list}\PYG{p}{,} \PYG{l+s+s1}{\PYGZsq{}}\PYG{l+s+s1}{c2}\PYG{l+s+s1}{\PYGZsq{}}\PYG{p}{:} \PYG{n}{c2\PYGZus{}list}\PYG{p}{\PYGZcb{}}\PYG{p}{)}
\PYG{n+ne}{\PYGZhy{}\PYGZhy{}\PYGZhy{}\PYGZhy{}\PYGZgt{} }\PYG{l+m+mi}{4} \PYG{n}{df}\PYG{o}{.}\PYG{n}{set\PYGZus{}index}\PYG{p}{(}\PYG{l+s+s1}{\PYGZsq{}}\PYG{l+s+s1}{c2}\PYG{l+s+s1}{\PYGZsq{}}\PYG{p}{,} \PYG{n}{verify\PYGZus{}integrity}\PYG{o}{=}\PYG{k+kc}{True}\PYG{p}{)} \PYG{c+c1}{\PYGZsh{} index 중복여부를 체크}

\PYG{n+nn}{\PYGZti{}\PYGZbs{}Anaconda3\PYGZbs{}lib\PYGZbs{}site\PYGZhy{}packages\PYGZbs{}pandas\PYGZbs{}util\PYGZbs{}\PYGZus{}decorators.py} in \PYG{n+ni}{wrapper}\PYG{n+nt}{(*args, **kwargs)}
\PYG{g+gWhitespace}{    }\PYG{l+m+mi}{309}                     \PYG{n}{stacklevel}\PYG{o}{=}\PYG{n}{stacklevel}\PYG{p}{,}
\PYG{g+gWhitespace}{    }\PYG{l+m+mi}{310}                 \PYG{p}{)}
\PYG{n+ne}{\PYGZhy{}\PYGZhy{}\PYGZgt{} }\PYG{l+m+mi}{311}             \PYG{k}{return} \PYG{n}{func}\PYG{p}{(}\PYG{o}{*}\PYG{n}{args}\PYG{p}{,} \PYG{o}{*}\PYG{o}{*}\PYG{n}{kwargs}\PYG{p}{)}
\PYG{g+gWhitespace}{    }\PYG{l+m+mi}{312} 
\PYG{g+gWhitespace}{    }\PYG{l+m+mi}{313}         \PYG{k}{return} \PYG{n}{wrapper}

\PYG{n+nn}{\PYGZti{}\PYGZbs{}Anaconda3\PYGZbs{}lib\PYGZbs{}site\PYGZhy{}packages\PYGZbs{}pandas\PYGZbs{}core\PYGZbs{}frame.py} in \PYG{n+ni}{set\PYGZus{}index}\PYG{n+nt}{(self, keys, drop, append, inplace, verify\PYGZus{}integrity)}
\PYG{g+gWhitespace}{   }\PYG{l+m+mi}{5508}         \PYG{k}{if} \PYG{n}{verify\PYGZus{}integrity} \PYG{o+ow}{and} \PYG{o+ow}{not} \PYG{n}{index}\PYG{o}{.}\PYG{n}{is\PYGZus{}unique}\PYG{p}{:}
\PYG{g+gWhitespace}{   }\PYG{l+m+mi}{5509}             \PYG{n}{duplicates} \PYG{o}{=} \PYG{n}{index}\PYG{p}{[}\PYG{n}{index}\PYG{o}{.}\PYG{n}{duplicated}\PYG{p}{(}\PYG{p}{)}\PYG{p}{]}\PYG{o}{.}\PYG{n}{unique}\PYG{p}{(}\PYG{p}{)}
\PYG{n+ne}{\PYGZhy{}\PYGZgt{} }\PYG{l+m+mi}{5510}             \PYG{k}{raise} \PYG{n+ne}{ValueError}\PYG{p}{(}\PYG{l+s+sa}{f}\PYG{l+s+s2}{\PYGZdq{}}\PYG{l+s+s2}{Index has duplicate keys: }\PYG{l+s+si}{\PYGZob{}}\PYG{n}{duplicates}\PYG{l+s+si}{\PYGZcb{}}\PYG{l+s+s2}{\PYGZdq{}}\PYG{p}{)}
\PYG{g+gWhitespace}{   }\PYG{l+m+mi}{5511} 
\PYG{g+gWhitespace}{   }\PYG{l+m+mi}{5512}         \PYG{c+c1}{\PYGZsh{} use set to handle duplicate column names gracefully in case of drop}

\PYG{n+ne}{ValueError}: Index has duplicate keys: Index([\PYGZsq{}a\PYGZsq{}], dtype=\PYGZsq{}object\PYGZsq{}, name=\PYGZsq{}c2\PYGZsq{})
\end{sphinxVerbatim}

\end{sphinxuseclass}\end{sphinxVerbatimOutput}

\end{sphinxuseclass}






\renewcommand{\indexname}{Index}
\printindex
\end{document}