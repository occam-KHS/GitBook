%% Generated by Sphinx.
\def\sphinxdocclass{jupyterBook}
\documentclass[letterpaper,10pt,english]{jupyterBook}
\ifdefined\pdfpxdimen
   \let\sphinxpxdimen\pdfpxdimen\else\newdimen\sphinxpxdimen
\fi \sphinxpxdimen=.75bp\relax
\ifdefined\pdfimageresolution
    \pdfimageresolution= \numexpr \dimexpr1in\relax/\sphinxpxdimen\relax
\fi
%% let collapsible pdf bookmarks panel have high depth per default
\PassOptionsToPackage{bookmarksdepth=5}{hyperref}
%% turn off hyperref patch of \index as sphinx.xdy xindy module takes care of
%% suitable \hyperpage mark-up, working around hyperref-xindy incompatibility
\PassOptionsToPackage{hyperindex=false}{hyperref}
%% memoir class requires extra handling
\makeatletter\@ifclassloaded{memoir}
{\ifdefined\memhyperindexfalse\memhyperindexfalse\fi}{}\makeatother

\PassOptionsToPackage{warn}{textcomp}

\catcode`^^^^00a0\active\protected\def^^^^00a0{\leavevmode\nobreak\ }
\usepackage{cmap}
\usepackage{fontspec}
\defaultfontfeatures[\rmfamily,\sffamily,\ttfamily]{}
\usepackage{amsmath,amssymb,amstext}
\usepackage{polyglossia}
\setmainlanguage{english}



\setmainfont{FreeSerif}[
  Extension      = .otf,
  UprightFont    = *,
  ItalicFont     = *Italic,
  BoldFont       = *Bold,
  BoldItalicFont = *BoldItalic
]
\setsansfont{FreeSans}[
  Extension      = .otf,
  UprightFont    = *,
  ItalicFont     = *Oblique,
  BoldFont       = *Bold,
  BoldItalicFont = *BoldOblique,
]
\setmonofont{FreeMono}[
  Extension      = .otf,
  UprightFont    = *,
  ItalicFont     = *Oblique,
  BoldFont       = *Bold,
  BoldItalicFont = *BoldOblique,
]



\usepackage[Bjarne]{fncychap}
\usepackage[,numfigreset=1,mathnumfig]{sphinx}

\fvset{fontsize=\small}
\usepackage{geometry}


% Include hyperref last.
\usepackage{hyperref}
% Fix anchor placement for figures with captions.
\usepackage{hypcap}% it must be loaded after hyperref.
% Set up styles of URL: it should be placed after hyperref.
\urlstyle{same}


\usepackage{sphinxmessages}



        % Start of preamble defined in sphinx-jupyterbook-latex %
         \usepackage[Latin,Greek]{ucharclasses}
        \usepackage{unicode-math}
        % fixing title of the toc
        \addto\captionsenglish{\renewcommand{\contentsname}{Contents}}
        \hypersetup{
            pdfencoding=auto,
            psdextra
        }
        % End of preamble defined in sphinx-jupyterbook-latex %
        

\title{소매업 사례}
\date{Jul 02, 2022}
\release{}
\author{KHS}
\newcommand{\sphinxlogo}{\vbox{}}
\renewcommand{\releasename}{}
\makeindex
\begin{document}

\pagestyle{empty}
\sphinxmaketitle
\pagestyle{plain}
\sphinxtableofcontents
\pagestyle{normal}
\phantomsection\label{\detokenize{chapter3/3.1.3_Use_Case::doc}}


\sphinxAtStartPar
문제해결을 위하여 데이터분석을 하는 경우가 대부분입니다.  처음 보험사 사례에서 설명드린 것과 같이 가설을 세우고, 가설을 검증하기 위하여 데이터분석을 합니다. 검증된 가설은 문제해결의 근거가 됩니다. 다른 접근법은 데이터베이스를 알고리즘으로 분석해 패턴을 찾아내는 방법인데, 이를 데이터마이닝 접근법이라고 부릅니다. 이 번 사례에서는 데이터 마이닝 사례 두 가지를 소개하겠습니다.

\sphinxAtStartPar
월마트는 미국의 큰 소매업체입니다. 우리나라 이마트 정도가 비슷할 것 같은데요. 카트에 사고 싶은 물건을 담고, 계산대에서 일괄로 지불합니다. 그 때 구매내역이 찍힌 영수증이 발행됩니다. 이 데이터는 월마트 내부 전산시스템에도 동일하게 저장이 되게 됩니다. 데이터분석가가 어떤 상품들이 같이 판매가 되는지 궁금해 분석을 해 보았습니다. 이 분석은 장바구니 분석(Market Basket Analysis)라고도 부릅니다. 이상하게 금요일 오후에 맥주와 아기 기저귀와 같은 영수증에 동일하게 찍히는 경우가 많다는 것이 분석 결과로 도출되었습니다. 이 사실을 영업총괄 매니저에게 보고를 했고, 총괄 매니저는 맥주 옆에 기저귀를 같이 진열을 해 보았습니다. 그 결과는 맥주와 아기 기저귀 매출이 두 배로 증가했습니다. 의아하게 생각한 매니저는 금요일 진열대 옆에서 어떤 고객들이 맥주와 기저귀를 같이 구매를 하는 지 관찰 해 보았습니다. 알고 보니, 금요일 퇴근한 젊은 아빠들이 스트레스를 받은 표정(와이프의 요청으로 퇴근 후에 피곤한 몸을 이끌고 마트에 온 것으로 추정)으로 기저귀를 사러왔다가 맥주도 같이 사가는 것이였습니다. 이렇듯 데이터마이닝으로 생각하지 못했던 통찰(Insight)를 얻을 수 도 있습니다.

\sphinxAtStartPar
이번에는 타겟 사례입니다. 타겟은 미국의 카탈로그 소매업체입니다. 타겟은 임산부를 위한 특별한 프로모션을 준비했고, 임산부에게 임신기간 중 필요한 다양한 상품을 소개하는 카탈로그 준비했습니다. 일단 임신을 하게되면, 출산과 육아기간 동안 필요한 제품들이 정해져 있는데요. 타겟은 그것을 노리고 대대적인 프로모션을 계획한 것입니다. 카탈로그는 독자 이름으로 우편 배달이 되는데요. 이 임산부용 카탈로그가 어느 한 여고생 집으로 배달이 된 것이였습니다. 그 사실은 안 학생 아버지는 화가 잔뜩 나서 회사에 전화를 걸어 항의했습니다. “우리 아이가 고등학생인데, 무슨 임산부 카탈로를 보내느냐? 당장 담당자가 직접와서 사과를 하지 않으면 업체를 고소하겠다”  그런데 몇 일 후 학생은 아버지에게 이런 이런 일로 임신을 했다고 고백을 하게되었습니다. 타겟의 고객 담당자는 몇 일 후 집으로 찾아왔고, 아버지는 이 사실을 이야기 할 수 밖에 없었습니다. 그렇다면, 타겟은 이 여고생이 임신한 사실을 어떻게 알았을까요?  타겟의 데이터 분석가는 임산부의 구매특성에 대하여 분석을 했었는데요. 대부분의 고객은 임신을 하게되면, 피부 로션과 헤어 샴푸를 화학성분이 없고, 향이 강하지 않은 오가닉 제품으로 변경한다는 사실을 발견했습니다. 바로 그 여고생이 제품을 오가닉으로 갑자기 변경한 고객 중의 하나였던 것입니다. 물론 연령을 고려하지 않은 타겟팅을 한 잘못이 있다고 생각됩니다. 타겟의 에피소드 역시 구매이력 데이터베이스를 분석하다가 예상하지 못한 것을 발견하고 마케팅에 활용한 데이터마이닝 사례 중 하나입니다.







\renewcommand{\indexname}{Index}
\printindex
\end{document}