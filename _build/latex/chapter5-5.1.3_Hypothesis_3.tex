%% Generated by Sphinx.
\def\sphinxdocclass{jupyterBook}
\documentclass[letterpaper,10pt,english]{jupyterBook}
\ifdefined\pdfpxdimen
   \let\sphinxpxdimen\pdfpxdimen\else\newdimen\sphinxpxdimen
\fi \sphinxpxdimen=.75bp\relax
\ifdefined\pdfimageresolution
    \pdfimageresolution= \numexpr \dimexpr1in\relax/\sphinxpxdimen\relax
\fi
%% let collapsible pdf bookmarks panel have high depth per default
\PassOptionsToPackage{bookmarksdepth=5}{hyperref}
%% turn off hyperref patch of \index as sphinx.xdy xindy module takes care of
%% suitable \hyperpage mark-up, working around hyperref-xindy incompatibility
\PassOptionsToPackage{hyperindex=false}{hyperref}
%% memoir class requires extra handling
\makeatletter\@ifclassloaded{memoir}
{\ifdefined\memhyperindexfalse\memhyperindexfalse\fi}{}\makeatother

\PassOptionsToPackage{warn}{textcomp}

\catcode`^^^^00a0\active\protected\def^^^^00a0{\leavevmode\nobreak\ }
\usepackage{cmap}
\usepackage{fontspec}
\defaultfontfeatures[\rmfamily,\sffamily,\ttfamily]{}
\usepackage{amsmath,amssymb,amstext}
\usepackage{polyglossia}
\setmainlanguage{english}



\setmainfont{FreeSerif}[
  Extension      = .otf,
  UprightFont    = *,
  ItalicFont     = *Italic,
  BoldFont       = *Bold,
  BoldItalicFont = *BoldItalic
]
\setsansfont{FreeSans}[
  Extension      = .otf,
  UprightFont    = *,
  ItalicFont     = *Oblique,
  BoldFont       = *Bold,
  BoldItalicFont = *BoldOblique,
]
\setmonofont{FreeMono}[
  Extension      = .otf,
  UprightFont    = *,
  ItalicFont     = *Oblique,
  BoldFont       = *Bold,
  BoldItalicFont = *BoldOblique,
]



\usepackage[Bjarne]{fncychap}
\usepackage[,numfigreset=1,mathnumfig]{sphinx}

\fvset{fontsize=\small}
\usepackage{geometry}


% Include hyperref last.
\usepackage{hyperref}
% Fix anchor placement for figures with captions.
\usepackage{hypcap}% it must be loaded after hyperref.
% Set up styles of URL: it should be placed after hyperref.
\urlstyle{same}


\usepackage{sphinxmessages}



        % Start of preamble defined in sphinx-jupyterbook-latex %
         \usepackage[Latin,Greek]{ucharclasses}
        \usepackage{unicode-math}
        % fixing title of the toc
        \addto\captionsenglish{\renewcommand{\contentsname}{Contents}}
        \hypersetup{
            pdfencoding=auto,
            psdextra
        }
        % End of preamble defined in sphinx-jupyterbook-latex %
        

\title{위 꼬리가 긴 양봉이 자주 발생한다.}
\date{Jul 02, 2022}
\release{}
\author{KHS}
\newcommand{\sphinxlogo}{\vbox{}}
\renewcommand{\releasename}{}
\makeindex
\begin{document}

\pagestyle{empty}
\sphinxmaketitle
\pagestyle{plain}
\sphinxtableofcontents
\pagestyle{normal}
\phantomsection\label{\detokenize{chapter5/5.1.3_Hypothesis_3::doc}}
\begin{sphinxuseclass}{cell}\begin{sphinxVerbatimInput}

\begin{sphinxuseclass}{cell_input}
\begin{sphinxVerbatim}[commandchars=\\\{\}]
\PYG{k+kn}{import} \PYG{n+nn}{FinanceDataReader} \PYG{k}{as} \PYG{n+nn}{fdr}
\PYG{o}{\PYGZpc{}}\PYG{k}{matplotlib} inline
\PYG{k+kn}{import} \PYG{n+nn}{matplotlib}\PYG{n+nn}{.}\PYG{n+nn}{pyplot} \PYG{k}{as} \PYG{n+nn}{plt}
\PYG{k+kn}{import} \PYG{n+nn}{pandas} \PYG{k}{as} \PYG{n+nn}{pd}
\PYG{k+kn}{import} \PYG{n+nn}{numpy} \PYG{k}{as} \PYG{n+nn}{np}
\PYG{k+kn}{import} \PYG{n+nn}{warnings}
\PYG{n}{warnings}\PYG{o}{.}\PYG{n}{filterwarnings}\PYG{p}{(}\PYG{l+s+s1}{\PYGZsq{}}\PYG{l+s+s1}{ignore}\PYG{l+s+s1}{\PYGZsq{}}\PYG{p}{)}

\PYG{n}{pd}\PYG{o}{.}\PYG{n}{options}\PYG{o}{.}\PYG{n}{display}\PYG{o}{.}\PYG{n}{float\PYGZus{}format} \PYG{o}{=} \PYG{l+s+s1}{\PYGZsq{}}\PYG{l+s+si}{\PYGZob{}:,.3f\PYGZcb{}}\PYG{l+s+s1}{\PYGZsq{}}\PYG{o}{.}\PYG{n}{format}
\end{sphinxVerbatim}

\end{sphinxuseclass}\end{sphinxVerbatimInput}

\end{sphinxuseclass}


\sphinxAtStartPar
위 꼬리는 종가보다 고가가 더 높이 위치해 있는 양봉입니다. 따라서 고가를 종가로 나눈 값이 1 보다 상당히 크면 위꼬리 양봉이라고 할 수 있습니다. 양봉의 조건은 종가가 시가보다 큰 것입니다. 이 것을 데이터로 표현합니다.

\begin{sphinxuseclass}{cell}\begin{sphinxVerbatimInput}

\begin{sphinxuseclass}{cell_input}
\begin{sphinxVerbatim}[commandchars=\\\{\}]
\PYG{n}{mdl\PYGZus{}data} \PYG{o}{=} \PYG{n}{pd}\PYG{o}{.}\PYG{n}{read\PYGZus{}pickle}\PYG{p}{(}\PYG{l+s+s1}{\PYGZsq{}}\PYG{l+s+s1}{mdl\PYGZus{}data.pkl}\PYG{l+s+s1}{\PYGZsq{}}\PYG{p}{)}
\PYG{n}{mdl\PYGZus{}data}\PYG{o}{.}\PYG{n}{head}\PYG{p}{(}\PYG{p}{)}\PYG{o}{.}\PYG{n}{style}\PYG{o}{.}\PYG{n}{set\PYGZus{}table\PYGZus{}attributes}\PYG{p}{(}\PYG{l+s+s1}{\PYGZsq{}}\PYG{l+s+s1}{style=}\PYG{l+s+s1}{\PYGZdq{}}\PYG{l+s+s1}{font\PYGZhy{}size: 12px}\PYG{l+s+s1}{\PYGZdq{}}\PYG{l+s+s1}{\PYGZsq{}}\PYG{p}{)}
\end{sphinxVerbatim}

\end{sphinxuseclass}\end{sphinxVerbatimInput}
\begin{sphinxVerbatimOutput}

\begin{sphinxuseclass}{cell_output}
\begin{sphinxVerbatim}[commandchars=\\\{\}]
\PYGZlt{}pandas.io.formats.style.Styler at 0x1a8d5107f70\PYGZgt{}
\end{sphinxVerbatim}

\end{sphinxuseclass}\end{sphinxVerbatimOutput}

\end{sphinxuseclass}
\begin{sphinxuseclass}{cell}\begin{sphinxVerbatimInput}

\begin{sphinxuseclass}{cell_input}
\begin{sphinxVerbatim}[commandchars=\\\{\}]
\PYG{n}{kosdaq\PYGZus{}list} \PYG{o}{=} \PYG{n}{pd}\PYG{o}{.}\PYG{n}{read\PYGZus{}pickle}\PYG{p}{(}\PYG{l+s+s1}{\PYGZsq{}}\PYG{l+s+s1}{kosdaq\PYGZus{}list.pkl}\PYG{l+s+s1}{\PYGZsq{}}\PYG{p}{)}

\PYG{n}{data\PYGZus{}h3} \PYG{o}{=} \PYG{n}{pd}\PYG{o}{.}\PYG{n}{DataFrame}\PYG{p}{(}\PYG{p}{)}

\PYG{k}{for} \PYG{n}{code} \PYG{o+ow}{in} \PYG{n}{kosdaq\PYGZus{}list}\PYG{p}{[}\PYG{l+s+s1}{\PYGZsq{}}\PYG{l+s+s1}{code}\PYG{l+s+s1}{\PYGZsq{}}\PYG{p}{]}\PYG{p}{:}

    \PYG{n}{data} \PYG{o}{=} \PYG{n}{mdl\PYGZus{}data}\PYG{p}{[}\PYG{n}{mdl\PYGZus{}data}\PYG{p}{[}\PYG{l+s+s1}{\PYGZsq{}}\PYG{l+s+s1}{code}\PYG{l+s+s1}{\PYGZsq{}}\PYG{p}{]}\PYG{o}{==}\PYG{n}{code}\PYG{p}{]}\PYG{o}{.}\PYG{n}{sort\PYGZus{}index}\PYG{p}{(}\PYG{p}{)}\PYG{o}{.}\PYG{n}{copy}\PYG{p}{(}\PYG{p}{)}
    
    \PYG{n}{data}\PYG{p}{[}\PYG{l+s+s1}{\PYGZsq{}}\PYG{l+s+s1}{positive\PYGZus{}candle}\PYG{l+s+s1}{\PYGZsq{}}\PYG{p}{]} \PYG{o}{=} \PYG{p}{(}\PYG{n}{data}\PYG{p}{[}\PYG{l+s+s1}{\PYGZsq{}}\PYG{l+s+s1}{close}\PYG{l+s+s1}{\PYGZsq{}}\PYG{p}{]} \PYG{o}{\PYGZgt{}} \PYG{n}{data}\PYG{p}{[}\PYG{l+s+s1}{\PYGZsq{}}\PYG{l+s+s1}{open}\PYG{l+s+s1}{\PYGZsq{}}\PYG{p}{]}\PYG{p}{)}\PYG{o}{.}\PYG{n}{astype}\PYG{p}{(}\PYG{n+nb}{int}\PYG{p}{)} \PYG{c+c1}{\PYGZsh{} 양봉}
    \PYG{n}{data}\PYG{p}{[}\PYG{l+s+s1}{\PYGZsq{}}\PYG{l+s+s1}{high/close}\PYG{l+s+s1}{\PYGZsq{}}\PYG{p}{]} \PYG{o}{=} \PYG{p}{(}\PYG{n}{data}\PYG{p}{[}\PYG{l+s+s1}{\PYGZsq{}}\PYG{l+s+s1}{positive\PYGZus{}candle}\PYG{l+s+s1}{\PYGZsq{}}\PYG{p}{]}\PYG{o}{==}\PYG{l+m+mi}{1}\PYG{p}{)}\PYG{o}{*}\PYG{p}{(}\PYG{n}{data}\PYG{p}{[}\PYG{l+s+s1}{\PYGZsq{}}\PYG{l+s+s1}{high}\PYG{l+s+s1}{\PYGZsq{}}\PYG{p}{]}\PYG{o}{/}\PYG{n}{data}\PYG{p}{[}\PYG{l+s+s1}{\PYGZsq{}}\PYG{l+s+s1}{close}\PYG{l+s+s1}{\PYGZsq{}}\PYG{p}{]} \PYG{o}{\PYGZgt{}} \PYG{l+m+mf}{1.1}\PYG{p}{)}\PYG{o}{.}\PYG{n}{astype}\PYG{p}{(}\PYG{n+nb}{int}\PYG{p}{)} \PYG{c+c1}{\PYGZsh{} 양봉이면서 고가가 종가보다 높게 위치 10\PYGZpc{} 이상 높은 경우}
    \PYG{n}{data}\PYG{p}{[}\PYG{l+s+s1}{\PYGZsq{}}\PYG{l+s+s1}{num\PYGZus{}high/close}\PYG{l+s+s1}{\PYGZsq{}}\PYG{p}{]} \PYG{o}{=}  \PYG{n}{data}\PYG{p}{[}\PYG{l+s+s1}{\PYGZsq{}}\PYG{l+s+s1}{high/close}\PYG{l+s+s1}{\PYGZsq{}}\PYG{p}{]}\PYG{o}{.}\PYG{n}{rolling}\PYG{p}{(}\PYG{l+m+mi}{20}\PYG{p}{)}\PYG{o}{.}\PYG{n}{sum}\PYG{p}{(}\PYG{p}{)}
       
    \PYG{n}{data}\PYG{p}{[}\PYG{l+s+s1}{\PYGZsq{}}\PYG{l+s+s1}{max\PYGZus{}close}\PYG{l+s+s1}{\PYGZsq{}}\PYG{p}{]}  \PYG{o}{=} \PYG{n}{data}\PYG{p}{[}\PYG{p}{[}\PYG{l+s+s1}{\PYGZsq{}}\PYG{l+s+s1}{close\PYGZus{}r1}\PYG{l+s+s1}{\PYGZsq{}}\PYG{p}{,}\PYG{l+s+s1}{\PYGZsq{}}\PYG{l+s+s1}{close\PYGZus{}r2}\PYG{l+s+s1}{\PYGZsq{}}\PYG{p}{,}\PYG{l+s+s1}{\PYGZsq{}}\PYG{l+s+s1}{close\PYGZus{}r3}\PYG{l+s+s1}{\PYGZsq{}}\PYG{p}{,}\PYG{l+s+s1}{\PYGZsq{}}\PYG{l+s+s1}{close\PYGZus{}r4}\PYG{l+s+s1}{\PYGZsq{}}\PYG{p}{,}\PYG{l+s+s1}{\PYGZsq{}}\PYG{l+s+s1}{close\PYGZus{}r5}\PYG{l+s+s1}{\PYGZsq{}}\PYG{p}{]}\PYG{p}{]}\PYG{o}{.}\PYG{n}{max}\PYG{p}{(}\PYG{n}{axis}\PYG{o}{=}\PYG{l+m+mi}{1}\PYG{p}{)} \PYG{c+c1}{\PYGZsh{} 5 영업일 종가 수익율 중 최고 값}
    \PYG{n}{data}\PYG{o}{.}\PYG{n}{dropna}\PYG{p}{(}\PYG{n}{subset}\PYG{o}{=}\PYG{p}{[}\PYG{l+s+s1}{\PYGZsq{}}\PYG{l+s+s1}{num\PYGZus{}high/close}\PYG{l+s+s1}{\PYGZsq{}}\PYG{p}{,}\PYG{l+s+s1}{\PYGZsq{}}\PYG{l+s+s1}{close\PYGZus{}r1}\PYG{l+s+s1}{\PYGZsq{}}\PYG{p}{,}\PYG{l+s+s1}{\PYGZsq{}}\PYG{l+s+s1}{close\PYGZus{}r2}\PYG{l+s+s1}{\PYGZsq{}}\PYG{p}{,}\PYG{l+s+s1}{\PYGZsq{}}\PYG{l+s+s1}{close\PYGZus{}r3}\PYG{l+s+s1}{\PYGZsq{}}\PYG{p}{,}\PYG{l+s+s1}{\PYGZsq{}}\PYG{l+s+s1}{close\PYGZus{}r4}\PYG{l+s+s1}{\PYGZsq{}}\PYG{p}{,}\PYG{l+s+s1}{\PYGZsq{}}\PYG{l+s+s1}{close\PYGZus{}r5}\PYG{l+s+s1}{\PYGZsq{}}\PYG{p}{]}\PYG{p}{,} \PYG{n}{inplace}\PYG{o}{=}\PYG{k+kc}{True}\PYG{p}{)} \PYG{c+c1}{\PYGZsh{} missing 이 있는 행은 제거  }
    
    \PYG{n}{data\PYGZus{}h3} \PYG{o}{=} \PYG{n}{pd}\PYG{o}{.}\PYG{n}{concat}\PYG{p}{(}\PYG{p}{[}\PYG{n}{data}\PYG{p}{,} \PYG{n}{data\PYGZus{}h3}\PYG{p}{]}\PYG{p}{,} \PYG{n}{axis}\PYG{o}{=}\PYG{l+m+mi}{0}\PYG{p}{)}

\PYG{n}{data\PYGZus{}h3}\PYG{o}{.}\PYG{n}{to\PYGZus{}pickle}\PYG{p}{(}\PYG{l+s+s1}{\PYGZsq{}}\PYG{l+s+s1}{data\PYGZus{}h3.pkl}\PYG{l+s+s1}{\PYGZsq{}}\PYG{p}{)}  
\end{sphinxVerbatim}

\end{sphinxuseclass}\end{sphinxVerbatimInput}

\end{sphinxuseclass}
\sphinxAtStartPar
 윗 꼬리가 긴 양봉이 많이 발생할 수 록 수익율에 좋은 영향을 주는 것으로 분석이 되었습니다.

\begin{sphinxuseclass}{cell}\begin{sphinxVerbatimInput}

\begin{sphinxuseclass}{cell_input}
\begin{sphinxVerbatim}[commandchars=\\\{\}]
\PYG{n}{data\PYGZus{}h3} \PYG{o}{=} \PYG{n}{pd}\PYG{o}{.}\PYG{n}{read\PYGZus{}pickle}\PYG{p}{(}\PYG{l+s+s1}{\PYGZsq{}}\PYG{l+s+s1}{data\PYGZus{}h3.pkl}\PYG{l+s+s1}{\PYGZsq{}}\PYG{p}{)}
\PYG{n+nb}{print}\PYG{p}{(}\PYG{n}{data\PYGZus{}h3}\PYG{o}{.}\PYG{n}{groupby}\PYG{p}{(}\PYG{l+s+s1}{\PYGZsq{}}\PYG{l+s+s1}{num\PYGZus{}high/close}\PYG{l+s+s1}{\PYGZsq{}}\PYG{p}{)}\PYG{p}{[}\PYG{l+s+s1}{\PYGZsq{}}\PYG{l+s+s1}{max\PYGZus{}close}\PYG{l+s+s1}{\PYGZsq{}}\PYG{p}{]}\PYG{o}{.}\PYG{n}{agg}\PYG{p}{(}\PYG{p}{[}\PYG{l+s+s1}{\PYGZsq{}}\PYG{l+s+s1}{count}\PYG{l+s+s1}{\PYGZsq{}}\PYG{p}{,}\PYG{l+s+s1}{\PYGZsq{}}\PYG{l+s+s1}{mean}\PYG{l+s+s1}{\PYGZsq{}}\PYG{p}{]}\PYG{p}{)}\PYG{p}{)}
\PYG{n}{data\PYGZus{}h3}\PYG{o}{.}\PYG{n}{groupby}\PYG{p}{(}\PYG{l+s+s1}{\PYGZsq{}}\PYG{l+s+s1}{num\PYGZus{}high/close}\PYG{l+s+s1}{\PYGZsq{}}\PYG{p}{)}\PYG{p}{[}\PYG{l+s+s1}{\PYGZsq{}}\PYG{l+s+s1}{max\PYGZus{}close}\PYG{l+s+s1}{\PYGZsq{}}\PYG{p}{]}\PYG{o}{.}\PYG{n}{mean}\PYG{p}{(}\PYG{p}{)}\PYG{o}{.}\PYG{n}{plot}\PYG{p}{(}\PYG{n}{kind}\PYG{o}{=}\PYG{l+s+s1}{\PYGZsq{}}\PYG{l+s+s1}{bar}\PYG{l+s+s1}{\PYGZsq{}}\PYG{p}{,} \PYG{n}{ylim}\PYG{o}{=}\PYG{p}{(}\PYG{l+m+mf}{0.9}\PYG{p}{,}\PYG{l+m+mf}{1.2}\PYG{p}{)}\PYG{p}{)} \PYG{c+c1}{\PYGZsh{} 막대그래프로 표현}
\end{sphinxVerbatim}

\end{sphinxuseclass}\end{sphinxVerbatimInput}
\begin{sphinxVerbatimOutput}

\begin{sphinxuseclass}{cell_output}
\begin{sphinxVerbatim}[commandchars=\\\{\}]
                 count  mean
num\PYGZus{}high/close              
0.000           355754 1.031
1.000            37734 1.043
2.000             5113 1.050
3.000              824 1.072
4.000              159 1.047
5.000               10 1.168
\end{sphinxVerbatim}

\begin{sphinxVerbatim}[commandchars=\\\{\}]
\PYGZlt{}AxesSubplot:xlabel=\PYGZsq{}num\PYGZus{}high/close\PYGZsq{}\PYGZgt{}
\end{sphinxVerbatim}

\noindent\sphinxincludegraphics{{5.1.3_Hypothesis_3_5_2}.png}

\end{sphinxuseclass}\end{sphinxVerbatimOutput}

\end{sphinxuseclass}
\sphinxAtStartPar
 윗 꼬리가 긴 양봉도 궁금하지만, 장대양봉은 어떨지도 궁금합니다. 이렇게 가설을 검증하는 과정에서 새로운 가설을 테스트하기도 합니다. 장대양봉이 과거 60일 동안 몇 번 발생했는지 카운트해보고, 장대양봉의 갯 수와 수익율 사이에 상관성이 있는 지 함 보겠습니다.

\begin{sphinxuseclass}{cell}\begin{sphinxVerbatimInput}

\begin{sphinxuseclass}{cell_input}
\begin{sphinxVerbatim}[commandchars=\\\{\}]
\PYG{n}{kosdaq\PYGZus{}list} \PYG{o}{=} \PYG{n}{pd}\PYG{o}{.}\PYG{n}{read\PYGZus{}pickle}\PYG{p}{(}\PYG{l+s+s1}{\PYGZsq{}}\PYG{l+s+s1}{kosdaq\PYGZus{}list.pkl}\PYG{l+s+s1}{\PYGZsq{}}\PYG{p}{)}

\PYG{n}{data\PYGZus{}h3} \PYG{o}{=} \PYG{n}{pd}\PYG{o}{.}\PYG{n}{DataFrame}\PYG{p}{(}\PYG{p}{)}

\PYG{k}{for} \PYG{n}{code} \PYG{o+ow}{in} \PYG{n}{kosdaq\PYGZus{}list}\PYG{p}{[}\PYG{l+s+s1}{\PYGZsq{}}\PYG{l+s+s1}{code}\PYG{l+s+s1}{\PYGZsq{}}\PYG{p}{]}\PYG{p}{:}

    \PYG{n}{data} \PYG{o}{=} \PYG{n}{mdl\PYGZus{}data}\PYG{p}{[}\PYG{n}{mdl\PYGZus{}data}\PYG{p}{[}\PYG{l+s+s1}{\PYGZsq{}}\PYG{l+s+s1}{code}\PYG{l+s+s1}{\PYGZsq{}}\PYG{p}{]}\PYG{o}{==}\PYG{n}{code}\PYG{p}{]}\PYG{o}{.}\PYG{n}{sort\PYGZus{}index}\PYG{p}{(}\PYG{p}{)}\PYG{o}{.}\PYG{n}{copy}\PYG{p}{(}\PYG{p}{)}
    
    \PYG{n}{data}\PYG{p}{[}\PYG{l+s+s1}{\PYGZsq{}}\PYG{l+s+s1}{positive\PYGZus{}candle}\PYG{l+s+s1}{\PYGZsq{}}\PYG{p}{]} \PYG{o}{=} \PYG{p}{(}\PYG{n}{data}\PYG{p}{[}\PYG{l+s+s1}{\PYGZsq{}}\PYG{l+s+s1}{close}\PYG{l+s+s1}{\PYGZsq{}}\PYG{p}{]} \PYG{o}{\PYGZgt{}} \PYG{n}{data}\PYG{p}{[}\PYG{l+s+s1}{\PYGZsq{}}\PYG{l+s+s1}{open}\PYG{l+s+s1}{\PYGZsq{}}\PYG{p}{]}\PYG{p}{)}\PYG{o}{.}\PYG{n}{astype}\PYG{p}{(}\PYG{n+nb}{int}\PYG{p}{)} \PYG{c+c1}{\PYGZsh{} 양봉}
    \PYG{n}{data}\PYG{p}{[}\PYG{l+s+s1}{\PYGZsq{}}\PYG{l+s+s1}{long\PYGZus{}candle}\PYG{l+s+s1}{\PYGZsq{}}\PYG{p}{]} \PYG{o}{=} \PYG{p}{(}\PYG{n}{data}\PYG{p}{[}\PYG{l+s+s1}{\PYGZsq{}}\PYG{l+s+s1}{positive\PYGZus{}candle}\PYG{l+s+s1}{\PYGZsq{}}\PYG{p}{]}\PYG{o}{==}\PYG{l+m+mi}{1}\PYG{p}{)}\PYG{o}{*}\PYG{p}{(}\PYG{n}{data}\PYG{p}{[}\PYG{l+s+s1}{\PYGZsq{}}\PYG{l+s+s1}{high}\PYG{l+s+s1}{\PYGZsq{}}\PYG{p}{]}\PYG{o}{==}\PYG{n}{data}\PYG{p}{[}\PYG{l+s+s1}{\PYGZsq{}}\PYG{l+s+s1}{close}\PYG{l+s+s1}{\PYGZsq{}}\PYG{p}{]}\PYG{p}{)}\PYG{o}{*}\PYGZbs{}
    \PYG{p}{(}\PYG{n}{data}\PYG{p}{[}\PYG{l+s+s1}{\PYGZsq{}}\PYG{l+s+s1}{low}\PYG{l+s+s1}{\PYGZsq{}}\PYG{p}{]}\PYG{o}{==}\PYG{n}{data}\PYG{p}{[}\PYG{l+s+s1}{\PYGZsq{}}\PYG{l+s+s1}{open}\PYG{l+s+s1}{\PYGZsq{}}\PYG{p}{]}\PYG{p}{)}\PYG{o}{*}\PYG{p}{(}\PYG{n}{data}\PYG{p}{[}\PYG{l+s+s1}{\PYGZsq{}}\PYG{l+s+s1}{close}\PYG{l+s+s1}{\PYGZsq{}}\PYG{p}{]}\PYG{o}{/}\PYG{n}{data}\PYG{p}{[}\PYG{l+s+s1}{\PYGZsq{}}\PYG{l+s+s1}{open}\PYG{l+s+s1}{\PYGZsq{}}\PYG{p}{]} \PYG{o}{\PYGZgt{}} \PYG{l+m+mf}{1.2}\PYG{p}{)}\PYG{o}{.}\PYG{n}{astype}\PYG{p}{(}\PYG{n+nb}{int}\PYG{p}{)} \PYG{c+c1}{\PYGZsh{} 장대 양봉을 데이터로 표현}
    
    \PYG{n}{data}\PYG{p}{[}\PYG{l+s+s1}{\PYGZsq{}}\PYG{l+s+s1}{num\PYGZus{}long}\PYG{l+s+s1}{\PYGZsq{}}\PYG{p}{]} \PYG{o}{=}  \PYG{n}{data}\PYG{p}{[}\PYG{l+s+s1}{\PYGZsq{}}\PYG{l+s+s1}{long\PYGZus{}candle}\PYG{l+s+s1}{\PYGZsq{}}\PYG{p}{]}\PYG{o}{.}\PYG{n}{rolling}\PYG{p}{(}\PYG{l+m+mi}{60}\PYG{p}{)}\PYG{o}{.}\PYG{n}{sum}\PYG{p}{(}\PYG{p}{)} \PYG{c+c1}{\PYGZsh{} 지난 20 일 동안 장대양봉의 갯 수}
       
    \PYG{n}{data}\PYG{p}{[}\PYG{l+s+s1}{\PYGZsq{}}\PYG{l+s+s1}{max\PYGZus{}close}\PYG{l+s+s1}{\PYGZsq{}}\PYG{p}{]}  \PYG{o}{=} \PYG{n}{data}\PYG{p}{[}\PYG{p}{[}\PYG{l+s+s1}{\PYGZsq{}}\PYG{l+s+s1}{close\PYGZus{}r1}\PYG{l+s+s1}{\PYGZsq{}}\PYG{p}{,}\PYG{l+s+s1}{\PYGZsq{}}\PYG{l+s+s1}{close\PYGZus{}r2}\PYG{l+s+s1}{\PYGZsq{}}\PYG{p}{,}\PYG{l+s+s1}{\PYGZsq{}}\PYG{l+s+s1}{close\PYGZus{}r3}\PYG{l+s+s1}{\PYGZsq{}}\PYG{p}{,}\PYG{l+s+s1}{\PYGZsq{}}\PYG{l+s+s1}{close\PYGZus{}r4}\PYG{l+s+s1}{\PYGZsq{}}\PYG{p}{,}\PYG{l+s+s1}{\PYGZsq{}}\PYG{l+s+s1}{close\PYGZus{}r5}\PYG{l+s+s1}{\PYGZsq{}}\PYG{p}{]}\PYG{p}{]}\PYG{o}{.}\PYG{n}{max}\PYG{p}{(}\PYG{n}{axis}\PYG{o}{=}\PYG{l+m+mi}{1}\PYG{p}{)} \PYG{c+c1}{\PYGZsh{} 5 영업일 종가 수익율 중 최고 값}
    \PYG{n}{data}\PYG{o}{.}\PYG{n}{dropna}\PYG{p}{(}\PYG{n}{subset}\PYG{o}{=}\PYG{p}{[}\PYG{l+s+s1}{\PYGZsq{}}\PYG{l+s+s1}{num\PYGZus{}long}\PYG{l+s+s1}{\PYGZsq{}}\PYG{p}{,}\PYG{l+s+s1}{\PYGZsq{}}\PYG{l+s+s1}{close\PYGZus{}r1}\PYG{l+s+s1}{\PYGZsq{}}\PYG{p}{,}\PYG{l+s+s1}{\PYGZsq{}}\PYG{l+s+s1}{close\PYGZus{}r2}\PYG{l+s+s1}{\PYGZsq{}}\PYG{p}{,}\PYG{l+s+s1}{\PYGZsq{}}\PYG{l+s+s1}{close\PYGZus{}r3}\PYG{l+s+s1}{\PYGZsq{}}\PYG{p}{,}\PYG{l+s+s1}{\PYGZsq{}}\PYG{l+s+s1}{close\PYGZus{}r4}\PYG{l+s+s1}{\PYGZsq{}}\PYG{p}{,}\PYG{l+s+s1}{\PYGZsq{}}\PYG{l+s+s1}{close\PYGZus{}r5}\PYG{l+s+s1}{\PYGZsq{}}\PYG{p}{]}\PYG{p}{,} \PYG{n}{inplace}\PYG{o}{=}\PYG{k+kc}{True}\PYG{p}{)} \PYG{c+c1}{\PYGZsh{} missing 이 있는 행은 제거  }
    
    \PYG{n}{data\PYGZus{}h3} \PYG{o}{=} \PYG{n}{pd}\PYG{o}{.}\PYG{n}{concat}\PYG{p}{(}\PYG{p}{[}\PYG{n}{data}\PYG{p}{,} \PYG{n}{data\PYGZus{}h3}\PYG{p}{]}\PYG{p}{,} \PYG{n}{axis}\PYG{o}{=}\PYG{l+m+mi}{0}\PYG{p}{)}

\PYG{n}{data\PYGZus{}h3}\PYG{o}{.}\PYG{n}{to\PYGZus{}pickle}\PYG{p}{(}\PYG{l+s+s1}{\PYGZsq{}}\PYG{l+s+s1}{data\PYGZus{}h3.pkl}\PYG{l+s+s1}{\PYGZsq{}}\PYG{p}{)}  
\end{sphinxVerbatim}

\end{sphinxuseclass}\end{sphinxVerbatimInput}

\end{sphinxuseclass}
\sphinxAtStartPar
 과거 60일 동안 장대양봉이 2 번 발생한 경우 좋은 수익율을 보여주고 있습니다.

\begin{sphinxuseclass}{cell}\begin{sphinxVerbatimInput}

\begin{sphinxuseclass}{cell_input}
\begin{sphinxVerbatim}[commandchars=\\\{\}]
\PYG{n}{data\PYGZus{}h3} \PYG{o}{=} \PYG{n}{pd}\PYG{o}{.}\PYG{n}{read\PYGZus{}pickle}\PYG{p}{(}\PYG{l+s+s1}{\PYGZsq{}}\PYG{l+s+s1}{data\PYGZus{}h3.pkl}\PYG{l+s+s1}{\PYGZsq{}}\PYG{p}{)}
\PYG{n+nb}{print}\PYG{p}{(}\PYG{n}{data\PYGZus{}h3}\PYG{o}{.}\PYG{n}{groupby}\PYG{p}{(}\PYG{l+s+s1}{\PYGZsq{}}\PYG{l+s+s1}{num\PYGZus{}long}\PYG{l+s+s1}{\PYGZsq{}}\PYG{p}{)}\PYG{p}{[}\PYG{l+s+s1}{\PYGZsq{}}\PYG{l+s+s1}{max\PYGZus{}close}\PYG{l+s+s1}{\PYGZsq{}}\PYG{p}{]}\PYG{o}{.}\PYG{n}{agg}\PYG{p}{(}\PYG{p}{[}\PYG{l+s+s1}{\PYGZsq{}}\PYG{l+s+s1}{count}\PYG{l+s+s1}{\PYGZsq{}}\PYG{p}{,}\PYG{l+s+s1}{\PYGZsq{}}\PYG{l+s+s1}{mean}\PYG{l+s+s1}{\PYGZsq{}}\PYG{p}{]}\PYG{p}{)}\PYG{p}{)}
\PYG{n}{data\PYGZus{}h3}\PYG{o}{.}\PYG{n}{groupby}\PYG{p}{(}\PYG{l+s+s1}{\PYGZsq{}}\PYG{l+s+s1}{num\PYGZus{}long}\PYG{l+s+s1}{\PYGZsq{}}\PYG{p}{)}\PYG{p}{[}\PYG{l+s+s1}{\PYGZsq{}}\PYG{l+s+s1}{max\PYGZus{}close}\PYG{l+s+s1}{\PYGZsq{}}\PYG{p}{]}\PYG{o}{.}\PYG{n}{mean}\PYG{p}{(}\PYG{p}{)}\PYG{o}{.}\PYG{n}{plot}\PYG{p}{(}\PYG{n}{kind}\PYG{o}{=}\PYG{l+s+s1}{\PYGZsq{}}\PYG{l+s+s1}{bar}\PYG{l+s+s1}{\PYGZsq{}}\PYG{p}{,} \PYG{n}{ylim}\PYG{o}{=}\PYG{p}{(}\PYG{l+m+mf}{0.9}\PYG{p}{,}\PYG{l+m+mf}{1.1}\PYG{p}{)}\PYG{p}{)}
\end{sphinxVerbatim}

\end{sphinxuseclass}\end{sphinxVerbatimInput}
\begin{sphinxVerbatimOutput}

\begin{sphinxuseclass}{cell_output}
\begin{sphinxVerbatim}[commandchars=\\\{\}]
           count  mean
num\PYGZus{}long              
0.000     337432 1.031
1.000       5394 1.047
2.000         88 1.056
\end{sphinxVerbatim}

\begin{sphinxVerbatim}[commandchars=\\\{\}]
\PYGZlt{}AxesSubplot:xlabel=\PYGZsq{}num\PYGZus{}long\PYGZsq{}\PYGZgt{}
\end{sphinxVerbatim}

\noindent\sphinxincludegraphics{{5.1.3_Hypothesis_3_9_2}.png}

\end{sphinxuseclass}\end{sphinxVerbatimOutput}

\end{sphinxuseclass}






\renewcommand{\indexname}{Index}
\printindex
\end{document}