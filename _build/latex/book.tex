%% Generated by Sphinx.
\def\sphinxdocclass{jupyterBook}
\documentclass[letterpaper,10pt,english]{jupyterBook}
\ifdefined\pdfpxdimen
   \let\sphinxpxdimen\pdfpxdimen\else\newdimen\sphinxpxdimen
\fi \sphinxpxdimen=.75bp\relax
\ifdefined\pdfimageresolution
    \pdfimageresolution= \numexpr \dimexpr1in\relax/\sphinxpxdimen\relax
\fi
%% let collapsible pdf bookmarks panel have high depth per default
\PassOptionsToPackage{bookmarksdepth=5}{hyperref}
%% turn off hyperref patch of \index as sphinx.xdy xindy module takes care of
%% suitable \hyperpage mark-up, working around hyperref-xindy incompatibility
\PassOptionsToPackage{hyperindex=false}{hyperref}
%% memoir class requires extra handling
\makeatletter\@ifclassloaded{memoir}
{\ifdefined\memhyperindexfalse\memhyperindexfalse\fi}{}\makeatother

\PassOptionsToPackage{warn}{textcomp}

\catcode`^^^^00a0\active\protected\def^^^^00a0{\leavevmode\nobreak\ }
\usepackage{cmap}
\usepackage{fontspec}
\defaultfontfeatures[\rmfamily,\sffamily,\ttfamily]{}
\usepackage{amsmath,amssymb,amstext}
\usepackage{polyglossia}
\setmainlanguage{english}



\setmainfont{FreeSerif}[
  Extension      = .otf,
  UprightFont    = *,
  ItalicFont     = *Italic,
  BoldFont       = *Bold,
  BoldItalicFont = *BoldItalic
]
\setsansfont{FreeSans}[
  Extension      = .otf,
  UprightFont    = *,
  ItalicFont     = *Oblique,
  BoldFont       = *Bold,
  BoldItalicFont = *BoldOblique,
]
\setmonofont{FreeMono}[
  Extension      = .otf,
  UprightFont    = *,
  ItalicFont     = *Oblique,
  BoldFont       = *Bold,
  BoldItalicFont = *BoldOblique,
]



\usepackage[Bjarne]{fncychap}
\usepackage[,numfigreset=1,mathnumfig]{sphinx}

\fvset{fontsize=\small}
\usepackage{geometry}


% Include hyperref last.
\usepackage{hyperref}
% Fix anchor placement for figures with captions.
\usepackage{hypcap}% it must be loaded after hyperref.
% Set up styles of URL: it should be placed after hyperref.
\urlstyle{same}

\addto\captionsenglish{\renewcommand{\contentsname}{chapter 1}}

\usepackage{sphinxmessages}



        % Start of preamble defined in sphinx-jupyterbook-latex %
         \usepackage[Latin,Greek]{ucharclasses}
        \usepackage{unicode-math}
        % fixing title of the toc
        \addto\captionsenglish{\renewcommand{\contentsname}{Contents}}
        \hypersetup{
            pdfencoding=auto,
            psdextra
        }
        % End of preamble defined in sphinx-jupyterbook-latex %
        

\title{나만의 추천 시스템}
\date{Jul 02, 2022}
\release{}
\author{KHS}
\newcommand{\sphinxlogo}{\vbox{}}
\renewcommand{\releasename}{}
\makeindex
\begin{document}

\pagestyle{empty}
\sphinxmaketitle
\pagestyle{plain}
\sphinxtableofcontents
\pagestyle{normal}
\phantomsection\label{\detokenize{intro::doc}}


\begin{DUlineblock}{0em}
\item[] \sphinxstylestrong{\Large (주식데이터로 데이터분석 배우기)}
\end{DUlineblock}


\part{chapter 1}


\chapter{\sphinxstylestrong{책을 시작하면서}}
\label{\detokenize{chapter1/1.1.0_SOP:id1}}\label{\detokenize{chapter1/1.1.0_SOP::doc}}



\chapter{데이터분석 환경의 변화}
\label{\detokenize{chapter1/1.2.0_Data_Science:id1}}\label{\detokenize{chapter1/1.2.0_Data_Science::doc}}


\sphinxAtStartPar
2000년 초반에는 CRM(Customer Relationship Management) 이  업계의 화두였습니다. 특히 데이터베이스마케팅을 할 수 있는 소프트웨어가 인기가 좋았습니다. 오라클(Oracle), 시불(Siebel) 등은 CRM 소프트웨어와 컨설팅을 경쟁적으로 시장에 팔았습니다. CRM 은 “신규고객 1 명 획득에 따른 비용 대비 수익보다, 기존 고객을 유지하는 비용 대비 수익이 훨씬 좋다”라는 기본 철학을 바탕으로 합니다. 하지만 고객 분석과 전략보다는 벤더의 소프트웨어 판매에 치중이 되다보니, 효과를 입증하기 어려웠고 무용론이 대두되었습니다.

\sphinxAtStartPar
또한 데이터분석은 개인 신용평가에도 활용이 많이 되었습니다. 과거 IMF (1997년) 이전에는 대출 신용한도나 신용카드 한도를 대출 담당직원이 심도있는 데이터 분석 없이 결정했습니다. 데이터에 근거하지 않고, 과거 경험으로 결정했습니다. IMF 이후, 많은 신용불량자가 생성되자 은행에서는 연체관리의 중요성을 인식하기 시작했습니다. 데이터에 근거한 통계적인 접근을 시작했습니다. 예를 들어, “이 고객은 통계적으로 연체할 확률이 70\% 이므로, 예상되는 손실이 오백만원이 될때까지 신용한도를 낮춰야한다. 이런 식의 접근입니다”

\sphinxAtStartPar
데이터 분석은 회사에서 중요한 업무를 담당했지만, 대우를 잘 받지는 못 했습니다. 똑똑한 신입이나 대리급에서 하는 일이라고 생각하는 임원이 대부분이였습니다. 따라서 지원부서의 역할이 강했고 승진과 성과급은 영업부서의 독차지였죠.

\sphinxAtStartPar
하지만, 시대가 변했습니다. 인터넷/모바일의 발달로 온라인 시장이 급격히 커졌습니다. 온라인은 영업부서의 역할을 축소시켰습니다. 은행은 점포보다는 모바일 채널을 통하여 고객에게 더 저렴하고 다양한 서비스를 제공할 수 있게 되었습니다.  딥러닝 기술의 발달로 사람의 판단이 필요없는 단순한 업무는 자동화가 가능해졌습니다. 즉 분석된 결과가 비용절감과 영업의 결과로 곧바로 연결이 되는 시대가 되어 가고 있습니다. 구글은 유투브 광고를 통하여 많은 수익을 올리고 있습니다. 데이터 분석(추천 알고리즘) 이 영업결과가 되는 대표적인 케이스라고 말씀드릴 수 있습니다.

\sphinxAtStartPar
제조업도 센서기술과 빅데이터 저장 기술의 발달로 데이터분석의 혜택을 보기 시작했습니다. 전체 프로세스에서 발생되는 데이터를 연결하여 불량 원인분석, 비용절감 및 생산증대에 데이터를 활용하고 있습니다. 특히, 데이터를 활용하여 최적으로 기계를 제어하는 부분까지 발전하고 있습니다.

\sphinxAtStartPar
분석 툴과 용어도 많은 변화가 있습니다. 과거에는 SAS 라는 통계분석툴을 주로 사용했습니다. 개인적으로도 SAS 를 이용하여 20년이상 일을 했습니다. 아침에 출근하면 왼쪽 모니터에는 엑셀, 오른쪽 모니터에는 SAS 가 있었습니다. 개인적으로 애착이 깊은 소프트웨어입니다. 요즘에는 파이썬이 대세입니다. 두가지 툴 사이에 큰 차이점은 SAS 는 분석에 특화되어 있지만, 파이썬은 분석뿐 아니라 자동화와 배포(예를 들면 웹서비스)까지 모든 서비스를 가능하게 합니다. 따라서 분석의 결과를 구현해서 성과로 보여주고 싶다면 파이썬이 좋은 툴입니다.

\sphinxAtStartPar
데이터분석이라고 부르던 업무 영역들이 확장되어 이제는 데이터사이언스라고 불리고 있습니다. 일부는 데이터분석가과 데이터사이언티스트를 구분하고 업무영역을 다르게 보는 시각이 있으나, 제 경험으로는 차이가 크게 없습니다. 요즘 데이터사이언티스트를 하시는 분들은 컴퓨터 혹은 소프트웨어 전공자 분들이 많으셔서 소프트웨어 개발도 잘 하십니다. 분석결과가 실행되어 결과가 되는 속도가 더욱 가속되고 있습니다.


\part{chapter 2}


\chapter{\sphinxstylestrong{파이썬 데이터분석 기초}}
\label{\detokenize{chapter2/2.1.0_Python_Basics:id1}}\label{\detokenize{chapter2/2.1.0_Python_Basics::doc}}
\sphinxAtStartPar
이번 장에서는 데이터 분석에 꼭 필요한 부분만 다룰 수 있도록 하겠습니다. 파이썬 기초 학습은 유튜브에서 좋은 강의를 쉽게 찾으실 수 있습니다. 이 장에서는 주가데이터 분석에 필요한 기술만 우선적으로 익히고, 배운 기술을 이용하여 래리 윌리암스의 변동성 돌파전략을 구현해 보겠습니다. 알고리즘 개발을 위한 파이썬 코드는 쥬피터노트북 환경에서 작성하고 실행을 하였습니다. 주피터노트북은 대화형으로 코딩을 할 수 있는 파이썬 에디터이기 때문에 데이터분석가 가장 선호하는 에디터입니다. 아나콘다를 설치하면 데이터분석을 위한 여러 패키지와 함께 주피터노트북이 설치됩니다. 아나콘다 설치 관련하여 다양한 유투브 동영상과 온라인 자료가 있습니다. 본서에는 7장(자동매매)에서 아나콘다 설치를 위한 가이드를 제공합니다. 아나콘다를 설치 못하신 독자는 아나콘다 설치 완료하시고 진행하시면 되겠습니다.


\section{Data Type}
\label{\detokenize{chapter2/2.1.1_Python_Basics:data-type}}\label{\detokenize{chapter2/2.1.1_Python_Basics::doc}}
\sphinxAtStartPar
먼저 데이터 타입에 대한 이해가 필요합니다. 주식 데이터 분석에서 활용할 데이터 타입은 숫자(Number),  문자열(String),  날짜(Date), 딕셔너리(Dictionary), 리스트(List), 시리즈(Series), 데이터프레임(DataFrame) 등이 있습니다. 각 타입의 형식은 아래와 같습니다.

\begin{sphinxuseclass}{cell}\begin{sphinxVerbatimInput}

\begin{sphinxuseclass}{cell_input}
\begin{sphinxVerbatim}[commandchars=\\\{\}]
\PYG{c+c1}{\PYGZsh{} Number}
\PYG{n}{n1} \PYG{o}{=} \PYG{l+m+mi}{123}
\PYG{n}{n2} \PYG{o}{=} \PYG{l+m+mi}{234}

\PYG{c+c1}{\PYGZsh{} String }
\PYG{n}{s1} \PYG{o}{=} \PYG{l+s+s1}{\PYGZsq{}}\PYG{l+s+s1}{string}\PYG{l+s+s1}{\PYGZsq{}}
\PYG{n}{s2} \PYG{o}{=} \PYG{l+s+s1}{\PYGZsq{}}\PYG{l+s+s1}{I am Tom}\PYG{l+s+s1}{\PYGZsq{}}

\PYG{c+c1}{\PYGZsh{} Date}
\PYG{k+kn}{import} \PYG{n+nn}{datetime}
\PYG{n}{d1} \PYG{o}{=} \PYG{n}{datetime}\PYG{o}{.}\PYG{n}{datetime}\PYG{p}{(}\PYG{l+m+mi}{2021}\PYG{p}{,} \PYG{l+m+mi}{1}\PYG{p}{,} \PYG{l+m+mi}{3}\PYG{p}{,} \PYG{l+m+mi}{0}\PYG{p}{,} \PYG{l+m+mi}{0}\PYG{p}{)}
\PYG{n}{yymmdd} \PYG{o}{=} \PYG{l+s+s1}{\PYGZsq{}}\PYG{l+s+s1}{2021\PYGZhy{}01\PYGZhy{}03}\PYG{l+s+s1}{\PYGZsq{}}
\PYG{n}{d2} \PYG{o}{=} \PYG{n}{datetime}\PYG{o}{.}\PYG{n}{datetime}\PYG{o}{.}\PYG{n}{strptime}\PYG{p}{(}\PYG{n}{yymmdd}\PYG{p}{,} \PYG{l+s+s1}{\PYGZsq{}}\PYG{l+s+s1}{\PYGZpc{}}\PYG{l+s+s1}{Y\PYGZhy{}}\PYG{l+s+s1}{\PYGZpc{}}\PYG{l+s+s1}{m\PYGZhy{}}\PYG{l+s+si}{\PYGZpc{}d}\PYG{l+s+s1}{\PYGZsq{}}\PYG{p}{)}

\PYG{c+c1}{\PYGZsh{} Dictionary}
\PYG{n}{dic1} \PYG{o}{=} \PYG{p}{\PYGZob{}}\PYG{l+s+s1}{\PYGZsq{}}\PYG{l+s+s1}{a}\PYG{l+s+s1}{\PYGZsq{}}\PYG{p}{:}\PYG{l+m+mi}{11}\PYG{p}{,} \PYG{l+s+s1}{\PYGZsq{}}\PYG{l+s+s1}{b}\PYG{l+s+s1}{\PYGZsq{}}\PYG{p}{:}\PYG{l+m+mi}{12}\PYG{p}{,} \PYG{l+s+s1}{\PYGZsq{}}\PYG{l+s+s1}{c}\PYG{l+s+s1}{\PYGZsq{}}\PYG{p}{:}\PYG{l+m+mi}{13}\PYG{p}{\PYGZcb{}}

\PYG{c+c1}{\PYGZsh{} List}
\PYG{n}{l1} \PYG{o}{=} \PYG{p}{[}\PYG{l+m+mi}{1}\PYG{p}{,}\PYG{l+m+mi}{2}\PYG{p}{,}\PYG{l+m+mi}{3}\PYG{p}{]}
\PYG{n}{l2} \PYG{o}{=}  \PYG{p}{[}\PYG{l+s+s1}{\PYGZsq{}}\PYG{l+s+s1}{a}\PYG{l+s+s1}{\PYGZsq{}}\PYG{p}{,}\PYG{l+s+s1}{\PYGZsq{}}\PYG{l+s+s1}{b}\PYG{l+s+s1}{\PYGZsq{}}\PYG{p}{,}\PYG{l+s+s1}{\PYGZsq{}}\PYG{l+s+s1}{c}\PYG{l+s+s1}{\PYGZsq{}}\PYG{p}{]}

\PYG{c+c1}{\PYGZsh{} Series}
\PYG{k+kn}{import} \PYG{n+nn}{pandas} \PYG{k}{as} \PYG{n+nn}{pd}
\PYG{n}{ss1} \PYG{o}{=} \PYG{p}{[}\PYG{l+m+mi}{11}\PYG{p}{,}\PYG{l+m+mi}{12}\PYG{p}{,}\PYG{l+m+mi}{13}\PYG{p}{,}\PYG{l+m+mi}{14}\PYG{p}{,}\PYG{l+m+mi}{15}\PYG{p}{]}
\PYG{n}{ss2} \PYG{o}{=} \PYG{n}{pd}\PYG{o}{.}\PYG{n}{Series}\PYG{p}{(}\PYG{n}{ss1}\PYG{p}{)}

\PYG{c+c1}{\PYGZsh{} DataFrame}
\PYG{n}{c\PYGZus{}1} \PYG{o}{=} \PYG{p}{[}\PYG{l+m+mi}{1}\PYG{p}{,}\PYG{l+m+mi}{2}\PYG{p}{,}\PYG{l+m+mi}{3}\PYG{p}{]}
\PYG{n}{c\PYGZus{}2} \PYG{o}{=}  \PYG{p}{[}\PYG{l+s+s1}{\PYGZsq{}}\PYG{l+s+s1}{a}\PYG{l+s+s1}{\PYGZsq{}}\PYG{p}{,}\PYG{l+s+s1}{\PYGZsq{}}\PYG{l+s+s1}{b}\PYG{l+s+s1}{\PYGZsq{}}\PYG{p}{,}\PYG{l+s+s1}{\PYGZsq{}}\PYG{l+s+s1}{c}\PYG{l+s+s1}{\PYGZsq{}}\PYG{p}{]}
\PYG{n}{df1} \PYG{o}{=} \PYG{n}{pd}\PYG{o}{.}\PYG{n}{DataFrame}\PYG{p}{(}\PYG{p}{\PYGZob{}}\PYG{l+s+s1}{\PYGZsq{}}\PYG{l+s+s1}{col1}\PYG{l+s+s1}{\PYGZsq{}}\PYG{p}{:} \PYG{n}{c\PYGZus{}1}\PYG{p}{,} \PYG{l+s+s1}{\PYGZsq{}}\PYG{l+s+s1}{col2}\PYG{l+s+s1}{\PYGZsq{}}\PYG{p}{:} \PYG{n}{c\PYGZus{}2}\PYG{p}{\PYGZcb{}}\PYG{p}{)}
\end{sphinxVerbatim}

\end{sphinxuseclass}\end{sphinxVerbatimInput}

\end{sphinxuseclass}

\section{String}
\label{\detokenize{chapter2/2.1.1_Python_Basics:string}}
\sphinxAtStartPar
문자열의 첫 글자부터 0, 1, 2 번째 문자 해당합니다. 예를 들어, ‘String’ 이란 문자열을 s1 이란 변수에서 저장한 경우, s1{[}0{]} 은 ‘s’ 에 해당하고, s1{[}1{]} 은 ‘t’ 에 해당합니다.

\begin{sphinxuseclass}{cell}\begin{sphinxVerbatimInput}

\begin{sphinxuseclass}{cell_input}
\begin{sphinxVerbatim}[commandchars=\\\{\}]
\PYG{c+c1}{\PYGZsh{} String }
\PYG{n}{s1} \PYG{o}{=} \PYG{l+s+s1}{\PYGZsq{}}\PYG{l+s+s1}{string}\PYG{l+s+s1}{\PYGZsq{}}
\PYG{n}{s2} \PYG{o}{=} \PYG{l+s+s1}{\PYGZsq{}}\PYG{l+s+s1}{I am Tom}\PYG{l+s+s1}{\PYGZsq{}}

\PYG{n+nb}{print}\PYG{p}{(}\PYG{n}{s1}\PYG{p}{)}
\PYG{n+nb}{print}\PYG{p}{(}\PYG{n}{s2}\PYG{p}{[}\PYG{l+m+mi}{0}\PYG{p}{]}\PYG{p}{,} \PYG{n}{s2}\PYG{p}{[}\PYG{l+m+mi}{5}\PYG{p}{]}\PYG{p}{)} \PYG{c+c1}{\PYGZsh{} I am Tom 의 첫번재[0], 6번째[5] 글자 반환. 대부분의 컴푸터 언어는 0 부터 시작.}
\end{sphinxVerbatim}

\end{sphinxuseclass}\end{sphinxVerbatimInput}
\begin{sphinxVerbatimOutput}

\begin{sphinxuseclass}{cell_output}
\begin{sphinxVerbatim}[commandchars=\\\{\}]
string
I T
\end{sphinxVerbatim}

\end{sphinxuseclass}\end{sphinxVerbatimOutput}

\end{sphinxuseclass}

\section{List}
\label{\detokenize{chapter2/2.1.1_Python_Basics:list}}
\sphinxAtStartPar
리스트는 여러 개의 원소를 대괄호 {[}{]} 에 넣은 형태입니다. 리스트도 문자열과 동일하게 0 부터 시작합니다.
아래 l1 에서 0 번째 원소는 1 이고, l2 에서는 ‘a’ 입니다. 그리고 {[}start\_point:end\_point{]} 를 형식으로 원소의 일부만 가져올 수 있습니다. 단, {[}start\_point:end\_point{]} 에서 원소는 end\_point 전까지 가져오는 사실만 유의하시면 됩니다.

\begin{sphinxuseclass}{cell}\begin{sphinxVerbatimInput}

\begin{sphinxuseclass}{cell_input}
\begin{sphinxVerbatim}[commandchars=\\\{\}]
\PYG{c+c1}{\PYGZsh{} List}
\PYG{n}{l1} \PYG{o}{=} \PYG{p}{[}\PYG{l+m+mi}{1}\PYG{p}{,}\PYG{l+m+mi}{2}\PYG{p}{,}\PYG{l+m+mi}{3}\PYG{p}{]}
\PYG{n}{l2} \PYG{o}{=}  \PYG{p}{[}\PYG{l+s+s1}{\PYGZsq{}}\PYG{l+s+s1}{a}\PYG{l+s+s1}{\PYGZsq{}}\PYG{p}{,}\PYG{l+s+s1}{\PYGZsq{}}\PYG{l+s+s1}{b}\PYG{l+s+s1}{\PYGZsq{}}\PYG{p}{,}\PYG{l+s+s1}{\PYGZsq{}}\PYG{l+s+s1}{c}\PYG{l+s+s1}{\PYGZsq{}}\PYG{p}{,}\PYG{l+s+s1}{\PYGZsq{}}\PYG{l+s+s1}{d}\PYG{l+s+s1}{\PYGZsq{}}\PYG{p}{]}

\PYG{n+nb}{print}\PYG{p}{(}\PYG{n}{l1}\PYG{p}{[}\PYG{l+m+mi}{0}\PYG{p}{]}\PYG{p}{)}
\PYG{n+nb}{print}\PYG{p}{(}\PYG{n}{l2}\PYG{p}{[}\PYG{l+m+mi}{0}\PYG{p}{]}\PYG{p}{)}
\PYG{n+nb}{print}\PYG{p}{(}\PYG{n}{l2}\PYG{p}{[}\PYG{p}{:}\PYG{l+m+mi}{2}\PYG{p}{]}\PYG{p}{)} \PYG{c+c1}{\PYGZsh{} 0 \PYGZti{} 1 번째 문자를 가져옴. 2 번째 포함되지 않음.}
\PYG{n+nb}{print}\PYG{p}{(}\PYG{n}{l2}\PYG{p}{[}\PYG{l+m+mi}{1}\PYG{p}{:}\PYG{l+m+mi}{3}\PYG{p}{]}\PYG{p}{)} \PYG{c+c1}{\PYGZsh{} 1 \PYGZti{} 2 번째 문자를 가져옴. 3 번째는 포함되지 않음}
\end{sphinxVerbatim}

\end{sphinxuseclass}\end{sphinxVerbatimInput}
\begin{sphinxVerbatimOutput}

\begin{sphinxuseclass}{cell_output}
\begin{sphinxVerbatim}[commandchars=\\\{\}]
1
a
[\PYGZsq{}a\PYGZsq{}, \PYGZsq{}b\PYGZsq{}]
[\PYGZsq{}b\PYGZsq{}, \PYGZsq{}c\PYGZsq{}]
\end{sphinxVerbatim}

\end{sphinxuseclass}\end{sphinxVerbatimOutput}

\end{sphinxuseclass}

\section{Number}
\label{\detokenize{chapter2/2.1.1_Python_Basics:number}}
\sphinxAtStartPar
숫자형은 정수형, 소숫점형으로 나눌 수 있으나, 아래와 같이 곧바로 사칙연산이 가능합니다.

\begin{sphinxuseclass}{cell}\begin{sphinxVerbatimInput}

\begin{sphinxuseclass}{cell_input}
\begin{sphinxVerbatim}[commandchars=\\\{\}]
\PYG{c+c1}{\PYGZsh{} Number}
\PYG{n}{n1} \PYG{o}{=} \PYG{l+m+mi}{123}
\PYG{n}{n2} \PYG{o}{=} \PYG{l+m+mi}{234}
\PYG{n+nb}{print}\PYG{p}{(}\PYG{n}{n1}\PYG{o}{*}\PYG{n}{n2}\PYG{p}{)}
\end{sphinxVerbatim}

\end{sphinxuseclass}\end{sphinxVerbatimInput}
\begin{sphinxVerbatimOutput}

\begin{sphinxuseclass}{cell_output}
\begin{sphinxVerbatim}[commandchars=\\\{\}]
28782
\end{sphinxVerbatim}

\end{sphinxuseclass}\end{sphinxVerbatimOutput}

\end{sphinxuseclass}

\section{Date}
\label{\detokenize{chapter2/2.1.1_Python_Basics:date}}
\sphinxAtStartPar
날짜형 데이터는 주식데이터 분석에 중요한 데이터형식입니다. 활용도가 아주 높습니다. 날짜 데이터를 활용하기 위해서는 먼저 datetime 패키지를 import 합니다. datetime 는 날짜 데이터를 다루기 위한 여러 메소드를 가지고 있습니다. 첫 번째 d1 변수에서 datetime.datetime 함수는 년, 월, 시 등의 숫자를 파이썬 날짜로 변형할 수 있게 해 줍니다. d2 변수는 문자열로 되어 있는 날짜를 파이썬 날짜로 변형하는 한 후 저장을 합니다. strptime 은 문자열을 파이썬 날짜로 변경해주는 메소드입니다. strptime 은 두 개의 인수가 필요한데요. 이 함수의 첫 번째 인자는 날짜 형태의 문자열, 두번 째 인자는 형식 포맷입니다. 문자열 날짜 포맷이 \%Y\sphinxhyphen{}\%m\sphinxhyphen{}\%d 형태라는 것을 인수로 알려줍니다. d3 는 다른 날짜 형식으로 되어 있는 문자열을 파이썬 날짜로 변경하는 법을 보여줍니다. d4 는 반대로 파이썬 날짜를 다시 문자열로 변경하는 법을 보여주고 있습니다. 이 때 함수는 strpftime 입니다. 마지막으로 timedelta 를 알아보겠습니다. timedelta 는 일정한 시간을 뒤로 이동한 결과를 반환합니다. 예를 들어 d1 에 hours=5 를 추가하면 2021년 1월 3일 0 시에서 2021년 1월 3일 5시로 변경됩니다. 아래 예시는 d1 에서 5 시간을 더 했을 때, 2 일 더했을 때 결과를 보여줍니다.

\begin{sphinxuseclass}{cell}\begin{sphinxVerbatimInput}

\begin{sphinxuseclass}{cell_input}
\begin{sphinxVerbatim}[commandchars=\\\{\}]
\PYG{c+c1}{\PYGZsh{} Date}
\PYG{k+kn}{import} \PYG{n+nn}{datetime}

\PYG{n}{d1} \PYG{o}{=} \PYG{n}{datetime}\PYG{o}{.}\PYG{n}{datetime}\PYG{p}{(}\PYG{l+m+mi}{2021}\PYG{p}{,} \PYG{l+m+mi}{1}\PYG{p}{,} \PYG{l+m+mi}{3}\PYG{p}{,} \PYG{l+m+mi}{0}\PYG{p}{,} \PYG{l+m+mi}{0}\PYG{p}{)}

\PYG{n}{yymmdd} \PYG{o}{=} \PYG{l+s+s1}{\PYGZsq{}}\PYG{l+s+s1}{2021\PYGZhy{}01\PYGZhy{}03}\PYG{l+s+s1}{\PYGZsq{}}
\PYG{n}{d2} \PYG{o}{=} \PYG{n}{datetime}\PYG{o}{.}\PYG{n}{datetime}\PYG{o}{.}\PYG{n}{strptime}\PYG{p}{(}\PYG{n}{yymmdd}\PYG{p}{,} \PYG{l+s+s1}{\PYGZsq{}}\PYG{l+s+s1}{\PYGZpc{}}\PYG{l+s+s1}{Y\PYGZhy{}}\PYG{l+s+s1}{\PYGZpc{}}\PYG{l+s+s1}{m\PYGZhy{}}\PYG{l+s+si}{\PYGZpc{}d}\PYG{l+s+s1}{\PYGZsq{}}\PYG{p}{)}

\PYG{n}{time\PYGZus{}point} \PYG{o}{=} \PYG{l+s+s1}{\PYGZsq{}}\PYG{l+s+s1}{2021/01/03 19:15:32}\PYG{l+s+s1}{\PYGZsq{}}
\PYG{n}{d3} \PYG{o}{=}  \PYG{n}{datetime}\PYG{o}{.}\PYG{n}{datetime}\PYG{o}{.}\PYG{n}{strptime}\PYG{p}{(}\PYG{n}{time\PYGZus{}point}\PYG{p}{,} \PYG{l+s+s1}{\PYGZsq{}}\PYG{l+s+s1}{\PYGZpc{}}\PYG{l+s+s1}{Y/}\PYG{l+s+s1}{\PYGZpc{}}\PYG{l+s+s1}{m/}\PYG{l+s+si}{\PYGZpc{}d}\PYG{l+s+s1}{ }\PYG{l+s+s1}{\PYGZpc{}}\PYG{l+s+s1}{H:}\PYG{l+s+s1}{\PYGZpc{}}\PYG{l+s+s1}{M:}\PYG{l+s+s1}{\PYGZpc{}}\PYG{l+s+s1}{S}\PYG{l+s+s1}{\PYGZsq{}}\PYG{p}{)}

\PYG{n}{d4} \PYG{o}{=} \PYG{n}{datetime}\PYG{o}{.}\PYG{n}{datetime}\PYG{o}{.}\PYG{n}{strftime}\PYG{p}{(}\PYG{n}{d3}\PYG{p}{,} \PYG{l+s+s1}{\PYGZsq{}}\PYG{l+s+s1}{\PYGZpc{}}\PYG{l+s+s1}{Y\PYGZhy{}}\PYG{l+s+s1}{\PYGZpc{}}\PYG{l+s+s1}{m\PYGZhy{}}\PYG{l+s+si}{\PYGZpc{}d}\PYG{l+s+s1}{\PYGZsq{}}\PYG{p}{)}

\PYG{n+nb}{print}\PYG{p}{(}\PYG{n}{d1}\PYG{p}{)}
\PYG{n+nb}{print}\PYG{p}{(}\PYG{n}{d2}\PYG{p}{)}
\PYG{n+nb}{print}\PYG{p}{(}\PYG{n}{d3}\PYG{p}{,} \PYG{n+nb}{type}\PYG{p}{(}\PYG{n}{d3}\PYG{p}{)}\PYG{p}{)}
\PYG{n+nb}{print}\PYG{p}{(}\PYG{n}{d4}\PYG{p}{,} \PYG{n+nb}{type}\PYG{p}{(}\PYG{n}{d4}\PYG{p}{)}\PYG{p}{)}    

\PYG{n+nb}{print}\PYG{p}{(}\PYG{n}{d1} \PYG{o}{+} \PYG{n}{datetime}\PYG{o}{.}\PYG{n}{timedelta}\PYG{p}{(}\PYG{n}{hours}\PYG{o}{=}\PYG{l+m+mi}{5}\PYG{p}{)}\PYG{p}{)}
\PYG{n+nb}{print}\PYG{p}{(}\PYG{n}{d1} \PYG{o}{+} \PYG{n}{datetime}\PYG{o}{.}\PYG{n}{timedelta}\PYG{p}{(}\PYG{n}{days}\PYG{o}{=}\PYG{l+m+mi}{2}\PYG{p}{)}\PYG{p}{)}
\end{sphinxVerbatim}

\end{sphinxuseclass}\end{sphinxVerbatimInput}
\begin{sphinxVerbatimOutput}

\begin{sphinxuseclass}{cell_output}
\begin{sphinxVerbatim}[commandchars=\\\{\}]
2021\PYGZhy{}01\PYGZhy{}03 00:00:00
2021\PYGZhy{}01\PYGZhy{}03 00:00:00
2021\PYGZhy{}01\PYGZhy{}03 19:15:32 \PYGZlt{}class \PYGZsq{}datetime.datetime\PYGZsq{}\PYGZgt{}
2021\PYGZhy{}01\PYGZhy{}03 \PYGZlt{}class \PYGZsq{}str\PYGZsq{}\PYGZgt{}
2021\PYGZhy{}01\PYGZhy{}03 05:00:00
2021\PYGZhy{}01\PYGZhy{}05 00:00:00
\end{sphinxVerbatim}

\end{sphinxuseclass}\end{sphinxVerbatimOutput}

\end{sphinxuseclass}

\section{Dictionary}
\label{\detokenize{chapter2/2.1.1_Python_Basics:dictionary}}
\sphinxAtStartPar
디셔너리는 key 와 value 가 있는 짝으로 있는 형태입니다. key 를 이용하여 원하는 값을 찾을 수 있어서, 프로그램에서 참조 값을 저장해 둘 때 아주 유용합니다. value 대신에 List 나 DataFrame 형식의 데이터도 넣을 수 도 있습니다. 아래 예시에서는 dic1 에서 키값 ‘a’ 에 해당하는 값은 11 입니다. 새로운 key 와 value 를 넣어서 기존의 Dictionary 에 추가할 수 있습니다. 아래 두 번째 예시는 ‘d’라는 key 에 value 14 를 추가하는 방법입니다. Dictionary 가 너무 커서 어떤 key 값들이 있는 지 알고 싶을 때는 key() 메소드를 사용합니다.

\begin{sphinxuseclass}{cell}\begin{sphinxVerbatimInput}

\begin{sphinxuseclass}{cell_input}
\begin{sphinxVerbatim}[commandchars=\\\{\}]
\PYG{c+c1}{\PYGZsh{} Dictionary}
\PYG{n}{dic1} \PYG{o}{=} \PYG{p}{\PYGZob{}}\PYG{l+s+s1}{\PYGZsq{}}\PYG{l+s+s1}{a}\PYG{l+s+s1}{\PYGZsq{}}\PYG{p}{:}\PYG{l+m+mi}{11}\PYG{p}{,} \PYG{l+s+s1}{\PYGZsq{}}\PYG{l+s+s1}{b}\PYG{l+s+s1}{\PYGZsq{}}\PYG{p}{:}\PYG{l+m+mi}{12}\PYG{p}{,} \PYG{l+s+s1}{\PYGZsq{}}\PYG{l+s+s1}{c}\PYG{l+s+s1}{\PYGZsq{}}\PYG{p}{:}\PYG{l+m+mi}{13}\PYG{p}{\PYGZcb{}}
\PYG{n+nb}{print}\PYG{p}{(}\PYG{n}{dic1}\PYG{p}{[}\PYG{l+s+s1}{\PYGZsq{}}\PYG{l+s+s1}{a}\PYG{l+s+s1}{\PYGZsq{}}\PYG{p}{]}\PYG{p}{)}

\PYG{n}{dic1}\PYG{p}{[}\PYG{l+s+s1}{\PYGZsq{}}\PYG{l+s+s1}{d}\PYG{l+s+s1}{\PYGZsq{}}\PYG{p}{]} \PYG{o}{=} \PYG{l+m+mi}{14}
\PYG{n+nb}{print}\PYG{p}{(}\PYG{n}{dic1}\PYG{p}{)}

\PYG{n+nb}{print}\PYG{p}{(}\PYG{n}{dic1}\PYG{o}{.}\PYG{n}{keys}\PYG{p}{(}\PYG{p}{)}\PYG{p}{)}
\end{sphinxVerbatim}

\end{sphinxuseclass}\end{sphinxVerbatimInput}
\begin{sphinxVerbatimOutput}

\begin{sphinxuseclass}{cell_output}
\begin{sphinxVerbatim}[commandchars=\\\{\}]
11
\PYGZob{}\PYGZsq{}a\PYGZsq{}: 11, \PYGZsq{}b\PYGZsq{}: 12, \PYGZsq{}c\PYGZsq{}: 13, \PYGZsq{}d\PYGZsq{}: 14\PYGZcb{}
dict\PYGZus{}keys([\PYGZsq{}a\PYGZsq{}, \PYGZsq{}b\PYGZsq{}, \PYGZsq{}c\PYGZsq{}, \PYGZsq{}d\PYGZsq{}])
\end{sphinxVerbatim}

\end{sphinxuseclass}\end{sphinxVerbatimOutput}

\end{sphinxuseclass}

\section{Series}
\label{\detokenize{chapter2/2.1.1_Python_Basics:series}}
\sphinxAtStartPar
Series 라는 데이터 타입을 이용하기 위해서는 Pandas 패키지를 이용합니다. Pandas 는 테이블 형태의 데이터를 다루는데 정말 강력한 패키지입니다. 먼저 ss1 이라는 리스트를 생성해 보겠습니다. ss1 이라는 리스트에 어떤 메소드를 사용할 수 있는 지 알기 위해서는 dir() 를 이용합니다.  dir() 를 하면 Built\sphinxhyphen{}in 함수 전체를 알 수 있습니다. append 부터 sort 까지 총 11 개의 함수가 나옵니다. 이 11 개 함수가 List 에서 쓸 수 있는 메소드입니다. 이번에는 ss1 를 Series 로 변경한 후, ss2 에 저장하겠습니다. 그리고 dir() 함수로 호출해 보겠습니다. 많은 메소스가 나열됩니다. 예를 들어 List 값의 평균을 알고 싶은데, List 는 평균을 구하는 메소드가 없습니다. 하지만 Series 에는 mean() 으로 평균값을 구할 수 있습니다.

\begin{sphinxuseclass}{cell}\begin{sphinxVerbatimInput}

\begin{sphinxuseclass}{cell_input}
\begin{sphinxVerbatim}[commandchars=\\\{\}]
\PYG{k+kn}{import} \PYG{n+nn}{pandas} \PYG{k}{as} \PYG{n+nn}{pd}

\PYG{n}{ss1} \PYG{o}{=} \PYG{p}{[}\PYG{l+m+mi}{11}\PYG{p}{,}\PYG{l+m+mi}{12}\PYG{p}{,}\PYG{l+m+mi}{13}\PYG{p}{,}\PYG{l+m+mi}{14}\PYG{p}{,}\PYG{l+m+mi}{15}\PYG{p}{]}
\PYG{n}{ss2} \PYG{o}{=} \PYG{n}{pd}\PYG{o}{.}\PYG{n}{Series}\PYG{p}{(}\PYG{n}{ss1}\PYG{p}{)}

\PYG{n+nb}{print}\PYG{p}{(}\PYG{n+nb}{dir}\PYG{p}{(}\PYG{n}{ss1}\PYG{p}{)}\PYG{p}{)} \PYG{c+c1}{\PYGZsh{} \PYGZsq{}append\PYGZsq{}, \PYGZsq{}clear\PYGZsq{}, \PYGZsq{}copy\PYGZsq{}, \PYGZsq{}count\PYGZsq{}, \PYGZsq{}extend\PYGZsq{}, \PYGZsq{}index\PYGZsq{}, \PYGZsq{}insert\PYGZsq{}, \PYGZsq{}pop\PYGZsq{}, \PYGZsq{}remove\PYGZsq{}, \PYGZsq{}reverse\PYGZsq{}, \PYGZsq{}sort\PYGZsq{} 등 사용가능}
\PYG{n+nb}{print}\PYG{p}{(}\PYG{l+s+s1}{\PYGZsq{}}\PYG{l+s+se}{\PYGZbs{}n}\PYG{l+s+s1}{\PYGZsq{}}\PYG{p}{)}

\PYG{n+nb}{print}\PYG{p}{(}\PYG{n+nb}{dir}\PYG{p}{(}\PYG{n}{ss2}\PYG{p}{)}\PYG{p}{)}
\PYG{n+nb}{print}\PYG{p}{(}\PYG{l+s+s1}{\PYGZsq{}}\PYG{l+s+se}{\PYGZbs{}n}\PYG{l+s+s1}{\PYGZsq{}}\PYG{p}{)}

\PYG{l+s+sd}{\PYGZsq{}\PYGZsq{}\PYGZsq{} ss1.mean() \PYGZhy{}\PYGZhy{}\PYGZgt{} 에러발생 \PYGZsq{}\PYGZsq{}\PYGZsq{}}
\PYG{n+nb}{print}\PYG{p}{(}\PYG{n}{ss2}\PYG{o}{.}\PYG{n}{mean}\PYG{p}{(}\PYG{p}{)}\PYG{p}{)} \PYG{c+c1}{\PYGZsh{} 평균값 13 반환}
\end{sphinxVerbatim}

\end{sphinxuseclass}\end{sphinxVerbatimInput}
\begin{sphinxVerbatimOutput}

\begin{sphinxuseclass}{cell_output}
\begin{sphinxVerbatim}[commandchars=\\\{\}]
[\PYGZsq{}\PYGZus{}\PYGZus{}add\PYGZus{}\PYGZus{}\PYGZsq{}, \PYGZsq{}\PYGZus{}\PYGZus{}class\PYGZus{}\PYGZus{}\PYGZsq{}, \PYGZsq{}\PYGZus{}\PYGZus{}class\PYGZus{}getitem\PYGZus{}\PYGZus{}\PYGZsq{}, \PYGZsq{}\PYGZus{}\PYGZus{}contains\PYGZus{}\PYGZus{}\PYGZsq{}, \PYGZsq{}\PYGZus{}\PYGZus{}delattr\PYGZus{}\PYGZus{}\PYGZsq{}, \PYGZsq{}\PYGZus{}\PYGZus{}delitem\PYGZus{}\PYGZus{}\PYGZsq{}, \PYGZsq{}\PYGZus{}\PYGZus{}dir\PYGZus{}\PYGZus{}\PYGZsq{}, \PYGZsq{}\PYGZus{}\PYGZus{}doc\PYGZus{}\PYGZus{}\PYGZsq{}, \PYGZsq{}\PYGZus{}\PYGZus{}eq\PYGZus{}\PYGZus{}\PYGZsq{}, \PYGZsq{}\PYGZus{}\PYGZus{}format\PYGZus{}\PYGZus{}\PYGZsq{}, \PYGZsq{}\PYGZus{}\PYGZus{}ge\PYGZus{}\PYGZus{}\PYGZsq{}, \PYGZsq{}\PYGZus{}\PYGZus{}getattribute\PYGZus{}\PYGZus{}\PYGZsq{}, \PYGZsq{}\PYGZus{}\PYGZus{}getitem\PYGZus{}\PYGZus{}\PYGZsq{}, \PYGZsq{}\PYGZus{}\PYGZus{}gt\PYGZus{}\PYGZus{}\PYGZsq{}, \PYGZsq{}\PYGZus{}\PYGZus{}hash\PYGZus{}\PYGZus{}\PYGZsq{}, \PYGZsq{}\PYGZus{}\PYGZus{}iadd\PYGZus{}\PYGZus{}\PYGZsq{}, \PYGZsq{}\PYGZus{}\PYGZus{}imul\PYGZus{}\PYGZus{}\PYGZsq{}, \PYGZsq{}\PYGZus{}\PYGZus{}init\PYGZus{}\PYGZus{}\PYGZsq{}, \PYGZsq{}\PYGZus{}\PYGZus{}init\PYGZus{}subclass\PYGZus{}\PYGZus{}\PYGZsq{}, \PYGZsq{}\PYGZus{}\PYGZus{}iter\PYGZus{}\PYGZus{}\PYGZsq{}, \PYGZsq{}\PYGZus{}\PYGZus{}le\PYGZus{}\PYGZus{}\PYGZsq{}, \PYGZsq{}\PYGZus{}\PYGZus{}len\PYGZus{}\PYGZus{}\PYGZsq{}, \PYGZsq{}\PYGZus{}\PYGZus{}lt\PYGZus{}\PYGZus{}\PYGZsq{}, \PYGZsq{}\PYGZus{}\PYGZus{}mul\PYGZus{}\PYGZus{}\PYGZsq{}, \PYGZsq{}\PYGZus{}\PYGZus{}ne\PYGZus{}\PYGZus{}\PYGZsq{}, \PYGZsq{}\PYGZus{}\PYGZus{}new\PYGZus{}\PYGZus{}\PYGZsq{}, \PYGZsq{}\PYGZus{}\PYGZus{}reduce\PYGZus{}\PYGZus{}\PYGZsq{}, \PYGZsq{}\PYGZus{}\PYGZus{}reduce\PYGZus{}ex\PYGZus{}\PYGZus{}\PYGZsq{}, \PYGZsq{}\PYGZus{}\PYGZus{}repr\PYGZus{}\PYGZus{}\PYGZsq{}, \PYGZsq{}\PYGZus{}\PYGZus{}reversed\PYGZus{}\PYGZus{}\PYGZsq{}, \PYGZsq{}\PYGZus{}\PYGZus{}rmul\PYGZus{}\PYGZus{}\PYGZsq{}, \PYGZsq{}\PYGZus{}\PYGZus{}setattr\PYGZus{}\PYGZus{}\PYGZsq{}, \PYGZsq{}\PYGZus{}\PYGZus{}setitem\PYGZus{}\PYGZus{}\PYGZsq{}, \PYGZsq{}\PYGZus{}\PYGZus{}sizeof\PYGZus{}\PYGZus{}\PYGZsq{}, \PYGZsq{}\PYGZus{}\PYGZus{}str\PYGZus{}\PYGZus{}\PYGZsq{}, \PYGZsq{}\PYGZus{}\PYGZus{}subclasshook\PYGZus{}\PYGZus{}\PYGZsq{}, \PYGZsq{}append\PYGZsq{}, \PYGZsq{}clear\PYGZsq{}, \PYGZsq{}copy\PYGZsq{}, \PYGZsq{}count\PYGZsq{}, \PYGZsq{}extend\PYGZsq{}, \PYGZsq{}index\PYGZsq{}, \PYGZsq{}insert\PYGZsq{}, \PYGZsq{}pop\PYGZsq{}, \PYGZsq{}remove\PYGZsq{}, \PYGZsq{}reverse\PYGZsq{}, \PYGZsq{}sort\PYGZsq{}]


[\PYGZsq{}T\PYGZsq{}, \PYGZsq{}\PYGZus{}AXIS\PYGZus{}LEN\PYGZsq{}, \PYGZsq{}\PYGZus{}AXIS\PYGZus{}ORDERS\PYGZsq{}, \PYGZsq{}\PYGZus{}AXIS\PYGZus{}REVERSED\PYGZsq{}, \PYGZsq{}\PYGZus{}AXIS\PYGZus{}TO\PYGZus{}AXIS\PYGZus{}NUMBER\PYGZsq{}, \PYGZsq{}\PYGZus{}HANDLED\PYGZus{}TYPES\PYGZsq{}, \PYGZsq{}\PYGZus{}\PYGZus{}abs\PYGZus{}\PYGZus{}\PYGZsq{}, \PYGZsq{}\PYGZus{}\PYGZus{}add\PYGZus{}\PYGZus{}\PYGZsq{}, \PYGZsq{}\PYGZus{}\PYGZus{}and\PYGZus{}\PYGZus{}\PYGZsq{}, \PYGZsq{}\PYGZus{}\PYGZus{}annotations\PYGZus{}\PYGZus{}\PYGZsq{}, \PYGZsq{}\PYGZus{}\PYGZus{}array\PYGZus{}\PYGZus{}\PYGZsq{}, \PYGZsq{}\PYGZus{}\PYGZus{}array\PYGZus{}priority\PYGZus{}\PYGZus{}\PYGZsq{}, \PYGZsq{}\PYGZus{}\PYGZus{}array\PYGZus{}ufunc\PYGZus{}\PYGZus{}\PYGZsq{}, \PYGZsq{}\PYGZus{}\PYGZus{}array\PYGZus{}wrap\PYGZus{}\PYGZus{}\PYGZsq{}, \PYGZsq{}\PYGZus{}\PYGZus{}bool\PYGZus{}\PYGZus{}\PYGZsq{}, \PYGZsq{}\PYGZus{}\PYGZus{}class\PYGZus{}\PYGZus{}\PYGZsq{}, \PYGZsq{}\PYGZus{}\PYGZus{}contains\PYGZus{}\PYGZus{}\PYGZsq{}, \PYGZsq{}\PYGZus{}\PYGZus{}copy\PYGZus{}\PYGZus{}\PYGZsq{}, \PYGZsq{}\PYGZus{}\PYGZus{}deepcopy\PYGZus{}\PYGZus{}\PYGZsq{}, \PYGZsq{}\PYGZus{}\PYGZus{}delattr\PYGZus{}\PYGZus{}\PYGZsq{}, \PYGZsq{}\PYGZus{}\PYGZus{}delitem\PYGZus{}\PYGZus{}\PYGZsq{}, \PYGZsq{}\PYGZus{}\PYGZus{}dict\PYGZus{}\PYGZus{}\PYGZsq{}, \PYGZsq{}\PYGZus{}\PYGZus{}dir\PYGZus{}\PYGZus{}\PYGZsq{}, \PYGZsq{}\PYGZus{}\PYGZus{}divmod\PYGZus{}\PYGZus{}\PYGZsq{}, \PYGZsq{}\PYGZus{}\PYGZus{}doc\PYGZus{}\PYGZus{}\PYGZsq{}, \PYGZsq{}\PYGZus{}\PYGZus{}eq\PYGZus{}\PYGZus{}\PYGZsq{}, \PYGZsq{}\PYGZus{}\PYGZus{}finalize\PYGZus{}\PYGZus{}\PYGZsq{}, \PYGZsq{}\PYGZus{}\PYGZus{}float\PYGZus{}\PYGZus{}\PYGZsq{}, \PYGZsq{}\PYGZus{}\PYGZus{}floordiv\PYGZus{}\PYGZus{}\PYGZsq{}, \PYGZsq{}\PYGZus{}\PYGZus{}format\PYGZus{}\PYGZus{}\PYGZsq{}, \PYGZsq{}\PYGZus{}\PYGZus{}ge\PYGZus{}\PYGZus{}\PYGZsq{}, \PYGZsq{}\PYGZus{}\PYGZus{}getattr\PYGZus{}\PYGZus{}\PYGZsq{}, \PYGZsq{}\PYGZus{}\PYGZus{}getattribute\PYGZus{}\PYGZus{}\PYGZsq{}, \PYGZsq{}\PYGZus{}\PYGZus{}getitem\PYGZus{}\PYGZus{}\PYGZsq{}, \PYGZsq{}\PYGZus{}\PYGZus{}getstate\PYGZus{}\PYGZus{}\PYGZsq{}, \PYGZsq{}\PYGZus{}\PYGZus{}gt\PYGZus{}\PYGZus{}\PYGZsq{}, \PYGZsq{}\PYGZus{}\PYGZus{}hash\PYGZus{}\PYGZus{}\PYGZsq{}, \PYGZsq{}\PYGZus{}\PYGZus{}iadd\PYGZus{}\PYGZus{}\PYGZsq{}, \PYGZsq{}\PYGZus{}\PYGZus{}iand\PYGZus{}\PYGZus{}\PYGZsq{}, \PYGZsq{}\PYGZus{}\PYGZus{}ifloordiv\PYGZus{}\PYGZus{}\PYGZsq{}, \PYGZsq{}\PYGZus{}\PYGZus{}imod\PYGZus{}\PYGZus{}\PYGZsq{}, \PYGZsq{}\PYGZus{}\PYGZus{}imul\PYGZus{}\PYGZus{}\PYGZsq{}, \PYGZsq{}\PYGZus{}\PYGZus{}init\PYGZus{}\PYGZus{}\PYGZsq{}, \PYGZsq{}\PYGZus{}\PYGZus{}init\PYGZus{}subclass\PYGZus{}\PYGZus{}\PYGZsq{}, \PYGZsq{}\PYGZus{}\PYGZus{}int\PYGZus{}\PYGZus{}\PYGZsq{}, \PYGZsq{}\PYGZus{}\PYGZus{}invert\PYGZus{}\PYGZus{}\PYGZsq{}, \PYGZsq{}\PYGZus{}\PYGZus{}ior\PYGZus{}\PYGZus{}\PYGZsq{}, \PYGZsq{}\PYGZus{}\PYGZus{}ipow\PYGZus{}\PYGZus{}\PYGZsq{}, \PYGZsq{}\PYGZus{}\PYGZus{}isub\PYGZus{}\PYGZus{}\PYGZsq{}, \PYGZsq{}\PYGZus{}\PYGZus{}iter\PYGZus{}\PYGZus{}\PYGZsq{}, \PYGZsq{}\PYGZus{}\PYGZus{}itruediv\PYGZus{}\PYGZus{}\PYGZsq{}, \PYGZsq{}\PYGZus{}\PYGZus{}ixor\PYGZus{}\PYGZus{}\PYGZsq{}, \PYGZsq{}\PYGZus{}\PYGZus{}le\PYGZus{}\PYGZus{}\PYGZsq{}, \PYGZsq{}\PYGZus{}\PYGZus{}len\PYGZus{}\PYGZus{}\PYGZsq{}, \PYGZsq{}\PYGZus{}\PYGZus{}long\PYGZus{}\PYGZus{}\PYGZsq{}, \PYGZsq{}\PYGZus{}\PYGZus{}lt\PYGZus{}\PYGZus{}\PYGZsq{}, \PYGZsq{}\PYGZus{}\PYGZus{}matmul\PYGZus{}\PYGZus{}\PYGZsq{}, \PYGZsq{}\PYGZus{}\PYGZus{}mod\PYGZus{}\PYGZus{}\PYGZsq{}, \PYGZsq{}\PYGZus{}\PYGZus{}module\PYGZus{}\PYGZus{}\PYGZsq{}, \PYGZsq{}\PYGZus{}\PYGZus{}mul\PYGZus{}\PYGZus{}\PYGZsq{}, \PYGZsq{}\PYGZus{}\PYGZus{}ne\PYGZus{}\PYGZus{}\PYGZsq{}, \PYGZsq{}\PYGZus{}\PYGZus{}neg\PYGZus{}\PYGZus{}\PYGZsq{}, \PYGZsq{}\PYGZus{}\PYGZus{}new\PYGZus{}\PYGZus{}\PYGZsq{}, \PYGZsq{}\PYGZus{}\PYGZus{}nonzero\PYGZus{}\PYGZus{}\PYGZsq{}, \PYGZsq{}\PYGZus{}\PYGZus{}or\PYGZus{}\PYGZus{}\PYGZsq{}, \PYGZsq{}\PYGZus{}\PYGZus{}pos\PYGZus{}\PYGZus{}\PYGZsq{}, \PYGZsq{}\PYGZus{}\PYGZus{}pow\PYGZus{}\PYGZus{}\PYGZsq{}, \PYGZsq{}\PYGZus{}\PYGZus{}radd\PYGZus{}\PYGZus{}\PYGZsq{}, \PYGZsq{}\PYGZus{}\PYGZus{}rand\PYGZus{}\PYGZus{}\PYGZsq{}, \PYGZsq{}\PYGZus{}\PYGZus{}rdivmod\PYGZus{}\PYGZus{}\PYGZsq{}, \PYGZsq{}\PYGZus{}\PYGZus{}reduce\PYGZus{}\PYGZus{}\PYGZsq{}, \PYGZsq{}\PYGZus{}\PYGZus{}reduce\PYGZus{}ex\PYGZus{}\PYGZus{}\PYGZsq{}, \PYGZsq{}\PYGZus{}\PYGZus{}repr\PYGZus{}\PYGZus{}\PYGZsq{}, \PYGZsq{}\PYGZus{}\PYGZus{}rfloordiv\PYGZus{}\PYGZus{}\PYGZsq{}, \PYGZsq{}\PYGZus{}\PYGZus{}rmatmul\PYGZus{}\PYGZus{}\PYGZsq{}, \PYGZsq{}\PYGZus{}\PYGZus{}rmod\PYGZus{}\PYGZus{}\PYGZsq{}, \PYGZsq{}\PYGZus{}\PYGZus{}rmul\PYGZus{}\PYGZus{}\PYGZsq{}, \PYGZsq{}\PYGZus{}\PYGZus{}ror\PYGZus{}\PYGZus{}\PYGZsq{}, \PYGZsq{}\PYGZus{}\PYGZus{}round\PYGZus{}\PYGZus{}\PYGZsq{}, \PYGZsq{}\PYGZus{}\PYGZus{}rpow\PYGZus{}\PYGZus{}\PYGZsq{}, \PYGZsq{}\PYGZus{}\PYGZus{}rsub\PYGZus{}\PYGZus{}\PYGZsq{}, \PYGZsq{}\PYGZus{}\PYGZus{}rtruediv\PYGZus{}\PYGZus{}\PYGZsq{}, \PYGZsq{}\PYGZus{}\PYGZus{}rxor\PYGZus{}\PYGZus{}\PYGZsq{}, \PYGZsq{}\PYGZus{}\PYGZus{}setattr\PYGZus{}\PYGZus{}\PYGZsq{}, \PYGZsq{}\PYGZus{}\PYGZus{}setitem\PYGZus{}\PYGZus{}\PYGZsq{}, \PYGZsq{}\PYGZus{}\PYGZus{}setstate\PYGZus{}\PYGZus{}\PYGZsq{}, \PYGZsq{}\PYGZus{}\PYGZus{}sizeof\PYGZus{}\PYGZus{}\PYGZsq{}, \PYGZsq{}\PYGZus{}\PYGZus{}str\PYGZus{}\PYGZus{}\PYGZsq{}, \PYGZsq{}\PYGZus{}\PYGZus{}sub\PYGZus{}\PYGZus{}\PYGZsq{}, \PYGZsq{}\PYGZus{}\PYGZus{}subclasshook\PYGZus{}\PYGZus{}\PYGZsq{}, \PYGZsq{}\PYGZus{}\PYGZus{}truediv\PYGZus{}\PYGZus{}\PYGZsq{}, \PYGZsq{}\PYGZus{}\PYGZus{}weakref\PYGZus{}\PYGZus{}\PYGZsq{}, \PYGZsq{}\PYGZus{}\PYGZus{}xor\PYGZus{}\PYGZus{}\PYGZsq{}, \PYGZsq{}\PYGZus{}accessors\PYGZsq{}, \PYGZsq{}\PYGZus{}accum\PYGZus{}func\PYGZsq{}, \PYGZsq{}\PYGZus{}add\PYGZus{}numeric\PYGZus{}operations\PYGZsq{}, \PYGZsq{}\PYGZus{}agg\PYGZus{}by\PYGZus{}level\PYGZsq{}, \PYGZsq{}\PYGZus{}agg\PYGZus{}examples\PYGZus{}doc\PYGZsq{}, \PYGZsq{}\PYGZus{}agg\PYGZus{}see\PYGZus{}also\PYGZus{}doc\PYGZsq{}, \PYGZsq{}\PYGZus{}align\PYGZus{}frame\PYGZsq{}, \PYGZsq{}\PYGZus{}align\PYGZus{}series\PYGZsq{}, \PYGZsq{}\PYGZus{}arith\PYGZus{}method\PYGZsq{}, \PYGZsq{}\PYGZus{}as\PYGZus{}manager\PYGZsq{}, \PYGZsq{}\PYGZus{}attrs\PYGZsq{}, \PYGZsq{}\PYGZus{}binop\PYGZsq{}, \PYGZsq{}\PYGZus{}can\PYGZus{}hold\PYGZus{}na\PYGZsq{}, \PYGZsq{}\PYGZus{}check\PYGZus{}inplace\PYGZus{}and\PYGZus{}allows\PYGZus{}duplicate\PYGZus{}labels\PYGZsq{}, \PYGZsq{}\PYGZus{}check\PYGZus{}inplace\PYGZus{}setting\PYGZsq{}, \PYGZsq{}\PYGZus{}check\PYGZus{}is\PYGZus{}chained\PYGZus{}assignment\PYGZus{}possible\PYGZsq{}, \PYGZsq{}\PYGZus{}check\PYGZus{}label\PYGZus{}or\PYGZus{}level\PYGZus{}ambiguity\PYGZsq{}, \PYGZsq{}\PYGZus{}check\PYGZus{}setitem\PYGZus{}copy\PYGZsq{}, \PYGZsq{}\PYGZus{}clear\PYGZus{}item\PYGZus{}cache\PYGZsq{}, \PYGZsq{}\PYGZus{}clip\PYGZus{}with\PYGZus{}one\PYGZus{}bound\PYGZsq{}, \PYGZsq{}\PYGZus{}clip\PYGZus{}with\PYGZus{}scalar\PYGZsq{}, \PYGZsq{}\PYGZus{}cmp\PYGZus{}method\PYGZsq{}, \PYGZsq{}\PYGZus{}consolidate\PYGZsq{}, \PYGZsq{}\PYGZus{}consolidate\PYGZus{}inplace\PYGZsq{}, \PYGZsq{}\PYGZus{}construct\PYGZus{}axes\PYGZus{}dict\PYGZsq{}, \PYGZsq{}\PYGZus{}construct\PYGZus{}axes\PYGZus{}from\PYGZus{}arguments\PYGZsq{}, \PYGZsq{}\PYGZus{}construct\PYGZus{}result\PYGZsq{}, \PYGZsq{}\PYGZus{}constructor\PYGZsq{}, \PYGZsq{}\PYGZus{}constructor\PYGZus{}expanddim\PYGZsq{}, \PYGZsq{}\PYGZus{}convert\PYGZsq{}, \PYGZsq{}\PYGZus{}convert\PYGZus{}dtypes\PYGZsq{}, \PYGZsq{}\PYGZus{}data\PYGZsq{}, \PYGZsq{}\PYGZus{}dir\PYGZus{}additions\PYGZsq{}, \PYGZsq{}\PYGZus{}dir\PYGZus{}deletions\PYGZsq{}, \PYGZsq{}\PYGZus{}drop\PYGZus{}axis\PYGZsq{}, \PYGZsq{}\PYGZus{}drop\PYGZus{}labels\PYGZus{}or\PYGZus{}levels\PYGZsq{}, \PYGZsq{}\PYGZus{}duplicated\PYGZsq{}, \PYGZsq{}\PYGZus{}find\PYGZus{}valid\PYGZus{}index\PYGZsq{}, \PYGZsq{}\PYGZus{}flags\PYGZsq{}, \PYGZsq{}\PYGZus{}from\PYGZus{}mgr\PYGZsq{}, \PYGZsq{}\PYGZus{}get\PYGZus{}axis\PYGZsq{}, \PYGZsq{}\PYGZus{}get\PYGZus{}axis\PYGZus{}name\PYGZsq{}, \PYGZsq{}\PYGZus{}get\PYGZus{}axis\PYGZus{}number\PYGZsq{}, \PYGZsq{}\PYGZus{}get\PYGZus{}axis\PYGZus{}resolvers\PYGZsq{}, \PYGZsq{}\PYGZus{}get\PYGZus{}block\PYGZus{}manager\PYGZus{}axis\PYGZsq{}, \PYGZsq{}\PYGZus{}get\PYGZus{}bool\PYGZus{}data\PYGZsq{}, \PYGZsq{}\PYGZus{}get\PYGZus{}cacher\PYGZsq{}, \PYGZsq{}\PYGZus{}get\PYGZus{}cleaned\PYGZus{}column\PYGZus{}resolvers\PYGZsq{}, \PYGZsq{}\PYGZus{}get\PYGZus{}index\PYGZus{}resolvers\PYGZsq{}, \PYGZsq{}\PYGZus{}get\PYGZus{}label\PYGZus{}or\PYGZus{}level\PYGZus{}values\PYGZsq{}, \PYGZsq{}\PYGZus{}get\PYGZus{}numeric\PYGZus{}data\PYGZsq{}, \PYGZsq{}\PYGZus{}get\PYGZus{}value\PYGZsq{}, \PYGZsq{}\PYGZus{}get\PYGZus{}values\PYGZsq{}, \PYGZsq{}\PYGZus{}get\PYGZus{}values\PYGZus{}tuple\PYGZsq{}, \PYGZsq{}\PYGZus{}get\PYGZus{}with\PYGZsq{}, \PYGZsq{}\PYGZus{}gotitem\PYGZsq{}, \PYGZsq{}\PYGZus{}hidden\PYGZus{}attrs\PYGZsq{}, \PYGZsq{}\PYGZus{}index\PYGZsq{}, \PYGZsq{}\PYGZus{}indexed\PYGZus{}same\PYGZsq{}, \PYGZsq{}\PYGZus{}info\PYGZus{}axis\PYGZsq{}, \PYGZsq{}\PYGZus{}info\PYGZus{}axis\PYGZus{}name\PYGZsq{}, \PYGZsq{}\PYGZus{}info\PYGZus{}axis\PYGZus{}number\PYGZsq{}, \PYGZsq{}\PYGZus{}init\PYGZus{}dict\PYGZsq{}, \PYGZsq{}\PYGZus{}init\PYGZus{}mgr\PYGZsq{}, \PYGZsq{}\PYGZus{}inplace\PYGZus{}method\PYGZsq{}, \PYGZsq{}\PYGZus{}internal\PYGZus{}names\PYGZsq{}, \PYGZsq{}\PYGZus{}internal\PYGZus{}names\PYGZus{}set\PYGZsq{}, \PYGZsq{}\PYGZus{}is\PYGZus{}cached\PYGZsq{}, \PYGZsq{}\PYGZus{}is\PYGZus{}copy\PYGZsq{}, \PYGZsq{}\PYGZus{}is\PYGZus{}label\PYGZus{}or\PYGZus{}level\PYGZus{}reference\PYGZsq{}, \PYGZsq{}\PYGZus{}is\PYGZus{}label\PYGZus{}reference\PYGZsq{}, \PYGZsq{}\PYGZus{}is\PYGZus{}level\PYGZus{}reference\PYGZsq{}, \PYGZsq{}\PYGZus{}is\PYGZus{}mixed\PYGZus{}type\PYGZsq{}, \PYGZsq{}\PYGZus{}is\PYGZus{}view\PYGZsq{}, \PYGZsq{}\PYGZus{}item\PYGZus{}cache\PYGZsq{}, \PYGZsq{}\PYGZus{}ixs\PYGZsq{}, \PYGZsq{}\PYGZus{}logical\PYGZus{}func\PYGZsq{}, \PYGZsq{}\PYGZus{}logical\PYGZus{}method\PYGZsq{}, \PYGZsq{}\PYGZus{}map\PYGZus{}values\PYGZsq{}, \PYGZsq{}\PYGZus{}maybe\PYGZus{}update\PYGZus{}cacher\PYGZsq{}, \PYGZsq{}\PYGZus{}memory\PYGZus{}usage\PYGZsq{}, \PYGZsq{}\PYGZus{}metadata\PYGZsq{}, \PYGZsq{}\PYGZus{}mgr\PYGZsq{}, \PYGZsq{}\PYGZus{}min\PYGZus{}count\PYGZus{}stat\PYGZus{}function\PYGZsq{}, \PYGZsq{}\PYGZus{}name\PYGZsq{}, \PYGZsq{}\PYGZus{}needs\PYGZus{}reindex\PYGZus{}multi\PYGZsq{}, \PYGZsq{}\PYGZus{}protect\PYGZus{}consolidate\PYGZsq{}, \PYGZsq{}\PYGZus{}reduce\PYGZsq{}, \PYGZsq{}\PYGZus{}reindex\PYGZus{}axes\PYGZsq{}, \PYGZsq{}\PYGZus{}reindex\PYGZus{}indexer\PYGZsq{}, \PYGZsq{}\PYGZus{}reindex\PYGZus{}multi\PYGZsq{}, \PYGZsq{}\PYGZus{}reindex\PYGZus{}with\PYGZus{}indexers\PYGZsq{}, \PYGZsq{}\PYGZus{}replace\PYGZus{}single\PYGZsq{}, \PYGZsq{}\PYGZus{}repr\PYGZus{}data\PYGZus{}resource\PYGZus{}\PYGZsq{}, \PYGZsq{}\PYGZus{}repr\PYGZus{}latex\PYGZus{}\PYGZsq{}, \PYGZsq{}\PYGZus{}reset\PYGZus{}cache\PYGZsq{}, \PYGZsq{}\PYGZus{}reset\PYGZus{}cacher\PYGZsq{}, \PYGZsq{}\PYGZus{}set\PYGZus{}as\PYGZus{}cached\PYGZsq{}, \PYGZsq{}\PYGZus{}set\PYGZus{}axis\PYGZsq{}, \PYGZsq{}\PYGZus{}set\PYGZus{}axis\PYGZus{}name\PYGZsq{}, \PYGZsq{}\PYGZus{}set\PYGZus{}axis\PYGZus{}nocheck\PYGZsq{}, \PYGZsq{}\PYGZus{}set\PYGZus{}is\PYGZus{}copy\PYGZsq{}, \PYGZsq{}\PYGZus{}set\PYGZus{}labels\PYGZsq{}, \PYGZsq{}\PYGZus{}set\PYGZus{}name\PYGZsq{}, \PYGZsq{}\PYGZus{}set\PYGZus{}value\PYGZsq{}, \PYGZsq{}\PYGZus{}set\PYGZus{}values\PYGZsq{}, \PYGZsq{}\PYGZus{}set\PYGZus{}with\PYGZsq{}, \PYGZsq{}\PYGZus{}set\PYGZus{}with\PYGZus{}engine\PYGZsq{}, \PYGZsq{}\PYGZus{}slice\PYGZsq{}, \PYGZsq{}\PYGZus{}stat\PYGZus{}axis\PYGZsq{}, \PYGZsq{}\PYGZus{}stat\PYGZus{}axis\PYGZus{}name\PYGZsq{}, \PYGZsq{}\PYGZus{}stat\PYGZus{}axis\PYGZus{}number\PYGZsq{}, \PYGZsq{}\PYGZus{}stat\PYGZus{}function\PYGZsq{}, \PYGZsq{}\PYGZus{}stat\PYGZus{}function\PYGZus{}ddof\PYGZsq{}, \PYGZsq{}\PYGZus{}take\PYGZus{}with\PYGZus{}is\PYGZus{}copy\PYGZsq{}, \PYGZsq{}\PYGZus{}typ\PYGZsq{}, \PYGZsq{}\PYGZus{}update\PYGZus{}inplace\PYGZsq{}, \PYGZsq{}\PYGZus{}validate\PYGZus{}dtype\PYGZsq{}, \PYGZsq{}\PYGZus{}values\PYGZsq{}, \PYGZsq{}\PYGZus{}where\PYGZsq{}, \PYGZsq{}abs\PYGZsq{}, \PYGZsq{}add\PYGZsq{}, \PYGZsq{}add\PYGZus{}prefix\PYGZsq{}, \PYGZsq{}add\PYGZus{}suffix\PYGZsq{}, \PYGZsq{}agg\PYGZsq{}, \PYGZsq{}aggregate\PYGZsq{}, \PYGZsq{}align\PYGZsq{}, \PYGZsq{}all\PYGZsq{}, \PYGZsq{}any\PYGZsq{}, \PYGZsq{}append\PYGZsq{}, \PYGZsq{}apply\PYGZsq{}, \PYGZsq{}argmax\PYGZsq{}, \PYGZsq{}argmin\PYGZsq{}, \PYGZsq{}argsort\PYGZsq{}, \PYGZsq{}array\PYGZsq{}, \PYGZsq{}asfreq\PYGZsq{}, \PYGZsq{}asof\PYGZsq{}, \PYGZsq{}astype\PYGZsq{}, \PYGZsq{}at\PYGZsq{}, \PYGZsq{}at\PYGZus{}time\PYGZsq{}, \PYGZsq{}attrs\PYGZsq{}, \PYGZsq{}autocorr\PYGZsq{}, \PYGZsq{}axes\PYGZsq{}, \PYGZsq{}backfill\PYGZsq{}, \PYGZsq{}between\PYGZsq{}, \PYGZsq{}between\PYGZus{}time\PYGZsq{}, \PYGZsq{}bfill\PYGZsq{}, \PYGZsq{}bool\PYGZsq{}, \PYGZsq{}clip\PYGZsq{}, \PYGZsq{}combine\PYGZsq{}, \PYGZsq{}combine\PYGZus{}first\PYGZsq{}, \PYGZsq{}compare\PYGZsq{}, \PYGZsq{}convert\PYGZus{}dtypes\PYGZsq{}, \PYGZsq{}copy\PYGZsq{}, \PYGZsq{}corr\PYGZsq{}, \PYGZsq{}count\PYGZsq{}, \PYGZsq{}cov\PYGZsq{}, \PYGZsq{}cummax\PYGZsq{}, \PYGZsq{}cummin\PYGZsq{}, \PYGZsq{}cumprod\PYGZsq{}, \PYGZsq{}cumsum\PYGZsq{}, \PYGZsq{}describe\PYGZsq{}, \PYGZsq{}diff\PYGZsq{}, \PYGZsq{}div\PYGZsq{}, \PYGZsq{}divide\PYGZsq{}, \PYGZsq{}divmod\PYGZsq{}, \PYGZsq{}dot\PYGZsq{}, \PYGZsq{}drop\PYGZsq{}, \PYGZsq{}drop\PYGZus{}duplicates\PYGZsq{}, \PYGZsq{}droplevel\PYGZsq{}, \PYGZsq{}dropna\PYGZsq{}, \PYGZsq{}dtype\PYGZsq{}, \PYGZsq{}dtypes\PYGZsq{}, \PYGZsq{}duplicated\PYGZsq{}, \PYGZsq{}empty\PYGZsq{}, \PYGZsq{}eq\PYGZsq{}, \PYGZsq{}equals\PYGZsq{}, \PYGZsq{}ewm\PYGZsq{}, \PYGZsq{}expanding\PYGZsq{}, \PYGZsq{}explode\PYGZsq{}, \PYGZsq{}factorize\PYGZsq{}, \PYGZsq{}ffill\PYGZsq{}, \PYGZsq{}fillna\PYGZsq{}, \PYGZsq{}filter\PYGZsq{}, \PYGZsq{}first\PYGZsq{}, \PYGZsq{}first\PYGZus{}valid\PYGZus{}index\PYGZsq{}, \PYGZsq{}flags\PYGZsq{}, \PYGZsq{}floordiv\PYGZsq{}, \PYGZsq{}ge\PYGZsq{}, \PYGZsq{}get\PYGZsq{}, \PYGZsq{}groupby\PYGZsq{}, \PYGZsq{}gt\PYGZsq{}, \PYGZsq{}hasnans\PYGZsq{}, \PYGZsq{}head\PYGZsq{}, \PYGZsq{}hist\PYGZsq{}, \PYGZsq{}iat\PYGZsq{}, \PYGZsq{}idxmax\PYGZsq{}, \PYGZsq{}idxmin\PYGZsq{}, \PYGZsq{}iloc\PYGZsq{}, \PYGZsq{}index\PYGZsq{}, \PYGZsq{}infer\PYGZus{}objects\PYGZsq{}, \PYGZsq{}interpolate\PYGZsq{}, \PYGZsq{}is\PYGZus{}monotonic\PYGZsq{}, \PYGZsq{}is\PYGZus{}monotonic\PYGZus{}decreasing\PYGZsq{}, \PYGZsq{}is\PYGZus{}monotonic\PYGZus{}increasing\PYGZsq{}, \PYGZsq{}is\PYGZus{}unique\PYGZsq{}, \PYGZsq{}isin\PYGZsq{}, \PYGZsq{}isna\PYGZsq{}, \PYGZsq{}isnull\PYGZsq{}, \PYGZsq{}item\PYGZsq{}, \PYGZsq{}items\PYGZsq{}, \PYGZsq{}iteritems\PYGZsq{}, \PYGZsq{}keys\PYGZsq{}, \PYGZsq{}kurt\PYGZsq{}, \PYGZsq{}kurtosis\PYGZsq{}, \PYGZsq{}last\PYGZsq{}, \PYGZsq{}last\PYGZus{}valid\PYGZus{}index\PYGZsq{}, \PYGZsq{}le\PYGZsq{}, \PYGZsq{}loc\PYGZsq{}, \PYGZsq{}lt\PYGZsq{}, \PYGZsq{}mad\PYGZsq{}, \PYGZsq{}map\PYGZsq{}, \PYGZsq{}mask\PYGZsq{}, \PYGZsq{}max\PYGZsq{}, \PYGZsq{}mean\PYGZsq{}, \PYGZsq{}median\PYGZsq{}, \PYGZsq{}memory\PYGZus{}usage\PYGZsq{}, \PYGZsq{}min\PYGZsq{}, \PYGZsq{}mod\PYGZsq{}, \PYGZsq{}mode\PYGZsq{}, \PYGZsq{}mul\PYGZsq{}, \PYGZsq{}multiply\PYGZsq{}, \PYGZsq{}name\PYGZsq{}, \PYGZsq{}nbytes\PYGZsq{}, \PYGZsq{}ndim\PYGZsq{}, \PYGZsq{}ne\PYGZsq{}, \PYGZsq{}nlargest\PYGZsq{}, \PYGZsq{}notna\PYGZsq{}, \PYGZsq{}notnull\PYGZsq{}, \PYGZsq{}nsmallest\PYGZsq{}, \PYGZsq{}nunique\PYGZsq{}, \PYGZsq{}pad\PYGZsq{}, \PYGZsq{}pct\PYGZus{}change\PYGZsq{}, \PYGZsq{}pipe\PYGZsq{}, \PYGZsq{}plot\PYGZsq{}, \PYGZsq{}pop\PYGZsq{}, \PYGZsq{}pow\PYGZsq{}, \PYGZsq{}prod\PYGZsq{}, \PYGZsq{}product\PYGZsq{}, \PYGZsq{}quantile\PYGZsq{}, \PYGZsq{}radd\PYGZsq{}, \PYGZsq{}rank\PYGZsq{}, \PYGZsq{}ravel\PYGZsq{}, \PYGZsq{}rdiv\PYGZsq{}, \PYGZsq{}rdivmod\PYGZsq{}, \PYGZsq{}reindex\PYGZsq{}, \PYGZsq{}reindex\PYGZus{}like\PYGZsq{}, \PYGZsq{}rename\PYGZsq{}, \PYGZsq{}rename\PYGZus{}axis\PYGZsq{}, \PYGZsq{}reorder\PYGZus{}levels\PYGZsq{}, \PYGZsq{}repeat\PYGZsq{}, \PYGZsq{}replace\PYGZsq{}, \PYGZsq{}resample\PYGZsq{}, \PYGZsq{}reset\PYGZus{}index\PYGZsq{}, \PYGZsq{}rfloordiv\PYGZsq{}, \PYGZsq{}rmod\PYGZsq{}, \PYGZsq{}rmul\PYGZsq{}, \PYGZsq{}rolling\PYGZsq{}, \PYGZsq{}round\PYGZsq{}, \PYGZsq{}rpow\PYGZsq{}, \PYGZsq{}rsub\PYGZsq{}, \PYGZsq{}rtruediv\PYGZsq{}, \PYGZsq{}sample\PYGZsq{}, \PYGZsq{}searchsorted\PYGZsq{}, \PYGZsq{}sem\PYGZsq{}, \PYGZsq{}set\PYGZus{}axis\PYGZsq{}, \PYGZsq{}set\PYGZus{}flags\PYGZsq{}, \PYGZsq{}shape\PYGZsq{}, \PYGZsq{}shift\PYGZsq{}, \PYGZsq{}size\PYGZsq{}, \PYGZsq{}skew\PYGZsq{}, \PYGZsq{}slice\PYGZus{}shift\PYGZsq{}, \PYGZsq{}sort\PYGZus{}index\PYGZsq{}, \PYGZsq{}sort\PYGZus{}values\PYGZsq{}, \PYGZsq{}squeeze\PYGZsq{}, \PYGZsq{}std\PYGZsq{}, \PYGZsq{}sub\PYGZsq{}, \PYGZsq{}subtract\PYGZsq{}, \PYGZsq{}sum\PYGZsq{}, \PYGZsq{}swapaxes\PYGZsq{}, \PYGZsq{}swaplevel\PYGZsq{}, \PYGZsq{}tail\PYGZsq{}, \PYGZsq{}take\PYGZsq{}, \PYGZsq{}to\PYGZus{}clipboard\PYGZsq{}, \PYGZsq{}to\PYGZus{}csv\PYGZsq{}, \PYGZsq{}to\PYGZus{}dict\PYGZsq{}, \PYGZsq{}to\PYGZus{}excel\PYGZsq{}, \PYGZsq{}to\PYGZus{}frame\PYGZsq{}, \PYGZsq{}to\PYGZus{}hdf\PYGZsq{}, \PYGZsq{}to\PYGZus{}json\PYGZsq{}, \PYGZsq{}to\PYGZus{}latex\PYGZsq{}, \PYGZsq{}to\PYGZus{}list\PYGZsq{}, \PYGZsq{}to\PYGZus{}markdown\PYGZsq{}, \PYGZsq{}to\PYGZus{}numpy\PYGZsq{}, \PYGZsq{}to\PYGZus{}period\PYGZsq{}, \PYGZsq{}to\PYGZus{}pickle\PYGZsq{}, \PYGZsq{}to\PYGZus{}sql\PYGZsq{}, \PYGZsq{}to\PYGZus{}string\PYGZsq{}, \PYGZsq{}to\PYGZus{}timestamp\PYGZsq{}, \PYGZsq{}to\PYGZus{}xarray\PYGZsq{}, \PYGZsq{}transform\PYGZsq{}, \PYGZsq{}transpose\PYGZsq{}, \PYGZsq{}truediv\PYGZsq{}, \PYGZsq{}truncate\PYGZsq{}, \PYGZsq{}tz\PYGZus{}convert\PYGZsq{}, \PYGZsq{}tz\PYGZus{}localize\PYGZsq{}, \PYGZsq{}unique\PYGZsq{}, \PYGZsq{}unstack\PYGZsq{}, \PYGZsq{}update\PYGZsq{}, \PYGZsq{}value\PYGZus{}counts\PYGZsq{}, \PYGZsq{}values\PYGZsq{}, \PYGZsq{}var\PYGZsq{}, \PYGZsq{}view\PYGZsq{}, \PYGZsq{}where\PYGZsq{}, \PYGZsq{}xs\PYGZsq{}]


13.0
\end{sphinxVerbatim}

\end{sphinxuseclass}\end{sphinxVerbatimOutput}

\end{sphinxuseclass}

\section{DataFrame}
\label{\detokenize{chapter2/2.1.1_Python_Basics:dataframe}}
\sphinxAtStartPar
DataFrame 은 Series의 확장으로, DataFrame 에서 한 개의 Series 는 하나의 Column 이 됩니다. DataFrame 은 여러 개 Column 이 모여 있는 테이블 형태의 데이터 형식입니다. 우선 Dictionary 로 DataFrame 을 만들어 보겠습니다. 두 개의 List \sphinxhyphen{} c1\_list 와 c2\_list 가 두 개의 key \sphinxhyphen{} ‘c1’, ‘c2’ 대응이 되는 Dictionary, dic\_c12 를 생성합니다. 그 다음 pd.DataFrame(dic\_c12) 와 같이 DataFrame 으로 데이터 타입을 변경합니다. 출력해보면 테이블 형태로 변경되었음을 알 수 있습니다. 그리고 이 DataFrame 을 df 라는 변수에 저장합니다. df 에서 한 컬럼만 자르면 다시 Series 로 변경됩니다. Series 보다 더 많은 메소드를 이용할 수 있습니다.

\begin{sphinxuseclass}{cell}\begin{sphinxVerbatimInput}

\begin{sphinxuseclass}{cell_input}
\begin{sphinxVerbatim}[commandchars=\\\{\}]
\PYG{n}{c1\PYGZus{}list} \PYG{o}{=} \PYG{p}{[}\PYG{l+m+mi}{11}\PYG{p}{,}\PYG{l+m+mi}{12}\PYG{p}{,}\PYG{l+m+mi}{13}\PYG{p}{,}\PYG{l+m+mi}{14}\PYG{p}{,}\PYG{l+m+mi}{15}\PYG{p}{]}
\PYG{n}{c2\PYGZus{}list} \PYG{o}{=} \PYG{p}{[}\PYG{l+s+s1}{\PYGZsq{}}\PYG{l+s+s1}{a}\PYG{l+s+s1}{\PYGZsq{}}\PYG{p}{,}\PYG{l+s+s1}{\PYGZsq{}}\PYG{l+s+s1}{b}\PYG{l+s+s1}{\PYGZsq{}}\PYG{p}{,}\PYG{l+s+s1}{\PYGZsq{}}\PYG{l+s+s1}{c}\PYG{l+s+s1}{\PYGZsq{}}\PYG{p}{,}\PYG{l+s+s1}{\PYGZsq{}}\PYG{l+s+s1}{d}\PYG{l+s+s1}{\PYGZsq{}}\PYG{p}{,}\PYG{l+s+s1}{\PYGZsq{}}\PYG{l+s+s1}{e}\PYG{l+s+s1}{\PYGZsq{}}\PYG{p}{]}

\PYG{n}{dic\PYGZus{}c12} \PYG{o}{=} \PYG{p}{\PYGZob{}}\PYG{l+s+s1}{\PYGZsq{}}\PYG{l+s+s1}{c1}\PYG{l+s+s1}{\PYGZsq{}}\PYG{p}{:} \PYG{n}{c1\PYGZus{}list}\PYG{p}{,} \PYG{l+s+s1}{\PYGZsq{}}\PYG{l+s+s1}{c2}\PYG{l+s+s1}{\PYGZsq{}}\PYG{p}{:} \PYG{n}{c2\PYGZus{}list}\PYG{p}{\PYGZcb{}}
\PYG{n}{df} \PYG{o}{=} \PYG{n}{pd}\PYG{o}{.}\PYG{n}{DataFrame}\PYG{p}{(}\PYG{n}{dic\PYGZus{}c12}\PYG{p}{)} \PYG{c+c1}{\PYGZsh{} DataFrame 으로 변경}

\PYG{n+nb}{print}\PYG{p}{(}\PYG{n}{df}\PYG{p}{)}
\PYG{n+nb}{print}\PYG{p}{(}\PYG{l+s+s1}{\PYGZsq{}}\PYG{l+s+se}{\PYGZbs{}n}\PYG{l+s+s1}{\PYGZsq{}}\PYG{p}{)}

\PYG{n+nb}{print}\PYG{p}{(}\PYG{n}{df}\PYG{p}{[}\PYG{l+s+s1}{\PYGZsq{}}\PYG{l+s+s1}{c1}\PYG{l+s+s1}{\PYGZsq{}}\PYG{p}{]}\PYG{p}{,} \PYG{n+nb}{type}\PYG{p}{(}\PYG{n}{df}\PYG{p}{[}\PYG{l+s+s1}{\PYGZsq{}}\PYG{l+s+s1}{c1}\PYG{l+s+s1}{\PYGZsq{}}\PYG{p}{]}\PYG{p}{)}\PYG{p}{)}
\end{sphinxVerbatim}

\end{sphinxuseclass}\end{sphinxVerbatimInput}
\begin{sphinxVerbatimOutput}

\begin{sphinxuseclass}{cell_output}
\begin{sphinxVerbatim}[commandchars=\\\{\}]
   c1 c2
0  11  a
1  12  b
2  13  c
3  14  d
4  15  e


0    11
1    12
2    13
3    14
4    15
Name: c1, dtype: int64 \PYGZlt{}class \PYGZsq{}pandas.core.series.Series\PYGZsq{}\PYGZgt{}
\end{sphinxVerbatim}

\end{sphinxuseclass}\end{sphinxVerbatimOutput}

\end{sphinxuseclass}

\section{Index}
\label{\detokenize{chapter2/2.1.2_Python_Basics:index}}\label{\detokenize{chapter2/2.1.2_Python_Basics::doc}}
\sphinxAtStartPar
데이터 처리에 중요한 역활을 하는 Index 에 대하여 알아보겠습니다. Index 는 우리말로 색인이라고 할 수 있을 것 같은데요. 색인은 무엇을 빨리 찾기 위해 순서대로 정리되어 있는 목록입니다. Index 는 색인처럼 어떤 값을 빨리 찾을 때도 필요하지만, 두 데이터를 어떤 값을 기준으로 결합하는데도 유용하게 쓰입니다. Index 는 Series 와 DataFrame 에 주로 활용됩니다. ss2 는 바로 이전 장에서 만든 Series 입니다. 출력을 해 보면 맨 왼쪽에 0 \textasciitilde{} 4 까지 값이 보이는데요. 이게 Index 입니다. 특별하게 지정하지 않으면 숫자 0 부터서 순서대로 들어가게 됩니다. 다음은 알파벳 Index 를 넣어서 ss3 를 생성하고 출력 해보겠습니다. 맨 왼쪽 index 값이 숫자가 아니라 알파벳으로 바뀌었습니다.

\begin{sphinxuseclass}{cell}\begin{sphinxVerbatimInput}

\begin{sphinxuseclass}{cell_input}
\begin{sphinxVerbatim}[commandchars=\\\{\}]
\PYG{k+kn}{import} \PYG{n+nn}{pandas} \PYG{k}{as} \PYG{n+nn}{pd}

\PYG{n}{ss1} \PYG{o}{=} \PYG{p}{[}\PYG{l+m+mi}{11}\PYG{p}{,}\PYG{l+m+mi}{12}\PYG{p}{,}\PYG{l+m+mi}{13}\PYG{p}{,}\PYG{l+m+mi}{14}\PYG{p}{,}\PYG{l+m+mi}{15}\PYG{p}{]}
\PYG{n}{ss2} \PYG{o}{=} \PYG{n}{pd}\PYG{o}{.}\PYG{n}{Series}\PYG{p}{(}\PYG{n}{ss1}\PYG{p}{)}
\PYG{n+nb}{print}\PYG{p}{(}\PYG{n}{ss2}\PYG{p}{)}

\PYG{n}{ss3} \PYG{o}{=} \PYG{n}{pd}\PYG{o}{.}\PYG{n}{Series}\PYG{p}{(}\PYG{n}{ss1}\PYG{p}{,} \PYG{n}{index}\PYG{o}{=}\PYG{p}{[}\PYG{l+s+s1}{\PYGZsq{}}\PYG{l+s+s1}{a}\PYG{l+s+s1}{\PYGZsq{}}\PYG{p}{,} \PYG{l+s+s1}{\PYGZsq{}}\PYG{l+s+s1}{b}\PYG{l+s+s1}{\PYGZsq{}}\PYG{p}{,} \PYG{l+s+s1}{\PYGZsq{}}\PYG{l+s+s1}{c}\PYG{l+s+s1}{\PYGZsq{}}\PYG{p}{,} \PYG{l+s+s1}{\PYGZsq{}}\PYG{l+s+s1}{d}\PYG{l+s+s1}{\PYGZsq{}}\PYG{p}{,} \PYG{l+s+s1}{\PYGZsq{}}\PYG{l+s+s1}{e}\PYG{l+s+s1}{\PYGZsq{}}\PYG{p}{]}\PYG{p}{)}
\PYG{n+nb}{print}\PYG{p}{(}\PYG{n}{ss3}\PYG{p}{)}
\end{sphinxVerbatim}

\end{sphinxuseclass}\end{sphinxVerbatimInput}
\begin{sphinxVerbatimOutput}

\begin{sphinxuseclass}{cell_output}
\begin{sphinxVerbatim}[commandchars=\\\{\}]
0    11
1    12
2    13
3    14
4    15
dtype: int64
a    11
b    12
c    13
d    14
e    15
dtype: int64
\end{sphinxVerbatim}

\end{sphinxuseclass}\end{sphinxVerbatimOutput}

\end{sphinxuseclass}

\section{Index 활용}
\label{\detokenize{chapter2/2.1.2_Python_Basics:id1}}
\sphinxAtStartPar
Index 의 본연의 기능은 찾기입니다. ss3.loc{[}인덱스값{]} 를 이용하여 원하는 값을 찾을 수 있습니다. 인덱스 ‘c’ 에 해당하는 값은 13입니다. ss3.loc{[}‘c’{]} 를 하면 13이 출력됩니다. 만약, 인덱스 ‘a’ 와 ‘c’ 를 다 찾고 싶으면 {[}‘a’, ‘c’{]} 와 같이 List 로 넣어주면 됩니다. loc 를 하지 않아도 같은 결과를 얻으시겠지만, loc 를 넣으면 ‘a’,’c’ 를 column 이 아니라 index 에서 찾는다는 것을 명확하게 해 줍니다.

\begin{sphinxuseclass}{cell}\begin{sphinxVerbatimInput}

\begin{sphinxuseclass}{cell_input}
\begin{sphinxVerbatim}[commandchars=\\\{\}]
\PYG{n+nb}{print}\PYG{p}{(}\PYG{n}{ss3}\PYG{o}{.}\PYG{n}{loc}\PYG{p}{[}\PYG{l+s+s1}{\PYGZsq{}}\PYG{l+s+s1}{a}\PYG{l+s+s1}{\PYGZsq{}}\PYG{p}{]}\PYG{p}{,} \PYG{n}{ss3}\PYG{p}{[}\PYG{l+s+s1}{\PYGZsq{}}\PYG{l+s+s1}{c}\PYG{l+s+s1}{\PYGZsq{}}\PYG{p}{]}\PYG{p}{)}
\PYG{n+nb}{print}\PYG{p}{(}\PYG{n}{ss3}\PYG{o}{.}\PYG{n}{loc}\PYG{p}{[}\PYG{p}{[}\PYG{l+s+s1}{\PYGZsq{}}\PYG{l+s+s1}{a}\PYG{l+s+s1}{\PYGZsq{}}\PYG{p}{,}\PYG{l+s+s1}{\PYGZsq{}}\PYG{l+s+s1}{c}\PYG{l+s+s1}{\PYGZsq{}}\PYG{p}{]}\PYG{p}{]}\PYG{p}{)}
\end{sphinxVerbatim}

\end{sphinxuseclass}\end{sphinxVerbatimInput}
\begin{sphinxVerbatimOutput}

\begin{sphinxuseclass}{cell_output}
\begin{sphinxVerbatim}[commandchars=\\\{\}]
11 13
a    11
c    13
dtype: int64
\end{sphinxVerbatim}

\end{sphinxuseclass}\end{sphinxVerbatimOutput}

\end{sphinxuseclass}
\sphinxAtStartPar
 DataFrame 에서도 동일하게 활용가능합니다. 먼저 df1 이라는 DataFrame 을 생성하고 출력합니다. Default Index 인 숫자 0 \textasciitilde{} 4 로 되어 있음을 확인할 수 있습니다. 이제 원하는 인덱스 s1 \textasciitilde{} s5 를 할당하고 df2 에 저장합니다. 출력 결과를 보니 df2 의 인덱스가 바뀌었습니다.

\sphinxAtStartPar
이번에는 원하는 값을 찾아보겠습니다. df2 의 index 가 ‘s3’ 인 c1 컬럼값을 알고 싶다면 df2.loc{[}‘s3’{]}{[}‘c1’{]} 이라고 하면 됩니다. 만약, c1 과 c2 둘다 출력하고 싶으면  df2.loc{[}‘s3’{]}{[}{[}‘c1’,’c2’{]}{]} 형태로 리스트로 입력합니다. 실수로 df2.loc{[}‘s3’{]}{[}‘c1’,’c2’{]} 로 입력을 하면 Pandas 패키지는 ‘c1,’c2’ 가 하나의 column 이름이라고 착각하게 되어 에러가 발생합니다.

\begin{sphinxuseclass}{cell}\begin{sphinxVerbatimInput}

\begin{sphinxuseclass}{cell_input}
\begin{sphinxVerbatim}[commandchars=\\\{\}]
\PYG{c+c1}{\PYGZsh{} DataFrame 생성}
\PYG{n}{c1\PYGZus{}list} \PYG{o}{=} \PYG{p}{[}\PYG{l+m+mi}{11}\PYG{p}{,}\PYG{l+m+mi}{12}\PYG{p}{,}\PYG{l+m+mi}{13}\PYG{p}{,}\PYG{l+m+mi}{14}\PYG{p}{,}\PYG{l+m+mi}{15}\PYG{p}{]}
\PYG{n}{c2\PYGZus{}list} \PYG{o}{=} \PYG{p}{[}\PYG{l+s+s1}{\PYGZsq{}}\PYG{l+s+s1}{a}\PYG{l+s+s1}{\PYGZsq{}}\PYG{p}{,}\PYG{l+s+s1}{\PYGZsq{}}\PYG{l+s+s1}{b}\PYG{l+s+s1}{\PYGZsq{}}\PYG{p}{,}\PYG{l+s+s1}{\PYGZsq{}}\PYG{l+s+s1}{c}\PYG{l+s+s1}{\PYGZsq{}}\PYG{p}{,}\PYG{l+s+s1}{\PYGZsq{}}\PYG{l+s+s1}{d}\PYG{l+s+s1}{\PYGZsq{}}\PYG{p}{,}\PYG{l+s+s1}{\PYGZsq{}}\PYG{l+s+s1}{e}\PYG{l+s+s1}{\PYGZsq{}}\PYG{p}{]}
\PYG{n}{df1} \PYG{o}{=} \PYG{n}{pd}\PYG{o}{.}\PYG{n}{DataFrame}\PYG{p}{(}\PYG{p}{\PYGZob{}}\PYG{l+s+s1}{\PYGZsq{}}\PYG{l+s+s1}{c1}\PYG{l+s+s1}{\PYGZsq{}}\PYG{p}{:} \PYG{n}{c1\PYGZus{}list}\PYG{p}{,} \PYG{l+s+s1}{\PYGZsq{}}\PYG{l+s+s1}{c2}\PYG{l+s+s1}{\PYGZsq{}}\PYG{p}{:} \PYG{n}{c2\PYGZus{}list}\PYG{p}{\PYGZcb{}}\PYG{p}{)}
\PYG{n+nb}{print}\PYG{p}{(}\PYG{n}{df1}\PYG{p}{)}

\PYG{n+nb}{print}\PYG{p}{(}\PYG{l+s+s1}{\PYGZsq{}}\PYG{l+s+se}{\PYGZbs{}n}\PYG{l+s+s1}{\PYGZsq{}}\PYG{p}{)}
\PYG{n}{df2} \PYG{o}{=} \PYG{n}{pd}\PYG{o}{.}\PYG{n}{DataFrame}\PYG{p}{(}\PYG{p}{\PYGZob{}}\PYG{l+s+s1}{\PYGZsq{}}\PYG{l+s+s1}{c1}\PYG{l+s+s1}{\PYGZsq{}}\PYG{p}{:} \PYG{n}{c1\PYGZus{}list}\PYG{p}{,} \PYG{l+s+s1}{\PYGZsq{}}\PYG{l+s+s1}{c2}\PYG{l+s+s1}{\PYGZsq{}}\PYG{p}{:} \PYG{n}{c2\PYGZus{}list}\PYG{p}{\PYGZcb{}}\PYG{p}{,} \PYG{n}{index}\PYG{o}{=}\PYG{p}{[}\PYG{l+s+s1}{\PYGZsq{}}\PYG{l+s+s1}{s1}\PYG{l+s+s1}{\PYGZsq{}}\PYG{p}{,}\PYG{l+s+s1}{\PYGZsq{}}\PYG{l+s+s1}{s2}\PYG{l+s+s1}{\PYGZsq{}}\PYG{p}{,}\PYG{l+s+s1}{\PYGZsq{}}\PYG{l+s+s1}{s3}\PYG{l+s+s1}{\PYGZsq{}}\PYG{p}{,}\PYG{l+s+s1}{\PYGZsq{}}\PYG{l+s+s1}{s4}\PYG{l+s+s1}{\PYGZsq{}}\PYG{p}{,}\PYG{l+s+s1}{\PYGZsq{}}\PYG{l+s+s1}{s4}\PYG{l+s+s1}{\PYGZsq{}}\PYG{p}{]}\PYG{p}{)}
\PYG{n+nb}{print}\PYG{p}{(}\PYG{n}{df2}\PYG{p}{)}

\PYG{n+nb}{print}\PYG{p}{(}\PYG{l+s+s1}{\PYGZsq{}}\PYG{l+s+se}{\PYGZbs{}n}\PYG{l+s+s1}{\PYGZsq{}}\PYG{p}{)}
\PYG{n+nb}{print}\PYG{p}{(}\PYG{n}{df2}\PYG{o}{.}\PYG{n}{loc}\PYG{p}{[}\PYG{l+s+s1}{\PYGZsq{}}\PYG{l+s+s1}{s3}\PYG{l+s+s1}{\PYGZsq{}}\PYG{p}{]}\PYG{p}{[}\PYG{l+s+s1}{\PYGZsq{}}\PYG{l+s+s1}{c1}\PYG{l+s+s1}{\PYGZsq{}}\PYG{p}{]}\PYG{p}{)} \PYG{c+c1}{\PYGZsh{} 13 출력}
\PYG{n+nb}{print}\PYG{p}{(}\PYG{n}{df2}\PYG{o}{.}\PYG{n}{loc}\PYG{p}{[}\PYG{l+s+s1}{\PYGZsq{}}\PYG{l+s+s1}{s3}\PYG{l+s+s1}{\PYGZsq{}}\PYG{p}{]}\PYG{p}{[}\PYG{p}{[}\PYG{l+s+s1}{\PYGZsq{}}\PYG{l+s+s1}{c1}\PYG{l+s+s1}{\PYGZsq{}}\PYG{p}{,}\PYG{l+s+s1}{\PYGZsq{}}\PYG{l+s+s1}{c2}\PYG{l+s+s1}{\PYGZsq{}}\PYG{p}{]}\PYG{p}{]}\PYG{p}{)} \PYG{c+c1}{\PYGZsh{} 13 과 c 출력}
\end{sphinxVerbatim}

\end{sphinxuseclass}\end{sphinxVerbatimInput}
\begin{sphinxVerbatimOutput}

\begin{sphinxuseclass}{cell_output}
\begin{sphinxVerbatim}[commandchars=\\\{\}]
   c1 c2
0  11  a
1  12  b
2  13  c
3  14  d
4  15  e


    c1 c2
s1  11  a
s2  12  b
s3  13  c
s4  14  d
s4  15  e


13
c1    13
c2     c
Name: s3, dtype: object
\end{sphinxVerbatim}

\end{sphinxuseclass}\end{sphinxVerbatimOutput}

\end{sphinxuseclass}

\section{Index 생성 및 추출}
\label{\detokenize{chapter2/2.1.2_Python_Basics:id2}}
\sphinxAtStartPar
set\_index 메소드로 기존의 column 을 index 로 만들 수 있습니다. set\_index(‘c2’) 처리 후, df2 를 출력하시면 df1 의 ‘c2’ 컬럼이 index 로 되어 있음을 확인할 수 있습니다.
이제 df2 의 index 값을 변경해 보겠습니다. 아래와 같이 DataFrame 의 Index를 호출하여 원하는 Index 로 교체도 가능합니다. 참고로 아래 df2 는 column 하나지만 현재 Series 가 아닌 DataFrame 입니다.

\begin{sphinxuseclass}{cell}\begin{sphinxVerbatimInput}

\begin{sphinxuseclass}{cell_input}
\begin{sphinxVerbatim}[commandchars=\\\{\}]
\PYG{n}{c1\PYGZus{}list} \PYG{o}{=} \PYG{p}{[}\PYG{l+m+mi}{11}\PYG{p}{,}\PYG{l+m+mi}{12}\PYG{p}{,}\PYG{l+m+mi}{13}\PYG{p}{,}\PYG{l+m+mi}{14}\PYG{p}{,}\PYG{l+m+mi}{15}\PYG{p}{]}
\PYG{n}{c2\PYGZus{}list} \PYG{o}{=} \PYG{p}{[}\PYG{l+s+s1}{\PYGZsq{}}\PYG{l+s+s1}{a}\PYG{l+s+s1}{\PYGZsq{}}\PYG{p}{,}\PYG{l+s+s1}{\PYGZsq{}}\PYG{l+s+s1}{b}\PYG{l+s+s1}{\PYGZsq{}}\PYG{p}{,}\PYG{l+s+s1}{\PYGZsq{}}\PYG{l+s+s1}{c}\PYG{l+s+s1}{\PYGZsq{}}\PYG{p}{,}\PYG{l+s+s1}{\PYGZsq{}}\PYG{l+s+s1}{d}\PYG{l+s+s1}{\PYGZsq{}}\PYG{p}{,}\PYG{l+s+s1}{\PYGZsq{}}\PYG{l+s+s1}{e}\PYG{l+s+s1}{\PYGZsq{}}\PYG{p}{]}
\PYG{n}{df1} \PYG{o}{=} \PYG{n}{pd}\PYG{o}{.}\PYG{n}{DataFrame}\PYG{p}{(}\PYG{p}{\PYGZob{}}\PYG{l+s+s1}{\PYGZsq{}}\PYG{l+s+s1}{c1}\PYG{l+s+s1}{\PYGZsq{}}\PYG{p}{:} \PYG{n}{c1\PYGZus{}list}\PYG{p}{,} \PYG{l+s+s1}{\PYGZsq{}}\PYG{l+s+s1}{c2}\PYG{l+s+s1}{\PYGZsq{}}\PYG{p}{:} \PYG{n}{c2\PYGZus{}list}\PYG{p}{\PYGZcb{}}\PYG{p}{)}
\PYG{n+nb}{print}\PYG{p}{(}\PYG{n}{df1}\PYG{p}{)}        

\PYG{n}{df2} \PYG{o}{=} \PYG{n}{df1}\PYG{o}{.}\PYG{n}{set\PYGZus{}index}\PYG{p}{(}\PYG{l+s+s1}{\PYGZsq{}}\PYG{l+s+s1}{c2}\PYG{l+s+s1}{\PYGZsq{}}\PYG{p}{)} \PYG{c+c1}{\PYGZsh{} c1 를 index 로 변경}
\PYG{n+nb}{print}\PYG{p}{(}\PYG{n}{df2}\PYG{p}{)}

\PYG{n+nb}{print}\PYG{p}{(}\PYG{l+s+s1}{\PYGZsq{}}\PYG{l+s+se}{\PYGZbs{}n}\PYG{l+s+s1}{\PYGZsq{}}\PYG{p}{)}
\PYG{n}{df2}\PYG{o}{.}\PYG{n}{index} \PYG{o}{=} \PYG{p}{[}\PYG{l+s+s1}{\PYGZsq{}}\PYG{l+s+s1}{ss1}\PYG{l+s+s1}{\PYGZsq{}}\PYG{p}{,} \PYG{l+s+s1}{\PYGZsq{}}\PYG{l+s+s1}{ss2}\PYG{l+s+s1}{\PYGZsq{}}\PYG{p}{,} \PYG{l+s+s1}{\PYGZsq{}}\PYG{l+s+s1}{ss3}\PYG{l+s+s1}{\PYGZsq{}}\PYG{p}{,} \PYG{l+s+s1}{\PYGZsq{}}\PYG{l+s+s1}{ss4}\PYG{l+s+s1}{\PYGZsq{}}\PYG{p}{,} \PYG{l+s+s1}{\PYGZsq{}}\PYG{l+s+s1}{ss5}\PYG{l+s+s1}{\PYGZsq{}}\PYG{p}{]}
\PYG{n+nb}{print}\PYG{p}{(}\PYG{n}{df2}\PYG{p}{,} \PYG{n+nb}{type}\PYG{p}{(}\PYG{n}{df2}\PYG{p}{)}\PYG{p}{)}
\end{sphinxVerbatim}

\end{sphinxuseclass}\end{sphinxVerbatimInput}
\begin{sphinxVerbatimOutput}

\begin{sphinxuseclass}{cell_output}
\begin{sphinxVerbatim}[commandchars=\\\{\}]
   c1 c2
0  11  a
1  12  b
2  13  c
3  14  d
4  15  e
    c1
c2    
a   11
b   12
c   13
d   14
e   15


     c1
ss1  11
ss2  12
ss3  13
ss4  14
ss5  15 \PYGZlt{}class \PYGZsq{}pandas.core.frame.DataFrame\PYGZsq{}\PYGZgt{}
\end{sphinxVerbatim}

\end{sphinxuseclass}\end{sphinxVerbatimOutput}

\end{sphinxuseclass}
\sphinxAtStartPar
 항상 두 데이터셋을 index 로 병합할 때는 index 에 중복이 있는지 확인을 할 필요가 있습니다. index 가 중복 여부를 체크하는 인수는 verify\_integriry 입니다. 아래는 중복이 있는 경우 에러를 발생시킵니다.

\begin{sphinxuseclass}{cell}\begin{sphinxVerbatimInput}

\begin{sphinxuseclass}{cell_input}
\begin{sphinxVerbatim}[commandchars=\\\{\}]
\PYG{n}{c1\PYGZus{}list} \PYG{o}{=} \PYG{p}{[}\PYG{l+m+mi}{11}\PYG{p}{,}\PYG{l+m+mi}{12}\PYG{p}{,}\PYG{l+m+mi}{13}\PYG{p}{,}\PYG{l+m+mi}{14}\PYG{p}{,}\PYG{l+m+mi}{15}\PYG{p}{]}
\PYG{n}{c2\PYGZus{}list} \PYG{o}{=} \PYG{p}{[}\PYG{l+s+s1}{\PYGZsq{}}\PYG{l+s+s1}{a}\PYG{l+s+s1}{\PYGZsq{}}\PYG{p}{,}\PYG{l+s+s1}{\PYGZsq{}}\PYG{l+s+s1}{a}\PYG{l+s+s1}{\PYGZsq{}}\PYG{p}{,}\PYG{l+s+s1}{\PYGZsq{}}\PYG{l+s+s1}{b}\PYG{l+s+s1}{\PYGZsq{}}\PYG{p}{,}\PYG{l+s+s1}{\PYGZsq{}}\PYG{l+s+s1}{c}\PYG{l+s+s1}{\PYGZsq{}}\PYG{p}{,}\PYG{l+s+s1}{\PYGZsq{}}\PYG{l+s+s1}{d}\PYG{l+s+s1}{\PYGZsq{}}\PYG{p}{]} \PYG{c+c1}{\PYGZsh{} 값에 중복이 있음}
\PYG{n}{df} \PYG{o}{=} \PYG{n}{pd}\PYG{o}{.}\PYG{n}{DataFrame}\PYG{p}{(}\PYG{p}{\PYGZob{}}\PYG{l+s+s1}{\PYGZsq{}}\PYG{l+s+s1}{c1}\PYG{l+s+s1}{\PYGZsq{}}\PYG{p}{:} \PYG{n}{c1\PYGZus{}list}\PYG{p}{,} \PYG{l+s+s1}{\PYGZsq{}}\PYG{l+s+s1}{c2}\PYG{l+s+s1}{\PYGZsq{}}\PYG{p}{:} \PYG{n}{c2\PYGZus{}list}\PYG{p}{\PYGZcb{}}\PYG{p}{)}
\PYG{n}{df}\PYG{o}{.}\PYG{n}{set\PYGZus{}index}\PYG{p}{(}\PYG{l+s+s1}{\PYGZsq{}}\PYG{l+s+s1}{c2}\PYG{l+s+s1}{\PYGZsq{}}\PYG{p}{,} \PYG{n}{verify\PYGZus{}integrity}\PYG{o}{=}\PYG{k+kc}{True}\PYG{p}{)} \PYG{c+c1}{\PYGZsh{} index 중복여부를 체크}
\end{sphinxVerbatim}

\end{sphinxuseclass}\end{sphinxVerbatimInput}
\begin{sphinxVerbatimOutput}

\begin{sphinxuseclass}{cell_output}
\begin{sphinxVerbatim}[commandchars=\\\{\}]
\PYG{g+gt}{\PYGZhy{}\PYGZhy{}\PYGZhy{}\PYGZhy{}\PYGZhy{}\PYGZhy{}\PYGZhy{}\PYGZhy{}\PYGZhy{}\PYGZhy{}\PYGZhy{}\PYGZhy{}\PYGZhy{}\PYGZhy{}\PYGZhy{}\PYGZhy{}\PYGZhy{}\PYGZhy{}\PYGZhy{}\PYGZhy{}\PYGZhy{}\PYGZhy{}\PYGZhy{}\PYGZhy{}\PYGZhy{}\PYGZhy{}\PYGZhy{}\PYGZhy{}\PYGZhy{}\PYGZhy{}\PYGZhy{}\PYGZhy{}\PYGZhy{}\PYGZhy{}\PYGZhy{}\PYGZhy{}\PYGZhy{}\PYGZhy{}\PYGZhy{}\PYGZhy{}\PYGZhy{}\PYGZhy{}\PYGZhy{}\PYGZhy{}\PYGZhy{}\PYGZhy{}\PYGZhy{}\PYGZhy{}\PYGZhy{}\PYGZhy{}\PYGZhy{}\PYGZhy{}\PYGZhy{}\PYGZhy{}\PYGZhy{}\PYGZhy{}\PYGZhy{}\PYGZhy{}\PYGZhy{}\PYGZhy{}\PYGZhy{}\PYGZhy{}\PYGZhy{}\PYGZhy{}\PYGZhy{}\PYGZhy{}\PYGZhy{}\PYGZhy{}\PYGZhy{}\PYGZhy{}\PYGZhy{}\PYGZhy{}\PYGZhy{}\PYGZhy{}\PYGZhy{}}
\PYG{n+ne}{ValueError}\PYG{g+gWhitespace}{                                }Traceback (most recent call last)
\PYG{o}{\PYGZti{}}\PYGZbs{}\PYG{n}{AppData}\PYGZbs{}\PYG{n}{Local}\PYGZbs{}\PYG{n}{Temp}\PYGZbs{}\PYG{n}{ipykernel\PYGZus{}3484}\PYGZbs{}\PYG{l+m+mf}{2078573242.}\PYG{n}{py} \PYG{o+ow}{in} \PYG{o}{\PYGZlt{}}\PYG{n}{module}\PYG{o}{\PYGZgt{}}
\PYG{g+gWhitespace}{      }\PYG{l+m+mi}{2} \PYG{n}{c2\PYGZus{}list} \PYG{o}{=} \PYG{p}{[}\PYG{l+s+s1}{\PYGZsq{}}\PYG{l+s+s1}{a}\PYG{l+s+s1}{\PYGZsq{}}\PYG{p}{,}\PYG{l+s+s1}{\PYGZsq{}}\PYG{l+s+s1}{a}\PYG{l+s+s1}{\PYGZsq{}}\PYG{p}{,}\PYG{l+s+s1}{\PYGZsq{}}\PYG{l+s+s1}{b}\PYG{l+s+s1}{\PYGZsq{}}\PYG{p}{,}\PYG{l+s+s1}{\PYGZsq{}}\PYG{l+s+s1}{c}\PYG{l+s+s1}{\PYGZsq{}}\PYG{p}{,}\PYG{l+s+s1}{\PYGZsq{}}\PYG{l+s+s1}{d}\PYG{l+s+s1}{\PYGZsq{}}\PYG{p}{]} \PYG{c+c1}{\PYGZsh{} 값에 중복이 있음}
\PYG{g+gWhitespace}{      }\PYG{l+m+mi}{3} \PYG{n}{df} \PYG{o}{=} \PYG{n}{pd}\PYG{o}{.}\PYG{n}{DataFrame}\PYG{p}{(}\PYG{p}{\PYGZob{}}\PYG{l+s+s1}{\PYGZsq{}}\PYG{l+s+s1}{c1}\PYG{l+s+s1}{\PYGZsq{}}\PYG{p}{:} \PYG{n}{c1\PYGZus{}list}\PYG{p}{,} \PYG{l+s+s1}{\PYGZsq{}}\PYG{l+s+s1}{c2}\PYG{l+s+s1}{\PYGZsq{}}\PYG{p}{:} \PYG{n}{c2\PYGZus{}list}\PYG{p}{\PYGZcb{}}\PYG{p}{)}
\PYG{n+ne}{\PYGZhy{}\PYGZhy{}\PYGZhy{}\PYGZhy{}\PYGZgt{} }\PYG{l+m+mi}{4} \PYG{n}{df}\PYG{o}{.}\PYG{n}{set\PYGZus{}index}\PYG{p}{(}\PYG{l+s+s1}{\PYGZsq{}}\PYG{l+s+s1}{c2}\PYG{l+s+s1}{\PYGZsq{}}\PYG{p}{,} \PYG{n}{verify\PYGZus{}integrity}\PYG{o}{=}\PYG{k+kc}{True}\PYG{p}{)} \PYG{c+c1}{\PYGZsh{} index 중복여부를 체크}

\PYG{n+nn}{\PYGZti{}\PYGZbs{}Anaconda3\PYGZbs{}lib\PYGZbs{}site\PYGZhy{}packages\PYGZbs{}pandas\PYGZbs{}util\PYGZbs{}\PYGZus{}decorators.py} in \PYG{n+ni}{wrapper}\PYG{n+nt}{(*args, **kwargs)}
\PYG{g+gWhitespace}{    }\PYG{l+m+mi}{309}                     \PYG{n}{stacklevel}\PYG{o}{=}\PYG{n}{stacklevel}\PYG{p}{,}
\PYG{g+gWhitespace}{    }\PYG{l+m+mi}{310}                 \PYG{p}{)}
\PYG{n+ne}{\PYGZhy{}\PYGZhy{}\PYGZgt{} }\PYG{l+m+mi}{311}             \PYG{k}{return} \PYG{n}{func}\PYG{p}{(}\PYG{o}{*}\PYG{n}{args}\PYG{p}{,} \PYG{o}{*}\PYG{o}{*}\PYG{n}{kwargs}\PYG{p}{)}
\PYG{g+gWhitespace}{    }\PYG{l+m+mi}{312} 
\PYG{g+gWhitespace}{    }\PYG{l+m+mi}{313}         \PYG{k}{return} \PYG{n}{wrapper}

\PYG{n+nn}{\PYGZti{}\PYGZbs{}Anaconda3\PYGZbs{}lib\PYGZbs{}site\PYGZhy{}packages\PYGZbs{}pandas\PYGZbs{}core\PYGZbs{}frame.py} in \PYG{n+ni}{set\PYGZus{}index}\PYG{n+nt}{(self, keys, drop, append, inplace, verify\PYGZus{}integrity)}
\PYG{g+gWhitespace}{   }\PYG{l+m+mi}{5508}         \PYG{k}{if} \PYG{n}{verify\PYGZus{}integrity} \PYG{o+ow}{and} \PYG{o+ow}{not} \PYG{n}{index}\PYG{o}{.}\PYG{n}{is\PYGZus{}unique}\PYG{p}{:}
\PYG{g+gWhitespace}{   }\PYG{l+m+mi}{5509}             \PYG{n}{duplicates} \PYG{o}{=} \PYG{n}{index}\PYG{p}{[}\PYG{n}{index}\PYG{o}{.}\PYG{n}{duplicated}\PYG{p}{(}\PYG{p}{)}\PYG{p}{]}\PYG{o}{.}\PYG{n}{unique}\PYG{p}{(}\PYG{p}{)}
\PYG{n+ne}{\PYGZhy{}\PYGZgt{} }\PYG{l+m+mi}{5510}             \PYG{k}{raise} \PYG{n+ne}{ValueError}\PYG{p}{(}\PYG{l+s+sa}{f}\PYG{l+s+s2}{\PYGZdq{}}\PYG{l+s+s2}{Index has duplicate keys: }\PYG{l+s+si}{\PYGZob{}}\PYG{n}{duplicates}\PYG{l+s+si}{\PYGZcb{}}\PYG{l+s+s2}{\PYGZdq{}}\PYG{p}{)}
\PYG{g+gWhitespace}{   }\PYG{l+m+mi}{5511} 
\PYG{g+gWhitespace}{   }\PYG{l+m+mi}{5512}         \PYG{c+c1}{\PYGZsh{} use set to handle duplicate column names gracefully in case of drop}

\PYG{n+ne}{ValueError}: Index has duplicate keys: Index([\PYGZsq{}a\PYGZsq{}], dtype=\PYGZsq{}object\PYGZsq{}, name=\PYGZsq{}c2\PYGZsq{})
\end{sphinxVerbatim}

\end{sphinxuseclass}\end{sphinxVerbatimOutput}

\end{sphinxuseclass}

\section{For Loop}
\label{\detokenize{chapter2/2.1.3_Python_Basics:for-loop}}\label{\detokenize{chapter2/2.1.3_Python_Basics::doc}}
\sphinxAtStartPar
컴퓨터를 잘 활용한다는 것의 컴퓨터의 3 가지 강점 \sphinxhyphen{} 기억, 반복, 계산을 잘 활용한다는 뜻입니다. 그 중에서도 인간보다 탁월한 능력이 바로 반복입니다. 컴퓨터는 수만번, 수천번의 반복도 금방 해 치웁니다. 이번에는 그 반복문을 배우겠습니다. 반복문 중에 for \textasciitilde{} in 구분이 가장 많이 활용됩니다. for \textasciitilde{} in 형식에서 in 다음에 List 를 넣으면 List 의 원소를 순서대로 꺼내어 처리합니다. 단순히 출력만 해보겠습니다. 다음에는 제곱한 값을 출력해 보겠습니다.

\begin{sphinxuseclass}{cell}\begin{sphinxVerbatimInput}

\begin{sphinxuseclass}{cell_input}
\begin{sphinxVerbatim}[commandchars=\\\{\}]
\PYG{n}{num\PYGZus{}list} \PYG{o}{=} \PYG{p}{[}\PYG{l+m+mi}{1}\PYG{p}{,}\PYG{l+m+mi}{2}\PYG{p}{,}\PYG{l+m+mi}{3}\PYG{p}{,}\PYG{l+m+mi}{4}\PYG{p}{,}\PYG{l+m+mi}{5}\PYG{p}{,}\PYG{l+m+mi}{6}\PYG{p}{]}
\PYG{k}{for} \PYG{n}{i} \PYG{o+ow}{in} \PYG{n}{num\PYGZus{}list}\PYG{p}{:}
    \PYG{n+nb}{print}\PYG{p}{(}\PYG{n}{i}\PYG{p}{)}

\PYG{n+nb}{print}\PYG{p}{(}\PYG{l+s+s1}{\PYGZsq{}}\PYG{l+s+se}{\PYGZbs{}n}\PYG{l+s+s1}{\PYGZsq{}}\PYG{p}{)}    
\PYG{k}{for} \PYG{n}{i} \PYG{o+ow}{in} \PYG{n}{num\PYGZus{}list}\PYG{p}{:}
    \PYG{n+nb}{print}\PYG{p}{(}\PYG{n}{i}\PYG{o}{*}\PYG{o}{*}\PYG{l+m+mi}{2}\PYG{p}{)}    
\end{sphinxVerbatim}

\end{sphinxuseclass}\end{sphinxVerbatimInput}
\begin{sphinxVerbatimOutput}

\begin{sphinxuseclass}{cell_output}
\begin{sphinxVerbatim}[commandchars=\\\{\}]
1
2
3
4
5
6


1
4
9
16
25
36
\end{sphinxVerbatim}

\end{sphinxuseclass}\end{sphinxVerbatimOutput}

\end{sphinxuseclass}


\begin{sphinxuseclass}{cell}\begin{sphinxVerbatimInput}

\begin{sphinxuseclass}{cell_input}
\begin{sphinxVerbatim}[commandchars=\\\{\}]
\PYG{c+c1}{\PYGZsh{} break}
\PYG{k}{for} \PYG{n}{i} \PYG{o+ow}{in} \PYG{n}{num\PYGZus{}list}\PYG{p}{:}
    \PYG{k}{if} \PYG{n}{i} \PYG{o}{==} \PYG{l+m+mi}{3}\PYG{p}{:}
        \PYG{k}{break}
    \PYG{n+nb}{print}\PYG{p}{(}\PYG{n}{i}\PYG{o}{*}\PYG{o}{*}\PYG{l+m+mi}{2}\PYG{p}{)}  
\end{sphinxVerbatim}

\end{sphinxuseclass}\end{sphinxVerbatimInput}
\begin{sphinxVerbatimOutput}

\begin{sphinxuseclass}{cell_output}
\begin{sphinxVerbatim}[commandchars=\\\{\}]
1
4
\end{sphinxVerbatim}

\end{sphinxuseclass}\end{sphinxVerbatimOutput}

\end{sphinxuseclass}
\begin{sphinxuseclass}{cell}\begin{sphinxVerbatimInput}

\begin{sphinxuseclass}{cell_input}
\begin{sphinxVerbatim}[commandchars=\\\{\}]
\PYG{c+c1}{\PYGZsh{} continue}
\PYG{k}{for} \PYG{n}{i} \PYG{o+ow}{in} \PYG{n}{num\PYGZus{}list}\PYG{p}{:}
    \PYG{k}{if} \PYG{n}{i} \PYG{o}{==} \PYG{l+m+mi}{3}\PYG{p}{:}
        \PYG{k}{continue}
    \PYG{n+nb}{print}\PYG{p}{(}\PYG{n}{i}\PYG{o}{*}\PYG{o}{*}\PYG{l+m+mi}{2}\PYG{p}{)}    
\end{sphinxVerbatim}

\end{sphinxuseclass}\end{sphinxVerbatimInput}
\begin{sphinxVerbatimOutput}

\begin{sphinxuseclass}{cell_output}
\begin{sphinxVerbatim}[commandchars=\\\{\}]
1
4
16
25
36
\end{sphinxVerbatim}

\end{sphinxuseclass}\end{sphinxVerbatimOutput}

\end{sphinxuseclass}

\section{While Loop}
\label{\detokenize{chapter2/2.1.3_Python_Basics:while-loop}}
\sphinxAtStartPar
While 반복문도 자주 활용됩니다. While 안의 조건이 만족하는 한, 계속 반복합니다. break 문으로 While Loop 를 빠져나올 수 있습니다.

\begin{sphinxuseclass}{cell}\begin{sphinxVerbatimInput}

\begin{sphinxuseclass}{cell_input}
\begin{sphinxVerbatim}[commandchars=\\\{\}]
\PYG{n}{i} \PYG{o}{=} \PYG{l+m+mi}{0}
\PYG{k}{while}\PYG{p}{(}\PYG{k+kc}{True}\PYG{p}{)}\PYG{p}{:}
    \PYG{n}{i} \PYG{o}{=} \PYG{n}{i} \PYG{o}{+} \PYG{l+m+mi}{1}
    \PYG{n+nb}{print}\PYG{p}{(}\PYG{n}{i}\PYG{o}{*}\PYG{o}{*}\PYG{l+m+mi}{2}\PYG{p}{)}
    \PYG{k}{if} \PYG{n}{i} \PYG{o}{==} \PYG{l+m+mi}{10}\PYG{p}{:}
        \PYG{k}{break}
\end{sphinxVerbatim}

\end{sphinxuseclass}\end{sphinxVerbatimInput}
\begin{sphinxVerbatimOutput}

\begin{sphinxuseclass}{cell_output}
\begin{sphinxVerbatim}[commandchars=\\\{\}]
1
4
9
16
25
36
49
64
81
100
\end{sphinxVerbatim}

\end{sphinxuseclass}\end{sphinxVerbatimOutput}

\end{sphinxuseclass}
\begin{sphinxuseclass}{cell}\begin{sphinxVerbatimInput}

\begin{sphinxuseclass}{cell_input}
\begin{sphinxVerbatim}[commandchars=\\\{\}]
\PYG{n}{i} \PYG{o}{=} \PYG{l+m+mi}{0}
\PYG{k}{while}\PYG{p}{(}\PYG{n}{i}\PYG{o}{\PYGZlt{}}\PYG{l+m+mi}{10}\PYG{p}{)}\PYG{p}{:}
    \PYG{n}{i} \PYG{o}{=} \PYG{n}{i} \PYG{o}{+} \PYG{l+m+mi}{1}
    \PYG{n+nb}{print}\PYG{p}{(}\PYG{n}{i}\PYG{o}{*}\PYG{o}{*}\PYG{l+m+mi}{2}\PYG{p}{)}
\end{sphinxVerbatim}

\end{sphinxuseclass}\end{sphinxVerbatimInput}
\begin{sphinxVerbatimOutput}

\begin{sphinxuseclass}{cell_output}
\begin{sphinxVerbatim}[commandchars=\\\{\}]
1
4
9
16
25
36
49
64
81
100
\end{sphinxVerbatim}

\end{sphinxuseclass}\end{sphinxVerbatimOutput}

\end{sphinxuseclass}

\section{If Condition}
\label{\detokenize{chapter2/2.1.4_Python_Basics:if-condition}}\label{\detokenize{chapter2/2.1.4_Python_Basics::doc}}
\sphinxAtStartPar
파이썬의 조건문 if \textasciitilde{} else 형식으로 다른 컴퓨터 언어와 다르지 않습니다. 단지 else if 부분은 줄여서 elif 로 씁니다. 아래 예제를 보시면 쉽게 이해가 되실 것으로 생각합니다.

\begin{sphinxuseclass}{cell}\begin{sphinxVerbatimInput}

\begin{sphinxuseclass}{cell_input}
\begin{sphinxVerbatim}[commandchars=\\\{\}]
\PYG{n}{a} \PYG{o}{=} \PYG{l+m+mi}{3}
\PYG{n}{b} \PYG{o}{=} \PYG{l+m+mi}{2}
\PYG{k}{if} \PYG{n}{a} \PYG{o}{\PYGZgt{}} \PYG{n}{b}\PYG{p}{:}
    \PYG{n+nb}{print}\PYG{p}{(}\PYG{l+s+s1}{\PYGZsq{}}\PYG{l+s+s1}{a \PYGZgt{} b}\PYG{l+s+s1}{\PYGZsq{}}\PYG{p}{)}
\PYG{k}{else}\PYG{p}{:}
    \PYG{n+nb}{print}\PYG{p}{(}\PYG{l+s+s1}{\PYGZsq{}}\PYG{l+s+s1}{a \PYGZlt{}= b}\PYG{l+s+s1}{\PYGZsq{}}\PYG{p}{)}
\end{sphinxVerbatim}

\end{sphinxuseclass}\end{sphinxVerbatimInput}
\begin{sphinxVerbatimOutput}

\begin{sphinxuseclass}{cell_output}
\begin{sphinxVerbatim}[commandchars=\\\{\}]
a \PYGZgt{} b
\end{sphinxVerbatim}

\end{sphinxuseclass}\end{sphinxVerbatimOutput}

\end{sphinxuseclass}
\begin{sphinxuseclass}{cell}\begin{sphinxVerbatimInput}

\begin{sphinxuseclass}{cell_input}
\begin{sphinxVerbatim}[commandchars=\\\{\}]
\PYG{n}{num\PYGZus{}list} \PYG{o}{=} \PYG{p}{[}\PYG{l+m+mi}{1}\PYG{p}{,}\PYG{l+m+mi}{2}\PYG{p}{,}\PYG{l+m+mi}{3}\PYG{p}{,}\PYG{l+m+mi}{4}\PYG{p}{,}\PYG{l+m+mi}{5}\PYG{p}{,}\PYG{l+m+mi}{6}\PYG{p}{]}

\PYG{k}{for} \PYG{n}{i} \PYG{o+ow}{in} \PYG{n}{num\PYGZus{}list}\PYG{p}{:}
    \PYG{k}{if} \PYG{n}{i} \PYG{o}{\PYGZlt{}} \PYG{l+m+mi}{3}\PYG{p}{:}
        \PYG{n+nb}{print}\PYG{p}{(}\PYG{n}{i}\PYG{p}{,} \PYG{l+s+s1}{\PYGZsq{}}\PYG{l+s+s1}{the number is less than 3}\PYG{l+s+s1}{\PYGZsq{}}\PYG{p}{)}
    \PYG{k}{elif} \PYG{n}{i} \PYG{o}{\PYGZgt{}} \PYG{l+m+mi}{3}\PYG{p}{:}
        \PYG{n+nb}{print}\PYG{p}{(}\PYG{n}{i}\PYG{p}{,} \PYG{l+s+s1}{\PYGZsq{}}\PYG{l+s+s1}{The number is greater than 3}\PYG{l+s+s1}{\PYGZsq{}}\PYG{p}{)}
    \PYG{k}{else}\PYG{p}{:}
        \PYG{n+nb}{print}\PYG{p}{(}\PYG{n}{i}\PYG{p}{,} \PYG{l+s+s1}{\PYGZsq{}}\PYG{l+s+s1}{The number is 3}\PYG{l+s+s1}{\PYGZsq{}}\PYG{p}{)}
\end{sphinxVerbatim}

\end{sphinxuseclass}\end{sphinxVerbatimInput}
\begin{sphinxVerbatimOutput}

\begin{sphinxuseclass}{cell_output}
\begin{sphinxVerbatim}[commandchars=\\\{\}]
1 the number is less than 3
2 the number is less than 3
3 The number is 3
4 The number is greater than 3
5 The number is greater than 3
6 The number is greater than 3
\end{sphinxVerbatim}

\end{sphinxuseclass}\end{sphinxVerbatimOutput}

\end{sphinxuseclass}
\begin{sphinxuseclass}{cell}\begin{sphinxVerbatimInput}

\begin{sphinxuseclass}{cell_input}
\begin{sphinxVerbatim}[commandchars=\\\{\}]
\PYG{k}{for} \PYG{n}{i} \PYG{o+ow}{in} \PYG{n}{num\PYGZus{}list}\PYG{p}{:}
    \PYG{k}{if} \PYG{n}{i} \PYG{o}{\PYGZlt{}} \PYG{l+m+mi}{3}\PYG{p}{:}
        \PYG{n+nb}{print}\PYG{p}{(}\PYG{n}{i}\PYG{p}{,} \PYG{l+s+s1}{\PYGZsq{}}\PYG{l+s+s1}{the number is less than 3}\PYG{l+s+s1}{\PYGZsq{}}\PYG{p}{)}
    \PYG{k}{elif} \PYG{n}{i} \PYG{o}{\PYGZgt{}} \PYG{l+m+mi}{3}\PYG{p}{:}
        \PYG{k}{pass}  \PYG{c+c1}{\PYGZsh{} 아무런 처리를 하지 않음}
    \PYG{k}{else}\PYG{p}{:}
        \PYG{n+nb}{print}\PYG{p}{(}\PYG{n}{i}\PYG{p}{,} \PYG{l+s+s1}{\PYGZsq{}}\PYG{l+s+s1}{The number is 3}\PYG{l+s+s1}{\PYGZsq{}}\PYG{p}{)}
\end{sphinxVerbatim}

\end{sphinxuseclass}\end{sphinxVerbatimInput}
\begin{sphinxVerbatimOutput}

\begin{sphinxuseclass}{cell_output}
\begin{sphinxVerbatim}[commandchars=\\\{\}]
1 the number is less than 3
2 the number is less than 3
3 The number is 3
\end{sphinxVerbatim}

\end{sphinxuseclass}\end{sphinxVerbatimOutput}

\end{sphinxuseclass}

\section{Functions}
\label{\detokenize{chapter2/2.1.5_Python_Basics:functions}}\label{\detokenize{chapter2/2.1.5_Python_Basics::doc}}
\sphinxAtStartPar
파이썬의 함수는 def 로 시작하고 결과값을 return 으로 반환합니다. 결과값의 반환은 여러 개도 가능합니다. 단, 함수 호출 후 결과 값을 받을 때, 함수가 return 하는 결과 값 갯수가 동일해야 합니다. 함수도 아래 예제를 보시면 쉽게 이해가 되시리라 생각합니다.

\begin{sphinxuseclass}{cell}\begin{sphinxVerbatimInput}

\begin{sphinxuseclass}{cell_input}
\begin{sphinxVerbatim}[commandchars=\\\{\}]
\PYG{k}{def} \PYG{n+nf}{cal}\PYG{p}{(}\PYG{n}{x}\PYG{p}{,} \PYG{n}{y}\PYG{p}{)}\PYG{p}{:}
    \PYG{n}{z} \PYG{o}{=} \PYG{n}{x} \PYG{o}{+} \PYG{n}{y}
    \PYG{k}{return} \PYG{n}{z}

\PYG{n}{result} \PYG{o}{=} \PYG{n}{cal}\PYG{p}{(}\PYG{l+m+mi}{2}\PYG{p}{,}\PYG{l+m+mi}{3}\PYG{p}{)}
\PYG{n+nb}{print}\PYG{p}{(}\PYG{n}{result}\PYG{p}{)}
\end{sphinxVerbatim}

\end{sphinxuseclass}\end{sphinxVerbatimInput}
\begin{sphinxVerbatimOutput}

\begin{sphinxuseclass}{cell_output}
\begin{sphinxVerbatim}[commandchars=\\\{\}]
5
\end{sphinxVerbatim}

\end{sphinxuseclass}\end{sphinxVerbatimOutput}

\end{sphinxuseclass}
\begin{sphinxuseclass}{cell}\begin{sphinxVerbatimInput}

\begin{sphinxuseclass}{cell_input}
\begin{sphinxVerbatim}[commandchars=\\\{\}]
\PYG{k}{def} \PYG{n+nf}{cal}\PYG{p}{(}\PYG{n}{x}\PYG{p}{,} \PYG{n}{y}\PYG{p}{)}\PYG{p}{:}
    \PYG{n}{z1} \PYG{o}{=} \PYG{n}{x} \PYG{o}{+} \PYG{n}{y}
    \PYG{n}{z2} \PYG{o}{=} \PYG{n}{x}\PYG{o}{*}\PYG{n}{y}
    \PYG{k}{return} \PYG{n}{z1}\PYG{p}{,} \PYG{n}{z2}

\PYG{n}{result1}\PYG{p}{,} \PYG{n}{result2} \PYG{o}{=} \PYG{n}{cal}\PYG{p}{(}\PYG{l+m+mi}{2}\PYG{p}{,}\PYG{l+m+mi}{3}\PYG{p}{)}
\PYG{n+nb}{print}\PYG{p}{(}\PYG{n}{result1}\PYG{p}{,} \PYG{n}{result2}\PYG{p}{)}
\end{sphinxVerbatim}

\end{sphinxuseclass}\end{sphinxVerbatimInput}
\begin{sphinxVerbatimOutput}

\begin{sphinxuseclass}{cell_output}
\begin{sphinxVerbatim}[commandchars=\\\{\}]
5 6
\end{sphinxVerbatim}

\end{sphinxuseclass}\end{sphinxVerbatimOutput}

\end{sphinxuseclass}
\begin{sphinxuseclass}{cell}\begin{sphinxVerbatimInput}

\begin{sphinxuseclass}{cell_input}
\begin{sphinxVerbatim}[commandchars=\\\{\}]
\PYG{k}{def} \PYG{n+nf}{cal}\PYG{p}{(}\PYG{n}{x}\PYG{p}{,} \PYG{n}{y}\PYG{p}{)}\PYG{p}{:}
    \PYG{k}{return}  \PYG{p}{(}\PYG{n}{x}\PYG{o}{+}\PYG{n}{y}\PYG{p}{)}\PYG{p}{,} \PYG{p}{(}\PYG{n}{x}\PYG{o}{*}\PYG{n}{y}\PYG{p}{)}\PYG{p}{,} \PYG{p}{(}\PYG{n}{x}\PYG{o}{*}\PYG{o}{*}\PYG{n}{y}\PYG{p}{)}

\PYG{n}{result1}\PYG{p}{,} \PYG{n}{result2}\PYG{p}{,} \PYG{n}{result3} \PYG{o}{=} \PYG{n}{cal}\PYG{p}{(}\PYG{l+m+mi}{2}\PYG{p}{,}\PYG{l+m+mi}{3}\PYG{p}{)}
\PYG{n+nb}{print}\PYG{p}{(}\PYG{n}{result1}\PYG{p}{,} \PYG{n}{result2}\PYG{p}{,} \PYG{n}{result3}\PYG{p}{)}
\end{sphinxVerbatim}

\end{sphinxuseclass}\end{sphinxVerbatimInput}
\begin{sphinxVerbatimOutput}

\begin{sphinxuseclass}{cell_output}
\begin{sphinxVerbatim}[commandchars=\\\{\}]
5 6 8
\end{sphinxVerbatim}

\end{sphinxuseclass}\end{sphinxVerbatimOutput}

\end{sphinxuseclass}

\chapter{\sphinxstylestrong{유용한 기능들}}
\label{\detokenize{chapter2/2.2.0_Useful_Techniques:id1}}\label{\detokenize{chapter2/2.2.0_Useful_Techniques::doc}}
\sphinxAtStartPar
이번 장에서는 유용한 데이터 핸들링 방법들을 배우겠습니다. 데이터 분석을 통해서 원하는 결과를 얻기 위해서는, 우선 분석이 가능한 형태의 데이터로 가공을 해야합니다.


\section{Append}
\label{\detokenize{chapter2/2.2.1_Useful_Techniques:append}}\label{\detokenize{chapter2/2.2.1_Useful_Techniques::doc}}
\sphinxAtStartPar
append 는 반복문에서 발생하는 값을 순차적으로 모으는데 유용합니다. 아래 예제는 반복문에서 추출된 원소를 제곱한 값을 계속 v\_list 리스트에 추가하는 코드입니다.

\begin{sphinxuseclass}{cell}\begin{sphinxVerbatimInput}

\begin{sphinxuseclass}{cell_input}
\begin{sphinxVerbatim}[commandchars=\\\{\}]
\PYG{n}{v\PYGZus{}list} \PYG{o}{=} \PYG{p}{[}\PYG{p}{]}

\PYG{n}{aa} \PYG{o}{=} \PYG{p}{[}\PYG{l+m+mi}{1}\PYG{p}{,} \PYG{l+m+mi}{2}\PYG{p}{,} \PYG{l+m+mi}{3}\PYG{p}{,} \PYG{l+m+mi}{4}\PYG{p}{,} \PYG{l+m+mi}{5}\PYG{p}{]}

\PYG{k}{for} \PYG{n}{a} \PYG{o+ow}{in} \PYG{n}{aa}\PYG{p}{:}
    \PYG{n}{v\PYGZus{}list}\PYG{o}{.}\PYG{n}{append}\PYG{p}{(}\PYG{n}{a}\PYG{o}{*}\PYG{o}{*}\PYG{l+m+mi}{2}\PYG{p}{)}
    
\PYG{n+nb}{print}\PYG{p}{(}\PYG{n}{v\PYGZus{}list}\PYG{p}{)}    
\end{sphinxVerbatim}

\end{sphinxuseclass}\end{sphinxVerbatimInput}
\begin{sphinxVerbatimOutput}

\begin{sphinxuseclass}{cell_output}
\begin{sphinxVerbatim}[commandchars=\\\{\}]
[1, 4, 9, 16, 25]
\end{sphinxVerbatim}

\end{sphinxuseclass}\end{sphinxVerbatimOutput}

\end{sphinxuseclass}


\begin{sphinxuseclass}{cell}\begin{sphinxVerbatimInput}

\begin{sphinxuseclass}{cell_input}
\begin{sphinxVerbatim}[commandchars=\\\{\}]
\PYG{k+kn}{import} \PYG{n+nn}{pandas} \PYG{k}{as} \PYG{n+nn}{pd}

\PYG{n}{r\PYGZus{}list} \PYG{o}{=} \PYG{p}{[}\PYG{p}{]}

\PYG{n}{c1\PYGZus{}list} \PYG{o}{=} \PYG{p}{[}\PYG{l+m+mi}{11}\PYG{p}{,}\PYG{l+m+mi}{12}\PYG{p}{,}\PYG{l+m+mi}{13}\PYG{p}{,}\PYG{l+m+mi}{14}\PYG{p}{,}\PYG{l+m+mi}{15}\PYG{p}{]}
\PYG{n}{c2\PYGZus{}list} \PYG{o}{=} \PYG{p}{[}\PYG{l+s+s1}{\PYGZsq{}}\PYG{l+s+s1}{a}\PYG{l+s+s1}{\PYGZsq{}}\PYG{p}{,}\PYG{l+s+s1}{\PYGZsq{}}\PYG{l+s+s1}{b}\PYG{l+s+s1}{\PYGZsq{}}\PYG{p}{,}\PYG{l+s+s1}{\PYGZsq{}}\PYG{l+s+s1}{c}\PYG{l+s+s1}{\PYGZsq{}}\PYG{p}{,}\PYG{l+s+s1}{\PYGZsq{}}\PYG{l+s+s1}{d}\PYG{l+s+s1}{\PYGZsq{}}\PYG{p}{,}\PYG{l+s+s1}{\PYGZsq{}}\PYG{l+s+s1}{e}\PYG{l+s+s1}{\PYGZsq{}}\PYG{p}{]}

\PYG{n}{df1} \PYG{o}{=} \PYG{n}{pd}\PYG{o}{.}\PYG{n}{DataFrame}\PYG{p}{(}\PYG{p}{\PYGZob{}}\PYG{l+s+s1}{\PYGZsq{}}\PYG{l+s+s1}{c1}\PYG{l+s+s1}{\PYGZsq{}}\PYG{p}{:} \PYG{n}{c1\PYGZus{}list}\PYG{p}{,} \PYG{l+s+s1}{\PYGZsq{}}\PYG{l+s+s1}{c2}\PYG{l+s+s1}{\PYGZsq{}}\PYG{p}{:} \PYG{n}{c2\PYGZus{}list}\PYG{p}{\PYGZcb{}}\PYG{p}{)}

\PYG{k}{for} \PYG{n}{i} \PYG{o+ow}{in} \PYG{n+nb}{list}\PYG{p}{(}\PYG{n}{df1}\PYG{p}{[}\PYG{l+s+s1}{\PYGZsq{}}\PYG{l+s+s1}{c1}\PYG{l+s+s1}{\PYGZsq{}}\PYG{p}{]}\PYG{p}{)}\PYG{p}{:} \PYG{c+c1}{\PYGZsh{} List 함수가 꼭 필요하지는 않음  df1[\PYGZsq{}c1\PYGZsq{}] =\PYGZgt{} [11,12,13,14,15]}
    \PYG{n}{r\PYGZus{}list}\PYG{o}{.}\PYG{n}{append}\PYG{p}{(}\PYG{n}{i}\PYG{o}{*}\PYG{o}{*}\PYG{l+m+mi}{2}\PYG{p}{)}
    
\PYG{n}{df1}\PYG{p}{[}\PYG{l+s+s1}{\PYGZsq{}}\PYG{l+s+s1}{c3}\PYG{l+s+s1}{\PYGZsq{}}\PYG{p}{]} \PYG{o}{=} \PYG{n}{r\PYGZus{}list} \PYG{c+c1}{\PYGZsh{} r\PYGZus{}list 갯수와 df1 갯수가 동일해야 함}

\PYG{n+nb}{print}\PYG{p}{(}\PYG{n}{df1}\PYG{p}{)}
\end{sphinxVerbatim}

\end{sphinxuseclass}\end{sphinxVerbatimInput}
\begin{sphinxVerbatimOutput}

\begin{sphinxuseclass}{cell_output}
\begin{sphinxVerbatim}[commandchars=\\\{\}]
   c1 c2   c3
0  11  a  121
1  12  b  144
2  13  c  169
3  14  d  196
4  15  e  225
\end{sphinxVerbatim}

\end{sphinxuseclass}\end{sphinxVerbatimOutput}

\end{sphinxuseclass}


\begin{sphinxuseclass}{cell}\begin{sphinxVerbatimInput}

\begin{sphinxuseclass}{cell_input}
\begin{sphinxVerbatim}[commandchars=\\\{\}]
\PYG{k+kn}{import} \PYG{n+nn}{pandas} \PYG{k}{as} \PYG{n+nn}{pd}

\PYG{n}{c1\PYGZus{}list} \PYG{o}{=} \PYG{p}{[}\PYG{l+m+mi}{11}\PYG{p}{,}\PYG{l+m+mi}{12}\PYG{p}{,}\PYG{l+m+mi}{13}\PYG{p}{,}\PYG{l+m+mi}{14}\PYG{p}{,}\PYG{l+m+mi}{15}\PYG{p}{]}
\PYG{n}{c2\PYGZus{}list} \PYG{o}{=} \PYG{p}{[}\PYG{l+s+s1}{\PYGZsq{}}\PYG{l+s+s1}{a}\PYG{l+s+s1}{\PYGZsq{}}\PYG{p}{,}\PYG{l+s+s1}{\PYGZsq{}}\PYG{l+s+s1}{b}\PYG{l+s+s1}{\PYGZsq{}}\PYG{p}{,}\PYG{l+s+s1}{\PYGZsq{}}\PYG{l+s+s1}{c}\PYG{l+s+s1}{\PYGZsq{}}\PYG{p}{,}\PYG{l+s+s1}{\PYGZsq{}}\PYG{l+s+s1}{d}\PYG{l+s+s1}{\PYGZsq{}}\PYG{p}{,}\PYG{l+s+s1}{\PYGZsq{}}\PYG{l+s+s1}{e}\PYG{l+s+s1}{\PYGZsq{}}\PYG{p}{]}
\PYG{n}{df1} \PYG{o}{=} \PYG{n}{pd}\PYG{o}{.}\PYG{n}{DataFrame}\PYG{p}{(}\PYG{p}{\PYGZob{}}\PYG{l+s+s1}{\PYGZsq{}}\PYG{l+s+s1}{c1}\PYG{l+s+s1}{\PYGZsq{}}\PYG{p}{:} \PYG{n}{c1\PYGZus{}list}\PYG{p}{,} \PYG{l+s+s1}{\PYGZsq{}}\PYG{l+s+s1}{c2}\PYG{l+s+s1}{\PYGZsq{}}\PYG{p}{:} \PYG{n}{c2\PYGZus{}list}\PYG{p}{\PYGZcb{}}\PYG{p}{)}

\PYG{n}{df1}\PYG{p}{[}\PYG{l+s+s1}{\PYGZsq{}}\PYG{l+s+s1}{c3}\PYG{l+s+s1}{\PYGZsq{}}\PYG{p}{]} \PYG{o}{=} \PYG{n}{df1}\PYG{p}{[}\PYG{l+s+s1}{\PYGZsq{}}\PYG{l+s+s1}{c1}\PYG{l+s+s1}{\PYGZsq{}}\PYG{p}{]}\PYG{o}{*}\PYG{o}{*}\PYG{l+m+mi}{2}
\PYG{n+nb}{print}\PYG{p}{(}\PYG{n}{df1}\PYG{p}{)}
\end{sphinxVerbatim}

\end{sphinxuseclass}\end{sphinxVerbatimInput}
\begin{sphinxVerbatimOutput}

\begin{sphinxuseclass}{cell_output}
\begin{sphinxVerbatim}[commandchars=\\\{\}]
   c1 c2   c3
0  11  a  121
1  12  b  144
2  13  c  169
3  14  d  196
4  15  e  225
\end{sphinxVerbatim}

\end{sphinxuseclass}\end{sphinxVerbatimOutput}

\end{sphinxuseclass}

\section{Concat 과 Merge}
\label{\detokenize{chapter2/2.2.2_Useful_Techniques:concat-merge}}\label{\detokenize{chapter2/2.2.2_Useful_Techniques::doc}}
\sphinxAtStartPar
Concat 과 Merge 는 두개 이상의 DataFrame/Series 을 키 값(매칭을 위한 값)으로 합칠 때 쓰는 메소드입니다.


\section{Concat}
\label{\detokenize{chapter2/2.2.2_Useful_Techniques:concat}}
\sphinxAtStartPar
먼저 concat 를 해 보겠습니다. concat 는 axis 라는 인수를 사용해서 위\sphinxhyphen{}아래로 합할 것인지, 좌\sphinxhyphen{}우로 합할 것인지 알려줍니다. 좌\sphinxhyphen{}우로 합치는 경우는 index(행) 를 기준으로 하고, 위\sphinxhyphen{}아래로 합치는 경우는 column(열) 을 기준으로 합니다. 다른 인수로는 join 이 있습니다. 주로 axis=1 로 병합(좌\sphinxhyphen{}우)하는 경우가 많은데요. 양쪽 데이터셋에 동시에 존재하는 index 만으로 합칠 때는 join=’inner’ 를 넣어주고, 모든 index 를 남기고 싶을 때는 join=’outer’ 를 넣어줍니다. 아래 예제에서 Series s1 과 Series s2 의 index 가 동일하므로, ‘inner’ 나 ‘outer’ 로 합쳐도 동일한 결과가 나옵니다. 두개의 데이터셋이 같은 지 체크하는 메소드는 equal 입니다. 체크한 결과 True 를 얻었습니다. 참고로 함수의 ()안에 커서를 놓고, ‘Shift+Tab’ 를 하면 활용 가능한 모든 인수와 설명이 나옵니다.

\begin{sphinxuseclass}{cell}\begin{sphinxVerbatimInput}

\begin{sphinxuseclass}{cell_input}
\begin{sphinxVerbatim}[commandchars=\\\{\}]
\PYG{k+kn}{import} \PYG{n+nn}{pandas} \PYG{k}{as} \PYG{n+nn}{pd}

\PYG{n}{s1} \PYG{o}{=} \PYG{n}{pd}\PYG{o}{.}\PYG{n}{Series}\PYG{p}{(}\PYG{p}{[}\PYG{l+m+mi}{1}\PYG{p}{,}\PYG{l+m+mi}{2}\PYG{p}{,}\PYG{l+m+mi}{3}\PYG{p}{,}\PYG{l+m+mi}{4}\PYG{p}{,}\PYG{l+m+mi}{5}\PYG{p}{]}\PYG{p}{,} \PYG{n}{name}\PYG{o}{=}\PYG{l+s+s1}{\PYGZsq{}}\PYG{l+s+s1}{s1}\PYG{l+s+s1}{\PYGZsq{}}\PYG{p}{)} \PYG{c+c1}{\PYGZsh{} 두 Series를 합친 후, 어느 Series 에서 알기위해 이름 지정}
\PYG{n}{s2} \PYG{o}{=} \PYG{n}{pd}\PYG{o}{.}\PYG{n}{Series}\PYG{p}{(}\PYG{p}{[}\PYG{l+s+s1}{\PYGZsq{}}\PYG{l+s+s1}{a}\PYG{l+s+s1}{\PYGZsq{}}\PYG{p}{,}\PYG{l+s+s1}{\PYGZsq{}}\PYG{l+s+s1}{b}\PYG{l+s+s1}{\PYGZsq{}}\PYG{p}{,}\PYG{l+s+s1}{\PYGZsq{}}\PYG{l+s+s1}{c}\PYG{l+s+s1}{\PYGZsq{}}\PYG{p}{,}\PYG{l+s+s1}{\PYGZsq{}}\PYG{l+s+s1}{d}\PYG{l+s+s1}{\PYGZsq{}}\PYG{p}{,}\PYG{l+s+s1}{\PYGZsq{}}\PYG{l+s+s1}{e}\PYG{l+s+s1}{\PYGZsq{}}\PYG{p}{]}\PYG{p}{,} \PYG{n}{name}\PYG{o}{=}\PYG{l+s+s1}{\PYGZsq{}}\PYG{l+s+s1}{s2}\PYG{l+s+s1}{\PYGZsq{}}\PYG{p}{)}

\PYG{n}{horizontal} \PYG{o}{=} \PYG{n}{pd}\PYG{o}{.}\PYG{n}{concat}\PYG{p}{(}\PYG{p}{[}\PYG{n}{s1}\PYG{p}{,} \PYG{n}{s2}\PYG{p}{]}\PYG{p}{,} \PYG{n}{axis}\PYG{o}{=}\PYG{l+m+mi}{1}\PYG{p}{)} \PYG{c+c1}{\PYGZsh{} axis=1 이면 index (행) 기준으로 합함. 즉. 좌\PYGZhy{}우로 합함}
\PYG{n+nb}{print}\PYG{p}{(}\PYG{n}{horizontal}\PYG{p}{)}

\PYG{n+nb}{print}\PYG{p}{(}\PYG{l+s+s1}{\PYGZsq{}}\PYG{l+s+se}{\PYGZbs{}n}\PYG{l+s+s1}{\PYGZsq{}}\PYG{p}{)}
\PYG{n}{vertical} \PYG{o}{=} \PYG{n}{pd}\PYG{o}{.}\PYG{n}{concat}\PYG{p}{(}\PYG{p}{[}\PYG{n}{s1}\PYG{p}{,} \PYG{n}{s2}\PYG{p}{]}\PYG{p}{,} \PYG{n}{axis}\PYG{o}{=}\PYG{l+m+mi}{0}\PYG{p}{)} \PYG{c+c1}{\PYGZsh{} axis=0 이면 column(열) 기준으로 합함. 즉 위\PYGZhy{}아래로 합함}
\PYG{n+nb}{print}\PYG{p}{(}\PYG{n}{vertical}\PYG{p}{)}

\PYG{n+nb}{print}\PYG{p}{(}\PYG{l+s+s1}{\PYGZsq{}}\PYG{l+s+se}{\PYGZbs{}n}\PYG{l+s+s1}{\PYGZsq{}}\PYG{p}{)}
\PYG{n}{vertical\PYGZus{}1} \PYG{o}{=} \PYG{n}{pd}\PYG{o}{.}\PYG{n}{concat}\PYG{p}{(}\PYG{p}{[}\PYG{n}{s1}\PYG{p}{,} \PYG{n}{s2}\PYG{p}{]}\PYG{p}{,} \PYG{n}{axis}\PYG{o}{=}\PYG{l+m+mi}{1}\PYG{p}{,} \PYG{n}{join}\PYG{o}{=}\PYG{l+s+s1}{\PYGZsq{}}\PYG{l+s+s1}{outer}\PYG{l+s+s1}{\PYGZsq{}}\PYG{p}{)} \PYG{c+c1}{\PYGZsh{} axis=1 인덱스 기준으로 합함. 양쪽 Series 에 존재하는 모든 index 는 남김}
\PYG{n}{vertical\PYGZus{}2} \PYG{o}{=} \PYG{n}{pd}\PYG{o}{.}\PYG{n}{concat}\PYG{p}{(}\PYG{p}{[}\PYG{n}{s1}\PYG{p}{,} \PYG{n}{s2}\PYG{p}{]}\PYG{p}{,} \PYG{n}{axis}\PYG{o}{=}\PYG{l+m+mi}{1}\PYG{p}{,} \PYG{n}{join}\PYG{o}{=}\PYG{l+s+s1}{\PYGZsq{}}\PYG{l+s+s1}{inner}\PYG{l+s+s1}{\PYGZsq{}}\PYG{p}{)} \PYG{c+c1}{\PYGZsh{} axis=1 인덱스 기준으로 합함. 양쪽 Series 에 동시에 존재하는 index 만 남김}
\PYG{n+nb}{print}\PYG{p}{(}\PYG{n}{vertical\PYGZus{}1}\PYG{o}{.}\PYG{n}{equals}\PYG{p}{(}\PYG{n}{vertical\PYGZus{}2}\PYG{p}{)}\PYG{p}{)} \PYG{c+c1}{\PYGZsh{} 두개의 DataFrame 이 서로 동일한지 체크. 인덱스가 동일하므로 동일 결과가 됨.}
\end{sphinxVerbatim}

\end{sphinxuseclass}\end{sphinxVerbatimInput}
\begin{sphinxVerbatimOutput}

\begin{sphinxuseclass}{cell_output}
\begin{sphinxVerbatim}[commandchars=\\\{\}]
   s1 s2
0   1  a
1   2  b
2   3  c
3   4  d
4   5  e


0    1
1    2
2    3
3    4
4    5
0    a
1    b
2    c
3    d
4    e
dtype: object


True
\end{sphinxVerbatim}

\end{sphinxuseclass}\end{sphinxVerbatimOutput}

\end{sphinxuseclass}
\sphinxAtStartPar
 concat 에서 두 Series 의 index 가 다르경우, 원하는 결과가 안 나온다는 것의 유의합니다. 아래 예제에서 index 가 서로 다른 Series 를 합쳐보겠습니다. join=’inner’ 조건에서는 동일한 index 가 없으므로 concat 후 결과가 없습니다. 단지 좌\sphinxhyphen{}우로 합치는 것이 목적이라면 기존의 index 를 제거하고 default index 인 숫자를 넣어주고 concat 하면 됩니다. 기존의 index 를 제거할 때는  reset\_index(drop=True) 를 합니다.

\begin{sphinxuseclass}{cell}\begin{sphinxVerbatimInput}

\begin{sphinxuseclass}{cell_input}
\begin{sphinxVerbatim}[commandchars=\\\{\}]
\PYG{n}{s3} \PYG{o}{=} \PYG{n}{pd}\PYG{o}{.}\PYG{n}{Series}\PYG{p}{(}\PYG{p}{[}\PYG{l+m+mi}{1}\PYG{p}{,}\PYG{l+m+mi}{2}\PYG{p}{,}\PYG{l+m+mi}{3}\PYG{p}{,}\PYG{l+m+mi}{4}\PYG{p}{,}\PYG{l+m+mi}{5}\PYG{p}{]}\PYG{p}{,} \PYG{n}{index} \PYG{o}{=} \PYG{p}{[}\PYG{l+s+s1}{\PYGZsq{}}\PYG{l+s+s1}{a}\PYG{l+s+s1}{\PYGZsq{}}\PYG{p}{,}\PYG{l+s+s1}{\PYGZsq{}}\PYG{l+s+s1}{b}\PYG{l+s+s1}{\PYGZsq{}}\PYG{p}{,}\PYG{l+s+s1}{\PYGZsq{}}\PYG{l+s+s1}{c}\PYG{l+s+s1}{\PYGZsq{}}\PYG{p}{,}\PYG{l+s+s1}{\PYGZsq{}}\PYG{l+s+s1}{d}\PYG{l+s+s1}{\PYGZsq{}}\PYG{p}{,}\PYG{l+s+s1}{\PYGZsq{}}\PYG{l+s+s1}{e}\PYG{l+s+s1}{\PYGZsq{}}\PYG{p}{]} \PYG{p}{,} \PYG{n}{name}\PYG{o}{=}\PYG{l+s+s1}{\PYGZsq{}}\PYG{l+s+s1}{s3}\PYG{l+s+s1}{\PYGZsq{}}\PYG{p}{)}
\PYG{n}{s4} \PYG{o}{=} \PYG{n}{pd}\PYG{o}{.}\PYG{n}{Series}\PYG{p}{(}\PYG{p}{[}\PYG{l+m+mi}{11}\PYG{p}{,}\PYG{l+m+mi}{12}\PYG{p}{,}\PYG{l+m+mi}{13}\PYG{p}{,}\PYG{l+m+mi}{14}\PYG{p}{,}\PYG{l+m+mi}{15}\PYG{p}{]}\PYG{p}{,} \PYG{n}{index} \PYG{o}{=} \PYG{p}{[}\PYG{l+s+s1}{\PYGZsq{}}\PYG{l+s+s1}{f}\PYG{l+s+s1}{\PYGZsq{}}\PYG{p}{,}\PYG{l+s+s1}{\PYGZsq{}}\PYG{l+s+s1}{g}\PYG{l+s+s1}{\PYGZsq{}}\PYG{p}{,}\PYG{l+s+s1}{\PYGZsq{}}\PYG{l+s+s1}{h}\PYG{l+s+s1}{\PYGZsq{}}\PYG{p}{,}\PYG{l+s+s1}{\PYGZsq{}}\PYG{l+s+s1}{i}\PYG{l+s+s1}{\PYGZsq{}}\PYG{p}{,}\PYG{l+s+s1}{\PYGZsq{}}\PYG{l+s+s1}{j}\PYG{l+s+s1}{\PYGZsq{}}\PYG{p}{]}\PYG{p}{,} \PYG{n}{name}\PYG{o}{=}\PYG{l+s+s1}{\PYGZsq{}}\PYG{l+s+s1}{s4}\PYG{l+s+s1}{\PYGZsq{}}\PYG{p}{)}

\PYG{n+nb}{print}\PYG{p}{(}\PYG{n}{pd}\PYG{o}{.}\PYG{n}{concat}\PYG{p}{(}\PYG{p}{[}\PYG{n}{s3}\PYG{p}{,} \PYG{n}{s4}\PYG{p}{]}\PYG{p}{,} \PYG{n}{axis}\PYG{o}{=}\PYG{l+m+mi}{1}\PYG{p}{,} \PYG{n}{join}\PYG{o}{=}\PYG{l+s+s1}{\PYGZsq{}}\PYG{l+s+s1}{inner}\PYG{l+s+s1}{\PYGZsq{}}\PYG{p}{)}\PYG{p}{)} \PYG{c+c1}{\PYGZsh{} axis=1 이면 인덱스 기준으로 합함. 즉. 좌\PYGZhy{}우로 합함}

\PYG{n+nb}{print}\PYG{p}{(}\PYG{l+s+s1}{\PYGZsq{}}\PYG{l+s+se}{\PYGZbs{}n}\PYG{l+s+s1}{\PYGZsq{}}\PYG{p}{)}
\PYG{n+nb}{print}\PYG{p}{(}\PYG{n}{pd}\PYG{o}{.}\PYG{n}{concat}\PYG{p}{(}\PYG{p}{[}\PYG{n}{s3}\PYG{o}{.}\PYG{n}{reset\PYGZus{}index}\PYG{p}{(}\PYG{n}{drop}\PYG{o}{=}\PYG{k+kc}{True}\PYG{p}{)}\PYG{p}{,} \PYG{n}{s4}\PYG{o}{.}\PYG{n}{reset\PYGZus{}index}\PYG{p}{(}\PYG{n}{drop}\PYG{o}{=}\PYG{k+kc}{True}\PYG{p}{)}\PYG{p}{]}\PYG{p}{,} \PYG{n}{axis}\PYG{o}{=}\PYG{l+m+mi}{1}\PYG{p}{,} \PYG{n}{join}\PYG{o}{=}\PYG{l+s+s1}{\PYGZsq{}}\PYG{l+s+s1}{inner}\PYG{l+s+s1}{\PYGZsq{}}\PYG{p}{)}\PYG{p}{)} \PYG{c+c1}{\PYGZsh{} axis=1 이면 인덱스 기준으로 합함. 즉. 좌\PYGZhy{}우로 합함}
\end{sphinxVerbatim}

\end{sphinxuseclass}\end{sphinxVerbatimInput}
\begin{sphinxVerbatimOutput}

\begin{sphinxuseclass}{cell_output}
\begin{sphinxVerbatim}[commandchars=\\\{\}]
Empty DataFrame
Columns: [s3, s4]
Index: []


   s3  s4
0   1  11
1   2  12
2   3  13
3   4  14
4   5  15
\end{sphinxVerbatim}

\end{sphinxuseclass}\end{sphinxVerbatimOutput}

\end{sphinxuseclass}

\section{Merge}
\label{\detokenize{chapter2/2.2.2_Useful_Techniques:merge}}
\sphinxAtStartPar
Index 가 동일하고, 단순한 병합일 때는 concat 를 쓰지만, 서로 다른 컬럼으로 병합을 할 때는 Merge 를 씁니다.
만약 두 데이터셋이 있고, 고객번호로 서로 Merge 하려고 한다고 합시다. 그런데 한 데이터셋에는 고객번호가 cust\_id 로 되어 있고, 다른 데이터셋에는 Cust\_Number 로 되어 있으면 concat 를 활용하기 어렵습니다. 이 경우는 merge 를 쓰는 것이 편리합니다. merge 는 넣어야하는 인수가 concat 보다많아, 단순한 병합은 concat 으로 합니다. 먼저 예제 DataFrame 을 생성합니다.

\begin{sphinxuseclass}{cell}\begin{sphinxVerbatimInput}

\begin{sphinxuseclass}{cell_input}
\begin{sphinxVerbatim}[commandchars=\\\{\}]
\PYG{n}{cust\PYGZus{}list} \PYG{o}{=} \PYG{p}{[}\PYG{l+m+mi}{10}\PYG{p}{,} \PYG{l+m+mi}{11}\PYG{p}{,} \PYG{l+m+mi}{12}\PYG{p}{,} \PYG{l+m+mi}{13}\PYG{p}{,} \PYG{l+m+mi}{14}\PYG{p}{,} \PYG{l+m+mi}{15}\PYG{p}{]}
\PYG{n}{product\PYGZus{}list} \PYG{o}{=} \PYG{p}{[}\PYG{l+s+s1}{\PYGZsq{}}\PYG{l+s+s1}{a}\PYG{l+s+s1}{\PYGZsq{}}\PYG{p}{,}\PYG{l+s+s1}{\PYGZsq{}}\PYG{l+s+s1}{b}\PYG{l+s+s1}{\PYGZsq{}}\PYG{p}{,}\PYG{l+s+s1}{\PYGZsq{}}\PYG{l+s+s1}{c}\PYG{l+s+s1}{\PYGZsq{}}\PYG{p}{,}\PYG{l+s+s1}{\PYGZsq{}}\PYG{l+s+s1}{d}\PYG{l+s+s1}{\PYGZsq{}}\PYG{p}{,}\PYG{l+s+s1}{\PYGZsq{}}\PYG{l+s+s1}{e}\PYG{l+s+s1}{\PYGZsq{}}\PYG{p}{,} \PYG{l+s+s1}{\PYGZsq{}}\PYG{l+s+s1}{f}\PYG{l+s+s1}{\PYGZsq{}}\PYG{p}{]}
\PYG{n}{df1} \PYG{o}{=} \PYG{n}{pd}\PYG{o}{.}\PYG{n}{DataFrame}\PYG{p}{(}\PYG{p}{\PYGZob{}}\PYG{l+s+s1}{\PYGZsq{}}\PYG{l+s+s1}{cust\PYGZus{}id}\PYG{l+s+s1}{\PYGZsq{}}\PYG{p}{:} \PYG{n}{cust\PYGZus{}list}\PYG{p}{,} \PYG{l+s+s1}{\PYGZsq{}}\PYG{l+s+s1}{product}\PYG{l+s+s1}{\PYGZsq{}}\PYG{p}{:} \PYG{n}{product\PYGZus{}list}\PYG{p}{\PYGZcb{}}\PYG{p}{)}

\PYG{n}{cust\PYGZus{}list} \PYG{o}{=} \PYG{p}{[}\PYG{l+m+mi}{12}\PYG{p}{,} \PYG{l+m+mi}{13}\PYG{p}{,} \PYG{l+m+mi}{14}\PYG{p}{,} \PYG{l+m+mi}{15}\PYG{p}{,} \PYG{l+m+mi}{16}\PYG{p}{,} \PYG{l+m+mi}{17}\PYG{p}{]}
\PYG{n}{grade\PYGZus{}list} \PYG{o}{=} \PYG{p}{[}\PYG{l+s+s1}{\PYGZsq{}}\PYG{l+s+s1}{p1}\PYG{l+s+s1}{\PYGZsq{}}\PYG{p}{,}\PYG{l+s+s1}{\PYGZsq{}}\PYG{l+s+s1}{p2}\PYG{l+s+s1}{\PYGZsq{}}\PYG{p}{,}\PYG{l+s+s1}{\PYGZsq{}}\PYG{l+s+s1}{p3}\PYG{l+s+s1}{\PYGZsq{}}\PYG{p}{,}\PYG{l+s+s1}{\PYGZsq{}}\PYG{l+s+s1}{p4}\PYG{l+s+s1}{\PYGZsq{}}\PYG{p}{,}\PYG{l+s+s1}{\PYGZsq{}}\PYG{l+s+s1}{p5}\PYG{l+s+s1}{\PYGZsq{}}\PYG{p}{,}\PYG{l+s+s1}{\PYGZsq{}}\PYG{l+s+s1}{p6}\PYG{l+s+s1}{\PYGZsq{}}\PYG{p}{]}
\PYG{n}{df2} \PYG{o}{=} \PYG{n}{pd}\PYG{o}{.}\PYG{n}{DataFrame}\PYG{p}{(}\PYG{p}{\PYGZob{}}\PYG{l+s+s1}{\PYGZsq{}}\PYG{l+s+s1}{cust\PYGZus{}number}\PYG{l+s+s1}{\PYGZsq{}}\PYG{p}{:} \PYG{n}{cust\PYGZus{}list}\PYG{p}{,} \PYG{l+s+s1}{\PYGZsq{}}\PYG{l+s+s1}{grade}\PYG{l+s+s1}{\PYGZsq{}}\PYG{p}{:} \PYG{n}{grade\PYGZus{}list}\PYG{p}{\PYGZcb{}}\PYG{p}{)}

\PYG{n+nb}{print}\PYG{p}{(}\PYG{n}{df1}\PYG{p}{)}
\PYG{n+nb}{print}\PYG{p}{(}\PYG{l+s+s1}{\PYGZsq{}}\PYG{l+s+se}{\PYGZbs{}n}\PYG{l+s+s1}{\PYGZsq{}}\PYG{p}{)}
\PYG{n+nb}{print}\PYG{p}{(}\PYG{n}{df2}\PYG{p}{)}
\end{sphinxVerbatim}

\end{sphinxuseclass}\end{sphinxVerbatimInput}
\begin{sphinxVerbatimOutput}

\begin{sphinxuseclass}{cell_output}
\begin{sphinxVerbatim}[commandchars=\\\{\}]
   cust\PYGZus{}id product
0       10       a
1       11       b
2       12       c
3       13       d
4       14       e
5       15       f


   cust\PYGZus{}number grade
0           12    p1
1           13    p2
2           14    p3
3           15    p4
4           16    p5
5           17    p6
\end{sphinxVerbatim}

\end{sphinxuseclass}\end{sphinxVerbatimOutput}

\end{sphinxuseclass}


\begin{sphinxuseclass}{cell}\begin{sphinxVerbatimInput}

\begin{sphinxuseclass}{cell_input}
\begin{sphinxVerbatim}[commandchars=\\\{\}]
\PYG{n+nb}{print}\PYG{p}{(}\PYG{n}{df1}\PYG{o}{.}\PYG{n}{set\PYGZus{}index}\PYG{p}{(}\PYG{l+s+s1}{\PYGZsq{}}\PYG{l+s+s1}{cust\PYGZus{}id}\PYG{l+s+s1}{\PYGZsq{}}\PYG{p}{)}\PYG{p}{)}
\PYG{n+nb}{print}\PYG{p}{(}\PYG{n}{df2}\PYG{o}{.}\PYG{n}{set\PYGZus{}index}\PYG{p}{(}\PYG{l+s+s1}{\PYGZsq{}}\PYG{l+s+s1}{cust\PYGZus{}number}\PYG{l+s+s1}{\PYGZsq{}}\PYG{p}{)}\PYG{p}{)}
\end{sphinxVerbatim}

\end{sphinxuseclass}\end{sphinxVerbatimInput}
\begin{sphinxVerbatimOutput}

\begin{sphinxuseclass}{cell_output}
\begin{sphinxVerbatim}[commandchars=\\\{\}]
        product
cust\PYGZus{}id        
10            a
11            b
12            c
13            d
14            e
15            f
            grade
cust\PYGZus{}number      
12             p1
13             p2
14             p3
15             p4
16             p5
17             p6
\end{sphinxVerbatim}

\end{sphinxuseclass}\end{sphinxVerbatimOutput}

\end{sphinxuseclass}


\begin{sphinxuseclass}{cell}\begin{sphinxVerbatimInput}

\begin{sphinxuseclass}{cell_input}
\begin{sphinxVerbatim}[commandchars=\\\{\}]
\PYG{n}{df1}\PYG{o}{.}\PYG{n}{set\PYGZus{}index}\PYG{p}{(}\PYG{l+s+s1}{\PYGZsq{}}\PYG{l+s+s1}{cust\PYGZus{}id}\PYG{l+s+s1}{\PYGZsq{}}\PYG{p}{)}\PYG{o}{.}\PYG{n}{merge}\PYG{p}{(}\PYG{n}{df2}\PYG{o}{.}\PYG{n}{set\PYGZus{}index}\PYG{p}{(}\PYG{l+s+s1}{\PYGZsq{}}\PYG{l+s+s1}{cust\PYGZus{}number}\PYG{l+s+s1}{\PYGZsq{}}\PYG{p}{)}\PYG{p}{,} \PYG{n}{left\PYGZus{}index}\PYG{o}{=}\PYG{k+kc}{True}\PYG{p}{,} \PYG{n}{right\PYGZus{}index}\PYG{o}{=}\PYG{k+kc}{True}\PYG{p}{,} \PYG{n}{how}\PYG{o}{=}\PYG{l+s+s1}{\PYGZsq{}}\PYG{l+s+s1}{inner}\PYG{l+s+s1}{\PYGZsq{}}\PYG{p}{)}
\end{sphinxVerbatim}

\end{sphinxuseclass}\end{sphinxVerbatimInput}
\begin{sphinxVerbatimOutput}

\begin{sphinxuseclass}{cell_output}
\begin{sphinxVerbatim}[commandchars=\\\{\}]
   product grade
12       c    p1
13       d    p2
14       e    p3
15       f    p4
\end{sphinxVerbatim}

\end{sphinxuseclass}\end{sphinxVerbatimOutput}

\end{sphinxuseclass}
\begin{sphinxuseclass}{cell}\begin{sphinxVerbatimInput}

\begin{sphinxuseclass}{cell_input}
\begin{sphinxVerbatim}[commandchars=\\\{\}]
\PYG{n}{pd}\PYG{o}{.}\PYG{n}{merge}\PYG{p}{(}\PYG{n}{left}\PYG{o}{=}\PYG{n}{df1}\PYG{o}{.}\PYG{n}{set\PYGZus{}index}\PYG{p}{(}\PYG{l+s+s1}{\PYGZsq{}}\PYG{l+s+s1}{cust\PYGZus{}id}\PYG{l+s+s1}{\PYGZsq{}}\PYG{p}{)}\PYG{p}{,} \PYG{n}{right}\PYG{o}{=}\PYG{n}{df2}\PYG{o}{.}\PYG{n}{set\PYGZus{}index}\PYG{p}{(}\PYG{l+s+s1}{\PYGZsq{}}\PYG{l+s+s1}{cust\PYGZus{}number}\PYG{l+s+s1}{\PYGZsq{}}\PYG{p}{)}\PYG{p}{,} \PYG{n}{left\PYGZus{}index}\PYG{o}{=}\PYG{k+kc}{True}\PYG{p}{,} \PYG{n}{right\PYGZus{}index}\PYG{o}{=}\PYG{k+kc}{True}\PYG{p}{,} \PYG{n}{how}\PYG{o}{=}\PYG{l+s+s1}{\PYGZsq{}}\PYG{l+s+s1}{inner}\PYG{l+s+s1}{\PYGZsq{}}\PYG{p}{)}
\end{sphinxVerbatim}

\end{sphinxuseclass}\end{sphinxVerbatimInput}
\begin{sphinxVerbatimOutput}

\begin{sphinxuseclass}{cell_output}
\begin{sphinxVerbatim}[commandchars=\\\{\}]
   product grade
12       c    p1
13       d    p2
14       e    p3
15       f    p4
\end{sphinxVerbatim}

\end{sphinxuseclass}\end{sphinxVerbatimOutput}

\end{sphinxuseclass}


\begin{sphinxuseclass}{cell}\begin{sphinxVerbatimInput}

\begin{sphinxuseclass}{cell_input}
\begin{sphinxVerbatim}[commandchars=\\\{\}]
\PYG{n}{df1}\PYG{o}{.}\PYG{n}{set\PYGZus{}index}\PYG{p}{(}\PYG{l+s+s1}{\PYGZsq{}}\PYG{l+s+s1}{cust\PYGZus{}id}\PYG{l+s+s1}{\PYGZsq{}}\PYG{p}{)}\PYG{o}{.}\PYG{n}{merge}\PYG{p}{(}\PYG{n}{df2}\PYG{o}{.}\PYG{n}{set\PYGZus{}index}\PYG{p}{(}\PYG{l+s+s1}{\PYGZsq{}}\PYG{l+s+s1}{cust\PYGZus{}number}\PYG{l+s+s1}{\PYGZsq{}}\PYG{p}{)}\PYG{p}{,} \PYG{n}{left\PYGZus{}index}\PYG{o}{=}\PYG{k+kc}{True}\PYG{p}{,} \PYG{n}{right\PYGZus{}index}\PYG{o}{=}\PYG{k+kc}{True}\PYG{p}{,} \PYG{n}{how}\PYG{o}{=}\PYG{l+s+s1}{\PYGZsq{}}\PYG{l+s+s1}{inner}\PYG{l+s+s1}{\PYGZsq{}}\PYG{p}{)}\PYG{o}{.}\PYG{n}{reset\PYGZus{}index}\PYG{p}{(}\PYG{p}{)}\PYG{o}{.}\PYG{n}{rename}\PYG{p}{(}\PYG{n}{columns}\PYG{o}{=}\PYG{p}{\PYGZob{}}\PYG{l+s+s1}{\PYGZsq{}}\PYG{l+s+s1}{index}\PYG{l+s+s1}{\PYGZsq{}}\PYG{p}{:}\PYG{l+s+s1}{\PYGZsq{}}\PYG{l+s+s1}{cust\PYGZus{}id}\PYG{l+s+s1}{\PYGZsq{}}\PYG{p}{\PYGZcb{}}\PYG{p}{)}
\end{sphinxVerbatim}

\end{sphinxuseclass}\end{sphinxVerbatimInput}
\begin{sphinxVerbatimOutput}

\begin{sphinxuseclass}{cell_output}
\begin{sphinxVerbatim}[commandchars=\\\{\}]
   cust\PYGZus{}id product grade
0       12       c    p1
1       13       d    p2
2       14       e    p3
3       15       f    p4
\end{sphinxVerbatim}

\end{sphinxuseclass}\end{sphinxVerbatimOutput}

\end{sphinxuseclass}

\section{Groupby}
\label{\detokenize{chapter2/2.2.3_Useful_Techniques:groupby}}\label{\detokenize{chapter2/2.2.3_Useful_Techniques::doc}}
\sphinxAtStartPar
Groupby 는 데이터를 요약할 때 많이 활용하는 기법입니다. 아래 예제에서 만들어진 DataFrame \sphinxhyphen{} df 의 ‘grp’ 컬럼을 이용하여 ‘a’, ‘b’, ‘c’ 등의 3 개의 그룹으로 나눌 수 있습니다.
먼저, 그룹을 무시하고 v1, v2 의 평균값을 알아봅니다. 그 다음, 그룹 별로 v1 과 v2 의 평균값을 알아봅니다.

\begin{sphinxuseclass}{cell}\begin{sphinxVerbatimInput}

\begin{sphinxuseclass}{cell_input}
\begin{sphinxVerbatim}[commandchars=\\\{\}]
\PYG{k+kn}{import} \PYG{n+nn}{pandas} \PYG{k}{as} \PYG{n+nn}{pd}

\PYG{n}{g\PYGZus{}list} \PYG{o}{=} \PYG{p}{[}\PYG{l+s+s1}{\PYGZsq{}}\PYG{l+s+s1}{a}\PYG{l+s+s1}{\PYGZsq{}}\PYG{p}{,}\PYG{l+s+s1}{\PYGZsq{}}\PYG{l+s+s1}{a}\PYG{l+s+s1}{\PYGZsq{}}\PYG{p}{,}\PYG{l+s+s1}{\PYGZsq{}}\PYG{l+s+s1}{a}\PYG{l+s+s1}{\PYGZsq{}}\PYG{p}{,}\PYG{l+s+s1}{\PYGZsq{}}\PYG{l+s+s1}{b}\PYG{l+s+s1}{\PYGZsq{}}\PYG{p}{,}\PYG{l+s+s1}{\PYGZsq{}}\PYG{l+s+s1}{b}\PYG{l+s+s1}{\PYGZsq{}}\PYG{p}{,}\PYG{l+s+s1}{\PYGZsq{}}\PYG{l+s+s1}{b}\PYG{l+s+s1}{\PYGZsq{}}\PYG{p}{,}\PYG{l+s+s1}{\PYGZsq{}}\PYG{l+s+s1}{c}\PYG{l+s+s1}{\PYGZsq{}}\PYG{p}{,}\PYG{l+s+s1}{\PYGZsq{}}\PYG{l+s+s1}{c}\PYG{l+s+s1}{\PYGZsq{}}\PYG{p}{,}\PYG{l+s+s1}{\PYGZsq{}}\PYG{l+s+s1}{c}\PYG{l+s+s1}{\PYGZsq{}}\PYG{p}{,}\PYG{l+s+s1}{\PYGZsq{}}\PYG{l+s+s1}{c}\PYG{l+s+s1}{\PYGZsq{}}\PYG{p}{]}
\PYG{n}{v1\PYGZus{}list} \PYG{o}{=} \PYG{p}{[}\PYG{l+m+mi}{1}\PYG{p}{,} \PYG{l+m+mi}{2}\PYG{p}{,} \PYG{l+m+mi}{3}\PYG{p}{,} \PYG{l+m+mi}{4}\PYG{p}{,} \PYG{l+m+mi}{5}\PYG{p}{,} \PYG{l+m+mi}{6}\PYG{p}{,} \PYG{l+m+mi}{7}\PYG{p}{,} \PYG{l+m+mi}{8}\PYG{p}{,} \PYG{l+m+mi}{9}\PYG{p}{,} \PYG{l+m+mi}{10}\PYG{p}{]}
\PYG{n}{v2\PYGZus{}list} \PYG{o}{=} \PYG{p}{[}\PYG{l+m+mi}{11}\PYG{p}{,} \PYG{l+m+mi}{12}\PYG{p}{,} \PYG{l+m+mi}{13}\PYG{p}{,} \PYG{l+m+mi}{14}\PYG{p}{,} \PYG{l+m+mi}{15}\PYG{p}{,} \PYG{l+m+mi}{16}\PYG{p}{,} \PYG{l+m+mi}{17}\PYG{p}{,} \PYG{l+m+mi}{18}\PYG{p}{,} \PYG{l+m+mi}{19}\PYG{p}{,} \PYG{l+m+mi}{20}\PYG{p}{]}

\PYG{n}{df} \PYG{o}{=}  \PYG{n}{pd}\PYG{o}{.}\PYG{n}{DataFrame}\PYG{p}{(}\PYG{p}{\PYGZob{}}\PYG{l+s+s1}{\PYGZsq{}}\PYG{l+s+s1}{grp}\PYG{l+s+s1}{\PYGZsq{}}\PYG{p}{:} \PYG{n}{g\PYGZus{}list}\PYG{p}{,} \PYG{l+s+s1}{\PYGZsq{}}\PYG{l+s+s1}{v1}\PYG{l+s+s1}{\PYGZsq{}}\PYG{p}{:} \PYG{n}{v1\PYGZus{}list}\PYG{p}{,} \PYG{l+s+s1}{\PYGZsq{}}\PYG{l+s+s1}{v2}\PYG{l+s+s1}{\PYGZsq{}}\PYG{p}{:} \PYG{n}{v2\PYGZus{}list}\PYG{p}{\PYGZcb{}}\PYG{p}{)} \PYG{c+c1}{\PYGZsh{} 그룹핑을 할 수 있는 컬럼을 가진 DataFrame 생성}
\end{sphinxVerbatim}

\end{sphinxuseclass}\end{sphinxVerbatimInput}

\end{sphinxuseclass}
\begin{sphinxuseclass}{cell}\begin{sphinxVerbatimInput}

\begin{sphinxuseclass}{cell_input}
\begin{sphinxVerbatim}[commandchars=\\\{\}]
\PYG{n}{df}\PYG{p}{[}\PYG{p}{[}\PYG{l+s+s1}{\PYGZsq{}}\PYG{l+s+s1}{v1}\PYG{l+s+s1}{\PYGZsq{}}\PYG{p}{,} \PYG{l+s+s1}{\PYGZsq{}}\PYG{l+s+s1}{v2}\PYG{l+s+s1}{\PYGZsq{}}\PYG{p}{]}\PYG{p}{]}\PYG{o}{.}\PYG{n}{mean}\PYG{p}{(}\PYG{p}{)} \PYG{c+c1}{\PYGZsh{} 전체 평균}
\end{sphinxVerbatim}

\end{sphinxuseclass}\end{sphinxVerbatimInput}
\begin{sphinxVerbatimOutput}

\begin{sphinxuseclass}{cell_output}
\begin{sphinxVerbatim}[commandchars=\\\{\}]
v1     5.5
v2    15.5
dtype: float64
\end{sphinxVerbatim}

\end{sphinxuseclass}\end{sphinxVerbatimOutput}

\end{sphinxuseclass}
\begin{sphinxuseclass}{cell}\begin{sphinxVerbatimInput}

\begin{sphinxuseclass}{cell_input}
\begin{sphinxVerbatim}[commandchars=\\\{\}]
\PYG{n}{df}\PYG{o}{.}\PYG{n}{groupby}\PYG{p}{(}\PYG{l+s+s1}{\PYGZsq{}}\PYG{l+s+s1}{grp}\PYG{l+s+s1}{\PYGZsq{}}\PYG{p}{)}\PYG{p}{[}\PYG{l+s+s1}{\PYGZsq{}}\PYG{l+s+s1}{v1}\PYG{l+s+s1}{\PYGZsq{}}\PYG{p}{]}\PYG{o}{.}\PYG{n}{mean}\PYG{p}{(}\PYG{p}{)} \PYG{c+c1}{\PYGZsh{} 그룹별 평균}
\end{sphinxVerbatim}

\end{sphinxuseclass}\end{sphinxVerbatimInput}
\begin{sphinxVerbatimOutput}

\begin{sphinxuseclass}{cell_output}
\begin{sphinxVerbatim}[commandchars=\\\{\}]
grp
a    2.0
b    5.0
c    8.5
Name: v1, dtype: float64
\end{sphinxVerbatim}

\end{sphinxuseclass}\end{sphinxVerbatimOutput}

\end{sphinxuseclass}


\begin{sphinxuseclass}{cell}\begin{sphinxVerbatimInput}

\begin{sphinxuseclass}{cell_input}
\begin{sphinxVerbatim}[commandchars=\\\{\}]
\PYG{n}{df}\PYG{o}{.}\PYG{n}{groupby}\PYG{p}{(}\PYG{l+s+s1}{\PYGZsq{}}\PYG{l+s+s1}{grp}\PYG{l+s+s1}{\PYGZsq{}}\PYG{p}{)}\PYG{p}{[}\PYG{l+s+s1}{\PYGZsq{}}\PYG{l+s+s1}{v1}\PYG{l+s+s1}{\PYGZsq{}}\PYG{p}{]}\PYG{o}{.}\PYG{n}{agg}\PYG{p}{(}\PYG{p}{[}\PYG{l+s+s1}{\PYGZsq{}}\PYG{l+s+s1}{mean}\PYG{l+s+s1}{\PYGZsq{}}\PYG{p}{,}\PYG{l+s+s1}{\PYGZsq{}}\PYG{l+s+s1}{max}\PYG{l+s+s1}{\PYGZsq{}}\PYG{p}{,}\PYG{l+s+s1}{\PYGZsq{}}\PYG{l+s+s1}{sum}\PYG{l+s+s1}{\PYGZsq{}}\PYG{p}{]}\PYG{p}{)}
\end{sphinxVerbatim}

\end{sphinxuseclass}\end{sphinxVerbatimInput}
\begin{sphinxVerbatimOutput}

\begin{sphinxuseclass}{cell_output}
\begin{sphinxVerbatim}[commandchars=\\\{\}]
     mean  max  sum
grp                
a     2.0    3    6
b     5.0    6   15
c     8.5   10   34
\end{sphinxVerbatim}

\end{sphinxuseclass}\end{sphinxVerbatimOutput}

\end{sphinxuseclass}


\begin{sphinxuseclass}{cell}\begin{sphinxVerbatimInput}

\begin{sphinxuseclass}{cell_input}
\begin{sphinxVerbatim}[commandchars=\\\{\}]
\PYG{n}{df}\PYG{o}{.}\PYG{n}{groupby}\PYG{p}{(}\PYG{l+s+s1}{\PYGZsq{}}\PYG{l+s+s1}{grp}\PYG{l+s+s1}{\PYGZsq{}}\PYG{p}{)}\PYG{p}{[}\PYG{p}{[}\PYG{l+s+s1}{\PYGZsq{}}\PYG{l+s+s1}{v1}\PYG{l+s+s1}{\PYGZsq{}}\PYG{p}{,}\PYG{l+s+s1}{\PYGZsq{}}\PYG{l+s+s1}{v2}\PYG{l+s+s1}{\PYGZsq{}}\PYG{p}{]}\PYG{p}{]}\PYG{o}{.}\PYG{n}{agg}\PYG{p}{(}\PYG{p}{[}\PYG{l+s+s1}{\PYGZsq{}}\PYG{l+s+s1}{mean}\PYG{l+s+s1}{\PYGZsq{}}\PYG{p}{,}\PYG{l+s+s1}{\PYGZsq{}}\PYG{l+s+s1}{max}\PYG{l+s+s1}{\PYGZsq{}}\PYG{p}{,}\PYG{l+s+s1}{\PYGZsq{}}\PYG{l+s+s1}{sum}\PYG{l+s+s1}{\PYGZsq{}}\PYG{p}{]}\PYG{p}{)}
\end{sphinxVerbatim}

\end{sphinxuseclass}\end{sphinxVerbatimInput}
\begin{sphinxVerbatimOutput}

\begin{sphinxuseclass}{cell_output}
\begin{sphinxVerbatim}[commandchars=\\\{\}]
      v1            v2        
    mean max sum  mean max sum
grp                           
a    2.0   3   6  12.0  13  36
b    5.0   6  15  15.0  16  45
c    8.5  10  34  18.5  20  74
\end{sphinxVerbatim}

\end{sphinxuseclass}\end{sphinxVerbatimOutput}

\end{sphinxuseclass}


\begin{sphinxuseclass}{cell}\begin{sphinxVerbatimInput}

\begin{sphinxuseclass}{cell_input}
\begin{sphinxVerbatim}[commandchars=\\\{\}]
\PYG{n}{s} \PYG{o}{=} \PYG{p}{\PYGZob{}}\PYG{l+s+s1}{\PYGZsq{}}\PYG{l+s+s1}{v1}\PYG{l+s+s1}{\PYGZsq{}}\PYG{p}{:}\PYG{l+s+s1}{\PYGZsq{}}\PYG{l+s+s1}{mean}\PYG{l+s+s1}{\PYGZsq{}}\PYG{p}{,} \PYG{l+s+s1}{\PYGZsq{}}\PYG{l+s+s1}{v2}\PYG{l+s+s1}{\PYGZsq{}}\PYG{p}{:}\PYG{l+s+s1}{\PYGZsq{}}\PYG{l+s+s1}{sum}\PYG{l+s+s1}{\PYGZsq{}}\PYG{p}{\PYGZcb{}}
\PYG{n}{df}\PYG{o}{.}\PYG{n}{groupby}\PYG{p}{(}\PYG{l+s+s1}{\PYGZsq{}}\PYG{l+s+s1}{grp}\PYG{l+s+s1}{\PYGZsq{}}\PYG{p}{)}\PYG{o}{.}\PYG{n}{agg}\PYG{p}{(}\PYG{n}{s}\PYG{p}{)}
\end{sphinxVerbatim}

\end{sphinxuseclass}\end{sphinxVerbatimInput}
\begin{sphinxVerbatimOutput}

\begin{sphinxuseclass}{cell_output}
\begin{sphinxVerbatim}[commandchars=\\\{\}]
      v1  v2
grp         
a    2.0  36
b    5.0  45
c    8.5  74
\end{sphinxVerbatim}

\end{sphinxuseclass}\end{sphinxVerbatimOutput}

\end{sphinxuseclass}


\begin{sphinxuseclass}{cell}\begin{sphinxVerbatimInput}

\begin{sphinxuseclass}{cell_input}
\begin{sphinxVerbatim}[commandchars=\\\{\}]
\PYG{n}{df}\PYG{o}{.}\PYG{n}{groupby}\PYG{p}{(}\PYG{l+s+s1}{\PYGZsq{}}\PYG{l+s+s1}{grp}\PYG{l+s+s1}{\PYGZsq{}}\PYG{p}{)}\PYG{p}{[}\PYG{l+s+s1}{\PYGZsq{}}\PYG{l+s+s1}{v1}\PYG{l+s+s1}{\PYGZsq{}}\PYG{p}{]}\PYG{o}{.}\PYG{n}{apply}\PYG{p}{(}\PYG{k}{lambda} \PYG{n}{x}\PYG{p}{:} \PYG{n}{x}\PYG{o}{.}\PYG{n}{max}\PYG{p}{(}\PYG{p}{)} \PYG{o}{\PYGZhy{}} \PYG{n}{x}\PYG{o}{.}\PYG{n}{mean}\PYG{p}{(}\PYG{p}{)}\PYG{p}{)}
\end{sphinxVerbatim}

\end{sphinxuseclass}\end{sphinxVerbatimInput}
\begin{sphinxVerbatimOutput}

\begin{sphinxuseclass}{cell_output}
\begin{sphinxVerbatim}[commandchars=\\\{\}]
grp
a    1.0
b    1.0
c    1.5
Name: v1, dtype: float64
\end{sphinxVerbatim}

\end{sphinxuseclass}\end{sphinxVerbatimOutput}

\end{sphinxuseclass}
\begin{sphinxuseclass}{cell}\begin{sphinxVerbatimInput}

\begin{sphinxuseclass}{cell_input}
\begin{sphinxVerbatim}[commandchars=\\\{\}]
\PYG{n}{df}\PYG{o}{.}\PYG{n}{groupby}\PYG{p}{(}\PYG{l+s+s1}{\PYGZsq{}}\PYG{l+s+s1}{grp}\PYG{l+s+s1}{\PYGZsq{}}\PYG{p}{)}\PYG{p}{[}\PYG{l+s+s1}{\PYGZsq{}}\PYG{l+s+s1}{v1}\PYG{l+s+s1}{\PYGZsq{}}\PYG{p}{]}\PYG{o}{.}\PYG{n}{transform}\PYG{p}{(}\PYG{k}{lambda} \PYG{n}{x}\PYG{p}{:} \PYG{n}{x}\PYG{o}{.}\PYG{n}{max}\PYG{p}{(}\PYG{p}{)} \PYG{o}{\PYGZhy{}} \PYG{n}{x}\PYG{o}{.}\PYG{n}{mean}\PYG{p}{(}\PYG{p}{)}\PYG{p}{)}
\end{sphinxVerbatim}

\end{sphinxuseclass}\end{sphinxVerbatimInput}
\begin{sphinxVerbatimOutput}

\begin{sphinxuseclass}{cell_output}
\begin{sphinxVerbatim}[commandchars=\\\{\}]
0    1.0
1    1.0
2    1.0
3    1.0
4    1.0
5    1.0
6    1.5
7    1.5
8    1.5
9    1.5
Name: v1, dtype: float64
\end{sphinxVerbatim}

\end{sphinxuseclass}\end{sphinxVerbatimOutput}

\end{sphinxuseclass}

\section{pd.cut / pd.qcut}
\label{\detokenize{chapter2/2.2.3_Useful_Techniques:br-pd-cut-pd-qcut}}
\sphinxAtStartPar
이번에는 그룹핑을 하기 위해 활용되는 pd.qcut() 혹은 pd.cut() 메소드에 대하여 알아보겠습니다. 어떤 변수를 그룹 별로 분석하고 싶습니다. 예를 들어, 섹터가 그룹이라면 groupby(‘섹터’){[}‘수익률’{]}.mean() 명령으로 섹터별로 각 섹터에 속하는 종목들의 수익율 평균울 구할 수 있습니다. 하지만 그룹 변수가 없고 연속형 변수를 구간으로 나누어 그룹화하고 싶은 경우 pd.cut() 나 pd.qcut() 을 이용합니다. pd.cut 은 직접 구간을 지정해 그룹을 만들고, pd.qcut 은 분위 수를 이용하여 구간을 만듭니다.

\begin{sphinxVerbatim}[commandchars=\\\{\}]
\PYG{n}{pd}\PYG{o}{.}\PYG{n}{qcut}\PYG{p}{(}\PYG{n}{Series}\PYG{p}{,} \PYG{n}{q}\PYG{o}{=}\PYG{l+m+mi}{10}\PYG{p}{)} \PYG{c+c1}{\PYGZsh{} 십 분위수로 구간 생성}
\PYG{n}{pd}\PYG{o}{.}\PYG{n}{cut}\PYG{p}{(}\PYG{n}{Series}\PYG{p}{,} \PYG{n}{bins}\PYG{o}{=}\PYG{p}{[}\PYG{n}{a1}\PYG{p}{,} \PYG{n}{a2}\PYG{p}{,} \PYG{n}{a3}\PYG{p}{]}\PYG{p}{)} \PYG{c+c1}{\PYGZsh{} bins 인수를 이용하면  (a1, a2], (a2, a3]로 구간 생성}
\end{sphinxVerbatim}

\sphinxAtStartPar
np.arange(100) 을 이용하여 0 부터 99까지 값을 생성한 후 pd.qcut 와 pd.cut 을 사용 해 보겠습니다.

\begin{sphinxuseclass}{cell}\begin{sphinxVerbatimInput}

\begin{sphinxuseclass}{cell_input}
\begin{sphinxVerbatim}[commandchars=\\\{\}]
\PYG{k+kn}{import} \PYG{n+nn}{numpy} \PYG{k}{as} \PYG{n+nn}{np}
\PYG{k+kn}{import} \PYG{n+nn}{pandas} \PYG{k}{as} \PYG{n+nn}{pd}

\PYG{n}{a\PYGZus{}list} \PYG{o}{=} \PYG{n}{np}\PYG{o}{.}\PYG{n}{arange}\PYG{p}{(}\PYG{l+m+mi}{100}\PYG{p}{)}
\PYG{n}{df} \PYG{o}{=} \PYG{n}{pd}\PYG{o}{.}\PYG{n}{DataFrame}\PYG{p}{(}\PYG{p}{\PYGZob{}}\PYG{l+s+s1}{\PYGZsq{}}\PYG{l+s+s1}{a}\PYG{l+s+s1}{\PYGZsq{}}\PYG{p}{:} \PYG{n}{a\PYGZus{}list}\PYG{p}{\PYGZcb{}}\PYG{p}{)} 
\PYG{n}{df}\PYG{o}{.}\PYG{n}{head}\PYG{p}{(}\PYG{p}{)}
\end{sphinxVerbatim}

\end{sphinxuseclass}\end{sphinxVerbatimInput}
\begin{sphinxVerbatimOutput}

\begin{sphinxuseclass}{cell_output}
\begin{sphinxVerbatim}[commandchars=\\\{\}]
   a
0  0
1  1
2  2
3  3
4  4
\end{sphinxVerbatim}

\end{sphinxuseclass}\end{sphinxVerbatimOutput}

\end{sphinxuseclass}
\sphinxAtStartPar
10 이하의 숫자와 90 초과의 숫자는 해댱 구간이 없어서 그룹핑이 되지 않았습니다.

\begin{sphinxuseclass}{cell}\begin{sphinxVerbatimInput}

\begin{sphinxuseclass}{cell_input}
\begin{sphinxVerbatim}[commandchars=\\\{\}]
\PYG{n}{rank} \PYG{o}{=} \PYG{n}{pd}\PYG{o}{.}\PYG{n}{cut}\PYG{p}{(}\PYG{n}{df}\PYG{p}{[}\PYG{l+s+s1}{\PYGZsq{}}\PYG{l+s+s1}{a}\PYG{l+s+s1}{\PYGZsq{}}\PYG{p}{]}\PYG{p}{,} \PYG{n}{bins}\PYG{o}{=}\PYG{p}{[}\PYG{l+m+mi}{10}\PYG{p}{,} \PYG{l+m+mi}{25}\PYG{p}{,} \PYG{l+m+mi}{75}\PYG{p}{,} \PYG{l+m+mi}{90}\PYG{p}{]}\PYG{p}{)}
\PYG{n}{df}\PYG{o}{.}\PYG{n}{groupby}\PYG{p}{(}\PYG{n}{rank}\PYG{p}{)}\PYG{p}{[}\PYG{l+s+s1}{\PYGZsq{}}\PYG{l+s+s1}{a}\PYG{l+s+s1}{\PYGZsq{}}\PYG{p}{]}\PYG{o}{.}\PYG{n}{agg}\PYG{p}{(}\PYG{p}{[}\PYG{l+s+s1}{\PYGZsq{}}\PYG{l+s+s1}{min}\PYG{l+s+s1}{\PYGZsq{}}\PYG{p}{,}\PYG{l+s+s1}{\PYGZsq{}}\PYG{l+s+s1}{max}\PYG{l+s+s1}{\PYGZsq{}}\PYG{p}{,}\PYG{l+s+s1}{\PYGZsq{}}\PYG{l+s+s1}{count}\PYG{l+s+s1}{\PYGZsq{}}\PYG{p}{]}\PYG{p}{)}
\end{sphinxVerbatim}

\end{sphinxuseclass}\end{sphinxVerbatimInput}
\begin{sphinxVerbatimOutput}

\begin{sphinxuseclass}{cell_output}
\begin{sphinxVerbatim}[commandchars=\\\{\}]
          min  max  count
a                        
(10, 25]   11   25     15
(25, 75]   26   75     50
(75, 90]   76   90     15
\end{sphinxVerbatim}

\end{sphinxuseclass}\end{sphinxVerbatimOutput}

\end{sphinxuseclass}
\sphinxAtStartPar
bins 구간이 모든 값을 포함하도록 하기 위해서는 아래와 같이 bins 를 설정합니다.

\begin{sphinxuseclass}{cell}\begin{sphinxVerbatimInput}

\begin{sphinxuseclass}{cell_input}
\begin{sphinxVerbatim}[commandchars=\\\{\}]
\PYG{n}{rank} \PYG{o}{=} \PYG{n}{pd}\PYG{o}{.}\PYG{n}{cut}\PYG{p}{(}\PYG{n}{df}\PYG{p}{[}\PYG{l+s+s1}{\PYGZsq{}}\PYG{l+s+s1}{a}\PYG{l+s+s1}{\PYGZsq{}}\PYG{p}{]}\PYG{p}{,} \PYG{n}{bins}\PYG{o}{=}\PYG{p}{[}\PYG{o}{\PYGZhy{}}\PYG{n}{np}\PYG{o}{.}\PYG{n}{inf}\PYG{p}{,} \PYG{l+m+mi}{10}\PYG{p}{,} \PYG{l+m+mi}{25}\PYG{p}{,} \PYG{l+m+mi}{75}\PYG{p}{,} \PYG{l+m+mi}{90}\PYG{p}{,} \PYG{n}{np}\PYG{o}{.}\PYG{n}{inf}\PYG{p}{]}\PYG{p}{)}
\PYG{n}{df}\PYG{o}{.}\PYG{n}{groupby}\PYG{p}{(}\PYG{n}{rank}\PYG{p}{)}\PYG{o}{.}\PYG{n}{agg}\PYG{p}{(}\PYG{p}{[}\PYG{l+s+s1}{\PYGZsq{}}\PYG{l+s+s1}{min}\PYG{l+s+s1}{\PYGZsq{}}\PYG{p}{,}\PYG{l+s+s1}{\PYGZsq{}}\PYG{l+s+s1}{max}\PYG{l+s+s1}{\PYGZsq{}}\PYG{p}{,}\PYG{l+s+s1}{\PYGZsq{}}\PYG{l+s+s1}{count}\PYG{l+s+s1}{\PYGZsq{}}\PYG{p}{]}\PYG{p}{)}
\end{sphinxVerbatim}

\end{sphinxuseclass}\end{sphinxVerbatimInput}
\begin{sphinxVerbatimOutput}

\begin{sphinxuseclass}{cell_output}
\begin{sphinxVerbatim}[commandchars=\\\{\}]
               a          
             min max count
a                         
(\PYGZhy{}inf, 10.0]   0  10    11
(10.0, 25.0]  11  25    15
(25.0, 75.0]  26  75    50
(75.0, 90.0]  76  90    15
(90.0, inf]   91  99     9
\end{sphinxVerbatim}

\end{sphinxuseclass}\end{sphinxVerbatimOutput}

\end{sphinxuseclass}
\sphinxAtStartPar
이번에는 제가 주로 활용하는 pd.qcut 입니다. pd.cut 은 bins 로 구간을 설정해야 하나,
qcut 는 q 인수로 분위수를 이용하여 구간을 만듭니다. 아래 결과를 보시면 q 의 역할을 아실 수 있을 것이라고 생각합니다.

\begin{sphinxuseclass}{cell}\begin{sphinxVerbatimInput}

\begin{sphinxuseclass}{cell_input}
\begin{sphinxVerbatim}[commandchars=\\\{\}]
\PYG{n}{rank} \PYG{o}{=} \PYG{n}{pd}\PYG{o}{.}\PYG{n}{qcut}\PYG{p}{(}\PYG{n}{df}\PYG{p}{[}\PYG{l+s+s1}{\PYGZsq{}}\PYG{l+s+s1}{a}\PYG{l+s+s1}{\PYGZsq{}}\PYG{p}{]}\PYG{p}{,} \PYG{n}{q}\PYG{o}{=}\PYG{l+m+mi}{10}\PYG{p}{)}
\PYG{n}{df}\PYG{o}{.}\PYG{n}{groupby}\PYG{p}{(}\PYG{n}{rank}\PYG{p}{)}\PYG{p}{[}\PYG{l+s+s1}{\PYGZsq{}}\PYG{l+s+s1}{a}\PYG{l+s+s1}{\PYGZsq{}}\PYG{p}{]}\PYG{o}{.}\PYG{n}{agg}\PYG{p}{(}\PYG{p}{[}\PYG{l+s+s1}{\PYGZsq{}}\PYG{l+s+s1}{min}\PYG{l+s+s1}{\PYGZsq{}}\PYG{p}{,}\PYG{l+s+s1}{\PYGZsq{}}\PYG{l+s+s1}{max}\PYG{l+s+s1}{\PYGZsq{}}\PYG{p}{,}\PYG{l+s+s1}{\PYGZsq{}}\PYG{l+s+s1}{count}\PYG{l+s+s1}{\PYGZsq{}}\PYG{p}{]}\PYG{p}{)}
\end{sphinxVerbatim}

\end{sphinxuseclass}\end{sphinxVerbatimInput}
\begin{sphinxVerbatimOutput}

\begin{sphinxuseclass}{cell_output}
\begin{sphinxVerbatim}[commandchars=\\\{\}]
               min  max  count
a                             
(\PYGZhy{}0.001, 9.9]    0    9     10
(9.9, 19.8]     10   19     10
(19.8, 29.7]    20   29     10
(29.7, 39.6]    30   39     10
(39.6, 49.5]    40   49     10
(49.5, 59.4]    50   59     10
(59.4, 69.3]    60   69     10
(69.3, 79.2]    70   79     10
(79.2, 89.1]    80   89     10
(89.1, 99.0]    90   99     10
\end{sphinxVerbatim}

\end{sphinxuseclass}\end{sphinxVerbatimOutput}

\end{sphinxuseclass}
\begin{sphinxuseclass}{cell}\begin{sphinxVerbatimInput}

\begin{sphinxuseclass}{cell_input}
\begin{sphinxVerbatim}[commandchars=\\\{\}]
\PYG{n}{rank} \PYG{o}{=} \PYG{n}{pd}\PYG{o}{.}\PYG{n}{qcut}\PYG{p}{(}\PYG{n}{df}\PYG{p}{[}\PYG{l+s+s1}{\PYGZsq{}}\PYG{l+s+s1}{a}\PYG{l+s+s1}{\PYGZsq{}}\PYG{p}{]}\PYG{p}{,} \PYG{n}{q}\PYG{o}{=}\PYG{l+m+mi}{5}\PYG{p}{)}
\PYG{n}{df}\PYG{o}{.}\PYG{n}{groupby}\PYG{p}{(}\PYG{n}{rank}\PYG{p}{)}\PYG{p}{[}\PYG{l+s+s1}{\PYGZsq{}}\PYG{l+s+s1}{a}\PYG{l+s+s1}{\PYGZsq{}}\PYG{p}{]}\PYG{o}{.}\PYG{n}{agg}\PYG{p}{(}\PYG{p}{[}\PYG{l+s+s1}{\PYGZsq{}}\PYG{l+s+s1}{min}\PYG{l+s+s1}{\PYGZsq{}}\PYG{p}{,}\PYG{l+s+s1}{\PYGZsq{}}\PYG{l+s+s1}{max}\PYG{l+s+s1}{\PYGZsq{}}\PYG{p}{,}\PYG{l+s+s1}{\PYGZsq{}}\PYG{l+s+s1}{count}\PYG{l+s+s1}{\PYGZsq{}}\PYG{p}{]}\PYG{p}{)}
\end{sphinxVerbatim}

\end{sphinxuseclass}\end{sphinxVerbatimInput}
\begin{sphinxVerbatimOutput}

\begin{sphinxuseclass}{cell_output}
\begin{sphinxVerbatim}[commandchars=\\\{\}]
                min  max  count
a                              
(\PYGZhy{}0.001, 19.8]    0   19     20
(19.8, 39.6]     20   39     20
(39.6, 59.4]     40   59     20
(59.4, 79.2]     60   79     20
(79.2, 99.0]     80   99     20
\end{sphinxVerbatim}

\end{sphinxuseclass}\end{sphinxVerbatimOutput}

\end{sphinxuseclass}

\section{Resample}
\label{\detokenize{chapter2/2.2.4_Useful_Techniques:resample}}\label{\detokenize{chapter2/2.2.4_Useful_Techniques::doc}}
\sphinxAtStartPar
Resample 은 시간데이터를 다른 시간 단위로 변경하고 싶을 때 활용합니다. 예를 들면, 초 단위 데이터를 일단위 혹은 월단위 데이터로 변경 할 수 있습니다. 연습을 위하여 시간 레벨의 데이터가 필요합니다. 시간레벨 데이터는 FinanceDataReader 패키지에서 제공하는 일봉 데이터를 활용하겠습니다. FinanceDataReader 는 이승준님이 금융자료 분석을 하시는 분들을 위하여 만들어 주신 정말 유용한 패키지입니다. 자세한 내용은 아래 링크에 설명이 되어 있습니다.
\sphinxurl{https://financedata.github.io/posts/finance-data-reader-users-guide.html} 또한, 이승준님이 Pycon 에서 엑셀에 비하여 파이썬의 장점에 대하여 강연하시는 내용이 유투브에 있습니다. \sphinxurl{https://www.youtube.com/watch?v=w7Q\_eKN5r-I}


\section{FinanceDataReader}
\label{\detokenize{chapter2/2.2.4_Useful_Techniques:financedatareader}}
\sphinxAtStartPar
FinanceDataReader 를 import 합니다. DataReader 함수에 종목코드, 시작일, 종료일을 인수로 넣어주면 아래와 같이 일봉데이터를 리턴합니다. 출력해보면 Date 가 index 로 되어 있음을 알 수 있습니다.

\begin{sphinxuseclass}{cell}\begin{sphinxVerbatimInput}

\begin{sphinxuseclass}{cell_input}
\begin{sphinxVerbatim}[commandchars=\\\{\}]
\PYG{k+kn}{import} \PYG{n+nn}{FinanceDataReader} \PYG{k}{as} \PYG{n+nn}{fdr}

\PYG{n}{code} \PYG{o}{=} \PYG{l+s+s1}{\PYGZsq{}}\PYG{l+s+s1}{005930}\PYG{l+s+s1}{\PYGZsq{}} \PYG{c+c1}{\PYGZsh{} 삼성전자}
\PYG{n}{stock\PYGZus{}data} \PYG{o}{=} \PYG{n}{fdr}\PYG{o}{.}\PYG{n}{DataReader}\PYG{p}{(}\PYG{n}{code}\PYG{p}{,} \PYG{n}{start}\PYG{o}{=}\PYG{l+s+s1}{\PYGZsq{}}\PYG{l+s+s1}{2021\PYGZhy{}01\PYGZhy{}03}\PYG{l+s+s1}{\PYGZsq{}}\PYG{p}{,} \PYG{n}{end}\PYG{o}{=}\PYG{l+s+s1}{\PYGZsq{}}\PYG{l+s+s1}{2021\PYGZhy{}12\PYGZhy{}31}\PYG{l+s+s1}{\PYGZsq{}}\PYG{p}{)} 

\PYG{n}{stock\PYGZus{}data}\PYG{o}{.}\PYG{n}{head}\PYG{p}{(}\PYG{p}{)}\PYG{o}{.}\PYG{n}{style}\PYG{o}{.}\PYG{n}{set\PYGZus{}table\PYGZus{}attributes}\PYG{p}{(}\PYG{l+s+s1}{\PYGZsq{}}\PYG{l+s+s1}{style=}\PYG{l+s+s1}{\PYGZdq{}}\PYG{l+s+s1}{font\PYGZhy{}size: 12px}\PYG{l+s+s1}{\PYGZdq{}}\PYG{l+s+s1}{\PYGZsq{}}\PYG{p}{)} \PYG{c+c1}{\PYGZsh{} head 메소드는 처음 5 row 만 출력합니다.}
\end{sphinxVerbatim}

\end{sphinxuseclass}\end{sphinxVerbatimInput}
\begin{sphinxVerbatimOutput}

\begin{sphinxuseclass}{cell_output}
\begin{sphinxVerbatim}[commandchars=\\\{\}]
\PYGZlt{}pandas.io.formats.style.Styler at 0x1eab9282310\PYGZgt{}
\end{sphinxVerbatim}

\end{sphinxuseclass}\end{sphinxVerbatimOutput}

\end{sphinxuseclass}


\begin{sphinxuseclass}{cell}\begin{sphinxVerbatimInput}

\begin{sphinxuseclass}{cell_input}
\begin{sphinxVerbatim}[commandchars=\\\{\}]
\PYG{k+kn}{import} \PYG{n+nn}{pandas} \PYG{k}{as} \PYG{n+nn}{pd}
\PYG{n}{pd}\PYG{o}{.}\PYG{n}{options}\PYG{o}{.}\PYG{n}{display}\PYG{o}{.}\PYG{n}{float\PYGZus{}format} \PYG{o}{=} \PYG{l+s+s1}{\PYGZsq{}}\PYG{l+s+si}{\PYGZob{}:,.0f\PYGZcb{}}\PYG{l+s+s1}{\PYGZsq{}}\PYG{o}{.}\PYG{n}{format}
\PYG{n}{stock\PYGZus{}data}\PYG{o}{.}\PYG{n}{resample}\PYG{p}{(}\PYG{l+s+s1}{\PYGZsq{}}\PYG{l+s+s1}{M}\PYG{l+s+s1}{\PYGZsq{}}\PYG{p}{)}\PYG{p}{[}\PYG{l+s+s1}{\PYGZsq{}}\PYG{l+s+s1}{Close}\PYG{l+s+s1}{\PYGZsq{}}\PYG{p}{]}\PYG{o}{.}\PYG{n}{agg}\PYG{p}{(}\PYG{p}{[}\PYG{l+s+s1}{\PYGZsq{}}\PYG{l+s+s1}{mean}\PYG{l+s+s1}{\PYGZsq{}}\PYG{p}{,}\PYG{l+s+s1}{\PYGZsq{}}\PYG{l+s+s1}{max}\PYG{l+s+s1}{\PYGZsq{}}\PYG{p}{,}\PYG{l+s+s1}{\PYGZsq{}}\PYG{l+s+s1}{min}\PYG{l+s+s1}{\PYGZsq{}}\PYG{p}{]}\PYG{p}{)}\PYG{o}{.}\PYG{n}{head}\PYG{p}{(}\PYG{p}{)}\PYG{o}{.}\PYG{n}{style}\PYG{o}{.}\PYG{n}{set\PYGZus{}table\PYGZus{}attributes}\PYG{p}{(}\PYG{l+s+s1}{\PYGZsq{}}\PYG{l+s+s1}{style=}\PYG{l+s+s1}{\PYGZdq{}}\PYG{l+s+s1}{font\PYGZhy{}size: 12px}\PYG{l+s+s1}{\PYGZdq{}}\PYG{l+s+s1}{\PYGZsq{}}\PYG{p}{)} \PYG{c+c1}{\PYGZsh{} 처음 5개만 출력}
\end{sphinxVerbatim}

\end{sphinxuseclass}\end{sphinxVerbatimInput}
\begin{sphinxVerbatimOutput}

\begin{sphinxuseclass}{cell_output}
\begin{sphinxVerbatim}[commandchars=\\\{\}]
\PYGZlt{}pandas.io.formats.style.Styler at 0x1eab69f9af0\PYGZgt{}
\end{sphinxVerbatim}

\end{sphinxuseclass}\end{sphinxVerbatimOutput}

\end{sphinxuseclass}
\sphinxAtStartPar
주별로 요약할 수 도 있습니다. 이번에는 resample(‘W’) 라고 해 줍니다. Resample 이 정말 유용한 기능이라는 것을 직감하셨을 것으로 생각합니다. 역시 한 주(월요일 \textasciitilde{} 일요일)의 마지막날이 Index 로 들어가 있습니다. 디폴트는 일요일입니다.

\begin{sphinxuseclass}{cell}\begin{sphinxVerbatimInput}

\begin{sphinxuseclass}{cell_input}
\begin{sphinxVerbatim}[commandchars=\\\{\}]
\PYG{n}{pd}\PYG{o}{.}\PYG{n}{options}\PYG{o}{.}\PYG{n}{display}\PYG{o}{.}\PYG{n}{float\PYGZus{}format} \PYG{o}{=} \PYG{l+s+s1}{\PYGZsq{}}\PYG{l+s+si}{\PYGZob{}:,.0f\PYGZcb{}}\PYG{l+s+s1}{\PYGZsq{}}\PYG{o}{.}\PYG{n}{format}
\PYG{n}{stock\PYGZus{}data}\PYG{o}{.}\PYG{n}{resample}\PYG{p}{(}\PYG{l+s+s1}{\PYGZsq{}}\PYG{l+s+s1}{W}\PYG{l+s+s1}{\PYGZsq{}}\PYG{p}{)}\PYG{p}{[}\PYG{l+s+s1}{\PYGZsq{}}\PYG{l+s+s1}{Close}\PYG{l+s+s1}{\PYGZsq{}}\PYG{p}{]}\PYG{o}{.}\PYG{n}{agg}\PYG{p}{(}\PYG{p}{[}\PYG{l+s+s1}{\PYGZsq{}}\PYG{l+s+s1}{mean}\PYG{l+s+s1}{\PYGZsq{}}\PYG{p}{,}\PYG{l+s+s1}{\PYGZsq{}}\PYG{l+s+s1}{max}\PYG{l+s+s1}{\PYGZsq{}}\PYG{p}{,}\PYG{l+s+s1}{\PYGZsq{}}\PYG{l+s+s1}{min}\PYG{l+s+s1}{\PYGZsq{}}\PYG{p}{]}\PYG{p}{)}\PYG{o}{.}\PYG{n}{head}\PYG{p}{(}\PYG{p}{)}\PYG{o}{.}\PYG{n}{style}\PYG{o}{.}\PYG{n}{set\PYGZus{}table\PYGZus{}attributes}\PYG{p}{(}\PYG{l+s+s1}{\PYGZsq{}}\PYG{l+s+s1}{style=}\PYG{l+s+s1}{\PYGZdq{}}\PYG{l+s+s1}{font\PYGZhy{}size: 12px}\PYG{l+s+s1}{\PYGZdq{}}\PYG{l+s+s1}{\PYGZsq{}}\PYG{p}{)}
\end{sphinxVerbatim}

\end{sphinxuseclass}\end{sphinxVerbatimInput}
\begin{sphinxVerbatimOutput}

\begin{sphinxuseclass}{cell_output}
\begin{sphinxVerbatim}[commandchars=\\\{\}]
\PYGZlt{}pandas.io.formats.style.Styler at 0x1eab947fdc0\PYGZgt{}
\end{sphinxVerbatim}

\end{sphinxuseclass}\end{sphinxVerbatimOutput}

\end{sphinxuseclass}

\section{Pickle}
\label{\detokenize{chapter2/2.2.5_Useful_Techniques:pickle}}\label{\detokenize{chapter2/2.2.5_Useful_Techniques::doc}}
\sphinxAtStartPar
Pickle 은 사전적으로 절여서 저장해 놓는다는 말인데요. 파이썬에서 데이터를 저장해 놓을 때 쓰는 패키지입니다. 파이썬 언어로 만들어진 데이터는 RAM 메모리에 존재합니다. 따라서, 컴퓨터가 꺼지면 자동으로 데이터가 사라지게 됩니다. 그래서, 저는 pickle 를 이용해서 데이터 작업 중간에 데이터를 저장합니다. 파이썬 DataFrame 의 저장은 csv, excel, json 등 다양한 형식으로 저장할 수 있으나, 파이썬의 데이터 타입을 손상시키지 않고, 원형대로 저장하고 불러올 수 있는 pickle 이 제일 편리합니다. 삼성전자 일봉데이터를 가져와서 피클로 저장해 보겠습니다.

\begin{sphinxuseclass}{cell}\begin{sphinxVerbatimInput}

\begin{sphinxuseclass}{cell_input}
\begin{sphinxVerbatim}[commandchars=\\\{\}]
\PYG{k+kn}{import} \PYG{n+nn}{FinanceDataReader} \PYG{k}{as} \PYG{n+nn}{fdr} 

\PYG{n}{code} \PYG{o}{=} \PYG{l+s+s1}{\PYGZsq{}}\PYG{l+s+s1}{005930}\PYG{l+s+s1}{\PYGZsq{}} \PYG{c+c1}{\PYGZsh{} 삼성전자}
\PYG{n}{stock\PYGZus{}data} \PYG{o}{=} \PYG{n}{fdr}\PYG{o}{.}\PYG{n}{DataReader}\PYG{p}{(}\PYG{n}{code}\PYG{p}{,} \PYG{n}{start}\PYG{o}{=}\PYG{l+s+s1}{\PYGZsq{}}\PYG{l+s+s1}{2021\PYGZhy{}01\PYGZhy{}03}\PYG{l+s+s1}{\PYGZsq{}}\PYG{p}{,} \PYG{n}{end}\PYG{o}{=}\PYG{l+s+s1}{\PYGZsq{}}\PYG{l+s+s1}{2021\PYGZhy{}12\PYGZhy{}31}\PYG{l+s+s1}{\PYGZsq{}}\PYG{p}{)} 

\PYG{n}{stock\PYGZus{}data}\PYG{o}{.}\PYG{n}{to\PYGZus{}pickle}\PYG{p}{(}\PYG{l+s+s1}{\PYGZsq{}}\PYG{l+s+s1}{stock\PYGZus{}data.pkl}\PYG{l+s+s1}{\PYGZsq{}}\PYG{p}{)} \PYG{c+c1}{\PYGZsh{} 디렉토리를 지정하지 않으면 현재 작업 폴더에 저장이 됩니다.}
\end{sphinxVerbatim}

\end{sphinxuseclass}\end{sphinxVerbatimInput}

\end{sphinxuseclass}


\begin{sphinxuseclass}{cell}\begin{sphinxVerbatimInput}

\begin{sphinxuseclass}{cell_input}
\begin{sphinxVerbatim}[commandchars=\\\{\}]
\PYG{k+kn}{import} \PYG{n+nn}{pandas} \PYG{k}{as} \PYG{n+nn}{pd}
\PYG{n}{stock\PYGZus{}data} \PYG{o}{=} \PYG{n}{pd}\PYG{o}{.}\PYG{n}{read\PYGZus{}pickle}\PYG{p}{(}\PYG{l+s+s1}{\PYGZsq{}}\PYG{l+s+s1}{stock\PYGZus{}data.pkl}\PYG{l+s+s1}{\PYGZsq{}}\PYG{p}{)}
\PYG{n}{stock\PYGZus{}data}\PYG{o}{.}\PYG{n}{head}\PYG{p}{(}\PYG{p}{)}\PYG{o}{.}\PYG{n}{style}\PYG{o}{.}\PYG{n}{set\PYGZus{}table\PYGZus{}attributes}\PYG{p}{(}\PYG{l+s+s1}{\PYGZsq{}}\PYG{l+s+s1}{style=}\PYG{l+s+s1}{\PYGZdq{}}\PYG{l+s+s1}{font\PYGZhy{}size: 12px}\PYG{l+s+s1}{\PYGZdq{}}\PYG{l+s+s1}{\PYGZsq{}}\PYG{p}{)}
\end{sphinxVerbatim}

\end{sphinxuseclass}\end{sphinxVerbatimInput}
\begin{sphinxVerbatimOutput}

\begin{sphinxuseclass}{cell_output}
\begin{sphinxVerbatim}[commandchars=\\\{\}]
\PYGZlt{}pandas.io.formats.style.Styler at 0x23a6792aa90\PYGZgt{}
\end{sphinxVerbatim}

\end{sphinxuseclass}\end{sphinxVerbatimOutput}

\end{sphinxuseclass}


\begin{sphinxuseclass}{cell}\begin{sphinxVerbatimInput}

\begin{sphinxuseclass}{cell_input}
\begin{sphinxVerbatim}[commandchars=\\\{\}]
\PYG{k+kn}{import} \PYG{n+nn}{pickle}

\PYG{k}{with} \PYG{n+nb}{open}\PYG{p}{(}\PYG{l+s+s1}{\PYGZsq{}}\PYG{l+s+s1}{stock\PYGZus{}data.pkl}\PYG{l+s+s1}{\PYGZsq{}}\PYG{p}{,} \PYG{l+s+s1}{\PYGZsq{}}\PYG{l+s+s1}{wb}\PYG{l+s+s1}{\PYGZsq{}}\PYG{p}{)} \PYG{k}{as} \PYG{n}{file}\PYG{p}{:}    \PYG{c+c1}{\PYGZsh{} Binary 파일로 저징}
    \PYG{n}{pickle}\PYG{o}{.}\PYG{n}{dump}\PYG{p}{(}\PYG{n}{stock\PYGZus{}data}\PYG{p}{,} \PYG{n}{file}\PYG{p}{)}
    
\PYG{k}{with} \PYG{n+nb}{open}\PYG{p}{(}\PYG{l+s+s1}{\PYGZsq{}}\PYG{l+s+s1}{stock\PYGZus{}data.pkl}\PYG{l+s+s1}{\PYGZsq{}}\PYG{p}{,} \PYG{l+s+s1}{\PYGZsq{}}\PYG{l+s+s1}{rb}\PYG{l+s+s1}{\PYGZsq{}}\PYG{p}{)} \PYG{k}{as} \PYG{n}{file}\PYG{p}{:}    \PYG{c+c1}{\PYGZsh{} 저장된 binary 파일 읽기}
    \PYG{n}{stock\PYGZus{}data} \PYG{o}{=} \PYG{n}{pickle}\PYG{o}{.}\PYG{n}{load}\PYG{p}{(}\PYG{n}{file}\PYG{p}{)}    
\end{sphinxVerbatim}

\end{sphinxuseclass}\end{sphinxVerbatimInput}

\end{sphinxuseclass}
\begin{sphinxuseclass}{cell}\begin{sphinxVerbatimInput}

\begin{sphinxuseclass}{cell_input}
\begin{sphinxVerbatim}[commandchars=\\\{\}]
\PYG{n}{stock\PYGZus{}data}\PYG{o}{.}\PYG{n}{head}\PYG{p}{(}\PYG{p}{)}\PYG{o}{.}\PYG{n}{style}\PYG{o}{.}\PYG{n}{set\PYGZus{}table\PYGZus{}attributes}\PYG{p}{(}\PYG{l+s+s1}{\PYGZsq{}}\PYG{l+s+s1}{style=}\PYG{l+s+s1}{\PYGZdq{}}\PYG{l+s+s1}{font\PYGZhy{}size: 12px}\PYG{l+s+s1}{\PYGZdq{}}\PYG{l+s+s1}{\PYGZsq{}}\PYG{p}{)}
\end{sphinxVerbatim}

\end{sphinxuseclass}\end{sphinxVerbatimInput}
\begin{sphinxVerbatimOutput}

\begin{sphinxuseclass}{cell_output}
\begin{sphinxVerbatim}[commandchars=\\\{\}]
\PYGZlt{}pandas.io.formats.style.Styler at 0x23a67948880\PYGZgt{}
\end{sphinxVerbatim}

\end{sphinxuseclass}\end{sphinxVerbatimOutput}

\end{sphinxuseclass}

\section{Shift}
\label{\detokenize{chapter2/2.2.6_Useful_Techniques:shift}}\label{\detokenize{chapter2/2.2.6_Useful_Techniques::doc}}
\sphinxAtStartPar
Shift 은 이전 row 나 이후 row 에 있는 값을 가져올 수 있는 메소드입니다. 일단 삼성전자 일봉을 가져오겠습니다.

\begin{sphinxuseclass}{cell}\begin{sphinxVerbatimInput}

\begin{sphinxuseclass}{cell_input}
\begin{sphinxVerbatim}[commandchars=\\\{\}]
\PYG{k+kn}{import} \PYG{n+nn}{FinanceDataReader} \PYG{k}{as} \PYG{n+nn}{fdr} 

\PYG{n}{code} \PYG{o}{=} \PYG{l+s+s1}{\PYGZsq{}}\PYG{l+s+s1}{005930}\PYG{l+s+s1}{\PYGZsq{}} \PYG{c+c1}{\PYGZsh{} 삼성전자}
\PYG{n}{stock\PYGZus{}data} \PYG{o}{=} \PYG{n}{fdr}\PYG{o}{.}\PYG{n}{DataReader}\PYG{p}{(}\PYG{n}{code}\PYG{p}{,} \PYG{n}{start}\PYG{o}{=}\PYG{l+s+s1}{\PYGZsq{}}\PYG{l+s+s1}{2021\PYGZhy{}01\PYGZhy{}03}\PYG{l+s+s1}{\PYGZsq{}}\PYG{p}{,} \PYG{n}{end}\PYG{o}{=}\PYG{l+s+s1}{\PYGZsq{}}\PYG{l+s+s1}{2021\PYGZhy{}12\PYGZhy{}31}\PYG{l+s+s1}{\PYGZsq{}}\PYG{p}{)} 

\PYG{n}{stock\PYGZus{}data}\PYG{o}{.}\PYG{n}{head}\PYG{p}{(}\PYG{p}{)}\PYG{o}{.}\PYG{n}{style}\PYG{o}{.}\PYG{n}{set\PYGZus{}table\PYGZus{}attributes}\PYG{p}{(}\PYG{l+s+s1}{\PYGZsq{}}\PYG{l+s+s1}{style=}\PYG{l+s+s1}{\PYGZdq{}}\PYG{l+s+s1}{font\PYGZhy{}size: 12px}\PYG{l+s+s1}{\PYGZdq{}}\PYG{l+s+s1}{\PYGZsq{}}\PYG{p}{)}
\end{sphinxVerbatim}

\end{sphinxuseclass}\end{sphinxVerbatimInput}
\begin{sphinxVerbatimOutput}

\begin{sphinxuseclass}{cell_output}
\begin{sphinxVerbatim}[commandchars=\\\{\}]
\PYGZlt{}pandas.io.formats.style.Styler at 0x14e71565940\PYGZgt{}
\end{sphinxVerbatim}

\end{sphinxuseclass}\end{sphinxVerbatimOutput}

\end{sphinxuseclass}


\begin{sphinxuseclass}{cell}\begin{sphinxVerbatimInput}

\begin{sphinxuseclass}{cell_input}
\begin{sphinxVerbatim}[commandchars=\\\{\}]
\PYG{n}{stock\PYGZus{}data}\PYG{p}{[}\PYG{l+s+s1}{\PYGZsq{}}\PYG{l+s+s1}{Previous Close}\PYG{l+s+s1}{\PYGZsq{}}\PYG{p}{]} \PYG{o}{=} \PYG{n}{stock\PYGZus{}data}\PYG{p}{[}\PYG{l+s+s1}{\PYGZsq{}}\PYG{l+s+s1}{Close}\PYG{l+s+s1}{\PYGZsq{}}\PYG{p}{]}\PYG{o}{.}\PYG{n}{shift}\PYG{p}{(}\PYG{l+m+mi}{1}\PYG{p}{)}
\PYG{n}{stock\PYGZus{}data}\PYG{o}{.}\PYG{n}{head}\PYG{p}{(}\PYG{l+m+mi}{6}\PYG{p}{)}\PYG{o}{.}\PYG{n}{style}\PYG{o}{.}\PYG{n}{set\PYGZus{}table\PYGZus{}attributes}\PYG{p}{(}\PYG{l+s+s1}{\PYGZsq{}}\PYG{l+s+s1}{style=}\PYG{l+s+s1}{\PYGZdq{}}\PYG{l+s+s1}{font\PYGZhy{}size: 12px}\PYG{l+s+s1}{\PYGZdq{}}\PYG{l+s+s1}{\PYGZsq{}}\PYG{p}{)}
\end{sphinxVerbatim}

\end{sphinxuseclass}\end{sphinxVerbatimInput}
\begin{sphinxVerbatimOutput}

\begin{sphinxuseclass}{cell_output}
\begin{sphinxVerbatim}[commandchars=\\\{\}]
\PYGZlt{}pandas.io.formats.style.Styler at 0x14e6ec2af40\PYGZgt{}
\end{sphinxVerbatim}

\end{sphinxuseclass}\end{sphinxVerbatimOutput}

\end{sphinxuseclass}


\sphinxAtStartPar
그 다음 수익율 데이터를 만들어 보겠습니다. 내일의 시가는 stock\_data{[}‘Open’{]}.shift(\sphinxhyphen{}1), 내일의 종가는 stock\_data{[}‘Close’{]}.shift(\sphinxhyphen{}1) 로 가져오면 됩니다. 결과를 컬럼 ‘return’ 에 넣겠습니다. shift(1) 는 전날의 정보를 shift(\sphinxhyphen{}1) 은 다음날의 데이터를 가져옵니다.

\begin{sphinxuseclass}{cell}\begin{sphinxVerbatimInput}

\begin{sphinxuseclass}{cell_input}
\begin{sphinxVerbatim}[commandchars=\\\{\}]
\PYG{n}{stock\PYGZus{}data}\PYG{p}{[}\PYG{l+s+s1}{\PYGZsq{}}\PYG{l+s+s1}{buy}\PYG{l+s+s1}{\PYGZsq{}}\PYG{p}{]} \PYG{o}{=} \PYG{p}{(}\PYG{n}{stock\PYGZus{}data}\PYG{p}{[}\PYG{l+s+s1}{\PYGZsq{}}\PYG{l+s+s1}{Close}\PYG{l+s+s1}{\PYGZsq{}}\PYG{p}{]} \PYG{o}{\PYGZgt{}} \PYG{n}{stock\PYGZus{}data}\PYG{p}{[}\PYG{l+s+s1}{\PYGZsq{}}\PYG{l+s+s1}{Previous Close}\PYG{l+s+s1}{\PYGZsq{}}\PYG{p}{]}\PYG{p}{)}\PYG{o}{.}\PYG{n}{astype}\PYG{p}{(}\PYG{n+nb}{int}\PYG{p}{)} \PYG{c+c1}{\PYGZsh{} 매수 시그널 생성}
\PYG{n}{stock\PYGZus{}data}\PYG{p}{[}\PYG{l+s+s1}{\PYGZsq{}}\PYG{l+s+s1}{return}\PYG{l+s+s1}{\PYGZsq{}}\PYG{p}{]} \PYG{o}{=} \PYG{n}{stock\PYGZus{}data}\PYG{p}{[}\PYG{l+s+s1}{\PYGZsq{}}\PYG{l+s+s1}{Close}\PYG{l+s+s1}{\PYGZsq{}}\PYG{p}{]}\PYG{o}{.}\PYG{n}{shift}\PYG{p}{(}\PYG{o}{\PYGZhy{}}\PYG{l+m+mi}{1}\PYG{p}{)} \PYG{o}{/} \PYG{n}{stock\PYGZus{}data}\PYG{p}{[}\PYG{l+s+s1}{\PYGZsq{}}\PYG{l+s+s1}{Open}\PYG{l+s+s1}{\PYGZsq{}}\PYG{p}{]}\PYG{o}{.}\PYG{n}{shift}\PYG{p}{(}\PYG{o}{\PYGZhy{}}\PYG{l+m+mi}{1}\PYG{p}{)} \PYG{c+c1}{\PYGZsh{} 전략의 수익율}
\PYG{n}{stock\PYGZus{}data}\PYG{o}{.}\PYG{n}{head}\PYG{p}{(}\PYG{l+m+mi}{6}\PYG{p}{)}\PYG{o}{.}\PYG{n}{style}\PYG{o}{.}\PYG{n}{set\PYGZus{}table\PYGZus{}attributes}\PYG{p}{(}\PYG{l+s+s1}{\PYGZsq{}}\PYG{l+s+s1}{style=}\PYG{l+s+s1}{\PYGZdq{}}\PYG{l+s+s1}{font\PYGZhy{}size: 12px}\PYG{l+s+s1}{\PYGZdq{}}\PYG{l+s+s1}{\PYGZsq{}}\PYG{p}{)}
\end{sphinxVerbatim}

\end{sphinxuseclass}\end{sphinxVerbatimInput}
\begin{sphinxVerbatimOutput}

\begin{sphinxuseclass}{cell_output}
\begin{sphinxVerbatim}[commandchars=\\\{\}]
\PYGZlt{}pandas.io.formats.style.Styler at 0x14e6ece0f40\PYGZgt{}
\end{sphinxVerbatim}

\end{sphinxuseclass}\end{sphinxVerbatimOutput}

\end{sphinxuseclass}


\begin{sphinxuseclass}{cell}\begin{sphinxVerbatimInput}

\begin{sphinxuseclass}{cell_input}
\begin{sphinxVerbatim}[commandchars=\\\{\}]
\PYG{k+kn}{import} \PYG{n+nn}{pandas} \PYG{k}{as} \PYG{n+nn}{pd}
\PYG{n}{pd}\PYG{o}{.}\PYG{n}{options}\PYG{o}{.}\PYG{n}{display}\PYG{o}{.}\PYG{n}{float\PYGZus{}format} \PYG{o}{=} \PYG{l+s+s1}{\PYGZsq{}}\PYG{l+s+si}{\PYGZob{}:,.3f\PYGZcb{}}\PYG{l+s+s1}{\PYGZsq{}}\PYG{o}{.}\PYG{n}{format}

\PYG{n}{stock\PYGZus{}data}\PYG{o}{.}\PYG{n}{dropna}\PYG{p}{(}\PYG{n}{inplace}\PYG{o}{=}\PYG{k+kc}{True}\PYG{p}{)} \PYG{c+c1}{\PYGZsh{} NaN(값 없음) 열 전부 제거}
\PYG{n+nb}{print}\PYG{p}{(}\PYG{n}{stock\PYGZus{}data}\PYG{o}{.}\PYG{n}{groupby}\PYG{p}{(}\PYG{l+s+s1}{\PYGZsq{}}\PYG{l+s+s1}{buy}\PYG{l+s+s1}{\PYGZsq{}}\PYG{p}{)}\PYG{p}{[}\PYG{l+s+s1}{\PYGZsq{}}\PYG{l+s+s1}{return}\PYG{l+s+s1}{\PYGZsq{}}\PYG{p}{]}\PYG{o}{.}\PYG{n}{mean}\PYG{p}{(}\PYG{p}{)}\PYG{p}{)} \PYG{c+c1}{\PYGZsh{} 평균 비교}
\PYG{n+nb}{print}\PYG{p}{(}\PYG{l+s+s1}{\PYGZsq{}}\PYG{l+s+se}{\PYGZbs{}n}\PYG{l+s+s1}{\PYGZsq{}}\PYG{p}{)}
\PYG{n+nb}{print}\PYG{p}{(}\PYG{n}{stock\PYGZus{}data}\PYG{o}{.}\PYG{n}{groupby}\PYG{p}{(}\PYG{l+s+s1}{\PYGZsq{}}\PYG{l+s+s1}{buy}\PYG{l+s+s1}{\PYGZsq{}}\PYG{p}{)}\PYG{p}{[}\PYG{l+s+s1}{\PYGZsq{}}\PYG{l+s+s1}{return}\PYG{l+s+s1}{\PYGZsq{}}\PYG{p}{]}\PYG{o}{.}\PYG{n}{describe}\PYG{p}{(}\PYG{p}{)}\PYG{p}{)} \PYG{c+c1}{\PYGZsh{} 분포 비교}
\end{sphinxVerbatim}

\end{sphinxuseclass}\end{sphinxVerbatimInput}
\begin{sphinxVerbatimOutput}

\begin{sphinxuseclass}{cell_output}
\begin{sphinxVerbatim}[commandchars=\\\{\}]
buy
0   0.999
1   0.998
Name: return, dtype: float64


      count  mean   std   min   25\PYGZpc{}   50\PYGZpc{}   75\PYGZpc{}   max
buy                                                  
0   136.000 0.999 0.011 0.970 0.992 1.000 1.005 1.033
1   110.000 0.998 0.013 0.975 0.990 0.997 1.004 1.066
\end{sphinxVerbatim}

\end{sphinxuseclass}\end{sphinxVerbatimOutput}

\end{sphinxuseclass}


\begin{sphinxuseclass}{cell}\begin{sphinxVerbatimInput}

\begin{sphinxuseclass}{cell_input}
\begin{sphinxVerbatim}[commandchars=\\\{\}]
\PYG{n+nb}{print}\PYG{p}{(}\PYG{n}{stock\PYGZus{}data}\PYG{o}{.}\PYG{n}{groupby}\PYG{p}{(}\PYG{l+s+s1}{\PYGZsq{}}\PYG{l+s+s1}{buy}\PYG{l+s+s1}{\PYGZsq{}}\PYG{p}{)}\PYG{p}{[}\PYG{l+s+s1}{\PYGZsq{}}\PYG{l+s+s1}{return}\PYG{l+s+s1}{\PYGZsq{}}\PYG{p}{]}\PYG{o}{.}\PYG{n}{prod}\PYG{p}{(}\PYG{p}{)}\PYG{p}{)}
\end{sphinxVerbatim}

\end{sphinxuseclass}\end{sphinxVerbatimInput}
\begin{sphinxVerbatimOutput}

\begin{sphinxuseclass}{cell_output}
\begin{sphinxVerbatim}[commandchars=\\\{\}]
buy
0   0.843
1   0.811
Name: return, dtype: float64
\end{sphinxVerbatim}

\end{sphinxuseclass}\end{sphinxVerbatimOutput}

\end{sphinxuseclass}

\section{Rolling}
\label{\detokenize{chapter2/2.2.7_Useful_Techniques:rolling}}\label{\detokenize{chapter2/2.2.7_Useful_Techniques::doc}}
\sphinxAtStartPar
주식을 하신 분들은 이동평균선에 대하여 많이 들어보셨을 것이라고 생각합니다. rolling 은 이동평균선을 간단하게 만들어줄 수 있는 메소드입니다. 예제를 보시면 금방 이해가 되 실 것이라고 생각합니다. 일단 삼성전자 일봉을 가져오겠습니다.

\begin{sphinxuseclass}{cell}\begin{sphinxVerbatimInput}

\begin{sphinxuseclass}{cell_input}
\begin{sphinxVerbatim}[commandchars=\\\{\}]
\PYG{k+kn}{import} \PYG{n+nn}{FinanceDataReader} \PYG{k}{as} \PYG{n+nn}{fdr} 

\PYG{n}{code} \PYG{o}{=} \PYG{l+s+s1}{\PYGZsq{}}\PYG{l+s+s1}{005930}\PYG{l+s+s1}{\PYGZsq{}} \PYG{c+c1}{\PYGZsh{} 삼성전자}
\PYG{n}{stock\PYGZus{}data} \PYG{o}{=} \PYG{n}{fdr}\PYG{o}{.}\PYG{n}{DataReader}\PYG{p}{(}\PYG{n}{code}\PYG{p}{,} \PYG{n}{start}\PYG{o}{=}\PYG{l+s+s1}{\PYGZsq{}}\PYG{l+s+s1}{2021\PYGZhy{}01\PYGZhy{}03}\PYG{l+s+s1}{\PYGZsq{}}\PYG{p}{,} \PYG{n}{end}\PYG{o}{=}\PYG{l+s+s1}{\PYGZsq{}}\PYG{l+s+s1}{2021\PYGZhy{}12\PYGZhy{}31}\PYG{l+s+s1}{\PYGZsq{}}\PYG{p}{)} 

\PYG{n}{stock\PYGZus{}data}\PYG{o}{.}\PYG{n}{head}\PYG{p}{(}\PYG{p}{)}\PYG{o}{.}\PYG{n}{style}\PYG{o}{.}\PYG{n}{set\PYGZus{}table\PYGZus{}attributes}\PYG{p}{(}\PYG{l+s+s1}{\PYGZsq{}}\PYG{l+s+s1}{style=}\PYG{l+s+s1}{\PYGZdq{}}\PYG{l+s+s1}{font\PYGZhy{}size: 12px}\PYG{l+s+s1}{\PYGZdq{}}\PYG{l+s+s1}{\PYGZsq{}}\PYG{p}{)}
\end{sphinxVerbatim}

\end{sphinxuseclass}\end{sphinxVerbatimInput}
\begin{sphinxVerbatimOutput}

\begin{sphinxuseclass}{cell_output}
\begin{sphinxVerbatim}[commandchars=\\\{\}]
\PYGZlt{}pandas.io.formats.style.Styler at 0x164cee62640\PYGZgt{}
\end{sphinxVerbatim}

\end{sphinxuseclass}\end{sphinxVerbatimOutput}

\end{sphinxuseclass}


\begin{sphinxuseclass}{cell}\begin{sphinxVerbatimInput}

\begin{sphinxuseclass}{cell_input}
\begin{sphinxVerbatim}[commandchars=\\\{\}]
\PYG{n}{stock\PYGZus{}data}\PYG{p}{[}\PYG{l+s+s1}{\PYGZsq{}}\PYG{l+s+s1}{5 day moving average}\PYG{l+s+s1}{\PYGZsq{}}\PYG{p}{]} \PYG{o}{=} \PYG{n}{stock\PYGZus{}data}\PYG{p}{[}\PYG{l+s+s1}{\PYGZsq{}}\PYG{l+s+s1}{Close}\PYG{l+s+s1}{\PYGZsq{}}\PYG{p}{]}\PYG{o}{.}\PYG{n}{rolling}\PYG{p}{(}\PYG{l+m+mi}{5}\PYG{p}{)}\PYG{o}{.}\PYG{n}{mean}\PYG{p}{(}\PYG{p}{)}
\PYG{n}{stock\PYGZus{}data}\PYG{o}{.}\PYG{n}{head}\PYG{p}{(}\PYG{l+m+mi}{6}\PYG{p}{)}\PYG{o}{.}\PYG{n}{style}\PYG{o}{.}\PYG{n}{set\PYGZus{}table\PYGZus{}attributes}\PYG{p}{(}\PYG{l+s+s1}{\PYGZsq{}}\PYG{l+s+s1}{style=}\PYG{l+s+s1}{\PYGZdq{}}\PYG{l+s+s1}{font\PYGZhy{}size: 12px}\PYG{l+s+s1}{\PYGZdq{}}\PYG{l+s+s1}{\PYGZsq{}}\PYG{p}{)}
\end{sphinxVerbatim}

\end{sphinxuseclass}\end{sphinxVerbatimInput}
\begin{sphinxVerbatimOutput}

\begin{sphinxuseclass}{cell_output}
\begin{sphinxVerbatim}[commandchars=\\\{\}]
\PYGZlt{}pandas.io.formats.style.Styler at 0x164cc568880\PYGZgt{}
\end{sphinxVerbatim}

\end{sphinxuseclass}\end{sphinxVerbatimOutput}

\end{sphinxuseclass}


\begin{sphinxuseclass}{cell}\begin{sphinxVerbatimInput}

\begin{sphinxuseclass}{cell_input}
\begin{sphinxVerbatim}[commandchars=\\\{\}]
\PYG{n}{stock\PYGZus{}data}\PYG{p}{[}\PYG{l+s+s1}{\PYGZsq{}}\PYG{l+s+s1}{20 day moving average}\PYG{l+s+s1}{\PYGZsq{}}\PYG{p}{]} \PYG{o}{=} \PYG{n}{stock\PYGZus{}data}\PYG{p}{[}\PYG{l+s+s1}{\PYGZsq{}}\PYG{l+s+s1}{Close}\PYG{l+s+s1}{\PYGZsq{}}\PYG{p}{]}\PYG{o}{.}\PYG{n}{rolling}\PYG{p}{(}\PYG{l+m+mi}{20}\PYG{p}{)}\PYG{o}{.}\PYG{n}{mean}\PYG{p}{(}\PYG{p}{)}
\PYG{n}{stock\PYGZus{}data}\PYG{o}{.}\PYG{n}{head}\PYG{p}{(}\PYG{l+m+mi}{21}\PYG{p}{)}\PYG{o}{.}\PYG{n}{style}\PYG{o}{.}\PYG{n}{set\PYGZus{}table\PYGZus{}attributes}\PYG{p}{(}\PYG{l+s+s1}{\PYGZsq{}}\PYG{l+s+s1}{style=}\PYG{l+s+s1}{\PYGZdq{}}\PYG{l+s+s1}{font\PYGZhy{}size: 12px}\PYG{l+s+s1}{\PYGZdq{}}\PYG{l+s+s1}{\PYGZsq{}}\PYG{p}{)}
\end{sphinxVerbatim}

\end{sphinxuseclass}\end{sphinxVerbatimInput}
\begin{sphinxVerbatimOutput}

\begin{sphinxuseclass}{cell_output}
\begin{sphinxVerbatim}[commandchars=\\\{\}]
\PYGZlt{}pandas.io.formats.style.Styler at 0x164cfdd35e0\PYGZgt{}
\end{sphinxVerbatim}

\end{sphinxuseclass}\end{sphinxVerbatimOutput}

\end{sphinxuseclass}


\begin{sphinxuseclass}{cell}\begin{sphinxVerbatimInput}

\begin{sphinxuseclass}{cell_input}
\begin{sphinxVerbatim}[commandchars=\\\{\}]
\PYG{n}{stock\PYGZus{}data}\PYG{o}{.}\PYG{n}{dropna}\PYG{p}{(}\PYG{n}{inplace}\PYG{o}{=}\PYG{k+kc}{True}\PYG{p}{)} \PYG{c+c1}{\PYGZsh{} NaN 이 있는 모든 row 제거}
\PYG{n}{stock\PYGZus{}data}\PYG{o}{.}\PYG{n}{head}\PYG{p}{(}\PYG{p}{)}\PYG{o}{.}\PYG{n}{style}\PYG{o}{.}\PYG{n}{set\PYGZus{}table\PYGZus{}attributes}\PYG{p}{(}\PYG{l+s+s1}{\PYGZsq{}}\PYG{l+s+s1}{style=}\PYG{l+s+s1}{\PYGZdq{}}\PYG{l+s+s1}{font\PYGZhy{}size: 12px}\PYG{l+s+s1}{\PYGZdq{}}\PYG{l+s+s1}{\PYGZsq{}}\PYG{p}{)}
\end{sphinxVerbatim}

\end{sphinxuseclass}\end{sphinxVerbatimInput}
\begin{sphinxVerbatimOutput}

\begin{sphinxuseclass}{cell_output}
\begin{sphinxVerbatim}[commandchars=\\\{\}]
\PYGZlt{}pandas.io.formats.style.Styler at 0x164cf076310\PYGZgt{}
\end{sphinxVerbatim}

\end{sphinxuseclass}\end{sphinxVerbatimOutput}

\end{sphinxuseclass}


\begin{sphinxuseclass}{cell}\begin{sphinxVerbatimInput}

\begin{sphinxuseclass}{cell_input}
\begin{sphinxVerbatim}[commandchars=\\\{\}]
\PYG{n}{stock\PYGZus{}data}\PYG{p}{[}\PYG{l+s+s1}{\PYGZsq{}}\PYG{l+s+s1}{cross\PYGZus{}flag}\PYG{l+s+s1}{\PYGZsq{}}\PYG{p}{]} \PYG{o}{=} \PYG{p}{(}\PYG{n}{stock\PYGZus{}data}\PYG{p}{[}\PYG{l+s+s1}{\PYGZsq{}}\PYG{l+s+s1}{5 day moving average}\PYG{l+s+s1}{\PYGZsq{}}\PYG{p}{]} \PYG{o}{\PYGZgt{}} \PYG{n}{stock\PYGZus{}data}\PYG{p}{[}\PYG{l+s+s1}{\PYGZsq{}}\PYG{l+s+s1}{20 day moving average}\PYG{l+s+s1}{\PYGZsq{}}\PYG{p}{]}\PYG{p}{)}\PYG{o}{.}\PYG{n}{astype}\PYG{p}{(}\PYG{n+nb}{int}\PYG{p}{)} \PYG{c+c1}{\PYGZsh{} True/False 결과 값을 1/0 으로 바꿔줌}
\PYG{n}{s} \PYG{o}{=} \PYG{n}{stock\PYGZus{}data}\PYG{p}{[}\PYG{p}{(}\PYG{n}{stock\PYGZus{}data}\PYG{p}{[}\PYG{l+s+s1}{\PYGZsq{}}\PYG{l+s+s1}{cross\PYGZus{}flag}\PYG{l+s+s1}{\PYGZsq{}}\PYG{p}{]}\PYG{o}{.}\PYG{n}{shift}\PYG{p}{(}\PYG{l+m+mi}{1}\PYG{p}{)}\PYG{o}{==}\PYG{l+m+mi}{0}\PYG{p}{)} \PYG{o}{\PYGZam{}} \PYG{p}{(}\PYG{n}{stock\PYGZus{}data}\PYG{p}{[}\PYG{l+s+s1}{\PYGZsq{}}\PYG{l+s+s1}{cross\PYGZus{}flag}\PYG{l+s+s1}{\PYGZsq{}}\PYG{p}{]}\PYG{o}{==}\PYG{l+m+mi}{1}\PYG{p}{)}\PYG{p}{]} \PYG{c+c1}{\PYGZsh{} 조건 \PYGZhy{} 전날에는 5일 이평선이 20일 이평선보다 작거나 같아는데, 당일은 5일 이평선이 20일 이평선 보다 커짐}
\PYG{n}{s}\PYG{o}{.}\PYG{n}{style}\PYG{o}{.}\PYG{n}{set\PYGZus{}table\PYGZus{}attributes}\PYG{p}{(}\PYG{l+s+s1}{\PYGZsq{}}\PYG{l+s+s1}{style=}\PYG{l+s+s1}{\PYGZdq{}}\PYG{l+s+s1}{font\PYGZhy{}size: 12px}\PYG{l+s+s1}{\PYGZdq{}}\PYG{l+s+s1}{\PYGZsq{}}\PYG{p}{)}
\end{sphinxVerbatim}

\end{sphinxuseclass}\end{sphinxVerbatimInput}
\begin{sphinxVerbatimOutput}

\begin{sphinxuseclass}{cell_output}
\begin{sphinxVerbatim}[commandchars=\\\{\}]
\PYGZlt{}pandas.io.formats.style.Styler at 0x164cee62400\PYGZgt{}
\end{sphinxVerbatim}

\end{sphinxuseclass}\end{sphinxVerbatimOutput}

\end{sphinxuseclass}

\chapter{\sphinxstylestrong{시각화}}
\label{\detokenize{chapter2/2.3.0_Visualization:id1}}\label{\detokenize{chapter2/2.3.0_Visualization::doc}}
\sphinxAtStartPar
이번 장에서는 데이터를 시각화하는 방법등을 배워보겠습니다.


\section{Pandas Plot}
\label{\detokenize{chapter2/2.3.1_Visualization:pandas-plot}}\label{\detokenize{chapter2/2.3.1_Visualization::doc}}
\sphinxAtStartPar
주식에서 많이 활용할 그래프는 Line Chart 와 Bar Chart 입니다. 보통 주가의 흐름은 Line Chart 로 표시하고, 거래량은 Bar Chat 로 표시합니다. 이 두 가지를 연습해 보겠습니다. 그래프는 DataFrame 에서도 만들 수 있습니다. 복잡한 그래프를 그리려면 Matplotlib 를 이용하는데요. 이번 섹션에는 Pandas 에서 제공하는 Plot 을 이용하겠습니다. 먼저 DataFrame 에서 제공하는 plot 메소드로 간단하게 그리는 법을 연습하겠습니다. 삼성전자 일봉을 가져옵니다.

\begin{sphinxuseclass}{cell}\begin{sphinxVerbatimInput}

\begin{sphinxuseclass}{cell_input}
\begin{sphinxVerbatim}[commandchars=\\\{\}]
\PYG{k+kn}{import} \PYG{n+nn}{FinanceDataReader} \PYG{k}{as} \PYG{n+nn}{fdr} 
\PYG{k+kn}{import} \PYG{n+nn}{pandas} \PYG{k}{as} \PYG{n+nn}{pd}

\PYG{n}{code} \PYG{o}{=} \PYG{l+s+s1}{\PYGZsq{}}\PYG{l+s+s1}{005930}\PYG{l+s+s1}{\PYGZsq{}} \PYG{c+c1}{\PYGZsh{} 삼성전자}
\PYG{n}{stock\PYGZus{}data} \PYG{o}{=} \PYG{n}{fdr}\PYG{o}{.}\PYG{n}{DataReader}\PYG{p}{(}\PYG{n}{code}\PYG{p}{,} \PYG{n}{start}\PYG{o}{=}\PYG{l+s+s1}{\PYGZsq{}}\PYG{l+s+s1}{2021\PYGZhy{}01\PYGZhy{}03}\PYG{l+s+s1}{\PYGZsq{}}\PYG{p}{,} \PYG{n}{end}\PYG{o}{=}\PYG{l+s+s1}{\PYGZsq{}}\PYG{l+s+s1}{2021\PYGZhy{}12\PYGZhy{}31}\PYG{l+s+s1}{\PYGZsq{}}\PYG{p}{)} 

\PYG{n}{stock\PYGZus{}data}\PYG{o}{.}\PYG{n}{head}\PYG{p}{(}\PYG{p}{)}\PYG{o}{.}\PYG{n}{style}\PYG{o}{.}\PYG{n}{set\PYGZus{}table\PYGZus{}attributes}\PYG{p}{(}\PYG{l+s+s1}{\PYGZsq{}}\PYG{l+s+s1}{style=}\PYG{l+s+s1}{\PYGZdq{}}\PYG{l+s+s1}{font\PYGZhy{}size: 12px}\PYG{l+s+s1}{\PYGZdq{}}\PYG{l+s+s1}{\PYGZsq{}}\PYG{p}{)}
\end{sphinxVerbatim}

\end{sphinxuseclass}\end{sphinxVerbatimInput}
\begin{sphinxVerbatimOutput}

\begin{sphinxuseclass}{cell_output}
\begin{sphinxVerbatim}[commandchars=\\\{\}]
             Open   High    Low  Close    Volume    Change
Date                                                      
2021\PYGZhy{}01\PYGZhy{}04  81000  84400  80200  83000  38655276  0.024691
2021\PYGZhy{}01\PYGZhy{}05  81600  83900  81600  83900  35335669  0.010843
2021\PYGZhy{}01\PYGZhy{}06  83300  84500  82100  82200  42089013 \PYGZhy{}0.020262
2021\PYGZhy{}01\PYGZhy{}07  82800  84200  82700  82900  32644642  0.008516
2021\PYGZhy{}01\PYGZhy{}08  83300  90000  83000  88800  59013307  0.071170
\end{sphinxVerbatim}

\end{sphinxuseclass}\end{sphinxVerbatimOutput}

\end{sphinxuseclass}
\sphinxAtStartPar
먼저 종가를 Line Chart 로 그려봅니다. 2021년 주가흐름이 내리막입니다.

\begin{sphinxuseclass}{cell}\begin{sphinxVerbatimInput}

\begin{sphinxuseclass}{cell_input}
\begin{sphinxVerbatim}[commandchars=\\\{\}]
\PYG{n}{stock\PYGZus{}data}\PYG{p}{[}\PYG{l+s+s1}{\PYGZsq{}}\PYG{l+s+s1}{Close}\PYG{l+s+s1}{\PYGZsq{}}\PYG{p}{]}\PYG{o}{.}\PYG{n}{plot}\PYG{p}{(}\PYG{p}{)}
\end{sphinxVerbatim}

\end{sphinxuseclass}\end{sphinxVerbatimInput}
\begin{sphinxVerbatimOutput}

\begin{sphinxuseclass}{cell_output}
\begin{sphinxVerbatim}[commandchars=\\\{\}]
\PYGZlt{}AxesSubplot:xlabel=\PYGZsq{}Date\PYGZsq{}\PYGZgt{}
\end{sphinxVerbatim}

\noindent\sphinxincludegraphics{{2.3.1_Visualization_3_1}.png}

\end{sphinxuseclass}\end{sphinxVerbatimOutput}

\end{sphinxuseclass}


\begin{sphinxuseclass}{cell}\begin{sphinxVerbatimInput}

\begin{sphinxuseclass}{cell_input}
\begin{sphinxVerbatim}[commandchars=\\\{\}]
\PYG{n}{stock\PYGZus{}data}\PYG{p}{[}\PYG{l+s+s1}{\PYGZsq{}}\PYG{l+s+s1}{Close}\PYG{l+s+s1}{\PYGZsq{}}\PYG{p}{]}\PYG{o}{.}\PYG{n}{plot}\PYG{p}{(}\PYG{n}{figsize}\PYG{o}{=}\PYG{p}{(}\PYG{l+m+mi}{15}\PYG{p}{,}\PYG{l+m+mi}{5}\PYG{p}{)}\PYG{p}{,} \PYG{n}{title} \PYG{o}{=} \PYG{l+s+s1}{\PYGZsq{}}\PYG{l+s+s1}{Samsung Electronics}\PYG{l+s+s1}{\PYGZsq{}}\PYG{p}{)}
\end{sphinxVerbatim}

\end{sphinxuseclass}\end{sphinxVerbatimInput}
\begin{sphinxVerbatimOutput}

\begin{sphinxuseclass}{cell_output}
\begin{sphinxVerbatim}[commandchars=\\\{\}]
\PYGZlt{}AxesSubplot:title=\PYGZob{}\PYGZsq{}center\PYGZsq{}:\PYGZsq{}Samsung Electronics\PYGZsq{}\PYGZcb{}, xlabel=\PYGZsq{}Date\PYGZsq{}\PYGZgt{}
\end{sphinxVerbatim}

\noindent\sphinxincludegraphics{{2.3.1_Visualization_5_1}.png}

\end{sphinxuseclass}\end{sphinxVerbatimOutput}

\end{sphinxuseclass}


\begin{sphinxuseclass}{cell}\begin{sphinxVerbatimInput}

\begin{sphinxuseclass}{cell_input}
\begin{sphinxVerbatim}[commandchars=\\\{\}]
\PYG{n}{stock\PYGZus{}data}\PYG{p}{[}\PYG{l+s+s1}{\PYGZsq{}}\PYG{l+s+s1}{Volume}\PYG{l+s+s1}{\PYGZsq{}}\PYG{p}{]}\PYG{o}{.}\PYG{n}{plot}\PYG{p}{(}\PYG{n}{kind}\PYG{o}{=}\PYG{l+s+s1}{\PYGZsq{}}\PYG{l+s+s1}{bar}\PYG{l+s+s1}{\PYGZsq{}}\PYG{p}{,} \PYG{n}{figsize}\PYG{o}{=}\PYG{p}{(}\PYG{l+m+mi}{15}\PYG{p}{,}\PYG{l+m+mi}{5}\PYG{p}{)}\PYG{p}{,} \PYG{n}{title} \PYG{o}{=} \PYG{l+s+s1}{\PYGZsq{}}\PYG{l+s+s1}{Samsung Electronics}\PYG{l+s+s1}{\PYGZsq{}}\PYG{p}{)}
\end{sphinxVerbatim}

\end{sphinxuseclass}\end{sphinxVerbatimInput}
\begin{sphinxVerbatimOutput}

\begin{sphinxuseclass}{cell_output}
\begin{sphinxVerbatim}[commandchars=\\\{\}]
\PYGZlt{}AxesSubplot:title=\PYGZob{}\PYGZsq{}center\PYGZsq{}:\PYGZsq{}Samsung Electronics\PYGZsq{}\PYGZcb{}, xlabel=\PYGZsq{}Date\PYGZsq{}\PYGZgt{}
\end{sphinxVerbatim}

\noindent\sphinxincludegraphics{{2.3.1_Visualization_7_1}.png}

\end{sphinxuseclass}\end{sphinxVerbatimOutput}

\end{sphinxuseclass}
\sphinxAtStartPar
Bar 별로 X 값(일) 을 표시하다 보니, X 축의 날짜가 보이질 않습니다. loc{[}시작일:종료일{]} 를 이용해서 1월의 거래량만을 보겠습니다.

\begin{sphinxuseclass}{cell}\begin{sphinxVerbatimInput}

\begin{sphinxuseclass}{cell_input}
\begin{sphinxVerbatim}[commandchars=\\\{\}]
\PYG{n}{stock\PYGZus{}data}\PYG{o}{.}\PYG{n}{loc}\PYG{p}{[}\PYG{l+s+s1}{\PYGZsq{}}\PYG{l+s+s1}{2021\PYGZhy{}01\PYGZhy{}04}\PYG{l+s+s1}{\PYGZsq{}}\PYG{p}{:}\PYG{l+s+s1}{\PYGZsq{}}\PYG{l+s+s1}{2021\PYGZhy{}01\PYGZhy{}31}\PYG{l+s+s1}{\PYGZsq{}}\PYG{p}{]}\PYG{p}{[}\PYG{l+s+s1}{\PYGZsq{}}\PYG{l+s+s1}{Volume}\PYG{l+s+s1}{\PYGZsq{}}\PYG{p}{]}\PYG{o}{.}\PYG{n}{plot}\PYG{p}{(}\PYG{n}{kind}\PYG{o}{=}\PYG{l+s+s1}{\PYGZsq{}}\PYG{l+s+s1}{bar}\PYG{l+s+s1}{\PYGZsq{}}\PYG{p}{,} \PYG{n}{figsize}\PYG{o}{=}\PYG{p}{(}\PYG{l+m+mi}{15}\PYG{p}{,}\PYG{l+m+mi}{5}\PYG{p}{)}\PYG{p}{,} \PYG{n}{title} \PYG{o}{=} \PYG{l+s+s1}{\PYGZsq{}}\PYG{l+s+s1}{Samsung Electronics}\PYG{l+s+s1}{\PYGZsq{}}\PYG{p}{)}
\end{sphinxVerbatim}

\end{sphinxuseclass}\end{sphinxVerbatimInput}
\begin{sphinxVerbatimOutput}

\begin{sphinxuseclass}{cell_output}
\begin{sphinxVerbatim}[commandchars=\\\{\}]
\PYGZlt{}AxesSubplot:title=\PYGZob{}\PYGZsq{}center\PYGZsq{}:\PYGZsq{}Samsung Electronics\PYGZsq{}\PYGZcb{}, xlabel=\PYGZsq{}Date\PYGZsq{}\PYGZgt{}
\end{sphinxVerbatim}

\noindent\sphinxincludegraphics{{2.3.1_Visualization_9_1}.png}

\end{sphinxuseclass}\end{sphinxVerbatimOutput}

\end{sphinxuseclass}
\sphinxAtStartPar
역시 X 축 값이 너무 깁니다. 년\sphinxhyphen{}월\sphinxhyphen{}일만 표시하고 싶습니다. 이번에는 stock\_data 의 인덱스를 strftime 을 이용해서 년\sphinxhyphen{}월\sphinxhyphen{}일 의 문자열로 바꿔주고 다시 그래프를 그립니다.

\begin{sphinxuseclass}{cell}\begin{sphinxVerbatimInput}

\begin{sphinxuseclass}{cell_input}
\begin{sphinxVerbatim}[commandchars=\\\{\}]
\PYG{k+kn}{import} \PYG{n+nn}{datetime}
\PYG{n}{stock\PYGZus{}data2} \PYG{o}{=} \PYG{n}{stock\PYGZus{}data}\PYG{o}{.}\PYG{n}{copy}\PYG{p}{(}\PYG{p}{)} \PYG{c+c1}{\PYGZsh{} 새로운 DataFrame 생성하고, 새로운 DataFrame 의 index 타입을 변경 }
\PYG{n}{stock\PYGZus{}data2}\PYG{o}{.}\PYG{n}{index} \PYG{o}{=} \PYG{p}{[}\PYG{n}{datetime}\PYG{o}{.}\PYG{n}{datetime}\PYG{o}{.}\PYG{n}{strftime}\PYG{p}{(}\PYG{n}{dt}\PYG{p}{,} \PYG{l+s+s1}{\PYGZsq{}}\PYG{l+s+s1}{\PYGZpc{}}\PYG{l+s+s1}{Y\PYGZhy{}}\PYG{l+s+s1}{\PYGZpc{}}\PYG{l+s+s1}{m\PYGZhy{}}\PYG{l+s+si}{\PYGZpc{}d}\PYG{l+s+s1}{\PYGZsq{}}\PYG{p}{)} \PYG{k}{for} \PYG{n}{dt} \PYG{o+ow}{in} \PYG{n}{stock\PYGZus{}data}\PYG{o}{.}\PYG{n}{index}\PYG{p}{]} \PYG{c+c1}{\PYGZsh{} Date 으로 되어 있는 index 값을 원하는 모양의 문자열로 변환}
\PYG{n}{stock\PYGZus{}data2}\PYG{o}{.}\PYG{n}{loc}\PYG{p}{[}\PYG{l+s+s1}{\PYGZsq{}}\PYG{l+s+s1}{2021\PYGZhy{}01\PYGZhy{}04}\PYG{l+s+s1}{\PYGZsq{}}\PYG{p}{:}\PYG{l+s+s1}{\PYGZsq{}}\PYG{l+s+s1}{2021\PYGZhy{}01\PYGZhy{}31}\PYG{l+s+s1}{\PYGZsq{}}\PYG{p}{]}\PYG{p}{[}\PYG{l+s+s1}{\PYGZsq{}}\PYG{l+s+s1}{Volume}\PYG{l+s+s1}{\PYGZsq{}}\PYG{p}{]}\PYG{o}{.}\PYG{n}{plot}\PYG{p}{(}\PYG{n}{kind}\PYG{o}{=}\PYG{l+s+s1}{\PYGZsq{}}\PYG{l+s+s1}{bar}\PYG{l+s+s1}{\PYGZsq{}}\PYG{p}{,} \PYG{n}{figsize}\PYG{o}{=}\PYG{p}{(}\PYG{l+m+mi}{15}\PYG{p}{,}\PYG{l+m+mi}{5}\PYG{p}{)}\PYG{p}{,} \PYG{n}{title} \PYG{o}{=} \PYG{l+s+s1}{\PYGZsq{}}\PYG{l+s+s1}{Samsung Electronics}\PYG{l+s+s1}{\PYGZsq{}}\PYG{p}{)}
\end{sphinxVerbatim}

\end{sphinxuseclass}\end{sphinxVerbatimInput}
\begin{sphinxVerbatimOutput}

\begin{sphinxuseclass}{cell_output}
\begin{sphinxVerbatim}[commandchars=\\\{\}]
\PYGZlt{}AxesSubplot:title=\PYGZob{}\PYGZsq{}center\PYGZsq{}:\PYGZsq{}Samsung Electronics\PYGZsq{}\PYGZcb{}\PYGZgt{}
\end{sphinxVerbatim}

\noindent\sphinxincludegraphics{{2.3.1_Visualization_11_1}.png}

\end{sphinxuseclass}\end{sphinxVerbatimOutput}

\end{sphinxuseclass}
\sphinxAtStartPar
 이제 주가 Line Chart 와 거래량 Bar Chat 를 한 Chart 에 그리고 싶은 욕구가 생깁니다. Pandas Plot 에서 가능은 한데 복잡합니다. 이 부분은 matplotlib 에서 하겠습니다.

\begin{sphinxuseclass}{cell}\begin{sphinxVerbatimInput}

\begin{sphinxuseclass}{cell_input}
\begin{sphinxVerbatim}[commandchars=\\\{\}]
\PYG{k+kn}{import} \PYG{n+nn}{matplotlib}\PYG{n+nn}{.}\PYG{n+nn}{pyplot} \PYG{k}{as} \PYG{n+nn}{plt}
\PYG{o}{\PYGZpc{}}\PYG{k}{matplotlib} inline
\end{sphinxVerbatim}

\end{sphinxuseclass}\end{sphinxVerbatimInput}

\end{sphinxuseclass}

\section{Matplotlib}
\label{\detokenize{chapter2/2.3.2_Visualization:matplotlib}}\label{\detokenize{chapter2/2.3.2_Visualization::doc}}
\sphinxAtStartPar
전 단원에서 Pandas 에서 제공하는 Plot 으로 Chart 를 그리는 연습을 했습니다. 이번에는 시각화 패키지인 Matplotlib 를 이용해서 Chart 를 만들어 보겠습니다. 다시 삼성전자 일봉을 가져옵니다.

\begin{sphinxuseclass}{cell}\begin{sphinxVerbatimInput}

\begin{sphinxuseclass}{cell_input}
\begin{sphinxVerbatim}[commandchars=\\\{\}]
\PYG{k+kn}{import} \PYG{n+nn}{FinanceDataReader} \PYG{k}{as} \PYG{n+nn}{fdr} 
\PYG{k+kn}{import} \PYG{n+nn}{pandas} \PYG{k}{as} \PYG{n+nn}{pd}

\PYG{n}{code} \PYG{o}{=} \PYG{l+s+s1}{\PYGZsq{}}\PYG{l+s+s1}{005930}\PYG{l+s+s1}{\PYGZsq{}} \PYG{c+c1}{\PYGZsh{} 삼성전자}
\PYG{n}{stock\PYGZus{}data} \PYG{o}{=} \PYG{n}{fdr}\PYG{o}{.}\PYG{n}{DataReader}\PYG{p}{(}\PYG{n}{code}\PYG{p}{,} \PYG{n}{start}\PYG{o}{=}\PYG{l+s+s1}{\PYGZsq{}}\PYG{l+s+s1}{2021\PYGZhy{}01\PYGZhy{}03}\PYG{l+s+s1}{\PYGZsq{}}\PYG{p}{,} \PYG{n}{end}\PYG{o}{=}\PYG{l+s+s1}{\PYGZsq{}}\PYG{l+s+s1}{2021\PYGZhy{}12\PYGZhy{}31}\PYG{l+s+s1}{\PYGZsq{}}\PYG{p}{)} 

\PYG{n}{stock\PYGZus{}data}\PYG{o}{.}\PYG{n}{head}\PYG{p}{(}\PYG{p}{)}\PYG{o}{.}\PYG{n}{style}\PYG{o}{.}\PYG{n}{set\PYGZus{}table\PYGZus{}attributes}\PYG{p}{(}\PYG{l+s+s1}{\PYGZsq{}}\PYG{l+s+s1}{style=}\PYG{l+s+s1}{\PYGZdq{}}\PYG{l+s+s1}{font\PYGZhy{}size: 12px}\PYG{l+s+s1}{\PYGZdq{}}\PYG{l+s+s1}{\PYGZsq{}}\PYG{p}{)}
\end{sphinxVerbatim}

\end{sphinxuseclass}\end{sphinxVerbatimInput}
\begin{sphinxVerbatimOutput}

\begin{sphinxuseclass}{cell_output}
\begin{sphinxVerbatim}[commandchars=\\\{\}]
             Open   High    Low  Close    Volume    Change
Date                                                      
2021\PYGZhy{}01\PYGZhy{}04  81000  84400  80200  83000  38655276  0.024691
2021\PYGZhy{}01\PYGZhy{}05  81600  83900  81600  83900  35335669  0.010843
2021\PYGZhy{}01\PYGZhy{}06  83300  84500  82100  82200  42089013 \PYGZhy{}0.020262
2021\PYGZhy{}01\PYGZhy{}07  82800  84200  82700  82900  32644642  0.008516
2021\PYGZhy{}01\PYGZhy{}08  83300  90000  83000  88800  59013307  0.071170
\end{sphinxVerbatim}

\end{sphinxuseclass}\end{sphinxVerbatimOutput}

\end{sphinxuseclass}
\sphinxAtStartPar
 Matplotlib 패키지를 import 합니다. 두번째 줄에 \%matplotlib inline 같이 적어줍니다. 두번째 줄은 쥬피터노트북의 아웃풋 창에 Chart 를 볼 수 있게 해주는 기능을 합니다. 먼저 plt.figure 을 이용하여 chart 의 크기를 결정해줍니다. plt.plot() 를 해보면 박스만 있습니다. 이제 chart 를 추가하겠습니다.

\begin{sphinxuseclass}{cell}\begin{sphinxVerbatimInput}

\begin{sphinxuseclass}{cell_input}
\begin{sphinxVerbatim}[commandchars=\\\{\}]
\PYG{k+kn}{import} \PYG{n+nn}{matplotlib}\PYG{n+nn}{.}\PYG{n+nn}{pyplot} \PYG{k}{as} \PYG{n+nn}{plt}
\PYG{o}{\PYGZpc{}}\PYG{k}{matplotlib} inline

\PYG{n}{plt}\PYG{o}{.}\PYG{n}{figure}\PYG{p}{(}\PYG{n}{figsize}\PYG{o}{=}\PYG{p}{(}\PYG{l+m+mi}{15}\PYG{p}{,}\PYG{l+m+mi}{5}\PYG{p}{)}\PYG{p}{)}
\PYG{n}{plt}\PYG{o}{.}\PYG{n}{plot}\PYG{p}{(}\PYG{p}{)}
\PYG{n}{plt}\PYG{o}{.}\PYG{n}{show}\PYG{p}{(}\PYG{p}{)}
\end{sphinxVerbatim}

\end{sphinxuseclass}\end{sphinxVerbatimInput}
\begin{sphinxVerbatimOutput}

\begin{sphinxuseclass}{cell_output}
\noindent\sphinxincludegraphics{{2.3.2_Visualization_3_0}.png}

\end{sphinxuseclass}\end{sphinxVerbatimOutput}

\end{sphinxuseclass}
\sphinxAtStartPar
 삼성전자 종가 line를 추가했습니다. plt.title 를 이용해서 제목도 넣어줍니다. color=’orangered’ 인수를 넣어 line 색상도 빨간 오렌지 색으로 바꿔줍니다.

\begin{sphinxuseclass}{cell}\begin{sphinxVerbatimInput}

\begin{sphinxuseclass}{cell_input}
\begin{sphinxVerbatim}[commandchars=\\\{\}]
\PYG{n}{plt}\PYG{o}{.}\PYG{n}{figure}\PYG{p}{(}\PYG{n}{figsize}\PYG{o}{=}\PYG{p}{(}\PYG{l+m+mi}{15}\PYG{p}{,}\PYG{l+m+mi}{5}\PYG{p}{)}\PYG{p}{)}
\PYG{n}{plt}\PYG{o}{.}\PYG{n}{title}\PYG{p}{(}\PYG{l+s+s1}{\PYGZsq{}}\PYG{l+s+s1}{Samsung Electronics}\PYG{l+s+s1}{\PYGZsq{}}\PYG{p}{)}
\PYG{n}{plt}\PYG{o}{.}\PYG{n}{plot}\PYG{p}{(}\PYG{n}{stock\PYGZus{}data}\PYG{p}{[}\PYG{l+s+s1}{\PYGZsq{}}\PYG{l+s+s1}{Close}\PYG{l+s+s1}{\PYGZsq{}}\PYG{p}{]}\PYG{p}{,} \PYG{n}{color}\PYG{o}{=}\PYG{l+s+s1}{\PYGZsq{}}\PYG{l+s+s1}{orangered}\PYG{l+s+s1}{\PYGZsq{}}\PYG{p}{)}
\PYG{n}{plt}\PYG{o}{.}\PYG{n}{show}\PYG{p}{(}\PYG{p}{)}
\end{sphinxVerbatim}

\end{sphinxuseclass}\end{sphinxVerbatimInput}
\begin{sphinxVerbatimOutput}

\begin{sphinxuseclass}{cell_output}
\noindent\sphinxincludegraphics{{2.3.2_Visualization_5_0}.png}

\end{sphinxuseclass}\end{sphinxVerbatimOutput}

\end{sphinxuseclass}
\sphinxAtStartPar
 이번에는 거래량 Bar Chart 를 추가합니다. 먼저 plt.subplots 에서 fig 와 ax 객체를 받아옵니다. fig 는 그래프의 사이즈 객체이고, ax 는 축 객체입니다. 주가와 거래량은 크기가 서로 틀리므로 두 개의 Y 축이 필요합니다. 원래의 축 ax 에 ax.twinx() 를 선언해서 새로운 축 ax2 을 만들어 줍니다. Bar Chart 는 ax2 축(오른쪽)에 그립니다.

\begin{sphinxuseclass}{cell}\begin{sphinxVerbatimInput}

\begin{sphinxuseclass}{cell_input}
\begin{sphinxVerbatim}[commandchars=\\\{\}]
\PYG{n}{fig}\PYG{p}{,} \PYG{n}{ax} \PYG{o}{=} \PYG{n}{plt}\PYG{o}{.}\PYG{n}{subplots}\PYG{p}{(}\PYG{n}{figsize}\PYG{o}{=}\PYG{p}{(}\PYG{l+m+mi}{15}\PYG{p}{,}\PYG{l+m+mi}{5}\PYG{p}{)}\PYG{p}{)}
\PYG{n}{plt}\PYG{o}{.}\PYG{n}{title}\PYG{p}{(}\PYG{l+s+s1}{\PYGZsq{}}\PYG{l+s+s1}{Samsung Electronics}\PYG{l+s+s1}{\PYGZsq{}}\PYG{p}{)}
\PYG{n}{ax}\PYG{o}{.}\PYG{n}{plot}\PYG{p}{(}\PYG{n}{stock\PYGZus{}data}\PYG{p}{[}\PYG{l+s+s1}{\PYGZsq{}}\PYG{l+s+s1}{Close}\PYG{l+s+s1}{\PYGZsq{}}\PYG{p}{]}\PYG{p}{,} \PYG{n}{color}\PYG{o}{=}\PYG{l+s+s1}{\PYGZsq{}}\PYG{l+s+s1}{orangered}\PYG{l+s+s1}{\PYGZsq{}}\PYG{p}{)}
\PYG{n}{ax2} \PYG{o}{=} \PYG{n}{ax}\PYG{o}{.}\PYG{n}{twinx}\PYG{p}{(}\PYG{p}{)}
\PYG{n}{ax2}\PYG{o}{.}\PYG{n}{bar}\PYG{p}{(}\PYG{n}{height}\PYG{o}{=}\PYG{n}{stock\PYGZus{}data}\PYG{p}{[}\PYG{l+s+s1}{\PYGZsq{}}\PYG{l+s+s1}{Volume}\PYG{l+s+s1}{\PYGZsq{}}\PYG{p}{]}\PYG{p}{,} \PYG{n}{x}\PYG{o}{=}\PYG{n}{stock\PYGZus{}data}\PYG{o}{.}\PYG{n}{index}\PYG{p}{)}
\PYG{n}{plt}\PYG{o}{.}\PYG{n}{show}\PYG{p}{(}\PYG{p}{)}
\end{sphinxVerbatim}

\end{sphinxuseclass}\end{sphinxVerbatimInput}
\begin{sphinxVerbatimOutput}

\begin{sphinxuseclass}{cell_output}
\noindent\sphinxincludegraphics{{2.3.2_Visualization_7_0}.png}

\end{sphinxuseclass}\end{sphinxVerbatimOutput}

\end{sphinxuseclass}
\sphinxAtStartPar
 만들어진 그래프에 set\_ylabel 로 왼쪽, 오른쪽 Y 축에 레이블을 추가합니다. 그리고  각 축 ax, ax2 에 legend(위치) 를 표시하도록 합니다.

\begin{sphinxuseclass}{cell}\begin{sphinxVerbatimInput}

\begin{sphinxuseclass}{cell_input}
\begin{sphinxVerbatim}[commandchars=\\\{\}]
\PYG{n}{fig}\PYG{p}{,} \PYG{n}{ax} \PYG{o}{=} \PYG{n}{plt}\PYG{o}{.}\PYG{n}{subplots}\PYG{p}{(}\PYG{n}{figsize}\PYG{o}{=}\PYG{p}{(}\PYG{l+m+mi}{15}\PYG{p}{,}\PYG{l+m+mi}{5}\PYG{p}{)}\PYG{p}{)}
\PYG{n}{plt}\PYG{o}{.}\PYG{n}{title}\PYG{p}{(}\PYG{l+s+s1}{\PYGZsq{}}\PYG{l+s+s1}{Samsung Electronics}\PYG{l+s+s1}{\PYGZsq{}}\PYG{p}{)}
\PYG{n}{ax}\PYG{o}{.}\PYG{n}{plot}\PYG{p}{(}\PYG{n}{stock\PYGZus{}data}\PYG{p}{[}\PYG{l+s+s1}{\PYGZsq{}}\PYG{l+s+s1}{Close}\PYG{l+s+s1}{\PYGZsq{}}\PYG{p}{]}\PYG{p}{,} \PYG{n}{color}\PYG{o}{=}\PYG{l+s+s1}{\PYGZsq{}}\PYG{l+s+s1}{orangered}\PYG{l+s+s1}{\PYGZsq{}}\PYG{p}{,} \PYG{n}{label}\PYG{o}{=}\PYG{l+s+s1}{\PYGZsq{}}\PYG{l+s+s1}{Price}\PYG{l+s+s1}{\PYGZsq{}}\PYG{p}{)} \PYG{c+c1}{\PYGZsh{} legend(범례)에 표시될 레이블 추가}
\PYG{n}{ax2} \PYG{o}{=} \PYG{n}{ax}\PYG{o}{.}\PYG{n}{twinx}\PYG{p}{(}\PYG{p}{)} \PYG{c+c1}{\PYGZsh{} 새로운 축 만듦}
\PYG{n}{ax2}\PYG{o}{.}\PYG{n}{bar}\PYG{p}{(}\PYG{n}{height}\PYG{o}{=}\PYG{n}{stock\PYGZus{}data}\PYG{p}{[}\PYG{l+s+s1}{\PYGZsq{}}\PYG{l+s+s1}{Volume}\PYG{l+s+s1}{\PYGZsq{}}\PYG{p}{]}\PYG{p}{,} \PYG{n}{x}\PYG{o}{=}\PYG{n}{stock\PYGZus{}data}\PYG{o}{.}\PYG{n}{index}\PYG{p}{,} \PYG{n}{label}\PYG{o}{=}\PYG{l+s+s1}{\PYGZsq{}}\PYG{l+s+s1}{Volume}\PYG{l+s+s1}{\PYGZsq{}}\PYG{p}{)} \PYG{c+c1}{\PYGZsh{} legend(범례)에 표시될 레이블 추가}
\PYG{n}{ax}\PYG{o}{.}\PYG{n}{set\PYGZus{}ylabel}\PYG{p}{(}\PYG{l+s+s1}{\PYGZsq{}}\PYG{l+s+s1}{Price}\PYG{l+s+s1}{\PYGZsq{}}\PYG{p}{)}
\PYG{n}{ax2}\PYG{o}{.}\PYG{n}{set\PYGZus{}ylabel}\PYG{p}{(}\PYG{l+s+s1}{\PYGZsq{}}\PYG{l+s+s1}{Volume}\PYG{l+s+s1}{\PYGZsq{}}\PYG{p}{)}
\PYG{n}{ax}\PYG{o}{.}\PYG{n}{legend}\PYG{p}{(}\PYG{n}{loc}\PYG{o}{=}\PYG{l+m+mi}{1}\PYG{p}{)} \PYG{c+c1}{\PYGZsh{} 범례 표시 () 안은 위치}
\PYG{n}{ax2}\PYG{o}{.}\PYG{n}{legend}\PYG{p}{(}\PYG{n}{loc}\PYG{o}{=}\PYG{l+m+mi}{2}\PYG{p}{)} \PYG{c+c1}{\PYGZsh{} 범례 표시 () 안은 위치 }
\PYG{n}{plt}\PYG{o}{.}\PYG{n}{show}\PYG{p}{(}\PYG{p}{)}
\end{sphinxVerbatim}

\end{sphinxuseclass}\end{sphinxVerbatimInput}
\begin{sphinxVerbatimOutput}

\begin{sphinxuseclass}{cell_output}
\noindent\sphinxincludegraphics{{2.3.2_Visualization_9_0}.png}

\end{sphinxuseclass}\end{sphinxVerbatimOutput}

\end{sphinxuseclass}

\chapter{\sphinxstylestrong{변동성 돌파전략 구현}}
\label{\detokenize{chapter2/2.4.1_Volatility_Breakout:id1}}\label{\detokenize{chapter2/2.4.1_Volatility_Breakout::doc}}
\sphinxAtStartPar
여기까지 배운 내용을 토대로 래리윌리암스의 변동성 돌파전략을 구현해보겠습니다.


\chapter{변동성 돌파전략}
\label{\detokenize{chapter2/2.4.1_Volatility_Breakout:id2}}
\sphinxAtStartPar
변동성 돌파전략은 래리 윌리암스가 개발한 전략인데요. 이 전략으로 윌리암스는 주식투자 대회에서 많은 상을 받았다고 하네요. 심지어 딸에게 이 전략을 전수해 주었다고 합니다. 전략은 아주 간단합니다. ‘전날 고가와 저가의 차’에 상수 K (0.4 \textasciitilde{} 0.6) 를 곱하여 변동성 값 V 를 만듭니다. 그리고 당일 장이 시작하면 시가에 이 변동성 값 V 를 더한 값을 매수 가격으로 설정합니다. 장 중에 매수 가격을 돌파하면 무조건 매수합니다. 그리고 다음날 장 시작할 때 전량 매도하는 전략입니다. 다음 링크는 변동성 돌파전략에 관련하여 참고할만한 블로그 입니다. \sphinxurl{https://blog.naver.com/niolpa/222436997945} 다시 삼성전자 일봉을 가져옵니다.

\begin{sphinxuseclass}{cell}\begin{sphinxVerbatimInput}

\begin{sphinxuseclass}{cell_input}
\begin{sphinxVerbatim}[commandchars=\\\{\}]
\PYG{k+kn}{import} \PYG{n+nn}{FinanceDataReader} \PYG{k}{as} \PYG{n+nn}{fdr} 

\PYG{n}{code} \PYG{o}{=} \PYG{l+s+s1}{\PYGZsq{}}\PYG{l+s+s1}{005930}\PYG{l+s+s1}{\PYGZsq{}} \PYG{c+c1}{\PYGZsh{} 삼성전자}
\PYG{n}{stock\PYGZus{}data} \PYG{o}{=} \PYG{n}{fdr}\PYG{o}{.}\PYG{n}{DataReader}\PYG{p}{(}\PYG{n}{code}\PYG{p}{,} \PYG{n}{start}\PYG{o}{=}\PYG{l+s+s1}{\PYGZsq{}}\PYG{l+s+s1}{2021\PYGZhy{}01\PYGZhy{}03}\PYG{l+s+s1}{\PYGZsq{}}\PYG{p}{,} \PYG{n}{end}\PYG{o}{=}\PYG{l+s+s1}{\PYGZsq{}}\PYG{l+s+s1}{2021\PYGZhy{}12\PYGZhy{}31}\PYG{l+s+s1}{\PYGZsq{}}\PYG{p}{)} 
\PYG{n}{stock\PYGZus{}data}\PYG{o}{.}\PYG{n}{head}\PYG{p}{(}\PYG{p}{)}\PYG{o}{.}\PYG{n}{style}\PYG{o}{.}\PYG{n}{set\PYGZus{}table\PYGZus{}attributes}\PYG{p}{(}\PYG{l+s+s1}{\PYGZsq{}}\PYG{l+s+s1}{style=}\PYG{l+s+s1}{\PYGZdq{}}\PYG{l+s+s1}{font\PYGZhy{}size: 12px}\PYG{l+s+s1}{\PYGZdq{}}\PYG{l+s+s1}{\PYGZsq{}}\PYG{p}{)}
\end{sphinxVerbatim}

\end{sphinxuseclass}\end{sphinxVerbatimInput}
\begin{sphinxVerbatimOutput}

\begin{sphinxuseclass}{cell_output}
\begin{sphinxVerbatim}[commandchars=\\\{\}]
\PYGZlt{}pandas.io.formats.style.Styler at 0x2ad7c659f40\PYGZgt{}
\end{sphinxVerbatim}

\end{sphinxuseclass}\end{sphinxVerbatimOutput}

\end{sphinxuseclass}
\sphinxAtStartPar
 K = 0.5 라고 하고 전날의 고가와 저가를 이용하여 변동성 값 V 를 구합니다. 그리고 시가를 더하여 매수가격을 만듭니다. shift(1) 은 바로 위에 있는 row 를 참조하게 됩니다. 따라서 전날 데이터가 됩니다.

\begin{sphinxuseclass}{cell}\begin{sphinxVerbatimInput}

\begin{sphinxuseclass}{cell_input}
\begin{sphinxVerbatim}[commandchars=\\\{\}]
\PYG{n}{K} \PYG{o}{=} \PYG{l+m+mf}{0.5}
\PYG{n}{stock\PYGZus{}data}\PYG{p}{[}\PYG{l+s+s1}{\PYGZsq{}}\PYG{l+s+s1}{v}\PYG{l+s+s1}{\PYGZsq{}}\PYG{p}{]} \PYG{o}{=} \PYG{p}{(}\PYG{n}{stock\PYGZus{}data}\PYG{p}{[}\PYG{l+s+s1}{\PYGZsq{}}\PYG{l+s+s1}{High}\PYG{l+s+s1}{\PYGZsq{}}\PYG{p}{]}\PYG{o}{.}\PYG{n}{shift}\PYG{p}{(}\PYG{l+m+mi}{1}\PYG{p}{)} \PYG{o}{\PYGZhy{}} \PYG{n}{stock\PYGZus{}data}\PYG{p}{[}\PYG{l+s+s1}{\PYGZsq{}}\PYG{l+s+s1}{Low}\PYG{l+s+s1}{\PYGZsq{}}\PYG{p}{]}\PYG{o}{.}\PYG{n}{shift}\PYG{p}{(}\PYG{l+m+mi}{1}\PYG{p}{)}\PYG{p}{)}\PYG{o}{*}\PYG{n}{K}  \PYG{c+c1}{\PYGZsh{} 전날 고가에서 저가를 뺀 값에 K 를 곱함}
\PYG{n}{stock\PYGZus{}data}\PYG{p}{[}\PYG{l+s+s1}{\PYGZsq{}}\PYG{l+s+s1}{buy\PYGZus{}price}\PYG{l+s+s1}{\PYGZsq{}}\PYG{p}{]} \PYG{o}{=} \PYG{n}{stock\PYGZus{}data}\PYG{p}{[}\PYG{l+s+s1}{\PYGZsq{}}\PYG{l+s+s1}{Open}\PYG{l+s+s1}{\PYGZsq{}}\PYG{p}{]} \PYG{o}{+} \PYG{n}{stock\PYGZus{}data}\PYG{p}{[}\PYG{l+s+s1}{\PYGZsq{}}\PYG{l+s+s1}{v}\PYG{l+s+s1}{\PYGZsq{}}\PYG{p}{]}  \PYG{c+c1}{\PYGZsh{} 변동성 값 V 에 당일 시가를 더하여 매수가를 만듦}
\end{sphinxVerbatim}

\end{sphinxuseclass}\end{sphinxVerbatimInput}

\end{sphinxuseclass}
\sphinxAtStartPar
 만약 buy\_price 가 당일 고가와 저가 사이의 값이라면 매수할 기회가 있었을 것입니다. 매수 여부를 ‘buy’ 라는 컬럼에 저장합니다. 그리고 수익율 ‘return’ 을 생성합니다. 수익율은 다음날 시가를 매수가격로 나눈 값이 됩니다.

\begin{sphinxuseclass}{cell}\begin{sphinxVerbatimInput}

\begin{sphinxuseclass}{cell_input}
\begin{sphinxVerbatim}[commandchars=\\\{\}]
\PYG{n}{stock\PYGZus{}data}\PYG{p}{[}\PYG{l+s+s1}{\PYGZsq{}}\PYG{l+s+s1}{buy}\PYG{l+s+s1}{\PYGZsq{}}\PYG{p}{]} \PYG{o}{=} \PYG{p}{(}\PYG{n}{stock\PYGZus{}data}\PYG{p}{[}\PYG{l+s+s1}{\PYGZsq{}}\PYG{l+s+s1}{High}\PYG{l+s+s1}{\PYGZsq{}}\PYG{p}{]} \PYG{o}{\PYGZgt{}} \PYG{n}{stock\PYGZus{}data}\PYG{p}{[}\PYG{l+s+s1}{\PYGZsq{}}\PYG{l+s+s1}{buy\PYGZus{}price}\PYG{l+s+s1}{\PYGZsq{}}\PYG{p}{]}\PYG{p}{)}\PYG{o}{*}\PYG{p}{(}\PYG{n}{stock\PYGZus{}data}\PYG{p}{[}\PYG{l+s+s1}{\PYGZsq{}}\PYG{l+s+s1}{Low}\PYG{l+s+s1}{\PYGZsq{}}\PYG{p}{]} \PYG{o}{\PYGZlt{}} \PYG{n}{stock\PYGZus{}data}\PYG{p}{[}\PYG{l+s+s1}{\PYGZsq{}}\PYG{l+s+s1}{buy\PYGZus{}price}\PYG{l+s+s1}{\PYGZsq{}}\PYG{p}{]}\PYG{p}{)}\PYG{o}{.}\PYG{n}{astype}\PYG{p}{(}\PYG{n+nb}{int}\PYG{p}{)} \PYG{c+c1}{\PYGZsh{} 매수 기회 있으면 1 아니면 0}
\PYG{n}{stock\PYGZus{}data}\PYG{p}{[}\PYG{l+s+s1}{\PYGZsq{}}\PYG{l+s+s1}{return}\PYG{l+s+s1}{\PYGZsq{}}\PYG{p}{]} \PYG{o}{=} \PYG{n}{stock\PYGZus{}data}\PYG{p}{[}\PYG{l+s+s1}{\PYGZsq{}}\PYG{l+s+s1}{Open}\PYG{l+s+s1}{\PYGZsq{}}\PYG{p}{]}\PYG{o}{.}\PYG{n}{shift}\PYG{p}{(}\PYG{o}{\PYGZhy{}}\PYG{l+m+mi}{1}\PYG{p}{)}\PYG{o}{/}\PYG{n}{stock\PYGZus{}data}\PYG{p}{[}\PYG{l+s+s1}{\PYGZsq{}}\PYG{l+s+s1}{buy\PYGZus{}price}\PYG{l+s+s1}{\PYGZsq{}}\PYG{p}{]} \PYG{c+c1}{\PYGZsh{} 다음 날 시가를 이용하여 수익율 계산}
\end{sphinxVerbatim}

\end{sphinxuseclass}\end{sphinxVerbatimInput}

\end{sphinxuseclass}
\sphinxAtStartPar
 이제 ‘buy’ = 1 인 날의 평균 수익율을 구해봅니다. 0.2\% 기대수익율(일) 을 얻을 수 있는 전략입니다. 여기서 기대 수익율이란 매수를 한 날 중 랜덤한 어떤 날의 기대 수익율입니다.

\begin{sphinxuseclass}{cell}\begin{sphinxVerbatimInput}

\begin{sphinxuseclass}{cell_input}
\begin{sphinxVerbatim}[commandchars=\\\{\}]
\PYG{n}{stock\PYGZus{}data}\PYG{o}{.}\PYG{n}{groupby}\PYG{p}{(}\PYG{l+s+s1}{\PYGZsq{}}\PYG{l+s+s1}{buy}\PYG{l+s+s1}{\PYGZsq{}}\PYG{p}{)}\PYG{p}{[}\PYG{l+s+s1}{\PYGZsq{}}\PYG{l+s+s1}{return}\PYG{l+s+s1}{\PYGZsq{}}\PYG{p}{]}\PYG{o}{.}\PYG{n}{mean}\PYG{p}{(}\PYG{p}{)}
\end{sphinxVerbatim}

\end{sphinxuseclass}\end{sphinxVerbatimInput}
\begin{sphinxVerbatimOutput}

\begin{sphinxuseclass}{cell_output}
\begin{sphinxVerbatim}[commandchars=\\\{\}]
buy
0    0.985780
1    1.002018
Name: return, dtype: float64
\end{sphinxVerbatim}

\end{sphinxuseclass}\end{sphinxVerbatimOutput}

\end{sphinxuseclass}
\sphinxAtStartPar
 다른 종목도 테스트할 수 있게 이 전략을 함수로 만들어 봅니다. 리턴은 평균수익율(일) 과 최대 손실율(일)로 하겠습니다.

\begin{sphinxuseclass}{cell}\begin{sphinxVerbatimInput}

\begin{sphinxuseclass}{cell_input}
\begin{sphinxVerbatim}[commandchars=\\\{\}]
\PYG{c+c1}{\PYGZsh{} 위 내용을 모아서 함수로 만듦}
\PYG{k}{def} \PYG{n+nf}{avg\PYGZus{}return}\PYG{p}{(}\PYG{n}{code}\PYG{p}{,} \PYG{n}{K}\PYG{p}{)}\PYG{p}{:}
    \PYG{n}{stock} \PYG{o}{=} \PYG{n}{fdr}\PYG{o}{.}\PYG{n}{DataReader}\PYG{p}{(}\PYG{n}{code}\PYG{p}{,}  \PYG{n}{start}\PYG{o}{=}\PYG{l+s+s1}{\PYGZsq{}}\PYG{l+s+s1}{2021\PYGZhy{}01\PYGZhy{}03}\PYG{l+s+s1}{\PYGZsq{}}\PYG{p}{,} \PYG{n}{end}\PYG{o}{=}\PYG{l+s+s1}{\PYGZsq{}}\PYG{l+s+s1}{2021\PYGZhy{}12\PYGZhy{}31}\PYG{l+s+s1}{\PYGZsq{}}\PYG{p}{)}    
    \PYG{n}{stock}\PYG{p}{[}\PYG{l+s+s1}{\PYGZsq{}}\PYG{l+s+s1}{v}\PYG{l+s+s1}{\PYGZsq{}}\PYG{p}{]} \PYG{o}{=} \PYG{p}{(}\PYG{n}{stock}\PYG{p}{[}\PYG{l+s+s1}{\PYGZsq{}}\PYG{l+s+s1}{High}\PYG{l+s+s1}{\PYGZsq{}}\PYG{p}{]}\PYG{o}{.}\PYG{n}{shift}\PYG{p}{(}\PYG{l+m+mi}{1}\PYG{p}{)} \PYG{o}{\PYGZhy{}} \PYG{n}{stock}\PYG{p}{[}\PYG{l+s+s1}{\PYGZsq{}}\PYG{l+s+s1}{Low}\PYG{l+s+s1}{\PYGZsq{}}\PYG{p}{]}\PYG{o}{.}\PYG{n}{shift}\PYG{p}{(}\PYG{l+m+mi}{1}\PYG{p}{)}\PYG{p}{)}\PYG{o}{*}\PYG{n}{K}
    \PYG{n}{stock}\PYG{p}{[}\PYG{l+s+s1}{\PYGZsq{}}\PYG{l+s+s1}{buy\PYGZus{}price}\PYG{l+s+s1}{\PYGZsq{}}\PYG{p}{]} \PYG{o}{=} \PYG{n}{stock}\PYG{p}{[}\PYG{l+s+s1}{\PYGZsq{}}\PYG{l+s+s1}{Open}\PYG{l+s+s1}{\PYGZsq{}}\PYG{p}{]} \PYG{o}{+} \PYG{n}{stock}\PYG{p}{[}\PYG{l+s+s1}{\PYGZsq{}}\PYG{l+s+s1}{v}\PYG{l+s+s1}{\PYGZsq{}}\PYG{p}{]}
    \PYG{n}{stock}\PYG{p}{[}\PYG{l+s+s1}{\PYGZsq{}}\PYG{l+s+s1}{buy}\PYG{l+s+s1}{\PYGZsq{}}\PYG{p}{]} \PYG{o}{=} \PYG{p}{(}\PYG{n}{stock\PYGZus{}data}\PYG{p}{[}\PYG{l+s+s1}{\PYGZsq{}}\PYG{l+s+s1}{High}\PYG{l+s+s1}{\PYGZsq{}}\PYG{p}{]} \PYG{o}{\PYGZgt{}} \PYG{n}{stock\PYGZus{}data}\PYG{p}{[}\PYG{l+s+s1}{\PYGZsq{}}\PYG{l+s+s1}{buy\PYGZus{}price}\PYG{l+s+s1}{\PYGZsq{}}\PYG{p}{]}\PYG{p}{)}\PYG{o}{*}\PYG{p}{(}\PYG{n}{stock\PYGZus{}data}\PYG{p}{[}\PYG{l+s+s1}{\PYGZsq{}}\PYG{l+s+s1}{Low}\PYG{l+s+s1}{\PYGZsq{}}\PYG{p}{]} \PYG{o}{\PYGZlt{}} \PYG{n}{stock\PYGZus{}data}\PYG{p}{[}\PYG{l+s+s1}{\PYGZsq{}}\PYG{l+s+s1}{buy\PYGZus{}price}\PYG{l+s+s1}{\PYGZsq{}}\PYG{p}{]}\PYG{p}{)}\PYG{o}{.}\PYG{n}{astype}\PYG{p}{(}\PYG{n+nb}{int}\PYG{p}{)}
    \PYG{n}{stock}\PYG{p}{[}\PYG{l+s+s1}{\PYGZsq{}}\PYG{l+s+s1}{return}\PYG{l+s+s1}{\PYGZsq{}}\PYG{p}{]} \PYG{o}{=} \PYG{n}{stock}\PYG{p}{[}\PYG{l+s+s1}{\PYGZsq{}}\PYG{l+s+s1}{Open}\PYG{l+s+s1}{\PYGZsq{}}\PYG{p}{]}\PYG{o}{.}\PYG{n}{shift}\PYG{p}{(}\PYG{o}{\PYGZhy{}}\PYG{l+m+mi}{1}\PYG{p}{)}\PYG{o}{/}\PYG{n}{stock}\PYG{p}{[}\PYG{l+s+s1}{\PYGZsq{}}\PYG{l+s+s1}{buy\PYGZus{}price}\PYG{l+s+s1}{\PYGZsq{}}\PYG{p}{]}
    \PYG{k}{return} \PYG{n}{stock}\PYG{p}{[}\PYG{n}{stock}\PYG{p}{[}\PYG{l+s+s1}{\PYGZsq{}}\PYG{l+s+s1}{buy}\PYG{l+s+s1}{\PYGZsq{}}\PYG{p}{]}\PYG{o}{==}\PYG{l+m+mi}{1}\PYG{p}{]}\PYG{p}{[}\PYG{l+s+s1}{\PYGZsq{}}\PYG{l+s+s1}{return}\PYG{l+s+s1}{\PYGZsq{}}\PYG{p}{]}\PYG{o}{.}\PYG{n}{mean}\PYG{p}{(}\PYG{p}{)}\PYG{p}{,} \PYG{n}{stock}\PYG{p}{[}\PYG{n}{stock}\PYG{p}{[}\PYG{l+s+s1}{\PYGZsq{}}\PYG{l+s+s1}{buy}\PYG{l+s+s1}{\PYGZsq{}}\PYG{p}{]}\PYG{o}{==}\PYG{l+m+mi}{1}\PYG{p}{]}\PYG{p}{[}\PYG{l+s+s1}{\PYGZsq{}}\PYG{l+s+s1}{return}\PYG{l+s+s1}{\PYGZsq{}}\PYG{p}{]}\PYG{o}{.}\PYG{n}{min}\PYG{p}{(}\PYG{p}{)}

\PYG{n}{a}\PYG{p}{,} \PYG{n}{b} \PYG{o}{=} \PYG{n}{avg\PYGZus{}return}\PYG{p}{(}\PYG{l+s+s1}{\PYGZsq{}}\PYG{l+s+s1}{005930}\PYG{l+s+s1}{\PYGZsq{}}\PYG{p}{,} \PYG{l+m+mf}{0.5}\PYG{p}{)}
\PYG{n+nb}{print}\PYG{p}{(}\PYG{n}{a}\PYG{p}{,} \PYG{n}{b}\PYG{p}{)}

\PYG{c+c1}{\PYGZsh{} 참고로 아래와 같이 f\PYGZhy{}string 이용하여 출력을 이쁘게 할 수 있습니다.}
\PYG{n+nb}{print}\PYG{p}{(}\PYG{l+s+sa}{f}\PYG{l+s+s1}{\PYGZsq{}}\PYG{l+s+s1}{ 평균 수익율: }\PYG{l+s+si}{\PYGZob{}}\PYG{p}{(}\PYG{n}{a}\PYG{o}{\PYGZhy{}}\PYG{l+m+mi}{1}\PYG{p}{)}\PYG{l+s+si}{:}\PYG{l+s+s1}{5.2\PYGZpc{}}\PYG{l+s+si}{\PYGZcb{}}\PYG{l+s+s1}{ 최대 손실: }\PYG{l+s+si}{\PYGZob{}}\PYG{p}{(}\PYG{n}{b}\PYG{o}{\PYGZhy{}}\PYG{l+m+mi}{1}\PYG{p}{)}\PYG{l+s+si}{:}\PYG{l+s+s1}{5.2\PYGZpc{}}\PYG{l+s+si}{\PYGZcb{}}\PYG{l+s+s1}{\PYGZsq{}}\PYG{p}{)}
\end{sphinxVerbatim}

\end{sphinxuseclass}\end{sphinxVerbatimInput}
\begin{sphinxVerbatimOutput}

\begin{sphinxuseclass}{cell_output}
\begin{sphinxVerbatim}[commandchars=\\\{\}]
1.002018419449279 0.9574350469872858
 평균 수익율: 0.20\PYGZpc{} 최대 손실: \PYGZhy{}4.26\PYGZpc{}
\end{sphinxVerbatim}

\end{sphinxuseclass}\end{sphinxVerbatimOutput}

\end{sphinxuseclass}
\sphinxAtStartPar
 다른 종목의 결과값도 함 보겠습니다. 네이버(035420)와 현대차(005380)를 함 볼까요? 삼성전자보다 더 안 좋은 결과가 나왔습니다.

\begin{sphinxuseclass}{cell}\begin{sphinxVerbatimInput}

\begin{sphinxuseclass}{cell_input}
\begin{sphinxVerbatim}[commandchars=\\\{\}]
\PYG{n}{a}\PYG{p}{,} \PYG{n}{b} \PYG{o}{=} \PYG{n}{avg\PYGZus{}return}\PYG{p}{(}\PYG{l+s+s1}{\PYGZsq{}}\PYG{l+s+s1}{035420}\PYG{l+s+s1}{\PYGZsq{}}\PYG{p}{,} \PYG{l+m+mf}{0.5}\PYG{p}{)}
\PYG{n+nb}{print}\PYG{p}{(}\PYG{n}{a}\PYG{p}{,} \PYG{n}{b}\PYG{p}{)}
\PYG{n+nb}{print}\PYG{p}{(}\PYG{l+s+sa}{f}\PYG{l+s+s1}{\PYGZsq{}}\PYG{l+s+s1}{네이버 평균 수익율: }\PYG{l+s+si}{\PYGZob{}}\PYG{p}{(}\PYG{n}{a}\PYG{o}{\PYGZhy{}}\PYG{l+m+mi}{1}\PYG{p}{)}\PYG{l+s+si}{:}\PYG{l+s+s1}{5.2\PYGZpc{}}\PYG{l+s+si}{\PYGZcb{}}\PYG{l+s+s1}{ 최대 손실: }\PYG{l+s+si}{\PYGZob{}}\PYG{p}{(}\PYG{n}{b}\PYG{o}{\PYGZhy{}}\PYG{l+m+mi}{1}\PYG{p}{)}\PYG{l+s+si}{:}\PYG{l+s+s1}{5.2\PYGZpc{}}\PYG{l+s+si}{\PYGZcb{}}\PYG{l+s+s1}{\PYGZsq{}}\PYG{p}{)}
\PYG{n+nb}{print}\PYG{p}{(}\PYG{l+s+s1}{\PYGZsq{}}\PYG{l+s+se}{\PYGZbs{}n}\PYG{l+s+s1}{\PYGZsq{}}\PYG{p}{)}

\PYG{n}{a}\PYG{p}{,} \PYG{n}{b} \PYG{o}{=} \PYG{n}{avg\PYGZus{}return}\PYG{p}{(}\PYG{l+s+s1}{\PYGZsq{}}\PYG{l+s+s1}{005380}\PYG{l+s+s1}{\PYGZsq{}}\PYG{p}{,} \PYG{l+m+mf}{0.5}\PYG{p}{)}
\PYG{n+nb}{print}\PYG{p}{(}\PYG{n}{a}\PYG{p}{,} \PYG{n}{b}\PYG{p}{)}
\PYG{n+nb}{print}\PYG{p}{(}\PYG{l+s+sa}{f}\PYG{l+s+s1}{\PYGZsq{}}\PYG{l+s+s1}{현대차 평균 수익율: }\PYG{l+s+si}{\PYGZob{}}\PYG{p}{(}\PYG{n}{a}\PYG{o}{\PYGZhy{}}\PYG{l+m+mi}{1}\PYG{p}{)}\PYG{l+s+si}{:}\PYG{l+s+s1}{5.2\PYGZpc{}}\PYG{l+s+si}{\PYGZcb{}}\PYG{l+s+s1}{ 최대 손실: }\PYG{l+s+si}{\PYGZob{}}\PYG{p}{(}\PYG{n}{b}\PYG{o}{\PYGZhy{}}\PYG{l+m+mi}{1}\PYG{p}{)}\PYG{l+s+si}{:}\PYG{l+s+s1}{5.2\PYGZpc{}}\PYG{l+s+si}{\PYGZcb{}}\PYG{l+s+s1}{\PYGZsq{}}\PYG{p}{)}
\end{sphinxVerbatim}

\end{sphinxuseclass}\end{sphinxVerbatimInput}
\begin{sphinxVerbatimOutput}

\begin{sphinxuseclass}{cell_output}
\begin{sphinxVerbatim}[commandchars=\\\{\}]
0.9902550213520191 0.9214501510574018
네이버 평균 수익율: \PYGZhy{}0.97\PYGZpc{} 최대 손실: \PYGZhy{}7.85\PYGZpc{}


0.9940220644638738 0.9295499021526419
현대차 평균 수익율: \PYGZhy{}0.60\PYGZpc{} 최대 손실: \PYGZhy{}7.05\PYGZpc{}
\end{sphinxVerbatim}

\end{sphinxuseclass}\end{sphinxVerbatimOutput}

\end{sphinxuseclass}
\sphinxAtStartPar
 이번에는 누적 수익율도 궁금합니다. 즉, 2021년 1월 3일 100 원을 투자하면 2021년 12월 31일 얼마가 되어 있을까요? 함수의 리턴 값에 누적 수익율을 추가합니다.

\begin{sphinxuseclass}{cell}\begin{sphinxVerbatimInput}

\begin{sphinxuseclass}{cell_input}
\begin{sphinxVerbatim}[commandchars=\\\{\}]
\PYG{k}{def} \PYG{n+nf}{avg\PYGZus{}return}\PYG{p}{(}\PYG{n}{code}\PYG{p}{,} \PYG{n}{K}\PYG{p}{)}\PYG{p}{:}
    \PYG{n}{stock} \PYG{o}{=} \PYG{n}{fdr}\PYG{o}{.}\PYG{n}{DataReader}\PYG{p}{(}\PYG{n}{code}\PYG{p}{,}  \PYG{n}{start}\PYG{o}{=}\PYG{l+s+s1}{\PYGZsq{}}\PYG{l+s+s1}{2021\PYGZhy{}01\PYGZhy{}03}\PYG{l+s+s1}{\PYGZsq{}}\PYG{p}{,} \PYG{n}{end}\PYG{o}{=}\PYG{l+s+s1}{\PYGZsq{}}\PYG{l+s+s1}{2021\PYGZhy{}12\PYGZhy{}31}\PYG{l+s+s1}{\PYGZsq{}}\PYG{p}{)}    
    \PYG{n}{stock}\PYG{p}{[}\PYG{l+s+s1}{\PYGZsq{}}\PYG{l+s+s1}{v}\PYG{l+s+s1}{\PYGZsq{}}\PYG{p}{]} \PYG{o}{=} \PYG{p}{(}\PYG{n}{stock}\PYG{p}{[}\PYG{l+s+s1}{\PYGZsq{}}\PYG{l+s+s1}{High}\PYG{l+s+s1}{\PYGZsq{}}\PYG{p}{]}\PYG{o}{.}\PYG{n}{shift}\PYG{p}{(}\PYG{l+m+mi}{1}\PYG{p}{)} \PYG{o}{\PYGZhy{}} \PYG{n}{stock}\PYG{p}{[}\PYG{l+s+s1}{\PYGZsq{}}\PYG{l+s+s1}{Low}\PYG{l+s+s1}{\PYGZsq{}}\PYG{p}{]}\PYG{o}{.}\PYG{n}{shift}\PYG{p}{(}\PYG{l+m+mi}{1}\PYG{p}{)}\PYG{p}{)}\PYG{o}{*}\PYG{n}{K}
    \PYG{n}{stock}\PYG{p}{[}\PYG{l+s+s1}{\PYGZsq{}}\PYG{l+s+s1}{buy\PYGZus{}price}\PYG{l+s+s1}{\PYGZsq{}}\PYG{p}{]} \PYG{o}{=} \PYG{n}{stock}\PYG{p}{[}\PYG{l+s+s1}{\PYGZsq{}}\PYG{l+s+s1}{Open}\PYG{l+s+s1}{\PYGZsq{}}\PYG{p}{]} \PYG{o}{+} \PYG{n}{stock}\PYG{p}{[}\PYG{l+s+s1}{\PYGZsq{}}\PYG{l+s+s1}{v}\PYG{l+s+s1}{\PYGZsq{}}\PYG{p}{]}
    \PYG{n}{stock}\PYG{p}{[}\PYG{l+s+s1}{\PYGZsq{}}\PYG{l+s+s1}{buy}\PYG{l+s+s1}{\PYGZsq{}}\PYG{p}{]} \PYG{o}{=} \PYG{p}{(}\PYG{n}{stock\PYGZus{}data}\PYG{p}{[}\PYG{l+s+s1}{\PYGZsq{}}\PYG{l+s+s1}{High}\PYG{l+s+s1}{\PYGZsq{}}\PYG{p}{]} \PYG{o}{\PYGZgt{}} \PYG{n}{stock\PYGZus{}data}\PYG{p}{[}\PYG{l+s+s1}{\PYGZsq{}}\PYG{l+s+s1}{buy\PYGZus{}price}\PYG{l+s+s1}{\PYGZsq{}}\PYG{p}{]}\PYG{p}{)}\PYG{o}{*}\PYG{p}{(}\PYG{n}{stock\PYGZus{}data}\PYG{p}{[}\PYG{l+s+s1}{\PYGZsq{}}\PYG{l+s+s1}{Low}\PYG{l+s+s1}{\PYGZsq{}}\PYG{p}{]} \PYG{o}{\PYGZlt{}} \PYG{n}{stock\PYGZus{}data}\PYG{p}{[}\PYG{l+s+s1}{\PYGZsq{}}\PYG{l+s+s1}{buy\PYGZus{}price}\PYG{l+s+s1}{\PYGZsq{}}\PYG{p}{]}\PYG{p}{)}\PYG{o}{.}\PYG{n}{astype}\PYG{p}{(}\PYG{n+nb}{int}\PYG{p}{)}
    \PYG{n}{stock}\PYG{p}{[}\PYG{l+s+s1}{\PYGZsq{}}\PYG{l+s+s1}{return}\PYG{l+s+s1}{\PYGZsq{}}\PYG{p}{]} \PYG{o}{=} \PYG{n}{stock}\PYG{p}{[}\PYG{l+s+s1}{\PYGZsq{}}\PYG{l+s+s1}{Open}\PYG{l+s+s1}{\PYGZsq{}}\PYG{p}{]}\PYG{o}{.}\PYG{n}{shift}\PYG{p}{(}\PYG{o}{\PYGZhy{}}\PYG{l+m+mi}{1}\PYG{p}{)}\PYG{o}{/}\PYG{n}{stock}\PYG{p}{[}\PYG{l+s+s1}{\PYGZsq{}}\PYG{l+s+s1}{buy\PYGZus{}price}\PYG{l+s+s1}{\PYGZsq{}}\PYG{p}{]}
    \PYG{k}{return} \PYG{n}{stock}\PYG{p}{[}\PYG{n}{stock}\PYG{p}{[}\PYG{l+s+s1}{\PYGZsq{}}\PYG{l+s+s1}{buy}\PYG{l+s+s1}{\PYGZsq{}}\PYG{p}{]}\PYG{o}{==}\PYG{l+m+mi}{1}\PYG{p}{]}\PYG{p}{[}\PYG{l+s+s1}{\PYGZsq{}}\PYG{l+s+s1}{return}\PYG{l+s+s1}{\PYGZsq{}}\PYG{p}{]}\PYG{o}{.}\PYG{n}{mean}\PYG{p}{(}\PYG{p}{)}\PYG{p}{,} \PYG{n}{stock}\PYG{p}{[}\PYG{n}{stock}\PYG{p}{[}\PYG{l+s+s1}{\PYGZsq{}}\PYG{l+s+s1}{buy}\PYG{l+s+s1}{\PYGZsq{}}\PYG{p}{]}\PYG{o}{==}\PYG{l+m+mi}{1}\PYG{p}{]}\PYG{p}{[}\PYG{l+s+s1}{\PYGZsq{}}\PYG{l+s+s1}{return}\PYG{l+s+s1}{\PYGZsq{}}\PYG{p}{]}\PYG{o}{.}\PYG{n}{min}\PYG{p}{(}\PYG{p}{)}\PYG{p}{,} \PYG{n}{stock}\PYG{p}{[}\PYG{n}{stock}\PYG{p}{[}\PYG{l+s+s1}{\PYGZsq{}}\PYG{l+s+s1}{buy}\PYG{l+s+s1}{\PYGZsq{}}\PYG{p}{]}\PYG{o}{==}\PYG{l+m+mi}{1}\PYG{p}{]}\PYG{p}{[}\PYG{l+s+s1}{\PYGZsq{}}\PYG{l+s+s1}{return}\PYG{l+s+s1}{\PYGZsq{}}\PYG{p}{]}\PYG{o}{.}\PYG{n}{prod}\PYG{p}{(}\PYG{p}{)}

\PYG{n}{a}\PYG{p}{,} \PYG{n}{b}\PYG{p}{,} \PYG{n}{c} \PYG{o}{=} \PYG{n}{avg\PYGZus{}return}\PYG{p}{(}\PYG{l+s+s1}{\PYGZsq{}}\PYG{l+s+s1}{005930}\PYG{l+s+s1}{\PYGZsq{}}\PYG{p}{,} \PYG{l+m+mf}{0.5}\PYG{p}{)}
\PYG{n+nb}{print}\PYG{p}{(}\PYG{n}{a}\PYG{p}{,} \PYG{n}{b}\PYG{p}{,} \PYG{n}{c}\PYG{p}{)}

\PYG{c+c1}{\PYGZsh{} 참고로 아래와 같이 f\PYGZhy{}string 이용하여 출력을 이쁘게 할 수 있습니다.}
\PYG{n+nb}{print}\PYG{p}{(}\PYG{l+s+sa}{f}\PYG{l+s+s1}{\PYGZsq{}}\PYG{l+s+s1}{ 평균 수익율: }\PYG{l+s+si}{\PYGZob{}}\PYG{n}{a}\PYG{l+s+si}{:}\PYG{l+s+s1}{5.2\PYGZpc{}}\PYG{l+s+si}{\PYGZcb{}}\PYG{l+s+s1}{ 최대 손실: }\PYG{l+s+si}{\PYGZob{}}\PYG{n}{b}\PYG{l+s+si}{:}\PYG{l+s+s1}{5.2\PYGZpc{}}\PYG{l+s+si}{\PYGZcb{}}\PYG{l+s+s1}{ 누적수익율: }\PYG{l+s+si}{\PYGZob{}}\PYG{n}{c}\PYG{l+s+si}{:}\PYG{l+s+s1}{5.2\PYGZpc{}}\PYG{l+s+si}{\PYGZcb{}}\PYG{l+s+s1}{\PYGZsq{}}\PYG{p}{)}
\end{sphinxVerbatim}

\end{sphinxuseclass}\end{sphinxVerbatimInput}
\begin{sphinxVerbatimOutput}

\begin{sphinxuseclass}{cell_output}
\begin{sphinxVerbatim}[commandchars=\\\{\}]
1.002018419449279 0.9574350469872858 1.1843972916348044
 평균 수익율: 100.20\PYGZpc{} 최대 손실: 95.74\PYGZpc{} 누적수익율: 118.44\PYGZpc{}
\end{sphinxVerbatim}

\end{sphinxuseclass}\end{sphinxVerbatimOutput}

\end{sphinxuseclass}
\sphinxAtStartPar
 네이버의 누적 수익율은 58.2\%, 현대차의 누적 수익율은 38.7\% 입니다. 즉 2021년 초에 각 각 100 원을 투자했다면 연말에 네이버는 40원, 현대차는 56원이 되어 있습니다. 실제 장에서는 원하는 가격에 매수 매도를 할 수 없으므로 실전 수익율은 아니겠지만 예상 수익율을 추정해 볼 수 있습니다.

\begin{sphinxuseclass}{cell}\begin{sphinxVerbatimInput}

\begin{sphinxuseclass}{cell_input}
\begin{sphinxVerbatim}[commandchars=\\\{\}]
\PYG{n}{a}\PYG{p}{,} \PYG{n}{b}\PYG{p}{,} \PYG{n}{c} \PYG{o}{=} \PYG{n}{avg\PYGZus{}return}\PYG{p}{(}\PYG{l+s+s1}{\PYGZsq{}}\PYG{l+s+s1}{035420}\PYG{l+s+s1}{\PYGZsq{}}\PYG{p}{,} \PYG{l+m+mf}{0.5}\PYG{p}{)}
\PYG{n+nb}{print}\PYG{p}{(}\PYG{l+s+sa}{f}\PYG{l+s+s1}{\PYGZsq{}}\PYG{l+s+s1}{네이버 평균 수익율: }\PYG{l+s+si}{\PYGZob{}}\PYG{n}{a}\PYG{l+s+si}{:}\PYG{l+s+s1}{5.2\PYGZpc{}}\PYG{l+s+si}{\PYGZcb{}}\PYG{l+s+s1}{ 최대 손실: }\PYG{l+s+si}{\PYGZob{}}\PYG{n}{b}\PYG{l+s+si}{:}\PYG{l+s+s1}{5.2\PYGZpc{}}\PYG{l+s+si}{\PYGZcb{}}\PYG{l+s+s1}{ 누적수익율: }\PYG{l+s+si}{\PYGZob{}}\PYG{n}{c}\PYG{l+s+si}{:}\PYG{l+s+s1}{5.2\PYGZpc{}}\PYG{l+s+si}{\PYGZcb{}}\PYG{l+s+s1}{\PYGZsq{}}\PYG{p}{)}

\PYG{n}{a}\PYG{p}{,} \PYG{n}{b}\PYG{p}{,} \PYG{n}{c} \PYG{o}{=} \PYG{n}{avg\PYGZus{}return}\PYG{p}{(}\PYG{l+s+s1}{\PYGZsq{}}\PYG{l+s+s1}{005380}\PYG{l+s+s1}{\PYGZsq{}}\PYG{p}{,} \PYG{l+m+mf}{0.5}\PYG{p}{)}
\PYG{n+nb}{print}\PYG{p}{(}\PYG{l+s+sa}{f}\PYG{l+s+s1}{\PYGZsq{}}\PYG{l+s+s1}{현대차 평균 수익율: }\PYG{l+s+si}{\PYGZob{}}\PYG{n}{a}\PYG{l+s+si}{:}\PYG{l+s+s1}{5.2\PYGZpc{}}\PYG{l+s+si}{\PYGZcb{}}\PYG{l+s+s1}{ 최대 손실: }\PYG{l+s+si}{\PYGZob{}}\PYG{n}{b}\PYG{l+s+si}{:}\PYG{l+s+s1}{5.2\PYGZpc{}}\PYG{l+s+si}{\PYGZcb{}}\PYG{l+s+s1}{ 누적수익율: }\PYG{l+s+si}{\PYGZob{}}\PYG{n}{c}\PYG{l+s+si}{:}\PYG{l+s+s1}{5.2\PYGZpc{}}\PYG{l+s+si}{\PYGZcb{}}\PYG{l+s+s1}{\PYGZsq{}}\PYG{p}{)}
\end{sphinxVerbatim}

\end{sphinxuseclass}\end{sphinxVerbatimInput}
\begin{sphinxVerbatimOutput}

\begin{sphinxuseclass}{cell_output}
\begin{sphinxVerbatim}[commandchars=\\\{\}]
네이버 평균 수익율: 99.03\PYGZpc{} 최대 손실: 92.15\PYGZpc{} 누적수익율: 40.87\PYGZpc{}
현대차 평균 수익율: 99.40\PYGZpc{} 최대 손실: 92.95\PYGZpc{} 누적수익율: 56.69\PYGZpc{}
\end{sphinxVerbatim}

\end{sphinxuseclass}\end{sphinxVerbatimOutput}

\end{sphinxuseclass}

\chapter{K 값 찾기}
\label{\detokenize{chapter2/2.4.1_Volatility_Breakout:k}}
\sphinxAtStartPar
이전 단원에서 래리 윌리암스의 변동성 돌파전략을 파이썬으로 구현해봤습니다. K 값을 왜 0.5 로 했을까? 다른 K 는 어떨까 궁금해 집니다. 래리윌리암스는 K 값을 0.4 \textasciitilde{} 0.6 으로 추천했습니다. K 가 높아지면 매수 가격이 올라가므로 매수가격에 살 수 있는 기회가 적어지는 문제가 있습니다. K 가 낮아지면 쉽게 매수를 하므로 과연 변동성 돌파를 하고 있는지 의심이 듭니다. 이 번에는 삼성전자 K 값이 얼마일 때, 가장 좋은 결과가 나오는 지 알아보겠습니다. 삼성전자 2021년 일봉과 전 단원에서 만들어 놓은 함수를 가져옵니다.

\begin{sphinxuseclass}{cell}\begin{sphinxVerbatimInput}

\begin{sphinxuseclass}{cell_input}
\begin{sphinxVerbatim}[commandchars=\\\{\}]
\PYG{k+kn}{import} \PYG{n+nn}{FinanceDataReader} \PYG{k}{as} \PYG{n+nn}{fdr} 

\PYG{n}{code} \PYG{o}{=} \PYG{l+s+s1}{\PYGZsq{}}\PYG{l+s+s1}{005930}\PYG{l+s+s1}{\PYGZsq{}} \PYG{c+c1}{\PYGZsh{} 삼성전자}
\PYG{n}{stock\PYGZus{}data} \PYG{o}{=} \PYG{n}{fdr}\PYG{o}{.}\PYG{n}{DataReader}\PYG{p}{(}\PYG{n}{code}\PYG{p}{,} \PYG{n}{start}\PYG{o}{=}\PYG{l+s+s1}{\PYGZsq{}}\PYG{l+s+s1}{2021\PYGZhy{}01\PYGZhy{}03}\PYG{l+s+s1}{\PYGZsq{}}\PYG{p}{,} \PYG{n}{end}\PYG{o}{=}\PYG{l+s+s1}{\PYGZsq{}}\PYG{l+s+s1}{2021\PYGZhy{}12\PYGZhy{}31}\PYG{l+s+s1}{\PYGZsq{}}\PYG{p}{)} 
\PYG{n}{stock\PYGZus{}data}\PYG{o}{.}\PYG{n}{head}\PYG{p}{(}\PYG{p}{)}\PYG{o}{.}\PYG{n}{style}\PYG{o}{.}\PYG{n}{set\PYGZus{}table\PYGZus{}attributes}\PYG{p}{(}\PYG{l+s+s1}{\PYGZsq{}}\PYG{l+s+s1}{style=}\PYG{l+s+s1}{\PYGZdq{}}\PYG{l+s+s1}{font\PYGZhy{}size: 12px}\PYG{l+s+s1}{\PYGZdq{}}\PYG{l+s+s1}{\PYGZsq{}}\PYG{p}{)}
\end{sphinxVerbatim}

\end{sphinxuseclass}\end{sphinxVerbatimInput}
\begin{sphinxVerbatimOutput}

\begin{sphinxuseclass}{cell_output}
\begin{sphinxVerbatim}[commandchars=\\\{\}]
\PYGZlt{}pandas.io.formats.style.Styler at 0x2ad7eefe700\PYGZgt{}
\end{sphinxVerbatim}

\end{sphinxuseclass}\end{sphinxVerbatimOutput}

\end{sphinxuseclass}
\begin{sphinxuseclass}{cell}\begin{sphinxVerbatimInput}

\begin{sphinxuseclass}{cell_input}
\begin{sphinxVerbatim}[commandchars=\\\{\}]
\PYG{k}{def} \PYG{n+nf}{avg\PYGZus{}return}\PYG{p}{(}\PYG{n}{code}\PYG{p}{,} \PYG{n}{K}\PYG{p}{)}\PYG{p}{:}
    \PYG{n}{stock} \PYG{o}{=} \PYG{n}{fdr}\PYG{o}{.}\PYG{n}{DataReader}\PYG{p}{(}\PYG{n}{code}\PYG{p}{,}  \PYG{n}{start}\PYG{o}{=}\PYG{l+s+s1}{\PYGZsq{}}\PYG{l+s+s1}{2021\PYGZhy{}01\PYGZhy{}03}\PYG{l+s+s1}{\PYGZsq{}}\PYG{p}{,} \PYG{n}{end}\PYG{o}{=}\PYG{l+s+s1}{\PYGZsq{}}\PYG{l+s+s1}{2021\PYGZhy{}12\PYGZhy{}31}\PYG{l+s+s1}{\PYGZsq{}}\PYG{p}{)}    
    \PYG{n}{stock}\PYG{p}{[}\PYG{l+s+s1}{\PYGZsq{}}\PYG{l+s+s1}{v}\PYG{l+s+s1}{\PYGZsq{}}\PYG{p}{]} \PYG{o}{=} \PYG{p}{(}\PYG{n}{stock}\PYG{p}{[}\PYG{l+s+s1}{\PYGZsq{}}\PYG{l+s+s1}{High}\PYG{l+s+s1}{\PYGZsq{}}\PYG{p}{]}\PYG{o}{.}\PYG{n}{shift}\PYG{p}{(}\PYG{l+m+mi}{1}\PYG{p}{)} \PYG{o}{\PYGZhy{}} \PYG{n}{stock}\PYG{p}{[}\PYG{l+s+s1}{\PYGZsq{}}\PYG{l+s+s1}{Low}\PYG{l+s+s1}{\PYGZsq{}}\PYG{p}{]}\PYG{o}{.}\PYG{n}{shift}\PYG{p}{(}\PYG{l+m+mi}{1}\PYG{p}{)}\PYG{p}{)}\PYG{o}{*}\PYG{n}{K}
    \PYG{n}{stock}\PYG{p}{[}\PYG{l+s+s1}{\PYGZsq{}}\PYG{l+s+s1}{buy\PYGZus{}price}\PYG{l+s+s1}{\PYGZsq{}}\PYG{p}{]} \PYG{o}{=} \PYG{n}{stock}\PYG{p}{[}\PYG{l+s+s1}{\PYGZsq{}}\PYG{l+s+s1}{Open}\PYG{l+s+s1}{\PYGZsq{}}\PYG{p}{]} \PYG{o}{+} \PYG{n}{stock}\PYG{p}{[}\PYG{l+s+s1}{\PYGZsq{}}\PYG{l+s+s1}{v}\PYG{l+s+s1}{\PYGZsq{}}\PYG{p}{]}
    \PYG{n}{stock}\PYG{p}{[}\PYG{l+s+s1}{\PYGZsq{}}\PYG{l+s+s1}{buy}\PYG{l+s+s1}{\PYGZsq{}}\PYG{p}{]} \PYG{o}{=} \PYG{p}{(}\PYG{n}{stock}\PYG{p}{[}\PYG{l+s+s1}{\PYGZsq{}}\PYG{l+s+s1}{High}\PYG{l+s+s1}{\PYGZsq{}}\PYG{p}{]} \PYG{o}{\PYGZgt{}} \PYG{n}{stock}\PYG{p}{[}\PYG{l+s+s1}{\PYGZsq{}}\PYG{l+s+s1}{buy\PYGZus{}price}\PYG{l+s+s1}{\PYGZsq{}}\PYG{p}{]}\PYG{p}{)}\PYG{o}{*}\PYG{p}{(}\PYG{n}{stock}\PYG{p}{[}\PYG{l+s+s1}{\PYGZsq{}}\PYG{l+s+s1}{Low}\PYG{l+s+s1}{\PYGZsq{}}\PYG{p}{]} \PYG{o}{\PYGZlt{}} \PYG{n}{stock}\PYG{p}{[}\PYG{l+s+s1}{\PYGZsq{}}\PYG{l+s+s1}{buy\PYGZus{}price}\PYG{l+s+s1}{\PYGZsq{}}\PYG{p}{]}\PYG{p}{)}\PYG{o}{.}\PYG{n}{astype}\PYG{p}{(}\PYG{n+nb}{int}\PYG{p}{)}
    \PYG{n}{stock}\PYG{p}{[}\PYG{l+s+s1}{\PYGZsq{}}\PYG{l+s+s1}{return}\PYG{l+s+s1}{\PYGZsq{}}\PYG{p}{]} \PYG{o}{=} \PYG{n}{stock}\PYG{p}{[}\PYG{l+s+s1}{\PYGZsq{}}\PYG{l+s+s1}{Open}\PYG{l+s+s1}{\PYGZsq{}}\PYG{p}{]}\PYG{o}{.}\PYG{n}{shift}\PYG{p}{(}\PYG{o}{\PYGZhy{}}\PYG{l+m+mi}{1}\PYG{p}{)}\PYG{o}{/}\PYG{n}{stock}\PYG{p}{[}\PYG{l+s+s1}{\PYGZsq{}}\PYG{l+s+s1}{buy\PYGZus{}price}\PYG{l+s+s1}{\PYGZsq{}}\PYG{p}{]}
    \PYG{k}{return} \PYG{n}{stock}\PYG{p}{[}\PYG{n}{stock}\PYG{p}{[}\PYG{l+s+s1}{\PYGZsq{}}\PYG{l+s+s1}{buy}\PYG{l+s+s1}{\PYGZsq{}}\PYG{p}{]}\PYG{o}{==}\PYG{l+m+mi}{1}\PYG{p}{]}\PYG{p}{[}\PYG{l+s+s1}{\PYGZsq{}}\PYG{l+s+s1}{return}\PYG{l+s+s1}{\PYGZsq{}}\PYG{p}{]}\PYG{o}{.}\PYG{n}{mean}\PYG{p}{(}\PYG{p}{)}\PYG{p}{,} \PYG{n}{stock}\PYG{p}{[}\PYG{n}{stock}\PYG{p}{[}\PYG{l+s+s1}{\PYGZsq{}}\PYG{l+s+s1}{buy}\PYG{l+s+s1}{\PYGZsq{}}\PYG{p}{]}\PYG{o}{==}\PYG{l+m+mi}{1}\PYG{p}{]}\PYG{p}{[}\PYG{l+s+s1}{\PYGZsq{}}\PYG{l+s+s1}{return}\PYG{l+s+s1}{\PYGZsq{}}\PYG{p}{]}\PYG{o}{.}\PYG{n}{min}\PYG{p}{(}\PYG{p}{)}

\PYG{n}{a}\PYG{p}{,} \PYG{n}{b} \PYG{o}{=} \PYG{n}{avg\PYGZus{}return}\PYG{p}{(}\PYG{l+s+s1}{\PYGZsq{}}\PYG{l+s+s1}{005930}\PYG{l+s+s1}{\PYGZsq{}}\PYG{p}{,} \PYG{l+m+mf}{0.5}\PYG{p}{)}

\PYG{n+nb}{print}\PYG{p}{(}\PYG{l+s+sa}{f}\PYG{l+s+s1}{\PYGZsq{}}\PYG{l+s+s1}{ 평균 수익율: }\PYG{l+s+si}{\PYGZob{}}\PYG{n}{a}\PYG{l+s+si}{:}\PYG{l+s+s1}{5.2\PYGZpc{}}\PYG{l+s+si}{\PYGZcb{}}\PYG{l+s+s1}{ 최대 손실: }\PYG{l+s+si}{\PYGZob{}}\PYG{n}{b}\PYG{l+s+si}{:}\PYG{l+s+s1}{5.2\PYGZpc{}}\PYG{l+s+si}{\PYGZcb{}}\PYG{l+s+s1}{\PYGZsq{}}\PYG{p}{)}
\end{sphinxVerbatim}

\end{sphinxuseclass}\end{sphinxVerbatimInput}
\begin{sphinxVerbatimOutput}

\begin{sphinxuseclass}{cell_output}
\begin{sphinxVerbatim}[commandchars=\\\{\}]
 평균 수익율: 100.20\PYGZpc{} 최대 손실: 95.74\PYGZpc{}
\end{sphinxVerbatim}

\end{sphinxuseclass}\end{sphinxVerbatimOutput}

\end{sphinxuseclass}
\sphinxAtStartPar
 이제 K 를 조금씩 올려가면서 평균 수익율이 최대가 되는 지점을 알아보겠습니다. K 늘 조금씩 증가시켜가면서 For Loop 를 이용하면 좋을 것 같습니다. 그리고 각 K 에서 평균수익율과 최대손실을 list 에 담습니다.
테스트 할 K 는 numpy 에서 제공하는 linspace(시작 값, 종료 값) 를 이용합니다. linspace 는 num(인수 중 하나) 을 지정하지 않으면 50개의 등 간격 구간으로 된 list 를 반환합니다.

\begin{sphinxuseclass}{cell}\begin{sphinxVerbatimInput}

\begin{sphinxuseclass}{cell_input}
\begin{sphinxVerbatim}[commandchars=\\\{\}]
\PYG{c+c1}{\PYGZsh{} Line space Test}
\PYG{k+kn}{import} \PYG{n+nn}{numpy} \PYG{k}{as} \PYG{n+nn}{np}
\PYG{n+nb}{print}\PYG{p}{(}\PYG{n}{np}\PYG{o}{.}\PYG{n}{linspace}\PYG{p}{(}\PYG{n}{start}\PYG{o}{=}\PYG{l+m+mi}{0}\PYG{p}{,} \PYG{n}{stop}\PYG{o}{=}\PYG{l+m+mi}{1}\PYG{p}{,} \PYG{n}{num}\PYG{o}{=}\PYG{l+m+mi}{10}\PYG{p}{)}\PYG{p}{)}

\PYG{n+nb}{print}\PYG{p}{(}\PYG{l+s+s1}{\PYGZsq{}}\PYG{l+s+se}{\PYGZbs{}n}\PYG{l+s+s1}{\PYGZsq{}}\PYG{p}{)}
\PYG{k+kn}{import} \PYG{n+nn}{matplotlib}\PYG{n+nn}{.}\PYG{n+nn}{pyplot} \PYG{k}{as} \PYG{n+nn}{plt}
\PYG{o}{\PYGZpc{}}\PYG{k}{matplotlib} inline
\PYG{k+kn}{import} \PYG{n+nn}{numpy} \PYG{k}{as} \PYG{n+nn}{np}
\PYG{k+kn}{import} \PYG{n+nn}{pandas} \PYG{k}{as} \PYG{n+nn}{pd}

\PYG{n}{k\PYGZus{}list} \PYG{o}{=} \PYG{p}{[}\PYG{p}{]}
\PYG{n}{r\PYGZus{}list} \PYG{o}{=} \PYG{p}{[}\PYG{p}{]}
\PYG{n}{w\PYGZus{}list} \PYG{o}{=} \PYG{p}{[}\PYG{p}{]}

\PYG{k}{for} \PYG{n}{k} \PYG{o+ow}{in} \PYG{n+nb}{list}\PYG{p}{(}\PYG{n}{np}\PYG{o}{.}\PYG{n}{linspace}\PYG{p}{(}\PYG{l+m+mf}{0.2}\PYG{p}{,} \PYG{l+m+mf}{0.8}\PYG{p}{)}\PYG{p}{)}\PYG{p}{:} \PYG{c+c1}{\PYGZsh{} 0.2 \PYGZti{} 0.8 까지 50 구간 list }
    \PYG{n}{r}\PYG{p}{,} \PYG{n}{w} \PYG{o}{=} \PYG{n}{avg\PYGZus{}return}\PYG{p}{(}\PYG{l+s+s1}{\PYGZsq{}}\PYG{l+s+s1}{005930}\PYG{l+s+s1}{\PYGZsq{}}\PYG{p}{,} \PYG{n}{k}\PYG{p}{)}

    \PYG{n}{k\PYGZus{}list}\PYG{o}{.}\PYG{n}{append}\PYG{p}{(}\PYG{n}{k}\PYG{p}{)}       
    \PYG{n}{r\PYGZus{}list}\PYG{o}{.}\PYG{n}{append}\PYG{p}{(}\PYG{n}{r}\PYG{p}{)}
    \PYG{n}{w\PYGZus{}list}\PYG{o}{.}\PYG{n}{append}\PYG{p}{(}\PYG{n}{w}\PYG{p}{)}

    
\PYG{n}{outcome} \PYG{o}{=} \PYG{n}{pd}\PYG{o}{.}\PYG{n}{DataFrame}\PYG{p}{(}\PYG{p}{\PYGZob{}}\PYG{l+s+s1}{\PYGZsq{}}\PYG{l+s+s1}{k}\PYG{l+s+s1}{\PYGZsq{}}\PYG{p}{:} \PYG{n}{k\PYGZus{}list}\PYG{p}{,} \PYG{l+s+s1}{\PYGZsq{}}\PYG{l+s+s1}{return}\PYG{l+s+s1}{\PYGZsq{}}\PYG{p}{:} \PYG{n}{r\PYGZus{}list}\PYG{p}{,} \PYG{l+s+s1}{\PYGZsq{}}\PYG{l+s+s1}{worst}\PYG{l+s+s1}{\PYGZsq{}}\PYG{p}{:} \PYG{n}{w\PYGZus{}list}\PYG{p}{\PYGZcb{}}\PYG{p}{)}    
\end{sphinxVerbatim}

\end{sphinxuseclass}\end{sphinxVerbatimInput}
\begin{sphinxVerbatimOutput}

\begin{sphinxuseclass}{cell_output}
\begin{sphinxVerbatim}[commandchars=\\\{\}]
[0.         0.11111111 0.22222222 0.33333333 0.44444444 0.55555556
 0.66666667 0.77777778 0.88888889 1.        ]
\end{sphinxVerbatim}

\end{sphinxuseclass}\end{sphinxVerbatimOutput}

\end{sphinxuseclass}
\sphinxAtStartPar
위 50 개 결과를 Pandas Plot 그려보겠습니다. 파란색 라인이 평균 기대 수익율이고 빨간색 라인이 최대 손실입니다. 둘 다 값이 높아야 좋은 전략일 것 입니다. 그래프를 보시면 K = 0.4 보다는 k = 0.6 이 더 좋은 선택으로 판단됩니다. 기대 수익율이 제일 높은 K 는 0.67 입니다.

\begin{sphinxuseclass}{cell}\begin{sphinxVerbatimInput}

\begin{sphinxuseclass}{cell_input}
\begin{sphinxVerbatim}[commandchars=\\\{\}]
\PYG{n}{plt}\PYG{o}{.}\PYG{n}{figure}\PYG{p}{(}\PYG{n}{figsize}\PYG{o}{=}\PYG{p}{(}\PYG{l+m+mi}{10}\PYG{p}{,}\PYG{l+m+mi}{4}\PYG{p}{)}\PYG{p}{)}
\PYG{n}{ax} \PYG{o}{=} \PYG{n}{outcome}\PYG{o}{.}\PYG{n}{set\PYGZus{}index}\PYG{p}{(}\PYG{l+s+s1}{\PYGZsq{}}\PYG{l+s+s1}{k}\PYG{l+s+s1}{\PYGZsq{}}\PYG{p}{)}\PYG{p}{[}\PYG{l+s+s1}{\PYGZsq{}}\PYG{l+s+s1}{return}\PYG{l+s+s1}{\PYGZsq{}}\PYG{p}{]}\PYG{o}{.}\PYG{n}{plot}\PYG{p}{(}\PYG{n}{color}\PYG{o}{=}\PYG{l+s+s1}{\PYGZsq{}}\PYG{l+s+s1}{blue}\PYG{l+s+s1}{\PYGZsq{}}\PYG{p}{)}
\PYG{n}{ax2} \PYG{o}{=} \PYG{n}{ax}\PYG{o}{.}\PYG{n}{twinx}\PYG{p}{(}\PYG{p}{)}
\PYG{n}{ax2} \PYG{o}{=} \PYG{n}{outcome}\PYG{o}{.}\PYG{n}{set\PYGZus{}index}\PYG{p}{(}\PYG{l+s+s1}{\PYGZsq{}}\PYG{l+s+s1}{k}\PYG{l+s+s1}{\PYGZsq{}}\PYG{p}{)}\PYG{p}{[}\PYG{l+s+s1}{\PYGZsq{}}\PYG{l+s+s1}{worst}\PYG{l+s+s1}{\PYGZsq{}}\PYG{p}{]}\PYG{o}{.}\PYG{n}{plot}\PYG{p}{(}\PYG{n}{color}\PYG{o}{=}\PYG{l+s+s1}{\PYGZsq{}}\PYG{l+s+s1}{red}\PYG{l+s+s1}{\PYGZsq{}}\PYG{p}{,} \PYG{n}{style}\PYG{o}{=}\PYG{l+s+s1}{\PYGZsq{}}\PYG{l+s+s1}{\PYGZhy{}\PYGZhy{}}\PYG{l+s+s1}{\PYGZsq{}}\PYG{p}{)}
\PYG{n}{plt}\PYG{o}{.}\PYG{n}{title}\PYG{p}{(}\PYG{l+s+s1}{\PYGZsq{}}\PYG{l+s+s1}{Avergage Return vs. The Worst Return}\PYG{l+s+s1}{\PYGZsq{}}\PYG{p}{)}
\PYG{n}{ax}\PYG{o}{.}\PYG{n}{legend}\PYG{p}{(}\PYG{n}{loc}\PYG{o}{=}\PYG{l+m+mi}{1}\PYG{p}{)}
\PYG{n}{ax2}\PYG{o}{.}\PYG{n}{legend}\PYG{p}{(}\PYG{n}{loc}\PYG{o}{=}\PYG{l+m+mi}{2}\PYG{p}{)}
\PYG{n}{ax}\PYG{o}{.}\PYG{n}{set\PYGZus{}ylabel}\PYG{p}{(}\PYG{l+s+s1}{\PYGZsq{}}\PYG{l+s+s1}{Avg. Return}\PYG{l+s+s1}{\PYGZsq{}}\PYG{p}{)}
\PYG{n}{ax2}\PYG{o}{.}\PYG{n}{set\PYGZus{}ylabel}\PYG{p}{(}\PYG{l+s+s1}{\PYGZsq{}}\PYG{l+s+s1}{The Worst Return}\PYG{l+s+s1}{\PYGZsq{}}\PYG{p}{)}
\PYG{n}{plt}\PYG{o}{.}\PYG{n}{show}\PYG{p}{(}\PYG{p}{)}
\end{sphinxVerbatim}

\end{sphinxuseclass}\end{sphinxVerbatimInput}
\begin{sphinxVerbatimOutput}

\begin{sphinxuseclass}{cell_output}
\noindent\sphinxincludegraphics{{2.4.1_Volatility_Breakout_23_0}.png}

\end{sphinxuseclass}\end{sphinxVerbatimOutput}

\end{sphinxuseclass}
\sphinxAtStartPar
 이번에는 위 라인을 rolling 을 이용하여 부드럽게 해 보겠습니다.

\begin{sphinxuseclass}{cell}\begin{sphinxVerbatimInput}

\begin{sphinxuseclass}{cell_input}
\begin{sphinxVerbatim}[commandchars=\\\{\}]
\PYG{n}{plt}\PYG{o}{.}\PYG{n}{figure}\PYG{p}{(}\PYG{n}{figsize}\PYG{o}{=}\PYG{p}{(}\PYG{l+m+mi}{10}\PYG{p}{,}\PYG{l+m+mi}{4}\PYG{p}{)}\PYG{p}{)}
\PYG{n}{ax} \PYG{o}{=} \PYG{n}{outcome}\PYG{o}{.}\PYG{n}{set\PYGZus{}index}\PYG{p}{(}\PYG{l+s+s1}{\PYGZsq{}}\PYG{l+s+s1}{k}\PYG{l+s+s1}{\PYGZsq{}}\PYG{p}{)}\PYG{p}{[}\PYG{l+s+s1}{\PYGZsq{}}\PYG{l+s+s1}{return}\PYG{l+s+s1}{\PYGZsq{}}\PYG{p}{]}\PYG{o}{.}\PYG{n}{rolling}\PYG{p}{(}\PYG{l+m+mi}{3}\PYG{p}{)}\PYG{o}{.}\PYG{n}{mean}\PYG{p}{(}\PYG{p}{)}\PYG{o}{.}\PYG{n}{plot}\PYG{p}{(}\PYG{n}{color}\PYG{o}{=}\PYG{l+s+s1}{\PYGZsq{}}\PYG{l+s+s1}{blue}\PYG{l+s+s1}{\PYGZsq{}}\PYG{p}{)}
\PYG{n}{ax2} \PYG{o}{=} \PYG{n}{ax}\PYG{o}{.}\PYG{n}{twinx}\PYG{p}{(}\PYG{p}{)}
\PYG{n}{ax2} \PYG{o}{=} \PYG{n}{outcome}\PYG{o}{.}\PYG{n}{set\PYGZus{}index}\PYG{p}{(}\PYG{l+s+s1}{\PYGZsq{}}\PYG{l+s+s1}{k}\PYG{l+s+s1}{\PYGZsq{}}\PYG{p}{)}\PYG{p}{[}\PYG{l+s+s1}{\PYGZsq{}}\PYG{l+s+s1}{worst}\PYG{l+s+s1}{\PYGZsq{}}\PYG{p}{]}\PYG{o}{.}\PYG{n}{rolling}\PYG{p}{(}\PYG{l+m+mi}{3}\PYG{p}{)}\PYG{o}{.}\PYG{n}{mean}\PYG{p}{(}\PYG{p}{)}\PYG{o}{.}\PYG{n}{plot}\PYG{p}{(}\PYG{n}{color}\PYG{o}{=}\PYG{l+s+s1}{\PYGZsq{}}\PYG{l+s+s1}{red}\PYG{l+s+s1}{\PYGZsq{}}\PYG{p}{,} \PYG{n}{style}\PYG{o}{=}\PYG{l+s+s1}{\PYGZsq{}}\PYG{l+s+s1}{\PYGZhy{}\PYGZhy{}}\PYG{l+s+s1}{\PYGZsq{}}\PYG{p}{)}
\PYG{n}{plt}\PYG{o}{.}\PYG{n}{title}\PYG{p}{(}\PYG{l+s+s1}{\PYGZsq{}}\PYG{l+s+s1}{Avergage Return vs. The Worst Return}\PYG{l+s+s1}{\PYGZsq{}}\PYG{p}{)}
\PYG{n}{ax}\PYG{o}{.}\PYG{n}{legend}\PYG{p}{(}\PYG{n}{loc}\PYG{o}{=}\PYG{l+m+mi}{1}\PYG{p}{)}
\PYG{n}{ax2}\PYG{o}{.}\PYG{n}{legend}\PYG{p}{(}\PYG{n}{loc}\PYG{o}{=}\PYG{l+m+mi}{2}\PYG{p}{)}
\PYG{n}{ax}\PYG{o}{.}\PYG{n}{set\PYGZus{}ylabel}\PYG{p}{(}\PYG{l+s+s1}{\PYGZsq{}}\PYG{l+s+s1}{Avg. Return}\PYG{l+s+s1}{\PYGZsq{}}\PYG{p}{)}
\PYG{n}{ax2}\PYG{o}{.}\PYG{n}{set\PYGZus{}ylabel}\PYG{p}{(}\PYG{l+s+s1}{\PYGZsq{}}\PYG{l+s+s1}{The Worst Return}\PYG{l+s+s1}{\PYGZsq{}}\PYG{p}{)}
\PYG{n}{plt}\PYG{o}{.}\PYG{n}{show}\PYG{p}{(}\PYG{p}{)}
\end{sphinxVerbatim}

\end{sphinxuseclass}\end{sphinxVerbatimInput}
\begin{sphinxVerbatimOutput}

\begin{sphinxuseclass}{cell_output}
\noindent\sphinxincludegraphics{{2.4.1_Volatility_Breakout_25_0}.png}

\end{sphinxuseclass}\end{sphinxVerbatimOutput}

\end{sphinxuseclass}
\sphinxAtStartPar
2021년 데이터에서는 K 가 0.4 근처보다는 0.6 근처가 더 좋은 전략으로 관찰되었습니다. 과연 2022년도 그럴지 궁금합니다. 2022년 1분기 데이터를 이용해 보겠습니다. 다른 결과가 나왔습니다. K= 0.5 가 더 좋은 것 같습니다. 과거에 좋은 K 가 현재에도 좋은 K 가 아닌 것 같습니다. 단순이 과거 K 를 이용하는 것이 좋은 방법이 아니라는 것을 알았습니다.

\begin{sphinxuseclass}{cell}\begin{sphinxVerbatimInput}

\begin{sphinxuseclass}{cell_input}
\begin{sphinxVerbatim}[commandchars=\\\{\}]
\PYG{k}{def} \PYG{n+nf}{avg\PYGZus{}return}\PYG{p}{(}\PYG{n}{code}\PYG{p}{,} \PYG{n}{K}\PYG{p}{)}\PYG{p}{:}
    \PYG{n}{stock} \PYG{o}{=} \PYG{n}{fdr}\PYG{o}{.}\PYG{n}{DataReader}\PYG{p}{(}\PYG{n}{code}\PYG{p}{,}  \PYG{n}{start}\PYG{o}{=}\PYG{l+s+s1}{\PYGZsq{}}\PYG{l+s+s1}{2022\PYGZhy{}01\PYGZhy{}03}\PYG{l+s+s1}{\PYGZsq{}}\PYG{p}{,} \PYG{n}{end}\PYG{o}{=}\PYG{l+s+s1}{\PYGZsq{}}\PYG{l+s+s1}{2022\PYGZhy{}03\PYGZhy{}31}\PYG{l+s+s1}{\PYGZsq{}}\PYG{p}{)}    \PYG{c+c1}{\PYGZsh{} 2022년 1분기 데이터}
    \PYG{n}{stock}\PYG{p}{[}\PYG{l+s+s1}{\PYGZsq{}}\PYG{l+s+s1}{v}\PYG{l+s+s1}{\PYGZsq{}}\PYG{p}{]} \PYG{o}{=} \PYG{p}{(}\PYG{n}{stock}\PYG{p}{[}\PYG{l+s+s1}{\PYGZsq{}}\PYG{l+s+s1}{High}\PYG{l+s+s1}{\PYGZsq{}}\PYG{p}{]}\PYG{o}{.}\PYG{n}{shift}\PYG{p}{(}\PYG{l+m+mi}{1}\PYG{p}{)} \PYG{o}{\PYGZhy{}} \PYG{n}{stock}\PYG{p}{[}\PYG{l+s+s1}{\PYGZsq{}}\PYG{l+s+s1}{Low}\PYG{l+s+s1}{\PYGZsq{}}\PYG{p}{]}\PYG{o}{.}\PYG{n}{shift}\PYG{p}{(}\PYG{l+m+mi}{1}\PYG{p}{)}\PYG{p}{)}\PYG{o}{*}\PYG{n}{K}
    \PYG{n}{stock}\PYG{p}{[}\PYG{l+s+s1}{\PYGZsq{}}\PYG{l+s+s1}{buy\PYGZus{}price}\PYG{l+s+s1}{\PYGZsq{}}\PYG{p}{]} \PYG{o}{=} \PYG{n}{stock}\PYG{p}{[}\PYG{l+s+s1}{\PYGZsq{}}\PYG{l+s+s1}{Open}\PYG{l+s+s1}{\PYGZsq{}}\PYG{p}{]} \PYG{o}{+} \PYG{n}{stock}\PYG{p}{[}\PYG{l+s+s1}{\PYGZsq{}}\PYG{l+s+s1}{v}\PYG{l+s+s1}{\PYGZsq{}}\PYG{p}{]}
    \PYG{n}{stock}\PYG{p}{[}\PYG{l+s+s1}{\PYGZsq{}}\PYG{l+s+s1}{buy}\PYG{l+s+s1}{\PYGZsq{}}\PYG{p}{]} \PYG{o}{=} \PYG{p}{(}\PYG{n}{stock}\PYG{p}{[}\PYG{l+s+s1}{\PYGZsq{}}\PYG{l+s+s1}{High}\PYG{l+s+s1}{\PYGZsq{}}\PYG{p}{]} \PYG{o}{\PYGZgt{}} \PYG{n}{stock}\PYG{p}{[}\PYG{l+s+s1}{\PYGZsq{}}\PYG{l+s+s1}{buy\PYGZus{}price}\PYG{l+s+s1}{\PYGZsq{}}\PYG{p}{]}\PYG{p}{)}\PYG{o}{*}\PYG{p}{(}\PYG{n}{stock}\PYG{p}{[}\PYG{l+s+s1}{\PYGZsq{}}\PYG{l+s+s1}{Low}\PYG{l+s+s1}{\PYGZsq{}}\PYG{p}{]} \PYG{o}{\PYGZlt{}} \PYG{n}{stock}\PYG{p}{[}\PYG{l+s+s1}{\PYGZsq{}}\PYG{l+s+s1}{buy\PYGZus{}price}\PYG{l+s+s1}{\PYGZsq{}}\PYG{p}{]}\PYG{p}{)}\PYG{o}{.}\PYG{n}{astype}\PYG{p}{(}\PYG{n+nb}{int}\PYG{p}{)}
    \PYG{n}{stock}\PYG{p}{[}\PYG{l+s+s1}{\PYGZsq{}}\PYG{l+s+s1}{return}\PYG{l+s+s1}{\PYGZsq{}}\PYG{p}{]} \PYG{o}{=} \PYG{n}{stock}\PYG{p}{[}\PYG{l+s+s1}{\PYGZsq{}}\PYG{l+s+s1}{Open}\PYG{l+s+s1}{\PYGZsq{}}\PYG{p}{]}\PYG{o}{.}\PYG{n}{shift}\PYG{p}{(}\PYG{o}{\PYGZhy{}}\PYG{l+m+mi}{1}\PYG{p}{)}\PYG{o}{/}\PYG{n}{stock}\PYG{p}{[}\PYG{l+s+s1}{\PYGZsq{}}\PYG{l+s+s1}{buy\PYGZus{}price}\PYG{l+s+s1}{\PYGZsq{}}\PYG{p}{]}
    \PYG{k}{return} \PYG{n}{stock}\PYG{p}{[}\PYG{n}{stock}\PYG{p}{[}\PYG{l+s+s1}{\PYGZsq{}}\PYG{l+s+s1}{buy}\PYG{l+s+s1}{\PYGZsq{}}\PYG{p}{]}\PYG{o}{==}\PYG{l+m+mi}{1}\PYG{p}{]}\PYG{p}{[}\PYG{l+s+s1}{\PYGZsq{}}\PYG{l+s+s1}{return}\PYG{l+s+s1}{\PYGZsq{}}\PYG{p}{]}\PYG{o}{.}\PYG{n}{mean}\PYG{p}{(}\PYG{p}{)}\PYG{p}{,} \PYG{n}{stock}\PYG{p}{[}\PYG{n}{stock}\PYG{p}{[}\PYG{l+s+s1}{\PYGZsq{}}\PYG{l+s+s1}{buy}\PYG{l+s+s1}{\PYGZsq{}}\PYG{p}{]}\PYG{o}{==}\PYG{l+m+mi}{1}\PYG{p}{]}\PYG{p}{[}\PYG{l+s+s1}{\PYGZsq{}}\PYG{l+s+s1}{return}\PYG{l+s+s1}{\PYGZsq{}}\PYG{p}{]}\PYG{o}{.}\PYG{n}{min}\PYG{p}{(}\PYG{p}{)}

\PYG{n}{k\PYGZus{}list} \PYG{o}{=} \PYG{p}{[}\PYG{p}{]}
\PYG{n}{r\PYGZus{}list} \PYG{o}{=} \PYG{p}{[}\PYG{p}{]}
\PYG{n}{w\PYGZus{}list} \PYG{o}{=} \PYG{p}{[}\PYG{p}{]}

\PYG{k}{for} \PYG{n}{k} \PYG{o+ow}{in} \PYG{n+nb}{list}\PYG{p}{(}\PYG{n}{np}\PYG{o}{.}\PYG{n}{linspace}\PYG{p}{(}\PYG{l+m+mf}{0.2}\PYG{p}{,} \PYG{l+m+mf}{0.8}\PYG{p}{)}\PYG{p}{)}\PYG{p}{:} \PYG{c+c1}{\PYGZsh{} 0.2 \PYGZti{} 0.8 까지 50 구간 list }
    \PYG{n}{r}\PYG{p}{,} \PYG{n}{w} \PYG{o}{=} \PYG{n}{avg\PYGZus{}return}\PYG{p}{(}\PYG{l+s+s1}{\PYGZsq{}}\PYG{l+s+s1}{005930}\PYG{l+s+s1}{\PYGZsq{}}\PYG{p}{,} \PYG{n}{k}\PYG{p}{)}

    \PYG{n}{k\PYGZus{}list}\PYG{o}{.}\PYG{n}{append}\PYG{p}{(}\PYG{n}{k}\PYG{p}{)}       
    \PYG{n}{r\PYGZus{}list}\PYG{o}{.}\PYG{n}{append}\PYG{p}{(}\PYG{n}{r}\PYG{p}{)}
    \PYG{n}{w\PYGZus{}list}\PYG{o}{.}\PYG{n}{append}\PYG{p}{(}\PYG{n}{w}\PYG{p}{)}
    
\PYG{n}{outcome} \PYG{o}{=} \PYG{n}{pd}\PYG{o}{.}\PYG{n}{DataFrame}\PYG{p}{(}\PYG{p}{\PYGZob{}}\PYG{l+s+s1}{\PYGZsq{}}\PYG{l+s+s1}{k}\PYG{l+s+s1}{\PYGZsq{}}\PYG{p}{:} \PYG{n}{k\PYGZus{}list}\PYG{p}{,} \PYG{l+s+s1}{\PYGZsq{}}\PYG{l+s+s1}{return}\PYG{l+s+s1}{\PYGZsq{}}\PYG{p}{:} \PYG{n}{r\PYGZus{}list}\PYG{p}{,} \PYG{l+s+s1}{\PYGZsq{}}\PYG{l+s+s1}{worst}\PYG{l+s+s1}{\PYGZsq{}}\PYG{p}{:} \PYG{n}{w\PYGZus{}list}\PYG{p}{\PYGZcb{}}\PYG{p}{)}   

\PYG{n}{plt}\PYG{o}{.}\PYG{n}{figure}\PYG{p}{(}\PYG{n}{figsize}\PYG{o}{=}\PYG{p}{(}\PYG{l+m+mi}{10}\PYG{p}{,}\PYG{l+m+mi}{4}\PYG{p}{)}\PYG{p}{)}
\PYG{n}{ax} \PYG{o}{=} \PYG{n}{outcome}\PYG{o}{.}\PYG{n}{set\PYGZus{}index}\PYG{p}{(}\PYG{l+s+s1}{\PYGZsq{}}\PYG{l+s+s1}{k}\PYG{l+s+s1}{\PYGZsq{}}\PYG{p}{)}\PYG{p}{[}\PYG{l+s+s1}{\PYGZsq{}}\PYG{l+s+s1}{return}\PYG{l+s+s1}{\PYGZsq{}}\PYG{p}{]}\PYG{o}{.}\PYG{n}{rolling}\PYG{p}{(}\PYG{l+m+mi}{3}\PYG{p}{)}\PYG{o}{.}\PYG{n}{mean}\PYG{p}{(}\PYG{p}{)}\PYG{o}{.}\PYG{n}{plot}\PYG{p}{(}\PYG{n}{color}\PYG{o}{=}\PYG{l+s+s1}{\PYGZsq{}}\PYG{l+s+s1}{blue}\PYG{l+s+s1}{\PYGZsq{}}\PYG{p}{)}
\PYG{n}{ax2} \PYG{o}{=} \PYG{n}{ax}\PYG{o}{.}\PYG{n}{twinx}\PYG{p}{(}\PYG{p}{)}
\PYG{n}{ax2} \PYG{o}{=} \PYG{n}{outcome}\PYG{o}{.}\PYG{n}{set\PYGZus{}index}\PYG{p}{(}\PYG{l+s+s1}{\PYGZsq{}}\PYG{l+s+s1}{k}\PYG{l+s+s1}{\PYGZsq{}}\PYG{p}{)}\PYG{p}{[}\PYG{l+s+s1}{\PYGZsq{}}\PYG{l+s+s1}{worst}\PYG{l+s+s1}{\PYGZsq{}}\PYG{p}{]}\PYG{o}{.}\PYG{n}{rolling}\PYG{p}{(}\PYG{l+m+mi}{3}\PYG{p}{)}\PYG{o}{.}\PYG{n}{mean}\PYG{p}{(}\PYG{p}{)}\PYG{o}{.}\PYG{n}{plot}\PYG{p}{(}\PYG{n}{color}\PYG{o}{=}\PYG{l+s+s1}{\PYGZsq{}}\PYG{l+s+s1}{red}\PYG{l+s+s1}{\PYGZsq{}}\PYG{p}{,} \PYG{n}{style}\PYG{o}{=}\PYG{l+s+s1}{\PYGZsq{}}\PYG{l+s+s1}{\PYGZhy{}\PYGZhy{}}\PYG{l+s+s1}{\PYGZsq{}}\PYG{p}{)}
\PYG{n}{plt}\PYG{o}{.}\PYG{n}{title}\PYG{p}{(}\PYG{l+s+s1}{\PYGZsq{}}\PYG{l+s+s1}{Avergage Return vs. The Worst Return}\PYG{l+s+s1}{\PYGZsq{}}\PYG{p}{)}
\PYG{n}{ax}\PYG{o}{.}\PYG{n}{legend}\PYG{p}{(}\PYG{n}{loc}\PYG{o}{=}\PYG{l+m+mi}{1}\PYG{p}{)}
\PYG{n}{ax2}\PYG{o}{.}\PYG{n}{legend}\PYG{p}{(}\PYG{n}{loc}\PYG{o}{=}\PYG{l+m+mi}{2}\PYG{p}{)}
\PYG{n}{ax}\PYG{o}{.}\PYG{n}{set\PYGZus{}ylabel}\PYG{p}{(}\PYG{l+s+s1}{\PYGZsq{}}\PYG{l+s+s1}{Avg. Return}\PYG{l+s+s1}{\PYGZsq{}}\PYG{p}{)}
\PYG{n}{ax2}\PYG{o}{.}\PYG{n}{set\PYGZus{}ylabel}\PYG{p}{(}\PYG{l+s+s1}{\PYGZsq{}}\PYG{l+s+s1}{The Worst Return}\PYG{l+s+s1}{\PYGZsq{}}\PYG{p}{)}
\PYG{n}{plt}\PYG{o}{.}\PYG{n}{show}\PYG{p}{(}\PYG{p}{)}
\end{sphinxVerbatim}

\end{sphinxuseclass}\end{sphinxVerbatimInput}
\begin{sphinxVerbatimOutput}

\begin{sphinxuseclass}{cell_output}
\noindent\sphinxincludegraphics{{2.4.1_Volatility_Breakout_27_0}.png}

\end{sphinxuseclass}\end{sphinxVerbatimOutput}

\end{sphinxuseclass}

\chapter{종목 찾기}
\label{\detokenize{chapter2/2.4.1_Volatility_Breakout:id3}}
\sphinxAtStartPar
이전 단원에서 최적의 K 를 찾아보았는데요. 이번에는 K 를 고정하고 최적의 종목을 찾아보겠습니다. 코스피로 한정해서 종목을 찾아보겠습니다. FinanceDataReaer 의 StockListing 메소드의 인수로 ‘KOSPI’ 를 넣으면 코스피 모든 종목을 반환합니다.

\begin{sphinxuseclass}{cell}\begin{sphinxVerbatimInput}

\begin{sphinxuseclass}{cell_input}
\begin{sphinxVerbatim}[commandchars=\\\{\}]
\PYG{k+kn}{import} \PYG{n+nn}{FinanceDataReader} \PYG{k}{as} \PYG{n+nn}{fdr} 
\PYG{k+kn}{import} \PYG{n+nn}{pandas} \PYG{k}{as} \PYG{n+nn}{pd}
\PYG{k+kn}{import} \PYG{n+nn}{numpy} \PYG{k}{as} \PYG{n+nn}{np}

\PYG{n}{kospi\PYGZus{}df} \PYG{o}{=} \PYG{n}{fdr}\PYG{o}{.}\PYG{n}{StockListing}\PYG{p}{(}\PYG{l+s+s1}{\PYGZsq{}}\PYG{l+s+s1}{KOSPI}\PYG{l+s+s1}{\PYGZsq{}}\PYG{p}{)}
\PYG{n}{kospi\PYGZus{}df}\PYG{o}{.}\PYG{n}{head}\PYG{p}{(}\PYG{p}{)}\PYG{o}{.}\PYG{n}{style}\PYG{o}{.}\PYG{n}{set\PYGZus{}table\PYGZus{}attributes}\PYG{p}{(}\PYG{l+s+s1}{\PYGZsq{}}\PYG{l+s+s1}{style=}\PYG{l+s+s1}{\PYGZdq{}}\PYG{l+s+s1}{font\PYGZhy{}size: 12px}\PYG{l+s+s1}{\PYGZdq{}}\PYG{l+s+s1}{\PYGZsq{}}\PYG{p}{)}
\end{sphinxVerbatim}

\end{sphinxuseclass}\end{sphinxVerbatimInput}
\begin{sphinxVerbatimOutput}

\begin{sphinxuseclass}{cell_output}
\begin{sphinxVerbatim}[commandchars=\\\{\}]
\PYGZlt{}pandas.io.formats.style.Styler at 0x2ad01acf790\PYGZgt{}
\end{sphinxVerbatim}

\end{sphinxuseclass}\end{sphinxVerbatimOutput}

\end{sphinxuseclass}
\sphinxAtStartPar
 nunique() 은 유니크한 종목 수를 세는 메소드입니다. 6,361 개의 종목이 있습니다. Sector 가 비어있는 종목의 경우는 일반적인 회사의 종목이 아닌 것인 것으로 보입니다. Sector 가 비어있는 종목은 제외하겠습니다. 이제 820 개의 종목만 남았습니다.

\begin{sphinxuseclass}{cell}\begin{sphinxVerbatimInput}

\begin{sphinxuseclass}{cell_input}
\begin{sphinxVerbatim}[commandchars=\\\{\}]
\PYG{n+nb}{print}\PYG{p}{(}\PYG{n}{kospi\PYGZus{}df}\PYG{p}{[}\PYG{l+s+s1}{\PYGZsq{}}\PYG{l+s+s1}{Symbol}\PYG{l+s+s1}{\PYGZsq{}}\PYG{p}{]}\PYG{o}{.}\PYG{n}{nunique}\PYG{p}{(}\PYG{p}{)}\PYG{p}{)}
\PYG{n+nb}{print}\PYG{p}{(}\PYG{n}{kospi\PYGZus{}df}\PYG{p}{[}\PYG{o}{\PYGZti{}}\PYG{n}{kospi\PYGZus{}df}\PYG{p}{[}\PYG{l+s+s1}{\PYGZsq{}}\PYG{l+s+s1}{Sector}\PYG{l+s+s1}{\PYGZsq{}}\PYG{p}{]}\PYG{o}{.}\PYG{n}{isnull}\PYG{p}{(}\PYG{p}{)}\PYG{p}{]}\PYG{p}{[}\PYG{l+s+s1}{\PYGZsq{}}\PYG{l+s+s1}{Symbol}\PYG{l+s+s1}{\PYGZsq{}}\PYG{p}{]}\PYG{o}{.}\PYG{n}{nunique}\PYG{p}{(}\PYG{p}{)}\PYG{p}{)}
\end{sphinxVerbatim}

\end{sphinxuseclass}\end{sphinxVerbatimInput}
\begin{sphinxVerbatimOutput}

\begin{sphinxuseclass}{cell_output}
\begin{sphinxVerbatim}[commandchars=\\\{\}]
6151
821
\end{sphinxVerbatim}

\end{sphinxuseclass}\end{sphinxVerbatimOutput}

\end{sphinxuseclass}
\sphinxAtStartPar
 kospi\_df 에서 필요한 컬럼 ‘Symbol’ 과 ‘Name’ 두 개만 kospi\_list DataFrame 에 저장합니다. 그리고 종목코드 ‘Symbol’ 과 ‘Name’ 을 각 각 ‘code’ 외 ‘name’ 으로 바꿔줍니다. 그리고 나중을 위해서 결과물을 pickle 파일로 저장도 합니다.

\begin{sphinxuseclass}{cell}\begin{sphinxVerbatimInput}

\begin{sphinxuseclass}{cell_input}
\begin{sphinxVerbatim}[commandchars=\\\{\}]
\PYG{n}{kospi\PYGZus{}list} \PYG{o}{=} \PYG{n}{kospi\PYGZus{}df}\PYG{p}{[}\PYG{o}{\PYGZti{}}\PYG{n}{kospi\PYGZus{}df}\PYG{p}{[}\PYG{l+s+s1}{\PYGZsq{}}\PYG{l+s+s1}{Sector}\PYG{l+s+s1}{\PYGZsq{}}\PYG{p}{]}\PYG{o}{.}\PYG{n}{isnull}\PYG{p}{(}\PYG{p}{)}\PYG{p}{]}\PYG{p}{[}\PYG{p}{[}\PYG{l+s+s1}{\PYGZsq{}}\PYG{l+s+s1}{Symbol}\PYG{l+s+s1}{\PYGZsq{}}\PYG{p}{,}\PYG{l+s+s1}{\PYGZsq{}}\PYG{l+s+s1}{Name}\PYG{l+s+s1}{\PYGZsq{}}\PYG{p}{]}\PYG{p}{]}\PYG{o}{.}\PYG{n}{rename}\PYG{p}{(}\PYG{n}{columns}\PYG{o}{=}\PYG{p}{\PYGZob{}}\PYG{l+s+s1}{\PYGZsq{}}\PYG{l+s+s1}{Symbol}\PYG{l+s+s1}{\PYGZsq{}}\PYG{p}{:}\PYG{l+s+s1}{\PYGZsq{}}\PYG{l+s+s1}{code}\PYG{l+s+s1}{\PYGZsq{}}\PYG{p}{,}\PYG{l+s+s1}{\PYGZsq{}}\PYG{l+s+s1}{Name}\PYG{l+s+s1}{\PYGZsq{}}\PYG{p}{:}\PYG{l+s+s1}{\PYGZsq{}}\PYG{l+s+s1}{name}\PYG{l+s+s1}{\PYGZsq{}}\PYG{p}{\PYGZcb{}}\PYG{p}{)}
\PYG{n}{kospi\PYGZus{}list}\PYG{o}{.}\PYG{n}{to\PYGZus{}pickle}\PYG{p}{(}\PYG{l+s+s1}{\PYGZsq{}}\PYG{l+s+s1}{kospi\PYGZus{}list.pkl}\PYG{l+s+s1}{\PYGZsq{}}\PYG{p}{)}
\PYG{n}{kospi\PYGZus{}list} \PYG{o}{=} \PYG{n}{pd}\PYG{o}{.}\PYG{n}{read\PYGZus{}pickle}\PYG{p}{(}\PYG{l+s+s1}{\PYGZsq{}}\PYG{l+s+s1}{kospi\PYGZus{}list.pkl}\PYG{l+s+s1}{\PYGZsq{}}\PYG{p}{)}
\end{sphinxVerbatim}

\end{sphinxuseclass}\end{sphinxVerbatimInput}

\end{sphinxuseclass}
\sphinxAtStartPar
 이제 변동성 돌파 전략의 수익율을 계산하는 함수를 불러옵니다. 일단 K 는 0.5 로 고정합니다. 모든 코스피 종목을 For\sphinxhyphen{}Loop 하면서 가장 수익율이 좋은 종목을 찾으면 됩니다. Loop 를 돌 때마다 종목이름, 종목코드, 평균수익율, 최대손실, 누적수익율을 list 에 저장합니다. 마지막으로 모든 list 를 모아서 하나의 DataFrame 으로 저장합니다. 800 개가 넘는 종목을 Loop 로 하나 씩 하니 시간이 많이 걸립니다. time 모듈을 이용해 총 데이터 처리시간도 측정해 봅니다. 2021년 변동성 돌파전략으로 매수할 수 있는 날이 50일 미만 경우는 무시하도록 if 문을 만들었습니다. 최종 결과를 pickle 로 저장합니다.

\begin{sphinxuseclass}{cell}\begin{sphinxVerbatimInput}

\begin{sphinxuseclass}{cell_input}
\begin{sphinxVerbatim}[commandchars=\\\{\}]
\PYG{k+kn}{import} \PYG{n+nn}{time}

\PYG{n}{start\PYGZus{}time} \PYG{o}{=} \PYG{n}{time}\PYG{o}{.}\PYG{n}{time}\PYG{p}{(}\PYG{p}{)}
\PYG{n}{K} \PYG{o}{=} \PYG{l+m+mf}{0.5}

\PYG{k}{def} \PYG{n+nf}{avg\PYGZus{}r2}\PYG{p}{(}\PYG{n}{code}\PYG{p}{,} \PYG{n}{K}\PYG{p}{)}\PYG{p}{:}
    \PYG{n}{stock} \PYG{o}{=} \PYG{n}{fdr}\PYG{o}{.}\PYG{n}{DataReader}\PYG{p}{(}\PYG{n}{code}\PYG{p}{,}  \PYG{n}{start}\PYG{o}{=}\PYG{l+s+s1}{\PYGZsq{}}\PYG{l+s+s1}{2021\PYGZhy{}01\PYGZhy{}03}\PYG{l+s+s1}{\PYGZsq{}}\PYG{p}{,} \PYG{n}{end}\PYG{o}{=}\PYG{l+s+s1}{\PYGZsq{}}\PYG{l+s+s1}{2021\PYGZhy{}12\PYGZhy{}31}\PYG{l+s+s1}{\PYGZsq{}}\PYG{p}{)}    
    \PYG{n}{stock}\PYG{p}{[}\PYG{l+s+s1}{\PYGZsq{}}\PYG{l+s+s1}{v}\PYG{l+s+s1}{\PYGZsq{}}\PYG{p}{]} \PYG{o}{=} \PYG{p}{(}\PYG{n}{stock}\PYG{p}{[}\PYG{l+s+s1}{\PYGZsq{}}\PYG{l+s+s1}{High}\PYG{l+s+s1}{\PYGZsq{}}\PYG{p}{]}\PYG{o}{.}\PYG{n}{shift}\PYG{p}{(}\PYG{l+m+mi}{1}\PYG{p}{)} \PYG{o}{\PYGZhy{}} \PYG{n}{stock}\PYG{p}{[}\PYG{l+s+s1}{\PYGZsq{}}\PYG{l+s+s1}{Low}\PYG{l+s+s1}{\PYGZsq{}}\PYG{p}{]}\PYG{o}{.}\PYG{n}{shift}\PYG{p}{(}\PYG{l+m+mi}{1}\PYG{p}{)}\PYG{p}{)}\PYG{o}{*}\PYG{n}{K}
    \PYG{n}{stock}\PYG{p}{[}\PYG{l+s+s1}{\PYGZsq{}}\PYG{l+s+s1}{buy\PYGZus{}price}\PYG{l+s+s1}{\PYGZsq{}}\PYG{p}{]} \PYG{o}{=} \PYG{n}{stock}\PYG{p}{[}\PYG{l+s+s1}{\PYGZsq{}}\PYG{l+s+s1}{Open}\PYG{l+s+s1}{\PYGZsq{}}\PYG{p}{]} \PYG{o}{+} \PYG{n}{stock}\PYG{p}{[}\PYG{l+s+s1}{\PYGZsq{}}\PYG{l+s+s1}{v}\PYG{l+s+s1}{\PYGZsq{}}\PYG{p}{]}
    \PYG{n}{stock}\PYG{p}{[}\PYG{l+s+s1}{\PYGZsq{}}\PYG{l+s+s1}{buy}\PYG{l+s+s1}{\PYGZsq{}}\PYG{p}{]} \PYG{o}{=} \PYG{p}{(}\PYG{n}{stock}\PYG{p}{[}\PYG{l+s+s1}{\PYGZsq{}}\PYG{l+s+s1}{High}\PYG{l+s+s1}{\PYGZsq{}}\PYG{p}{]} \PYG{o}{\PYGZgt{}} \PYG{n}{stock}\PYG{p}{[}\PYG{l+s+s1}{\PYGZsq{}}\PYG{l+s+s1}{buy\PYGZus{}price}\PYG{l+s+s1}{\PYGZsq{}}\PYG{p}{]}\PYG{p}{)}\PYG{o}{*}\PYG{p}{(}\PYG{n}{stock}\PYG{p}{[}\PYG{l+s+s1}{\PYGZsq{}}\PYG{l+s+s1}{Low}\PYG{l+s+s1}{\PYGZsq{}}\PYG{p}{]} \PYG{o}{\PYGZlt{}} \PYG{n}{stock}\PYG{p}{[}\PYG{l+s+s1}{\PYGZsq{}}\PYG{l+s+s1}{buy\PYGZus{}price}\PYG{l+s+s1}{\PYGZsq{}}\PYG{p}{]}\PYG{p}{)}\PYG{o}{.}\PYG{n}{astype}\PYG{p}{(}\PYG{n+nb}{int}\PYG{p}{)}
    \PYG{n}{stock}\PYG{p}{[}\PYG{l+s+s1}{\PYGZsq{}}\PYG{l+s+s1}{return}\PYG{l+s+s1}{\PYGZsq{}}\PYG{p}{]} \PYG{o}{=} \PYG{n}{stock}\PYG{p}{[}\PYG{l+s+s1}{\PYGZsq{}}\PYG{l+s+s1}{Open}\PYG{l+s+s1}{\PYGZsq{}}\PYG{p}{]}\PYG{o}{.}\PYG{n}{shift}\PYG{p}{(}\PYG{o}{\PYGZhy{}}\PYG{l+m+mi}{1}\PYG{p}{)}\PYG{o}{/}\PYG{n}{stock}\PYG{p}{[}\PYG{l+s+s1}{\PYGZsq{}}\PYG{l+s+s1}{buy\PYGZus{}price}\PYG{l+s+s1}{\PYGZsq{}}\PYG{p}{]}
    \PYG{n}{n} \PYG{o}{=} \PYG{n+nb}{len}\PYG{p}{(}\PYG{n}{stock}\PYG{p}{[}\PYG{n}{stock}\PYG{p}{[}\PYG{l+s+s1}{\PYGZsq{}}\PYG{l+s+s1}{buy}\PYG{l+s+s1}{\PYGZsq{}}\PYG{p}{]}\PYG{o}{==}\PYG{l+m+mi}{1}\PYG{p}{]}\PYG{p}{)}
    \PYG{n}{r} \PYG{o}{=} \PYG{n}{stock}\PYG{p}{[}\PYG{n}{stock}\PYG{p}{[}\PYG{l+s+s1}{\PYGZsq{}}\PYG{l+s+s1}{buy}\PYG{l+s+s1}{\PYGZsq{}}\PYG{p}{]}\PYG{o}{==}\PYG{l+m+mi}{1}\PYG{p}{]}\PYG{p}{[}\PYG{l+s+s1}{\PYGZsq{}}\PYG{l+s+s1}{return}\PYG{l+s+s1}{\PYGZsq{}}\PYG{p}{]}\PYG{o}{.}\PYG{n}{mean}\PYG{p}{(}\PYG{p}{)}
    \PYG{n}{w} \PYG{o}{=} \PYG{n}{stock}\PYG{p}{[}\PYG{n}{stock}\PYG{p}{[}\PYG{l+s+s1}{\PYGZsq{}}\PYG{l+s+s1}{buy}\PYG{l+s+s1}{\PYGZsq{}}\PYG{p}{]}\PYG{o}{==}\PYG{l+m+mi}{1}\PYG{p}{]}\PYG{p}{[}\PYG{l+s+s1}{\PYGZsq{}}\PYG{l+s+s1}{return}\PYG{l+s+s1}{\PYGZsq{}}\PYG{p}{]}\PYG{o}{.}\PYG{n}{min}\PYG{p}{(}\PYG{p}{)}
    \PYG{n}{c} \PYG{o}{=} \PYG{n}{stock}\PYG{p}{[}\PYG{n}{stock}\PYG{p}{[}\PYG{l+s+s1}{\PYGZsq{}}\PYG{l+s+s1}{buy}\PYG{l+s+s1}{\PYGZsq{}}\PYG{p}{]}\PYG{o}{==}\PYG{l+m+mi}{1}\PYG{p}{]}\PYG{p}{[}\PYG{l+s+s1}{\PYGZsq{}}\PYG{l+s+s1}{return}\PYG{l+s+s1}{\PYGZsq{}}\PYG{p}{]}\PYG{o}{.}\PYG{n}{prod}\PYG{p}{(}\PYG{p}{)}
    \PYG{k}{return} \PYG{n}{n}\PYG{p}{,} \PYG{n}{r}\PYG{p}{,} \PYG{n}{w}\PYG{p}{,} \PYG{n}{c}

\PYG{n}{code\PYGZus{}list} \PYG{o}{=} \PYG{p}{[}\PYG{p}{]}
\PYG{n}{name\PYGZus{}list} \PYG{o}{=} \PYG{p}{[}\PYG{p}{]}
\PYG{n}{return\PYGZus{}list} \PYG{o}{=} \PYG{p}{[}\PYG{p}{]}
\PYG{n}{worst\PYGZus{}list} \PYG{o}{=} \PYG{p}{[}\PYG{p}{]}
\PYG{n}{cumul\PYGZus{}list} \PYG{o}{=} \PYG{p}{[}\PYG{p}{]}

\PYG{k}{for} \PYG{n}{code}\PYG{p}{,} \PYG{n}{name} \PYG{o+ow}{in} \PYG{n+nb}{zip}\PYG{p}{(}\PYG{n}{kospi\PYGZus{}list}\PYG{p}{[}\PYG{l+s+s1}{\PYGZsq{}}\PYG{l+s+s1}{code}\PYG{l+s+s1}{\PYGZsq{}}\PYG{p}{]}\PYG{p}{,} \PYG{n}{kospi\PYGZus{}list}\PYG{p}{[}\PYG{l+s+s1}{\PYGZsq{}}\PYG{l+s+s1}{name}\PYG{l+s+s1}{\PYGZsq{}}\PYG{p}{]}\PYG{p}{)}\PYG{p}{:}

    \PYG{n}{n}\PYG{p}{,} \PYG{n}{r}\PYG{p}{,} \PYG{n}{w}\PYG{p}{,} \PYG{n}{c} \PYG{o}{=} \PYG{n}{avg\PYGZus{}r2}\PYG{p}{(}\PYG{n}{code}\PYG{p}{,} \PYG{n}{K}\PYG{p}{)}

    \PYG{k}{if} \PYG{n}{n} \PYG{o}{\PYGZgt{}}\PYG{o}{=} \PYG{l+m+mi}{50}\PYG{p}{:} \PYG{c+c1}{\PYGZsh{} 최소한 50일 이상 거래일 존재해야 진행}

        \PYG{n}{code\PYGZus{}list}\PYG{o}{.}\PYG{n}{append}\PYG{p}{(}\PYG{n}{code}\PYG{p}{)}
        \PYG{n}{name\PYGZus{}list}\PYG{o}{.}\PYG{n}{append}\PYG{p}{(}\PYG{n}{name}\PYG{p}{)}
        \PYG{n}{return\PYGZus{}list}\PYG{o}{.}\PYG{n}{append}\PYG{p}{(}\PYG{n}{r}\PYG{p}{)}
        \PYG{n}{worst\PYGZus{}list}\PYG{o}{.}\PYG{n}{append}\PYG{p}{(}\PYG{n}{w}\PYG{p}{)}
        \PYG{n}{cumul\PYGZus{}list}\PYG{o}{.}\PYG{n}{append}\PYG{p}{(}\PYG{n}{c}\PYG{p}{)}
        
    \PYG{k}{else}\PYG{p}{:}
        \PYG{k}{continue}

\PYG{n}{outcome} \PYG{o}{=} \PYG{n}{pd}\PYG{o}{.}\PYG{n}{DataFrame}\PYG{p}{(}\PYG{p}{\PYGZob{}}\PYG{l+s+s1}{\PYGZsq{}}\PYG{l+s+s1}{code}\PYG{l+s+s1}{\PYGZsq{}}\PYG{p}{:} \PYG{n}{code\PYGZus{}list}\PYG{p}{,} \PYG{l+s+s1}{\PYGZsq{}}\PYG{l+s+s1}{name}\PYG{l+s+s1}{\PYGZsq{}}\PYG{p}{:} \PYG{n}{name\PYGZus{}list}\PYG{p}{,} \PYG{l+s+s1}{\PYGZsq{}}\PYG{l+s+s1}{return}\PYG{l+s+s1}{\PYGZsq{}}\PYG{p}{:} \PYG{n}{return\PYGZus{}list}\PYG{p}{,} \PYG{l+s+s1}{\PYGZsq{}}\PYG{l+s+s1}{worst}\PYG{l+s+s1}{\PYGZsq{}}\PYG{p}{:} \PYG{n}{worst\PYGZus{}list}\PYG{p}{,} \PYG{l+s+s1}{\PYGZsq{}}\PYG{l+s+s1}{cumul}\PYG{l+s+s1}{\PYGZsq{}}\PYG{p}{:} \PYG{n}{cumul\PYGZus{}list}\PYG{p}{\PYGZcb{}}\PYG{p}{)}    
\PYG{n}{outcome}\PYG{o}{.}\PYG{n}{to\PYGZus{}pickle}\PYG{p}{(}\PYG{l+s+s1}{\PYGZsq{}}\PYG{l+s+s1}{outcome.pkl}\PYG{l+s+s1}{\PYGZsq{}}\PYG{p}{)}

\PYG{n+nb}{print}\PYG{p}{(}\PYG{l+s+sa}{f}\PYG{l+s+s1}{\PYGZsq{}}\PYG{l+s+s1}{총 프로세싱 시간 }\PYG{l+s+si}{\PYGZob{}}\PYG{n}{time}\PYG{o}{.}\PYG{n}{time}\PYG{p}{(}\PYG{p}{)} \PYG{o}{\PYGZhy{}} \PYG{n}{start\PYGZus{}time}\PYG{l+s+si}{\PYGZcb{}}\PYG{l+s+s1}{\PYGZsq{}}\PYG{p}{)}
\end{sphinxVerbatim}

\end{sphinxuseclass}\end{sphinxVerbatimInput}
\begin{sphinxVerbatimOutput}

\begin{sphinxuseclass}{cell_output}
\begin{sphinxVerbatim}[commandchars=\\\{\}]
총 프로세싱 시간 125.55444192886353
\end{sphinxVerbatim}

\end{sphinxuseclass}\end{sphinxVerbatimOutput}

\end{sphinxuseclass}
\sphinxAtStartPar
 저장된 결과물 outcome 을 ‘return’ 값의 내림차순으로 함 보겠습니다. 수익율이 좋은 종목 Top 3 는 ‘조일알미늄’, ‘한전기술’, ‘포스코스틸리온’ 이였습니다. Top 10 에서 최대손실율을 동시에 고려하면, 2021년에는 ‘미원화학’이 좋아 보입니다. 미원화학을 2021년 초부터 변동성 돌파전략으로 매수 매도를 했으면 2021년 연말에는 원금의 2.9배가 되어 있었을 것입니다. 하지만, 지나간 일입니다. 이 책의 4장부터는 데이터 분석으로 미래를 예측하는 방법을 다룹니다.

\begin{sphinxuseclass}{cell}\begin{sphinxVerbatimInput}

\begin{sphinxuseclass}{cell_input}
\begin{sphinxVerbatim}[commandchars=\\\{\}]
\PYG{n}{outcome}\PYG{o}{.}\PYG{n}{sort\PYGZus{}values}\PYG{p}{(}\PYG{n}{by}\PYG{o}{=}\PYG{l+s+s1}{\PYGZsq{}}\PYG{l+s+s1}{return}\PYG{l+s+s1}{\PYGZsq{}}\PYG{p}{,} \PYG{n}{ascending}\PYG{o}{=}\PYG{k+kc}{False}\PYG{p}{)}\PYG{o}{.}\PYG{n}{head}\PYG{p}{(}\PYG{l+m+mi}{10}\PYG{p}{)}\PYG{o}{.}\PYG{n}{style}\PYG{o}{.}\PYG{n}{set\PYGZus{}table\PYGZus{}attributes}\PYG{p}{(}\PYG{l+s+s1}{\PYGZsq{}}\PYG{l+s+s1}{style=}\PYG{l+s+s1}{\PYGZdq{}}\PYG{l+s+s1}{font\PYGZhy{}size: 12px}\PYG{l+s+s1}{\PYGZdq{}}\PYG{l+s+s1}{\PYGZsq{}}\PYG{p}{)}
\end{sphinxVerbatim}

\end{sphinxuseclass}\end{sphinxVerbatimInput}
\begin{sphinxVerbatimOutput}

\begin{sphinxuseclass}{cell_output}
\begin{sphinxVerbatim}[commandchars=\\\{\}]
\PYGZlt{}pandas.io.formats.style.Styler at 0x2ad01a34370\PYGZgt{}
\end{sphinxVerbatim}

\end{sphinxuseclass}\end{sphinxVerbatimOutput}

\end{sphinxuseclass}
\sphinxAtStartPar
 미원화학의 2021년 주가흐름을 함 보겠습니다. 2021년 3월에 급등이 있었습니다.

\begin{sphinxuseclass}{cell}\begin{sphinxVerbatimInput}

\begin{sphinxuseclass}{cell_input}
\begin{sphinxVerbatim}[commandchars=\\\{\}]
\PYG{n}{code} \PYG{o}{=} \PYG{l+s+s1}{\PYGZsq{}}\PYG{l+s+s1}{134380}\PYG{l+s+s1}{\PYGZsq{}} \PYG{c+c1}{\PYGZsh{} 미원화학}
\PYG{n}{stock\PYGZus{}data} \PYG{o}{=} \PYG{n}{fdr}\PYG{o}{.}\PYG{n}{DataReader}\PYG{p}{(}\PYG{n}{code}\PYG{p}{,} \PYG{n}{start}\PYG{o}{=}\PYG{l+s+s1}{\PYGZsq{}}\PYG{l+s+s1}{2021\PYGZhy{}01\PYGZhy{}03}\PYG{l+s+s1}{\PYGZsq{}}\PYG{p}{,} \PYG{n}{end}\PYG{o}{=}\PYG{l+s+s1}{\PYGZsq{}}\PYG{l+s+s1}{2021\PYGZhy{}12\PYGZhy{}31}\PYG{l+s+s1}{\PYGZsq{}}\PYG{p}{)} 
\PYG{n}{stock\PYGZus{}data}\PYG{p}{[}\PYG{l+s+s1}{\PYGZsq{}}\PYG{l+s+s1}{Close}\PYG{l+s+s1}{\PYGZsq{}}\PYG{p}{]}\PYG{o}{.}\PYG{n}{plot}\PYG{p}{(}\PYG{p}{)}
\end{sphinxVerbatim}

\end{sphinxuseclass}\end{sphinxVerbatimInput}
\begin{sphinxVerbatimOutput}

\begin{sphinxuseclass}{cell_output}
\begin{sphinxVerbatim}[commandchars=\\\{\}]
\PYGZlt{}AxesSubplot:xlabel=\PYGZsq{}Date\PYGZsq{}\PYGZgt{}
\end{sphinxVerbatim}

\noindent\sphinxincludegraphics{{2.4.1_Volatility_Breakout_39_1}.png}

\end{sphinxuseclass}\end{sphinxVerbatimOutput}

\end{sphinxuseclass}

\part{chapter 3}


\chapter{\sphinxstylestrong{데이터분석 활용사례}}
\label{\detokenize{chapter3/3.1.0_Use_Case:id1}}\label{\detokenize{chapter3/3.1.0_Use_Case::doc}}
\sphinxAtStartPar
데이터 분석이 현장에서 어떻게 활용되고 있는 지 산업별로 케이스를 만들어 보았습니다.


\section{보험사 사례}
\label{\detokenize{chapter3/3.1.1_Use_Case:id1}}\label{\detokenize{chapter3/3.1.1_Use_Case::doc}}
\sphinxAtStartPar
산업 분야과 관계없이 데이터분석가로 일하는 과정은 대부분 비슷한 일련의 과정을 겪습니다.
\begin{enumerate}
\sphinxsetlistlabels{\arabic}{enumi}{enumii}{}{.}%
\item {} 
\sphinxAtStartPar
문제점 파악

\item {} 
\sphinxAtStartPar
정보 수집 및 전문가 인터뷰

\item {} 
\sphinxAtStartPar
가설 설정

\item {} 
\sphinxAtStartPar
가설 검증을 위한 데이터 수집 및 분석

\item {} 
\sphinxAtStartPar
검정과정에서 발견된 결과을 이용하여 해결책 개발

\item {} 
\sphinxAtStartPar
해결책 테스트 및 해결책의 효과 측정

\end{enumerate}

\sphinxAtStartPar
쉬운 이해를 위하여 보험회사 사례를 들어보겠습니다. 보험회사 영업회의 시간입니다. 영업 상무님이 코로나로 대면 채널이 어려워 텔레마케팅을 시도해 볼 계획인데, 워낙 반응율이 낮아(반응율 1\%, 즉 100 명에게 전화하면 1명이 보험가입), 걱정이라고 하십니다. 그리고 신입 데이터사이언티스인 홍철에게 좋은 방법이 있겠냐고 물어봅니다. 홍철은 고민하다가

\sphinxAtStartPar
(1) 문제점파악: 전화로 보험을 잘 구매할 고객 군을 타겟팅해서 텔레마케팅을 하면 어떻겠냐고 대답합니다. 상무님은 그게 된다면 좋겠지만, 가능하겠냐고 반문하셨고. 이런 저런 이야기로 회의가 마무리 되었습니다. 홍철은 회의 후 자리로 돌아와 고민에 빠졌습니다. 신입으로 데이터사이언티스의 가치를 보여줄 좋은 기회인데, 어떻게 텔레마케팅에 반응율이 좋은 고객군을 찾아낼 수 있을까?

\sphinxAtStartPar
(2) 정보 수집 및 전문가 인터뷰: 일단 전화영업 담당자 인터뷰를 통해 관련 지식과 가설 설정에 도움이 될 만한 정보를 수집해 봅니다. 전화영업 담당자는 주로 인구통계에 의한 결과를 공유해 줍니다. 고령자고, 남성이 더 반응율이 좋다고 합니다. 아주 좋은 정보를 얻었습니다. 또 다른 담당자를 만났습니다. 이 분은 누구에게 전화를 하는 것보다는 어떤 텔레마케터가 전화를 하느냐가 더 중요하다고 합니다. 성과가 좋은 텔레마케터는 연령대에 상관없이 좋은 반응율을 보인다고 합니다. 또 좋은 정보를 얻었습니다. 텔레마케터 지인이 있어, 개인적으로 만나봤습니다. 이 분은 일단 전화를 받을 시간이 있는 사람에게 전화를 해야 한다고 합니다. 아무래도 블루칼라보다는 사무직이 전화받을 시간이 있는 것 같다라고 귀띰을 해 줍니다. 홍철은 소득에 관련해서도 물어봅니다. 하지만 전화를 받는 사람의 소득은 잘 모르겠다고 합니다.

\sphinxAtStartPar
(3)  가설 설정: 현재까지 정보를 바탕으로 몇 가지 가설을 세웁니다.​ 여기서 가설이란 “아마 이런 이유일 때문일 것이다” 하고 추정해보는 것입니다. 예를 들어 의사가 환자를 진단하는 절차도 비슷할 것입니다. 환자가 들어왔습니다. “아랫 배 많이 아픕니다” 라고 이야기를 합니다. 그러면, 의사는 (1) 상한 음식을 먹어서 장염이 발생했나? (2) 화장실을 자주 못가서 그렇나? (3) 아랫 배에 충격이 있었나? 등 여러가지 가설을 설정하고 환자와의 대화를 통하여 해답을 얻을 것입니다.

\sphinxAtStartPar
텔레마케터 담당자와 인터뷰를 통하여 새울 수 있는 가설은 아래와 같습니다. 좋은 가설은 업무 경험에서 나옵니다.
\begin{itemize}
\item {} 
\sphinxAtStartPar
고령자일수록, 남성일 수록 반응율이 좋다.

\item {} 
\sphinxAtStartPar
반응율은 연령대와 상관없이 텔레마케터의 능력에 달려있다.

\item {} 
\sphinxAtStartPar
전화를 받을 시간적 여유가 있는 사람이 반응율이 좋다.

\item {} 
\sphinxAtStartPar
소득이 많을 수록 반응율이 좋다.

\end{itemize}

\sphinxAtStartPar
위 가설을 증명하기 위해서는 데이터를 수집해야 합니다. 하지만, 신규고객을 대상으로 테스트 마케팅을 하지 않는 이상, 위 정보를 얻을 수 는 없습니다. 가장 좋은 방법은 기존 고객을 대상으로 한 과거 캠페인 데이터를 수집하는 것입니다. 과거 기존 고객을 대상으로 한 캠페인 로그파일을 추출합니다. 기존 고객을 대상으로 교차판매 캠페인이므로 반응(신규 보험가입) 여부와 고객 프로파일이 존재합니다.
​
(4) 가설 검증을 위한 데이터 수집 및 분석:  연령별, 성별로 반응율은 분석합니다. 각 텔레마케터 별 연령, 성별 분석도 합니다. 텔레마케터의 프로파일과 대상고객사의 프로파일도 비교합니다. 시간적인 여유가 있는 고객인지는 모르겠습니다. 하지만, 직업군으로 추정해볼 수있습니다. 다행이 보험심사에 수집한 직업군 정보가 있습니다. 사무직이 현장직보다는 시간적인 여유가 있을 거라고 생각합니다. 직업군별로 반응율을 분석합니다. 또 고객의 소득은 모르겠습니다. 하지만, 거주지의 특성으로 소득을 추정해 봅니다. 상식적으로 도곡동 타워팰리스 거주자가 중소도시 아파트 거주자보다는 고소득일 확율이 높습니다.

\sphinxAtStartPar
(5) 검정과정에서 발견된 결과를 이용하여 해결책 개발: 가설 검정 분석에서 여러가지 분석결과를 얻었습니다. 연령별, 성별 평균 반응율, 직업군별 평균 반응율, 소득별 평균 반응율이 알게 되었습니다. 반응율에 유의미한 변수(피쳐) 등을 알아내었고, 반응율이 높은 고객군을 만들어보기로 하였습니다. 분류할 변수가 많아 고객군을 추출하기가 좀 어렵습니다. 이를 해결하기 위해 통계 스코어링 모델을 만들기로 합니다. 반응은 예/아니오는 이진 분류이므로 로지스틱회귀모델을 만들어서 고객 스코어링를 합니다. 그리고 스코어가 높은 순으로 마케팅 대상고객을 선정합니다.

\sphinxAtStartPar
(6) 해결책 테스트 및 해결책의 효과 측정: 로지스틱 회귀모델이 얼마나 효과있는 지 알기 위해서 약 1천명의 고객은 랜덤하게 추출하고, 1천명은 모델 스코어에 의해 추출합니다. 그리고 테스트 텔레마케팅을 하고 반응율의 차이를 비교합니다. 랜덤하게 뽑힌 대상은 이전과 동일하게 1\%의 반응율을 보였고, 모델을 통하여 뽑인 대상은 2\% 반응율을 보였습니다. 즉 2천명을 랜덤하게 뽑았으면 20명의 신규고객을 얻었을 것이고, 회귀 모델로 2천명을 뽑았으면 40명의 고객이 생겼을 것입니다. 이 번 캠페인에서는 천명씩 테스트 했으므로 30명(10명 + 20명) 의 고객이 생겼습니다. 즉 모델로 10명의 고객을 더 획득하였고, 한 고객이 가져오는 현금흐름의 현재가치가 100 만원이라면, 이번 테스트 마케팅에서 보여준 모델의 가치는 1천만원 됩니다.

\sphinxAtStartPar
참고로 데이터분석가 해결책을 만드는 방법은 맥킨지 컨설팅이 해결책을 제시하는 방법과 유사합니다. 맥킨지 컨설팅이 고객의 문제를 해결하기 위해서는 중요하게 생각하는 3 요소는 다음과 같습니다. 첫번째, 선입견없이 아무것도 모른다고 생각하고 문제에 접근할 것(zero based), 두 번째, 생각을 MECE(Mutually Exclusive Collectively Exhaustive) 하게 구조화할 것. 세번째, 가설기반으로 분석하고 해결책을 만들 것(Hypothesis driven). 이 중에 세번째가 빠르게 해결책을 찾는 핵심입니다. 데이터 마이닝이라는 접근법도 있습니다. 가능한 모든데이터를 한 곳에 집중시켜 분석함으로써 알 수 없었던 새로운 통찰을 알아내는 방법인데요. 대표적인 예가 월마트의 기저기와 맥주 에피소드(별도 에피소드 설명)입니다. 하지만, 이 방법은 시간이 오래걸리고, 유의미한 결과를 얻지 못하는 경우도 종종 있습니다. 사례 3 번에서 간단하게 소개하도록 하겠습니다.


\section{신용카드사 사례}
\label{\detokenize{chapter3/3.1.2_Use_Case:id1}}\label{\detokenize{chapter3/3.1.2_Use_Case::doc}}
\sphinxAtStartPar
이 번 사례는 약간 기술적인 내용을 다루어 보겠습니다. 김대리는 머신러닝 석사를 취득하고, 신용카드 사에 입사한 인재입니다. 김대리는 고객데이터 분석을 통해 신용카드 신청자의 연체가능성을 추정하고, 이에 따라 신용카드의 신용한도를 결정하는 신용관리 부서에서 일하고 있습니다. 보통 신용한도는 연체 가능성에 따라 결정하게 됩니다. 연체가능성이 낮으면 높은 신용한도를, 연체가능성이 높으면 낮은 신용한도를 받게 되는 것입니다. 연체 가능성을 파악하기 위해서 신용카드 신청자의 다양한 정보를 분석합니다. 크게 두 가지 데이터 소스가 있습니다. 카드 발급 신청서에 기입한 개인 정보와 크레딧 뷰로우(신용 금융 정보를 집중 관리 하는 기관) 데이터입니다. 신용카드 신청서에는 연령, 성별, 주소, 직업 등의 정보를 기입하도록 되어 있습니다. 크레딧 뷰로우에서는 신청자의 타 은행 신용 정보가 공유되고 있습니다. 대출을 받아보신 분은 경험하셨을 것이라고 생각합니다. 대출 신청을 하면, 담당 금융사는 타 금융사의 대출정보도 조회할 수 있습니다. 이 정보를 크레딧뷰로우가 제공하게 됩니다. 신용 대출액 혹은 카드 현금서비스 사용여부 등이 대표적인 예입니다. 신용카드사에서는 이런 모든 정보를 종합하여 정교한 연체 확률 모델을 개발, 관리하고 있으며, 신규 가입 요청이 들어오면, 신청자의 데이터가 연체 확률 예측 모델의 입력변수로 들어가게 되고, 예측 모델은 연체 확률을 추정하게 됩니다. 연체 예측 모델을 만들기 위해 신용관리 부서에서는 다양한 가설 검증과 데이터 분석으로 예측모델을 개발하고 운영하고 있습니다.

\sphinxAtStartPar
어느날, 재미교포 카스트로씨가 카드사를 방문을 했습니다. 미국에 어렸을 적에 부모님과 같이 이민을 갔다가 다시 한국에 역이민을 왔는데 신용카드가 필요하다는 것입니다. 김대리는 고민에 빠졌습니다. 카스트로씨 같은 경우는 아직 국내 신용거래가 없어, 크레딧 뷰로에 데이터가 존재하지 않습니다. 카드 신청서 개인 정보도 제한적입니다. 예를 들어, 자가 보유 여부, 주택담보대출 여부 등의 정보가 없습니다. 즉, 운용하고 있는 연체 확률 모델을 활용할 수 가 없는 것입니다.

\sphinxAtStartPar
김대리는 이 문제를 해결하기 위해 사내에 있는 통계자료를 수집했습니다. 수집된 통계자료는 연령대별, 성별, 거주지별, 직장별로 평균 연체확률(아래 그림) 이 있습니다. 카스트로씨가 제공할 수 있는 개인정보는 (1) 연령, (2) 성별, (3) 거주지, (4) 직장정보가 전부였습니다. 김대리가 수집한 통계자료는 아래와 같습니다. 그리고 카스트로씨에 해당하는 부분을 회색으로 표시했습니다. 아래 정보를 이용하여 카스트로씨의 연체 확률을 추정할 수 있을까요?

\sphinxAtStartPar


\sphinxAtStartPar
좋은 아이디어가 떠오르지 않았습니다. 김 대리는 이전 직장상사 오 부장님께 전화를 걸었습니다. 오 부장님은 신용분석으로 경력이 20년이상 되신 분이라, 비슷한 경험이 있으실 것이라고 생각했는데, 정말 좋은 해결책을 알려주셨습니다. 오즈(odds) 를 이용한 방법이였습니다. 오즈(odds)란 ‘이벤트가 일어나지 않을 확률’ 대비 ‘이벤트가 일어날 확률’을 의미합니다. 김대리의 문제에서 오즈(odds) 는 (연체할 확률 / 연체하지 않을 확률) 로 계산이 될 수 있습니다. 아래는 오즈(odds) 계산 결과입니다.

\sphinxAtStartPar


\sphinxAtStartPar
또한, 오즈비(odds ratio) 에 대한 이해가 필요합니다. 카스트로씨 연령에 대한 오즈(odds) 는 0.026 이고, 전체 오즈(odds) 는 0.027 이므로, 전체 대비 연령의 오즈비(odds ratio) 는 0.026 를 0.027 로 나눈 0.961 입니다 . 이 값을 의미는 카스트로씨의 연체에 대한 odds 는 전체 odds 대비 96.1\% 낮다고 해석할 수 있습니다. 따라서 아무런 정보가 없는 카스트로씨의 오즈는 0.027 이였지만, 카스트로씨가 30대라는 사실을 알면 우리는 오즈를 조금 줄일 수 있습니다. 카스트로씨가 30대라는 정보를 입수하면, 카스트로씨의 오즈(odds) 는 0.027 * (0.026/0.027) 로 변경됩니다.

\sphinxAtStartPar
같은 방법으로 많은 정보가 많을 수록 카스트로씨의  odds 가 구체화 됩니다. 연령, 성별, 거주지, 지역 정보를 반영하면 카스트로 씨의 odds ratio 는 아래 공식으로 계산을 할 수 있습니다. 카스트로씨의 odds = (전체 odds) * (연령 odds ratio) * (성별 odds ratio) * (거주지 odds ratio) * (직장 odds ratio). 이 공식의 계산 결과는 0.029 가 됩니다. 오즈(odds) 는 P / (1\sphinxhyphen{}P)  즉, ‘연체할 확률’ / ‘연체하지 않을 확률’이므로 P 로 풀어쓰면, 연체 확률 P 는 odds / (1 + odds) 가 됩니다. P 를 계산하면 카스트로씨가 연체할 확률은 2.79\% 가 됩니다. 따라서 김대리는 연체확률 2.79\% 에 해당하는 신용한도를 부여하면 합리적인 결정이라고 할 수 있습니다. 특히 신용관리 부서는 “왜 그런 결론을 내렸는지?” 에 대하여 고객에게 설명을 할 수 있어야 합니다. 예를 들면 “내 옆집은 신용한도가 천만원인데 나는 오백만원이가요? 등의 민원이 있을 수 있습니다. 머신러닝 기반 모델 들은 명확한 해석이 불가능해서 이런 종류의 민원을 근본적으로 해결하지 못합니다. 통계 기반의 모델은 결과에 대한 설명이 가능하므로 이런 종류의 민원에 대응이 가능합니다. 따라서 많은 금융기관이 통계적인 모델링 방식을 아직 선호하고 있습니다.

\sphinxAtStartPar
이 사례에서 데이터 분석을 공부하셨던 분들은 로지스틱 회귀분석과 비슷하다고 느끼 셨을 것입니다. 맞습니다. 위 해결책은 로지스틱 회귀분석 모델링과 동일합니다.
참고로 로지스틱 회귀모델은

\sphinxAtStartPar
\( \ln(odds) \) 를 \( _X \) 의 선형조합 \( (b_0 + b_1 \cdot x_1 + b_2 \cdot x_2 + b_3 \cdot x_3 + b_4  \cdot x_4) \)  의 형태로 설명하는 모델입니다.

\sphinxAtStartPar
\( (카스트로씨 \ odds) \ 는 \ (전체 \ odds) * (연령\ odds\ ratio) * (성별\ odds\ ratio) * (거주지\ odds\ ratio) * (직장\ odds\ ratio) \)로 추정할 수 있다고 사례에서 설명드렸습니다.

\sphinxAtStartPar
양변에 \(log\) 를 씌우면,

\sphinxAtStartPar
\( \ln(카스트로씨\ odds) = \ln(전체\ odds)  + \ln(연령\ odds\ ratio) + \ln(성별\ odds\ ratio)\)\( + \ln(거주지\ odds\ ratio) + \ln(직장\ odds\ ratio) + \ln(직장\ odds\ ratio) \)

\sphinxAtStartPar
따라서,

\sphinxAtStartPar
\( \ln(전체\ odds)\ 는 \ b_0 \),  \( \ln(연령\ odds\ ratio)\ 는\ (b_1 \cdot x_1) \) 에 해당한다는 것을 알 수 있습니다.\( x_1 \) 이 (0,1) 의 바이너리 값이라면 \( b_1 \) 은 해당 연령의 \( odds\ ratio \) 에 \( log \) 를 한 값임을 알 수 있습니다.




\section{소매업 사례}
\label{\detokenize{chapter3/3.1.3_Use_Case:id1}}\label{\detokenize{chapter3/3.1.3_Use_Case::doc}}
\sphinxAtStartPar
문제해결을 위하여 데이터분석을 하는 경우가 대부분입니다.  처음 보험사 사례에서 설명드린 것과 같이 가설을 세우고, 가설을 검증하기 위하여 데이터분석을 합니다. 검증된 가설은 문제해결의 근거가 됩니다. 다른 접근법은 데이터베이스를 알고리즘으로 분석해 패턴을 찾아내는 방법인데, 이를 데이터마이닝 접근법이라고 부릅니다. 이 번 사례에서는 데이터 마이닝 사례 두 가지를 소개하겠습니다.

\sphinxAtStartPar
월마트는 미국의 큰 소매업체입니다. 우리나라 이마트 정도가 비슷할 것 같은데요. 카트에 사고 싶은 물건을 담고, 계산대에서 일괄로 지불합니다. 그 때 구매내역이 찍힌 영수증이 발행됩니다. 이 데이터는 월마트 내부 전산시스템에도 동일하게 저장이 되게 됩니다. 데이터분석가가 어떤 상품들이 같이 판매가 되는지 궁금해 분석을 해 보았습니다. 이 분석은 장바구니 분석(Market Basket Analysis)라고도 부릅니다. 이상하게 금요일 오후에 맥주와 아기 기저귀와 같은 영수증에 동일하게 찍히는 경우가 많다는 것이 분석 결과로 도출되었습니다. 이 사실을 영업총괄 매니저에게 보고를 했고, 총괄 매니저는 맥주 옆에 기저귀를 같이 진열을 해 보았습니다. 그 결과는 맥주와 아기 기저귀 매출이 두 배로 증가했습니다. 의아하게 생각한 매니저는 금요일 진열대 옆에서 어떤 고객들이 맥주와 기저귀를 같이 구매를 하는 지 관찰 해 보았습니다. 알고 보니, 금요일 퇴근한 젊은 아빠들이 스트레스를 받은 표정(와이프의 요청으로 퇴근 후에 피곤한 몸을 이끌고 마트에 온 것으로 추정)으로 기저귀를 사러왔다가 맥주도 같이 사가는 것이였습니다. 이렇듯 데이터마이닝으로 생각하지 못했던 통찰(Insight)를 얻을 수 도 있습니다.

\sphinxAtStartPar
이번에는 타겟 사례입니다. 타겟은 미국의 카탈로그 소매업체입니다. 타겟은 임산부를 위한 특별한 프로모션을 준비했고, 임산부에게 임신기간 중 필요한 다양한 상품을 소개하는 카탈로그 준비했습니다. 일단 임신을 하게되면, 출산과 육아기간 동안 필요한 제품들이 정해져 있는데요. 타겟은 그것을 노리고 대대적인 프로모션을 계획한 것입니다. 카탈로그는 독자 이름으로 우편 배달이 되는데요. 이 임산부용 카탈로그가 어느 한 여고생 집으로 배달이 된 것이였습니다. 그 사실은 안 학생 아버지는 화가 잔뜩 나서 회사에 전화를 걸어 항의했습니다. “우리 아이가 고등학생인데, 무슨 임산부 카탈로를 보내느냐? 당장 담당자가 직접와서 사과를 하지 않으면 업체를 고소하겠다”  그런데 몇 일 후 학생은 아버지에게 이런 이런 일로 임신을 했다고 고백을 하게되었습니다. 타겟의 고객 담당자는 몇 일 후 집으로 찾아왔고, 아버지는 이 사실을 이야기 할 수 밖에 없었습니다. 그렇다면, 타겟은 이 여고생이 임신한 사실을 어떻게 알았을까요?  타겟의 데이터 분석가는 임산부의 구매특성에 대하여 분석을 했었는데요. 대부분의 고객은 임신을 하게되면, 피부 로션과 헤어 샴푸를 화학성분이 없고, 향이 강하지 않은 오가닉 제품으로 변경한다는 사실을 발견했습니다. 바로 그 여고생이 제품을 오가닉으로 갑자기 변경한 고객 중의 하나였던 것입니다. 물론 연령을 고려하지 않은 타겟팅을 한 잘못이 있다고 생각됩니다. 타겟의 에피소드 역시 구매이력 데이터베이스를 분석하다가 예상하지 못한 것을 발견하고 마케팅에 활용한 데이터마이닝 사례 중 하나입니다.


\section{제조업 사례 (난이도 상)}
\label{\detokenize{chapter3/3.1.4_Use_Case:id1}}\label{\detokenize{chapter3/3.1.4_Use_Case::doc}}
\sphinxAtStartPar
제조업의 데이터분석은 금융업이나 소매업보다 조금 더 복잡합니다.  이 제조업 사례의 문제를 풀기 위해서는 두 가지 지식이 필요한데요. 하나는 최소자승법(Least Square Method)이고 다른 하나는 선형계획법(Linear Programming) 입니다.  철강업체 A 에서 환경관리를 담당하고 있는 여과장은 용해 후에 최종적으로 납품되는 Alloy(금속합금) 의 성분에 독성이 강한 X 물질이 다량 함유 되어 있다는 것을 알았습니다. 이 독성물질 X 는 함유율이 임계점 0.05\% 이상되었을 때 독성이 강하게 나타납니다. 따라서, 최종 합금 Alloy 에 독성 물질이 0.05\% 이하가 되도록 관리를 하고 싶습니다. Alloy 는 10 개의 공급 업체에서 금속 Scrap (스크랩) 를 납품받은 후, 몇 개의 업체 Scrap 을 선별 후, 용해해서 만듭니다. 첫 번째 문제는 각 업체에서 공급하는 Scrap 에 함유되어 있는 X 물질의 함유율을 정확히 모르고, 두 번째 문제는 X 물질의 함유율이 낮을 수 록 Scrap 단가가 비싸다는 것입니다. X 물질 함유율이 낮은 Scrap 만을 골라서 용해를 하면 독성 물질 X 함유율을 0.05\% 이하로 관리할 수 있으나, 비용 증가의 문제가 생깁니다. 결국 풀고자하는 문제는 최소한의 비용으로 독성물질의 함유율이 0.05\% 이하가 되도록 Alloy 합금을 만드는 것입니다.

\sphinxAtStartPar
여과장은 아래와 같이 배치( Batch) 데이터베이스를 만들었습니다. 아래 테이블은 샘플 배치 3 개를 보여주고 있습니다. 예를 들어 첫 번째 배치 B123 에는 3 개 업체의 Scrap 이 각 13톤, 5톤, 12톤이 투입되었습니다. 최종적으로 30 톤의 Alloy 가 만들어졌는데요. 독성물질 X 의 함유량을 측정해 보니 0.07\% 가 들어있었고, 이 배치는 0.05\% 를 넘어갔으므로 불합격입니다.  만약 여과장이 각 업체  Scrap 의 X 함유율 P1 \textasciitilde{} P10 을 알고 있다면, 최소비용으로 Alloy 를 만드는 방법을 선형계획법을 풀 수 가 있습니다. 먼저 선형 계획법으로 문제를 풀어보겠습니다.



\sphinxAtStartPar
위 문제를 선형 계획법으로 도식화하면 아래와 같습니다. 배치의 총 비용을 목적함수으로 하고, 목적함수를 최소화하는 투입량  Wi  을 찾습니다. 단, 투입 후 물질 X 의 함유량이 0.05\% 이하여야 한다는 제약이 있습니다.  아래에서 각 \( W,\ C,\ P \) 는  다음을 의미합니다.
\( W \): Weight, \( C \): Cost,  \( P \): Proportion

\sphinxAtStartPar
\sphinxstylestrong{Minimize} \( \sum\ \) \( [ W_i \) * \( C_i  \ ] \) (비용 목적함수) for \( vendor = i \)

\sphinxAtStartPar
\sphinxstylestrong{(제약식)}

\sphinxAtStartPar
\( [ W_i \)* \( P_i  \ ] \) \(\ <= \) \(∑ \ W_i \ * \ 0.05\% \) (물질 X 의 함유량 제약)\( \sum\ W_i \)  =  (필요한 총 톤 수 제약)

\sphinxAtStartPar
이제 여과장은 각 업체의 물질 X 함유량 P 만 알면 위 선형계획법을 활용하여 최소비용으로 Alloy 를 생산하는 업체별 Scrap 투입량  W 를 알아낼 수 있습니다. 파이썬에서 선형계획법(Linear Programming) 은 PuLp 라이브러리에서 구현할 수 있습니다..

\sphinxAtStartPar
문제는 어떻게  함유량 P1 \textasciitilde{} P10 를 알아내는냐입니다. 사실 이 문제를 푸는 방법은 여러가지가 있습니다. 여과장은 최소자승법으로 문제를 풀었는데요. 어떻게 풀었는지 함 보겠습니다.  만약 우리가 P1 \textasciitilde{} P10 을 알고 있다면 각 배치별 물질 X 의 총량은 아래와 같이 추정할 수 있을 것입니다. 그렇다면 추정치 함유량과 실제 함유량의 차이가 최소가 되는 P1 \textasciitilde{} P10 을 찾는 것입니다.




\part{chapter 4}


\chapter{\sphinxstylestrong{데이터분석으로 나만의 전략 만들기}}
\label{\detokenize{chapter4/4.0.0_My_Strategy:id1}}\label{\detokenize{chapter4/4.0.0_My_Strategy::doc}}

\begin{enumerate}
\sphinxsetlistlabels{\arabic}{enumi}{enumii}{}{.}%
\item {} 
\sphinxAtStartPar
문제점 파악과 정의

\item {} 
\sphinxAtStartPar
정보 수집 및 전문가 인터뷰

\item {} 
\sphinxAtStartPar
가설 설정

\item {} 
\sphinxAtStartPar
가설 검증을 위한 데이터 수집 및 분석

\item {} 
\sphinxAtStartPar
검정과정에서 발견된 결과물을 이용하여 해결책 개발

\item {} 
\sphinxAtStartPar
해결책 테스트 및 해결책의 효과 측정

\end{enumerate}

\sphinxAtStartPar
 문제의 정의부터 해결책 개발까지의 과정을 도식화해 보았습니다. 문제를 먼저 구조화하고 가설을 만들면 각 가설 검증이 왜 필요한지 이해하기 쉽고, 가설들의 중복을 피할 수 있습니다.

\sphinxAtStartPar


\sphinxAtStartPar


\sphinxAtStartPar


\sphinxAtStartPar


\sphinxAtStartPar



\chapter{\sphinxstylestrong{문제점 파악}}
\label{\detokenize{chapter4/4.1.0_My_Strategy:id1}}\label{\detokenize{chapter4/4.1.0_My_Strategy::doc}}
\sphinxAtStartPar
주식 매매에 데이터 분석을 활용하는 이유는 간단합니다. 단기 경험으로는 주가가 상승할 종목을 찾기가 상당히 어렵다는 것입니다. 일단 종목이 수 천개 입니다. 수천 개의 종목을 수작업으로 종목별 분석은 거의 불가능할 것 같습니다. 엑셀의 도움을 받을 수 도 있겠으나, 엑셀로 수천개의 종목과 몇 년치 데이터를 다루는 것도 어렵습니다.  따라서 파이썬을 이용해 데이터를 수집하고 분석하여 확률적으로 주가가 상승할 종목을 찾는 알고리즘을 만들고자 합니다. 주식매매 관련하여 데이터분석이 의사결정에 도움을 줄 수 있는 부분은 아래와 같습니다.
\begin{enumerate}
\sphinxsetlistlabels{\arabic}{enumi}{enumii}{}{.}%
\item {} 
\sphinxAtStartPar
매수 종목 선정

\item {} 
\sphinxAtStartPar
매수 및 매도 시점 결정

\end{enumerate}



\sphinxAtStartPar
이제 문제점이 파악되었으면 타겟(예측할) 변수의 결정이 가능합니다.

\sphinxAtStartPar
모델링에서 예측하고자하는 변수를 타겟변수(통계모델에서는 종속변수)라고 합니다. 단기매매를 위한 예측모델에서 우리의 타겟변수는 매수 후 1 주일(5 영업일) 동안의 주가변동입니다. 타겟변수를 어떻게 정의하는냐는 예측모델의 성공과 실패를 결정하는 중요한 요소입니다. 여기서 데이터 분석의 경험이 많은 분석가와 신입 분석가의 차이가 발생하게 됩니다.

\sphinxAtStartPar
모델의 선택에 대하여 시계열 모델을 우선적으로 생각해 볼 수 있습니다. 시계열 모델은 (T\sphinxhyphen{}n) .. (T\sphinxhyphen{}3), (T\sphinxhyphen{}2), (T\sphinxhyphen{}1) 주가 정보를 이용하여 T 시점의 주가 예측하는 방법입니다. 시계열 모델의 경우는 가격 자체를 예측하기 보다는 가격의 변동성(예를 들면 수익율) 을 시계열 변수로 해야 정상성(Stationary) 를 만족하게 됩니다. 정상성은 통계 시계열 모델에서 특히 중요합니다. 또한 머신러닝 및 딥러닝 시계열 모델 등도 학습이 잘 되려면 정상적을 만족해야 합니다. (시계열 모델에서 왜 정상성을 확보해야 하는 이유에 대하여 별도 기술). 하지만 시계열모델은 과거의 주가 흐름이 미래에도 반복한다는 기본 가정이 있습니다.  문제는 주가흐름은 이 가정을 만족하기가 어렵다는 것입니다.  따라서 시계열모델 접근보다는 과거 주가정보의 특징이 요약된 피쳐를 입력변수로 활용하고 매수 후 일 주일간(5 영업일)의 수익률을 예측하는 모델을 구현할 것입니다.​

\sphinxAtStartPar
가상의 익절 수익율(예를 들어 익절 5\%)을 설정하고 매도를 합니다. 매도 후, 수익이 발생한 경우는 1, 나머지는 0 인 타겟변수를 생성합니다. 타겟변수의 특이값(예를 들어 10\% 이상 익절) 이 모두 1 로 치환되므로 모델의 입력변수가 타겟변수의 변화에 민감하게 반응하지 않게 됩니다. 일반적으로 이진 분류 모델의 경우 타겟변수에 입력변수가 민감에게 반응하지 않기 때문에 모델 적합이 잘됩니다. 이는 오버피팅(과적합)을 피하기 위한 한 방법이기도 합니다. 예측모델링을 공부하신 분들은 잘 아시겠지만, 이진 타겟변수에 대하여 다양한 모델을 활용할 수 있습니다. 이진 분류의 문제는 로지스틱 회귀분석이 대표적인 모델입니다. 요약하면 과거의 주가 및 거래량 정보를 요약하며 피쳐를 만든 후, 피쳐들을 이용하여 수익/손해를 추정하는 모델입니다.


\chapter{\sphinxstylestrong{정보 수집 및 전문가 인터뷰}}
\label{\detokenize{chapter4/4.2.0_My_Strategy:id1}}\label{\detokenize{chapter4/4.2.0_My_Strategy::doc}}



\chapter{\sphinxstylestrong{가설 설정}}
\label{\detokenize{chapter4/4.3.0_My_Strategy:id1}}\label{\detokenize{chapter4/4.3.0_My_Strategy::doc}}
\sphinxAtStartPar
유투브및 여러 블로그에서 얻은 정보를 바탕으로 몇 가지 가설을 세웠습니다. 여기서 가설이란 이런 경우 주가가 상승한다하는 가정입니다. 아래는 제가 찾은 정보입니다. 독자분들은 더 좋은 가설을 세우실 수 있을 것이라고 생각합니다. 가설의 깊이가 예측 모델의 성능을 좌우합니다.
\begin{itemize}
\item {} 
\sphinxAtStartPar
변동성이 크고 거래량이 몰린 종목이 주가가 상승한다.

\item {} 
\sphinxAtStartPar
5일 이동 평균선이 오늘 종가보다 위에 위치해 있다.

\item {} 
\sphinxAtStartPar
위 꼬리가 긴 양봉이 자주 발생한다.

\item {} 
\sphinxAtStartPar
거래량이 종종 크게 터지며 매집의 흔적을 보인다.

\item {} 
\sphinxAtStartPar
주가지수보다 더 좋은 수익율을 자주 보여준다.

\item {} 
\sphinxAtStartPar
동종업계의 평균 수익률보다 좋은 수익률을 보여준다.

\item {} 
\sphinxAtStartPar
개인투자자보다는 투신/사모펀드 등이 매수를 많이 한다.

\end{itemize}

\sphinxAtStartPar
마지막 가설(투자자)의 경우는 증권사 API 등을 활용하여 데이터를 추출을 해야하므로 별도의 공부가 필요합니다. 모델의 입력변수로 활용하지는 않겠습니다.


\chapter{\sphinxstylestrong{데이터 수집 및 분석}}
\label{\detokenize{chapter4/4.4.0_Data_Collection:id1}}\label{\detokenize{chapter4/4.4.0_Data_Collection::doc}}
\sphinxAtStartPar
이 부분이 데이터분석에서 가장 많은 시간이 필요한 부분입니다. 데이터를 수집해야하고 분석할 수 있는 모양으로 데이터 가공을 해야합니다. 다행이 일봉데이터는 데이터 크렌징은 필요없습니다. 하지만, 판다스 라이브러리를 이용하여 많은 가공이 필요합니다. 이 부분은 피쳐 엔지니어링이라고도 부릅니다.

\sphinxAtStartPar
일봉 데이터 수집은 웹크롤링으로 수집할 수 도 있고, 파이썬 라이브러리로도 수집이 가능합니다. 일봉 데이터에는 시종고저 값과 거래량 값이 기본적으로 제공됩니다. 우리는 이렇게 5개의 데이터를 이용하여 위 가설을 검정 해 볼 것 입니다.

\begin{sphinxuseclass}{cell}\begin{sphinxVerbatimInput}

\begin{sphinxuseclass}{cell_input}
\begin{sphinxVerbatim}[commandchars=\\\{\}]
\PYG{k+kn}{import} \PYG{n+nn}{FinanceDataReader} \PYG{k}{as} \PYG{n+nn}{fdr}
\PYG{k+kn}{import} \PYG{n+nn}{matplotlib}\PYG{n+nn}{.}\PYG{n+nn}{pyplot} \PYG{k}{as} \PYG{n+nn}{plt}
\PYG{o}{\PYGZpc{}}\PYG{k}{matplotlib} inline
\PYG{k+kn}{import} \PYG{n+nn}{os}
\PYG{k+kn}{import} \PYG{n+nn}{FinanceDataReader} \PYG{k}{as} \PYG{n+nn}{fdr} 
\PYG{k+kn}{import} \PYG{n+nn}{pandas} \PYG{k}{as} \PYG{n+nn}{pd}
\PYG{k+kn}{import} \PYG{n+nn}{numpy} \PYG{k}{as} \PYG{n+nn}{np}
\PYG{k+kn}{import} \PYG{n+nn}{requests}
\PYG{k+kn}{import} \PYG{n+nn}{bs4}

\PYG{n}{pd}\PYG{o}{.}\PYG{n}{options}\PYG{o}{.}\PYG{n}{display}\PYG{o}{.}\PYG{n}{float\PYGZus{}format} \PYG{o}{=} \PYG{l+s+s1}{\PYGZsq{}}\PYG{l+s+si}{\PYGZob{}:,.3f\PYGZcb{}}\PYG{l+s+s1}{\PYGZsq{}}\PYG{o}{.}\PYG{n}{format}
\end{sphinxVerbatim}

\end{sphinxuseclass}\end{sphinxVerbatimInput}

\end{sphinxuseclass}

\section{FinanceDataReader 로 일봉 데이터 가져오기}
\label{\detokenize{chapter4/4.4.1_Data_Collection:financedatareader}}\label{\detokenize{chapter4/4.4.1_Data_Collection::doc}}
\sphinxAtStartPar
가설 분석과 수익율 예측 모델링은 변동성이 큰 코스닥 종목만을 대상으로 하겠습니다.
가설검정을 위하여 과거 수 개월치의 일봉데이터가 필요합니다. 우선 데이터를 종목별로 가져오기 위해서 FinanceDataReader 의 Stocklisting 메소드에서 코스닥의 종목 코드와 정보를 불러옵니다.

\begin{sphinxuseclass}{cell}\begin{sphinxVerbatimInput}

\begin{sphinxuseclass}{cell_input}
\begin{sphinxVerbatim}[commandchars=\\\{\}]
\PYG{n}{kosdaq\PYGZus{}df} \PYG{o}{=} \PYG{n}{fdr}\PYG{o}{.}\PYG{n}{StockListing}\PYG{p}{(}\PYG{l+s+s1}{\PYGZsq{}}\PYG{l+s+s1}{KOSDAQ}\PYG{l+s+s1}{\PYGZsq{}}\PYG{p}{)}
\PYG{n}{kosdaq\PYGZus{}df}\PYG{o}{.}\PYG{n}{head}\PYG{p}{(}\PYG{p}{)}\PYG{o}{.}\PYG{n}{style}\PYG{o}{.}\PYG{n}{set\PYGZus{}table\PYGZus{}attributes}\PYG{p}{(}\PYG{l+s+s1}{\PYGZsq{}}\PYG{l+s+s1}{style=}\PYG{l+s+s1}{\PYGZdq{}}\PYG{l+s+s1}{font\PYGZhy{}size: 12px}\PYG{l+s+s1}{\PYGZdq{}}\PYG{l+s+s1}{\PYGZsq{}}\PYG{p}{)}
\end{sphinxVerbatim}

\end{sphinxuseclass}\end{sphinxVerbatimInput}
\begin{sphinxVerbatimOutput}

\begin{sphinxuseclass}{cell_output}
\begin{sphinxVerbatim}[commandchars=\\\{\}]
\PYGZlt{}pandas.io.formats.style.Styler at 0x199abda1f40\PYGZgt{}
\end{sphinxVerbatim}

\end{sphinxuseclass}\end{sphinxVerbatimOutput}

\end{sphinxuseclass}
\sphinxAtStartPar
 섹터가 정의되지 않은 종목과 2021년 1월 1일 이후 상장된 종목은 제외하겠습니다. 종 1422 개의 종목이 있습니다. 독자분이 책을 보시는 시점에는 종목 수가 바뀌어 있을 것입니다.kosdaq\_df 에서 필요한 컬럼 ‘Symbol’ 과 ‘Name’ 두 개만 kosdaq\_list 에 저장합니다. 그리고 종목코드 ‘Symbol’ 과 ‘Name’ 을 각 각 ‘code’ 외 ‘name’ 으로 바꿔줍니다. 그리고 나중을 위해서 결과물을 pickle 파일로 저장도 합니다.

\begin{sphinxuseclass}{cell}\begin{sphinxVerbatimInput}

\begin{sphinxuseclass}{cell_input}
\begin{sphinxVerbatim}[commandchars=\\\{\}]
\PYG{n+nb}{print}\PYG{p}{(}\PYG{n}{kosdaq\PYGZus{}df}\PYG{p}{[}\PYG{l+s+s1}{\PYGZsq{}}\PYG{l+s+s1}{Symbol}\PYG{l+s+s1}{\PYGZsq{}}\PYG{p}{]}\PYG{o}{.}\PYG{n}{nunique}\PYG{p}{(}\PYG{p}{)}\PYG{p}{)}

\PYG{n}{c1} \PYG{o}{=} \PYG{p}{(}\PYG{n}{kosdaq\PYGZus{}df}\PYG{p}{[}\PYG{l+s+s1}{\PYGZsq{}}\PYG{l+s+s1}{ListingDate}\PYG{l+s+s1}{\PYGZsq{}}\PYG{p}{]}\PYG{o}{\PYGZgt{}}\PYG{l+s+s1}{\PYGZsq{}}\PYG{l+s+s1}{2021\PYGZhy{}01\PYGZhy{}01}\PYG{l+s+s1}{\PYGZsq{}}\PYG{p}{)} \PYG{c+c1}{\PYGZsh{} 2021년 1월 1일 이후 상장된 종목}
\PYG{n}{c2} \PYG{o}{=} \PYG{p}{(}\PYG{n}{kosdaq\PYGZus{}df}\PYG{p}{[}\PYG{l+s+s1}{\PYGZsq{}}\PYG{l+s+s1}{Sector}\PYG{l+s+s1}{\PYGZsq{}}\PYG{p}{]}\PYG{o}{.}\PYG{n}{isnull}\PYG{p}{(}\PYG{p}{)}\PYG{p}{)} \PYG{c+c1}{\PYGZsh{} 섹터 값이 비어있음}
\PYG{n+nb}{print}\PYG{p}{(}\PYG{n}{kosdaq\PYGZus{}df}\PYG{p}{[}\PYG{o}{\PYGZti{}}\PYG{n}{c1} \PYG{o}{\PYGZam{}} \PYG{o}{\PYGZti{}}\PYG{n}{c2}\PYG{p}{]}\PYG{p}{[}\PYG{l+s+s1}{\PYGZsq{}}\PYG{l+s+s1}{Symbol}\PYG{l+s+s1}{\PYGZsq{}}\PYG{p}{]}\PYG{o}{.}\PYG{n}{nunique}\PYG{p}{(}\PYG{p}{)}\PYG{p}{)}  \PYG{c+c1}{\PYGZsh{} c1 이 아니고 c2 가 아닌 종목의 갯 수}

\PYG{n}{kosdaq\PYGZus{}list} \PYG{o}{=} \PYG{n}{kosdaq\PYGZus{}df}\PYG{p}{[}\PYG{o}{\PYGZti{}}\PYG{n}{c1} \PYG{o}{\PYGZam{}} \PYG{o}{\PYGZti{}}\PYG{n}{c2}\PYG{p}{]}\PYG{p}{[}\PYG{p}{[}\PYG{l+s+s1}{\PYGZsq{}}\PYG{l+s+s1}{Symbol}\PYG{l+s+s1}{\PYGZsq{}}\PYG{p}{,}\PYG{l+s+s1}{\PYGZsq{}}\PYG{l+s+s1}{Name}\PYG{l+s+s1}{\PYGZsq{}}\PYG{p}{,}\PYG{l+s+s1}{\PYGZsq{}}\PYG{l+s+s1}{Sector}\PYG{l+s+s1}{\PYGZsq{}}\PYG{p}{]}\PYG{p}{]}\PYG{o}{.}\PYG{n}{rename}\PYG{p}{(}\PYG{n}{columns}\PYG{o}{=}\PYG{p}{\PYGZob{}}\PYG{l+s+s1}{\PYGZsq{}}\PYG{l+s+s1}{Symbol}\PYG{l+s+s1}{\PYGZsq{}}\PYG{p}{:}\PYG{l+s+s1}{\PYGZsq{}}\PYG{l+s+s1}{code}\PYG{l+s+s1}{\PYGZsq{}}\PYG{p}{,}\PYG{l+s+s1}{\PYGZsq{}}\PYG{l+s+s1}{Name}\PYG{l+s+s1}{\PYGZsq{}}\PYG{p}{:}\PYG{l+s+s1}{\PYGZsq{}}\PYG{l+s+s1}{name}\PYG{l+s+s1}{\PYGZsq{}}\PYG{p}{,}\PYG{l+s+s1}{\PYGZsq{}}\PYG{l+s+s1}{Sector}\PYG{l+s+s1}{\PYGZsq{}}\PYG{p}{:}\PYG{l+s+s1}{\PYGZsq{}}\PYG{l+s+s1}{sector}\PYG{l+s+s1}{\PYGZsq{}}\PYG{p}{\PYGZcb{}}\PYG{p}{)}
\PYG{n}{kosdaq\PYGZus{}list}\PYG{o}{.}\PYG{n}{to\PYGZus{}pickle}\PYG{p}{(}\PYG{l+s+s1}{\PYGZsq{}}\PYG{l+s+s1}{kosdaq\PYGZus{}list.pkl}\PYG{l+s+s1}{\PYGZsq{}}\PYG{p}{)}
\end{sphinxVerbatim}

\end{sphinxuseclass}\end{sphinxVerbatimInput}
\begin{sphinxVerbatimOutput}

\begin{sphinxuseclass}{cell_output}
\begin{sphinxVerbatim}[commandchars=\\\{\}]
1579
1417
\end{sphinxVerbatim}

\end{sphinxuseclass}\end{sphinxVerbatimOutput}

\end{sphinxuseclass}
\sphinxAtStartPar
 저장한 pickle 파일을 읽고, sector 가 몇개나 있는 지 세어봅니다.

\begin{sphinxuseclass}{cell}\begin{sphinxVerbatimInput}

\begin{sphinxuseclass}{cell_input}
\begin{sphinxVerbatim}[commandchars=\\\{\}]
\PYG{n}{kosdaq\PYGZus{}list} \PYG{o}{=} \PYG{n}{pd}\PYG{o}{.}\PYG{n}{read\PYGZus{}pickle}\PYG{p}{(}\PYG{l+s+s1}{\PYGZsq{}}\PYG{l+s+s1}{kosdaq\PYGZus{}list.pkl}\PYG{l+s+s1}{\PYGZsq{}}\PYG{p}{)}
\PYG{n}{kosdaq\PYGZus{}list}\PYG{p}{[}\PYG{l+s+s1}{\PYGZsq{}}\PYG{l+s+s1}{sector}\PYG{l+s+s1}{\PYGZsq{}}\PYG{p}{]}\PYG{o}{.}\PYG{n}{nunique}\PYG{p}{(}\PYG{p}{)}
\end{sphinxVerbatim}

\end{sphinxuseclass}\end{sphinxVerbatimInput}
\begin{sphinxVerbatimOutput}

\begin{sphinxuseclass}{cell_output}
\begin{sphinxVerbatim}[commandchars=\\\{\}]
132
\end{sphinxVerbatim}

\end{sphinxuseclass}\end{sphinxVerbatimOutput}

\end{sphinxuseclass}
\sphinxAtStartPar
 For Loop 에서 kosdaq\_list 의 종목코드와 종목이름을 하나씩 불러서 DataReader 로 2021년 1월 3일부터 2022년 3월 31일까지 일봉데이터를 수집합니다.

\begin{sphinxuseclass}{cell}\begin{sphinxVerbatimInput}

\begin{sphinxuseclass}{cell_input}
\begin{sphinxVerbatim}[commandchars=\\\{\}]
\PYG{n}{price\PYGZus{}data} \PYG{o}{=} \PYG{n}{pd}\PYG{o}{.}\PYG{n}{DataFrame}\PYG{p}{(}\PYG{p}{)}

\PYG{k}{for} \PYG{n}{code}\PYG{p}{,} \PYG{n}{name} \PYG{o+ow}{in} \PYG{n+nb}{zip}\PYG{p}{(}\PYG{n}{kosdaq\PYGZus{}list}\PYG{p}{[}\PYG{l+s+s1}{\PYGZsq{}}\PYG{l+s+s1}{code}\PYG{l+s+s1}{\PYGZsq{}}\PYG{p}{]}\PYG{p}{,} \PYG{n}{kosdaq\PYGZus{}list}\PYG{p}{[}\PYG{l+s+s1}{\PYGZsq{}}\PYG{l+s+s1}{name}\PYG{l+s+s1}{\PYGZsq{}}\PYG{p}{]}\PYG{p}{)}\PYG{p}{:}  \PYG{c+c1}{\PYGZsh{} 코스닥 모든 종목에서 대하여 반복}
    \PYG{n}{daily\PYGZus{}price} \PYG{o}{=} \PYG{n}{fdr}\PYG{o}{.}\PYG{n}{DataReader}\PYG{p}{(}\PYG{n}{code}\PYG{p}{,}  \PYG{n}{start}\PYG{o}{=}\PYG{l+s+s1}{\PYGZsq{}}\PYG{l+s+s1}{2021\PYGZhy{}01\PYGZhy{}03}\PYG{l+s+s1}{\PYGZsq{}}\PYG{p}{,} \PYG{n}{end}\PYG{o}{=}\PYG{l+s+s1}{\PYGZsq{}}\PYG{l+s+s1}{2022\PYGZhy{}03\PYGZhy{}31}\PYG{l+s+s1}{\PYGZsq{}}\PYG{p}{)} \PYG{c+c1}{\PYGZsh{} 종목, 일봉, 데이터 갯수}
    \PYG{n}{daily\PYGZus{}price}\PYG{p}{[}\PYG{l+s+s1}{\PYGZsq{}}\PYG{l+s+s1}{code}\PYG{l+s+s1}{\PYGZsq{}}\PYG{p}{]} \PYG{o}{=} \PYG{n}{code}
    \PYG{n}{daily\PYGZus{}price}\PYG{p}{[}\PYG{l+s+s1}{\PYGZsq{}}\PYG{l+s+s1}{name}\PYG{l+s+s1}{\PYGZsq{}}\PYG{p}{]} \PYG{o}{=} \PYG{n}{name}
    \PYG{n}{price\PYGZus{}data} \PYG{o}{=} \PYG{n}{pd}\PYG{o}{.}\PYG{n}{concat}\PYG{p}{(}\PYG{p}{[}\PYG{n}{price\PYGZus{}data}\PYG{p}{,} \PYG{n}{daily\PYGZus{}price}\PYG{p}{]}\PYG{p}{,} \PYG{n}{axis}\PYG{o}{=}\PYG{l+m+mi}{0}\PYG{p}{)}   

\PYG{n}{price\PYGZus{}data}\PYG{o}{.}\PYG{n}{index}\PYG{o}{.}\PYG{n}{name} \PYG{o}{=} \PYG{l+s+s1}{\PYGZsq{}}\PYG{l+s+s1}{date}\PYG{l+s+s1}{\PYGZsq{}}
\PYG{n}{price\PYGZus{}data}\PYG{o}{.}\PYG{n}{columns}\PYG{o}{=} \PYG{n}{price\PYGZus{}data}\PYG{o}{.}\PYG{n}{columns}\PYG{o}{.}\PYG{n}{str}\PYG{o}{.}\PYG{n}{lower}\PYG{p}{(}\PYG{p}{)} \PYG{c+c1}{\PYGZsh{} 컬럼 이름 소문자로 변경}
\PYG{n}{price\PYGZus{}data}\PYG{o}{.}\PYG{n}{to\PYGZus{}pickle}\PYG{p}{(}\PYG{l+s+s1}{\PYGZsq{}}\PYG{l+s+s1}{stock\PYGZus{}data\PYGZus{}from\PYGZus{}fdr.pkl}\PYG{l+s+s1}{\PYGZsq{}}\PYG{p}{)}
\end{sphinxVerbatim}

\end{sphinxuseclass}\end{sphinxVerbatimInput}

\end{sphinxuseclass}
\sphinxAtStartPar
 저장한 pickle 파일을 다시 읽어 첫 5 라인을 head 메소드로 찍어보면 아래와 같습니다. 여기서 date가 인덱스로 처리되어 있다는 것을 기억해주시면 좋습니다. 타이핑 편의를 위해 컬럼이름을 소문자료 변경하겠습니다.

\begin{sphinxuseclass}{cell}\begin{sphinxVerbatimInput}

\begin{sphinxuseclass}{cell_input}
\begin{sphinxVerbatim}[commandchars=\\\{\}]
\PYG{n}{price\PYGZus{}data} \PYG{o}{=} \PYG{n}{pd}\PYG{o}{.}\PYG{n}{read\PYGZus{}pickle}\PYG{p}{(}\PYG{l+s+s1}{\PYGZsq{}}\PYG{l+s+s1}{stock\PYGZus{}data\PYGZus{}from\PYGZus{}fdr.pkl}\PYG{l+s+s1}{\PYGZsq{}}\PYG{p}{)}
\PYG{n}{price\PYGZus{}data}\PYG{o}{.}\PYG{n}{head}\PYG{p}{(}\PYG{p}{)}\PYG{o}{.}\PYG{n}{style}\PYG{o}{.}\PYG{n}{set\PYGZus{}table\PYGZus{}attributes}\PYG{p}{(}\PYG{l+s+s1}{\PYGZsq{}}\PYG{l+s+s1}{style=}\PYG{l+s+s1}{\PYGZdq{}}\PYG{l+s+s1}{font\PYGZhy{}size: 12px}\PYG{l+s+s1}{\PYGZdq{}}\PYG{l+s+s1}{\PYGZsq{}}\PYG{p}{)}
\end{sphinxVerbatim}

\end{sphinxuseclass}\end{sphinxVerbatimInput}
\begin{sphinxVerbatimOutput}

\begin{sphinxuseclass}{cell_output}
\begin{sphinxVerbatim}[commandchars=\\\{\}]
\PYGZlt{}pandas.io.formats.style.Styler at 0x199b0f463a0\PYGZgt{}
\end{sphinxVerbatim}

\end{sphinxuseclass}\end{sphinxVerbatimOutput}

\end{sphinxuseclass}
\sphinxAtStartPar
 몇 개의 종목이 있고, 각 종목별 일봉의 갯 수 가 몇 개인지 확인해 보겠습니다. 종목 수는 1417 개, 307 개의 일봉이 있습니다.

\begin{sphinxuseclass}{cell}\begin{sphinxVerbatimInput}

\begin{sphinxuseclass}{cell_input}
\begin{sphinxVerbatim}[commandchars=\\\{\}]
\PYG{n+nb}{print}\PYG{p}{(}\PYG{n}{price\PYGZus{}data}\PYG{p}{[}\PYG{l+s+s1}{\PYGZsq{}}\PYG{l+s+s1}{code}\PYG{l+s+s1}{\PYGZsq{}}\PYG{p}{]}\PYG{o}{.}\PYG{n}{nunique}\PYG{p}{(}\PYG{p}{)}\PYG{p}{)}
\PYG{n+nb}{print}\PYG{p}{(}\PYG{n}{price\PYGZus{}data}\PYG{o}{.}\PYG{n}{groupby}\PYG{p}{(}\PYG{l+s+s1}{\PYGZsq{}}\PYG{l+s+s1}{code}\PYG{l+s+s1}{\PYGZsq{}}\PYG{p}{)}\PYG{p}{[}\PYG{l+s+s1}{\PYGZsq{}}\PYG{l+s+s1}{close}\PYG{l+s+s1}{\PYGZsq{}}\PYG{p}{]}\PYG{o}{.}\PYG{n}{count}\PYG{p}{(}\PYG{p}{)}\PYG{o}{.}\PYG{n}{agg}\PYG{p}{(}\PYG{p}{[}\PYG{l+s+s1}{\PYGZsq{}}\PYG{l+s+s1}{min}\PYG{l+s+s1}{\PYGZsq{}}\PYG{p}{,}\PYG{l+s+s1}{\PYGZsq{}}\PYG{l+s+s1}{max}\PYG{l+s+s1}{\PYGZsq{}}\PYG{p}{]}\PYG{p}{)}\PYG{p}{)}
\end{sphinxVerbatim}

\end{sphinxuseclass}\end{sphinxVerbatimInput}
\begin{sphinxVerbatimOutput}

\begin{sphinxuseclass}{cell_output}
\begin{sphinxVerbatim}[commandchars=\\\{\}]
1417
min    307
max    307
Name: close, dtype: int64
\end{sphinxVerbatim}

\end{sphinxuseclass}\end{sphinxVerbatimOutput}

\end{sphinxuseclass}



\section{네이버 증권 웹크롤링으로 일봉 가져오기}
\label{\detokenize{chapter4/4.4.1_Data_Collection:id1}}
\sphinxAtStartPar
이 번에는 네이버 증권 차트 \sphinxstyleemphasis{(네이버 차트 예시 필요)} 에서 데이터를 가져오는 방법도 시도해 보겠습니다. 웹 크롤링은 코드가 복잡합니다. 첫 번째 방법인 FinanceDataReader 로 추출하는 방법을 추천드립니다.다시 pickle 파일을 읽습니다. make\_price\_data 함수는 ‘종목’, ‘추출단위’, ‘데이터 건수’ 를 인자로 네이버증권에서 데이터를 가져오는 함수입니다. 인자는 작은 따옴표에 넣어야 합니다. 셀트리온 헬스케어(091990) 의 일봉 데이터를 최근 300 일 가져오고 싶다면  make\_price\_data(‘091990’, ‘day’, ‘300’) 와 같이 호출합니다. 이 함수를 for 문을 이용해 모든 코스닥 종목에서 대하여 호출하고, 각 결과를 price\_data 라는 데이터프레임에 담습니다.
for 문을 돌리고 결과를 concat 함수로 연속으로 저장하는 방법은 자주 활용되는 기법입니다.

\begin{sphinxuseclass}{cell}\begin{sphinxVerbatimInput}

\begin{sphinxuseclass}{cell_input}
\begin{sphinxVerbatim}[commandchars=\\\{\}]
\PYG{c+c1}{\PYGZsh{} 네이버 증권 차트에서 데이터 크롤링}

\PYG{n}{kosdaq\PYGZus{}list} \PYG{o}{=} \PYG{n}{pd}\PYG{o}{.}\PYG{n}{read\PYGZus{}pickle}\PYG{p}{(}\PYG{l+s+s1}{\PYGZsq{}}\PYG{l+s+s1}{kosdaq\PYGZus{}list.pkl}\PYG{l+s+s1}{\PYGZsq{}}\PYG{p}{)}

\PYG{k}{def} \PYG{n+nf}{make\PYGZus{}price\PYGZus{}data}\PYG{p}{(}\PYG{n}{code}\PYG{p}{,} \PYG{n}{name}\PYG{p}{,} \PYG{n}{timeframe}\PYG{p}{,} \PYG{n}{count}\PYG{p}{)}\PYG{p}{:}
    \PYG{n}{url} \PYG{o}{=} \PYG{l+s+s1}{\PYGZsq{}}\PYG{l+s+s1}{https://fchart.stock.naver.com/sise.nhn?symbol=}\PYG{l+s+s1}{\PYGZsq{}} \PYG{o}{+} \PYG{n}{code} \PYG{o}{+} \PYG{l+s+s1}{\PYGZsq{}}\PYG{l+s+s1}{\PYGZam{}timeframe=}\PYG{l+s+s1}{\PYGZsq{}} \PYG{o}{+} \PYG{n}{timeframe} \PYG{o}{+} \PYG{l+s+s1}{\PYGZsq{}}\PYG{l+s+s1}{\PYGZam{}count=}\PYG{l+s+s1}{\PYGZsq{}} \PYG{o}{+} \PYG{n}{count} \PYG{o}{+} \PYG{l+s+s1}{\PYGZsq{}}\PYG{l+s+s1}{\PYGZam{}requestType=0}\PYG{l+s+s1}{\PYGZsq{}}
    \PYG{n}{price\PYGZus{}data} \PYG{o}{=} \PYG{n}{requests}\PYG{o}{.}\PYG{n}{get}\PYG{p}{(}\PYG{n}{url}\PYG{p}{)}
    \PYG{n}{price\PYGZus{}data\PYGZus{}bs} \PYG{o}{=} \PYG{n}{bs4}\PYG{o}{.}\PYG{n}{BeautifulSoup}\PYG{p}{(}\PYG{n}{price\PYGZus{}data}\PYG{o}{.}\PYG{n}{text}\PYG{p}{,} \PYG{l+s+s1}{\PYGZsq{}}\PYG{l+s+s1}{lxml}\PYG{l+s+s1}{\PYGZsq{}}\PYG{p}{)}
    \PYG{n}{item\PYGZus{}list} \PYG{o}{=} \PYG{n}{price\PYGZus{}data\PYGZus{}bs}\PYG{o}{.}\PYG{n}{find\PYGZus{}all}\PYG{p}{(}\PYG{l+s+s1}{\PYGZsq{}}\PYG{l+s+s1}{item}\PYG{l+s+s1}{\PYGZsq{}}\PYG{p}{)}

    \PYG{n}{date\PYGZus{}list} \PYG{o}{=} \PYG{p}{[}\PYG{p}{]} 
    \PYG{n}{open\PYGZus{}list} \PYG{o}{=} \PYG{p}{[}\PYG{p}{]}
    \PYG{n}{high\PYGZus{}list} \PYG{o}{=} \PYG{p}{[}\PYG{p}{]}
    \PYG{n}{low\PYGZus{}list} \PYG{o}{=} \PYG{p}{[}\PYG{p}{]}
    \PYG{n}{close\PYGZus{}list} \PYG{o}{=} \PYG{p}{[}\PYG{p}{]}
    \PYG{n}{trade\PYGZus{}list} \PYG{o}{=} \PYG{p}{[}\PYG{p}{]}

    \PYG{k}{for} \PYG{n}{item} \PYG{o+ow}{in} \PYG{n}{item\PYGZus{}list}\PYG{p}{:}
        \PYG{n}{data} \PYG{o}{=} \PYG{n}{item}\PYG{p}{[}\PYG{l+s+s1}{\PYGZsq{}}\PYG{l+s+s1}{data}\PYG{l+s+s1}{\PYGZsq{}}\PYG{p}{]}\PYG{o}{.}\PYG{n}{split}\PYG{p}{(}\PYG{l+s+s1}{\PYGZsq{}}\PYG{l+s+s1}{|}\PYG{l+s+s1}{\PYGZsq{}}\PYG{p}{)}
        \PYG{n}{date\PYGZus{}list}\PYG{o}{.}\PYG{n}{append}\PYG{p}{(}\PYG{n}{data}\PYG{p}{[}\PYG{l+m+mi}{0}\PYG{p}{]}\PYG{p}{)}
        \PYG{n}{open\PYGZus{}list}\PYG{o}{.}\PYG{n}{append}\PYG{p}{(}\PYG{n}{data}\PYG{p}{[}\PYG{l+m+mi}{1}\PYG{p}{]}\PYG{p}{)}
        \PYG{n}{high\PYGZus{}list}\PYG{o}{.}\PYG{n}{append}\PYG{p}{(}\PYG{n}{data}\PYG{p}{[}\PYG{l+m+mi}{2}\PYG{p}{]}\PYG{p}{)}
        \PYG{n}{low\PYGZus{}list}\PYG{o}{.}\PYG{n}{append}\PYG{p}{(}\PYG{n}{data}\PYG{p}{[}\PYG{l+m+mi}{3}\PYG{p}{]}\PYG{p}{)}
        \PYG{n}{close\PYGZus{}list}\PYG{o}{.}\PYG{n}{append}\PYG{p}{(}\PYG{n}{data}\PYG{p}{[}\PYG{l+m+mi}{4}\PYG{p}{]}\PYG{p}{)}
        \PYG{n}{trade\PYGZus{}list}\PYG{o}{.}\PYG{n}{append}\PYG{p}{(}\PYG{n}{data}\PYG{p}{[}\PYG{l+m+mi}{5}\PYG{p}{]}\PYG{p}{)}        

    \PYG{n}{price\PYGZus{}df} \PYG{o}{=} \PYG{n}{pd}\PYG{o}{.}\PYG{n}{DataFrame}\PYG{p}{(}\PYG{p}{\PYGZob{}}\PYG{l+s+s1}{\PYGZsq{}}\PYG{l+s+s1}{open}\PYG{l+s+s1}{\PYGZsq{}}\PYG{p}{:} \PYG{n}{open\PYGZus{}list}\PYG{p}{,} \PYG{l+s+s1}{\PYGZsq{}}\PYG{l+s+s1}{high}\PYG{l+s+s1}{\PYGZsq{}}\PYG{p}{:} \PYG{n}{high\PYGZus{}list}\PYG{p}{,} \PYG{l+s+s1}{\PYGZsq{}}\PYG{l+s+s1}{low}\PYG{l+s+s1}{\PYGZsq{}}\PYG{p}{:} \PYG{n}{low\PYGZus{}list}\PYG{p}{,} \PYG{l+s+s1}{\PYGZsq{}}\PYG{l+s+s1}{close}\PYG{l+s+s1}{\PYGZsq{}}\PYG{p}{:} \PYG{n}{close\PYGZus{}list}\PYG{p}{,} \PYG{l+s+s1}{\PYGZsq{}}\PYG{l+s+s1}{volume}\PYG{l+s+s1}{\PYGZsq{}}\PYG{p}{:} \PYG{n}{trade\PYGZus{}list}\PYG{p}{\PYGZcb{}}\PYG{p}{,} \PYG{n}{index}\PYG{o}{=}\PYG{n}{date\PYGZus{}list}\PYG{p}{)}            
    \PYG{n}{price\PYGZus{}df}\PYG{p}{[}\PYG{l+s+s1}{\PYGZsq{}}\PYG{l+s+s1}{code}\PYG{l+s+s1}{\PYGZsq{}}\PYG{p}{]} \PYG{o}{=} \PYG{n}{code}
    \PYG{n}{price\PYGZus{}df}\PYG{p}{[}\PYG{l+s+s1}{\PYGZsq{}}\PYG{l+s+s1}{name}\PYG{l+s+s1}{\PYGZsq{}}\PYG{p}{]} \PYG{o}{=} \PYG{n}{name}
    \PYG{n}{num\PYGZus{}vars} \PYG{o}{=} \PYG{p}{[}\PYG{l+s+s1}{\PYGZsq{}}\PYG{l+s+s1}{open}\PYG{l+s+s1}{\PYGZsq{}}\PYG{p}{,}\PYG{l+s+s1}{\PYGZsq{}}\PYG{l+s+s1}{high}\PYG{l+s+s1}{\PYGZsq{}}\PYG{p}{,}\PYG{l+s+s1}{\PYGZsq{}}\PYG{l+s+s1}{low}\PYG{l+s+s1}{\PYGZsq{}}\PYG{p}{,}\PYG{l+s+s1}{\PYGZsq{}}\PYG{l+s+s1}{close}\PYG{l+s+s1}{\PYGZsq{}}\PYG{p}{,}\PYG{l+s+s1}{\PYGZsq{}}\PYG{l+s+s1}{volume}\PYG{l+s+s1}{\PYGZsq{}}\PYG{p}{]}
    \PYG{n}{char\PYGZus{}vars} \PYG{o}{=} \PYG{p}{[}\PYG{l+s+s1}{\PYGZsq{}}\PYG{l+s+s1}{code}\PYG{l+s+s1}{\PYGZsq{}}\PYG{p}{,}\PYG{l+s+s1}{\PYGZsq{}}\PYG{l+s+s1}{name}\PYG{l+s+s1}{\PYGZsq{}}\PYG{p}{]}
    \PYG{n}{price\PYGZus{}df} \PYG{o}{=} \PYG{n}{price\PYGZus{}df}\PYG{o}{.}\PYG{n}{reindex}\PYG{p}{(}\PYG{n}{columns} \PYG{o}{=} \PYG{n}{char\PYGZus{}vars} \PYG{o}{+} \PYG{n}{num\PYGZus{}vars}\PYG{p}{)}

    \PYG{k}{for} \PYG{n}{var} \PYG{o+ow}{in} \PYG{n}{num\PYGZus{}vars}\PYG{p}{:}
        \PYG{n}{price\PYGZus{}df}\PYG{p}{[}\PYG{n}{var}\PYG{p}{]} \PYG{o}{=} \PYG{n}{pd}\PYG{o}{.}\PYG{n}{to\PYGZus{}numeric}\PYG{p}{(}\PYG{n}{price\PYGZus{}df}\PYG{p}{[}\PYG{n}{var}\PYG{p}{]}\PYG{p}{,} \PYG{n}{errors}\PYG{o}{=}\PYG{l+s+s1}{\PYGZsq{}}\PYG{l+s+s1}{coerce}\PYG{l+s+s1}{\PYGZsq{}}\PYG{p}{)}

    \PYG{n}{price\PYGZus{}df}\PYG{o}{.}\PYG{n}{index} \PYG{o}{=} \PYG{n}{pd}\PYG{o}{.}\PYG{n}{to\PYGZus{}datetime}\PYG{p}{(}\PYG{n}{price\PYGZus{}df}\PYG{o}{.}\PYG{n}{index}\PYG{p}{,} \PYG{n}{errors}\PYG{o}{=}\PYG{l+s+s1}{\PYGZsq{}}\PYG{l+s+s1}{coerce}\PYG{l+s+s1}{\PYGZsq{}}\PYG{p}{)}

    \PYG{k}{return} \PYG{n}{price\PYGZus{}df}

\PYG{n}{price\PYGZus{}data} \PYG{o}{=} \PYG{n}{pd}\PYG{o}{.}\PYG{n}{DataFrame}\PYG{p}{(}\PYG{p}{)}

\PYG{k}{for} \PYG{n}{code}\PYG{p}{,} \PYG{n}{name} \PYG{o+ow}{in} \PYG{n+nb}{zip}\PYG{p}{(}\PYG{n}{kosdaq\PYGZus{}list}\PYG{p}{[}\PYG{l+s+s1}{\PYGZsq{}}\PYG{l+s+s1}{code}\PYG{l+s+s1}{\PYGZsq{}}\PYG{p}{]}\PYG{p}{,} \PYG{n}{kosdaq\PYGZus{}list}\PYG{p}{[}\PYG{l+s+s1}{\PYGZsq{}}\PYG{l+s+s1}{name}\PYG{l+s+s1}{\PYGZsq{}}\PYG{p}{]}\PYG{p}{)}\PYG{p}{:}  \PYG{c+c1}{\PYGZsh{} 코스닥 모든 종목에서 대하여 반복}
    \PYG{n}{daily\PYGZus{}price} \PYG{o}{=} \PYG{n}{make\PYGZus{}price\PYGZus{}data}\PYG{p}{(}\PYG{n}{code}\PYG{p}{,} \PYG{n}{name}\PYG{p}{,} \PYG{l+s+s1}{\PYGZsq{}}\PYG{l+s+s1}{day}\PYG{l+s+s1}{\PYGZsq{}}\PYG{p}{,} \PYG{l+s+s1}{\PYGZsq{}}\PYG{l+s+s1}{307}\PYG{l+s+s1}{\PYGZsq{}}\PYG{p}{)} \PYG{c+c1}{\PYGZsh{} 종목, 일봉, 데이터 갯수}
    \PYG{n}{price\PYGZus{}data} \PYG{o}{=} \PYG{n}{pd}\PYG{o}{.}\PYG{n}{concat}\PYG{p}{(}\PYG{p}{[}\PYG{n}{price\PYGZus{}data}\PYG{p}{,} \PYG{n}{daily\PYGZus{}price}\PYG{p}{]}\PYG{p}{,} \PYG{n}{axis}\PYG{o}{=}\PYG{l+m+mi}{0}\PYG{p}{)}   

\PYG{n}{price\PYGZus{}data}\PYG{o}{.}\PYG{n}{index}\PYG{o}{.}\PYG{n}{name} \PYG{o}{=} \PYG{l+s+s1}{\PYGZsq{}}\PYG{l+s+s1}{date}\PYG{l+s+s1}{\PYGZsq{}}
\PYG{n}{price\PYGZus{}data}\PYG{o}{.}\PYG{n}{to\PYGZus{}pickle}\PYG{p}{(}\PYG{l+s+s1}{\PYGZsq{}}\PYG{l+s+s1}{stock\PYGZus{}data\PYGZus{}from\PYGZus{}naver.pkl}\PYG{l+s+s1}{\PYGZsq{}}\PYG{p}{)}
\end{sphinxVerbatim}

\end{sphinxuseclass}\end{sphinxVerbatimInput}

\end{sphinxuseclass}
\sphinxAtStartPar
 저장한 pickle 파일을 다시 읽어 첫 5 라인을 head 메소드로 찍어보면 아래와 같습니다. 여기서 date가 인덱스로 처리되어 있다는 것을 기억해주시면 좋습니다.

\begin{sphinxuseclass}{cell}\begin{sphinxVerbatimInput}

\begin{sphinxuseclass}{cell_input}
\begin{sphinxVerbatim}[commandchars=\\\{\}]
\PYG{n}{price\PYGZus{}data} \PYG{o}{=} \PYG{n}{pd}\PYG{o}{.}\PYG{n}{read\PYGZus{}pickle}\PYG{p}{(}\PYG{l+s+s1}{\PYGZsq{}}\PYG{l+s+s1}{stock\PYGZus{}data\PYGZus{}from\PYGZus{}naver.pkl}\PYG{l+s+s1}{\PYGZsq{}}\PYG{p}{)}
\PYG{n}{price\PYGZus{}data}\PYG{o}{.}\PYG{n}{head}\PYG{p}{(}\PYG{p}{)}\PYG{o}{.}\PYG{n}{style}\PYG{o}{.}\PYG{n}{set\PYGZus{}table\PYGZus{}attributes}\PYG{p}{(}\PYG{l+s+s1}{\PYGZsq{}}\PYG{l+s+s1}{style=}\PYG{l+s+s1}{\PYGZdq{}}\PYG{l+s+s1}{font\PYGZhy{}size: 12px}\PYG{l+s+s1}{\PYGZdq{}}\PYG{l+s+s1}{\PYGZsq{}}\PYG{p}{)}
\end{sphinxVerbatim}

\end{sphinxuseclass}\end{sphinxVerbatimInput}
\begin{sphinxVerbatimOutput}

\begin{sphinxuseclass}{cell_output}
\begin{sphinxVerbatim}[commandchars=\\\{\}]
\PYGZlt{}pandas.io.formats.style.Styler at 0x1c862384a90\PYGZgt{}
\end{sphinxVerbatim}

\end{sphinxuseclass}\end{sphinxVerbatimOutput}

\end{sphinxuseclass}
\sphinxAtStartPar
 몇 개의 종목이 있고, 각 종목별 일봉의 갯 수 가 몇 개인지 확인해 보겠습니다. 종목 수는 1422 개, 307 개의 일봉이 있습니다.

\begin{sphinxuseclass}{cell}\begin{sphinxVerbatimInput}

\begin{sphinxuseclass}{cell_input}
\begin{sphinxVerbatim}[commandchars=\\\{\}]
\PYG{n+nb}{print}\PYG{p}{(}\PYG{n}{price\PYGZus{}data}\PYG{p}{[}\PYG{l+s+s1}{\PYGZsq{}}\PYG{l+s+s1}{code}\PYG{l+s+s1}{\PYGZsq{}}\PYG{p}{]}\PYG{o}{.}\PYG{n}{nunique}\PYG{p}{(}\PYG{p}{)}\PYG{p}{)}
\PYG{n+nb}{print}\PYG{p}{(}\PYG{n}{price\PYGZus{}data}\PYG{o}{.}\PYG{n}{groupby}\PYG{p}{(}\PYG{l+s+s1}{\PYGZsq{}}\PYG{l+s+s1}{code}\PYG{l+s+s1}{\PYGZsq{}}\PYG{p}{)}\PYG{p}{[}\PYG{l+s+s1}{\PYGZsq{}}\PYG{l+s+s1}{close}\PYG{l+s+s1}{\PYGZsq{}}\PYG{p}{]}\PYG{o}{.}\PYG{n}{count}\PYG{p}{(}\PYG{p}{)}\PYG{o}{.}\PYG{n}{agg}\PYG{p}{(}\PYG{p}{[}\PYG{l+s+s1}{\PYGZsq{}}\PYG{l+s+s1}{min}\PYG{l+s+s1}{\PYGZsq{}}\PYG{p}{,}\PYG{l+s+s1}{\PYGZsq{}}\PYG{l+s+s1}{max}\PYG{l+s+s1}{\PYGZsq{}}\PYG{p}{]}\PYG{p}{)}\PYG{p}{)}
\end{sphinxVerbatim}

\end{sphinxuseclass}\end{sphinxVerbatimInput}
\begin{sphinxVerbatimOutput}

\begin{sphinxuseclass}{cell_output}
\begin{sphinxVerbatim}[commandchars=\\\{\}]
1417
min    307
max    307
Name: close, dtype: int64
\end{sphinxVerbatim}

\end{sphinxuseclass}\end{sphinxVerbatimOutput}

\end{sphinxuseclass}
\begin{sphinxuseclass}{cell}\begin{sphinxVerbatimInput}

\begin{sphinxuseclass}{cell_input}
\begin{sphinxVerbatim}[commandchars=\\\{\}]
\PYG{k+kn}{import} \PYG{n+nn}{FinanceDataReader} \PYG{k}{as} \PYG{n+nn}{fdr}
\PYG{k+kn}{import} \PYG{n+nn}{pandas} \PYG{k}{as} \PYG{n+nn}{pd}
\PYG{k+kn}{import} \PYG{n+nn}{matplotlib}\PYG{n+nn}{.}\PYG{n+nn}{pyplot} \PYG{k}{as} \PYG{n+nn}{plt}
\PYG{o}{\PYGZpc{}}\PYG{k}{matplotlib} inline
\PYG{n}{pd}\PYG{o}{.}\PYG{n}{options}\PYG{o}{.}\PYG{n}{display}\PYG{o}{.}\PYG{n}{float\PYGZus{}format} \PYG{o}{=} \PYG{l+s+s1}{\PYGZsq{}}\PYG{l+s+si}{\PYGZob{}:,.3f\PYGZcb{}}\PYG{l+s+s1}{\PYGZsq{}}\PYG{o}{.}\PYG{n}{format}
\end{sphinxVerbatim}

\end{sphinxuseclass}\end{sphinxVerbatimInput}

\end{sphinxuseclass}

\section{코스닥 인덱스 데이터}
\label{\detokenize{chapter4/4.4.2_Data_Collection:id1}}\label{\detokenize{chapter4/4.4.2_Data_Collection::doc}}
\sphinxAtStartPar
코스닥 인덱스 데이터는 FinanceDataReader 로 데이터를 수집해 보겠습니다.
사용법에 대한 설명은 아래 링크에 자세하게 되어 있습니다.
\sphinxurl{https://financedata.github.io/posts/finance-data-reader-users-guide.html}

\sphinxAtStartPar
FinanceDataReader 는 국내 주식 데이터 뿐만 아니라 해외 데이터도 수집이 가능합니다. 환율, 암호화폐 등의 데이터도 제공됩니다.
이성준님이 개발해서 무료로 제공해 주시는 파이썬 라이브러리입니다. 금융데이터를 쉽게 수집할 수 있게 해 주신 이성준님께 다시 한 번 깊은 감사를 드립니다.

\sphinxAtStartPar
FinanceDataReader 를 fdr 이름으로 import 하시고, fdr.DataReader 함수에서 KQ11 를 호출하시면 결과값을 얻을 수 있습니다.
fdr.DataReader(‘KQ11’, ‘2021’) 에서 ‘KQ11’ 는 코스닥 지수 종목을 의미하고 ‘2021’ 은 2021 년부터 데이터를 가져오라는 뜻 입니다.

\sphinxAtStartPar
Pandas 에서 제공하는 plot 를 이용하여 2021년 부터 코스닥 지수 시계열 데이터를 그려보았습니다.

\begin{sphinxuseclass}{cell}\begin{sphinxVerbatimInput}

\begin{sphinxuseclass}{cell_input}
\begin{sphinxVerbatim}[commandchars=\\\{\}]
\PYG{n}{kosdaq\PYGZus{}index} \PYG{o}{=} \PYG{n}{fdr}\PYG{o}{.}\PYG{n}{DataReader}\PYG{p}{(}\PYG{l+s+s1}{\PYGZsq{}}\PYG{l+s+s1}{KQ11}\PYG{l+s+s1}{\PYGZsq{}}\PYG{p}{,} \PYG{l+s+s1}{\PYGZsq{}}\PYG{l+s+s1}{2021}\PYG{l+s+s1}{\PYGZsq{}}\PYG{p}{)} \PYG{c+c1}{\PYGZsh{} 데이터 호출}
\PYG{n}{kosdaq\PYGZus{}index}\PYG{o}{.}\PYG{n}{columns} \PYG{o}{=} \PYG{p}{[}\PYG{l+s+s1}{\PYGZsq{}}\PYG{l+s+s1}{close}\PYG{l+s+s1}{\PYGZsq{}}\PYG{p}{,}\PYG{l+s+s1}{\PYGZsq{}}\PYG{l+s+s1}{open}\PYG{l+s+s1}{\PYGZsq{}}\PYG{p}{,}\PYG{l+s+s1}{\PYGZsq{}}\PYG{l+s+s1}{high}\PYG{l+s+s1}{\PYGZsq{}}\PYG{p}{,}\PYG{l+s+s1}{\PYGZsq{}}\PYG{l+s+s1}{low}\PYG{l+s+s1}{\PYGZsq{}}\PYG{p}{,}\PYG{l+s+s1}{\PYGZsq{}}\PYG{l+s+s1}{volume}\PYG{l+s+s1}{\PYGZsq{}}\PYG{p}{,}\PYG{l+s+s1}{\PYGZsq{}}\PYG{l+s+s1}{change}\PYG{l+s+s1}{\PYGZsq{}}\PYG{p}{]} \PYG{c+c1}{\PYGZsh{} 컬럼명 변경}
\PYG{n}{kosdaq\PYGZus{}index}\PYG{o}{.}\PYG{n}{index}\PYG{o}{.}\PYG{n}{name}\PYG{o}{=}\PYG{l+s+s1}{\PYGZsq{}}\PYG{l+s+s1}{date}\PYG{l+s+s1}{\PYGZsq{}} \PYG{c+c1}{\PYGZsh{} 인덱스 이름 생성}
\PYG{n}{kosdaq\PYGZus{}index}\PYG{o}{.}\PYG{n}{sort\PYGZus{}index}\PYG{p}{(}\PYG{n}{inplace}\PYG{o}{=}\PYG{k+kc}{True}\PYG{p}{)} \PYG{c+c1}{\PYGZsh{} 인덱스(날짜) 로 정렬 }
\PYG{n}{kosdaq\PYGZus{}index}\PYG{p}{[}\PYG{l+s+s1}{\PYGZsq{}}\PYG{l+s+s1}{kosdaq\PYGZus{}return}\PYG{l+s+s1}{\PYGZsq{}}\PYG{p}{]} \PYG{o}{=} \PYG{n}{kosdaq\PYGZus{}index}\PYG{p}{[}\PYG{l+s+s1}{\PYGZsq{}}\PYG{l+s+s1}{close}\PYG{l+s+s1}{\PYGZsq{}}\PYG{p}{]}\PYG{o}{/}\PYG{n}{kosdaq\PYGZus{}index}\PYG{p}{[}\PYG{l+s+s1}{\PYGZsq{}}\PYG{l+s+s1}{close}\PYG{l+s+s1}{\PYGZsq{}}\PYG{p}{]}\PYG{o}{.}\PYG{n}{shift}\PYG{p}{(}\PYG{l+m+mi}{1}\PYG{p}{)} \PYG{c+c1}{\PYGZsh{} 수익율 : 전 날 종가대비 당일 종가}
\PYG{n}{kosdaq\PYGZus{}index}\PYG{o}{.}\PYG{n}{to\PYGZus{}pickle}\PYG{p}{(}\PYG{l+s+s1}{\PYGZsq{}}\PYG{l+s+s1}{kosdaq\PYGZus{}index.pkl}\PYG{l+s+s1}{\PYGZsq{}}\PYG{p}{)} \PYG{c+c1}{\PYGZsh{} 피클로 저장}

\PYG{c+c1}{\PYGZsh{} 차트 생성}
\PYG{n}{kosdaq\PYGZus{}index}\PYG{p}{[}\PYG{l+s+s1}{\PYGZsq{}}\PYG{l+s+s1}{close}\PYG{l+s+s1}{\PYGZsq{}}\PYG{p}{]}\PYG{o}{.}\PYG{n}{plot}\PYG{p}{(}\PYG{n}{figsize}\PYG{o}{=}\PYG{p}{(}\PYG{l+m+mi}{20}\PYG{p}{,}\PYG{l+m+mi}{5}\PYG{p}{)}\PYG{p}{)}
\PYG{n}{plt}\PYG{o}{.}\PYG{n}{title}\PYG{p}{(}\PYG{l+s+s1}{\PYGZsq{}}\PYG{l+s+s1}{KOSDAQ Index}\PYG{l+s+s1}{\PYGZsq{}}\PYG{p}{)}
\end{sphinxVerbatim}

\end{sphinxuseclass}\end{sphinxVerbatimInput}
\begin{sphinxVerbatimOutput}

\begin{sphinxuseclass}{cell_output}
\begin{sphinxVerbatim}[commandchars=\\\{\}]
Text(0.5, 1.0, \PYGZsq{}KOSDAQ Index\PYGZsq{})
\end{sphinxVerbatim}

\noindent\sphinxincludegraphics{{4.4.2_Data_Collection_2_1}.png}

\end{sphinxuseclass}\end{sphinxVerbatimOutput}

\end{sphinxuseclass}
\sphinxAtStartPar
일별 수익율 그래프도 함 그려보겠습니다. 2021년 3월부터 2021년 8월까지는 수익율의 변동성이 비교적 적어보입니다.

\begin{sphinxuseclass}{cell}\begin{sphinxVerbatimInput}

\begin{sphinxuseclass}{cell_input}
\begin{sphinxVerbatim}[commandchars=\\\{\}]
\PYG{c+c1}{\PYGZsh{} 차트 생성}
\PYG{n}{kosdaq\PYGZus{}index}\PYG{p}{[}\PYG{l+s+s1}{\PYGZsq{}}\PYG{l+s+s1}{kosdaq\PYGZus{}return}\PYG{l+s+s1}{\PYGZsq{}}\PYG{p}{]}\PYG{o}{.}\PYG{n}{plot}\PYG{p}{(}\PYG{n}{figsize}\PYG{o}{=}\PYG{p}{(}\PYG{l+m+mi}{20}\PYG{p}{,}\PYG{l+m+mi}{5}\PYG{p}{)}\PYG{p}{,} \PYG{n}{color}\PYG{o}{=}\PYG{l+s+s1}{\PYGZsq{}}\PYG{l+s+s1}{orangered}\PYG{l+s+s1}{\PYGZsq{}}\PYG{p}{,} \PYG{n}{style}\PYG{o}{=}\PYG{l+s+s1}{\PYGZsq{}}\PYG{l+s+s1}{\PYGZhy{}\PYGZhy{}}\PYG{l+s+s1}{\PYGZsq{}}\PYG{p}{)}
\PYG{n}{plt}\PYG{o}{.}\PYG{n}{title}\PYG{p}{(}\PYG{l+s+s1}{\PYGZsq{}}\PYG{l+s+s1}{KOSDAQ Index Daily Return}\PYG{l+s+s1}{\PYGZsq{}}\PYG{p}{)}
\end{sphinxVerbatim}

\end{sphinxuseclass}\end{sphinxVerbatimInput}
\begin{sphinxVerbatimOutput}

\begin{sphinxuseclass}{cell_output}
\begin{sphinxVerbatim}[commandchars=\\\{\}]
Text(0.5, 1.0, \PYGZsq{}KOSDAQ Index Daily Return\PYGZsq{})
\end{sphinxVerbatim}

\noindent\sphinxincludegraphics{{4.4.2_Data_Collection_4_1}.png}

\end{sphinxuseclass}\end{sphinxVerbatimOutput}

\end{sphinxuseclass}
\sphinxAtStartPar
저장된 Pickle 파일을 읽어서 첫 5 행 출력해 봅니다.

\begin{sphinxuseclass}{cell}\begin{sphinxVerbatimInput}

\begin{sphinxuseclass}{cell_input}
\begin{sphinxVerbatim}[commandchars=\\\{\}]
\PYG{n}{kosdaq\PYGZus{}index} \PYG{o}{=} \PYG{n}{pd}\PYG{o}{.}\PYG{n}{read\PYGZus{}pickle}\PYG{p}{(}\PYG{l+s+s1}{\PYGZsq{}}\PYG{l+s+s1}{kosdaq\PYGZus{}index.pkl}\PYG{l+s+s1}{\PYGZsq{}}\PYG{p}{)} 
\PYG{n}{kosdaq\PYGZus{}index}\PYG{o}{.}\PYG{n}{head}\PYG{p}{(}\PYG{p}{)}\PYG{o}{.}\PYG{n}{style}\PYG{o}{.}\PYG{n}{set\PYGZus{}table\PYGZus{}attributes}\PYG{p}{(}\PYG{l+s+s1}{\PYGZsq{}}\PYG{l+s+s1}{style=}\PYG{l+s+s1}{\PYGZdq{}}\PYG{l+s+s1}{font\PYGZhy{}size: 12px}\PYG{l+s+s1}{\PYGZdq{}}\PYG{l+s+s1}{\PYGZsq{}}\PYG{p}{)}
\end{sphinxVerbatim}

\end{sphinxuseclass}\end{sphinxVerbatimInput}
\begin{sphinxVerbatimOutput}

\begin{sphinxuseclass}{cell_output}
\begin{sphinxVerbatim}[commandchars=\\\{\}]
\PYGZlt{}pandas.io.formats.style.Styler at 0x1de8d29b850\PYGZgt{}
\end{sphinxVerbatim}

\end{sphinxuseclass}\end{sphinxVerbatimOutput}

\end{sphinxuseclass}
\begin{sphinxuseclass}{cell}\begin{sphinxVerbatimInput}

\begin{sphinxuseclass}{cell_input}
\begin{sphinxVerbatim}[commandchars=\\\{\}]
\PYG{k+kn}{import} \PYG{n+nn}{FinanceDataReader} \PYG{k}{as} \PYG{n+nn}{fdr}
\PYG{o}{\PYGZpc{}}\PYG{k}{matplotlib} inline
\PYG{k+kn}{import} \PYG{n+nn}{matplotlib}\PYG{n+nn}{.}\PYG{n+nn}{pyplot} \PYG{k}{as} \PYG{n+nn}{plt}
\PYG{k+kn}{import} \PYG{n+nn}{pandas} \PYG{k}{as} \PYG{n+nn}{pd}
\PYG{k+kn}{import} \PYG{n+nn}{numpy} \PYG{k}{as} \PYG{n+nn}{np}
\PYG{n}{pd}\PYG{o}{.}\PYG{n}{options}\PYG{o}{.}\PYG{n}{display}\PYG{o}{.}\PYG{n}{float\PYGZus{}format} \PYG{o}{=} \PYG{l+s+s1}{\PYGZsq{}}\PYG{l+s+si}{\PYGZob{}:,.3f\PYGZcb{}}\PYG{l+s+s1}{\PYGZsq{}}\PYG{o}{.}\PYG{n}{format}
\end{sphinxVerbatim}

\end{sphinxuseclass}\end{sphinxVerbatimInput}

\end{sphinxuseclass}

\section{종목별 일봉 데이터와 코스피 지수 데이터와 결합}
\label{\detokenize{chapter4/4.4.3_Data_Collection:id1}}\label{\detokenize{chapter4/4.4.3_Data_Collection::doc}}
\sphinxAtStartPar
앞에서 저장한 종목 리스트, 코스닥 종목별 주가 데이터와 지수 데이터를 읽습니다. 인덱스(날짜) 의 최소값과 최대값을 확인해 봅니다.

\begin{sphinxuseclass}{cell}\begin{sphinxVerbatimInput}

\begin{sphinxuseclass}{cell_input}
\begin{sphinxVerbatim}[commandchars=\\\{\}]
\PYG{n}{price\PYGZus{}data} \PYG{o}{=} \PYG{n}{pd}\PYG{o}{.}\PYG{n}{read\PYGZus{}pickle}\PYG{p}{(}\PYG{l+s+s1}{\PYGZsq{}}\PYG{l+s+s1}{stock\PYGZus{}data\PYGZus{}from\PYGZus{}fdr.pkl}\PYG{l+s+s1}{\PYGZsq{}}\PYG{p}{)} \PYG{c+c1}{\PYGZsh{} 주가 정보}
\PYG{n}{kosdaq\PYGZus{}index} \PYG{o}{=} \PYG{n}{pd}\PYG{o}{.}\PYG{n}{read\PYGZus{}pickle}\PYG{p}{(}\PYG{l+s+s1}{\PYGZsq{}}\PYG{l+s+s1}{kosdaq\PYGZus{}index.pkl}\PYG{l+s+s1}{\PYGZsq{}}\PYG{p}{)} \PYG{c+c1}{\PYGZsh{} 지수 정보}
\PYG{n}{kosdaq\PYGZus{}list} \PYG{o}{=} \PYG{n}{pd}\PYG{o}{.}\PYG{n}{read\PYGZus{}pickle}\PYG{p}{(}\PYG{l+s+s1}{\PYGZsq{}}\PYG{l+s+s1}{kosdaq\PYGZus{}list.pkl}\PYG{l+s+s1}{\PYGZsq{}}\PYG{p}{)} \PYG{c+c1}{\PYGZsh{} 종목 정보}

\PYG{n+nb}{print}\PYG{p}{(}\PYG{n}{price\PYGZus{}data}\PYG{o}{.}\PYG{n}{index}\PYG{o}{.}\PYG{n}{min}\PYG{p}{(}\PYG{p}{)}\PYG{p}{,} \PYG{n}{price\PYGZus{}data}\PYG{o}{.}\PYG{n}{index}\PYG{o}{.}\PYG{n}{max}\PYG{p}{(}\PYG{p}{)}\PYG{p}{)}
\PYG{n+nb}{print}\PYG{p}{(}\PYG{n}{kosdaq\PYGZus{}index}\PYG{o}{.}\PYG{n}{index}\PYG{o}{.}\PYG{n}{min}\PYG{p}{(}\PYG{p}{)}\PYG{p}{,} \PYG{n}{kosdaq\PYGZus{}index}\PYG{o}{.}\PYG{n}{index}\PYG{o}{.}\PYG{n}{max}\PYG{p}{(}\PYG{p}{)}\PYG{p}{)}
\end{sphinxVerbatim}

\end{sphinxuseclass}\end{sphinxVerbatimInput}
\begin{sphinxVerbatimOutput}

\begin{sphinxuseclass}{cell_output}
\begin{sphinxVerbatim}[commandchars=\\\{\}]
2021\PYGZhy{}01\PYGZhy{}04 00:00:00 2022\PYGZhy{}03\PYGZhy{}31 00:00:00
2021\PYGZhy{}01\PYGZhy{}04 00:00:00 2022\PYGZhy{}06\PYGZhy{}24 00:00:00
\end{sphinxVerbatim}

\end{sphinxuseclass}\end{sphinxVerbatimOutput}

\end{sphinxuseclass}
\sphinxAtStartPar
 나중에 검정할 가설 중 하나가 “주가가 상승할 확률이 높은 종목은 마켓이 안 좋을 때(즉 지표가 빠질 때) 수익율이 좋았다” 입니다. 이 가설을 검증하기 위해 두 데이타셋을 병합합니다. 두 데이터를 종목별 날짜별로 병합을 해야 ‘종목 수익율’과 ‘코스닥 지수 수익율’을 비교할 수 있습니다.

\sphinxAtStartPar
price\_data 를 기준으로 kosdaq\_index 데이터의 지수 수익율을 추가합니다. price\_data 에 날짜를 Index 로 left merge 를 하면 주가지수 정보를 추가할 수 있습니다.

\begin{sphinxuseclass}{cell}\begin{sphinxVerbatimInput}

\begin{sphinxuseclass}{cell_input}
\begin{sphinxVerbatim}[commandchars=\\\{\}]
\PYG{n}{merged} \PYG{o}{=} \PYG{n}{price\PYGZus{}data}\PYG{o}{.}\PYG{n}{merge}\PYG{p}{(}\PYG{n}{kosdaq\PYGZus{}index}\PYG{p}{[}\PYG{l+s+s1}{\PYGZsq{}}\PYG{l+s+s1}{kosdaq\PYGZus{}return}\PYG{l+s+s1}{\PYGZsq{}}\PYG{p}{]}\PYG{p}{,} \PYG{n}{left\PYGZus{}index}\PYG{o}{=}\PYG{k+kc}{True}\PYG{p}{,} \PYG{n}{right\PYGZus{}index}\PYG{o}{=}\PYG{k+kc}{True}\PYG{p}{,} \PYG{n}{how}\PYG{o}{=}\PYG{l+s+s1}{\PYGZsq{}}\PYG{l+s+s1}{left}\PYG{l+s+s1}{\PYGZsq{}}\PYG{p}{)}
\PYG{n}{merged}\PYG{o}{.}\PYG{n}{head}\PYG{p}{(}\PYG{p}{)}\PYG{o}{.}\PYG{n}{style}\PYG{o}{.}\PYG{n}{set\PYGZus{}table\PYGZus{}attributes}\PYG{p}{(}\PYG{l+s+s1}{\PYGZsq{}}\PYG{l+s+s1}{style=}\PYG{l+s+s1}{\PYGZdq{}}\PYG{l+s+s1}{font\PYGZhy{}size: 12px}\PYG{l+s+s1}{\PYGZdq{}}\PYG{l+s+s1}{\PYGZsq{}}\PYG{p}{)}
\end{sphinxVerbatim}

\end{sphinxuseclass}\end{sphinxVerbatimInput}
\begin{sphinxVerbatimOutput}

\begin{sphinxuseclass}{cell_output}
\begin{sphinxVerbatim}[commandchars=\\\{\}]
\PYGZlt{}pandas.io.formats.style.Styler at 0x25487cdbf40\PYGZgt{}
\end{sphinxVerbatim}

\end{sphinxuseclass}\end{sphinxVerbatimOutput}

\end{sphinxuseclass}
\sphinxAtStartPar
가설 검정을 위해 미리 컬럼을 생성합니다. 코스닥 지수 수익율이 1 보다 적을 때, 종목의 수익율이 1 보다 크면 1, 아니면 0 을 생성합니다. 그 값을 win\_market 이라는 새로운 컬럼에 저장합니다. 아래오와 같이 np.where 구문을 사용했는데요.

\begin{sphinxVerbatim}[commandchars=\\\{\}]
\PYG{n}{stock\PYGZus{}return}\PYG{p}{[}\PYG{l+s+s1}{\PYGZsq{}}\PYG{l+s+s1}{win\PYGZus{}market}\PYG{l+s+s1}{\PYGZsq{}}\PYG{p}{]} \PYG{o}{=} \PYG{n}{np}\PYG{o}{.}\PYG{n}{where}\PYG{p}{(}\PYG{p}{(}\PYG{n}{c1}\PYG{o}{\PYGZam{}}\PYG{n}{c2}\PYG{p}{)}\PYG{p}{,} \PYG{l+m+mi}{1}\PYG{p}{,} \PYG{l+m+mi}{0}\PYG{p}{)}
\end{sphinxVerbatim}

\sphinxAtStartPar
이 메소드는 np.where(조건, 조건이 참일 때 값, 조건이 거짓일 때 값)와 같이 처리를 합니다.

\begin{sphinxuseclass}{cell}\begin{sphinxVerbatimInput}

\begin{sphinxuseclass}{cell_input}
\begin{sphinxVerbatim}[commandchars=\\\{\}]
\PYG{n}{return\PYGZus{}all} \PYG{o}{=} \PYG{n}{pd}\PYG{o}{.}\PYG{n}{DataFrame}\PYG{p}{(}\PYG{p}{)}

\PYG{k}{for} \PYG{n}{code} \PYG{o+ow}{in} \PYG{n}{kosdaq\PYGZus{}list}\PYG{p}{[}\PYG{l+s+s1}{\PYGZsq{}}\PYG{l+s+s1}{code}\PYG{l+s+s1}{\PYGZsq{}}\PYG{p}{]}\PYG{p}{:}  
    
    \PYG{n}{stock\PYGZus{}return} \PYG{o}{=} \PYG{n}{merged}\PYG{p}{[}\PYG{n}{merged}\PYG{p}{[}\PYG{l+s+s1}{\PYGZsq{}}\PYG{l+s+s1}{code}\PYG{l+s+s1}{\PYGZsq{}}\PYG{p}{]}\PYG{o}{==}\PYG{n}{code}\PYG{p}{]}\PYG{o}{.}\PYG{n}{sort\PYGZus{}index}\PYG{p}{(}\PYG{p}{)}
    \PYG{n}{stock\PYGZus{}return}\PYG{p}{[}\PYG{l+s+s1}{\PYGZsq{}}\PYG{l+s+s1}{return}\PYG{l+s+s1}{\PYGZsq{}}\PYG{p}{]} \PYG{o}{=} \PYG{n}{stock\PYGZus{}return}\PYG{p}{[}\PYG{l+s+s1}{\PYGZsq{}}\PYG{l+s+s1}{close}\PYG{l+s+s1}{\PYGZsq{}}\PYG{p}{]}\PYG{o}{/}\PYG{n}{stock\PYGZus{}return}\PYG{p}{[}\PYG{l+s+s1}{\PYGZsq{}}\PYG{l+s+s1}{close}\PYG{l+s+s1}{\PYGZsq{}}\PYG{p}{]}\PYG{o}{.}\PYG{n}{shift}\PYG{p}{(}\PYG{l+m+mi}{1}\PYG{p}{)} \PYG{c+c1}{\PYGZsh{} 종목별 전일 종가 대비 당일 종가 수익율}
    \PYG{n}{c1} \PYG{o}{=} \PYG{p}{(}\PYG{n}{stock\PYGZus{}return}\PYG{p}{[}\PYG{l+s+s1}{\PYGZsq{}}\PYG{l+s+s1}{kosdaq\PYGZus{}return}\PYG{l+s+s1}{\PYGZsq{}}\PYG{p}{]} \PYG{o}{\PYGZlt{}} \PYG{l+m+mi}{1}\PYG{p}{)} \PYG{c+c1}{\PYGZsh{} 수익율 1 보다 작음. 당일 종가가 전일 종가보다 낮음 (코스닥 지표)}
    \PYG{n}{c2} \PYG{o}{=} \PYG{p}{(}\PYG{n}{stock\PYGZus{}return}\PYG{p}{[}\PYG{l+s+s1}{\PYGZsq{}}\PYG{l+s+s1}{return}\PYG{l+s+s1}{\PYGZsq{}}\PYG{p}{]} \PYG{o}{\PYGZgt{}} \PYG{l+m+mi}{1}\PYG{p}{)} \PYG{c+c1}{\PYGZsh{} 수익율 1 보다 큼. 당일 종가가 전일 종가보다 큼 (개별 종목)}
    \PYG{n}{stock\PYGZus{}return}\PYG{p}{[}\PYG{l+s+s1}{\PYGZsq{}}\PYG{l+s+s1}{win\PYGZus{}market}\PYG{l+s+s1}{\PYGZsq{}}\PYG{p}{]} \PYG{o}{=} \PYG{n}{np}\PYG{o}{.}\PYG{n}{where}\PYG{p}{(}\PYG{p}{(}\PYG{n}{c1}\PYG{o}{\PYGZam{}}\PYG{n}{c2}\PYG{p}{)}\PYG{p}{,} \PYG{l+m+mi}{1}\PYG{p}{,} \PYG{l+m+mi}{0}\PYG{p}{)} \PYG{c+c1}{\PYGZsh{} C1 과 C2 조건을 동시에 만족하면 1, 아니면 0}
    \PYG{n}{return\PYGZus{}all} \PYG{o}{=} \PYG{n}{pd}\PYG{o}{.}\PYG{n}{concat}\PYG{p}{(}\PYG{p}{[}\PYG{n}{return\PYGZus{}all}\PYG{p}{,} \PYG{n}{stock\PYGZus{}return}\PYG{p}{]}\PYG{p}{,} \PYG{n}{axis}\PYG{o}{=}\PYG{l+m+mi}{0}\PYG{p}{)} 
    
\PYG{n}{return\PYGZus{}all}\PYG{o}{.}\PYG{n}{to\PYGZus{}pickle}\PYG{p}{(}\PYG{l+s+s1}{\PYGZsq{}}\PYG{l+s+s1}{return\PYGZus{}all.pkl}\PYG{l+s+s1}{\PYGZsq{}}\PYG{p}{)}       
\end{sphinxVerbatim}

\end{sphinxuseclass}\end{sphinxVerbatimInput}

\end{sphinxuseclass}
\sphinxAtStartPar
 값이 잘 들어갔는 지 head 메소드로 첫 번째 행 5 개를 출력해 봅니다.

\begin{sphinxuseclass}{cell}\begin{sphinxVerbatimInput}

\begin{sphinxuseclass}{cell_input}
\begin{sphinxVerbatim}[commandchars=\\\{\}]
\PYG{n}{return\PYGZus{}all} \PYG{o}{=} \PYG{n}{pd}\PYG{o}{.}\PYG{n}{read\PYGZus{}pickle}\PYG{p}{(}\PYG{l+s+s1}{\PYGZsq{}}\PYG{l+s+s1}{return\PYGZus{}all.pkl}\PYG{l+s+s1}{\PYGZsq{}}\PYG{p}{)}  
\PYG{n}{return\PYGZus{}all}\PYG{o}{.}\PYG{n}{head}\PYG{p}{(}\PYG{p}{)}\PYG{o}{.}\PYG{n}{style}\PYG{o}{.}\PYG{n}{set\PYGZus{}table\PYGZus{}attributes}\PYG{p}{(}\PYG{l+s+s1}{\PYGZsq{}}\PYG{l+s+s1}{style=}\PYG{l+s+s1}{\PYGZdq{}}\PYG{l+s+s1}{font\PYGZhy{}size: 12px}\PYG{l+s+s1}{\PYGZdq{}}\PYG{l+s+s1}{\PYGZsq{}}\PYG{p}{)}
\end{sphinxVerbatim}

\end{sphinxuseclass}\end{sphinxVerbatimInput}
\begin{sphinxVerbatimOutput}

\begin{sphinxuseclass}{cell_output}
\begin{sphinxVerbatim}[commandchars=\\\{\}]
\PYGZlt{}pandas.io.formats.style.Styler at 0x25489ff2850\PYGZgt{}
\end{sphinxVerbatim}

\end{sphinxuseclass}\end{sphinxVerbatimOutput}

\end{sphinxuseclass}
\sphinxAtStartPar
가설 검정 시 자세히 다루겠지만, win\_market 의 비율과 종목별 수익율과의 관계를 간단하게 조사하겠습니다. 이번에 scatter plot 를 함 그려보겠습니다. Scatter plot 에는 x 축의 값과 y 축의 값을 인수로 넣어주면 됩니다. 그래프를 보니 두 값 사이에 상관성이 높아 보입니다.

\begin{sphinxuseclass}{cell}\begin{sphinxVerbatimInput}

\begin{sphinxuseclass}{cell_input}
\begin{sphinxVerbatim}[commandchars=\\\{\}]
\PYG{n}{plt}\PYG{o}{.}\PYG{n}{figure}\PYG{p}{(}\PYG{n}{figsize}\PYG{o}{=}\PYG{p}{(}\PYG{l+m+mi}{10}\PYG{p}{,}\PYG{l+m+mi}{6}\PYG{p}{)}\PYG{p}{)}
\PYG{n}{x} \PYG{o}{=} \PYG{n}{return\PYGZus{}all}\PYG{o}{.}\PYG{n}{groupby}\PYG{p}{(}\PYG{l+s+s1}{\PYGZsq{}}\PYG{l+s+s1}{code}\PYG{l+s+s1}{\PYGZsq{}}\PYG{p}{)}\PYG{p}{[}\PYG{l+s+s1}{\PYGZsq{}}\PYG{l+s+s1}{win\PYGZus{}market}\PYG{l+s+s1}{\PYGZsq{}}\PYG{p}{]}\PYG{o}{.}\PYG{n}{mean}\PYG{p}{(}\PYG{p}{)} \PYG{c+c1}{\PYGZsh{} 종목별 win\PYGZus{}market의 비율}
\PYG{n}{y} \PYG{o}{=} \PYG{n}{return\PYGZus{}all}\PYG{o}{.}\PYG{n}{groupby}\PYG{p}{(}\PYG{l+s+s1}{\PYGZsq{}}\PYG{l+s+s1}{code}\PYG{l+s+s1}{\PYGZsq{}}\PYG{p}{)}\PYG{p}{[}\PYG{l+s+s1}{\PYGZsq{}}\PYG{l+s+s1}{return}\PYG{l+s+s1}{\PYGZsq{}}\PYG{p}{]}\PYG{o}{.}\PYG{n}{mean}\PYG{p}{(}\PYG{p}{)} \PYG{c+c1}{\PYGZsh{} 종목별 평균 수익율 }
\PYG{n}{plt}\PYG{o}{.}\PYG{n}{scatter}\PYG{p}{(}\PYG{n}{x} \PYG{o}{=} \PYG{n}{x} \PYG{p}{,} \PYG{n}{y}\PYG{o}{=} \PYG{n}{y}\PYG{p}{,} \PYG{n}{s}\PYG{o}{=}\PYG{l+m+mi}{1}\PYG{p}{,} \PYG{n}{color}\PYG{o}{=}\PYG{l+s+s1}{\PYGZsq{}}\PYG{l+s+s1}{red}\PYG{l+s+s1}{\PYGZsq{}}\PYG{p}{)}
\PYG{n}{plt}\PYG{o}{.}\PYG{n}{xlabel}\PYG{p}{(}\PYG{l+s+s1}{\PYGZsq{}}\PYG{l+s+si}{\PYGZpc{} o}\PYG{l+s+s1}{f Win Market}\PYG{l+s+s1}{\PYGZsq{}}\PYG{p}{)}
\PYG{n}{plt}\PYG{o}{.}\PYG{n}{ylabel}\PYG{p}{(}\PYG{l+s+s1}{\PYGZsq{}}\PYG{l+s+s1}{Avg. Return}\PYG{l+s+s1}{\PYGZsq{}}\PYG{p}{)}
\end{sphinxVerbatim}

\end{sphinxuseclass}\end{sphinxVerbatimInput}
\begin{sphinxVerbatimOutput}

\begin{sphinxuseclass}{cell_output}
\begin{sphinxVerbatim}[commandchars=\\\{\}]
Text(0, 0.5, \PYGZsq{}Avg. Return\PYGZsq{})
\end{sphinxVerbatim}

\noindent\sphinxincludegraphics{{4.4.3_Data_Collection_10_1}.png}

\end{sphinxuseclass}\end{sphinxVerbatimOutput}

\end{sphinxuseclass}
\begin{sphinxuseclass}{cell}\begin{sphinxVerbatimInput}

\begin{sphinxuseclass}{cell_input}
\begin{sphinxVerbatim}[commandchars=\\\{\}]
\PYG{k+kn}{import} \PYG{n+nn}{FinanceDataReader} \PYG{k}{as} \PYG{n+nn}{fdr}
\PYG{o}{\PYGZpc{}}\PYG{k}{matplotlib} inline
\PYG{k+kn}{import} \PYG{n+nn}{matplotlib}\PYG{n+nn}{.}\PYG{n+nn}{pyplot} \PYG{k}{as} \PYG{n+nn}{plt}
\PYG{k+kn}{import} \PYG{n+nn}{pandas} \PYG{k}{as} \PYG{n+nn}{pd}
\PYG{k+kn}{import} \PYG{n+nn}{numpy} \PYG{k}{as} \PYG{n+nn}{np}
\PYG{k+kn}{import} \PYG{n+nn}{datetime}
\PYG{n}{pd}\PYG{o}{.}\PYG{n}{options}\PYG{o}{.}\PYG{n}{display}\PYG{o}{.}\PYG{n}{float\PYGZus{}format} \PYG{o}{=} \PYG{l+s+s1}{\PYGZsq{}}\PYG{l+s+si}{\PYGZob{}:,.2f\PYGZcb{}}\PYG{l+s+s1}{\PYGZsq{}}\PYG{o}{.}\PYG{n}{format}
\PYG{n}{pd}\PYG{o}{.}\PYG{n}{set\PYGZus{}option}\PYG{p}{(}\PYG{l+s+s1}{\PYGZsq{}}\PYG{l+s+s1}{display.expand\PYGZus{}frame\PYGZus{}repr}\PYG{l+s+s1}{\PYGZsq{}}\PYG{p}{,} \PYG{k+kc}{False}\PYG{p}{)}
\end{sphinxVerbatim}

\end{sphinxuseclass}\end{sphinxVerbatimInput}

\end{sphinxuseclass}

\section{가설 검증을 위한 데이터 처리}
\label{\detokenize{chapter4/4.4.4_Data_Processing:id1}}\label{\detokenize{chapter4/4.4.4_Data_Processing::doc}}
\sphinxAtStartPar
앞서 만든 return\_all (주가 데이터에 지수데이터가 추가된 파일) 을 아래와 같이 로드하고, Missing Data 는 제거합니다.

\begin{sphinxuseclass}{cell}\begin{sphinxVerbatimInput}

\begin{sphinxuseclass}{cell_input}
\begin{sphinxVerbatim}[commandchars=\\\{\}]
\PYG{n}{return\PYGZus{}all} \PYG{o}{=} \PYG{n}{pd}\PYG{o}{.}\PYG{n}{read\PYGZus{}pickle}\PYG{p}{(}\PYG{l+s+s1}{\PYGZsq{}}\PYG{l+s+s1}{return\PYGZus{}all.pkl}\PYG{l+s+s1}{\PYGZsq{}}\PYG{p}{)}\PYG{o}{.}\PYG{n}{dropna}\PYG{p}{(}\PYG{p}{)}  
\PYG{n}{return\PYGZus{}all}\PYG{o}{.}\PYG{n}{index} \PYG{o}{=} \PYG{p}{[}\PYG{n}{datetime}\PYG{o}{.}\PYG{n}{datetime}\PYG{o}{.}\PYG{n}{strftime}\PYG{p}{(}\PYG{n}{dt}\PYG{p}{,} \PYG{l+s+s1}{\PYGZsq{}}\PYG{l+s+s1}{\PYGZpc{}}\PYG{l+s+s1}{Y\PYGZhy{}}\PYG{l+s+s1}{\PYGZpc{}}\PYG{l+s+s1}{m\PYGZhy{}}\PYG{l+s+si}{\PYGZpc{}d}\PYG{l+s+s1}{\PYGZsq{}}\PYG{p}{)} \PYG{k}{for} \PYG{n}{dt} \PYG{o+ow}{in} \PYG{n}{return\PYGZus{}all}\PYG{o}{.}\PYG{n}{index}\PYG{p}{]}
\end{sphinxVerbatim}

\end{sphinxuseclass}\end{sphinxVerbatimInput}

\end{sphinxuseclass}
\begin{sphinxuseclass}{cell}\begin{sphinxVerbatimInput}

\begin{sphinxuseclass}{cell_input}
\begin{sphinxVerbatim}[commandchars=\\\{\}]
\PYG{n}{return\PYGZus{}all}\PYG{o}{.}\PYG{n}{head}\PYG{p}{(}\PYG{p}{)}\PYG{o}{.}\PYG{n}{style}\PYG{o}{.}\PYG{n}{set\PYGZus{}table\PYGZus{}attributes}\PYG{p}{(}\PYG{l+s+s1}{\PYGZsq{}}\PYG{l+s+s1}{style=}\PYG{l+s+s1}{\PYGZdq{}}\PYG{l+s+s1}{font\PYGZhy{}size: 12px}\PYG{l+s+s1}{\PYGZdq{}}\PYG{l+s+s1}{\PYGZsq{}}\PYG{p}{)}
\end{sphinxVerbatim}

\end{sphinxuseclass}\end{sphinxVerbatimInput}
\begin{sphinxVerbatimOutput}

\begin{sphinxuseclass}{cell_output}
\begin{sphinxVerbatim}[commandchars=\\\{\}]
\PYGZlt{}pandas.io.formats.style.Styler at 0x218fd39b5b0\PYGZgt{}
\end{sphinxVerbatim}

\end{sphinxuseclass}\end{sphinxVerbatimOutput}

\end{sphinxuseclass}
\sphinxAtStartPar
일주일(5영업일)을 수익율의 관찰 기간으로 하고, 관찰 기간 동안 주가 상승이 있으면 저희가 세운 가설들을 유의미한 가설로 판단하겠습니다. 여기서 주가 상승의 기준은  “종가 매수 일부터 다음 5 영업일 동안 최고 종가 수익율” 하겠습니다.

\sphinxAtStartPar
첫 번째 종목 060310 에 대하여 처리를 먼저 해 보겠습니다. df{[}‘close’{]} * shift(\sphinxhyphen{}1) 은 다음 영업일의 종가 수익율을 참조하고, df{[}‘close’{]}*shift(\sphinxhyphen{}2) 은 그 다음의 영업일의 종가 수익율을 참조합니다. 따라서 매수 후 2 영업일 후, 종가 수익율은 \{ df{[}‘close’{]} * shift(\sphinxhyphen{}1) \} * \{ df{[}‘close’{]} * shift(\sphinxhyphen{}2) \} 로 계산됩니다. 이렇게 1 영업일, 2 영업일, 3 영업일, 4 영업일, 5 영업일 후 종가 수익율을 새로운 컬럼에 생성하고, 그 중에서 가장 큰 수익율을 고르면 됩니다. 생성된 컬럼 중 가장 큰 값은 max(axis=1) 로 찾습니다. 참고로 max() 에서는 axis=0 이 Default 라서 axis=1 로 정해주지 않으면 열에서 가장 큰 값을 찾게 됩니다. 이 부분을 유의해 주세요.

\begin{sphinxuseclass}{cell}\begin{sphinxVerbatimInput}

\begin{sphinxuseclass}{cell_input}
\begin{sphinxVerbatim}[commandchars=\\\{\}]
\PYG{n}{s} \PYG{o}{=} \PYG{l+s+s1}{\PYGZsq{}}\PYG{l+s+s1}{060310}\PYG{l+s+s1}{\PYGZsq{}}
\PYG{n}{df} \PYG{o}{=} \PYG{n}{return\PYGZus{}all}\PYG{p}{[}\PYG{n}{return\PYGZus{}all}\PYG{p}{[}\PYG{l+s+s1}{\PYGZsq{}}\PYG{l+s+s1}{code}\PYG{l+s+s1}{\PYGZsq{}}\PYG{p}{]}\PYG{o}{==}\PYG{n}{s}\PYG{p}{]}\PYG{o}{.}\PYG{n}{sort\PYGZus{}index}\PYG{p}{(}\PYG{p}{)}\PYG{o}{.}\PYG{n}{copy}\PYG{p}{(}\PYG{p}{)}

\PYG{n}{df}\PYG{p}{[}\PYG{l+s+s1}{\PYGZsq{}}\PYG{l+s+s1}{close\PYGZus{}r1}\PYG{l+s+s1}{\PYGZsq{}}\PYG{p}{]} \PYG{o}{=} \PYG{n}{df}\PYG{p}{[}\PYG{l+s+s1}{\PYGZsq{}}\PYG{l+s+s1}{close}\PYG{l+s+s1}{\PYGZsq{}}\PYG{p}{]}\PYG{o}{.}\PYG{n}{shift}\PYG{p}{(}\PYG{o}{\PYGZhy{}}\PYG{l+m+mi}{1}\PYG{p}{)}\PYG{o}{/}\PYG{n}{df}\PYG{p}{[}\PYG{l+s+s1}{\PYGZsq{}}\PYG{l+s+s1}{close}\PYG{l+s+s1}{\PYGZsq{}}\PYG{p}{]} \PYG{c+c1}{\PYGZsh{} 1 일후 종가 수익률}
\PYG{n}{df}\PYG{p}{[}\PYG{l+s+s1}{\PYGZsq{}}\PYG{l+s+s1}{close\PYGZus{}r2}\PYG{l+s+s1}{\PYGZsq{}}\PYG{p}{]} \PYG{o}{=} \PYG{n}{df}\PYG{p}{[}\PYG{l+s+s1}{\PYGZsq{}}\PYG{l+s+s1}{close}\PYG{l+s+s1}{\PYGZsq{}}\PYG{p}{]}\PYG{o}{.}\PYG{n}{shift}\PYG{p}{(}\PYG{o}{\PYGZhy{}}\PYG{l+m+mi}{2}\PYG{p}{)}\PYG{o}{/}\PYG{n}{df}\PYG{p}{[}\PYG{l+s+s1}{\PYGZsq{}}\PYG{l+s+s1}{close}\PYG{l+s+s1}{\PYGZsq{}}\PYG{p}{]} \PYG{c+c1}{\PYGZsh{} 2 일후 종가 수익률}
\PYG{n}{df}\PYG{p}{[}\PYG{l+s+s1}{\PYGZsq{}}\PYG{l+s+s1}{close\PYGZus{}r3}\PYG{l+s+s1}{\PYGZsq{}}\PYG{p}{]} \PYG{o}{=} \PYG{n}{df}\PYG{p}{[}\PYG{l+s+s1}{\PYGZsq{}}\PYG{l+s+s1}{close}\PYG{l+s+s1}{\PYGZsq{}}\PYG{p}{]}\PYG{o}{.}\PYG{n}{shift}\PYG{p}{(}\PYG{o}{\PYGZhy{}}\PYG{l+m+mi}{3}\PYG{p}{)}\PYG{o}{/}\PYG{n}{df}\PYG{p}{[}\PYG{l+s+s1}{\PYGZsq{}}\PYG{l+s+s1}{close}\PYG{l+s+s1}{\PYGZsq{}}\PYG{p}{]} \PYG{c+c1}{\PYGZsh{} 3 일후 종가 수익률}
\PYG{n}{df}\PYG{p}{[}\PYG{l+s+s1}{\PYGZsq{}}\PYG{l+s+s1}{close\PYGZus{}r4}\PYG{l+s+s1}{\PYGZsq{}}\PYG{p}{]} \PYG{o}{=} \PYG{n}{df}\PYG{p}{[}\PYG{l+s+s1}{\PYGZsq{}}\PYG{l+s+s1}{close}\PYG{l+s+s1}{\PYGZsq{}}\PYG{p}{]}\PYG{o}{.}\PYG{n}{shift}\PYG{p}{(}\PYG{o}{\PYGZhy{}}\PYG{l+m+mi}{4}\PYG{p}{)}\PYG{o}{/}\PYG{n}{df}\PYG{p}{[}\PYG{l+s+s1}{\PYGZsq{}}\PYG{l+s+s1}{close}\PYG{l+s+s1}{\PYGZsq{}}\PYG{p}{]} \PYG{c+c1}{\PYGZsh{} 4 일후 종가 수익률}
\PYG{n}{df}\PYG{p}{[}\PYG{l+s+s1}{\PYGZsq{}}\PYG{l+s+s1}{close\PYGZus{}r5}\PYG{l+s+s1}{\PYGZsq{}}\PYG{p}{]} \PYG{o}{=} \PYG{n}{df}\PYG{p}{[}\PYG{l+s+s1}{\PYGZsq{}}\PYG{l+s+s1}{close}\PYG{l+s+s1}{\PYGZsq{}}\PYG{p}{]}\PYG{o}{.}\PYG{n}{shift}\PYG{p}{(}\PYG{o}{\PYGZhy{}}\PYG{l+m+mi}{5}\PYG{p}{)}\PYG{o}{/}\PYG{n}{df}\PYG{p}{[}\PYG{l+s+s1}{\PYGZsq{}}\PYG{l+s+s1}{close}\PYG{l+s+s1}{\PYGZsq{}}\PYG{p}{]} \PYG{c+c1}{\PYGZsh{} 5 일후 종가 수익률}

\PYG{l+s+sd}{\PYGZsq{}\PYGZsq{}\PYGZsq{} 위 코드와 같은 결과}
\PYG{l+s+sd}{df[\PYGZsq{}return\PYGZus{}1\PYGZsq{}] = df[\PYGZsq{}return\PYGZsq{}].shift(\PYGZhy{}1)}
\PYG{l+s+sd}{df[\PYGZsq{}return\PYGZus{}2\PYGZsq{}] = df[\PYGZsq{}return\PYGZsq{}].shift(\PYGZhy{}2)*df[\PYGZsq{}return\PYGZsq{}].shift(\PYGZhy{}1)}
\PYG{l+s+sd}{df[\PYGZsq{}return\PYGZus{}3\PYGZsq{}] = df[\PYGZsq{}return\PYGZsq{}].shift(\PYGZhy{}3)*df[\PYGZsq{}return\PYGZsq{}].shift(\PYGZhy{}2)*df[\PYGZsq{}return\PYGZsq{}].shift(\PYGZhy{}1)}
\PYG{l+s+sd}{df[\PYGZsq{}return\PYGZus{}4\PYGZsq{}] = df[\PYGZsq{}return\PYGZsq{}].shift(\PYGZhy{}4)*df[\PYGZsq{}return\PYGZsq{}].shift(\PYGZhy{}3)*df[\PYGZsq{}return\PYGZsq{}].shift(\PYGZhy{}2)*df[\PYGZsq{}return\PYGZsq{}].shift(\PYGZhy{}1)}
\PYG{l+s+sd}{df[\PYGZsq{}return\PYGZus{}5\PYGZsq{}] = df[\PYGZsq{}return\PYGZsq{}].shift(\PYGZhy{}5)*df[\PYGZsq{}return\PYGZsq{}].shift(\PYGZhy{}4)*df[\PYGZsq{}return\PYGZsq{}].shift(\PYGZhy{}3)*df[\PYGZsq{}return\PYGZsq{}].shift(\PYGZhy{}2)*df[\PYGZsq{}return\PYGZsq{}].shift(\PYGZhy{}1)}
\PYG{l+s+sd}{\PYGZsq{}\PYGZsq{}\PYGZsq{}}

\PYG{n}{df}\PYG{p}{[}\PYG{l+s+s1}{\PYGZsq{}}\PYG{l+s+s1}{target}\PYG{l+s+s1}{\PYGZsq{}}\PYG{p}{]} \PYG{o}{=} \PYG{n}{df}\PYG{p}{[}\PYG{p}{[}\PYG{l+s+s1}{\PYGZsq{}}\PYG{l+s+s1}{close\PYGZus{}r1}\PYG{l+s+s1}{\PYGZsq{}}\PYG{p}{,}\PYG{l+s+s1}{\PYGZsq{}}\PYG{l+s+s1}{close\PYGZus{}r2}\PYG{l+s+s1}{\PYGZsq{}}\PYG{p}{,}\PYG{l+s+s1}{\PYGZsq{}}\PYG{l+s+s1}{close\PYGZus{}r3}\PYG{l+s+s1}{\PYGZsq{}}\PYG{p}{,}\PYG{l+s+s1}{\PYGZsq{}}\PYG{l+s+s1}{close\PYGZus{}r4}\PYG{l+s+s1}{\PYGZsq{}}\PYG{p}{,}\PYG{l+s+s1}{\PYGZsq{}}\PYG{l+s+s1}{close\PYGZus{}r5}\PYG{l+s+s1}{\PYGZsq{}}\PYG{p}{]}\PYG{p}{]}\PYG{o}{.}\PYG{n}{max}\PYG{p}{(}\PYG{n}{axis}\PYG{o}{=}\PYG{l+m+mi}{1}\PYG{p}{)} \PYG{c+c1}{\PYGZsh{} 주어지 컬럼에서 최대 값을 찾고 \PYGZsq{}target\PYGZsq{} 에 저장}
\PYG{n}{df}\PYG{o}{.}\PYG{n}{dropna}\PYG{p}{(}\PYG{n}{subset}\PYG{o}{=}\PYG{p}{[}\PYG{l+s+s1}{\PYGZsq{}}\PYG{l+s+s1}{close\PYGZus{}r1}\PYG{l+s+s1}{\PYGZsq{}}\PYG{p}{,}\PYG{l+s+s1}{\PYGZsq{}}\PYG{l+s+s1}{close\PYGZus{}r2}\PYG{l+s+s1}{\PYGZsq{}}\PYG{p}{,}\PYG{l+s+s1}{\PYGZsq{}}\PYG{l+s+s1}{close\PYGZus{}r3}\PYG{l+s+s1}{\PYGZsq{}}\PYG{p}{,}\PYG{l+s+s1}{\PYGZsq{}}\PYG{l+s+s1}{close\PYGZus{}r4}\PYG{l+s+s1}{\PYGZsq{}}\PYG{p}{,}\PYG{l+s+s1}{\PYGZsq{}}\PYG{l+s+s1}{close\PYGZus{}r5}\PYG{l+s+s1}{\PYGZsq{}}\PYG{p}{]}\PYG{p}{,} \PYG{n}{inplace}\PYG{o}{=}\PYG{k+kc}{True}\PYG{p}{)} \PYG{c+c1}{\PYGZsh{} 주어진 컬럼 중에 missing 값이 있으면 행을 제거(dropna)하고, 자신을 덮어 씀(inplace=True).}
\end{sphinxVerbatim}

\end{sphinxuseclass}\end{sphinxVerbatimInput}

\end{sphinxuseclass}
\begin{sphinxuseclass}{cell}\begin{sphinxVerbatimInput}

\begin{sphinxuseclass}{cell_input}
\begin{sphinxVerbatim}[commandchars=\\\{\}]
\PYG{n}{df}\PYG{o}{.}\PYG{n}{head}\PYG{p}{(}\PYG{l+m+mi}{10}\PYG{p}{)}\PYG{o}{.}\PYG{n}{style}\PYG{o}{.}\PYG{n}{set\PYGZus{}table\PYGZus{}attributes}\PYG{p}{(}\PYG{l+s+s1}{\PYGZsq{}}\PYG{l+s+s1}{style=}\PYG{l+s+s1}{\PYGZdq{}}\PYG{l+s+s1}{font\PYGZhy{}size: 12px}\PYG{l+s+s1}{\PYGZdq{}}\PYG{l+s+s1}{\PYGZsq{}}\PYG{p}{)}
\end{sphinxVerbatim}

\end{sphinxuseclass}\end{sphinxVerbatimInput}
\begin{sphinxVerbatimOutput}

\begin{sphinxuseclass}{cell_output}
\begin{sphinxVerbatim}[commandchars=\\\{\}]
\PYGZlt{}pandas.io.formats.style.Styler at 0x200186c8610\PYGZgt{}
\end{sphinxVerbatim}

\end{sphinxuseclass}\end{sphinxVerbatimOutput}

\end{sphinxuseclass}
\sphinxAtStartPar
이제 모든 종목에 대하여 For loop 로 매수 종가로 매도 시 수익율을 최대값을 생성합니다. ‘max\_close’ 의 분포를 보니 평균은 1.033, 최소값 0.326, 최대값 3.703 입니다. 단, max\_close 는 가설 검정으로 활용할 지표입니다. 매수 후, 몇 번 째 영업일이 최고 수익율인지 알 수 없기 때문에 기간 중 최고 수익율을 이용합니다.

\begin{sphinxuseclass}{cell}\begin{sphinxVerbatimInput}

\begin{sphinxuseclass}{cell_input}
\begin{sphinxVerbatim}[commandchars=\\\{\}]
\PYG{n}{kosdaq\PYGZus{}list} \PYG{o}{=} \PYG{n}{pd}\PYG{o}{.}\PYG{n}{read\PYGZus{}pickle}\PYG{p}{(}\PYG{l+s+s1}{\PYGZsq{}}\PYG{l+s+s1}{kosdaq\PYGZus{}list.pkl}\PYG{l+s+s1}{\PYGZsq{}}\PYG{p}{)}

\PYG{n}{mdl\PYGZus{}data} \PYG{o}{=} \PYG{n}{pd}\PYG{o}{.}\PYG{n}{DataFrame}\PYG{p}{(}\PYG{p}{)}

\PYG{k}{for} \PYG{n}{code} \PYG{o+ow}{in} \PYG{n}{kosdaq\PYGZus{}list}\PYG{p}{[}\PYG{l+s+s1}{\PYGZsq{}}\PYG{l+s+s1}{code}\PYG{l+s+s1}{\PYGZsq{}}\PYG{p}{]}\PYG{p}{:}
    \PYG{n}{df} \PYG{o}{=} \PYG{n}{return\PYGZus{}all}\PYG{p}{[}\PYG{n}{return\PYGZus{}all}\PYG{p}{[}\PYG{l+s+s1}{\PYGZsq{}}\PYG{l+s+s1}{code}\PYG{l+s+s1}{\PYGZsq{}}\PYG{p}{]}\PYG{o}{==}\PYG{n}{code}\PYG{p}{]}\PYG{o}{.}\PYG{n}{sort\PYGZus{}index}\PYG{p}{(}\PYG{p}{)}\PYG{o}{.}\PYG{n}{copy}\PYG{p}{(}\PYG{p}{)}

    \PYG{n}{df}\PYG{p}{[}\PYG{l+s+s1}{\PYGZsq{}}\PYG{l+s+s1}{close\PYGZus{}r1}\PYG{l+s+s1}{\PYGZsq{}}\PYG{p}{]} \PYG{o}{=} \PYG{n}{df}\PYG{p}{[}\PYG{l+s+s1}{\PYGZsq{}}\PYG{l+s+s1}{close}\PYG{l+s+s1}{\PYGZsq{}}\PYG{p}{]}\PYG{o}{.}\PYG{n}{shift}\PYG{p}{(}\PYG{o}{\PYGZhy{}}\PYG{l+m+mi}{1}\PYG{p}{)}\PYG{o}{/}\PYG{n}{df}\PYG{p}{[}\PYG{l+s+s1}{\PYGZsq{}}\PYG{l+s+s1}{close}\PYG{l+s+s1}{\PYGZsq{}}\PYG{p}{]}
    \PYG{n}{df}\PYG{p}{[}\PYG{l+s+s1}{\PYGZsq{}}\PYG{l+s+s1}{close\PYGZus{}r2}\PYG{l+s+s1}{\PYGZsq{}}\PYG{p}{]} \PYG{o}{=} \PYG{n}{df}\PYG{p}{[}\PYG{l+s+s1}{\PYGZsq{}}\PYG{l+s+s1}{close}\PYG{l+s+s1}{\PYGZsq{}}\PYG{p}{]}\PYG{o}{.}\PYG{n}{shift}\PYG{p}{(}\PYG{o}{\PYGZhy{}}\PYG{l+m+mi}{2}\PYG{p}{)}\PYG{o}{/}\PYG{n}{df}\PYG{p}{[}\PYG{l+s+s1}{\PYGZsq{}}\PYG{l+s+s1}{close}\PYG{l+s+s1}{\PYGZsq{}}\PYG{p}{]}
    \PYG{n}{df}\PYG{p}{[}\PYG{l+s+s1}{\PYGZsq{}}\PYG{l+s+s1}{close\PYGZus{}r3}\PYG{l+s+s1}{\PYGZsq{}}\PYG{p}{]} \PYG{o}{=} \PYG{n}{df}\PYG{p}{[}\PYG{l+s+s1}{\PYGZsq{}}\PYG{l+s+s1}{close}\PYG{l+s+s1}{\PYGZsq{}}\PYG{p}{]}\PYG{o}{.}\PYG{n}{shift}\PYG{p}{(}\PYG{o}{\PYGZhy{}}\PYG{l+m+mi}{3}\PYG{p}{)}\PYG{o}{/}\PYG{n}{df}\PYG{p}{[}\PYG{l+s+s1}{\PYGZsq{}}\PYG{l+s+s1}{close}\PYG{l+s+s1}{\PYGZsq{}}\PYG{p}{]}
    \PYG{n}{df}\PYG{p}{[}\PYG{l+s+s1}{\PYGZsq{}}\PYG{l+s+s1}{close\PYGZus{}r4}\PYG{l+s+s1}{\PYGZsq{}}\PYG{p}{]} \PYG{o}{=} \PYG{n}{df}\PYG{p}{[}\PYG{l+s+s1}{\PYGZsq{}}\PYG{l+s+s1}{close}\PYG{l+s+s1}{\PYGZsq{}}\PYG{p}{]}\PYG{o}{.}\PYG{n}{shift}\PYG{p}{(}\PYG{o}{\PYGZhy{}}\PYG{l+m+mi}{4}\PYG{p}{)}\PYG{o}{/}\PYG{n}{df}\PYG{p}{[}\PYG{l+s+s1}{\PYGZsq{}}\PYG{l+s+s1}{close}\PYG{l+s+s1}{\PYGZsq{}}\PYG{p}{]}
    \PYG{n}{df}\PYG{p}{[}\PYG{l+s+s1}{\PYGZsq{}}\PYG{l+s+s1}{close\PYGZus{}r5}\PYG{l+s+s1}{\PYGZsq{}}\PYG{p}{]} \PYG{o}{=} \PYG{n}{df}\PYG{p}{[}\PYG{l+s+s1}{\PYGZsq{}}\PYG{l+s+s1}{close}\PYG{l+s+s1}{\PYGZsq{}}\PYG{p}{]}\PYG{o}{.}\PYG{n}{shift}\PYG{p}{(}\PYG{o}{\PYGZhy{}}\PYG{l+m+mi}{5}\PYG{p}{)}\PYG{o}{/}\PYG{n}{df}\PYG{p}{[}\PYG{l+s+s1}{\PYGZsq{}}\PYG{l+s+s1}{close}\PYG{l+s+s1}{\PYGZsq{}}\PYG{p}{]}

    \PYG{n}{df}\PYG{p}{[}\PYG{l+s+s1}{\PYGZsq{}}\PYG{l+s+s1}{max\PYGZus{}close}\PYG{l+s+s1}{\PYGZsq{}}\PYG{p}{]} \PYG{o}{=} \PYG{n}{df}\PYG{p}{[}\PYG{p}{[}\PYG{l+s+s1}{\PYGZsq{}}\PYG{l+s+s1}{close\PYGZus{}r1}\PYG{l+s+s1}{\PYGZsq{}}\PYG{p}{,}\PYG{l+s+s1}{\PYGZsq{}}\PYG{l+s+s1}{close\PYGZus{}r2}\PYG{l+s+s1}{\PYGZsq{}}\PYG{p}{,}\PYG{l+s+s1}{\PYGZsq{}}\PYG{l+s+s1}{close\PYGZus{}r3}\PYG{l+s+s1}{\PYGZsq{}}\PYG{p}{,}\PYG{l+s+s1}{\PYGZsq{}}\PYG{l+s+s1}{close\PYGZus{}r4}\PYG{l+s+s1}{\PYGZsq{}}\PYG{p}{,}\PYG{l+s+s1}{\PYGZsq{}}\PYG{l+s+s1}{close\PYGZus{}r5}\PYG{l+s+s1}{\PYGZsq{}}\PYG{p}{]}\PYG{p}{]}\PYG{o}{.}\PYG{n}{max}\PYG{p}{(}\PYG{n}{axis}\PYG{o}{=}\PYG{l+m+mi}{1}\PYG{p}{)} \PYG{c+c1}{\PYGZsh{} 주어지 컬럼에서 최대 값을 찾음}
    \PYG{n}{df}\PYG{o}{.}\PYG{n}{dropna}\PYG{p}{(}\PYG{n}{subset}\PYG{o}{=}\PYG{p}{[}\PYG{l+s+s1}{\PYGZsq{}}\PYG{l+s+s1}{close\PYGZus{}r1}\PYG{l+s+s1}{\PYGZsq{}}\PYG{p}{,}\PYG{l+s+s1}{\PYGZsq{}}\PYG{l+s+s1}{close\PYGZus{}r2}\PYG{l+s+s1}{\PYGZsq{}}\PYG{p}{,}\PYG{l+s+s1}{\PYGZsq{}}\PYG{l+s+s1}{close\PYGZus{}r3}\PYG{l+s+s1}{\PYGZsq{}}\PYG{p}{,}\PYG{l+s+s1}{\PYGZsq{}}\PYG{l+s+s1}{close\PYGZus{}r4}\PYG{l+s+s1}{\PYGZsq{}}\PYG{p}{,}\PYG{l+s+s1}{\PYGZsq{}}\PYG{l+s+s1}{close\PYGZus{}r5}\PYG{l+s+s1}{\PYGZsq{}}\PYG{p}{]}\PYG{p}{,} \PYG{n}{inplace}\PYG{o}{=}\PYG{k+kc}{True}\PYG{p}{)} \PYG{c+c1}{\PYGZsh{} 주어진 컬럼 중에 missing 값이 있으면 행을 제거(dropna)하고, 자신을 덮어 씀(inplace=True).}
    
    \PYG{n}{mdl\PYGZus{}data} \PYG{o}{=} \PYG{n}{pd}\PYG{o}{.}\PYG{n}{concat}\PYG{p}{(}\PYG{p}{[}\PYG{n}{mdl\PYGZus{}data}\PYG{p}{,} \PYG{n}{df}\PYG{p}{]}\PYG{p}{,} \PYG{n}{axis}\PYG{o}{=}\PYG{l+m+mi}{0}\PYG{p}{)}    
    
\PYG{n}{mdl\PYGZus{}data}\PYG{o}{.}\PYG{n}{to\PYGZus{}pickle}\PYG{p}{(}\PYG{l+s+s1}{\PYGZsq{}}\PYG{l+s+s1}{mdl\PYGZus{}data.pkl}\PYG{l+s+s1}{\PYGZsq{}}\PYG{p}{)}
\end{sphinxVerbatim}

\end{sphinxuseclass}\end{sphinxVerbatimInput}

\end{sphinxuseclass}
\sphinxAtStartPar
 ‘max\_close’ 의 분포를 확인합니다.

\begin{sphinxuseclass}{cell}\begin{sphinxVerbatimInput}

\begin{sphinxuseclass}{cell_input}
\begin{sphinxVerbatim}[commandchars=\\\{\}]
\PYG{n}{mdl\PYGZus{}data} \PYG{o}{=} \PYG{n}{pd}\PYG{o}{.}\PYG{n}{read\PYGZus{}pickle}\PYG{p}{(}\PYG{l+s+s1}{\PYGZsq{}}\PYG{l+s+s1}{mdl\PYGZus{}data.pkl}\PYG{l+s+s1}{\PYGZsq{}}\PYG{p}{)}
\PYG{n+nb}{print}\PYG{p}{(}\PYG{n}{mdl\PYGZus{}data}\PYG{p}{[}\PYG{l+s+s1}{\PYGZsq{}}\PYG{l+s+s1}{max\PYGZus{}close}\PYG{l+s+s1}{\PYGZsq{}}\PYG{p}{]}\PYG{o}{.}\PYG{n}{describe}\PYG{p}{(}\PYG{n}{percentiles}\PYG{o}{=}\PYG{p}{[}\PYG{l+m+mf}{0.1}\PYG{p}{,} \PYG{l+m+mf}{0.2}\PYG{p}{,} \PYG{l+m+mf}{0.5}\PYG{p}{,} \PYG{l+m+mf}{0.8}\PYG{p}{,} \PYG{l+m+mf}{0.9}\PYG{p}{]}\PYG{p}{)}\PYG{p}{)}
\end{sphinxVerbatim}

\end{sphinxuseclass}\end{sphinxVerbatimInput}
\begin{sphinxVerbatimOutput}

\begin{sphinxuseclass}{cell_output}
\begin{sphinxVerbatim}[commandchars=\\\{\}]
count   426,517.00
mean          1.03
std           0.07
min           0.33
10\PYGZpc{}           0.98
20\PYGZpc{}           0.99
50\PYGZpc{}           1.02
80\PYGZpc{}           1.06
90\PYGZpc{}           1.10
max           3.70
Name: max\PYGZus{}close, dtype: float64
\end{sphinxVerbatim}

\end{sphinxuseclass}\end{sphinxVerbatimOutput}

\end{sphinxuseclass}

\section{ 매도 전략 데이터 프로세싱}
\label{\detokenize{chapter4/4.4.4_Data_Processing:br}}
\sphinxAtStartPar
모델 개발을 위해서는 매도 전략에 따는 수익을 계산을 할 수 있어야 합니다. 이번 장에서는 기본적인 몇 가지 전략의 수익율을 계산해보겠습니다. 저장해 둔 mdl\_data pickle 파일을 읽습니다. mdl\_data 는 수익률 결과값이 있는 데이터입니다.

\begin{sphinxuseclass}{cell}\begin{sphinxVerbatimInput}

\begin{sphinxuseclass}{cell_input}
\begin{sphinxVerbatim}[commandchars=\\\{\}]
\PYG{n}{mdl\PYGZus{}data} \PYG{o}{=} \PYG{n}{pd}\PYG{o}{.}\PYG{n}{read\PYGZus{}pickle}\PYG{p}{(}\PYG{l+s+s1}{\PYGZsq{}}\PYG{l+s+s1}{mdl\PYGZus{}data.pkl}\PYG{l+s+s1}{\PYGZsq{}}\PYG{p}{)}
\end{sphinxVerbatim}

\end{sphinxuseclass}\end{sphinxVerbatimInput}

\end{sphinxuseclass}
\sphinxAtStartPar
 \sphinxstylestrong{매도 전략 1 \sphinxhyphen{} 모든 종목 종가 매수 후, 5 영업일 기간 6\% 익절 매도}한가지 전략을 테스트 해 보겠습니다. 모든 종목을 같은 금액으로 매일 종가 매수합니다. 매수 후 5 영업일 동안 수익율이 6\% 이상되면 곧바로 익절합니다. 나머지 종목은 5 영업일에 전부 종가 매도하면 수익율은 어떻게 될까요?

\begin{sphinxuseclass}{cell}\begin{sphinxVerbatimInput}

\begin{sphinxuseclass}{cell_input}
\begin{sphinxVerbatim}[commandchars=\\\{\}]
\PYG{n}{kosdaq\PYGZus{}list} \PYG{o}{=} \PYG{n}{pd}\PYG{o}{.}\PYG{n}{read\PYGZus{}pickle}\PYG{p}{(}\PYG{l+s+s1}{\PYGZsq{}}\PYG{l+s+s1}{kosdaq\PYGZus{}list.pkl}\PYG{l+s+s1}{\PYGZsq{}}\PYG{p}{)}

\PYG{n}{data\PYGZus{}all\PYGZus{}5} \PYG{o}{=} \PYG{n}{pd}\PYG{o}{.}\PYG{n}{DataFrame}\PYG{p}{(}\PYG{p}{)}

\PYG{n}{ub} \PYG{o}{=} \PYG{l+m+mf}{1.06}

\PYG{k}{for} \PYG{n}{code} \PYG{o+ow}{in} \PYG{n}{kosdaq\PYGZus{}list}\PYG{p}{[}\PYG{l+s+s1}{\PYGZsq{}}\PYG{l+s+s1}{code}\PYG{l+s+s1}{\PYGZsq{}}\PYG{p}{]}\PYG{p}{:}
    
    \PYG{c+c1}{\PYGZsh{} 종목별 처리}
    \PYG{n}{data} \PYG{o}{=} \PYG{n}{mdl\PYGZus{}data}\PYG{p}{[}\PYG{n}{mdl\PYGZus{}data}\PYG{p}{[}\PYG{l+s+s1}{\PYGZsq{}}\PYG{l+s+s1}{code}\PYG{l+s+s1}{\PYGZsq{}}\PYG{p}{]}\PYG{o}{==}\PYG{n}{code}\PYG{p}{]}\PYG{o}{.}\PYG{n}{sort\PYGZus{}index}\PYG{p}{(}\PYG{p}{)}\PYG{o}{.}\PYG{n}{copy}\PYG{p}{(}\PYG{p}{)}
    
    \PYG{c+c1}{\PYGZsh{} 고가, 저가, 종가 수익율}
    \PYG{k}{for} \PYG{n}{i} \PYG{o+ow}{in} \PYG{p}{[}\PYG{l+m+mi}{1}\PYG{p}{,} \PYG{l+m+mi}{2}\PYG{p}{,} \PYG{l+m+mi}{3}\PYG{p}{,} \PYG{l+m+mi}{4}\PYG{p}{,} \PYG{l+m+mi}{5}\PYG{p}{]}\PYG{p}{:}

        \PYG{n}{data}\PYG{p}{[}\PYG{l+s+s1}{\PYGZsq{}}\PYG{l+s+s1}{high\PYGZus{}r}\PYG{l+s+s1}{\PYGZsq{}} \PYG{o}{+} \PYG{n+nb}{str}\PYG{p}{(}\PYG{n}{i}\PYG{p}{)}\PYG{p}{]} \PYG{o}{=} \PYG{n}{data}\PYG{p}{[}\PYG{l+s+s1}{\PYGZsq{}}\PYG{l+s+s1}{high}\PYG{l+s+s1}{\PYGZsq{}}\PYG{p}{]}\PYG{o}{.}\PYG{n}{shift}\PYG{p}{(}\PYG{o}{\PYGZhy{}}\PYG{l+m+mi}{1}\PYG{o}{*}\PYG{n}{i}\PYG{p}{)}\PYG{o}{/}\PYG{n}{data}\PYG{p}{[}\PYG{l+s+s1}{\PYGZsq{}}\PYG{l+s+s1}{close}\PYG{l+s+s1}{\PYGZsq{}}\PYG{p}{]}      
        \PYG{n}{data}\PYG{p}{[}\PYG{l+s+s1}{\PYGZsq{}}\PYG{l+s+s1}{low\PYGZus{}r}\PYG{l+s+s1}{\PYGZsq{}} \PYG{o}{+} \PYG{n+nb}{str}\PYG{p}{(}\PYG{n}{i}\PYG{p}{)}\PYG{p}{]} \PYG{o}{=} \PYG{n}{data}\PYG{p}{[}\PYG{l+s+s1}{\PYGZsq{}}\PYG{l+s+s1}{low}\PYG{l+s+s1}{\PYGZsq{}}\PYG{p}{]}\PYG{o}{.}\PYG{n}{shift}\PYG{p}{(}\PYG{o}{\PYGZhy{}}\PYG{l+m+mi}{1}\PYG{o}{*}\PYG{n}{i}\PYG{p}{)}\PYG{o}{/}\PYG{n}{data}\PYG{p}{[}\PYG{l+s+s1}{\PYGZsq{}}\PYG{l+s+s1}{close}\PYG{l+s+s1}{\PYGZsq{}}\PYG{p}{]}   
        \PYG{n}{data}\PYG{p}{[}\PYG{l+s+s1}{\PYGZsq{}}\PYG{l+s+s1}{close\PYGZus{}r}\PYG{l+s+s1}{\PYGZsq{}} \PYG{o}{+} \PYG{n+nb}{str}\PYG{p}{(}\PYG{n}{i}\PYG{p}{)}\PYG{p}{]} \PYG{o}{=} \PYG{n}{data}\PYG{p}{[}\PYG{l+s+s1}{\PYGZsq{}}\PYG{l+s+s1}{close}\PYG{l+s+s1}{\PYGZsq{}}\PYG{p}{]}\PYG{o}{.}\PYG{n}{shift}\PYG{p}{(}\PYG{o}{\PYGZhy{}}\PYG{l+m+mi}{1}\PYG{o}{*}\PYG{n}{i}\PYG{p}{)}\PYG{o}{/}\PYG{n}{data}\PYG{p}{[}\PYG{l+s+s1}{\PYGZsq{}}\PYG{l+s+s1}{close}\PYG{l+s+s1}{\PYGZsq{}}\PYG{p}{]}    
        
    \PYG{n}{data}\PYG{p}{[}\PYG{l+s+s1}{\PYGZsq{}}\PYG{l+s+s1}{max\PYGZus{}high}\PYG{l+s+s1}{\PYGZsq{}}\PYG{p}{]}  \PYG{o}{=} \PYG{p}{(}\PYG{n}{data}\PYG{p}{[}\PYG{p}{[}\PYG{l+s+s1}{\PYGZsq{}}\PYG{l+s+s1}{high\PYGZus{}r1}\PYG{l+s+s1}{\PYGZsq{}}\PYG{p}{,}\PYG{l+s+s1}{\PYGZsq{}}\PYG{l+s+s1}{high\PYGZus{}r2}\PYG{l+s+s1}{\PYGZsq{}}\PYG{p}{,}\PYG{l+s+s1}{\PYGZsq{}}\PYG{l+s+s1}{high\PYGZus{}r3}\PYG{l+s+s1}{\PYGZsq{}}\PYG{p}{,}\PYG{l+s+s1}{\PYGZsq{}}\PYG{l+s+s1}{high\PYGZus{}r4}\PYG{l+s+s1}{\PYGZsq{}}\PYG{p}{,}\PYG{l+s+s1}{\PYGZsq{}}\PYG{l+s+s1}{high\PYGZus{}r5}\PYG{l+s+s1}{\PYGZsq{}}\PYG{p}{]}\PYG{p}{]}\PYG{o}{.}\PYG{n}{max}\PYG{p}{(}\PYG{n}{axis}\PYG{o}{=}\PYG{l+m+mi}{1}\PYG{p}{)} \PYG{o}{\PYGZgt{}} \PYG{n}{ub}\PYG{p}{)}\PYG{o}{.}\PYG{n}{astype}\PYG{p}{(}\PYG{n+nb}{int}\PYG{p}{)} \PYG{c+c1}{\PYGZsh{} 5 영업일 최고가 중 최고가         }
    \PYG{n}{data}\PYG{p}{[}\PYG{l+s+s1}{\PYGZsq{}}\PYG{l+s+s1}{ub\PYGZus{}return}\PYG{l+s+s1}{\PYGZsq{}}\PYG{p}{]} \PYG{o}{=} \PYG{n}{np}\PYG{o}{.}\PYG{n}{where}\PYG{p}{(}\PYG{n}{data}\PYG{p}{[}\PYG{l+s+s1}{\PYGZsq{}}\PYG{l+s+s1}{max\PYGZus{}high}\PYG{l+s+s1}{\PYGZsq{}}\PYG{p}{]}\PYG{o}{==}\PYG{l+m+mi}{1}\PYG{p}{,} \PYG{n}{ub}\PYG{p}{,} \PYG{n}{data}\PYG{p}{[}\PYG{l+s+s1}{\PYGZsq{}}\PYG{l+s+s1}{close\PYGZus{}r5}\PYG{l+s+s1}{\PYGZsq{}}\PYG{p}{]}\PYG{p}{)} \PYG{c+c1}{\PYGZsh{} 종가 수익률이 6\PYGZpc{} 이면 매도, 아니면 마지막 5 영업일 수익률}
       
    \PYG{n}{data}\PYG{o}{.}\PYG{n}{dropna}\PYG{p}{(}\PYG{n}{subset}\PYG{o}{=}\PYG{p}{[}\PYG{l+s+s1}{\PYGZsq{}}\PYG{l+s+s1}{close\PYGZus{}r1}\PYG{l+s+s1}{\PYGZsq{}}\PYG{p}{,}\PYG{l+s+s1}{\PYGZsq{}}\PYG{l+s+s1}{close\PYGZus{}r2}\PYG{l+s+s1}{\PYGZsq{}}\PYG{p}{,}\PYG{l+s+s1}{\PYGZsq{}}\PYG{l+s+s1}{close\PYGZus{}r3}\PYG{l+s+s1}{\PYGZsq{}}\PYG{p}{,}\PYG{l+s+s1}{\PYGZsq{}}\PYG{l+s+s1}{close\PYGZus{}r4}\PYG{l+s+s1}{\PYGZsq{}}\PYG{p}{,}\PYG{l+s+s1}{\PYGZsq{}}\PYG{l+s+s1}{close\PYGZus{}r5}\PYG{l+s+s1}{\PYGZsq{}}\PYG{p}{]}\PYG{p}{,} \PYG{n}{inplace}\PYG{o}{=}\PYG{k+kc}{True}\PYG{p}{)}   
    \PYG{n}{data\PYGZus{}all\PYGZus{}5} \PYG{o}{=} \PYG{n}{pd}\PYG{o}{.}\PYG{n}{concat}\PYG{p}{(}\PYG{p}{[}\PYG{n}{data}\PYG{p}{,} \PYG{n}{data\PYGZus{}all\PYGZus{}5}\PYG{p}{]}\PYG{p}{,} \PYG{n}{axis}\PYG{o}{=}\PYG{l+m+mi}{0}\PYG{p}{)}

\PYG{n}{data\PYGZus{}all\PYGZus{}5}\PYG{o}{.}\PYG{n}{to\PYGZus{}pickle}\PYG{p}{(}\PYG{l+s+s1}{\PYGZsq{}}\PYG{l+s+s1}{data\PYGZus{}all\PYGZus{}5.pkl}\PYG{l+s+s1}{\PYGZsq{}}\PYG{p}{)} 
\end{sphinxVerbatim}

\end{sphinxuseclass}\end{sphinxVerbatimInput}

\end{sphinxuseclass}
\begin{sphinxuseclass}{cell}\begin{sphinxVerbatimInput}

\begin{sphinxuseclass}{cell_input}
\begin{sphinxVerbatim}[commandchars=\\\{\}]
\PYG{n}{data\PYGZus{}all\PYGZus{}5}\PYG{o}{.}\PYG{n}{head}\PYG{p}{(}\PYG{p}{)}\PYG{o}{.}\PYG{n}{style}\PYG{o}{.}\PYG{n}{set\PYGZus{}table\PYGZus{}attributes}\PYG{p}{(}\PYG{l+s+s1}{\PYGZsq{}}\PYG{l+s+s1}{style=}\PYG{l+s+s1}{\PYGZdq{}}\PYG{l+s+s1}{font\PYGZhy{}size: 12px}\PYG{l+s+s1}{\PYGZdq{}}\PYG{l+s+s1}{\PYGZsq{}}\PYG{p}{)}
\end{sphinxVerbatim}

\end{sphinxuseclass}\end{sphinxVerbatimInput}
\begin{sphinxVerbatimOutput}

\begin{sphinxuseclass}{cell_output}
\begin{sphinxVerbatim}[commandchars=\\\{\}]
\PYGZlt{}pandas.io.formats.style.Styler at 0x200186cb670\PYGZgt{}
\end{sphinxVerbatim}

\end{sphinxuseclass}\end{sphinxVerbatimOutput}

\end{sphinxuseclass}
\sphinxAtStartPar
 수익률의 분포를 확인합니다. 수익이 되는 전략은 아닙니다.

\begin{sphinxuseclass}{cell}\begin{sphinxVerbatimInput}

\begin{sphinxuseclass}{cell_input}
\begin{sphinxVerbatim}[commandchars=\\\{\}]
\PYG{n}{data\PYGZus{}all\PYGZus{}5} \PYG{o}{=} \PYG{n}{pd}\PYG{o}{.}\PYG{n}{read\PYGZus{}pickle}\PYG{p}{(}\PYG{l+s+s1}{\PYGZsq{}}\PYG{l+s+s1}{data\PYGZus{}all\PYGZus{}5.pkl}\PYG{l+s+s1}{\PYGZsq{}}\PYG{p}{)}
\PYG{n+nb}{print}\PYG{p}{(}\PYG{n}{data\PYGZus{}all\PYGZus{}5}\PYG{p}{[}\PYG{l+s+s1}{\PYGZsq{}}\PYG{l+s+s1}{ub\PYGZus{}return}\PYG{l+s+s1}{\PYGZsq{}}\PYG{p}{]}\PYG{o}{.}\PYG{n}{describe}\PYG{p}{(}\PYG{n}{percentiles}\PYG{o}{=}\PYG{p}{[}\PYG{l+m+mf}{0.01}\PYG{p}{,} \PYG{l+m+mf}{0.1}\PYG{p}{,} \PYG{l+m+mf}{0.5}\PYG{p}{,} \PYG{l+m+mf}{0.9}\PYG{p}{,} \PYG{l+m+mf}{0.99}\PYG{p}{]}\PYG{p}{)}\PYG{p}{)}
\PYG{n+nb}{print}\PYG{p}{(}\PYG{n}{data\PYGZus{}all\PYGZus{}5}\PYG{o}{.}\PYG{n}{groupby}\PYG{p}{(}\PYG{l+s+s1}{\PYGZsq{}}\PYG{l+s+s1}{max\PYGZus{}high}\PYG{l+s+s1}{\PYGZsq{}}\PYG{p}{)}\PYG{p}{[}\PYG{l+s+s1}{\PYGZsq{}}\PYG{l+s+s1}{ub\PYGZus{}return}\PYG{l+s+s1}{\PYGZsq{}}\PYG{p}{]}\PYG{o}{.}\PYG{n}{describe}\PYG{p}{(}\PYG{p}{)}\PYG{p}{)}
\end{sphinxVerbatim}

\end{sphinxuseclass}\end{sphinxVerbatimInput}
\begin{sphinxVerbatimOutput}

\begin{sphinxuseclass}{cell_output}
\begin{sphinxVerbatim}[commandchars=\\\{\}]
count   419,432.00
mean          1.00
std           0.06
min           0.29
1\PYGZpc{}            0.84
10\PYGZpc{}           0.93
50\PYGZpc{}           1.00
90\PYGZpc{}           1.06
99\PYGZpc{}           1.06
max           1.06
Name: ub\PYGZus{}return, dtype: float64
              count  mean  std  min  25\PYGZpc{}  50\PYGZpc{}  75\PYGZpc{}  max
max\PYGZus{}high                                               
0        280,145.00  0.97 0.05 0.29 0.95 0.98 1.00 1.06
1        139,287.00  1.06 0.00 1.06 1.06 1.06 1.06 1.06
\end{sphinxVerbatim}

\end{sphinxuseclass}\end{sphinxVerbatimOutput}

\end{sphinxuseclass}
\sphinxAtStartPar
 \sphinxstylestrong{매도 전략 2 \sphinxhyphen{} 모든 종목을 종가 매수 후, 아래와 같은 순서로 매도}
\begin{enumerate}
\sphinxsetlistlabels{\arabic}{enumi}{enumii}{}{.}%
\item {} 
\sphinxAtStartPar
익일 고가가 당일 고가 보다 크면 2 영업일 시가 매도

\item {} 
\sphinxAtStartPar
1 조건 만족하지 않으면 2 영업일 종가 매도

\end{enumerate}

\sphinxAtStartPar
위와 같은 매도 전략은 수익율이 어떻게 될까요?

\begin{sphinxuseclass}{cell}\begin{sphinxVerbatimInput}

\begin{sphinxuseclass}{cell_input}
\begin{sphinxVerbatim}[commandchars=\\\{\}]
\PYG{n}{kosdaq\PYGZus{}list} \PYG{o}{=} \PYG{n}{pd}\PYG{o}{.}\PYG{n}{read\PYGZus{}pickle}\PYG{p}{(}\PYG{l+s+s1}{\PYGZsq{}}\PYG{l+s+s1}{kosdaq\PYGZus{}list.pkl}\PYG{l+s+s1}{\PYGZsq{}}\PYG{p}{)}

\PYG{n}{data\PYGZus{}all\PYGZus{}5} \PYG{o}{=} \PYG{n}{pd}\PYG{o}{.}\PYG{n}{DataFrame}\PYG{p}{(}\PYG{p}{)}

\PYG{k}{def} \PYG{n+nf}{final\PYGZus{}r}\PYG{p}{(}\PYG{n}{x}\PYG{p}{)}\PYG{p}{:}
    
    \PYG{k}{if}   \PYG{n}{x}\PYG{p}{[}\PYG{l+s+s1}{\PYGZsq{}}\PYG{l+s+s1}{high\PYGZus{}r0}\PYG{l+s+s1}{\PYGZsq{}}\PYG{p}{]} \PYG{o}{\PYGZlt{}} \PYG{n}{x}\PYG{p}{[}\PYG{l+s+s1}{\PYGZsq{}}\PYG{l+s+s1}{high\PYGZus{}r1}\PYG{l+s+s1}{\PYGZsq{}}\PYG{p}{]}\PYG{p}{:}  \PYG{c+c1}{\PYGZsh{}  (당일 고가/매수 종가) 비율이 (익일 고가/매수 종가) 비율 값이 작으면 2 영업일 시가 매도     }
        \PYG{k}{return} \PYG{n}{x}\PYG{p}{[}\PYG{l+s+s1}{\PYGZsq{}}\PYG{l+s+s1}{open\PYGZus{}r2}\PYG{l+s+s1}{\PYGZsq{}}\PYG{p}{]}    
    
    \PYG{k}{else}\PYG{p}{:}
        \PYG{k}{return} \PYG{n}{x}\PYG{p}{[}\PYG{l+s+s1}{\PYGZsq{}}\PYG{l+s+s1}{close\PYGZus{}r2}\PYG{l+s+s1}{\PYGZsq{}}\PYG{p}{]} \PYG{c+c1}{\PYGZsh{} 매도 안된 종목은 전부 2 영업일 종가 매도         }
    
\PYG{k}{for} \PYG{n}{code} \PYG{o+ow}{in} \PYG{n}{kosdaq\PYGZus{}list}\PYG{p}{[}\PYG{l+s+s1}{\PYGZsq{}}\PYG{l+s+s1}{code}\PYG{l+s+s1}{\PYGZsq{}}\PYG{p}{]}\PYG{p}{:}    
    
    \PYG{c+c1}{\PYGZsh{} 종목별 처리}
    \PYG{n}{data} \PYG{o}{=} \PYG{n}{mdl\PYGZus{}data}\PYG{p}{[}\PYG{n}{mdl\PYGZus{}data}\PYG{p}{[}\PYG{l+s+s1}{\PYGZsq{}}\PYG{l+s+s1}{code}\PYG{l+s+s1}{\PYGZsq{}}\PYG{p}{]}\PYG{o}{==}\PYG{n}{code}\PYG{p}{]}\PYG{o}{.}\PYG{n}{sort\PYGZus{}index}\PYG{p}{(}\PYG{p}{)}\PYG{o}{.}\PYG{n}{copy}\PYG{p}{(}\PYG{p}{)}
    
    \PYG{c+c1}{\PYGZsh{} 최고/최저 수익율}
    \PYG{k}{for} \PYG{n}{i} \PYG{o+ow}{in} \PYG{p}{[}\PYG{l+m+mi}{0}\PYG{p}{,} \PYG{l+m+mi}{1}\PYG{p}{,} \PYG{l+m+mi}{2}\PYG{p}{]}\PYG{p}{:}

        \PYG{n}{data}\PYG{p}{[}\PYG{l+s+s1}{\PYGZsq{}}\PYG{l+s+s1}{high\PYGZus{}r}\PYG{l+s+s1}{\PYGZsq{}} \PYG{o}{+} \PYG{n+nb}{str}\PYG{p}{(}\PYG{n}{i}\PYG{p}{)}\PYG{p}{]} \PYG{o}{=} \PYG{n}{data}\PYG{p}{[}\PYG{l+s+s1}{\PYGZsq{}}\PYG{l+s+s1}{high}\PYG{l+s+s1}{\PYGZsq{}}\PYG{p}{]}\PYG{o}{.}\PYG{n}{shift}\PYG{p}{(}\PYG{o}{\PYGZhy{}}\PYG{l+m+mi}{1}\PYG{o}{*}\PYG{n}{i}\PYG{p}{)}\PYG{o}{/}\PYG{n}{data}\PYG{p}{[}\PYG{l+s+s1}{\PYGZsq{}}\PYG{l+s+s1}{close}\PYG{l+s+s1}{\PYGZsq{}}\PYG{p}{]}        
        \PYG{n}{data}\PYG{p}{[}\PYG{l+s+s1}{\PYGZsq{}}\PYG{l+s+s1}{close\PYGZus{}r}\PYG{l+s+s1}{\PYGZsq{}} \PYG{o}{+} \PYG{n+nb}{str}\PYG{p}{(}\PYG{n}{i}\PYG{p}{)}\PYG{p}{]} \PYG{o}{=} \PYG{n}{data}\PYG{p}{[}\PYG{l+s+s1}{\PYGZsq{}}\PYG{l+s+s1}{close}\PYG{l+s+s1}{\PYGZsq{}}\PYG{p}{]}\PYG{o}{.}\PYG{n}{shift}\PYG{p}{(}\PYG{o}{\PYGZhy{}}\PYG{l+m+mi}{1}\PYG{o}{*}\PYG{n}{i}\PYG{p}{)}\PYG{o}{/}\PYG{n}{data}\PYG{p}{[}\PYG{l+s+s1}{\PYGZsq{}}\PYG{l+s+s1}{close}\PYG{l+s+s1}{\PYGZsq{}}\PYG{p}{]}
        \PYG{n}{data}\PYG{p}{[}\PYG{l+s+s1}{\PYGZsq{}}\PYG{l+s+s1}{open\PYGZus{}r}\PYG{l+s+s1}{\PYGZsq{}} \PYG{o}{+} \PYG{n+nb}{str}\PYG{p}{(}\PYG{n}{i}\PYG{p}{)}\PYG{p}{]} \PYG{o}{=} \PYG{n}{data}\PYG{p}{[}\PYG{l+s+s1}{\PYGZsq{}}\PYG{l+s+s1}{open}\PYG{l+s+s1}{\PYGZsq{}}\PYG{p}{]}\PYG{o}{.}\PYG{n}{shift}\PYG{p}{(}\PYG{o}{\PYGZhy{}}\PYG{l+m+mi}{1}\PYG{o}{*}\PYG{n}{i}\PYG{p}{)}\PYG{o}{/}\PYG{n}{data}\PYG{p}{[}\PYG{l+s+s1}{\PYGZsq{}}\PYG{l+s+s1}{close}\PYG{l+s+s1}{\PYGZsq{}}\PYG{p}{]}
        
    \PYG{n}{data}\PYG{p}{[}\PYG{l+s+s1}{\PYGZsq{}}\PYG{l+s+s1}{final\PYGZus{}return}\PYG{l+s+s1}{\PYGZsq{}}\PYG{p}{]} \PYG{o}{=} \PYG{n}{data}\PYG{o}{.}\PYG{n}{apply}\PYG{p}{(}\PYG{n}{final\PYGZus{}r}\PYG{p}{,} \PYG{n}{axis}\PYG{o}{=}\PYG{l+m+mi}{1}\PYG{p}{)} \PYG{c+c1}{\PYGZsh{} 각 row 에 대하여 final\PYGZus{}r 함수를 적용}
                                                                                                                                                 
    \PYG{n}{data}\PYG{o}{.}\PYG{n}{dropna}\PYG{p}{(}\PYG{n}{subset}\PYG{o}{=}\PYG{p}{[}\PYG{l+s+s1}{\PYGZsq{}}\PYG{l+s+s1}{close\PYGZus{}r0}\PYG{l+s+s1}{\PYGZsq{}}\PYG{p}{,}\PYG{l+s+s1}{\PYGZsq{}}\PYG{l+s+s1}{close\PYGZus{}r1}\PYG{l+s+s1}{\PYGZsq{}}\PYG{p}{,} \PYG{l+s+s1}{\PYGZsq{}}\PYG{l+s+s1}{close\PYGZus{}r2}\PYG{l+s+s1}{\PYGZsq{}}\PYG{p}{,}\PYG{l+s+s1}{\PYGZsq{}}\PYG{l+s+s1}{high\PYGZus{}r0}\PYG{l+s+s1}{\PYGZsq{}}\PYG{p}{,} \PYG{l+s+s1}{\PYGZsq{}}\PYG{l+s+s1}{high\PYGZus{}r1}\PYG{l+s+s1}{\PYGZsq{}}\PYG{p}{,} \PYG{l+s+s1}{\PYGZsq{}}\PYG{l+s+s1}{open\PYGZus{}r2}\PYG{l+s+s1}{\PYGZsq{}}\PYG{p}{]}\PYG{p}{,} \PYG{n}{inplace}\PYG{o}{=}\PYG{k+kc}{True}\PYG{p}{)}   \PYG{c+c1}{\PYGZsh{} 데아터 처리 중 missing 값이 사용된 경우는 제거}
    \PYG{n}{data\PYGZus{}all\PYGZus{}5} \PYG{o}{=} \PYG{n}{pd}\PYG{o}{.}\PYG{n}{concat}\PYG{p}{(}\PYG{p}{[}\PYG{n}{data}\PYG{p}{,} \PYG{n}{data\PYGZus{}all\PYGZus{}5}\PYG{p}{]}\PYG{p}{,} \PYG{n}{axis}\PYG{o}{=}\PYG{l+m+mi}{0}\PYG{p}{)}

\PYG{n}{data\PYGZus{}all\PYGZus{}5}\PYG{o}{.}\PYG{n}{to\PYGZus{}pickle}\PYG{p}{(}\PYG{l+s+s1}{\PYGZsq{}}\PYG{l+s+s1}{data\PYGZus{}all\PYGZus{}5.pkl}\PYG{l+s+s1}{\PYGZsq{}}\PYG{p}{)}    
\end{sphinxVerbatim}

\end{sphinxuseclass}\end{sphinxVerbatimInput}

\end{sphinxuseclass}
\sphinxAtStartPar
 수익률을 확인합니다.

\begin{sphinxuseclass}{cell}\begin{sphinxVerbatimInput}

\begin{sphinxuseclass}{cell_input}
\begin{sphinxVerbatim}[commandchars=\\\{\}]
\PYG{n}{data\PYGZus{}all\PYGZus{}5} \PYG{o}{=} \PYG{n}{pd}\PYG{o}{.}\PYG{n}{read\PYGZus{}pickle}\PYG{p}{(}\PYG{l+s+s1}{\PYGZsq{}}\PYG{l+s+s1}{data\PYGZus{}all\PYGZus{}5.pkl}\PYG{l+s+s1}{\PYGZsq{}}\PYG{p}{)}  
\PYG{n}{data\PYGZus{}all\PYGZus{}5}\PYG{p}{[}\PYG{l+s+s1}{\PYGZsq{}}\PYG{l+s+s1}{final\PYGZus{}return}\PYG{l+s+s1}{\PYGZsq{}}\PYG{p}{]}\PYG{o}{.}\PYG{n}{describe}\PYG{p}{(}\PYG{n}{percentiles}\PYG{o}{=}\PYG{p}{[}\PYG{l+m+mf}{0.01}\PYG{p}{,} \PYG{l+m+mf}{0.1}\PYG{p}{,} \PYG{l+m+mf}{0.5}\PYG{p}{,} \PYG{l+m+mf}{0.9}\PYG{p}{,} \PYG{l+m+mf}{0.99}\PYG{p}{]}\PYG{p}{)}
\end{sphinxVerbatim}

\end{sphinxuseclass}\end{sphinxVerbatimInput}
\begin{sphinxVerbatimOutput}

\begin{sphinxuseclass}{cell_output}
\begin{sphinxVerbatim}[commandchars=\\\{\}]
count   423,683.00
mean          1.00
std           0.05
min           0.00
1\PYGZpc{}            0.89
10\PYGZpc{}           0.96
50\PYGZpc{}           1.00
90\PYGZpc{}           1.04
99\PYGZpc{}           1.15
max           1.69
Name: final\PYGZus{}return, dtype: float64
\end{sphinxVerbatim}

\end{sphinxuseclass}\end{sphinxVerbatimOutput}

\end{sphinxuseclass}

\section{ 매수 전략 데이터 프로세싱}
\label{\detokenize{chapter4/4.4.4_Data_Processing:id2}}
\sphinxAtStartPar
모델 개발을 위해서는 매수 전략에 따라 매수 종목을 결정할 수 있어야 합니다. 이번 장에서는 기본적인 매수 종목을 찾는 데이터처리를 진행해 보겠습니다. 결과 수익률 데이터가 있는
mdl\_data pickle 파일을 읽습니다.

\begin{sphinxuseclass}{cell}\begin{sphinxVerbatimInput}

\begin{sphinxuseclass}{cell_input}
\begin{sphinxVerbatim}[commandchars=\\\{\}]
\PYG{n}{mdl\PYGZus{}data} \PYG{o}{=} \PYG{n}{pd}\PYG{o}{.}\PYG{n}{read\PYGZus{}pickle}\PYG{p}{(}\PYG{l+s+s1}{\PYGZsq{}}\PYG{l+s+s1}{mdl\PYGZus{}data.pkl}\PYG{l+s+s1}{\PYGZsq{}}\PYG{p}{)}
\end{sphinxVerbatim}

\end{sphinxuseclass}\end{sphinxVerbatimInput}

\end{sphinxuseclass}
\sphinxAtStartPar
 \sphinxstylestrong{매수 전략 1 \sphinxhyphen{} 시장 수익율보다 더 좋은 수익율을 보인 종목을 매수}시장 수익율보다 더 좋은 수익율을 보인 종목을 알기 위해 4.4.5 절에 ‘win\_market’ 이라는 변수를 생성했습니다. 이것을 이용할 것인데요. 더 의미있는 지표를 생성하기 위해서 과거 60일 누적 합을 보겠습니다. 수익율은 max\_close(5 영업일 중 최고 종가 수익율) 이용하겠습니다.

\begin{sphinxuseclass}{cell}\begin{sphinxVerbatimInput}

\begin{sphinxuseclass}{cell_input}
\begin{sphinxVerbatim}[commandchars=\\\{\}]
\PYG{n}{kosdaq\PYGZus{}list} \PYG{o}{=} \PYG{n}{pd}\PYG{o}{.}\PYG{n}{read\PYGZus{}pickle}\PYG{p}{(}\PYG{l+s+s1}{\PYGZsq{}}\PYG{l+s+s1}{kosdaq\PYGZus{}list.pkl}\PYG{l+s+s1}{\PYGZsq{}}\PYG{p}{)}

\PYG{n}{data\PYGZus{}all\PYGZus{}6} \PYG{o}{=} \PYG{n}{pd}\PYG{o}{.}\PYG{n}{DataFrame}\PYG{p}{(}\PYG{p}{)}

\PYG{k}{for} \PYG{n}{code} \PYG{o+ow}{in} \PYG{n}{kosdaq\PYGZus{}list}\PYG{p}{[}\PYG{l+s+s1}{\PYGZsq{}}\PYG{l+s+s1}{code}\PYG{l+s+s1}{\PYGZsq{}}\PYG{p}{]}\PYG{p}{:}
    
    \PYG{c+c1}{\PYGZsh{} 종목별 처리}
    \PYG{n}{data} \PYG{o}{=} \PYG{n}{mdl\PYGZus{}data}\PYG{p}{[}\PYG{n}{mdl\PYGZus{}data}\PYG{p}{[}\PYG{l+s+s1}{\PYGZsq{}}\PYG{l+s+s1}{code}\PYG{l+s+s1}{\PYGZsq{}}\PYG{p}{]}\PYG{o}{==}\PYG{n}{code}\PYG{p}{]}\PYG{o}{.}\PYG{n}{sort\PYGZus{}index}\PYG{p}{(}\PYG{p}{)}\PYG{o}{.}\PYG{n}{copy}\PYG{p}{(}\PYG{p}{)}
    
    \PYG{c+c1}{\PYGZsh{} 과거 60일 win\PYGZus{}market 누적 합}
    \PYG{n}{data}\PYG{p}{[}\PYG{l+s+s1}{\PYGZsq{}}\PYG{l+s+s1}{win\PYGZus{}market\PYGZus{}sum}\PYG{l+s+s1}{\PYGZsq{}}\PYG{p}{]} \PYG{o}{=} \PYG{n}{data}\PYG{p}{[}\PYG{l+s+s1}{\PYGZsq{}}\PYG{l+s+s1}{win\PYGZus{}market}\PYG{l+s+s1}{\PYGZsq{}}\PYG{p}{]}\PYG{o}{.}\PYG{n}{rolling}\PYG{p}{(}\PYG{l+m+mi}{60}\PYG{p}{)}\PYG{o}{.}\PYG{n}{sum}\PYG{p}{(}\PYG{p}{)} \PYG{c+c1}{\PYGZsh{} 과거 60일 누적합}
    
    \PYG{c+c1}{\PYGZsh{} 고가, 저가, 종가 수익율}
    \PYG{k}{for} \PYG{n}{i} \PYG{o+ow}{in} \PYG{p}{[}\PYG{l+m+mi}{1}\PYG{p}{,}\PYG{l+m+mi}{2}\PYG{p}{,}\PYG{l+m+mi}{3}\PYG{p}{,}\PYG{l+m+mi}{4}\PYG{p}{,}\PYG{l+m+mi}{5}\PYG{p}{]}\PYG{p}{:}

        \PYG{n}{data}\PYG{p}{[}\PYG{l+s+s1}{\PYGZsq{}}\PYG{l+s+s1}{high\PYGZus{}r}\PYG{l+s+s1}{\PYGZsq{}} \PYG{o}{+} \PYG{n+nb}{str}\PYG{p}{(}\PYG{n}{i}\PYG{p}{)}\PYG{p}{]} \PYG{o}{=} \PYG{n}{data}\PYG{p}{[}\PYG{l+s+s1}{\PYGZsq{}}\PYG{l+s+s1}{high}\PYG{l+s+s1}{\PYGZsq{}}\PYG{p}{]}\PYG{o}{.}\PYG{n}{shift}\PYG{p}{(}\PYG{o}{\PYGZhy{}}\PYG{l+m+mi}{1}\PYG{o}{*}\PYG{n}{i}\PYG{p}{)}\PYG{o}{/}\PYG{n}{data}\PYG{p}{[}\PYG{l+s+s1}{\PYGZsq{}}\PYG{l+s+s1}{close}\PYG{l+s+s1}{\PYGZsq{}}\PYG{p}{]}      
        \PYG{n}{data}\PYG{p}{[}\PYG{l+s+s1}{\PYGZsq{}}\PYG{l+s+s1}{low\PYGZus{}r}\PYG{l+s+s1}{\PYGZsq{}} \PYG{o}{+} \PYG{n+nb}{str}\PYG{p}{(}\PYG{n}{i}\PYG{p}{)}\PYG{p}{]} \PYG{o}{=} \PYG{n}{data}\PYG{p}{[}\PYG{l+s+s1}{\PYGZsq{}}\PYG{l+s+s1}{low}\PYG{l+s+s1}{\PYGZsq{}}\PYG{p}{]}\PYG{o}{.}\PYG{n}{shift}\PYG{p}{(}\PYG{o}{\PYGZhy{}}\PYG{l+m+mi}{1}\PYG{o}{*}\PYG{n}{i}\PYG{p}{)}\PYG{o}{/}\PYG{n}{data}\PYG{p}{[}\PYG{l+s+s1}{\PYGZsq{}}\PYG{l+s+s1}{close}\PYG{l+s+s1}{\PYGZsq{}}\PYG{p}{]}   
        \PYG{n}{data}\PYG{p}{[}\PYG{l+s+s1}{\PYGZsq{}}\PYG{l+s+s1}{close\PYGZus{}r}\PYG{l+s+s1}{\PYGZsq{}} \PYG{o}{+} \PYG{n+nb}{str}\PYG{p}{(}\PYG{n}{i}\PYG{p}{)}\PYG{p}{]} \PYG{o}{=} \PYG{n}{data}\PYG{p}{[}\PYG{l+s+s1}{\PYGZsq{}}\PYG{l+s+s1}{close}\PYG{l+s+s1}{\PYGZsq{}}\PYG{p}{]}\PYG{o}{.}\PYG{n}{shift}\PYG{p}{(}\PYG{o}{\PYGZhy{}}\PYG{l+m+mi}{1}\PYG{o}{*}\PYG{n}{i}\PYG{p}{)}\PYG{o}{/}\PYG{n}{data}\PYG{p}{[}\PYG{l+s+s1}{\PYGZsq{}}\PYG{l+s+s1}{close}\PYG{l+s+s1}{\PYGZsq{}}\PYG{p}{]}    
        
    \PYG{n}{data}\PYG{p}{[}\PYG{l+s+s1}{\PYGZsq{}}\PYG{l+s+s1}{max\PYGZus{}close}\PYG{l+s+s1}{\PYGZsq{}}\PYG{p}{]}  \PYG{o}{=} \PYG{n}{data}\PYG{p}{[}\PYG{p}{[}\PYG{l+s+s1}{\PYGZsq{}}\PYG{l+s+s1}{close\PYGZus{}r1}\PYG{l+s+s1}{\PYGZsq{}}\PYG{p}{,}\PYG{l+s+s1}{\PYGZsq{}}\PYG{l+s+s1}{close\PYGZus{}r2}\PYG{l+s+s1}{\PYGZsq{}}\PYG{p}{,}\PYG{l+s+s1}{\PYGZsq{}}\PYG{l+s+s1}{close\PYGZus{}r3}\PYG{l+s+s1}{\PYGZsq{}}\PYG{p}{,}\PYG{l+s+s1}{\PYGZsq{}}\PYG{l+s+s1}{close\PYGZus{}r4}\PYG{l+s+s1}{\PYGZsq{}}\PYG{p}{,}\PYG{l+s+s1}{\PYGZsq{}}\PYG{l+s+s1}{close\PYGZus{}r5}\PYG{l+s+s1}{\PYGZsq{}}\PYG{p}{]}\PYG{p}{]}\PYG{o}{.}\PYG{n}{max}\PYG{p}{(}\PYG{n}{axis}\PYG{o}{=}\PYG{l+m+mi}{1}\PYG{p}{)} \PYG{c+c1}{\PYGZsh{} 5 영업일 종가 수익율 중 최고 값}
    \PYG{n}{data}\PYG{o}{.}\PYG{n}{dropna}\PYG{p}{(}\PYG{n}{subset}\PYG{o}{=}\PYG{p}{[}\PYG{l+s+s1}{\PYGZsq{}}\PYG{l+s+s1}{win\PYGZus{}market\PYGZus{}sum}\PYG{l+s+s1}{\PYGZsq{}}\PYG{p}{,}\PYG{l+s+s1}{\PYGZsq{}}\PYG{l+s+s1}{close\PYGZus{}r1}\PYG{l+s+s1}{\PYGZsq{}}\PYG{p}{,}\PYG{l+s+s1}{\PYGZsq{}}\PYG{l+s+s1}{close\PYGZus{}r2}\PYG{l+s+s1}{\PYGZsq{}}\PYG{p}{,}\PYG{l+s+s1}{\PYGZsq{}}\PYG{l+s+s1}{close\PYGZus{}r3}\PYG{l+s+s1}{\PYGZsq{}}\PYG{p}{,}\PYG{l+s+s1}{\PYGZsq{}}\PYG{l+s+s1}{close\PYGZus{}r4}\PYG{l+s+s1}{\PYGZsq{}}\PYG{p}{,}\PYG{l+s+s1}{\PYGZsq{}}\PYG{l+s+s1}{close\PYGZus{}r5}\PYG{l+s+s1}{\PYGZsq{}}\PYG{p}{]}\PYG{p}{,} \PYG{n}{inplace}\PYG{o}{=}\PYG{k+kc}{True}\PYG{p}{)} \PYG{c+c1}{\PYGZsh{} missing 이 있는 행은 제거   }
 
    \PYG{n}{data\PYGZus{}all\PYGZus{}6} \PYG{o}{=} \PYG{n}{pd}\PYG{o}{.}\PYG{n}{concat}\PYG{p}{(}\PYG{p}{[}\PYG{n}{data}\PYG{p}{,} \PYG{n}{data\PYGZus{}all\PYGZus{}6}\PYG{p}{]}\PYG{p}{,} \PYG{n}{axis}\PYG{o}{=}\PYG{l+m+mi}{0}\PYG{p}{)}

\PYG{n}{data\PYGZus{}all\PYGZus{}6}\PYG{o}{.}\PYG{n}{to\PYGZus{}pickle}\PYG{p}{(}\PYG{l+s+s1}{\PYGZsq{}}\PYG{l+s+s1}{data\PYGZus{}all\PYGZus{}6.pkl}\PYG{l+s+s1}{\PYGZsq{}}\PYG{p}{)}    
\PYG{n}{data\PYGZus{}all\PYGZus{}6}\PYG{o}{.}\PYG{n}{head}\PYG{p}{(}\PYG{p}{)}\PYG{o}{.}\PYG{n}{style}\PYG{o}{.}\PYG{n}{set\PYGZus{}table\PYGZus{}attributes}\PYG{p}{(}\PYG{l+s+s1}{\PYGZsq{}}\PYG{l+s+s1}{style=}\PYG{l+s+s1}{\PYGZdq{}}\PYG{l+s+s1}{font\PYGZhy{}size: 12px}\PYG{l+s+s1}{\PYGZdq{}}\PYG{l+s+s1}{\PYGZsq{}}\PYG{p}{)}
\end{sphinxVerbatim}

\end{sphinxuseclass}\end{sphinxVerbatimInput}
\begin{sphinxVerbatimOutput}

\begin{sphinxuseclass}{cell_output}
\begin{sphinxVerbatim}[commandchars=\\\{\}]
\PYGZlt{}pandas.io.formats.style.Styler at 0x2188364a1c0\PYGZgt{}
\end{sphinxVerbatim}

\end{sphinxuseclass}\end{sphinxVerbatimOutput}

\end{sphinxuseclass}
\sphinxAtStartPar
 win\_market\_sum 에 따른 수익률의 변화를 확인합니다. win\_market\_sum 이 클수록 수익률이 높아지는 경향이 있다는 것을 확인했습니다.

\begin{sphinxuseclass}{cell}\begin{sphinxVerbatimInput}

\begin{sphinxuseclass}{cell_input}
\begin{sphinxVerbatim}[commandchars=\\\{\}]
\PYG{n}{data\PYGZus{}all\PYGZus{}6} \PYG{o}{=} \PYG{n}{pd}\PYG{o}{.}\PYG{n}{read\PYGZus{}pickle}\PYG{p}{(}\PYG{l+s+s1}{\PYGZsq{}}\PYG{l+s+s1}{data\PYGZus{}all\PYGZus{}6.pkl}\PYG{l+s+s1}{\PYGZsq{}}\PYG{p}{)}    
\PYG{n}{ranks} \PYG{o}{=} \PYG{n}{pd}\PYG{o}{.}\PYG{n}{qcut}\PYG{p}{(}\PYG{n}{data\PYGZus{}all\PYGZus{}6}\PYG{p}{[}\PYG{l+s+s1}{\PYGZsq{}}\PYG{l+s+s1}{win\PYGZus{}market\PYGZus{}sum}\PYG{l+s+s1}{\PYGZsq{}}\PYG{p}{]}\PYG{p}{,} \PYG{n}{q}\PYG{o}{=}\PYG{l+m+mi}{8}\PYG{p}{)}
\PYG{n+nb}{print}\PYG{p}{(}\PYG{n}{data\PYGZus{}all\PYGZus{}6}\PYG{o}{.}\PYG{n}{groupby}\PYG{p}{(}\PYG{n}{ranks}\PYG{p}{)}\PYG{p}{[}\PYG{l+s+s1}{\PYGZsq{}}\PYG{l+s+s1}{max\PYGZus{}close}\PYG{l+s+s1}{\PYGZsq{}}\PYG{p}{]}\PYG{o}{.}\PYG{n}{mean}\PYG{p}{(}\PYG{p}{)}\PYG{p}{)}
\PYG{n}{data\PYGZus{}all\PYGZus{}6}\PYG{o}{.}\PYG{n}{groupby}\PYG{p}{(}\PYG{n}{ranks}\PYG{p}{)}\PYG{p}{[}\PYG{l+s+s1}{\PYGZsq{}}\PYG{l+s+s1}{max\PYGZus{}close}\PYG{l+s+s1}{\PYGZsq{}}\PYG{p}{]}\PYG{o}{.}\PYG{n}{mean}\PYG{p}{(}\PYG{p}{)}\PYG{o}{.}\PYG{n}{plot}\PYG{p}{(}\PYG{n}{figsize}\PYG{o}{=}\PYG{p}{(}\PYG{l+m+mi}{12}\PYG{p}{,}\PYG{l+m+mi}{5}\PYG{p}{)}\PYG{p}{)}
\end{sphinxVerbatim}

\end{sphinxuseclass}\end{sphinxVerbatimInput}
\begin{sphinxVerbatimOutput}

\begin{sphinxuseclass}{cell_output}
\begin{sphinxVerbatim}[commandchars=\\\{\}]
win\PYGZus{}market\PYGZus{}sum
(\PYGZhy{}0.001, 4.0]   1.02
(4.0, 5.0]      1.03
(5.0, 6.0]      1.03
(6.0, 7.0]      1.03
(7.0, 8.0]      1.03
(8.0, 9.0]      1.03
(9.0, 11.0]     1.04
(11.0, 22.0]    1.04
Name: max\PYGZus{}close, dtype: float64
\end{sphinxVerbatim}

\begin{sphinxVerbatim}[commandchars=\\\{\}]
\PYGZlt{}AxesSubplot:xlabel=\PYGZsq{}win\PYGZus{}market\PYGZus{}sum\PYGZsq{}\PYGZgt{}
\end{sphinxVerbatim}

\noindent\sphinxincludegraphics{{4.4.4_Data_Processing_27_2}.png}

\end{sphinxuseclass}\end{sphinxVerbatimOutput}

\end{sphinxuseclass}
\sphinxAtStartPar
 \sphinxstylestrong{매수 전략 2 \sphinxhyphen{} 섹터 평균 수익율보다 더 높은 수익율을 보인 종목을 매수}kosdaq\_list 에 있는 종목별 섹터 정보를 이용하겠습니다. 우선, 종목별 최근 60일 평균 수익율을 rolling 함수를 이용하여 으로 계산합니다. for Loop 을 이용하여 종목에 섹터 정보를 추가합니다.

\begin{sphinxuseclass}{cell}\begin{sphinxVerbatimInput}

\begin{sphinxuseclass}{cell_input}
\begin{sphinxVerbatim}[commandchars=\\\{\}]
\PYG{n}{kosdaq\PYGZus{}list} \PYG{o}{=} \PYG{n}{pd}\PYG{o}{.}\PYG{n}{read\PYGZus{}pickle}\PYG{p}{(}\PYG{l+s+s1}{\PYGZsq{}}\PYG{l+s+s1}{kosdaq\PYGZus{}list.pkl}\PYG{l+s+s1}{\PYGZsq{}}\PYG{p}{)}

\PYG{n}{data\PYGZus{}all\PYGZus{}6} \PYG{o}{=} \PYG{n}{pd}\PYG{o}{.}\PYG{n}{DataFrame}\PYG{p}{(}\PYG{p}{)}

\PYG{k}{for} \PYG{n}{code}\PYG{p}{,} \PYG{n}{sector} \PYG{o+ow}{in} \PYG{n+nb}{zip}\PYG{p}{(}\PYG{n}{kosdaq\PYGZus{}list}\PYG{p}{[}\PYG{l+s+s1}{\PYGZsq{}}\PYG{l+s+s1}{code}\PYG{l+s+s1}{\PYGZsq{}}\PYG{p}{]}\PYG{p}{,} \PYG{n}{kosdaq\PYGZus{}list}\PYG{p}{[}\PYG{l+s+s1}{\PYGZsq{}}\PYG{l+s+s1}{sector}\PYG{l+s+s1}{\PYGZsq{}}\PYG{p}{]}\PYG{p}{)}\PYG{p}{:}
    
    \PYG{c+c1}{\PYGZsh{} 종목별 처리}
    \PYG{n}{data} \PYG{o}{=} \PYG{n}{mdl\PYGZus{}data}\PYG{p}{[}\PYG{n}{mdl\PYGZus{}data}\PYG{p}{[}\PYG{l+s+s1}{\PYGZsq{}}\PYG{l+s+s1}{code}\PYG{l+s+s1}{\PYGZsq{}}\PYG{p}{]}\PYG{o}{==}\PYG{n}{code}\PYG{p}{]}\PYG{o}{.}\PYG{n}{sort\PYGZus{}index}\PYG{p}{(}\PYG{p}{)}\PYG{o}{.}\PYG{n}{copy}\PYG{p}{(}\PYG{p}{)}
    \PYG{n}{data}\PYG{o}{.}\PYG{n}{dropna}\PYG{p}{(}\PYG{n}{inplace}\PYG{o}{=}\PYG{k+kc}{True}\PYG{p}{)}
    
    \PYG{c+c1}{\PYGZsh{} 최근 60일 평균 수익율            }
    \PYG{n}{data}\PYG{p}{[}\PYG{l+s+s1}{\PYGZsq{}}\PYG{l+s+s1}{return\PYGZus{}mean}\PYG{l+s+s1}{\PYGZsq{}}\PYG{p}{]} \PYG{o}{=} \PYG{n}{data}\PYG{p}{[}\PYG{l+s+s1}{\PYGZsq{}}\PYG{l+s+s1}{return}\PYG{l+s+s1}{\PYGZsq{}}\PYG{p}{]}\PYG{o}{.}\PYG{n}{rolling}\PYG{p}{(}\PYG{l+m+mi}{60}\PYG{p}{)}\PYG{o}{.}\PYG{n}{mean}\PYG{p}{(}\PYG{p}{)} \PYG{c+c1}{\PYGZsh{} 종목별 최근 60 일 수익율의 평균}
    \PYG{n}{data}\PYG{p}{[}\PYG{l+s+s1}{\PYGZsq{}}\PYG{l+s+s1}{sector}\PYG{l+s+s1}{\PYGZsq{}}\PYG{p}{]} \PYG{o}{=} \PYG{n}{sector}     
  
    \PYG{n}{data}\PYG{o}{.}\PYG{n}{dropna}\PYG{p}{(}\PYG{n}{subset}\PYG{o}{=}\PYG{p}{[}\PYG{l+s+s1}{\PYGZsq{}}\PYG{l+s+s1}{return\PYGZus{}mean}\PYG{l+s+s1}{\PYGZsq{}}\PYG{p}{]}\PYG{p}{,} \PYG{n}{inplace}\PYG{o}{=}\PYG{k+kc}{True}\PYG{p}{)}    
    \PYG{n}{data\PYGZus{}all\PYGZus{}6} \PYG{o}{=} \PYG{n}{pd}\PYG{o}{.}\PYG{n}{concat}\PYG{p}{(}\PYG{p}{[}\PYG{n}{data}\PYG{p}{,} \PYG{n}{data\PYGZus{}all\PYGZus{}6}\PYG{p}{]}\PYG{p}{,} \PYG{n}{axis}\PYG{o}{=}\PYG{l+m+mi}{0}\PYG{p}{)}

\PYG{n}{data\PYGZus{}all\PYGZus{}6}\PYG{o}{.}\PYG{n}{to\PYGZus{}pickle}\PYG{p}{(}\PYG{l+s+s1}{\PYGZsq{}}\PYG{l+s+s1}{data\PYGZus{}all\PYGZus{}6.pkl}\PYG{l+s+s1}{\PYGZsq{}}\PYG{p}{)}   
\end{sphinxVerbatim}

\end{sphinxuseclass}\end{sphinxVerbatimInput}

\end{sphinxuseclass}
\begin{sphinxuseclass}{cell}\begin{sphinxVerbatimInput}

\begin{sphinxuseclass}{cell_input}
\begin{sphinxVerbatim}[commandchars=\\\{\}]
\PYG{n}{data\PYGZus{}all\PYGZus{}6} \PYG{o}{=} \PYG{n}{pd}\PYG{o}{.}\PYG{n}{read\PYGZus{}pickle}\PYG{p}{(}\PYG{l+s+s1}{\PYGZsq{}}\PYG{l+s+s1}{data\PYGZus{}all\PYGZus{}6.pkl}\PYG{l+s+s1}{\PYGZsq{}}\PYG{p}{)} 
\PYG{n}{data\PYGZus{}all\PYGZus{}6}\PYG{o}{.}\PYG{n}{head}\PYG{p}{(}\PYG{p}{)}\PYG{o}{.}\PYG{n}{style}\PYG{o}{.}\PYG{n}{set\PYGZus{}table\PYGZus{}attributes}\PYG{p}{(}\PYG{l+s+s1}{\PYGZsq{}}\PYG{l+s+s1}{style=}\PYG{l+s+s1}{\PYGZdq{}}\PYG{l+s+s1}{font\PYGZhy{}size: 12px}\PYG{l+s+s1}{\PYGZdq{}}\PYG{l+s+s1}{\PYGZsq{}}\PYG{p}{)}
\end{sphinxVerbatim}

\end{sphinxuseclass}\end{sphinxVerbatimInput}
\begin{sphinxVerbatimOutput}

\begin{sphinxuseclass}{cell_output}
\begin{sphinxVerbatim}[commandchars=\\\{\}]
\PYGZlt{}pandas.io.formats.style.Styler at 0x2188435c190\PYGZgt{}
\end{sphinxVerbatim}

\end{sphinxuseclass}\end{sphinxVerbatimOutput}

\end{sphinxuseclass}
\sphinxAtStartPar
최근 60 일 평균수익율 정보를 섹터 별, 일 별로 요약한 값을 추가합니다. 이때 apply 대신 Transform 함수가 이용되었습니다. apply 는 그룹의 숫자 만큼 행을 리턴하나, transform 은 그룹핑 하기 전의 행 수 를 리턴합니다. 그 값을 ‘return over sector’ 라는 변수에 저장합니다.

\begin{sphinxuseclass}{cell}\begin{sphinxVerbatimInput}

\begin{sphinxuseclass}{cell_input}
\begin{sphinxVerbatim}[commandchars=\\\{\}]
\PYG{n}{data\PYGZus{}all\PYGZus{}6}\PYG{p}{[}\PYG{l+s+s1}{\PYGZsq{}}\PYG{l+s+s1}{sector\PYGZus{}return}\PYG{l+s+s1}{\PYGZsq{}}\PYG{p}{]} \PYG{o}{=} \PYG{n}{data\PYGZus{}all\PYGZus{}6}\PYG{o}{.}\PYG{n}{groupby}\PYG{p}{(}\PYG{p}{[}\PYG{l+s+s1}{\PYGZsq{}}\PYG{l+s+s1}{sector}\PYG{l+s+s1}{\PYGZsq{}}\PYG{p}{,} \PYG{n}{data\PYGZus{}all\PYGZus{}6}\PYG{o}{.}\PYG{n}{index}\PYG{p}{]}\PYG{p}{)}\PYG{p}{[}\PYG{l+s+s1}{\PYGZsq{}}\PYG{l+s+s1}{return}\PYG{l+s+s1}{\PYGZsq{}}\PYG{p}{]}\PYG{o}{.}\PYG{n}{transform}\PYG{p}{(}\PYG{k}{lambda} \PYG{n}{x}\PYG{p}{:} \PYG{n}{x}\PYG{o}{.}\PYG{n}{mean}\PYG{p}{(}\PYG{p}{)}\PYG{p}{)}
\PYG{n}{data\PYGZus{}all\PYGZus{}6}\PYG{p}{[}\PYG{l+s+s1}{\PYGZsq{}}\PYG{l+s+s1}{return over sector}\PYG{l+s+s1}{\PYGZsq{}}\PYG{p}{]} \PYG{o}{=} \PYG{p}{(}\PYG{n}{data\PYGZus{}all\PYGZus{}6}\PYG{p}{[}\PYG{l+s+s1}{\PYGZsq{}}\PYG{l+s+s1}{return}\PYG{l+s+s1}{\PYGZsq{}}\PYG{p}{]}\PYG{o}{/}\PYG{n}{data\PYGZus{}all\PYGZus{}6}\PYG{p}{[}\PYG{l+s+s1}{\PYGZsq{}}\PYG{l+s+s1}{sector\PYGZus{}return}\PYG{l+s+s1}{\PYGZsq{}}\PYG{p}{]}\PYG{p}{)} \PYG{c+c1}{\PYGZsh{} 섹터의 평균 수익률 대비 종목 수익률의 비율}
\end{sphinxVerbatim}

\end{sphinxuseclass}\end{sphinxVerbatimInput}

\end{sphinxuseclass}
\sphinxAtStartPar
결과를 보니, 섹터를 이용하여 종목을 선정할 때는 섹터 평균 수익율보다 많이 높거나, 많이 낮는 종목을 선정하는 것이 수익율이 좋게 나왔습니다. 섹터 평균 수익율 대비 종목 수익율은 미래 수익율 예측에 도움이 되는 정보입니다.

\begin{sphinxuseclass}{cell}\begin{sphinxVerbatimInput}

\begin{sphinxuseclass}{cell_input}
\begin{sphinxVerbatim}[commandchars=\\\{\}]
\PYG{n}{pd}\PYG{o}{.}\PYG{n}{options}\PYG{o}{.}\PYG{n}{display}\PYG{o}{.}\PYG{n}{float\PYGZus{}format} \PYG{o}{=} \PYG{l+s+s1}{\PYGZsq{}}\PYG{l+s+si}{\PYGZob{}:,.3f\PYGZcb{}}\PYG{l+s+s1}{\PYGZsq{}}\PYG{o}{.}\PYG{n}{format}
\PYG{n}{ranks} \PYG{o}{=} \PYG{n}{pd}\PYG{o}{.}\PYG{n}{qcut}\PYG{p}{(}\PYG{n}{data\PYGZus{}all\PYGZus{}6}\PYG{p}{[}\PYG{l+s+s1}{\PYGZsq{}}\PYG{l+s+s1}{return over sector}\PYG{l+s+s1}{\PYGZsq{}}\PYG{p}{]}\PYG{p}{,} \PYG{n}{q}\PYG{o}{=}\PYG{l+m+mi}{10}\PYG{p}{)}
\PYG{n+nb}{print}\PYG{p}{(}\PYG{n}{data\PYGZus{}all\PYGZus{}6}\PYG{o}{.}\PYG{n}{groupby}\PYG{p}{(}\PYG{n}{ranks}\PYG{p}{)}\PYG{p}{[}\PYG{l+s+s1}{\PYGZsq{}}\PYG{l+s+s1}{max\PYGZus{}close}\PYG{l+s+s1}{\PYGZsq{}}\PYG{p}{]}\PYG{o}{.}\PYG{n}{describe}\PYG{p}{(}\PYG{n}{percentiles}\PYG{o}{=}\PYG{p}{[}\PYG{l+m+mf}{0.01}\PYG{p}{,} \PYG{l+m+mf}{0.99}\PYG{p}{]}\PYG{p}{)}\PYG{p}{)}
\PYG{n}{data\PYGZus{}all\PYGZus{}6}\PYG{o}{.}\PYG{n}{groupby}\PYG{p}{(}\PYG{n}{ranks}\PYG{p}{)}\PYG{p}{[}\PYG{l+s+s1}{\PYGZsq{}}\PYG{l+s+s1}{max\PYGZus{}close}\PYG{l+s+s1}{\PYGZsq{}}\PYG{p}{]}\PYG{o}{.}\PYG{n}{mean}\PYG{p}{(}\PYG{p}{)}\PYG{o}{.}\PYG{n}{plot}\PYG{p}{(}\PYG{n}{figsize}\PYG{o}{=}\PYG{p}{(}\PYG{l+m+mi}{12}\PYG{p}{,}\PYG{l+m+mi}{5}\PYG{p}{)}\PYG{p}{)}
\end{sphinxVerbatim}

\end{sphinxuseclass}\end{sphinxVerbatimInput}
\begin{sphinxVerbatimOutput}

\begin{sphinxuseclass}{cell_output}
\begin{sphinxVerbatim}[commandchars=\\\{\}]
                        count  mean   std   min    1\PYGZpc{}   50\PYGZpc{}   99\PYGZpc{}   max
return over sector                                                     
(0.378, 0.974]     34,292.000 1.042 0.086 0.700 0.920 1.022 1.384 2.968
(0.974, 0.984]     34,291.000 1.034 0.067 0.702 0.945 1.018 1.302 2.171
(0.984, 0.99]      34,292.000 1.030 0.061 0.700 0.949 1.016 1.278 2.330
(0.99, 0.994]      34,291.000 1.028 0.059 0.701 0.952 1.014 1.265 2.269
(0.994, 0.999]     34,291.000 1.027 0.058 0.708 0.954 1.013 1.256 2.729
(0.999, 1.002]     34,291.000 1.029 0.066 0.700 0.949 1.013 1.295 3.701
(1.002, 1.007]     34,292.000 1.027 0.059 0.700 0.951 1.012 1.255 3.027
(1.007, 1.013]     34,291.000 1.028 0.062 0.700 0.949 1.013 1.276 3.380
(1.013, 1.026]     34,291.000 1.031 0.067 0.326 0.944 1.014 1.307 2.412
(1.026, 1.399]     34,292.000 1.042 0.103 0.700 0.910 1.019 1.419 3.703
\end{sphinxVerbatim}

\begin{sphinxVerbatim}[commandchars=\\\{\}]
\PYGZlt{}AxesSubplot:xlabel=\PYGZsq{}return over sector\PYGZsq{}\PYGZgt{}
\end{sphinxVerbatim}

\noindent\sphinxincludegraphics{{4.4.4_Data_Processing_34_2}.png}

\end{sphinxuseclass}\end{sphinxVerbatimOutput}

\end{sphinxuseclass}
\sphinxAtStartPar
한 섹터에 최소 10 개 이상의 종목이 있어야 섹터의 평균 수익율이 의미가 있을 것 같습니다. 10개 이상의 종목이 있는 섹터만을 매수 대상으로 해서 다시 수익율을 계산해봅니다. 같은 결과를 얻었습니다. 섹터의 평균 수익율보다 아주 낮거나 높은 종목의 수익율의 상승이 높았습니다. 그래프 곡선이 더 부드러워졌습니다.

\begin{sphinxuseclass}{cell}\begin{sphinxVerbatimInput}

\begin{sphinxuseclass}{cell_input}
\begin{sphinxVerbatim}[commandchars=\\\{\}]
\PYG{n}{sector\PYGZus{}count} \PYG{o}{=} \PYG{n}{data\PYGZus{}all\PYGZus{}6}\PYG{o}{.}\PYG{n}{groupby}\PYG{p}{(}\PYG{l+s+s1}{\PYGZsq{}}\PYG{l+s+s1}{sector}\PYG{l+s+s1}{\PYGZsq{}}\PYG{p}{)}\PYG{p}{[}\PYG{l+s+s1}{\PYGZsq{}}\PYG{l+s+s1}{code}\PYG{l+s+s1}{\PYGZsq{}}\PYG{p}{]}\PYG{o}{.}\PYG{n}{nunique}\PYG{p}{(}\PYG{p}{)}\PYG{o}{.}\PYG{n}{sort\PYGZus{}values}\PYG{p}{(}\PYG{p}{)}
\PYG{n}{data\PYGZus{}all\PYGZus{}6x} \PYG{o}{=} \PYG{n}{data\PYGZus{}all\PYGZus{}6}\PYG{p}{[}\PYG{n}{data\PYGZus{}all\PYGZus{}6}\PYG{p}{[}\PYG{l+s+s1}{\PYGZsq{}}\PYG{l+s+s1}{sector}\PYG{l+s+s1}{\PYGZsq{}}\PYG{p}{]}\PYG{o}{.}\PYG{n}{isin}\PYG{p}{(}\PYG{n}{sector\PYGZus{}count}\PYG{p}{[}\PYG{n}{sector\PYGZus{}count}\PYG{o}{\PYGZgt{}}\PYG{o}{=}\PYG{l+m+mi}{10}\PYG{p}{]}\PYG{o}{.}\PYG{n}{index}\PYG{p}{)}\PYG{p}{]}
\PYG{n}{ranks} \PYG{o}{=} \PYG{n}{pd}\PYG{o}{.}\PYG{n}{qcut}\PYG{p}{(}\PYG{n}{data\PYGZus{}all\PYGZus{}6x}\PYG{p}{[}\PYG{l+s+s1}{\PYGZsq{}}\PYG{l+s+s1}{return over sector}\PYG{l+s+s1}{\PYGZsq{}}\PYG{p}{]}\PYG{p}{,} \PYG{n}{q}\PYG{o}{=}\PYG{l+m+mi}{10}\PYG{p}{)}
\PYG{n+nb}{print}\PYG{p}{(}\PYG{n}{data\PYGZus{}all\PYGZus{}6x}\PYG{o}{.}\PYG{n}{groupby}\PYG{p}{(}\PYG{n}{ranks}\PYG{p}{)}\PYG{p}{[}\PYG{l+s+s1}{\PYGZsq{}}\PYG{l+s+s1}{max\PYGZus{}close}\PYG{l+s+s1}{\PYGZsq{}}\PYG{p}{]}\PYG{o}{.}\PYG{n}{describe}\PYG{p}{(}\PYG{n}{percentiles}\PYG{o}{=}\PYG{p}{[}\PYG{l+m+mf}{0.01}\PYG{p}{,} \PYG{l+m+mf}{0.99}\PYG{p}{]}\PYG{p}{)}\PYG{p}{)}
\PYG{n}{data\PYGZus{}all\PYGZus{}6x}\PYG{o}{.}\PYG{n}{groupby}\PYG{p}{(}\PYG{n}{ranks}\PYG{p}{)}\PYG{p}{[}\PYG{l+s+s1}{\PYGZsq{}}\PYG{l+s+s1}{max\PYGZus{}close}\PYG{l+s+s1}{\PYGZsq{}}\PYG{p}{]}\PYG{o}{.}\PYG{n}{mean}\PYG{p}{(}\PYG{p}{)}\PYG{o}{.}\PYG{n}{plot}\PYG{p}{(}\PYG{n}{figsize}\PYG{o}{=}\PYG{p}{(}\PYG{l+m+mi}{12}\PYG{p}{,}\PYG{l+m+mi}{5}\PYG{p}{)}\PYG{p}{)}
\end{sphinxVerbatim}

\end{sphinxuseclass}\end{sphinxVerbatimInput}
\begin{sphinxVerbatimOutput}

\begin{sphinxuseclass}{cell_output}
\begin{sphinxVerbatim}[commandchars=\\\{\}]
                        count  mean   std   min    1\PYGZpc{}   50\PYGZpc{}   99\PYGZpc{}   max
return over sector                                                     
(0.688, 0.973]     26,887.000 1.042 0.087 0.700 0.918 1.022 1.388 2.968
(0.973, 0.983]     26,886.000 1.034 0.067 0.702 0.944 1.019 1.299 2.171
(0.983, 0.989]     26,886.000 1.030 0.060 0.704 0.948 1.016 1.267 2.286
(0.989, 0.994]     26,886.000 1.028 0.060 0.700 0.952 1.014 1.269 2.194
(0.994, 0.998]     26,889.000 1.027 0.058 0.819 0.954 1.013 1.255 2.729
(0.998, 1.002]     26,883.000 1.026 0.059 0.700 0.954 1.012 1.257 2.652
(1.002, 1.007]     26,886.000 1.027 0.062 0.700 0.950 1.012 1.257 3.027
(1.007, 1.014]     26,886.000 1.028 0.062 0.700 0.949 1.013 1.285 3.380
(1.014, 1.027]     26,886.000 1.031 0.068 0.700 0.943 1.014 1.315 2.412
(1.027, 1.399]     26,887.000 1.043 0.105 0.700 0.909 1.019 1.423 3.703
\end{sphinxVerbatim}

\begin{sphinxVerbatim}[commandchars=\\\{\}]
\PYGZlt{}AxesSubplot:xlabel=\PYGZsq{}return over sector\PYGZsq{}\PYGZgt{}
\end{sphinxVerbatim}

\noindent\sphinxincludegraphics{{4.4.4_Data_Processing_36_2}.png}

\end{sphinxuseclass}\end{sphinxVerbatimOutput}

\end{sphinxuseclass}

\part{chapter 5}


\chapter{\sphinxstylestrong{가설 검정}}
\label{\detokenize{chapter5/5.1.0_Hypothesis:id1}}\label{\detokenize{chapter5/5.1.0_Hypothesis::doc}}
\sphinxAtStartPar
이번 장에서는 아래 가설들을 하나씩 데이터를 이용하여 검정해 보시는 시간입니다. 유의미한 것으로 증명된 가설은 주가를 예측하는 모델에 입력 피쳐로 들어가게 됩니다.
\begin{itemize}
\item {} 
\sphinxAtStartPar
변동성이 크고 거래량이 몰린 종목이 주가가 상승한다.

\item {} 
\sphinxAtStartPar
5일 이동 평균선이 오늘 종가보다 위에 위치해 있다.

\item {} 
\sphinxAtStartPar
위 꼬리가 긴 양봉이 자주 발생한다.

\item {} 
\sphinxAtStartPar
거래량이 종종 크게 터지며 매집의 흔적을 보인다.

\item {} 
\sphinxAtStartPar
주가지수보다 더 좋은 수익율을 자주 보여준다.

\item {} 
\sphinxAtStartPar
동종업계의 평균 수익률보다 좋은 수익률을 보여준다.

\item {} 
\sphinxAtStartPar
개인투자자보다는 투신/사모펀드 등이 매수를 많이 한다.

\end{itemize}

\begin{sphinxuseclass}{cell}\begin{sphinxVerbatimInput}

\begin{sphinxuseclass}{cell_input}
\begin{sphinxVerbatim}[commandchars=\\\{\}]
\PYG{k+kn}{import} \PYG{n+nn}{FinanceDataReader} \PYG{k}{as} \PYG{n+nn}{fdr}
\PYG{o}{\PYGZpc{}}\PYG{k}{matplotlib} inline
\PYG{k+kn}{import} \PYG{n+nn}{matplotlib}\PYG{n+nn}{.}\PYG{n+nn}{pyplot} \PYG{k}{as} \PYG{n+nn}{plt}
\PYG{k+kn}{import} \PYG{n+nn}{pandas} \PYG{k}{as} \PYG{n+nn}{pd}
\PYG{k+kn}{import} \PYG{n+nn}{numpy} \PYG{k}{as} \PYG{n+nn}{np}
\PYG{n}{pd}\PYG{o}{.}\PYG{n}{options}\PYG{o}{.}\PYG{n}{display}\PYG{o}{.}\PYG{n}{float\PYGZus{}format} \PYG{o}{=} \PYG{l+s+s1}{\PYGZsq{}}\PYG{l+s+si}{\PYGZob{}:,.3f\PYGZcb{}}\PYG{l+s+s1}{\PYGZsq{}}\PYG{o}{.}\PYG{n}{format}
\end{sphinxVerbatim}

\end{sphinxuseclass}\end{sphinxVerbatimInput}

\end{sphinxuseclass}

\section{가격 변동성이 크고 거래량이 몰린 종목이 주가가 상승한다.}
\label{\detokenize{chapter5/5.1.1_Hypothesis_1:id1}}\label{\detokenize{chapter5/5.1.1_Hypothesis_1::doc}}
\sphinxAtStartPar
“가격 변동성이 크고 거래량이 몰린 종목이 주가가 상승한다” 라는 가설을 증명하기 위해서는 “가격 변동성이 크다”, “거래량이 몰린다” 등을 표현하는 변수가 필요합니다. 먼저 일봉데이터를 불러옵니다.

\begin{sphinxuseclass}{cell}\begin{sphinxVerbatimInput}

\begin{sphinxuseclass}{cell_input}
\begin{sphinxVerbatim}[commandchars=\\\{\}]
\PYG{n}{mdl\PYGZus{}data} \PYG{o}{=} \PYG{n}{pd}\PYG{o}{.}\PYG{n}{read\PYGZus{}pickle}\PYG{p}{(}\PYG{l+s+s1}{\PYGZsq{}}\PYG{l+s+s1}{mdl\PYGZus{}data.pkl}\PYG{l+s+s1}{\PYGZsq{}}\PYG{p}{)}
\PYG{n}{mdl\PYGZus{}data}\PYG{o}{.}\PYG{n}{head}\PYG{p}{(}\PYG{p}{)}\PYG{o}{.}\PYG{n}{style}\PYG{o}{.}\PYG{n}{set\PYGZus{}table\PYGZus{}attributes}\PYG{p}{(}\PYG{l+s+s1}{\PYGZsq{}}\PYG{l+s+s1}{style=}\PYG{l+s+s1}{\PYGZdq{}}\PYG{l+s+s1}{font\PYGZhy{}size: 12px}\PYG{l+s+s1}{\PYGZdq{}}\PYG{l+s+s1}{\PYGZsq{}}\PYG{p}{)}
\end{sphinxVerbatim}

\end{sphinxuseclass}\end{sphinxVerbatimInput}
\begin{sphinxVerbatimOutput}

\begin{sphinxuseclass}{cell_output}
\begin{sphinxVerbatim}[commandchars=\\\{\}]
\PYGZlt{}pandas.io.formats.style.Styler at 0x2a0c840a5b0\PYGZgt{}
\end{sphinxVerbatim}

\end{sphinxuseclass}\end{sphinxVerbatimOutput}

\end{sphinxuseclass}
\sphinxAtStartPar
첫 번째 종목 060310 (종목이름 3S) 에 대하여 가격 변동성 변수를 만들어 보겠습니다. 전 5일 종가의 평균(price\_mean), 전 5일 종가의 표준편차(price\_std)를 먼저 구합니다. 그리고, 전 5일의 평균 및 표준편차 대비 당일 종가의 수준을 표준화해서 보여주는 값이 ‘price\_z’ 입니다. price\_z 값이 \sphinxhyphen{}1.96 와 +1.96 안에 값이면 95\% 신뢰구간 안에 들어갑니다. 즉 \sphinxhyphen{}1.96 보다 작거나, 1.96 보다 크면(100 번중 5번 미만으로 일어날 확율) 당일의 종가는 직전 5일의 움직임에 비해 아주 특별하다고 생각할 수 있습니다.

\begin{sphinxuseclass}{cell}\begin{sphinxVerbatimInput}

\begin{sphinxuseclass}{cell_input}
\begin{sphinxVerbatim}[commandchars=\\\{\}]
\PYG{n}{df} \PYG{o}{=} \PYG{n}{mdl\PYGZus{}data}\PYG{p}{[}\PYG{n}{mdl\PYGZus{}data}\PYG{p}{[}\PYG{l+s+s1}{\PYGZsq{}}\PYG{l+s+s1}{code}\PYG{l+s+s1}{\PYGZsq{}}\PYG{p}{]}\PYG{o}{==}\PYG{l+s+s1}{\PYGZsq{}}\PYG{l+s+s1}{060310}\PYG{l+s+s1}{\PYGZsq{}}\PYG{p}{]}\PYG{o}{.}\PYG{n}{copy}\PYG{p}{(}\PYG{p}{)} \PYG{c+c1}{\PYGZsh{} 종목 060310 선택}
\PYG{n}{df}\PYG{p}{[}\PYG{l+s+s1}{\PYGZsq{}}\PYG{l+s+s1}{price\PYGZus{}mean}\PYG{l+s+s1}{\PYGZsq{}}\PYG{p}{]} \PYG{o}{=} \PYG{n}{df}\PYG{p}{[}\PYG{l+s+s1}{\PYGZsq{}}\PYG{l+s+s1}{close}\PYG{l+s+s1}{\PYGZsq{}}\PYG{p}{]}\PYG{o}{.}\PYG{n}{rolling}\PYG{p}{(}\PYG{l+m+mi}{5}\PYG{p}{)}\PYG{o}{.}\PYG{n}{mean}\PYG{p}{(}\PYG{p}{)} \PYG{c+c1}{\PYGZsh{} 직전 5일 종가의 평균}
\PYG{n}{df}\PYG{p}{[}\PYG{l+s+s1}{\PYGZsq{}}\PYG{l+s+s1}{price\PYGZus{}std}\PYG{l+s+s1}{\PYGZsq{}}\PYG{p}{]} \PYG{o}{=} \PYG{n}{df}\PYG{p}{[}\PYG{l+s+s1}{\PYGZsq{}}\PYG{l+s+s1}{close}\PYG{l+s+s1}{\PYGZsq{}}\PYG{p}{]}\PYG{o}{.}\PYG{n}{rolling}\PYG{p}{(}\PYG{l+m+mi}{5}\PYG{p}{)}\PYG{o}{.}\PYG{n}{std}\PYG{p}{(}\PYG{p}{)} \PYG{c+c1}{\PYGZsh{} 직전 5일 종가의 표준편차}
\PYG{n}{df}\PYG{p}{[}\PYG{l+s+s1}{\PYGZsq{}}\PYG{l+s+s1}{price\PYGZus{}z}\PYG{l+s+s1}{\PYGZsq{}}\PYG{p}{]} \PYG{o}{=} \PYG{p}{(}\PYG{n}{df}\PYG{p}{[}\PYG{l+s+s1}{\PYGZsq{}}\PYG{l+s+s1}{close}\PYG{l+s+s1}{\PYGZsq{}}\PYG{p}{]} \PYG{o}{\PYGZhy{}} \PYG{n}{df}\PYG{p}{[}\PYG{l+s+s1}{\PYGZsq{}}\PYG{l+s+s1}{price\PYGZus{}mean}\PYG{l+s+s1}{\PYGZsq{}}\PYG{p}{]}\PYG{p}{)}\PYG{o}{/}\PYG{n}{df}\PYG{p}{[}\PYG{l+s+s1}{\PYGZsq{}}\PYG{l+s+s1}{price\PYGZus{}std}\PYG{l+s+s1}{\PYGZsq{}}\PYG{p}{]} \PYG{c+c1}{\PYGZsh{} 직전 5일 종가의 평균 및 표준편차 대비 오늘 종가의 위치}
\PYG{n}{df}\PYG{p}{[}\PYG{p}{[}\PYG{l+s+s1}{\PYGZsq{}}\PYG{l+s+s1}{close}\PYG{l+s+s1}{\PYGZsq{}}\PYG{p}{,}\PYG{l+s+s1}{\PYGZsq{}}\PYG{l+s+s1}{price\PYGZus{}mean}\PYG{l+s+s1}{\PYGZsq{}}\PYG{p}{,}\PYG{l+s+s1}{\PYGZsq{}}\PYG{l+s+s1}{price\PYGZus{}std}\PYG{l+s+s1}{\PYGZsq{}}\PYG{p}{,}\PYG{l+s+s1}{\PYGZsq{}}\PYG{l+s+s1}{price\PYGZus{}z}\PYG{l+s+s1}{\PYGZsq{}}\PYG{p}{]}\PYG{p}{]}\PYG{o}{.}\PYG{n}{head}\PYG{p}{(}\PYG{l+m+mi}{10}\PYG{p}{)}\PYG{o}{.}\PYG{n}{style}\PYG{o}{.}\PYG{n}{set\PYGZus{}table\PYGZus{}attributes}\PYG{p}{(}\PYG{l+s+s1}{\PYGZsq{}}\PYG{l+s+s1}{style=}\PYG{l+s+s1}{\PYGZdq{}}\PYG{l+s+s1}{font\PYGZhy{}size: 12px}\PYG{l+s+s1}{\PYGZdq{}}\PYG{l+s+s1}{\PYGZsq{}}\PYG{p}{)}
\end{sphinxVerbatim}

\end{sphinxuseclass}\end{sphinxVerbatimInput}
\begin{sphinxVerbatimOutput}

\begin{sphinxuseclass}{cell_output}
\begin{sphinxVerbatim}[commandchars=\\\{\}]
\PYGZlt{}pandas.io.formats.style.Styler at 0x2a0ce533040\PYGZgt{}
\end{sphinxVerbatim}

\end{sphinxuseclass}\end{sphinxVerbatimOutput}

\end{sphinxuseclass}
\sphinxAtStartPar
 전 20일로 비교 구간을 바꾸고 전 종목에 대하여 동일한 계산을 합니다. 그리고 그 결과를 data\_h1 에 담습니다.

\begin{sphinxuseclass}{cell}\begin{sphinxVerbatimInput}

\begin{sphinxuseclass}{cell_input}
\begin{sphinxVerbatim}[commandchars=\\\{\}]
\PYG{n}{kosdaq\PYGZus{}list} \PYG{o}{=} \PYG{n}{pd}\PYG{o}{.}\PYG{n}{read\PYGZus{}pickle}\PYG{p}{(}\PYG{l+s+s1}{\PYGZsq{}}\PYG{l+s+s1}{kosdaq\PYGZus{}list.pkl}\PYG{l+s+s1}{\PYGZsq{}}\PYG{p}{)}

\PYG{n}{data\PYGZus{}h1} \PYG{o}{=} \PYG{n}{pd}\PYG{o}{.}\PYG{n}{DataFrame}\PYG{p}{(}\PYG{p}{)}

\PYG{k}{for} \PYG{n}{code} \PYG{o+ow}{in} \PYG{n}{kosdaq\PYGZus{}list}\PYG{p}{[}\PYG{l+s+s1}{\PYGZsq{}}\PYG{l+s+s1}{code}\PYG{l+s+s1}{\PYGZsq{}}\PYG{p}{]}\PYG{p}{:}

    \PYG{n}{data} \PYG{o}{=} \PYG{n}{mdl\PYGZus{}data}\PYG{p}{[}\PYG{n}{mdl\PYGZus{}data}\PYG{p}{[}\PYG{l+s+s1}{\PYGZsq{}}\PYG{l+s+s1}{code}\PYG{l+s+s1}{\PYGZsq{}}\PYG{p}{]}\PYG{o}{==}\PYG{n}{code}\PYG{p}{]}\PYG{o}{.}\PYG{n}{sort\PYGZus{}index}\PYG{p}{(}\PYG{p}{)}\PYG{o}{.}\PYG{n}{copy}\PYG{p}{(}\PYG{p}{)}
    \PYG{n}{data}\PYG{p}{[}\PYG{l+s+s1}{\PYGZsq{}}\PYG{l+s+s1}{price\PYGZus{}mean}\PYG{l+s+s1}{\PYGZsq{}}\PYG{p}{]} \PYG{o}{=} \PYG{n}{data}\PYG{p}{[}\PYG{l+s+s1}{\PYGZsq{}}\PYG{l+s+s1}{close}\PYG{l+s+s1}{\PYGZsq{}}\PYG{p}{]}\PYG{o}{.}\PYG{n}{rolling}\PYG{p}{(}\PYG{l+m+mi}{20}\PYG{p}{)}\PYG{o}{.}\PYG{n}{mean}\PYG{p}{(}\PYG{p}{)} \PYG{c+c1}{\PYGZsh{} 전 20일 평균}
    \PYG{n}{data}\PYG{p}{[}\PYG{l+s+s1}{\PYGZsq{}}\PYG{l+s+s1}{price\PYGZus{}std}\PYG{l+s+s1}{\PYGZsq{}}\PYG{p}{]} \PYG{o}{=} \PYG{n}{data}\PYG{p}{[}\PYG{l+s+s1}{\PYGZsq{}}\PYG{l+s+s1}{close}\PYG{l+s+s1}{\PYGZsq{}}\PYG{p}{]}\PYG{o}{.}\PYG{n}{rolling}\PYG{p}{(}\PYG{l+m+mi}{20}\PYG{p}{)}\PYG{o}{.}\PYG{n}{std}\PYG{p}{(}\PYG{n}{ddof}\PYG{o}{=}\PYG{l+m+mi}{0}\PYG{p}{)} \PYG{c+c1}{\PYGZsh{} 전 20일 표준편차}
    \PYG{n}{data}\PYG{p}{[}\PYG{l+s+s1}{\PYGZsq{}}\PYG{l+s+s1}{price\PYGZus{}z}\PYG{l+s+s1}{\PYGZsq{}}\PYG{p}{]} \PYG{o}{=} \PYG{p}{(}\PYG{n}{data}\PYG{p}{[}\PYG{l+s+s1}{\PYGZsq{}}\PYG{l+s+s1}{close}\PYG{l+s+s1}{\PYGZsq{}}\PYG{p}{]} \PYG{o}{\PYGZhy{}} \PYG{n}{data}\PYG{p}{[}\PYG{l+s+s1}{\PYGZsq{}}\PYG{l+s+s1}{price\PYGZus{}mean}\PYG{l+s+s1}{\PYGZsq{}}\PYG{p}{]}\PYG{p}{)}\PYG{o}{/}\PYG{n}{data}\PYG{p}{[}\PYG{l+s+s1}{\PYGZsq{}}\PYG{l+s+s1}{price\PYGZus{}std}\PYG{l+s+s1}{\PYGZsq{}}\PYG{p}{]}  \PYG{c+c1}{\PYGZsh{} 표준화된 Z 값 생성  }
    
    \PYG{n}{data}\PYG{p}{[}\PYG{l+s+s1}{\PYGZsq{}}\PYG{l+s+s1}{volume\PYGZus{}mean}\PYG{l+s+s1}{\PYGZsq{}}\PYG{p}{]} \PYG{o}{=} \PYG{n}{data}\PYG{p}{[}\PYG{l+s+s1}{\PYGZsq{}}\PYG{l+s+s1}{volume}\PYG{l+s+s1}{\PYGZsq{}}\PYG{p}{]}\PYG{o}{.}\PYG{n}{rolling}\PYG{p}{(}\PYG{l+m+mi}{20}\PYG{p}{)}\PYG{o}{.}\PYG{n}{mean}\PYG{p}{(}\PYG{p}{)} \PYG{c+c1}{\PYGZsh{} 전 20일 평균}
    \PYG{n}{data}\PYG{p}{[}\PYG{l+s+s1}{\PYGZsq{}}\PYG{l+s+s1}{volume\PYGZus{}std}\PYG{l+s+s1}{\PYGZsq{}}\PYG{p}{]} \PYG{o}{=} \PYG{n}{data}\PYG{p}{[}\PYG{l+s+s1}{\PYGZsq{}}\PYG{l+s+s1}{volume}\PYG{l+s+s1}{\PYGZsq{}}\PYG{p}{]}\PYG{o}{.}\PYG{n}{rolling}\PYG{p}{(}\PYG{l+m+mi}{20}\PYG{p}{)}\PYG{o}{.}\PYG{n}{std}\PYG{p}{(}\PYG{n}{ddof}\PYG{o}{=}\PYG{l+m+mi}{0}\PYG{p}{)} \PYG{c+c1}{\PYGZsh{} 전 20일 표준편차}
    \PYG{n}{data}\PYG{p}{[}\PYG{l+s+s1}{\PYGZsq{}}\PYG{l+s+s1}{volume\PYGZus{}z}\PYG{l+s+s1}{\PYGZsq{}}\PYG{p}{]} \PYG{o}{=} \PYG{p}{(}\PYG{n}{data}\PYG{p}{[}\PYG{l+s+s1}{\PYGZsq{}}\PYG{l+s+s1}{volume}\PYG{l+s+s1}{\PYGZsq{}}\PYG{p}{]} \PYG{o}{\PYGZhy{}} \PYG{n}{data}\PYG{p}{[}\PYG{l+s+s1}{\PYGZsq{}}\PYG{l+s+s1}{volume\PYGZus{}mean}\PYG{l+s+s1}{\PYGZsq{}}\PYG{p}{]}\PYG{p}{)}\PYG{o}{/}\PYG{n}{data}\PYG{p}{[}\PYG{l+s+s1}{\PYGZsq{}}\PYG{l+s+s1}{volume\PYGZus{}std}\PYG{l+s+s1}{\PYGZsq{}}\PYG{p}{]}  \PYG{c+c1}{\PYGZsh{} 표준화된 Z 값 생성  }
       
    \PYG{n}{data}\PYG{p}{[}\PYG{l+s+s1}{\PYGZsq{}}\PYG{l+s+s1}{max\PYGZus{}close}\PYG{l+s+s1}{\PYGZsq{}}\PYG{p}{]}  \PYG{o}{=} \PYG{n}{data}\PYG{p}{[}\PYG{p}{[}\PYG{l+s+s1}{\PYGZsq{}}\PYG{l+s+s1}{close\PYGZus{}r1}\PYG{l+s+s1}{\PYGZsq{}}\PYG{p}{,}\PYG{l+s+s1}{\PYGZsq{}}\PYG{l+s+s1}{close\PYGZus{}r2}\PYG{l+s+s1}{\PYGZsq{}}\PYG{p}{,}\PYG{l+s+s1}{\PYGZsq{}}\PYG{l+s+s1}{close\PYGZus{}r3}\PYG{l+s+s1}{\PYGZsq{}}\PYG{p}{,}\PYG{l+s+s1}{\PYGZsq{}}\PYG{l+s+s1}{close\PYGZus{}r4}\PYG{l+s+s1}{\PYGZsq{}}\PYG{p}{,}\PYG{l+s+s1}{\PYGZsq{}}\PYG{l+s+s1}{close\PYGZus{}r5}\PYG{l+s+s1}{\PYGZsq{}}\PYG{p}{]}\PYG{p}{]}\PYG{o}{.}\PYG{n}{max}\PYG{p}{(}\PYG{n}{axis}\PYG{o}{=}\PYG{l+m+mi}{1}\PYG{p}{)} \PYG{c+c1}{\PYGZsh{} 5 영업일 종가 수익율 중 최고 값}
    \PYG{n}{data}\PYG{o}{.}\PYG{n}{dropna}\PYG{p}{(}\PYG{n}{subset}\PYG{o}{=}\PYG{p}{[}\PYG{l+s+s1}{\PYGZsq{}}\PYG{l+s+s1}{price\PYGZus{}z}\PYG{l+s+s1}{\PYGZsq{}}\PYG{p}{,}\PYG{l+s+s1}{\PYGZsq{}}\PYG{l+s+s1}{volume\PYGZus{}z}\PYG{l+s+s1}{\PYGZsq{}}\PYG{p}{,}\PYG{l+s+s1}{\PYGZsq{}}\PYG{l+s+s1}{close\PYGZus{}r1}\PYG{l+s+s1}{\PYGZsq{}}\PYG{p}{,}\PYG{l+s+s1}{\PYGZsq{}}\PYG{l+s+s1}{close\PYGZus{}r2}\PYG{l+s+s1}{\PYGZsq{}}\PYG{p}{,}\PYG{l+s+s1}{\PYGZsq{}}\PYG{l+s+s1}{close\PYGZus{}r3}\PYG{l+s+s1}{\PYGZsq{}}\PYG{p}{,}\PYG{l+s+s1}{\PYGZsq{}}\PYG{l+s+s1}{close\PYGZus{}r4}\PYG{l+s+s1}{\PYGZsq{}}\PYG{p}{,}\PYG{l+s+s1}{\PYGZsq{}}\PYG{l+s+s1}{close\PYGZus{}r5}\PYG{l+s+s1}{\PYGZsq{}}\PYG{p}{]}\PYG{p}{,} \PYG{n}{inplace}\PYG{o}{=}\PYG{k+kc}{True}\PYG{p}{)} \PYG{c+c1}{\PYGZsh{} missing 이 있는 행은 제거  }
    
    \PYG{n}{data} \PYG{o}{=} \PYG{n}{data}\PYG{p}{[}\PYG{p}{(}\PYG{n}{data}\PYG{p}{[}\PYG{l+s+s1}{\PYGZsq{}}\PYG{l+s+s1}{price\PYGZus{}std}\PYG{l+s+s1}{\PYGZsq{}}\PYG{p}{]}\PYG{o}{!=}\PYG{l+m+mi}{0}\PYG{p}{)} \PYG{o}{\PYGZam{}} \PYG{p}{(}\PYG{n}{data}\PYG{p}{[}\PYG{l+s+s1}{\PYGZsq{}}\PYG{l+s+s1}{volume\PYGZus{}std}\PYG{l+s+s1}{\PYGZsq{}}\PYG{p}{]}\PYG{o}{!=}\PYG{l+m+mi}{0}\PYG{p}{)}\PYG{p}{]} \PYG{c+c1}{\PYGZsh{} 0 으로 나누는 상황은 없도록 함.}
    
    \PYG{n}{data\PYGZus{}h1} \PYG{o}{=} \PYG{n}{pd}\PYG{o}{.}\PYG{n}{concat}\PYG{p}{(}\PYG{p}{[}\PYG{n}{data}\PYG{p}{,} \PYG{n}{data\PYGZus{}h1}\PYG{p}{]}\PYG{p}{,} \PYG{n}{axis}\PYG{o}{=}\PYG{l+m+mi}{0}\PYG{p}{)}

\PYG{n}{data\PYGZus{}h1}\PYG{o}{.}\PYG{n}{to\PYGZus{}pickle}\PYG{p}{(}\PYG{l+s+s1}{\PYGZsq{}}\PYG{l+s+s1}{data\PYGZus{}h1.pkl}\PYG{l+s+s1}{\PYGZsq{}}\PYG{p}{)}  
\end{sphinxVerbatim}

\end{sphinxuseclass}\end{sphinxVerbatimInput}

\end{sphinxuseclass}
\begin{sphinxuseclass}{cell}\begin{sphinxVerbatimInput}

\begin{sphinxuseclass}{cell_input}
\begin{sphinxVerbatim}[commandchars=\\\{\}]
\PYG{n}{data\PYGZus{}h1} \PYG{o}{=} \PYG{n}{pd}\PYG{o}{.}\PYG{n}{read\PYGZus{}pickle}\PYG{p}{(}\PYG{l+s+s1}{\PYGZsq{}}\PYG{l+s+s1}{data\PYGZus{}h1.pkl}\PYG{l+s+s1}{\PYGZsq{}}\PYG{p}{)}  
\PYG{n+nb}{print}\PYG{p}{(}\PYG{n}{data\PYGZus{}h1}\PYG{p}{[}\PYG{l+s+s1}{\PYGZsq{}}\PYG{l+s+s1}{price\PYGZus{}z}\PYG{l+s+s1}{\PYGZsq{}}\PYG{p}{]}\PYG{o}{.}\PYG{n}{agg}\PYG{p}{(}\PYG{p}{[}\PYG{l+s+s1}{\PYGZsq{}}\PYG{l+s+s1}{min}\PYG{l+s+s1}{\PYGZsq{}}\PYG{p}{,}\PYG{l+s+s1}{\PYGZsq{}}\PYG{l+s+s1}{max}\PYG{l+s+s1}{\PYGZsq{}}\PYG{p}{]}\PYG{p}{)}\PYG{p}{)} \PYG{c+c1}{\PYGZsh{} 최소값과 최대값을 확인함}
\PYG{n+nb}{print}\PYG{p}{(}\PYG{n}{data\PYGZus{}h1}\PYG{p}{[}\PYG{l+s+s1}{\PYGZsq{}}\PYG{l+s+s1}{volume\PYGZus{}z}\PYG{l+s+s1}{\PYGZsq{}}\PYG{p}{]}\PYG{o}{.}\PYG{n}{agg}\PYG{p}{(}\PYG{p}{[}\PYG{l+s+s1}{\PYGZsq{}}\PYG{l+s+s1}{min}\PYG{l+s+s1}{\PYGZsq{}}\PYG{p}{,}\PYG{l+s+s1}{\PYGZsq{}}\PYG{l+s+s1}{max}\PYG{l+s+s1}{\PYGZsq{}}\PYG{p}{]}\PYG{p}{)}\PYG{p}{)}
\end{sphinxVerbatim}

\end{sphinxuseclass}\end{sphinxVerbatimInput}
\begin{sphinxVerbatimOutput}

\begin{sphinxuseclass}{cell_output}
\begin{sphinxVerbatim}[commandchars=\\\{\}]
min   \PYGZhy{}4.359
max    4.359
Name: price\PYGZus{}z, dtype: float64
min   \PYGZhy{}2.568
max    4.359
Name: volume\PYGZus{}z, dtype: float64
\end{sphinxVerbatim}

\end{sphinxuseclass}\end{sphinxVerbatimOutput}

\end{sphinxuseclass}
\sphinxAtStartPar
 price\_z 에 따른 종가 최고 수익률의 변화를 확인합니다. 최근 20일 종가의 평균 대비 오늘 종가가 낮거나 높은 경우 좋은 수익률을 기대할 수 있습니다.

\begin{sphinxuseclass}{cell}\begin{sphinxVerbatimInput}

\begin{sphinxuseclass}{cell_input}
\begin{sphinxVerbatim}[commandchars=\\\{\}]
\PYG{n}{rank} \PYG{o}{=} \PYG{n}{pd}\PYG{o}{.}\PYG{n}{qcut}\PYG{p}{(}\PYG{n}{data\PYGZus{}h1}\PYG{p}{[}\PYG{l+s+s1}{\PYGZsq{}}\PYG{l+s+s1}{price\PYGZus{}z}\PYG{l+s+s1}{\PYGZsq{}}\PYG{p}{]}\PYG{p}{,} \PYG{n}{q}\PYG{o}{=}\PYG{l+m+mi}{10}\PYG{p}{,} \PYG{n}{labels}\PYG{o}{=}\PYG{n+nb}{range}\PYG{p}{(}\PYG{l+m+mi}{10}\PYG{p}{)}\PYG{p}{)}
\PYG{n}{data\PYGZus{}h1}\PYG{o}{.}\PYG{n}{groupby}\PYG{p}{(}\PYG{n}{rank}\PYG{p}{)}\PYG{p}{[}\PYG{l+s+s1}{\PYGZsq{}}\PYG{l+s+s1}{max\PYGZus{}close}\PYG{l+s+s1}{\PYGZsq{}}\PYG{p}{]}\PYG{o}{.}\PYG{n}{mean}\PYG{p}{(}\PYG{p}{)}\PYG{o}{.}\PYG{n}{plot}\PYG{p}{(}\PYG{p}{)}
\end{sphinxVerbatim}

\end{sphinxuseclass}\end{sphinxVerbatimInput}
\begin{sphinxVerbatimOutput}

\begin{sphinxuseclass}{cell_output}
\begin{sphinxVerbatim}[commandchars=\\\{\}]
\PYGZlt{}AxesSubplot:xlabel=\PYGZsq{}price\PYGZus{}z\PYGZsq{}\PYGZgt{}
\end{sphinxVerbatim}

\noindent\sphinxincludegraphics{{5.1.1_Hypothesis_1_9_1}.png}

\end{sphinxuseclass}\end{sphinxVerbatimOutput}

\end{sphinxuseclass}
\sphinxAtStartPar
최근 20일 대비 거래량이 많을 수 록 더 좋은 수익률을 기대할 수 있습니다.

\begin{sphinxuseclass}{cell}\begin{sphinxVerbatimInput}

\begin{sphinxuseclass}{cell_input}
\begin{sphinxVerbatim}[commandchars=\\\{\}]
\PYG{n}{rank} \PYG{o}{=} \PYG{n}{pd}\PYG{o}{.}\PYG{n}{qcut}\PYG{p}{(}\PYG{n}{data\PYGZus{}h1}\PYG{p}{[}\PYG{l+s+s1}{\PYGZsq{}}\PYG{l+s+s1}{volume\PYGZus{}z}\PYG{l+s+s1}{\PYGZsq{}}\PYG{p}{]}\PYG{p}{,} \PYG{n}{q}\PYG{o}{=}\PYG{l+m+mi}{10}\PYG{p}{,} \PYG{n}{labels}\PYG{o}{=}\PYG{n+nb}{range}\PYG{p}{(}\PYG{l+m+mi}{10}\PYG{p}{)}\PYG{p}{)}
\PYG{n}{data\PYGZus{}h1}\PYG{o}{.}\PYG{n}{groupby}\PYG{p}{(}\PYG{n}{rank}\PYG{p}{)}\PYG{p}{[}\PYG{l+s+s1}{\PYGZsq{}}\PYG{l+s+s1}{max\PYGZus{}close}\PYG{l+s+s1}{\PYGZsq{}}\PYG{p}{]}\PYG{o}{.}\PYG{n}{mean}\PYG{p}{(}\PYG{p}{)}\PYG{o}{.}\PYG{n}{plot}\PYG{p}{(}\PYG{p}{)}
\end{sphinxVerbatim}

\end{sphinxuseclass}\end{sphinxVerbatimInput}
\begin{sphinxVerbatimOutput}

\begin{sphinxuseclass}{cell_output}
\begin{sphinxVerbatim}[commandchars=\\\{\}]
\PYGZlt{}AxesSubplot:xlabel=\PYGZsq{}volume\PYGZus{}z\PYGZsq{}\PYGZgt{}
\end{sphinxVerbatim}

\noindent\sphinxincludegraphics{{5.1.1_Hypothesis_1_11_1}.png}

\end{sphinxuseclass}\end{sphinxVerbatimOutput}

\end{sphinxuseclass}
\sphinxAtStartPar
 종가의 표준화 값 price\_z 와 거래량의 표준화 값 volume\_z 이 서로 직교하는 테이블로 구성하고 평균 수익율을 보니, 가격이 변동성이 높고, 거래량이 몰리는 종목은 평균 수익율이 더 높다는 것이 확인되었습니다.

\begin{sphinxuseclass}{cell}\begin{sphinxVerbatimInput}

\begin{sphinxuseclass}{cell_input}
\begin{sphinxVerbatim}[commandchars=\\\{\}]
\PYG{n}{rank1}  \PYG{o}{=} \PYG{n}{pd}\PYG{o}{.}\PYG{n}{qcut}\PYG{p}{(}\PYG{n}{data\PYGZus{}h1}\PYG{p}{[}\PYG{l+s+s1}{\PYGZsq{}}\PYG{l+s+s1}{price\PYGZus{}z}\PYG{l+s+s1}{\PYGZsq{}}\PYG{p}{]}\PYG{p}{,} \PYG{n}{q}\PYG{o}{=}\PYG{l+m+mi}{5}\PYG{p}{,} \PYG{n}{labels}\PYG{o}{=}\PYG{n+nb}{range}\PYG{p}{(}\PYG{l+m+mi}{5}\PYG{p}{)}\PYG{p}{)}
\PYG{n}{rank2}  \PYG{o}{=} \PYG{n}{pd}\PYG{o}{.}\PYG{n}{qcut}\PYG{p}{(}\PYG{n}{data\PYGZus{}h1}\PYG{p}{[}\PYG{l+s+s1}{\PYGZsq{}}\PYG{l+s+s1}{volume\PYGZus{}z}\PYG{l+s+s1}{\PYGZsq{}}\PYG{p}{]}\PYG{p}{,} \PYG{n}{q}\PYG{o}{=}\PYG{l+m+mi}{5}\PYG{p}{,} \PYG{n}{labels}\PYG{o}{=}\PYG{n+nb}{range}\PYG{p}{(}\PYG{l+m+mi}{5}\PYG{p}{)}\PYG{p}{)}

\PYG{n}{data\PYGZus{}h1}\PYG{o}{.}\PYG{n}{groupby}\PYG{p}{(}\PYG{p}{[}\PYG{n}{rank1}\PYG{p}{,} \PYG{n}{rank2}\PYG{p}{]}\PYG{p}{)}\PYG{p}{[}\PYG{l+s+s1}{\PYGZsq{}}\PYG{l+s+s1}{max\PYGZus{}close}\PYG{l+s+s1}{\PYGZsq{}}\PYG{p}{]}\PYG{o}{.}\PYG{n}{mean}\PYG{p}{(}\PYG{p}{)}\PYG{o}{.}\PYG{n}{unstack}\PYG{p}{(}\PYG{p}{)}\PYG{o}{.}\PYG{n}{style}\PYG{o}{.}\PYG{n}{set\PYGZus{}table\PYGZus{}attributes}\PYG{p}{(}\PYG{l+s+s1}{\PYGZsq{}}\PYG{l+s+s1}{style=}\PYG{l+s+s1}{\PYGZdq{}}\PYG{l+s+s1}{font\PYGZhy{}size: 12px}\PYG{l+s+s1}{\PYGZdq{}}\PYG{l+s+s1}{\PYGZsq{}}\PYG{p}{)}
\end{sphinxVerbatim}

\end{sphinxuseclass}\end{sphinxVerbatimInput}
\begin{sphinxVerbatimOutput}

\begin{sphinxuseclass}{cell_output}
\begin{sphinxVerbatim}[commandchars=\\\{\}]
\PYGZlt{}pandas.io.formats.style.Styler at 0x2e6c0df4bb0\PYGZgt{}
\end{sphinxVerbatim}

\end{sphinxuseclass}\end{sphinxVerbatimOutput}

\end{sphinxuseclass}
\begin{sphinxuseclass}{cell}\begin{sphinxVerbatimInput}

\begin{sphinxuseclass}{cell_input}
\begin{sphinxVerbatim}[commandchars=\\\{\}]
\PYG{k+kn}{import} \PYG{n+nn}{FinanceDataReader} \PYG{k}{as} \PYG{n+nn}{fdr}
\PYG{o}{\PYGZpc{}}\PYG{k}{matplotlib} inline
\PYG{k+kn}{import} \PYG{n+nn}{matplotlib}\PYG{n+nn}{.}\PYG{n+nn}{pyplot} \PYG{k}{as} \PYG{n+nn}{plt}
\PYG{k+kn}{import} \PYG{n+nn}{pandas} \PYG{k}{as} \PYG{n+nn}{pd}
\PYG{k+kn}{import} \PYG{n+nn}{numpy} \PYG{k}{as} \PYG{n+nn}{np}
\PYG{k+kn}{import} \PYG{n+nn}{warnings}
\PYG{n}{warnings}\PYG{o}{.}\PYG{n}{filterwarnings}\PYG{p}{(}\PYG{l+s+s1}{\PYGZsq{}}\PYG{l+s+s1}{ignore}\PYG{l+s+s1}{\PYGZsq{}}\PYG{p}{)}

\PYG{n}{pd}\PYG{o}{.}\PYG{n}{options}\PYG{o}{.}\PYG{n}{display}\PYG{o}{.}\PYG{n}{float\PYGZus{}format} \PYG{o}{=} \PYG{l+s+s1}{\PYGZsq{}}\PYG{l+s+si}{\PYGZob{}:,.3f\PYGZcb{}}\PYG{l+s+s1}{\PYGZsq{}}\PYG{o}{.}\PYG{n}{format}
\end{sphinxVerbatim}

\end{sphinxuseclass}\end{sphinxVerbatimInput}

\end{sphinxuseclass}

\section{5일 이동 평균선이 오늘 종가보다 위에 위치해 있다.}
\label{\detokenize{chapter5/5.1.2_Hypothesis_2:id1}}\label{\detokenize{chapter5/5.1.2_Hypothesis_2::doc}}
\sphinxAtStartPar
rolling(5) 을 이용하여 이동평균선을 만듭니다. 그리고 당일의 종가보다 크면, 1 아니면 0 인 변수 ‘flag’ 을 생성합니다. 이 가설은 검증이 쉬운 것 같습니다.

\begin{sphinxuseclass}{cell}\begin{sphinxVerbatimInput}

\begin{sphinxuseclass}{cell_input}
\begin{sphinxVerbatim}[commandchars=\\\{\}]
\PYG{n}{mdl\PYGZus{}data} \PYG{o}{=} \PYG{n}{pd}\PYG{o}{.}\PYG{n}{read\PYGZus{}pickle}\PYG{p}{(}\PYG{l+s+s1}{\PYGZsq{}}\PYG{l+s+s1}{mdl\PYGZus{}data.pkl}\PYG{l+s+s1}{\PYGZsq{}}\PYG{p}{)} \PYG{c+c1}{\PYGZsh{} 수익률 결과가 있는 데이터}
\PYG{n}{mdl\PYGZus{}data}\PYG{o}{.}\PYG{n}{head}\PYG{p}{(}\PYG{p}{)}\PYG{o}{.}\PYG{n}{style}\PYG{o}{.}\PYG{n}{set\PYGZus{}table\PYGZus{}attributes}\PYG{p}{(}\PYG{l+s+s1}{\PYGZsq{}}\PYG{l+s+s1}{style=}\PYG{l+s+s1}{\PYGZdq{}}\PYG{l+s+s1}{font\PYGZhy{}size: 12px}\PYG{l+s+s1}{\PYGZdq{}}\PYG{l+s+s1}{\PYGZsq{}}\PYG{p}{)}
\end{sphinxVerbatim}

\end{sphinxuseclass}\end{sphinxVerbatimInput}
\begin{sphinxVerbatimOutput}

\begin{sphinxuseclass}{cell_output}
\begin{sphinxVerbatim}[commandchars=\\\{\}]
\PYGZlt{}pandas.io.formats.style.Styler at 0x1d3829c5580\PYGZgt{}
\end{sphinxVerbatim}

\end{sphinxuseclass}\end{sphinxVerbatimOutput}

\end{sphinxuseclass}
\begin{sphinxuseclass}{cell}\begin{sphinxVerbatimInput}

\begin{sphinxuseclass}{cell_input}
\begin{sphinxVerbatim}[commandchars=\\\{\}]
\PYG{n}{kosdaq\PYGZus{}list} \PYG{o}{=} \PYG{n}{pd}\PYG{o}{.}\PYG{n}{read\PYGZus{}pickle}\PYG{p}{(}\PYG{l+s+s1}{\PYGZsq{}}\PYG{l+s+s1}{kosdaq\PYGZus{}list.pkl}\PYG{l+s+s1}{\PYGZsq{}}\PYG{p}{)}

\PYG{n}{data\PYGZus{}h2} \PYG{o}{=} \PYG{n}{pd}\PYG{o}{.}\PYG{n}{DataFrame}\PYG{p}{(}\PYG{p}{)}

\PYG{k}{for} \PYG{n}{code} \PYG{o+ow}{in} \PYG{n}{kosdaq\PYGZus{}list}\PYG{p}{[}\PYG{l+s+s1}{\PYGZsq{}}\PYG{l+s+s1}{code}\PYG{l+s+s1}{\PYGZsq{}}\PYG{p}{]}\PYG{p}{:}

    \PYG{n}{data} \PYG{o}{=} \PYG{n}{mdl\PYGZus{}data}\PYG{p}{[}\PYG{n}{mdl\PYGZus{}data}\PYG{p}{[}\PYG{l+s+s1}{\PYGZsq{}}\PYG{l+s+s1}{code}\PYG{l+s+s1}{\PYGZsq{}}\PYG{p}{]}\PYG{o}{==}\PYG{n}{code}\PYG{p}{]}\PYG{o}{.}\PYG{n}{sort\PYGZus{}index}\PYG{p}{(}\PYG{p}{)}\PYG{o}{.}\PYG{n}{copy}\PYG{p}{(}\PYG{p}{)}
    
    \PYG{n}{data}\PYG{p}{[}\PYG{l+s+s1}{\PYGZsq{}}\PYG{l+s+s1}{5day\PYGZus{}ma}\PYG{l+s+s1}{\PYGZsq{}}\PYG{p}{]} \PYG{o}{=} \PYG{n}{data}\PYG{p}{[}\PYG{l+s+s1}{\PYGZsq{}}\PYG{l+s+s1}{close}\PYG{l+s+s1}{\PYGZsq{}}\PYG{p}{]}\PYG{o}{.}\PYG{n}{rolling}\PYG{p}{(}\PYG{l+m+mi}{5}\PYG{p}{)}\PYG{o}{.}\PYG{n}{mean}\PYG{p}{(}\PYG{p}{)} \PYG{c+c1}{\PYGZsh{} 5일 이동평균선}
    \PYG{n}{data}\PYG{p}{[}\PYG{l+s+s1}{\PYGZsq{}}\PYG{l+s+s1}{flag}\PYG{l+s+s1}{\PYGZsq{}}\PYG{p}{]} \PYG{o}{=} \PYG{p}{(}\PYG{n}{data}\PYG{p}{[}\PYG{l+s+s1}{\PYGZsq{}}\PYG{l+s+s1}{close}\PYG{l+s+s1}{\PYGZsq{}}\PYG{p}{]} \PYG{o}{\PYGZlt{}} \PYG{n}{data}\PYG{p}{[}\PYG{l+s+s1}{\PYGZsq{}}\PYG{l+s+s1}{5day\PYGZus{}ma}\PYG{l+s+s1}{\PYGZsq{}}\PYG{p}{]}\PYG{p}{)}\PYG{o}{.}\PYG{n}{astype}\PYG{p}{(}\PYG{n+nb}{int}\PYG{p}{)} \PYG{c+c1}{\PYGZsh{} 5일 이동평균선이 종가보다 크면 1, 아니면 0}

       
    \PYG{n}{data}\PYG{p}{[}\PYG{l+s+s1}{\PYGZsq{}}\PYG{l+s+s1}{max\PYGZus{}close}\PYG{l+s+s1}{\PYGZsq{}}\PYG{p}{]}  \PYG{o}{=} \PYG{n}{data}\PYG{p}{[}\PYG{p}{[}\PYG{l+s+s1}{\PYGZsq{}}\PYG{l+s+s1}{close\PYGZus{}r1}\PYG{l+s+s1}{\PYGZsq{}}\PYG{p}{,}\PYG{l+s+s1}{\PYGZsq{}}\PYG{l+s+s1}{close\PYGZus{}r2}\PYG{l+s+s1}{\PYGZsq{}}\PYG{p}{,}\PYG{l+s+s1}{\PYGZsq{}}\PYG{l+s+s1}{close\PYGZus{}r3}\PYG{l+s+s1}{\PYGZsq{}}\PYG{p}{,}\PYG{l+s+s1}{\PYGZsq{}}\PYG{l+s+s1}{close\PYGZus{}r4}\PYG{l+s+s1}{\PYGZsq{}}\PYG{p}{,}\PYG{l+s+s1}{\PYGZsq{}}\PYG{l+s+s1}{close\PYGZus{}r5}\PYG{l+s+s1}{\PYGZsq{}}\PYG{p}{]}\PYG{p}{]}\PYG{o}{.}\PYG{n}{max}\PYG{p}{(}\PYG{n}{axis}\PYG{o}{=}\PYG{l+m+mi}{1}\PYG{p}{)} \PYG{c+c1}{\PYGZsh{} 5 영업일 종가 수익율 중 최고 값}
    \PYG{n}{data}\PYG{o}{.}\PYG{n}{dropna}\PYG{p}{(}\PYG{n}{subset}\PYG{o}{=}\PYG{p}{[}\PYG{l+s+s1}{\PYGZsq{}}\PYG{l+s+s1}{5day\PYGZus{}ma}\PYG{l+s+s1}{\PYGZsq{}}\PYG{p}{,}\PYG{l+s+s1}{\PYGZsq{}}\PYG{l+s+s1}{close\PYGZus{}r1}\PYG{l+s+s1}{\PYGZsq{}}\PYG{p}{,}\PYG{l+s+s1}{\PYGZsq{}}\PYG{l+s+s1}{close\PYGZus{}r2}\PYG{l+s+s1}{\PYGZsq{}}\PYG{p}{,}\PYG{l+s+s1}{\PYGZsq{}}\PYG{l+s+s1}{close\PYGZus{}r3}\PYG{l+s+s1}{\PYGZsq{}}\PYG{p}{,}\PYG{l+s+s1}{\PYGZsq{}}\PYG{l+s+s1}{close\PYGZus{}r4}\PYG{l+s+s1}{\PYGZsq{}}\PYG{p}{,}\PYG{l+s+s1}{\PYGZsq{}}\PYG{l+s+s1}{close\PYGZus{}r5}\PYG{l+s+s1}{\PYGZsq{}}\PYG{p}{]}\PYG{p}{,} \PYG{n}{inplace}\PYG{o}{=}\PYG{k+kc}{True}\PYG{p}{)} \PYG{c+c1}{\PYGZsh{} missing 이 있는 행은 제거  }
    
    \PYG{n}{data\PYGZus{}h2} \PYG{o}{=} \PYG{n}{pd}\PYG{o}{.}\PYG{n}{concat}\PYG{p}{(}\PYG{p}{[}\PYG{n}{data}\PYG{p}{,} \PYG{n}{data\PYGZus{}h2}\PYG{p}{]}\PYG{p}{,} \PYG{n}{axis}\PYG{o}{=}\PYG{l+m+mi}{0}\PYG{p}{)}

\PYG{n}{data\PYGZus{}h2}\PYG{o}{.}\PYG{n}{to\PYGZus{}pickle}\PYG{p}{(}\PYG{l+s+s1}{\PYGZsq{}}\PYG{l+s+s1}{data\PYGZus{}h2.pkl}\PYG{l+s+s1}{\PYGZsq{}}\PYG{p}{)}  
\end{sphinxVerbatim}

\end{sphinxuseclass}\end{sphinxVerbatimInput}

\end{sphinxuseclass}
\sphinxAtStartPar
 ‘flag’ 가 0 인 경우와 1 인 경우를 비교해보니 이 가설은 데이터가 강하게 뒷받침하지 못하고 있습니다.

\begin{sphinxuseclass}{cell}\begin{sphinxVerbatimInput}

\begin{sphinxuseclass}{cell_input}
\begin{sphinxVerbatim}[commandchars=\\\{\}]
\PYG{n}{data\PYGZus{}h2} \PYG{o}{=} \PYG{n}{pd}\PYG{o}{.}\PYG{n}{read\PYGZus{}pickle}\PYG{p}{(}\PYG{l+s+s1}{\PYGZsq{}}\PYG{l+s+s1}{data\PYGZus{}h2.pkl}\PYG{l+s+s1}{\PYGZsq{}}\PYG{p}{)}
\PYG{n}{data\PYGZus{}h2}\PYG{o}{.}\PYG{n}{groupby}\PYG{p}{(}\PYG{l+s+s1}{\PYGZsq{}}\PYG{l+s+s1}{flag}\PYG{l+s+s1}{\PYGZsq{}}\PYG{p}{)}\PYG{p}{[}\PYG{l+s+s1}{\PYGZsq{}}\PYG{l+s+s1}{max\PYGZus{}close}\PYG{l+s+s1}{\PYGZsq{}}\PYG{p}{]}\PYG{o}{.}\PYG{n}{describe}\PYG{p}{(}\PYG{p}{)}\PYG{o}{.}\PYG{n}{style}\PYG{o}{.}\PYG{n}{set\PYGZus{}table\PYGZus{}attributes}\PYG{p}{(}\PYG{l+s+s1}{\PYGZsq{}}\PYG{l+s+s1}{style=}\PYG{l+s+s1}{\PYGZdq{}}\PYG{l+s+s1}{font\PYGZhy{}size: 12px}\PYG{l+s+s1}{\PYGZdq{}}\PYG{l+s+s1}{\PYGZsq{}}\PYG{p}{)}\PYG{o}{.}\PYG{n}{format}\PYG{p}{(}\PYG{n}{precision}\PYG{o}{=}\PYG{l+m+mi}{3}\PYG{p}{)}
\end{sphinxVerbatim}

\end{sphinxuseclass}\end{sphinxVerbatimInput}
\begin{sphinxVerbatimOutput}

\begin{sphinxuseclass}{cell_output}
\begin{sphinxVerbatim}[commandchars=\\\{\}]
\PYGZlt{}pandas.io.formats.style.Styler at 0x1d388bc3a30\PYGZgt{}
\end{sphinxVerbatim}

\end{sphinxuseclass}\end{sphinxVerbatimOutput}

\end{sphinxuseclass}
\sphinxAtStartPar
T\sphinxhyphen{}Test 를 해보겠습니다. T\sphinxhyphen{}Test 는 두 집단의 평균이 서로 유의미하게 다른 지 확인하는 검정입니다. 귀무가설이 “두 집단의 평균이 같다” 이기 때문에,  p \sphinxhyphen{}value 가 유의수준(0.01) 보다 작으면 귀무가설을 기각합니다. 결과를 보니 P\sphinxhyphen{}Value 가 유의수준(0.01) 보다 큽니다. 따라서 귀무가설을 기각할 수 없습니다. 즉, flag 가 0 인 집단과 1 인 집단간의 차가 유의미하지 않은 것으로 판단됩니다. 왜 각 집단에서 샘플을 200 개만 뽑아서 테스트를 하는 지 궁금한 독자도 있으실 것 같습니다. 통계 검정은 샘플의 수가 많아지면 p value 가 작게 나오는 경향이 있습니다. 그렇게 되면 유의미하게 차이가 없는데도, 서로 다르다고 통계 결과가 나오게됩니다.

\begin{sphinxuseclass}{cell}\begin{sphinxVerbatimInput}

\begin{sphinxuseclass}{cell_input}
\begin{sphinxVerbatim}[commandchars=\\\{\}]
\PYG{k+kn}{from} \PYG{n+nn}{scipy} \PYG{k+kn}{import} \PYG{n}{stats}
\PYG{n}{a} \PYG{o}{=} \PYG{n}{data\PYGZus{}h2}\PYG{p}{[}\PYG{n}{data\PYGZus{}h2}\PYG{p}{[}\PYG{l+s+s1}{\PYGZsq{}}\PYG{l+s+s1}{flag}\PYG{l+s+s1}{\PYGZsq{}}\PYG{p}{]}\PYG{o}{==}\PYG{l+m+mi}{0}\PYG{p}{]}\PYG{p}{[}\PYG{l+s+s1}{\PYGZsq{}}\PYG{l+s+s1}{max\PYGZus{}close}\PYG{l+s+s1}{\PYGZsq{}}\PYG{p}{]}\PYG{o}{.}\PYG{n}{sample}\PYG{p}{(}\PYG{l+m+mi}{200}\PYG{p}{)}
\PYG{n}{b} \PYG{o}{=} \PYG{n}{data\PYGZus{}h2}\PYG{p}{[}\PYG{n}{data\PYGZus{}h2}\PYG{p}{[}\PYG{l+s+s1}{\PYGZsq{}}\PYG{l+s+s1}{flag}\PYG{l+s+s1}{\PYGZsq{}}\PYG{p}{]}\PYG{o}{==}\PYG{l+m+mi}{1}\PYG{p}{]}\PYG{p}{[}\PYG{l+s+s1}{\PYGZsq{}}\PYG{l+s+s1}{max\PYGZus{}close}\PYG{l+s+s1}{\PYGZsq{}}\PYG{p}{]}\PYG{o}{.}\PYG{n}{sample}\PYG{p}{(}\PYG{l+m+mi}{200}\PYG{p}{)}

\PYG{n}{stats}\PYG{o}{.}\PYG{n}{ttest\PYGZus{}ind}\PYG{p}{(}\PYG{n}{a}\PYG{p}{,} \PYG{n}{b}\PYG{p}{,} \PYG{n}{equal\PYGZus{}var}\PYG{o}{=}\PYG{k+kc}{False}\PYG{p}{)}
\end{sphinxVerbatim}

\end{sphinxuseclass}\end{sphinxVerbatimInput}
\begin{sphinxVerbatimOutput}

\begin{sphinxuseclass}{cell_output}
\begin{sphinxVerbatim}[commandchars=\\\{\}]
Ttest\PYGZus{}indResult(statistic=\PYGZhy{}1.8358785283648644, pvalue=0.06714153725869931)
\end{sphinxVerbatim}

\end{sphinxuseclass}\end{sphinxVerbatimOutput}

\end{sphinxuseclass}
\sphinxAtStartPar
 위 가설은 비교적 증명하기가 쉬웠습니다. 이번에는 5일선과 20일 이동평균선이 만나는 골든크로스에서 매수를 하면 어떤지 보겠습니다. 골든 크로스에서 매수한다고 더 좋은 수익율을 보장하지 않는 것 같습니다.

\begin{sphinxuseclass}{cell}\begin{sphinxVerbatimInput}

\begin{sphinxuseclass}{cell_input}
\begin{sphinxVerbatim}[commandchars=\\\{\}]
\PYG{n}{kosdaq\PYGZus{}list} \PYG{o}{=} \PYG{n}{pd}\PYG{o}{.}\PYG{n}{read\PYGZus{}pickle}\PYG{p}{(}\PYG{l+s+s1}{\PYGZsq{}}\PYG{l+s+s1}{kosdaq\PYGZus{}list.pkl}\PYG{l+s+s1}{\PYGZsq{}}\PYG{p}{)}

\PYG{n}{data\PYGZus{}h2} \PYG{o}{=} \PYG{n}{pd}\PYG{o}{.}\PYG{n}{DataFrame}\PYG{p}{(}\PYG{p}{)}

\PYG{k}{for} \PYG{n}{code} \PYG{o+ow}{in} \PYG{n}{kosdaq\PYGZus{}list}\PYG{p}{[}\PYG{l+s+s1}{\PYGZsq{}}\PYG{l+s+s1}{code}\PYG{l+s+s1}{\PYGZsq{}}\PYG{p}{]}\PYG{p}{:}

    \PYG{n}{data} \PYG{o}{=} \PYG{n}{mdl\PYGZus{}data}\PYG{p}{[}\PYG{n}{mdl\PYGZus{}data}\PYG{p}{[}\PYG{l+s+s1}{\PYGZsq{}}\PYG{l+s+s1}{code}\PYG{l+s+s1}{\PYGZsq{}}\PYG{p}{]}\PYG{o}{==}\PYG{n}{code}\PYG{p}{]}\PYG{o}{.}\PYG{n}{sort\PYGZus{}index}\PYG{p}{(}\PYG{p}{)}\PYG{o}{.}\PYG{n}{copy}\PYG{p}{(}\PYG{p}{)}
    \PYG{n}{data}\PYG{p}{[}\PYG{l+s+s1}{\PYGZsq{}}\PYG{l+s+s1}{5day\PYGZus{}ma}\PYG{l+s+s1}{\PYGZsq{}}\PYG{p}{]} \PYG{o}{=} \PYG{n}{data}\PYG{p}{[}\PYG{l+s+s1}{\PYGZsq{}}\PYG{l+s+s1}{close}\PYG{l+s+s1}{\PYGZsq{}}\PYG{p}{]}\PYG{o}{.}\PYG{n}{rolling}\PYG{p}{(}\PYG{l+m+mi}{5}\PYG{p}{)}\PYG{o}{.}\PYG{n}{mean}\PYG{p}{(}\PYG{p}{)} \PYG{c+c1}{\PYGZsh{} 5일 이동평균선}
    \PYG{n}{data}\PYG{p}{[}\PYG{l+s+s1}{\PYGZsq{}}\PYG{l+s+s1}{20day\PYGZus{}ma}\PYG{l+s+s1}{\PYGZsq{}}\PYG{p}{]} \PYG{o}{=} \PYG{n}{data}\PYG{p}{[}\PYG{l+s+s1}{\PYGZsq{}}\PYG{l+s+s1}{close}\PYG{l+s+s1}{\PYGZsq{}}\PYG{p}{]}\PYG{o}{.}\PYG{n}{rolling}\PYG{p}{(}\PYG{l+m+mi}{20}\PYG{p}{)}\PYG{o}{.}\PYG{n}{mean}\PYG{p}{(}\PYG{p}{)} \PYG{c+c1}{\PYGZsh{} 20일 이동평균선}
    \PYG{n}{data}\PYG{p}{[}\PYG{l+s+s1}{\PYGZsq{}}\PYG{l+s+s1}{golden\PYGZus{}cross}\PYG{l+s+s1}{\PYGZsq{}}\PYG{p}{]} \PYG{o}{=} \PYG{p}{(}\PYG{n}{data}\PYG{p}{[}\PYG{l+s+s1}{\PYGZsq{}}\PYG{l+s+s1}{5day\PYGZus{}ma}\PYG{l+s+s1}{\PYGZsq{}}\PYG{p}{]}\PYG{o}{.}\PYG{n}{shift}\PYG{p}{(}\PYG{l+m+mi}{1}\PYG{p}{)} \PYG{o}{\PYGZlt{}} \PYG{n}{data}\PYG{p}{[}\PYG{l+s+s1}{\PYGZsq{}}\PYG{l+s+s1}{20day\PYGZus{}ma}\PYG{l+s+s1}{\PYGZsq{}}\PYG{p}{]}\PYG{o}{.}\PYG{n}{shift}\PYG{p}{(}\PYG{l+m+mi}{1}\PYG{p}{)}\PYG{p}{)}\PYG{o}{*}\PYG{p}{(}\PYG{n}{data}\PYG{p}{[}\PYG{l+s+s1}{\PYGZsq{}}\PYG{l+s+s1}{5day\PYGZus{}ma}\PYG{l+s+s1}{\PYGZsq{}}\PYG{p}{]} \PYG{o}{\PYGZgt{}} \PYG{n}{data}\PYG{p}{[}\PYG{l+s+s1}{\PYGZsq{}}\PYG{l+s+s1}{20day\PYGZus{}ma}\PYG{l+s+s1}{\PYGZsq{}}\PYG{p}{]}\PYG{p}{)}\PYG{o}{.}\PYG{n}{astype}\PYG{p}{(}\PYG{n+nb}{int}\PYG{p}{)} \PYG{c+c1}{\PYGZsh{} 5일선이 20일 이동평균선보다 작았다가 커지는 시점}
       
    \PYG{n}{data}\PYG{p}{[}\PYG{l+s+s1}{\PYGZsq{}}\PYG{l+s+s1}{max\PYGZus{}close}\PYG{l+s+s1}{\PYGZsq{}}\PYG{p}{]}  \PYG{o}{=} \PYG{n}{data}\PYG{p}{[}\PYG{p}{[}\PYG{l+s+s1}{\PYGZsq{}}\PYG{l+s+s1}{close\PYGZus{}r1}\PYG{l+s+s1}{\PYGZsq{}}\PYG{p}{,}\PYG{l+s+s1}{\PYGZsq{}}\PYG{l+s+s1}{close\PYGZus{}r2}\PYG{l+s+s1}{\PYGZsq{}}\PYG{p}{,}\PYG{l+s+s1}{\PYGZsq{}}\PYG{l+s+s1}{close\PYGZus{}r3}\PYG{l+s+s1}{\PYGZsq{}}\PYG{p}{,}\PYG{l+s+s1}{\PYGZsq{}}\PYG{l+s+s1}{close\PYGZus{}r4}\PYG{l+s+s1}{\PYGZsq{}}\PYG{p}{,}\PYG{l+s+s1}{\PYGZsq{}}\PYG{l+s+s1}{close\PYGZus{}r5}\PYG{l+s+s1}{\PYGZsq{}}\PYG{p}{]}\PYG{p}{]}\PYG{o}{.}\PYG{n}{max}\PYG{p}{(}\PYG{n}{axis}\PYG{o}{=}\PYG{l+m+mi}{1}\PYG{p}{)} \PYG{c+c1}{\PYGZsh{} 5 영업일 종가 수익율 중 최고 값}
    \PYG{n}{data}\PYG{o}{.}\PYG{n}{dropna}\PYG{p}{(}\PYG{n}{subset}\PYG{o}{=}\PYG{p}{[}\PYG{l+s+s1}{\PYGZsq{}}\PYG{l+s+s1}{5day\PYGZus{}ma}\PYG{l+s+s1}{\PYGZsq{}}\PYG{p}{,}\PYG{l+s+s1}{\PYGZsq{}}\PYG{l+s+s1}{20day\PYGZus{}ma}\PYG{l+s+s1}{\PYGZsq{}}\PYG{p}{,}\PYG{l+s+s1}{\PYGZsq{}}\PYG{l+s+s1}{golden\PYGZus{}cross}\PYG{l+s+s1}{\PYGZsq{}}\PYG{p}{,}\PYG{l+s+s1}{\PYGZsq{}}\PYG{l+s+s1}{close\PYGZus{}r1}\PYG{l+s+s1}{\PYGZsq{}}\PYG{p}{,}\PYG{l+s+s1}{\PYGZsq{}}\PYG{l+s+s1}{close\PYGZus{}r2}\PYG{l+s+s1}{\PYGZsq{}}\PYG{p}{,}\PYG{l+s+s1}{\PYGZsq{}}\PYG{l+s+s1}{close\PYGZus{}r3}\PYG{l+s+s1}{\PYGZsq{}}\PYG{p}{,}\PYG{l+s+s1}{\PYGZsq{}}\PYG{l+s+s1}{close\PYGZus{}r4}\PYG{l+s+s1}{\PYGZsq{}}\PYG{p}{,}\PYG{l+s+s1}{\PYGZsq{}}\PYG{l+s+s1}{close\PYGZus{}r5}\PYG{l+s+s1}{\PYGZsq{}}\PYG{p}{]}\PYG{p}{,} \PYG{n}{inplace}\PYG{o}{=}\PYG{k+kc}{True}\PYG{p}{)} \PYG{c+c1}{\PYGZsh{} missing 이 있는 행은 제거  }
    
    \PYG{n}{data\PYGZus{}h2} \PYG{o}{=} \PYG{n}{pd}\PYG{o}{.}\PYG{n}{concat}\PYG{p}{(}\PYG{p}{[}\PYG{n}{data}\PYG{p}{,} \PYG{n}{data\PYGZus{}h2}\PYG{p}{]}\PYG{p}{,} \PYG{n}{axis}\PYG{o}{=}\PYG{l+m+mi}{0}\PYG{p}{)}

\PYG{n}{data\PYGZus{}h2}\PYG{o}{.}\PYG{n}{to\PYGZus{}pickle}\PYG{p}{(}\PYG{l+s+s1}{\PYGZsq{}}\PYG{l+s+s1}{data\PYGZus{}h2.pkl}\PYG{l+s+s1}{\PYGZsq{}}\PYG{p}{)}  
\end{sphinxVerbatim}

\end{sphinxuseclass}\end{sphinxVerbatimInput}

\end{sphinxuseclass}
\begin{sphinxuseclass}{cell}\begin{sphinxVerbatimInput}

\begin{sphinxuseclass}{cell_input}
\begin{sphinxVerbatim}[commandchars=\\\{\}]
\PYG{n}{data\PYGZus{}h2} \PYG{o}{=} \PYG{n}{pd}\PYG{o}{.}\PYG{n}{read\PYGZus{}pickle}\PYG{p}{(}\PYG{l+s+s1}{\PYGZsq{}}\PYG{l+s+s1}{data\PYGZus{}h2.pkl}\PYG{l+s+s1}{\PYGZsq{}}\PYG{p}{)}
\PYG{n}{data\PYGZus{}h2}\PYG{o}{.}\PYG{n}{groupby}\PYG{p}{(}\PYG{l+s+s1}{\PYGZsq{}}\PYG{l+s+s1}{golden\PYGZus{}cross}\PYG{l+s+s1}{\PYGZsq{}}\PYG{p}{)}\PYG{p}{[}\PYG{l+s+s1}{\PYGZsq{}}\PYG{l+s+s1}{max\PYGZus{}close}\PYG{l+s+s1}{\PYGZsq{}}\PYG{p}{]}\PYG{o}{.}\PYG{n}{describe}\PYG{p}{(}\PYG{p}{)}\PYG{o}{.}\PYG{n}{style}\PYG{o}{.}\PYG{n}{set\PYGZus{}table\PYGZus{}attributes}\PYG{p}{(}\PYG{l+s+s1}{\PYGZsq{}}\PYG{l+s+s1}{style=}\PYG{l+s+s1}{\PYGZdq{}}\PYG{l+s+s1}{font\PYGZhy{}size: 12px}\PYG{l+s+s1}{\PYGZdq{}}\PYG{l+s+s1}{\PYGZsq{}}\PYG{p}{)}\PYG{o}{.}\PYG{n}{format}\PYG{p}{(}\PYG{n}{precision}\PYG{o}{=}\PYG{l+m+mi}{3}\PYG{p}{)}
\end{sphinxVerbatim}

\end{sphinxuseclass}\end{sphinxVerbatimInput}
\begin{sphinxVerbatimOutput}

\begin{sphinxuseclass}{cell_output}
\begin{sphinxVerbatim}[commandchars=\\\{\}]
\PYGZlt{}pandas.io.formats.style.Styler at 0x1d3934be070\PYGZgt{}
\end{sphinxVerbatim}

\end{sphinxuseclass}\end{sphinxVerbatimOutput}

\end{sphinxuseclass}
\begin{sphinxuseclass}{cell}\begin{sphinxVerbatimInput}

\begin{sphinxuseclass}{cell_input}
\begin{sphinxVerbatim}[commandchars=\\\{\}]
\PYG{k+kn}{import} \PYG{n+nn}{FinanceDataReader} \PYG{k}{as} \PYG{n+nn}{fdr}
\PYG{o}{\PYGZpc{}}\PYG{k}{matplotlib} inline
\PYG{k+kn}{import} \PYG{n+nn}{matplotlib}\PYG{n+nn}{.}\PYG{n+nn}{pyplot} \PYG{k}{as} \PYG{n+nn}{plt}
\PYG{k+kn}{import} \PYG{n+nn}{pandas} \PYG{k}{as} \PYG{n+nn}{pd}
\PYG{k+kn}{import} \PYG{n+nn}{numpy} \PYG{k}{as} \PYG{n+nn}{np}
\PYG{k+kn}{import} \PYG{n+nn}{warnings}
\PYG{n}{warnings}\PYG{o}{.}\PYG{n}{filterwarnings}\PYG{p}{(}\PYG{l+s+s1}{\PYGZsq{}}\PYG{l+s+s1}{ignore}\PYG{l+s+s1}{\PYGZsq{}}\PYG{p}{)}

\PYG{n}{pd}\PYG{o}{.}\PYG{n}{options}\PYG{o}{.}\PYG{n}{display}\PYG{o}{.}\PYG{n}{float\PYGZus{}format} \PYG{o}{=} \PYG{l+s+s1}{\PYGZsq{}}\PYG{l+s+si}{\PYGZob{}:,.3f\PYGZcb{}}\PYG{l+s+s1}{\PYGZsq{}}\PYG{o}{.}\PYG{n}{format}
\end{sphinxVerbatim}

\end{sphinxuseclass}\end{sphinxVerbatimInput}

\end{sphinxuseclass}

\section{위 꼬리가 긴 양봉이 자주 발생한다.}
\label{\detokenize{chapter5/5.1.3_Hypothesis_3:id1}}\label{\detokenize{chapter5/5.1.3_Hypothesis_3::doc}}
\sphinxAtStartPar
위 꼬리는 종가보다 고가가 더 높이 위치해 있는 양봉입니다. 따라서 고가를 종가로 나눈 값이 1 보다 상당히 크면 위꼬리 양봉이라고 할 수 있습니다. 양봉의 조건은 종가가 시가보다 큰 것입니다. 이 것을 데이터로 표현합니다.

\begin{sphinxuseclass}{cell}\begin{sphinxVerbatimInput}

\begin{sphinxuseclass}{cell_input}
\begin{sphinxVerbatim}[commandchars=\\\{\}]
\PYG{n}{mdl\PYGZus{}data} \PYG{o}{=} \PYG{n}{pd}\PYG{o}{.}\PYG{n}{read\PYGZus{}pickle}\PYG{p}{(}\PYG{l+s+s1}{\PYGZsq{}}\PYG{l+s+s1}{mdl\PYGZus{}data.pkl}\PYG{l+s+s1}{\PYGZsq{}}\PYG{p}{)}
\PYG{n}{mdl\PYGZus{}data}\PYG{o}{.}\PYG{n}{head}\PYG{p}{(}\PYG{p}{)}\PYG{o}{.}\PYG{n}{style}\PYG{o}{.}\PYG{n}{set\PYGZus{}table\PYGZus{}attributes}\PYG{p}{(}\PYG{l+s+s1}{\PYGZsq{}}\PYG{l+s+s1}{style=}\PYG{l+s+s1}{\PYGZdq{}}\PYG{l+s+s1}{font\PYGZhy{}size: 12px}\PYG{l+s+s1}{\PYGZdq{}}\PYG{l+s+s1}{\PYGZsq{}}\PYG{p}{)}
\end{sphinxVerbatim}

\end{sphinxuseclass}\end{sphinxVerbatimInput}
\begin{sphinxVerbatimOutput}

\begin{sphinxuseclass}{cell_output}
\begin{sphinxVerbatim}[commandchars=\\\{\}]
\PYGZlt{}pandas.io.formats.style.Styler at 0x1a8d5107f70\PYGZgt{}
\end{sphinxVerbatim}

\end{sphinxuseclass}\end{sphinxVerbatimOutput}

\end{sphinxuseclass}
\begin{sphinxuseclass}{cell}\begin{sphinxVerbatimInput}

\begin{sphinxuseclass}{cell_input}
\begin{sphinxVerbatim}[commandchars=\\\{\}]
\PYG{n}{kosdaq\PYGZus{}list} \PYG{o}{=} \PYG{n}{pd}\PYG{o}{.}\PYG{n}{read\PYGZus{}pickle}\PYG{p}{(}\PYG{l+s+s1}{\PYGZsq{}}\PYG{l+s+s1}{kosdaq\PYGZus{}list.pkl}\PYG{l+s+s1}{\PYGZsq{}}\PYG{p}{)}

\PYG{n}{data\PYGZus{}h3} \PYG{o}{=} \PYG{n}{pd}\PYG{o}{.}\PYG{n}{DataFrame}\PYG{p}{(}\PYG{p}{)}

\PYG{k}{for} \PYG{n}{code} \PYG{o+ow}{in} \PYG{n}{kosdaq\PYGZus{}list}\PYG{p}{[}\PYG{l+s+s1}{\PYGZsq{}}\PYG{l+s+s1}{code}\PYG{l+s+s1}{\PYGZsq{}}\PYG{p}{]}\PYG{p}{:}

    \PYG{n}{data} \PYG{o}{=} \PYG{n}{mdl\PYGZus{}data}\PYG{p}{[}\PYG{n}{mdl\PYGZus{}data}\PYG{p}{[}\PYG{l+s+s1}{\PYGZsq{}}\PYG{l+s+s1}{code}\PYG{l+s+s1}{\PYGZsq{}}\PYG{p}{]}\PYG{o}{==}\PYG{n}{code}\PYG{p}{]}\PYG{o}{.}\PYG{n}{sort\PYGZus{}index}\PYG{p}{(}\PYG{p}{)}\PYG{o}{.}\PYG{n}{copy}\PYG{p}{(}\PYG{p}{)}
    
    \PYG{n}{data}\PYG{p}{[}\PYG{l+s+s1}{\PYGZsq{}}\PYG{l+s+s1}{positive\PYGZus{}candle}\PYG{l+s+s1}{\PYGZsq{}}\PYG{p}{]} \PYG{o}{=} \PYG{p}{(}\PYG{n}{data}\PYG{p}{[}\PYG{l+s+s1}{\PYGZsq{}}\PYG{l+s+s1}{close}\PYG{l+s+s1}{\PYGZsq{}}\PYG{p}{]} \PYG{o}{\PYGZgt{}} \PYG{n}{data}\PYG{p}{[}\PYG{l+s+s1}{\PYGZsq{}}\PYG{l+s+s1}{open}\PYG{l+s+s1}{\PYGZsq{}}\PYG{p}{]}\PYG{p}{)}\PYG{o}{.}\PYG{n}{astype}\PYG{p}{(}\PYG{n+nb}{int}\PYG{p}{)} \PYG{c+c1}{\PYGZsh{} 양봉}
    \PYG{n}{data}\PYG{p}{[}\PYG{l+s+s1}{\PYGZsq{}}\PYG{l+s+s1}{high/close}\PYG{l+s+s1}{\PYGZsq{}}\PYG{p}{]} \PYG{o}{=} \PYG{p}{(}\PYG{n}{data}\PYG{p}{[}\PYG{l+s+s1}{\PYGZsq{}}\PYG{l+s+s1}{positive\PYGZus{}candle}\PYG{l+s+s1}{\PYGZsq{}}\PYG{p}{]}\PYG{o}{==}\PYG{l+m+mi}{1}\PYG{p}{)}\PYG{o}{*}\PYG{p}{(}\PYG{n}{data}\PYG{p}{[}\PYG{l+s+s1}{\PYGZsq{}}\PYG{l+s+s1}{high}\PYG{l+s+s1}{\PYGZsq{}}\PYG{p}{]}\PYG{o}{/}\PYG{n}{data}\PYG{p}{[}\PYG{l+s+s1}{\PYGZsq{}}\PYG{l+s+s1}{close}\PYG{l+s+s1}{\PYGZsq{}}\PYG{p}{]} \PYG{o}{\PYGZgt{}} \PYG{l+m+mf}{1.1}\PYG{p}{)}\PYG{o}{.}\PYG{n}{astype}\PYG{p}{(}\PYG{n+nb}{int}\PYG{p}{)} \PYG{c+c1}{\PYGZsh{} 양봉이면서 고가가 종가보다 높게 위치 10\PYGZpc{} 이상 높은 경우}
    \PYG{n}{data}\PYG{p}{[}\PYG{l+s+s1}{\PYGZsq{}}\PYG{l+s+s1}{num\PYGZus{}high/close}\PYG{l+s+s1}{\PYGZsq{}}\PYG{p}{]} \PYG{o}{=}  \PYG{n}{data}\PYG{p}{[}\PYG{l+s+s1}{\PYGZsq{}}\PYG{l+s+s1}{high/close}\PYG{l+s+s1}{\PYGZsq{}}\PYG{p}{]}\PYG{o}{.}\PYG{n}{rolling}\PYG{p}{(}\PYG{l+m+mi}{20}\PYG{p}{)}\PYG{o}{.}\PYG{n}{sum}\PYG{p}{(}\PYG{p}{)}
       
    \PYG{n}{data}\PYG{p}{[}\PYG{l+s+s1}{\PYGZsq{}}\PYG{l+s+s1}{max\PYGZus{}close}\PYG{l+s+s1}{\PYGZsq{}}\PYG{p}{]}  \PYG{o}{=} \PYG{n}{data}\PYG{p}{[}\PYG{p}{[}\PYG{l+s+s1}{\PYGZsq{}}\PYG{l+s+s1}{close\PYGZus{}r1}\PYG{l+s+s1}{\PYGZsq{}}\PYG{p}{,}\PYG{l+s+s1}{\PYGZsq{}}\PYG{l+s+s1}{close\PYGZus{}r2}\PYG{l+s+s1}{\PYGZsq{}}\PYG{p}{,}\PYG{l+s+s1}{\PYGZsq{}}\PYG{l+s+s1}{close\PYGZus{}r3}\PYG{l+s+s1}{\PYGZsq{}}\PYG{p}{,}\PYG{l+s+s1}{\PYGZsq{}}\PYG{l+s+s1}{close\PYGZus{}r4}\PYG{l+s+s1}{\PYGZsq{}}\PYG{p}{,}\PYG{l+s+s1}{\PYGZsq{}}\PYG{l+s+s1}{close\PYGZus{}r5}\PYG{l+s+s1}{\PYGZsq{}}\PYG{p}{]}\PYG{p}{]}\PYG{o}{.}\PYG{n}{max}\PYG{p}{(}\PYG{n}{axis}\PYG{o}{=}\PYG{l+m+mi}{1}\PYG{p}{)} \PYG{c+c1}{\PYGZsh{} 5 영업일 종가 수익율 중 최고 값}
    \PYG{n}{data}\PYG{o}{.}\PYG{n}{dropna}\PYG{p}{(}\PYG{n}{subset}\PYG{o}{=}\PYG{p}{[}\PYG{l+s+s1}{\PYGZsq{}}\PYG{l+s+s1}{num\PYGZus{}high/close}\PYG{l+s+s1}{\PYGZsq{}}\PYG{p}{,}\PYG{l+s+s1}{\PYGZsq{}}\PYG{l+s+s1}{close\PYGZus{}r1}\PYG{l+s+s1}{\PYGZsq{}}\PYG{p}{,}\PYG{l+s+s1}{\PYGZsq{}}\PYG{l+s+s1}{close\PYGZus{}r2}\PYG{l+s+s1}{\PYGZsq{}}\PYG{p}{,}\PYG{l+s+s1}{\PYGZsq{}}\PYG{l+s+s1}{close\PYGZus{}r3}\PYG{l+s+s1}{\PYGZsq{}}\PYG{p}{,}\PYG{l+s+s1}{\PYGZsq{}}\PYG{l+s+s1}{close\PYGZus{}r4}\PYG{l+s+s1}{\PYGZsq{}}\PYG{p}{,}\PYG{l+s+s1}{\PYGZsq{}}\PYG{l+s+s1}{close\PYGZus{}r5}\PYG{l+s+s1}{\PYGZsq{}}\PYG{p}{]}\PYG{p}{,} \PYG{n}{inplace}\PYG{o}{=}\PYG{k+kc}{True}\PYG{p}{)} \PYG{c+c1}{\PYGZsh{} missing 이 있는 행은 제거  }
    
    \PYG{n}{data\PYGZus{}h3} \PYG{o}{=} \PYG{n}{pd}\PYG{o}{.}\PYG{n}{concat}\PYG{p}{(}\PYG{p}{[}\PYG{n}{data}\PYG{p}{,} \PYG{n}{data\PYGZus{}h3}\PYG{p}{]}\PYG{p}{,} \PYG{n}{axis}\PYG{o}{=}\PYG{l+m+mi}{0}\PYG{p}{)}

\PYG{n}{data\PYGZus{}h3}\PYG{o}{.}\PYG{n}{to\PYGZus{}pickle}\PYG{p}{(}\PYG{l+s+s1}{\PYGZsq{}}\PYG{l+s+s1}{data\PYGZus{}h3.pkl}\PYG{l+s+s1}{\PYGZsq{}}\PYG{p}{)}  
\end{sphinxVerbatim}

\end{sphinxuseclass}\end{sphinxVerbatimInput}

\end{sphinxuseclass}
\sphinxAtStartPar
 윗 꼬리가 긴 양봉이 많이 발생할 수 록 수익율에 좋은 영향을 주는 것으로 분석이 되었습니다.

\begin{sphinxuseclass}{cell}\begin{sphinxVerbatimInput}

\begin{sphinxuseclass}{cell_input}
\begin{sphinxVerbatim}[commandchars=\\\{\}]
\PYG{n}{data\PYGZus{}h3} \PYG{o}{=} \PYG{n}{pd}\PYG{o}{.}\PYG{n}{read\PYGZus{}pickle}\PYG{p}{(}\PYG{l+s+s1}{\PYGZsq{}}\PYG{l+s+s1}{data\PYGZus{}h3.pkl}\PYG{l+s+s1}{\PYGZsq{}}\PYG{p}{)}
\PYG{n+nb}{print}\PYG{p}{(}\PYG{n}{data\PYGZus{}h3}\PYG{o}{.}\PYG{n}{groupby}\PYG{p}{(}\PYG{l+s+s1}{\PYGZsq{}}\PYG{l+s+s1}{num\PYGZus{}high/close}\PYG{l+s+s1}{\PYGZsq{}}\PYG{p}{)}\PYG{p}{[}\PYG{l+s+s1}{\PYGZsq{}}\PYG{l+s+s1}{max\PYGZus{}close}\PYG{l+s+s1}{\PYGZsq{}}\PYG{p}{]}\PYG{o}{.}\PYG{n}{agg}\PYG{p}{(}\PYG{p}{[}\PYG{l+s+s1}{\PYGZsq{}}\PYG{l+s+s1}{count}\PYG{l+s+s1}{\PYGZsq{}}\PYG{p}{,}\PYG{l+s+s1}{\PYGZsq{}}\PYG{l+s+s1}{mean}\PYG{l+s+s1}{\PYGZsq{}}\PYG{p}{]}\PYG{p}{)}\PYG{p}{)}
\PYG{n}{data\PYGZus{}h3}\PYG{o}{.}\PYG{n}{groupby}\PYG{p}{(}\PYG{l+s+s1}{\PYGZsq{}}\PYG{l+s+s1}{num\PYGZus{}high/close}\PYG{l+s+s1}{\PYGZsq{}}\PYG{p}{)}\PYG{p}{[}\PYG{l+s+s1}{\PYGZsq{}}\PYG{l+s+s1}{max\PYGZus{}close}\PYG{l+s+s1}{\PYGZsq{}}\PYG{p}{]}\PYG{o}{.}\PYG{n}{mean}\PYG{p}{(}\PYG{p}{)}\PYG{o}{.}\PYG{n}{plot}\PYG{p}{(}\PYG{n}{kind}\PYG{o}{=}\PYG{l+s+s1}{\PYGZsq{}}\PYG{l+s+s1}{bar}\PYG{l+s+s1}{\PYGZsq{}}\PYG{p}{,} \PYG{n}{ylim}\PYG{o}{=}\PYG{p}{(}\PYG{l+m+mf}{0.9}\PYG{p}{,}\PYG{l+m+mf}{1.2}\PYG{p}{)}\PYG{p}{)} \PYG{c+c1}{\PYGZsh{} 막대그래프로 표현}
\end{sphinxVerbatim}

\end{sphinxuseclass}\end{sphinxVerbatimInput}
\begin{sphinxVerbatimOutput}

\begin{sphinxuseclass}{cell_output}
\begin{sphinxVerbatim}[commandchars=\\\{\}]
                 count  mean
num\PYGZus{}high/close              
0.000           355754 1.031
1.000            37734 1.043
2.000             5113 1.050
3.000              824 1.072
4.000              159 1.047
5.000               10 1.168
\end{sphinxVerbatim}

\begin{sphinxVerbatim}[commandchars=\\\{\}]
\PYGZlt{}AxesSubplot:xlabel=\PYGZsq{}num\PYGZus{}high/close\PYGZsq{}\PYGZgt{}
\end{sphinxVerbatim}

\noindent\sphinxincludegraphics{{5.1.3_Hypothesis_3_5_2}.png}

\end{sphinxuseclass}\end{sphinxVerbatimOutput}

\end{sphinxuseclass}
\sphinxAtStartPar
 윗 꼬리가 긴 양봉도 궁금하지만, 장대양봉은 어떨지도 궁금합니다. 이렇게 가설을 검증하는 과정에서 새로운 가설을 테스트하기도 합니다. 장대양봉이 과거 60일 동안 몇 번 발생했는지 카운트해보고, 장대양봉의 갯 수와 수익율 사이에 상관성이 있는 지 함 보겠습니다.

\begin{sphinxuseclass}{cell}\begin{sphinxVerbatimInput}

\begin{sphinxuseclass}{cell_input}
\begin{sphinxVerbatim}[commandchars=\\\{\}]
\PYG{n}{kosdaq\PYGZus{}list} \PYG{o}{=} \PYG{n}{pd}\PYG{o}{.}\PYG{n}{read\PYGZus{}pickle}\PYG{p}{(}\PYG{l+s+s1}{\PYGZsq{}}\PYG{l+s+s1}{kosdaq\PYGZus{}list.pkl}\PYG{l+s+s1}{\PYGZsq{}}\PYG{p}{)}

\PYG{n}{data\PYGZus{}h3} \PYG{o}{=} \PYG{n}{pd}\PYG{o}{.}\PYG{n}{DataFrame}\PYG{p}{(}\PYG{p}{)}

\PYG{k}{for} \PYG{n}{code} \PYG{o+ow}{in} \PYG{n}{kosdaq\PYGZus{}list}\PYG{p}{[}\PYG{l+s+s1}{\PYGZsq{}}\PYG{l+s+s1}{code}\PYG{l+s+s1}{\PYGZsq{}}\PYG{p}{]}\PYG{p}{:}

    \PYG{n}{data} \PYG{o}{=} \PYG{n}{mdl\PYGZus{}data}\PYG{p}{[}\PYG{n}{mdl\PYGZus{}data}\PYG{p}{[}\PYG{l+s+s1}{\PYGZsq{}}\PYG{l+s+s1}{code}\PYG{l+s+s1}{\PYGZsq{}}\PYG{p}{]}\PYG{o}{==}\PYG{n}{code}\PYG{p}{]}\PYG{o}{.}\PYG{n}{sort\PYGZus{}index}\PYG{p}{(}\PYG{p}{)}\PYG{o}{.}\PYG{n}{copy}\PYG{p}{(}\PYG{p}{)}
    
    \PYG{n}{data}\PYG{p}{[}\PYG{l+s+s1}{\PYGZsq{}}\PYG{l+s+s1}{positive\PYGZus{}candle}\PYG{l+s+s1}{\PYGZsq{}}\PYG{p}{]} \PYG{o}{=} \PYG{p}{(}\PYG{n}{data}\PYG{p}{[}\PYG{l+s+s1}{\PYGZsq{}}\PYG{l+s+s1}{close}\PYG{l+s+s1}{\PYGZsq{}}\PYG{p}{]} \PYG{o}{\PYGZgt{}} \PYG{n}{data}\PYG{p}{[}\PYG{l+s+s1}{\PYGZsq{}}\PYG{l+s+s1}{open}\PYG{l+s+s1}{\PYGZsq{}}\PYG{p}{]}\PYG{p}{)}\PYG{o}{.}\PYG{n}{astype}\PYG{p}{(}\PYG{n+nb}{int}\PYG{p}{)} \PYG{c+c1}{\PYGZsh{} 양봉}
    \PYG{n}{data}\PYG{p}{[}\PYG{l+s+s1}{\PYGZsq{}}\PYG{l+s+s1}{long\PYGZus{}candle}\PYG{l+s+s1}{\PYGZsq{}}\PYG{p}{]} \PYG{o}{=} \PYG{p}{(}\PYG{n}{data}\PYG{p}{[}\PYG{l+s+s1}{\PYGZsq{}}\PYG{l+s+s1}{positive\PYGZus{}candle}\PYG{l+s+s1}{\PYGZsq{}}\PYG{p}{]}\PYG{o}{==}\PYG{l+m+mi}{1}\PYG{p}{)}\PYG{o}{*}\PYG{p}{(}\PYG{n}{data}\PYG{p}{[}\PYG{l+s+s1}{\PYGZsq{}}\PYG{l+s+s1}{high}\PYG{l+s+s1}{\PYGZsq{}}\PYG{p}{]}\PYG{o}{==}\PYG{n}{data}\PYG{p}{[}\PYG{l+s+s1}{\PYGZsq{}}\PYG{l+s+s1}{close}\PYG{l+s+s1}{\PYGZsq{}}\PYG{p}{]}\PYG{p}{)}\PYG{o}{*}\PYGZbs{}
    \PYG{p}{(}\PYG{n}{data}\PYG{p}{[}\PYG{l+s+s1}{\PYGZsq{}}\PYG{l+s+s1}{low}\PYG{l+s+s1}{\PYGZsq{}}\PYG{p}{]}\PYG{o}{==}\PYG{n}{data}\PYG{p}{[}\PYG{l+s+s1}{\PYGZsq{}}\PYG{l+s+s1}{open}\PYG{l+s+s1}{\PYGZsq{}}\PYG{p}{]}\PYG{p}{)}\PYG{o}{*}\PYG{p}{(}\PYG{n}{data}\PYG{p}{[}\PYG{l+s+s1}{\PYGZsq{}}\PYG{l+s+s1}{close}\PYG{l+s+s1}{\PYGZsq{}}\PYG{p}{]}\PYG{o}{/}\PYG{n}{data}\PYG{p}{[}\PYG{l+s+s1}{\PYGZsq{}}\PYG{l+s+s1}{open}\PYG{l+s+s1}{\PYGZsq{}}\PYG{p}{]} \PYG{o}{\PYGZgt{}} \PYG{l+m+mf}{1.2}\PYG{p}{)}\PYG{o}{.}\PYG{n}{astype}\PYG{p}{(}\PYG{n+nb}{int}\PYG{p}{)} \PYG{c+c1}{\PYGZsh{} 장대 양봉을 데이터로 표현}
    
    \PYG{n}{data}\PYG{p}{[}\PYG{l+s+s1}{\PYGZsq{}}\PYG{l+s+s1}{num\PYGZus{}long}\PYG{l+s+s1}{\PYGZsq{}}\PYG{p}{]} \PYG{o}{=}  \PYG{n}{data}\PYG{p}{[}\PYG{l+s+s1}{\PYGZsq{}}\PYG{l+s+s1}{long\PYGZus{}candle}\PYG{l+s+s1}{\PYGZsq{}}\PYG{p}{]}\PYG{o}{.}\PYG{n}{rolling}\PYG{p}{(}\PYG{l+m+mi}{60}\PYG{p}{)}\PYG{o}{.}\PYG{n}{sum}\PYG{p}{(}\PYG{p}{)} \PYG{c+c1}{\PYGZsh{} 지난 20 일 동안 장대양봉의 갯 수}
       
    \PYG{n}{data}\PYG{p}{[}\PYG{l+s+s1}{\PYGZsq{}}\PYG{l+s+s1}{max\PYGZus{}close}\PYG{l+s+s1}{\PYGZsq{}}\PYG{p}{]}  \PYG{o}{=} \PYG{n}{data}\PYG{p}{[}\PYG{p}{[}\PYG{l+s+s1}{\PYGZsq{}}\PYG{l+s+s1}{close\PYGZus{}r1}\PYG{l+s+s1}{\PYGZsq{}}\PYG{p}{,}\PYG{l+s+s1}{\PYGZsq{}}\PYG{l+s+s1}{close\PYGZus{}r2}\PYG{l+s+s1}{\PYGZsq{}}\PYG{p}{,}\PYG{l+s+s1}{\PYGZsq{}}\PYG{l+s+s1}{close\PYGZus{}r3}\PYG{l+s+s1}{\PYGZsq{}}\PYG{p}{,}\PYG{l+s+s1}{\PYGZsq{}}\PYG{l+s+s1}{close\PYGZus{}r4}\PYG{l+s+s1}{\PYGZsq{}}\PYG{p}{,}\PYG{l+s+s1}{\PYGZsq{}}\PYG{l+s+s1}{close\PYGZus{}r5}\PYG{l+s+s1}{\PYGZsq{}}\PYG{p}{]}\PYG{p}{]}\PYG{o}{.}\PYG{n}{max}\PYG{p}{(}\PYG{n}{axis}\PYG{o}{=}\PYG{l+m+mi}{1}\PYG{p}{)} \PYG{c+c1}{\PYGZsh{} 5 영업일 종가 수익율 중 최고 값}
    \PYG{n}{data}\PYG{o}{.}\PYG{n}{dropna}\PYG{p}{(}\PYG{n}{subset}\PYG{o}{=}\PYG{p}{[}\PYG{l+s+s1}{\PYGZsq{}}\PYG{l+s+s1}{num\PYGZus{}long}\PYG{l+s+s1}{\PYGZsq{}}\PYG{p}{,}\PYG{l+s+s1}{\PYGZsq{}}\PYG{l+s+s1}{close\PYGZus{}r1}\PYG{l+s+s1}{\PYGZsq{}}\PYG{p}{,}\PYG{l+s+s1}{\PYGZsq{}}\PYG{l+s+s1}{close\PYGZus{}r2}\PYG{l+s+s1}{\PYGZsq{}}\PYG{p}{,}\PYG{l+s+s1}{\PYGZsq{}}\PYG{l+s+s1}{close\PYGZus{}r3}\PYG{l+s+s1}{\PYGZsq{}}\PYG{p}{,}\PYG{l+s+s1}{\PYGZsq{}}\PYG{l+s+s1}{close\PYGZus{}r4}\PYG{l+s+s1}{\PYGZsq{}}\PYG{p}{,}\PYG{l+s+s1}{\PYGZsq{}}\PYG{l+s+s1}{close\PYGZus{}r5}\PYG{l+s+s1}{\PYGZsq{}}\PYG{p}{]}\PYG{p}{,} \PYG{n}{inplace}\PYG{o}{=}\PYG{k+kc}{True}\PYG{p}{)} \PYG{c+c1}{\PYGZsh{} missing 이 있는 행은 제거  }
    
    \PYG{n}{data\PYGZus{}h3} \PYG{o}{=} \PYG{n}{pd}\PYG{o}{.}\PYG{n}{concat}\PYG{p}{(}\PYG{p}{[}\PYG{n}{data}\PYG{p}{,} \PYG{n}{data\PYGZus{}h3}\PYG{p}{]}\PYG{p}{,} \PYG{n}{axis}\PYG{o}{=}\PYG{l+m+mi}{0}\PYG{p}{)}

\PYG{n}{data\PYGZus{}h3}\PYG{o}{.}\PYG{n}{to\PYGZus{}pickle}\PYG{p}{(}\PYG{l+s+s1}{\PYGZsq{}}\PYG{l+s+s1}{data\PYGZus{}h3.pkl}\PYG{l+s+s1}{\PYGZsq{}}\PYG{p}{)}  
\end{sphinxVerbatim}

\end{sphinxuseclass}\end{sphinxVerbatimInput}

\end{sphinxuseclass}
\sphinxAtStartPar
 과거 60일 동안 장대양봉이 2 번 발생한 경우 좋은 수익율을 보여주고 있습니다.

\begin{sphinxuseclass}{cell}\begin{sphinxVerbatimInput}

\begin{sphinxuseclass}{cell_input}
\begin{sphinxVerbatim}[commandchars=\\\{\}]
\PYG{n}{data\PYGZus{}h3} \PYG{o}{=} \PYG{n}{pd}\PYG{o}{.}\PYG{n}{read\PYGZus{}pickle}\PYG{p}{(}\PYG{l+s+s1}{\PYGZsq{}}\PYG{l+s+s1}{data\PYGZus{}h3.pkl}\PYG{l+s+s1}{\PYGZsq{}}\PYG{p}{)}
\PYG{n+nb}{print}\PYG{p}{(}\PYG{n}{data\PYGZus{}h3}\PYG{o}{.}\PYG{n}{groupby}\PYG{p}{(}\PYG{l+s+s1}{\PYGZsq{}}\PYG{l+s+s1}{num\PYGZus{}long}\PYG{l+s+s1}{\PYGZsq{}}\PYG{p}{)}\PYG{p}{[}\PYG{l+s+s1}{\PYGZsq{}}\PYG{l+s+s1}{max\PYGZus{}close}\PYG{l+s+s1}{\PYGZsq{}}\PYG{p}{]}\PYG{o}{.}\PYG{n}{agg}\PYG{p}{(}\PYG{p}{[}\PYG{l+s+s1}{\PYGZsq{}}\PYG{l+s+s1}{count}\PYG{l+s+s1}{\PYGZsq{}}\PYG{p}{,}\PYG{l+s+s1}{\PYGZsq{}}\PYG{l+s+s1}{mean}\PYG{l+s+s1}{\PYGZsq{}}\PYG{p}{]}\PYG{p}{)}\PYG{p}{)}
\PYG{n}{data\PYGZus{}h3}\PYG{o}{.}\PYG{n}{groupby}\PYG{p}{(}\PYG{l+s+s1}{\PYGZsq{}}\PYG{l+s+s1}{num\PYGZus{}long}\PYG{l+s+s1}{\PYGZsq{}}\PYG{p}{)}\PYG{p}{[}\PYG{l+s+s1}{\PYGZsq{}}\PYG{l+s+s1}{max\PYGZus{}close}\PYG{l+s+s1}{\PYGZsq{}}\PYG{p}{]}\PYG{o}{.}\PYG{n}{mean}\PYG{p}{(}\PYG{p}{)}\PYG{o}{.}\PYG{n}{plot}\PYG{p}{(}\PYG{n}{kind}\PYG{o}{=}\PYG{l+s+s1}{\PYGZsq{}}\PYG{l+s+s1}{bar}\PYG{l+s+s1}{\PYGZsq{}}\PYG{p}{,} \PYG{n}{ylim}\PYG{o}{=}\PYG{p}{(}\PYG{l+m+mf}{0.9}\PYG{p}{,}\PYG{l+m+mf}{1.1}\PYG{p}{)}\PYG{p}{)}
\end{sphinxVerbatim}

\end{sphinxuseclass}\end{sphinxVerbatimInput}
\begin{sphinxVerbatimOutput}

\begin{sphinxuseclass}{cell_output}
\begin{sphinxVerbatim}[commandchars=\\\{\}]
           count  mean
num\PYGZus{}long              
0.000     337432 1.031
1.000       5394 1.047
2.000         88 1.056
\end{sphinxVerbatim}

\begin{sphinxVerbatim}[commandchars=\\\{\}]
\PYGZlt{}AxesSubplot:xlabel=\PYGZsq{}num\PYGZus{}long\PYGZsq{}\PYGZgt{}
\end{sphinxVerbatim}

\noindent\sphinxincludegraphics{{5.1.3_Hypothesis_3_9_2}.png}

\end{sphinxuseclass}\end{sphinxVerbatimOutput}

\end{sphinxuseclass}
\begin{sphinxuseclass}{cell}\begin{sphinxVerbatimInput}

\begin{sphinxuseclass}{cell_input}
\begin{sphinxVerbatim}[commandchars=\\\{\}]
\PYG{k+kn}{import} \PYG{n+nn}{FinanceDataReader} \PYG{k}{as} \PYG{n+nn}{fdr}
\PYG{o}{\PYGZpc{}}\PYG{k}{matplotlib} inline
\PYG{k+kn}{import} \PYG{n+nn}{matplotlib}\PYG{n+nn}{.}\PYG{n+nn}{pyplot} \PYG{k}{as} \PYG{n+nn}{plt}
\PYG{k+kn}{import} \PYG{n+nn}{pandas} \PYG{k}{as} \PYG{n+nn}{pd}
\PYG{k+kn}{import} \PYG{n+nn}{numpy} \PYG{k}{as} \PYG{n+nn}{np}
\PYG{n}{pd}\PYG{o}{.}\PYG{n}{options}\PYG{o}{.}\PYG{n}{display}\PYG{o}{.}\PYG{n}{float\PYGZus{}format} \PYG{o}{=} \PYG{l+s+s1}{\PYGZsq{}}\PYG{l+s+si}{\PYGZob{}:,.3f\PYGZcb{}}\PYG{l+s+s1}{\PYGZsq{}}\PYG{o}{.}\PYG{n}{format}
\end{sphinxVerbatim}

\end{sphinxuseclass}\end{sphinxVerbatimInput}

\end{sphinxuseclass}

\section{거래량이 종종 터지며, 매집의 흔적을 보인다.}
\label{\detokenize{chapter5/5.1.4_Hypothesis_4:id1}}\label{\detokenize{chapter5/5.1.4_Hypothesis_4::doc}}
\sphinxAtStartPar
양봉이면서 거래량이 갑자기 증가하는 날을 카운트하고, 수익율과의 상관관계를 보겠습니다.

\begin{sphinxuseclass}{cell}\begin{sphinxVerbatimInput}

\begin{sphinxuseclass}{cell_input}
\begin{sphinxVerbatim}[commandchars=\\\{\}]
\PYG{n}{mdl\PYGZus{}data} \PYG{o}{=} \PYG{n}{pd}\PYG{o}{.}\PYG{n}{read\PYGZus{}pickle}\PYG{p}{(}\PYG{l+s+s1}{\PYGZsq{}}\PYG{l+s+s1}{mdl\PYGZus{}data.pkl}\PYG{l+s+s1}{\PYGZsq{}}\PYG{p}{)}
\PYG{n}{mdl\PYGZus{}data}\PYG{o}{.}\PYG{n}{head}\PYG{p}{(}\PYG{p}{)}\PYG{o}{.}\PYG{n}{style}\PYG{o}{.}\PYG{n}{set\PYGZus{}table\PYGZus{}attributes}\PYG{p}{(}\PYG{l+s+s1}{\PYGZsq{}}\PYG{l+s+s1}{style=}\PYG{l+s+s1}{\PYGZdq{}}\PYG{l+s+s1}{font\PYGZhy{}size: 12px}\PYG{l+s+s1}{\PYGZdq{}}\PYG{l+s+s1}{\PYGZsq{}}\PYG{p}{)}
\end{sphinxVerbatim}

\end{sphinxuseclass}\end{sphinxVerbatimInput}
\begin{sphinxVerbatimOutput}

\begin{sphinxuseclass}{cell_output}
\begin{sphinxVerbatim}[commandchars=\\\{\}]
\PYGZlt{}pandas.io.formats.style.Styler at 0x2126ac19f40\PYGZgt{}
\end{sphinxVerbatim}

\end{sphinxuseclass}\end{sphinxVerbatimOutput}

\end{sphinxuseclass}
\begin{sphinxuseclass}{cell}\begin{sphinxVerbatimInput}

\begin{sphinxuseclass}{cell_input}
\begin{sphinxVerbatim}[commandchars=\\\{\}]
\PYG{n}{kosdaq\PYGZus{}list} \PYG{o}{=} \PYG{n}{pd}\PYG{o}{.}\PYG{n}{read\PYGZus{}pickle}\PYG{p}{(}\PYG{l+s+s1}{\PYGZsq{}}\PYG{l+s+s1}{kosdaq\PYGZus{}list.pkl}\PYG{l+s+s1}{\PYGZsq{}}\PYG{p}{)}

\PYG{n}{data\PYGZus{}h4} \PYG{o}{=} \PYG{n}{pd}\PYG{o}{.}\PYG{n}{DataFrame}\PYG{p}{(}\PYG{p}{)}

\PYG{k}{for} \PYG{n}{code} \PYG{o+ow}{in} \PYG{n}{kosdaq\PYGZus{}list}\PYG{p}{[}\PYG{l+s+s1}{\PYGZsq{}}\PYG{l+s+s1}{code}\PYG{l+s+s1}{\PYGZsq{}}\PYG{p}{]}\PYG{p}{:}

    \PYG{n}{data} \PYG{o}{=} \PYG{n}{mdl\PYGZus{}data}\PYG{p}{[}\PYG{n}{mdl\PYGZus{}data}\PYG{p}{[}\PYG{l+s+s1}{\PYGZsq{}}\PYG{l+s+s1}{code}\PYG{l+s+s1}{\PYGZsq{}}\PYG{p}{]}\PYG{o}{==}\PYG{n}{code}\PYG{p}{]}\PYG{o}{.}\PYG{n}{sort\PYGZus{}index}\PYG{p}{(}\PYG{p}{)}\PYG{o}{.}\PYG{n}{copy}\PYG{p}{(}\PYG{p}{)}
    
    \PYG{n}{data}\PYG{p}{[}\PYG{l+s+s1}{\PYGZsq{}}\PYG{l+s+s1}{volume\PYGZus{}mean}\PYG{l+s+s1}{\PYGZsq{}}\PYG{p}{]} \PYG{o}{=} \PYG{n}{data}\PYG{p}{[}\PYG{l+s+s1}{\PYGZsq{}}\PYG{l+s+s1}{volume}\PYG{l+s+s1}{\PYGZsq{}}\PYG{p}{]}\PYG{o}{.}\PYG{n}{rolling}\PYG{p}{(}\PYG{l+m+mi}{60}\PYG{p}{)}\PYG{o}{.}\PYG{n}{mean}\PYG{p}{(}\PYG{p}{)} \PYG{c+c1}{\PYGZsh{} 60일 이동평균값}
    \PYG{n}{data}\PYG{p}{[}\PYG{l+s+s1}{\PYGZsq{}}\PYG{l+s+s1}{volume\PYGZus{}std}\PYG{l+s+s1}{\PYGZsq{}}\PYG{p}{]} \PYG{o}{=} \PYG{n}{data}\PYG{p}{[}\PYG{l+s+s1}{\PYGZsq{}}\PYG{l+s+s1}{volume}\PYG{l+s+s1}{\PYGZsq{}}\PYG{p}{]}\PYG{o}{.}\PYG{n}{rolling}\PYG{p}{(}\PYG{l+m+mi}{60}\PYG{p}{)}\PYG{o}{.}\PYG{n}{std}\PYG{p}{(}\PYG{p}{)} \PYG{c+c1}{\PYGZsh{} 60일 이동평균값}
    \PYG{n}{data}\PYG{p}{[}\PYG{l+s+s1}{\PYGZsq{}}\PYG{l+s+s1}{volume\PYGZus{}z}\PYG{l+s+s1}{\PYGZsq{}}\PYG{p}{]} \PYG{o}{=} \PYG{p}{(}\PYG{n}{data}\PYG{p}{[}\PYG{l+s+s1}{\PYGZsq{}}\PYG{l+s+s1}{volume}\PYG{l+s+s1}{\PYGZsq{}}\PYG{p}{]} \PYG{o}{\PYGZhy{}} \PYG{n}{data}\PYG{p}{[}\PYG{l+s+s1}{\PYGZsq{}}\PYG{l+s+s1}{volume\PYGZus{}mean}\PYG{l+s+s1}{\PYGZsq{}}\PYG{p}{]}\PYG{p}{)}\PYG{o}{/}\PYG{n}{data}\PYG{p}{[}\PYG{l+s+s1}{\PYGZsq{}}\PYG{l+s+s1}{volume\PYGZus{}std}\PYG{l+s+s1}{\PYGZsq{}}\PYG{p}{]} \PYG{c+c1}{\PYGZsh{} 거래량은 종목과 주가에 따라 다르기 떄문에 표준화한 값이 필요함}
    \PYG{n}{data}\PYG{p}{[}\PYG{l+s+s1}{\PYGZsq{}}\PYG{l+s+s1}{z\PYGZgt{}1.96}\PYG{l+s+s1}{\PYGZsq{}}\PYG{p}{]} \PYG{o}{=} \PYG{p}{(}\PYG{n}{data}\PYG{p}{[}\PYG{l+s+s1}{\PYGZsq{}}\PYG{l+s+s1}{close}\PYG{l+s+s1}{\PYGZsq{}}\PYG{p}{]} \PYG{o}{\PYGZgt{}} \PYG{n}{data}\PYG{p}{[}\PYG{l+s+s1}{\PYGZsq{}}\PYG{l+s+s1}{open}\PYG{l+s+s1}{\PYGZsq{}}\PYG{p}{]}\PYG{p}{)}\PYG{o}{*}\PYG{p}{(}\PYG{n}{data}\PYG{p}{[}\PYG{l+s+s1}{\PYGZsq{}}\PYG{l+s+s1}{volume\PYGZus{}z}\PYG{l+s+s1}{\PYGZsq{}}\PYG{p}{]} \PYG{o}{\PYGZgt{}} \PYG{l+m+mf}{1.65}\PYG{p}{)}\PYG{o}{.}\PYG{n}{astype}\PYG{p}{(}\PYG{n+nb}{int}\PYG{p}{)} \PYG{c+c1}{\PYGZsh{} 양봉이면서 거래량이 90\PYGZpc{}신뢰구간을 벗어난 날}
    \PYG{n}{data}\PYG{p}{[}\PYG{l+s+s1}{\PYGZsq{}}\PYG{l+s+s1}{num\PYGZus{}z\PYGZgt{}1.96}\PYG{l+s+s1}{\PYGZsq{}}\PYG{p}{]} \PYG{o}{=}  \PYG{n}{data}\PYG{p}{[}\PYG{l+s+s1}{\PYGZsq{}}\PYG{l+s+s1}{z\PYGZgt{}1.96}\PYG{l+s+s1}{\PYGZsq{}}\PYG{p}{]}\PYG{o}{.}\PYG{n}{rolling}\PYG{p}{(}\PYG{l+m+mi}{60}\PYG{p}{)}\PYG{o}{.}\PYG{n}{sum}\PYG{p}{(}\PYG{p}{)}  \PYG{c+c1}{\PYGZsh{} 양봉이면서 거래량이 90\PYGZpc{} 신뢰구간을 벗어난 날을 카운트}
       
    \PYG{n}{data}\PYG{p}{[}\PYG{l+s+s1}{\PYGZsq{}}\PYG{l+s+s1}{max\PYGZus{}close}\PYG{l+s+s1}{\PYGZsq{}}\PYG{p}{]}  \PYG{o}{=} \PYG{n}{data}\PYG{p}{[}\PYG{p}{[}\PYG{l+s+s1}{\PYGZsq{}}\PYG{l+s+s1}{close\PYGZus{}r1}\PYG{l+s+s1}{\PYGZsq{}}\PYG{p}{,}\PYG{l+s+s1}{\PYGZsq{}}\PYG{l+s+s1}{close\PYGZus{}r2}\PYG{l+s+s1}{\PYGZsq{}}\PYG{p}{,}\PYG{l+s+s1}{\PYGZsq{}}\PYG{l+s+s1}{close\PYGZus{}r3}\PYG{l+s+s1}{\PYGZsq{}}\PYG{p}{,}\PYG{l+s+s1}{\PYGZsq{}}\PYG{l+s+s1}{close\PYGZus{}r4}\PYG{l+s+s1}{\PYGZsq{}}\PYG{p}{,}\PYG{l+s+s1}{\PYGZsq{}}\PYG{l+s+s1}{close\PYGZus{}r5}\PYG{l+s+s1}{\PYGZsq{}}\PYG{p}{]}\PYG{p}{]}\PYG{o}{.}\PYG{n}{max}\PYG{p}{(}\PYG{n}{axis}\PYG{o}{=}\PYG{l+m+mi}{1}\PYG{p}{)} \PYG{c+c1}{\PYGZsh{} 5 영업일 종가 수익율 중 최고 값}
    \PYG{n}{data}\PYG{o}{.}\PYG{n}{dropna}\PYG{p}{(}\PYG{n}{subset}\PYG{o}{=}\PYG{p}{[}\PYG{l+s+s1}{\PYGZsq{}}\PYG{l+s+s1}{volume\PYGZus{}mean}\PYG{l+s+s1}{\PYGZsq{}}\PYG{p}{,}\PYG{l+s+s1}{\PYGZsq{}}\PYG{l+s+s1}{close\PYGZus{}r1}\PYG{l+s+s1}{\PYGZsq{}}\PYG{p}{,}\PYG{l+s+s1}{\PYGZsq{}}\PYG{l+s+s1}{close\PYGZus{}r2}\PYG{l+s+s1}{\PYGZsq{}}\PYG{p}{,}\PYG{l+s+s1}{\PYGZsq{}}\PYG{l+s+s1}{close\PYGZus{}r3}\PYG{l+s+s1}{\PYGZsq{}}\PYG{p}{,}\PYG{l+s+s1}{\PYGZsq{}}\PYG{l+s+s1}{close\PYGZus{}r4}\PYG{l+s+s1}{\PYGZsq{}}\PYG{p}{,}\PYG{l+s+s1}{\PYGZsq{}}\PYG{l+s+s1}{close\PYGZus{}r5}\PYG{l+s+s1}{\PYGZsq{}}\PYG{p}{]}\PYG{p}{,} \PYG{n}{inplace}\PYG{o}{=}\PYG{k+kc}{True}\PYG{p}{)} \PYG{c+c1}{\PYGZsh{} missing 이 있는 행은 제거  }
    
    \PYG{n}{data\PYGZus{}h4} \PYG{o}{=} \PYG{n}{pd}\PYG{o}{.}\PYG{n}{concat}\PYG{p}{(}\PYG{p}{[}\PYG{n}{data}\PYG{p}{,} \PYG{n}{data\PYGZus{}h4}\PYG{p}{]}\PYG{p}{,} \PYG{n}{axis}\PYG{o}{=}\PYG{l+m+mi}{0}\PYG{p}{)}

\PYG{n}{data\PYGZus{}h4}\PYG{o}{.}\PYG{n}{to\PYGZus{}pickle}\PYG{p}{(}\PYG{l+s+s1}{\PYGZsq{}}\PYG{l+s+s1}{data\PYGZus{}h4.pkl}\PYG{l+s+s1}{\PYGZsq{}}\PYG{p}{)}  
\end{sphinxVerbatim}

\end{sphinxuseclass}\end{sphinxVerbatimInput}

\end{sphinxuseclass}
\sphinxAtStartPar
 거래량이 갑자기 많아지고 양봉인 날을 카운트하고 그 갯 수에 따라 수익율의 변화를 봤습니다. 전체적으로 거래량이 갑자기 증가하는 날이 많을 수 록 수익율이 증가하는 패턴을 보여줍니다. 결과의 마지막 num\_z 가 15일인 경우는 수익율이 급강하했는데요. 실제로 너무 많으면 수익율이 안 좋은 것인지 여부는 해당 레코드 수(47개)가 많지 않아 신뢰하기 어렵습니다.

\begin{sphinxuseclass}{cell}\begin{sphinxVerbatimInput}

\begin{sphinxuseclass}{cell_input}
\begin{sphinxVerbatim}[commandchars=\\\{\}]
\PYG{n}{data\PYGZus{}h4} \PYG{o}{=} \PYG{n}{pd}\PYG{o}{.}\PYG{n}{read\PYGZus{}pickle}\PYG{p}{(}\PYG{l+s+s1}{\PYGZsq{}}\PYG{l+s+s1}{data\PYGZus{}h4.pkl}\PYG{l+s+s1}{\PYGZsq{}}\PYG{p}{)} 
\PYG{n+nb}{print}\PYG{p}{(}\PYG{n}{data\PYGZus{}h4}\PYG{o}{.}\PYG{n}{groupby}\PYG{p}{(}\PYG{l+s+s1}{\PYGZsq{}}\PYG{l+s+s1}{num\PYGZus{}z\PYGZgt{}1.96}\PYG{l+s+s1}{\PYGZsq{}}\PYG{p}{)}\PYG{p}{[}\PYG{l+s+s1}{\PYGZsq{}}\PYG{l+s+s1}{max\PYGZus{}close}\PYG{l+s+s1}{\PYGZsq{}}\PYG{p}{]}\PYG{o}{.}\PYG{n}{agg}\PYG{p}{(}\PYG{p}{[}\PYG{l+s+s1}{\PYGZsq{}}\PYG{l+s+s1}{count}\PYG{l+s+s1}{\PYGZsq{}}\PYG{p}{,}\PYG{l+s+s1}{\PYGZsq{}}\PYG{l+s+s1}{mean}\PYG{l+s+s1}{\PYGZsq{}}\PYG{p}{]}\PYG{p}{)}\PYG{p}{)}
\PYG{n}{data\PYGZus{}h4}\PYG{o}{.}\PYG{n}{groupby}\PYG{p}{(}\PYG{l+s+s1}{\PYGZsq{}}\PYG{l+s+s1}{num\PYGZus{}z\PYGZgt{}1.96}\PYG{l+s+s1}{\PYGZsq{}}\PYG{p}{)}\PYG{p}{[}\PYG{l+s+s1}{\PYGZsq{}}\PYG{l+s+s1}{max\PYGZus{}close}\PYG{l+s+s1}{\PYGZsq{}}\PYG{p}{]}\PYG{o}{.}\PYG{n}{mean}\PYG{p}{(}\PYG{p}{)}\PYG{o}{.}\PYG{n}{plot}\PYG{p}{(}\PYG{n}{figsize}\PYG{o}{=}\PYG{p}{(}\PYG{l+m+mi}{12}\PYG{p}{,}\PYG{l+m+mi}{5}\PYG{p}{)}\PYG{p}{,} \PYG{n}{kind}\PYG{o}{=}\PYG{l+s+s1}{\PYGZsq{}}\PYG{l+s+s1}{bar}\PYG{l+s+s1}{\PYGZsq{}}\PYG{p}{,} \PYG{n}{ylim}\PYG{o}{=}\PYG{p}{(}\PYG{l+m+mf}{1.01}\PYG{p}{,} \PYG{l+m+mf}{1.05}\PYG{p}{)}\PYG{p}{)}
\end{sphinxVerbatim}

\end{sphinxuseclass}\end{sphinxVerbatimInput}
\begin{sphinxVerbatimOutput}

\begin{sphinxuseclass}{cell_output}
\begin{sphinxVerbatim}[commandchars=\\\{\}]
             count  mean
num\PYGZus{}z\PYGZgt{}1.96              
0.000       105320 1.027
1.000        78911 1.031
2.000        57860 1.032
3.000        39476 1.035
4.000        25244 1.035
5.000        15428 1.038
6.000         9250 1.040
7.000         4841 1.045
8.000         3210 1.048
9.000         1505 1.044
10.000         904 1.046
11.000         450 1.047
12.000         318 1.035
13.000          56 1.046
14.000          94 1.048
15.000          47 1.002
\end{sphinxVerbatim}

\begin{sphinxVerbatim}[commandchars=\\\{\}]
\PYGZlt{}AxesSubplot:xlabel=\PYGZsq{}num\PYGZus{}z\PYGZgt{}1.96\PYGZsq{}\PYGZgt{}
\end{sphinxVerbatim}

\noindent\sphinxincludegraphics{{5.1.4_Hypothesis_4_5_2}.png}

\end{sphinxuseclass}\end{sphinxVerbatimOutput}

\end{sphinxuseclass}
\begin{sphinxuseclass}{cell}\begin{sphinxVerbatimInput}

\begin{sphinxuseclass}{cell_input}
\begin{sphinxVerbatim}[commandchars=\\\{\}]
\PYG{k+kn}{import} \PYG{n+nn}{FinanceDataReader} \PYG{k}{as} \PYG{n+nn}{fdr}
\PYG{o}{\PYGZpc{}}\PYG{k}{matplotlib} inline
\PYG{k+kn}{import} \PYG{n+nn}{matplotlib}\PYG{n+nn}{.}\PYG{n+nn}{pyplot} \PYG{k}{as} \PYG{n+nn}{plt}
\PYG{k+kn}{import} \PYG{n+nn}{pandas} \PYG{k}{as} \PYG{n+nn}{pd}
\PYG{k+kn}{import} \PYG{n+nn}{numpy} \PYG{k}{as} \PYG{n+nn}{np}

\PYG{n}{pd}\PYG{o}{.}\PYG{n}{options}\PYG{o}{.}\PYG{n}{display}\PYG{o}{.}\PYG{n}{float\PYGZus{}format} \PYG{o}{=} \PYG{l+s+s1}{\PYGZsq{}}\PYG{l+s+si}{\PYGZob{}:,.3f\PYGZcb{}}\PYG{l+s+s1}{\PYGZsq{}}\PYG{o}{.}\PYG{n}{format}
\end{sphinxVerbatim}

\end{sphinxuseclass}\end{sphinxVerbatimInput}

\end{sphinxuseclass}

\section{주가지수보다 더 좋은 수익율을 자주 보여준다.}
\label{\detokenize{chapter5/5.1.5_Hypothesis_5:id1}}\label{\detokenize{chapter5/5.1.5_Hypothesis_5::doc}}
\begin{sphinxuseclass}{cell}\begin{sphinxVerbatimInput}

\begin{sphinxuseclass}{cell_input}
\begin{sphinxVerbatim}[commandchars=\\\{\}]
\PYG{n}{mdl\PYGZus{}data} \PYG{o}{=} \PYG{n}{pd}\PYG{o}{.}\PYG{n}{read\PYGZus{}pickle}\PYG{p}{(}\PYG{l+s+s1}{\PYGZsq{}}\PYG{l+s+s1}{mdl\PYGZus{}data.pkl}\PYG{l+s+s1}{\PYGZsq{}}\PYG{p}{)}
\PYG{n}{mdl\PYGZus{}data}\PYG{o}{.}\PYG{n}{head}\PYG{p}{(}\PYG{p}{)}\PYG{o}{.}\PYG{n}{style}\PYG{o}{.}\PYG{n}{set\PYGZus{}table\PYGZus{}attributes}\PYG{p}{(}\PYG{l+s+s1}{\PYGZsq{}}\PYG{l+s+s1}{style=}\PYG{l+s+s1}{\PYGZdq{}}\PYG{l+s+s1}{font\PYGZhy{}size: 12px}\PYG{l+s+s1}{\PYGZdq{}}\PYG{l+s+s1}{\PYGZsq{}}\PYG{p}{)}\PYG{o}{.}\PYG{n}{format}\PYG{p}{(}\PYG{n}{precision}\PYG{o}{=}\PYG{l+m+mi}{3}\PYG{p}{)}
\end{sphinxVerbatim}

\end{sphinxuseclass}\end{sphinxVerbatimInput}
\begin{sphinxVerbatimOutput}

\begin{sphinxuseclass}{cell_output}
\begin{sphinxVerbatim}[commandchars=\\\{\}]
\PYGZlt{}pandas.io.formats.style.Styler at 0x198f86a8fa0\PYGZgt{}
\end{sphinxVerbatim}

\end{sphinxuseclass}\end{sphinxVerbatimOutput}

\end{sphinxuseclass}
\sphinxAtStartPar
 이전 장에서 일봉 데이터에 KOSDAQ 주가지수 데이터를 추가한 후, 주가지수 수익율이 1 보다 작은 날, 종목 수익율이 1 보다 크면 win\_market 이라는 변수에 1 을 담아 두도록 했습니다. win\_market 의 과거 60일 동안 합계와 미래 수익율과의 관계를 보겠습니다. 별도로 주가지수 수익률 대비 종목 수익율의 비율을 새로운 변수로 만들어, 미래 수익율과의 상관관계도 볼 수 있도록 하겠습니다.

\begin{sphinxuseclass}{cell}\begin{sphinxVerbatimInput}

\begin{sphinxuseclass}{cell_input}
\begin{sphinxVerbatim}[commandchars=\\\{\}]
\PYG{n}{kosdaq\PYGZus{}list} \PYG{o}{=} \PYG{n}{pd}\PYG{o}{.}\PYG{n}{read\PYGZus{}pickle}\PYG{p}{(}\PYG{l+s+s1}{\PYGZsq{}}\PYG{l+s+s1}{kosdaq\PYGZus{}list.pkl}\PYG{l+s+s1}{\PYGZsq{}}\PYG{p}{)}

\PYG{n}{data\PYGZus{}h5} \PYG{o}{=} \PYG{n}{pd}\PYG{o}{.}\PYG{n}{DataFrame}\PYG{p}{(}\PYG{p}{)}

\PYG{k}{for} \PYG{n}{code} \PYG{o+ow}{in} \PYG{n}{kosdaq\PYGZus{}list}\PYG{p}{[}\PYG{l+s+s1}{\PYGZsq{}}\PYG{l+s+s1}{code}\PYG{l+s+s1}{\PYGZsq{}}\PYG{p}{]}\PYG{p}{:}
    
    \PYG{c+c1}{\PYGZsh{} 종목별 처리}
    \PYG{n}{data} \PYG{o}{=} \PYG{n}{mdl\PYGZus{}data}\PYG{p}{[}\PYG{n}{mdl\PYGZus{}data}\PYG{p}{[}\PYG{l+s+s1}{\PYGZsq{}}\PYG{l+s+s1}{code}\PYG{l+s+s1}{\PYGZsq{}}\PYG{p}{]}\PYG{o}{==}\PYG{n}{code}\PYG{p}{]}\PYG{o}{.}\PYG{n}{sort\PYGZus{}index}\PYG{p}{(}\PYG{p}{)}\PYG{o}{.}\PYG{n}{copy}\PYG{p}{(}\PYG{p}{)}
    
    \PYG{c+c1}{\PYGZsh{} 과거 60일 win\PYGZus{}market 누적 합}
    \PYG{n}{data}\PYG{p}{[}\PYG{l+s+s1}{\PYGZsq{}}\PYG{l+s+s1}{num\PYGZus{}win\PYGZus{}market}\PYG{l+s+s1}{\PYGZsq{}}\PYG{p}{]} \PYG{o}{=} \PYG{n}{data}\PYG{p}{[}\PYG{l+s+s1}{\PYGZsq{}}\PYG{l+s+s1}{win\PYGZus{}market}\PYG{l+s+s1}{\PYGZsq{}}\PYG{p}{]}\PYG{o}{.}\PYG{n}{rolling}\PYG{p}{(}\PYG{l+m+mi}{60}\PYG{p}{)}\PYG{o}{.}\PYG{n}{sum}\PYG{p}{(}\PYG{p}{)} \PYG{c+c1}{\PYGZsh{} 주가지수 수익율이 1 보다 작을 때, 종목 수익율이 1 보다 큰 날 수}
    \PYG{n}{data}\PYG{p}{[}\PYG{l+s+s1}{\PYGZsq{}}\PYG{l+s+s1}{pct\PYGZus{}win\PYGZus{}market}\PYG{l+s+s1}{\PYGZsq{}}\PYG{p}{]} \PYG{o}{=} \PYG{p}{(}\PYG{n}{data}\PYG{p}{[}\PYG{l+s+s1}{\PYGZsq{}}\PYG{l+s+s1}{return}\PYG{l+s+s1}{\PYGZsq{}}\PYG{p}{]}\PYG{o}{/}\PYG{n}{data}\PYG{p}{[}\PYG{l+s+s1}{\PYGZsq{}}\PYG{l+s+s1}{kosdaq\PYGZus{}return}\PYG{l+s+s1}{\PYGZsq{}}\PYG{p}{]}\PYG{p}{)}\PYG{o}{.}\PYG{n}{rolling}\PYG{p}{(}\PYG{l+m+mi}{60}\PYG{p}{)}\PYG{o}{.}\PYG{n}{mean}\PYG{p}{(}\PYG{p}{)} \PYG{c+c1}{\PYGZsh{} 주가지수 수익율 대비 종목 수익율}
        
    
    \PYG{c+c1}{\PYGZsh{} 고가, 저가, 종가 수익율}
    \PYG{k}{for} \PYG{n}{i} \PYG{o+ow}{in} \PYG{p}{[}\PYG{l+m+mi}{1}\PYG{p}{,}\PYG{l+m+mi}{2}\PYG{p}{,}\PYG{l+m+mi}{3}\PYG{p}{,}\PYG{l+m+mi}{4}\PYG{p}{,}\PYG{l+m+mi}{5}\PYG{p}{]}\PYG{p}{:}

        \PYG{n}{data}\PYG{p}{[}\PYG{l+s+s1}{\PYGZsq{}}\PYG{l+s+s1}{high\PYGZus{}r}\PYG{l+s+s1}{\PYGZsq{}} \PYG{o}{+} \PYG{n+nb}{str}\PYG{p}{(}\PYG{n}{i}\PYG{p}{)}\PYG{p}{]} \PYG{o}{=} \PYG{n}{data}\PYG{p}{[}\PYG{l+s+s1}{\PYGZsq{}}\PYG{l+s+s1}{high}\PYG{l+s+s1}{\PYGZsq{}}\PYG{p}{]}\PYG{o}{.}\PYG{n}{shift}\PYG{p}{(}\PYG{o}{\PYGZhy{}}\PYG{l+m+mi}{1}\PYG{o}{*}\PYG{n}{i}\PYG{p}{)}\PYG{o}{/}\PYG{n}{data}\PYG{p}{[}\PYG{l+s+s1}{\PYGZsq{}}\PYG{l+s+s1}{close}\PYG{l+s+s1}{\PYGZsq{}}\PYG{p}{]}      
        \PYG{n}{data}\PYG{p}{[}\PYG{l+s+s1}{\PYGZsq{}}\PYG{l+s+s1}{low\PYGZus{}r}\PYG{l+s+s1}{\PYGZsq{}} \PYG{o}{+} \PYG{n+nb}{str}\PYG{p}{(}\PYG{n}{i}\PYG{p}{)}\PYG{p}{]} \PYG{o}{=} \PYG{n}{data}\PYG{p}{[}\PYG{l+s+s1}{\PYGZsq{}}\PYG{l+s+s1}{low}\PYG{l+s+s1}{\PYGZsq{}}\PYG{p}{]}\PYG{o}{.}\PYG{n}{shift}\PYG{p}{(}\PYG{o}{\PYGZhy{}}\PYG{l+m+mi}{1}\PYG{o}{*}\PYG{n}{i}\PYG{p}{)}\PYG{o}{/}\PYG{n}{data}\PYG{p}{[}\PYG{l+s+s1}{\PYGZsq{}}\PYG{l+s+s1}{close}\PYG{l+s+s1}{\PYGZsq{}}\PYG{p}{]}   
        \PYG{n}{data}\PYG{p}{[}\PYG{l+s+s1}{\PYGZsq{}}\PYG{l+s+s1}{close\PYGZus{}r}\PYG{l+s+s1}{\PYGZsq{}} \PYG{o}{+} \PYG{n+nb}{str}\PYG{p}{(}\PYG{n}{i}\PYG{p}{)}\PYG{p}{]} \PYG{o}{=} \PYG{n}{data}\PYG{p}{[}\PYG{l+s+s1}{\PYGZsq{}}\PYG{l+s+s1}{close}\PYG{l+s+s1}{\PYGZsq{}}\PYG{p}{]}\PYG{o}{.}\PYG{n}{shift}\PYG{p}{(}\PYG{o}{\PYGZhy{}}\PYG{l+m+mi}{1}\PYG{o}{*}\PYG{n}{i}\PYG{p}{)}\PYG{o}{/}\PYG{n}{data}\PYG{p}{[}\PYG{l+s+s1}{\PYGZsq{}}\PYG{l+s+s1}{close}\PYG{l+s+s1}{\PYGZsq{}}\PYG{p}{]}    
        
    \PYG{n}{data}\PYG{p}{[}\PYG{l+s+s1}{\PYGZsq{}}\PYG{l+s+s1}{max\PYGZus{}close}\PYG{l+s+s1}{\PYGZsq{}}\PYG{p}{]}  \PYG{o}{=} \PYG{n}{data}\PYG{p}{[}\PYG{p}{[}\PYG{l+s+s1}{\PYGZsq{}}\PYG{l+s+s1}{close\PYGZus{}r1}\PYG{l+s+s1}{\PYGZsq{}}\PYG{p}{,}\PYG{l+s+s1}{\PYGZsq{}}\PYG{l+s+s1}{close\PYGZus{}r2}\PYG{l+s+s1}{\PYGZsq{}}\PYG{p}{,}\PYG{l+s+s1}{\PYGZsq{}}\PYG{l+s+s1}{close\PYGZus{}r3}\PYG{l+s+s1}{\PYGZsq{}}\PYG{p}{,}\PYG{l+s+s1}{\PYGZsq{}}\PYG{l+s+s1}{close\PYGZus{}r4}\PYG{l+s+s1}{\PYGZsq{}}\PYG{p}{,}\PYG{l+s+s1}{\PYGZsq{}}\PYG{l+s+s1}{close\PYGZus{}r5}\PYG{l+s+s1}{\PYGZsq{}}\PYG{p}{]}\PYG{p}{]}\PYG{o}{.}\PYG{n}{max}\PYG{p}{(}\PYG{n}{axis}\PYG{o}{=}\PYG{l+m+mi}{1}\PYG{p}{)} \PYG{c+c1}{\PYGZsh{} 5 영업일 종가 수익율 중 최고 값}
    \PYG{n}{data}\PYG{o}{.}\PYG{n}{dropna}\PYG{p}{(}\PYG{n}{subset}\PYG{o}{=}\PYG{p}{[}\PYG{l+s+s1}{\PYGZsq{}}\PYG{l+s+s1}{num\PYGZus{}win\PYGZus{}market}\PYG{l+s+s1}{\PYGZsq{}}\PYG{p}{,}\PYG{l+s+s1}{\PYGZsq{}}\PYG{l+s+s1}{close\PYGZus{}r1}\PYG{l+s+s1}{\PYGZsq{}}\PYG{p}{,}\PYG{l+s+s1}{\PYGZsq{}}\PYG{l+s+s1}{close\PYGZus{}r2}\PYG{l+s+s1}{\PYGZsq{}}\PYG{p}{,}\PYG{l+s+s1}{\PYGZsq{}}\PYG{l+s+s1}{close\PYGZus{}r3}\PYG{l+s+s1}{\PYGZsq{}}\PYG{p}{,}\PYG{l+s+s1}{\PYGZsq{}}\PYG{l+s+s1}{close\PYGZus{}r4}\PYG{l+s+s1}{\PYGZsq{}}\PYG{p}{,}\PYG{l+s+s1}{\PYGZsq{}}\PYG{l+s+s1}{close\PYGZus{}r5}\PYG{l+s+s1}{\PYGZsq{}}\PYG{p}{]}\PYG{p}{,} \PYG{n}{inplace}\PYG{o}{=}\PYG{k+kc}{True}\PYG{p}{)} \PYG{c+c1}{\PYGZsh{} missing 이 있는 행은 제거   }
 
    \PYG{n}{data\PYGZus{}h5} \PYG{o}{=} \PYG{n}{pd}\PYG{o}{.}\PYG{n}{concat}\PYG{p}{(}\PYG{p}{[}\PYG{n}{data}\PYG{p}{,} \PYG{n}{data\PYGZus{}h5}\PYG{p}{]}\PYG{p}{,} \PYG{n}{axis}\PYG{o}{=}\PYG{l+m+mi}{0}\PYG{p}{)}

\PYG{n}{data\PYGZus{}h5}\PYG{o}{.}\PYG{n}{to\PYGZus{}pickle}\PYG{p}{(}\PYG{l+s+s1}{\PYGZsq{}}\PYG{l+s+s1}{data\PYGZus{}h5.pkl}\PYG{l+s+s1}{\PYGZsq{}}\PYG{p}{)}    
\PYG{n}{data\PYGZus{}h5}\PYG{o}{.}\PYG{n}{head}\PYG{p}{(}\PYG{p}{)}\PYG{o}{.}\PYG{n}{style}\PYG{o}{.}\PYG{n}{set\PYGZus{}table\PYGZus{}attributes}\PYG{p}{(}\PYG{l+s+s1}{\PYGZsq{}}\PYG{l+s+s1}{style=}\PYG{l+s+s1}{\PYGZdq{}}\PYG{l+s+s1}{font\PYGZhy{}size: 12px}\PYG{l+s+s1}{\PYGZdq{}}\PYG{l+s+s1}{\PYGZsq{}}\PYG{p}{)}\PYG{o}{.}\PYG{n}{format}\PYG{p}{(}\PYG{n}{precision}\PYG{o}{=}\PYG{l+m+mi}{3}\PYG{p}{)}   
\end{sphinxVerbatim}

\end{sphinxuseclass}\end{sphinxVerbatimInput}
\begin{sphinxVerbatimOutput}

\begin{sphinxuseclass}{cell_output}
\begin{sphinxVerbatim}[commandchars=\\\{\}]
\PYGZlt{}pandas.io.formats.style.Styler at 0x1988126f940\PYGZgt{}
\end{sphinxVerbatim}

\end{sphinxuseclass}\end{sphinxVerbatimOutput}

\end{sphinxuseclass}
\sphinxAtStartPar
 예상한 바와 같이 주가지수가 빠질 때, 수익율이 좋았던 종목들은 미래 수익율이 좋게 나타났습니다.

\begin{sphinxuseclass}{cell}\begin{sphinxVerbatimInput}

\begin{sphinxuseclass}{cell_input}
\begin{sphinxVerbatim}[commandchars=\\\{\}]
\PYG{n}{data\PYGZus{}h5} \PYG{o}{=} \PYG{n}{pd}\PYG{o}{.}\PYG{n}{read\PYGZus{}pickle}\PYG{p}{(}\PYG{l+s+s1}{\PYGZsq{}}\PYG{l+s+s1}{data\PYGZus{}h5.pkl}\PYG{l+s+s1}{\PYGZsq{}}\PYG{p}{)}    
\PYG{n}{ranks} \PYG{o}{=} \PYG{n}{pd}\PYG{o}{.}\PYG{n}{qcut}\PYG{p}{(}\PYG{n}{data\PYGZus{}h5}\PYG{p}{[}\PYG{l+s+s1}{\PYGZsq{}}\PYG{l+s+s1}{num\PYGZus{}win\PYGZus{}market}\PYG{l+s+s1}{\PYGZsq{}}\PYG{p}{]}\PYG{p}{,} \PYG{n}{q}\PYG{o}{=}\PYG{l+m+mi}{8}\PYG{p}{)}
\PYG{n+nb}{print}\PYG{p}{(}\PYG{n}{data\PYGZus{}h5}\PYG{o}{.}\PYG{n}{groupby}\PYG{p}{(}\PYG{n}{ranks}\PYG{p}{)}\PYG{p}{[}\PYG{l+s+s1}{\PYGZsq{}}\PYG{l+s+s1}{max\PYGZus{}close}\PYG{l+s+s1}{\PYGZsq{}}\PYG{p}{]}\PYG{o}{.}\PYG{n}{mean}\PYG{p}{(}\PYG{p}{)}\PYG{p}{)}
\PYG{n}{data\PYGZus{}h5}\PYG{o}{.}\PYG{n}{groupby}\PYG{p}{(}\PYG{n}{ranks}\PYG{p}{)}\PYG{p}{[}\PYG{l+s+s1}{\PYGZsq{}}\PYG{l+s+s1}{max\PYGZus{}close}\PYG{l+s+s1}{\PYGZsq{}}\PYG{p}{]}\PYG{o}{.}\PYG{n}{mean}\PYG{p}{(}\PYG{p}{)}\PYG{o}{.}\PYG{n}{plot}\PYG{p}{(}\PYG{n}{figsize}\PYG{o}{=}\PYG{p}{(}\PYG{l+m+mi}{12}\PYG{p}{,}\PYG{l+m+mi}{5}\PYG{p}{)}\PYG{p}{)}
\end{sphinxVerbatim}

\end{sphinxuseclass}\end{sphinxVerbatimInput}
\begin{sphinxVerbatimOutput}

\begin{sphinxuseclass}{cell_output}
\begin{sphinxVerbatim}[commandchars=\\\{\}]
num\PYGZus{}win\PYGZus{}market
(\PYGZhy{}0.001, 4.0]   1.022
(4.0, 5.0]      1.031
(5.0, 6.0]      1.031
(6.0, 7.0]      1.032
(7.0, 8.0]      1.032
(8.0, 9.0]      1.034
(9.0, 11.0]     1.036
(11.0, 22.0]    1.040
Name: max\PYGZus{}close, dtype: float64
\end{sphinxVerbatim}

\begin{sphinxVerbatim}[commandchars=\\\{\}]
\PYGZlt{}AxesSubplot:xlabel=\PYGZsq{}num\PYGZus{}win\PYGZus{}market\PYGZsq{}\PYGZgt{}
\end{sphinxVerbatim}

\noindent\sphinxincludegraphics{{5.1.5_Hypothesis_5_6_2}.png}

\end{sphinxuseclass}\end{sphinxVerbatimOutput}

\end{sphinxuseclass}
\sphinxAtStartPar
 주가지수 수익율 대비 종목수익율의 경우는 아주 크거나 작을 때 수익율이 좋게 나타났습니다.

\begin{sphinxuseclass}{cell}\begin{sphinxVerbatimInput}

\begin{sphinxuseclass}{cell_input}
\begin{sphinxVerbatim}[commandchars=\\\{\}]
\PYG{n}{ranks} \PYG{o}{=} \PYG{n}{pd}\PYG{o}{.}\PYG{n}{qcut}\PYG{p}{(}\PYG{n}{data\PYGZus{}h5}\PYG{p}{[}\PYG{l+s+s1}{\PYGZsq{}}\PYG{l+s+s1}{pct\PYGZus{}win\PYGZus{}market}\PYG{l+s+s1}{\PYGZsq{}}\PYG{p}{]}\PYG{p}{,} \PYG{n}{q}\PYG{o}{=}\PYG{l+m+mi}{10}\PYG{p}{)}
\PYG{n+nb}{print}\PYG{p}{(}\PYG{n}{data\PYGZus{}h5}\PYG{o}{.}\PYG{n}{groupby}\PYG{p}{(}\PYG{n}{ranks}\PYG{p}{)}\PYG{p}{[}\PYG{l+s+s1}{\PYGZsq{}}\PYG{l+s+s1}{max\PYGZus{}close}\PYG{l+s+s1}{\PYGZsq{}}\PYG{p}{]}\PYG{o}{.}\PYG{n}{mean}\PYG{p}{(}\PYG{p}{)}\PYG{p}{)}
\PYG{n}{data\PYGZus{}h5}\PYG{o}{.}\PYG{n}{groupby}\PYG{p}{(}\PYG{n}{ranks}\PYG{p}{)}\PYG{p}{[}\PYG{l+s+s1}{\PYGZsq{}}\PYG{l+s+s1}{max\PYGZus{}close}\PYG{l+s+s1}{\PYGZsq{}}\PYG{p}{]}\PYG{o}{.}\PYG{n}{mean}\PYG{p}{(}\PYG{p}{)}\PYG{o}{.}\PYG{n}{plot}\PYG{p}{(}\PYG{n}{figsize}\PYG{o}{=}\PYG{p}{(}\PYG{l+m+mi}{12}\PYG{p}{,}\PYG{l+m+mi}{5}\PYG{p}{)}\PYG{p}{)}
\end{sphinxVerbatim}

\end{sphinxuseclass}\end{sphinxVerbatimInput}
\begin{sphinxVerbatimOutput}

\begin{sphinxuseclass}{cell_output}
\begin{sphinxVerbatim}[commandchars=\\\{\}]
pct\PYGZus{}win\PYGZus{}market
(0.9747, 0.9964]   1.039
(0.9964, 0.9976]   1.029
(0.9976, 0.9985]   1.026
(0.9985, 0.9992]   1.024
(0.9992, 1.0]      1.025
(1.0, 1.0007]      1.026
(1.0007, 1.0016]   1.028
(1.0016, 1.0029]   1.031
(1.0029, 1.0052]   1.038
(1.0052, 1.0432]   1.049
Name: max\PYGZus{}close, dtype: float64
\end{sphinxVerbatim}

\begin{sphinxVerbatim}[commandchars=\\\{\}]
\PYGZlt{}AxesSubplot:xlabel=\PYGZsq{}pct\PYGZus{}win\PYGZus{}market\PYGZsq{}\PYGZgt{}
\end{sphinxVerbatim}

\noindent\sphinxincludegraphics{{5.1.5_Hypothesis_5_8_2}.png}

\end{sphinxuseclass}\end{sphinxVerbatimOutput}

\end{sphinxuseclass}
\begin{sphinxuseclass}{cell}\begin{sphinxVerbatimInput}

\begin{sphinxuseclass}{cell_input}
\begin{sphinxVerbatim}[commandchars=\\\{\}]
\PYG{k+kn}{import} \PYG{n+nn}{FinanceDataReader} \PYG{k}{as} \PYG{n+nn}{fdr}
\PYG{o}{\PYGZpc{}}\PYG{k}{matplotlib} inline
\PYG{k+kn}{import} \PYG{n+nn}{matplotlib}\PYG{n+nn}{.}\PYG{n+nn}{pyplot} \PYG{k}{as} \PYG{n+nn}{plt}
\PYG{k+kn}{import} \PYG{n+nn}{pandas} \PYG{k}{as} \PYG{n+nn}{pd}
\PYG{k+kn}{import} \PYG{n+nn}{numpy} \PYG{k}{as} \PYG{n+nn}{np}

\PYG{n}{pd}\PYG{o}{.}\PYG{n}{options}\PYG{o}{.}\PYG{n}{display}\PYG{o}{.}\PYG{n}{float\PYGZus{}format} \PYG{o}{=} \PYG{l+s+s1}{\PYGZsq{}}\PYG{l+s+si}{\PYGZob{}:,.3f\PYGZcb{}}\PYG{l+s+s1}{\PYGZsq{}}\PYG{o}{.}\PYG{n}{format}
\end{sphinxVerbatim}

\end{sphinxuseclass}\end{sphinxVerbatimInput}

\end{sphinxuseclass}

\section{동종업계 평균 수익률보다 더 좋은 수익률을 보여준다.}
\label{\detokenize{chapter5/5.1.6_Hypothesis_6:id1}}\label{\detokenize{chapter5/5.1.6_Hypothesis_6::doc}}
\begin{sphinxuseclass}{cell}\begin{sphinxVerbatimInput}

\begin{sphinxuseclass}{cell_input}
\begin{sphinxVerbatim}[commandchars=\\\{\}]
\PYG{n}{mdl\PYGZus{}data} \PYG{o}{=} \PYG{n}{pd}\PYG{o}{.}\PYG{n}{read\PYGZus{}pickle}\PYG{p}{(}\PYG{l+s+s1}{\PYGZsq{}}\PYG{l+s+s1}{mdl\PYGZus{}data.pkl}\PYG{l+s+s1}{\PYGZsq{}}\PYG{p}{)}
\PYG{n}{mdl\PYGZus{}data}\PYG{o}{.}\PYG{n}{head}\PYG{p}{(}\PYG{p}{)}\PYG{o}{.}\PYG{n}{style}\PYG{o}{.}\PYG{n}{set\PYGZus{}table\PYGZus{}attributes}\PYG{p}{(}\PYG{l+s+s1}{\PYGZsq{}}\PYG{l+s+s1}{style=}\PYG{l+s+s1}{\PYGZdq{}}\PYG{l+s+s1}{font\PYGZhy{}size: 12px}\PYG{l+s+s1}{\PYGZdq{}}\PYG{l+s+s1}{\PYGZsq{}}\PYG{p}{)}\PYG{o}{.}\PYG{n}{format}\PYG{p}{(}\PYG{n}{precision}\PYG{o}{=}\PYG{l+m+mi}{3}\PYG{p}{)}
\end{sphinxVerbatim}

\end{sphinxuseclass}\end{sphinxVerbatimInput}
\begin{sphinxVerbatimOutput}

\begin{sphinxuseclass}{cell_output}
\begin{sphinxVerbatim}[commandchars=\\\{\}]
\PYGZlt{}pandas.io.formats.style.Styler at 0x2c749ab8df0\PYGZgt{}
\end{sphinxVerbatim}

\end{sphinxuseclass}\end{sphinxVerbatimOutput}

\end{sphinxuseclass}
\sphinxAtStartPar
 우선 과거 60일 평균 수익율 값을 return\_mean 에 저장합니다. 그리고 종목에 sector 정보를 추가합니다.

\begin{sphinxuseclass}{cell}\begin{sphinxVerbatimInput}

\begin{sphinxuseclass}{cell_input}
\begin{sphinxVerbatim}[commandchars=\\\{\}]
\PYG{n}{kosdaq\PYGZus{}list} \PYG{o}{=} \PYG{n}{pd}\PYG{o}{.}\PYG{n}{read\PYGZus{}pickle}\PYG{p}{(}\PYG{l+s+s1}{\PYGZsq{}}\PYG{l+s+s1}{kosdaq\PYGZus{}list.pkl}\PYG{l+s+s1}{\PYGZsq{}}\PYG{p}{)}

\PYG{n}{data\PYGZus{}h6} \PYG{o}{=} \PYG{n}{pd}\PYG{o}{.}\PYG{n}{DataFrame}\PYG{p}{(}\PYG{p}{)}

\PYG{k}{for} \PYG{n}{code}\PYG{p}{,} \PYG{n}{sector} \PYG{o+ow}{in} \PYG{n+nb}{zip}\PYG{p}{(}\PYG{n}{kosdaq\PYGZus{}list}\PYG{p}{[}\PYG{l+s+s1}{\PYGZsq{}}\PYG{l+s+s1}{code}\PYG{l+s+s1}{\PYGZsq{}}\PYG{p}{]}\PYG{p}{,} \PYG{n}{kosdaq\PYGZus{}list}\PYG{p}{[}\PYG{l+s+s1}{\PYGZsq{}}\PYG{l+s+s1}{sector}\PYG{l+s+s1}{\PYGZsq{}}\PYG{p}{]}\PYG{p}{)}\PYG{p}{:}
    
    \PYG{c+c1}{\PYGZsh{} 종목별 처리}
    \PYG{n}{data} \PYG{o}{=} \PYG{n}{mdl\PYGZus{}data}\PYG{p}{[}\PYG{n}{mdl\PYGZus{}data}\PYG{p}{[}\PYG{l+s+s1}{\PYGZsq{}}\PYG{l+s+s1}{code}\PYG{l+s+s1}{\PYGZsq{}}\PYG{p}{]}\PYG{o}{==}\PYG{n}{code}\PYG{p}{]}\PYG{o}{.}\PYG{n}{sort\PYGZus{}index}\PYG{p}{(}\PYG{p}{)}\PYG{o}{.}\PYG{n}{copy}\PYG{p}{(}\PYG{p}{)}
    \PYG{n}{data}\PYG{o}{.}\PYG{n}{dropna}\PYG{p}{(}\PYG{n}{inplace}\PYG{o}{=}\PYG{k+kc}{True}\PYG{p}{)}
    
    \PYG{c+c1}{\PYGZsh{} 최근 60일 평균 수익율            }
    \PYG{n}{data}\PYG{p}{[}\PYG{l+s+s1}{\PYGZsq{}}\PYG{l+s+s1}{return\PYGZus{}mean}\PYG{l+s+s1}{\PYGZsq{}}\PYG{p}{]} \PYG{o}{=} \PYG{n}{data}\PYG{p}{[}\PYG{l+s+s1}{\PYGZsq{}}\PYG{l+s+s1}{return}\PYG{l+s+s1}{\PYGZsq{}}\PYG{p}{]}\PYG{o}{.}\PYG{n}{rolling}\PYG{p}{(}\PYG{l+m+mi}{60}\PYG{p}{)}\PYG{o}{.}\PYG{n}{mean}\PYG{p}{(}\PYG{p}{)} \PYG{c+c1}{\PYGZsh{} 종목별 최근 60 일 수익율의 평균}
    \PYG{n}{data}\PYG{p}{[}\PYG{l+s+s1}{\PYGZsq{}}\PYG{l+s+s1}{sector}\PYG{l+s+s1}{\PYGZsq{}}\PYG{p}{]} \PYG{o}{=} \PYG{n}{sector}  \PYG{c+c1}{\PYGZsh{} 섹터 정보   }
  
    \PYG{n}{data}\PYG{o}{.}\PYG{n}{dropna}\PYG{p}{(}\PYG{n}{subset}\PYG{o}{=}\PYG{p}{[}\PYG{l+s+s1}{\PYGZsq{}}\PYG{l+s+s1}{return\PYGZus{}mean}\PYG{l+s+s1}{\PYGZsq{}}\PYG{p}{]}\PYG{p}{,} \PYG{n}{inplace}\PYG{o}{=}\PYG{k+kc}{True}\PYG{p}{)}    
    \PYG{n}{data\PYGZus{}h6} \PYG{o}{=} \PYG{n}{pd}\PYG{o}{.}\PYG{n}{concat}\PYG{p}{(}\PYG{p}{[}\PYG{n}{data}\PYG{p}{,} \PYG{n}{data\PYGZus{}h6}\PYG{p}{]}\PYG{p}{,} \PYG{n}{axis}\PYG{o}{=}\PYG{l+m+mi}{0}\PYG{p}{)}

    
\PYG{n}{data\PYGZus{}h6}\PYG{p}{[}\PYG{l+s+s1}{\PYGZsq{}}\PYG{l+s+s1}{sector\PYGZus{}return}\PYG{l+s+s1}{\PYGZsq{}}\PYG{p}{]} \PYG{o}{=} \PYG{n}{data\PYGZus{}h6}\PYG{o}{.}\PYG{n}{groupby}\PYG{p}{(}\PYG{p}{[}\PYG{l+s+s1}{\PYGZsq{}}\PYG{l+s+s1}{sector}\PYG{l+s+s1}{\PYGZsq{}}\PYG{p}{,} \PYG{n}{data\PYGZus{}h6}\PYG{o}{.}\PYG{n}{index}\PYG{p}{]}\PYG{p}{)}\PYG{p}{[}\PYG{l+s+s1}{\PYGZsq{}}\PYG{l+s+s1}{return}\PYG{l+s+s1}{\PYGZsq{}}\PYG{p}{]}\PYG{o}{.}\PYG{n}{transform}\PYG{p}{(}\PYG{k}{lambda} \PYG{n}{x}\PYG{p}{:} \PYG{n}{x}\PYG{o}{.}\PYG{n}{mean}\PYG{p}{(}\PYG{p}{)}\PYG{p}{)} \PYG{c+c1}{\PYGZsh{} 섹터와 날짜별 평균 값}
\PYG{n}{data\PYGZus{}h6}\PYG{p}{[}\PYG{l+s+s1}{\PYGZsq{}}\PYG{l+s+s1}{return over sector}\PYG{l+s+s1}{\PYGZsq{}}\PYG{p}{]} \PYG{o}{=} \PYG{p}{(}\PYG{n}{data\PYGZus{}h6}\PYG{p}{[}\PYG{l+s+s1}{\PYGZsq{}}\PYG{l+s+s1}{return}\PYG{l+s+s1}{\PYGZsq{}}\PYG{p}{]}\PYG{o}{/}\PYG{n}{data\PYGZus{}h6}\PYG{p}{[}\PYG{l+s+s1}{\PYGZsq{}}\PYG{l+s+s1}{sector\PYGZus{}return}\PYG{l+s+s1}{\PYGZsq{}}\PYG{p}{]}\PYG{p}{)}

\PYG{n}{data\PYGZus{}h6}\PYG{o}{.}\PYG{n}{to\PYGZus{}pickle}\PYG{p}{(}\PYG{l+s+s1}{\PYGZsq{}}\PYG{l+s+s1}{data\PYGZus{}h6.pkl}\PYG{l+s+s1}{\PYGZsq{}}\PYG{p}{)}  
\end{sphinxVerbatim}

\end{sphinxuseclass}\end{sphinxVerbatimInput}

\end{sphinxuseclass}
\sphinxAtStartPar
 종목이 몇 개 없는 섹터는 평균의 의미가 없으므로 섹터에 종목이 최소한 10 개 이상이 있는 섹터만 보겠습니다.

\begin{sphinxuseclass}{cell}\begin{sphinxVerbatimInput}

\begin{sphinxuseclass}{cell_input}
\begin{sphinxVerbatim}[commandchars=\\\{\}]
\PYG{n}{data\PYGZus{}h6} \PYG{o}{=} \PYG{n}{pd}\PYG{o}{.}\PYG{n}{read\PYGZus{}pickle}\PYG{p}{(}\PYG{l+s+s1}{\PYGZsq{}}\PYG{l+s+s1}{data\PYGZus{}h6.pkl}\PYG{l+s+s1}{\PYGZsq{}}\PYG{p}{)}  
\PYG{n}{sector\PYGZus{}count} \PYG{o}{=} \PYG{n}{data\PYGZus{}h6}\PYG{o}{.}\PYG{n}{groupby}\PYG{p}{(}\PYG{l+s+s1}{\PYGZsq{}}\PYG{l+s+s1}{sector}\PYG{l+s+s1}{\PYGZsq{}}\PYG{p}{)}\PYG{p}{[}\PYG{l+s+s1}{\PYGZsq{}}\PYG{l+s+s1}{code}\PYG{l+s+s1}{\PYGZsq{}}\PYG{p}{]}\PYG{o}{.}\PYG{n}{nunique}\PYG{p}{(}\PYG{p}{)}\PYG{o}{.}\PYG{n}{sort\PYGZus{}values}\PYG{p}{(}\PYG{p}{)} \PYG{c+c1}{\PYGZsh{} 섹터 별로 종목 수 계산}
\PYG{n}{data\PYGZus{}h6x} \PYG{o}{=} \PYG{n}{data\PYGZus{}h6}\PYG{p}{[}\PYG{n}{data\PYGZus{}h6}\PYG{p}{[}\PYG{l+s+s1}{\PYGZsq{}}\PYG{l+s+s1}{sector}\PYG{l+s+s1}{\PYGZsq{}}\PYG{p}{]}\PYG{o}{.}\PYG{n}{isin}\PYG{p}{(}\PYG{n}{sector\PYGZus{}count}\PYG{p}{[}\PYG{n}{sector\PYGZus{}count}\PYG{o}{\PYGZgt{}}\PYG{o}{=}\PYG{l+m+mi}{10}\PYG{p}{]}\PYG{o}{.}\PYG{n}{index}\PYG{p}{)}\PYG{p}{]}\PYG{o}{.}\PYG{n}{copy}\PYG{p}{(}\PYG{p}{)} \PYG{c+c1}{\PYGZsh{} 섹터 별로 10개 이상이 있는 종목이 있는 섹터만 추출}
\end{sphinxVerbatim}

\end{sphinxuseclass}\end{sphinxVerbatimInput}

\end{sphinxuseclass}
\sphinxAtStartPar
 섹터 평균 수익율 대비 종목 수익율이 아주 낮거나, 높은 경우에 미래 수익률이 높게 나왔습니다. 종목 수익률이 섹터 평균 수익률과 비슷한 경우(‘return over sector’ 값이 1 근처인 경우)는 예상 수익율이 낮게 나타나고 있습니다.

\begin{sphinxuseclass}{cell}\begin{sphinxVerbatimInput}

\begin{sphinxuseclass}{cell_input}
\begin{sphinxVerbatim}[commandchars=\\\{\}]
\PYG{n}{data\PYGZus{}h6x}\PYG{p}{[}\PYG{l+s+s1}{\PYGZsq{}}\PYG{l+s+s1}{sector\PYGZus{}return}\PYG{l+s+s1}{\PYGZsq{}}\PYG{p}{]} \PYG{o}{=} \PYG{n}{data\PYGZus{}h6x}\PYG{o}{.}\PYG{n}{groupby}\PYG{p}{(}\PYG{p}{[}\PYG{l+s+s1}{\PYGZsq{}}\PYG{l+s+s1}{sector}\PYG{l+s+s1}{\PYGZsq{}}\PYG{p}{,} \PYG{n}{data\PYGZus{}h6x}\PYG{o}{.}\PYG{n}{index}\PYG{p}{]}\PYG{p}{)}\PYG{p}{[}\PYG{l+s+s1}{\PYGZsq{}}\PYG{l+s+s1}{return}\PYG{l+s+s1}{\PYGZsq{}}\PYG{p}{]}\PYG{o}{.}\PYG{n}{transform}\PYG{p}{(}\PYG{k}{lambda} \PYG{n}{x}\PYG{p}{:} \PYG{n}{x}\PYG{o}{.}\PYG{n}{mean}\PYG{p}{(}\PYG{p}{)}\PYG{p}{)}
\PYG{n}{data\PYGZus{}h6x}\PYG{p}{[}\PYG{l+s+s1}{\PYGZsq{}}\PYG{l+s+s1}{return over sector}\PYG{l+s+s1}{\PYGZsq{}}\PYG{p}{]} \PYG{o}{=} \PYG{p}{(}\PYG{n}{data\PYGZus{}h6x}\PYG{p}{[}\PYG{l+s+s1}{\PYGZsq{}}\PYG{l+s+s1}{return}\PYG{l+s+s1}{\PYGZsq{}}\PYG{p}{]}\PYG{o}{/}\PYG{n}{data\PYGZus{}h6x}\PYG{p}{[}\PYG{l+s+s1}{\PYGZsq{}}\PYG{l+s+s1}{sector\PYGZus{}return}\PYG{l+s+s1}{\PYGZsq{}}\PYG{p}{]}\PYG{p}{)}
\PYG{n}{ranks} \PYG{o}{=} \PYG{n}{pd}\PYG{o}{.}\PYG{n}{qcut}\PYG{p}{(}\PYG{n}{data\PYGZus{}h6x}\PYG{p}{[}\PYG{l+s+s1}{\PYGZsq{}}\PYG{l+s+s1}{return over sector}\PYG{l+s+s1}{\PYGZsq{}}\PYG{p}{]}\PYG{p}{,} \PYG{n}{q}\PYG{o}{=}\PYG{l+m+mi}{10}\PYG{p}{)}
\PYG{n+nb}{print}\PYG{p}{(}\PYG{n}{data\PYGZus{}h6x}\PYG{o}{.}\PYG{n}{groupby}\PYG{p}{(}\PYG{n}{ranks}\PYG{p}{)}\PYG{p}{[}\PYG{l+s+s1}{\PYGZsq{}}\PYG{l+s+s1}{max\PYGZus{}close}\PYG{l+s+s1}{\PYGZsq{}}\PYG{p}{]}\PYG{o}{.}\PYG{n}{describe}\PYG{p}{(}\PYG{n}{percentiles}\PYG{o}{=}\PYG{p}{[}\PYG{l+m+mf}{0.1}\PYG{p}{,} \PYG{l+m+mf}{0.9}\PYG{p}{]}\PYG{p}{)}\PYG{p}{)}
\PYG{n}{data\PYGZus{}h6x}\PYG{o}{.}\PYG{n}{groupby}\PYG{p}{(}\PYG{n}{ranks}\PYG{p}{)}\PYG{p}{[}\PYG{l+s+s1}{\PYGZsq{}}\PYG{l+s+s1}{max\PYGZus{}close}\PYG{l+s+s1}{\PYGZsq{}}\PYG{p}{]}\PYG{o}{.}\PYG{n}{mean}\PYG{p}{(}\PYG{p}{)}\PYG{o}{.}\PYG{n}{plot}\PYG{p}{(}\PYG{n}{figsize}\PYG{o}{=}\PYG{p}{(}\PYG{l+m+mi}{12}\PYG{p}{,}\PYG{l+m+mi}{5}\PYG{p}{)}\PYG{p}{)}
\end{sphinxVerbatim}

\end{sphinxuseclass}\end{sphinxVerbatimInput}
\begin{sphinxVerbatimOutput}

\begin{sphinxuseclass}{cell_output}
\begin{sphinxVerbatim}[commandchars=\\\{\}]
                        count  mean   std   min   10\PYGZpc{}   50\PYGZpc{}   90\PYGZpc{}   max
return over sector                                                     
(0.688, 0.973]     26,887.000 1.042 0.087 0.700 0.974 1.022 1.126 2.968
(0.973, 0.983]     26,886.000 1.034 0.067 0.702 0.982 1.019 1.099 2.171
(0.983, 0.989]     26,886.000 1.030 0.060 0.704 0.984 1.016 1.087 2.286
(0.989, 0.994]     26,886.000 1.028 0.060 0.700 0.985 1.014 1.083 2.194
(0.994, 0.998]     26,889.000 1.027 0.058 0.819 0.985 1.013 1.080 2.729
(0.998, 1.002]     26,883.000 1.026 0.059 0.700 0.985 1.012 1.078 2.652
(1.002, 1.007]     26,886.000 1.027 0.062 0.700 0.984 1.012 1.080 3.027
(1.007, 1.014]     26,886.000 1.028 0.062 0.700 0.983 1.013 1.084 3.380
(1.014, 1.027]     26,886.000 1.031 0.068 0.700 0.980 1.014 1.093 2.412
(1.027, 1.399]     26,887.000 1.043 0.105 0.700 0.970 1.019 1.136 3.703
\end{sphinxVerbatim}

\begin{sphinxVerbatim}[commandchars=\\\{\}]
\PYGZlt{}AxesSubplot:xlabel=\PYGZsq{}return over sector\PYGZsq{}\PYGZgt{}
\end{sphinxVerbatim}

\noindent\sphinxincludegraphics{{5.1.6_Hypothesis_6_8_2}.png}

\end{sphinxuseclass}\end{sphinxVerbatimOutput}

\end{sphinxuseclass}

\chapter{\sphinxstylestrong{해결책 개발}}
\label{\detokenize{chapter5/5.2.0_Solution_Details:id1}}\label{\detokenize{chapter5/5.2.0_Solution_Details::doc}}
\sphinxAtStartPar
가설 검증을 통하여 각 가설이 유의미한지 알아보았습니다. 가설 검정과정에서 얻은 지식을 이용하여  종목을 찾고, 매매 수익을 실현하고 싶습니다. 하지만 유미의한 가설들을 동시에 활용하여 종목을 뽑아내기는 쉬운 일이 아닙니다.  단지 2 개의 가설을 동시에 고려하는 것도 어렵습니다.  이것을 가능하게 해 주는 것이 예측 모델인데요. 가설들이 입력변수가 되는 예측 모델을 만들어 보겠습니다.

\sphinxAtStartPar
매수 후 5 영업일 이내 주가 상승 여부를 알 수 있는 예측력이 좋은 모델이 개발되면, 매일 모델을 실행시켜 종목 추천을 모델로 부터 받을 것입니다. 정규장이 3시 30에 종료되므로 3시 30분에 모델을 돌리고 종목 추천을 받겠습니다. 그리고 익일 정해진 지정가에 매수를 하겠습니다. 매수와 매도는 자동매매로 구현해볼 계획입니다.

\begin{sphinxuseclass}{cell}\begin{sphinxVerbatimInput}

\begin{sphinxuseclass}{cell_input}
\begin{sphinxVerbatim}[commandchars=\\\{\}]
\PYG{k+kn}{import} \PYG{n+nn}{FinanceDataReader} \PYG{k}{as} \PYG{n+nn}{fdr}
\PYG{o}{\PYGZpc{}}\PYG{k}{matplotlib} inline
\PYG{k+kn}{import} \PYG{n+nn}{matplotlib}\PYG{n+nn}{.}\PYG{n+nn}{pyplot} \PYG{k}{as} \PYG{n+nn}{plt}
\PYG{k+kn}{import} \PYG{n+nn}{pandas} \PYG{k}{as} \PYG{n+nn}{pd}
\PYG{k+kn}{import} \PYG{n+nn}{numpy} \PYG{k}{as} \PYG{n+nn}{np}

\PYG{n}{pd}\PYG{o}{.}\PYG{n}{options}\PYG{o}{.}\PYG{n}{display}\PYG{o}{.}\PYG{n}{float\PYGZus{}format} \PYG{o}{=} \PYG{l+s+s1}{\PYGZsq{}}\PYG{l+s+si}{\PYGZob{}:,.3f\PYGZcb{}}\PYG{l+s+s1}{\PYGZsq{}}\PYG{o}{.}\PYG{n}{format}
\end{sphinxVerbatim}

\end{sphinxuseclass}\end{sphinxVerbatimInput}

\end{sphinxuseclass}

\section{피처 엔지니어링}
\label{\detokenize{chapter5/5.2.1_Feature_Engineering:id1}}\label{\detokenize{chapter5/5.2.1_Feature_Engineering::doc}}
\sphinxAtStartPar
가설 검정에서 만들었던 모든 피쳐(변수)를 정리해 보겠습니다. 이제 예측 모델링을 위한 데이터가 준비되었습니다. 예측모델링에 활용한 데이터의 기간은 2021년 1월 5일부터 2022년 3월 24일까지입니다.

\begin{sphinxuseclass}{cell}\begin{sphinxVerbatimInput}

\begin{sphinxuseclass}{cell_input}
\begin{sphinxVerbatim}[commandchars=\\\{\}]
\PYG{n}{mdl\PYGZus{}data} \PYG{o}{=} \PYG{n}{pd}\PYG{o}{.}\PYG{n}{read\PYGZus{}pickle}\PYG{p}{(}\PYG{l+s+s1}{\PYGZsq{}}\PYG{l+s+s1}{mdl\PYGZus{}data.pkl}\PYG{l+s+s1}{\PYGZsq{}}\PYG{p}{)} \PYG{c+c1}{\PYGZsh{} 수익률 결과값이 있는 데이터}
\PYG{n}{mdl\PYGZus{}data}\PYG{o}{.}\PYG{n}{head}\PYG{p}{(}\PYG{p}{)}
\PYG{n+nb}{print}\PYG{p}{(}\PYG{n}{mdl\PYGZus{}data}\PYG{o}{.}\PYG{n}{index}\PYG{o}{.}\PYG{n}{min}\PYG{p}{(}\PYG{p}{)}\PYG{p}{,} \PYG{n}{mdl\PYGZus{}data}\PYG{o}{.}\PYG{n}{index}\PYG{o}{.}\PYG{n}{max}\PYG{p}{(}\PYG{p}{)}\PYG{p}{)}
\end{sphinxVerbatim}

\end{sphinxuseclass}\end{sphinxVerbatimInput}
\begin{sphinxVerbatimOutput}

\begin{sphinxuseclass}{cell_output}
\begin{sphinxVerbatim}[commandchars=\\\{\}]
2021\PYGZhy{}01\PYGZhy{}05 2022\PYGZhy{}03\PYGZhy{}24
\end{sphinxVerbatim}

\end{sphinxuseclass}\end{sphinxVerbatimOutput}

\end{sphinxuseclass}
\sphinxAtStartPar
 가설검정에서 만들었던 모든 피쳐를 정리합니다. 단, \sphinxstyleemphasis{“5일 이동평균선이 종가보다 위에 있다”} 는 유의미하지 않았으므로 제외입니다. 결과를 feature\_all 이라는 데이터프레임에 저장합니다.

\begin{sphinxuseclass}{cell}\begin{sphinxVerbatimInput}

\begin{sphinxuseclass}{cell_input}
\begin{sphinxVerbatim}[commandchars=\\\{\}]
\PYG{n}{kosdaq\PYGZus{}list} \PYG{o}{=} \PYG{n}{pd}\PYG{o}{.}\PYG{n}{read\PYGZus{}pickle}\PYG{p}{(}\PYG{l+s+s1}{\PYGZsq{}}\PYG{l+s+s1}{kosdaq\PYGZus{}list.pkl}\PYG{l+s+s1}{\PYGZsq{}}\PYG{p}{)}

\PYG{n}{feature\PYGZus{}all} \PYG{o}{=} \PYG{n}{pd}\PYG{o}{.}\PYG{n}{DataFrame}\PYG{p}{(}\PYG{p}{)}

\PYG{k}{for} \PYG{n}{code}\PYG{p}{,} \PYG{n}{sector} \PYG{o+ow}{in} \PYG{n+nb}{zip}\PYG{p}{(}\PYG{n}{kosdaq\PYGZus{}list}\PYG{p}{[}\PYG{l+s+s1}{\PYGZsq{}}\PYG{l+s+s1}{code}\PYG{l+s+s1}{\PYGZsq{}}\PYG{p}{]}\PYG{p}{,} \PYG{n}{kosdaq\PYGZus{}list}\PYG{p}{[}\PYG{l+s+s1}{\PYGZsq{}}\PYG{l+s+s1}{sector}\PYG{l+s+s1}{\PYGZsq{}}\PYG{p}{]}\PYG{p}{)}\PYG{p}{:}

    \PYG{n}{data} \PYG{o}{=} \PYG{n}{mdl\PYGZus{}data}\PYG{p}{[}\PYG{n}{mdl\PYGZus{}data}\PYG{p}{[}\PYG{l+s+s1}{\PYGZsq{}}\PYG{l+s+s1}{code}\PYG{l+s+s1}{\PYGZsq{}}\PYG{p}{]}\PYG{o}{==}\PYG{n}{code}\PYG{p}{]}\PYG{o}{.}\PYG{n}{sort\PYGZus{}index}\PYG{p}{(}\PYG{p}{)}\PYG{o}{.}\PYG{n}{copy}\PYG{p}{(}\PYG{p}{)}
    
    
    \PYG{c+c1}{\PYGZsh{} 가격변동성이 크고, 거래량이 몰린 종목이 주가가 상승한다}
    \PYG{n}{data}\PYG{p}{[}\PYG{l+s+s1}{\PYGZsq{}}\PYG{l+s+s1}{price\PYGZus{}mean}\PYG{l+s+s1}{\PYGZsq{}}\PYG{p}{]} \PYG{o}{=} \PYG{n}{data}\PYG{p}{[}\PYG{l+s+s1}{\PYGZsq{}}\PYG{l+s+s1}{close}\PYG{l+s+s1}{\PYGZsq{}}\PYG{p}{]}\PYG{o}{.}\PYG{n}{rolling}\PYG{p}{(}\PYG{l+m+mi}{20}\PYG{p}{)}\PYG{o}{.}\PYG{n}{mean}\PYG{p}{(}\PYG{p}{)}
    \PYG{n}{data}\PYG{p}{[}\PYG{l+s+s1}{\PYGZsq{}}\PYG{l+s+s1}{price\PYGZus{}std}\PYG{l+s+s1}{\PYGZsq{}}\PYG{p}{]} \PYG{o}{=} \PYG{n}{data}\PYG{p}{[}\PYG{l+s+s1}{\PYGZsq{}}\PYG{l+s+s1}{close}\PYG{l+s+s1}{\PYGZsq{}}\PYG{p}{]}\PYG{o}{.}\PYG{n}{rolling}\PYG{p}{(}\PYG{l+m+mi}{20}\PYG{p}{)}\PYG{o}{.}\PYG{n}{std}\PYG{p}{(}\PYG{n}{ddof}\PYG{o}{=}\PYG{l+m+mi}{0}\PYG{p}{)}
    \PYG{n}{data}\PYG{p}{[}\PYG{l+s+s1}{\PYGZsq{}}\PYG{l+s+s1}{price\PYGZus{}z}\PYG{l+s+s1}{\PYGZsq{}}\PYG{p}{]} \PYG{o}{=} \PYG{p}{(}\PYG{n}{data}\PYG{p}{[}\PYG{l+s+s1}{\PYGZsq{}}\PYG{l+s+s1}{close}\PYG{l+s+s1}{\PYGZsq{}}\PYG{p}{]} \PYG{o}{\PYGZhy{}} \PYG{n}{data}\PYG{p}{[}\PYG{l+s+s1}{\PYGZsq{}}\PYG{l+s+s1}{price\PYGZus{}mean}\PYG{l+s+s1}{\PYGZsq{}}\PYG{p}{]}\PYG{p}{)}\PYG{o}{/}\PYG{n}{data}\PYG{p}{[}\PYG{l+s+s1}{\PYGZsq{}}\PYG{l+s+s1}{price\PYGZus{}std}\PYG{l+s+s1}{\PYGZsq{}}\PYG{p}{]}    
    \PYG{n}{data}\PYG{p}{[}\PYG{l+s+s1}{\PYGZsq{}}\PYG{l+s+s1}{volume\PYGZus{}mean}\PYG{l+s+s1}{\PYGZsq{}}\PYG{p}{]} \PYG{o}{=} \PYG{n}{data}\PYG{p}{[}\PYG{l+s+s1}{\PYGZsq{}}\PYG{l+s+s1}{volume}\PYG{l+s+s1}{\PYGZsq{}}\PYG{p}{]}\PYG{o}{.}\PYG{n}{rolling}\PYG{p}{(}\PYG{l+m+mi}{20}\PYG{p}{)}\PYG{o}{.}\PYG{n}{mean}\PYG{p}{(}\PYG{p}{)}
    \PYG{n}{data}\PYG{p}{[}\PYG{l+s+s1}{\PYGZsq{}}\PYG{l+s+s1}{volume\PYGZus{}std}\PYG{l+s+s1}{\PYGZsq{}}\PYG{p}{]} \PYG{o}{=} \PYG{n}{data}\PYG{p}{[}\PYG{l+s+s1}{\PYGZsq{}}\PYG{l+s+s1}{volume}\PYG{l+s+s1}{\PYGZsq{}}\PYG{p}{]}\PYG{o}{.}\PYG{n}{rolling}\PYG{p}{(}\PYG{l+m+mi}{20}\PYG{p}{)}\PYG{o}{.}\PYG{n}{std}\PYG{p}{(}\PYG{n}{ddof}\PYG{o}{=}\PYG{l+m+mi}{0}\PYG{p}{)}
    \PYG{n}{data}\PYG{p}{[}\PYG{l+s+s1}{\PYGZsq{}}\PYG{l+s+s1}{volume\PYGZus{}z}\PYG{l+s+s1}{\PYGZsq{}}\PYG{p}{]} \PYG{o}{=} \PYG{p}{(}\PYG{n}{data}\PYG{p}{[}\PYG{l+s+s1}{\PYGZsq{}}\PYG{l+s+s1}{volume}\PYG{l+s+s1}{\PYGZsq{}}\PYG{p}{]} \PYG{o}{\PYGZhy{}} \PYG{n}{data}\PYG{p}{[}\PYG{l+s+s1}{\PYGZsq{}}\PYG{l+s+s1}{volume\PYGZus{}mean}\PYG{l+s+s1}{\PYGZsq{}}\PYG{p}{]}\PYG{p}{)}\PYG{o}{/}\PYG{n}{data}\PYG{p}{[}\PYG{l+s+s1}{\PYGZsq{}}\PYG{l+s+s1}{volume\PYGZus{}std}\PYG{l+s+s1}{\PYGZsq{}}\PYG{p}{]}      
    
    
    \PYG{c+c1}{\PYGZsh{} 위꼬리가 긴 양봉이 자주 발생한다.}
    \PYG{n}{data}\PYG{p}{[}\PYG{l+s+s1}{\PYGZsq{}}\PYG{l+s+s1}{positive\PYGZus{}candle}\PYG{l+s+s1}{\PYGZsq{}}\PYG{p}{]} \PYG{o}{=} \PYG{p}{(}\PYG{n}{data}\PYG{p}{[}\PYG{l+s+s1}{\PYGZsq{}}\PYG{l+s+s1}{close}\PYG{l+s+s1}{\PYGZsq{}}\PYG{p}{]} \PYG{o}{\PYGZgt{}} \PYG{n}{data}\PYG{p}{[}\PYG{l+s+s1}{\PYGZsq{}}\PYG{l+s+s1}{open}\PYG{l+s+s1}{\PYGZsq{}}\PYG{p}{]}\PYG{p}{)}\PYG{o}{.}\PYG{n}{astype}\PYG{p}{(}\PYG{n+nb}{int}\PYG{p}{)} \PYG{c+c1}{\PYGZsh{} 양봉}
    \PYG{n}{data}\PYG{p}{[}\PYG{l+s+s1}{\PYGZsq{}}\PYG{l+s+s1}{high/close}\PYG{l+s+s1}{\PYGZsq{}}\PYG{p}{]} \PYG{o}{=} \PYG{p}{(}\PYG{n}{data}\PYG{p}{[}\PYG{l+s+s1}{\PYGZsq{}}\PYG{l+s+s1}{positive\PYGZus{}candle}\PYG{l+s+s1}{\PYGZsq{}}\PYG{p}{]}\PYG{o}{==}\PYG{l+m+mi}{1}\PYG{p}{)}\PYG{o}{*}\PYG{p}{(}\PYG{n}{data}\PYG{p}{[}\PYG{l+s+s1}{\PYGZsq{}}\PYG{l+s+s1}{high}\PYG{l+s+s1}{\PYGZsq{}}\PYG{p}{]}\PYG{o}{/}\PYG{n}{data}\PYG{p}{[}\PYG{l+s+s1}{\PYGZsq{}}\PYG{l+s+s1}{close}\PYG{l+s+s1}{\PYGZsq{}}\PYG{p}{]} \PYG{o}{\PYGZgt{}} \PYG{l+m+mf}{1.1}\PYG{p}{)}\PYG{o}{.}\PYG{n}{astype}\PYG{p}{(}\PYG{n+nb}{int}\PYG{p}{)} \PYG{c+c1}{\PYGZsh{} 양봉이면서 고가가 종가보다 높게 위치}
    \PYG{n}{data}\PYG{p}{[}\PYG{l+s+s1}{\PYGZsq{}}\PYG{l+s+s1}{num\PYGZus{}high/close}\PYG{l+s+s1}{\PYGZsq{}}\PYG{p}{]} \PYG{o}{=}  \PYG{n}{data}\PYG{p}{[}\PYG{l+s+s1}{\PYGZsq{}}\PYG{l+s+s1}{high/close}\PYG{l+s+s1}{\PYGZsq{}}\PYG{p}{]}\PYG{o}{.}\PYG{n}{rolling}\PYG{p}{(}\PYG{l+m+mi}{20}\PYG{p}{)}\PYG{o}{.}\PYG{n}{sum}\PYG{p}{(}\PYG{p}{)}
    \PYG{n}{data}\PYG{p}{[}\PYG{l+s+s1}{\PYGZsq{}}\PYG{l+s+s1}{long\PYGZus{}candle}\PYG{l+s+s1}{\PYGZsq{}}\PYG{p}{]} \PYG{o}{=} \PYG{p}{(}\PYG{n}{data}\PYG{p}{[}\PYG{l+s+s1}{\PYGZsq{}}\PYG{l+s+s1}{positive\PYGZus{}candle}\PYG{l+s+s1}{\PYGZsq{}}\PYG{p}{]}\PYG{o}{==}\PYG{l+m+mi}{1}\PYG{p}{)}\PYG{o}{*}\PYG{p}{(}\PYG{n}{data}\PYG{p}{[}\PYG{l+s+s1}{\PYGZsq{}}\PYG{l+s+s1}{high}\PYG{l+s+s1}{\PYGZsq{}}\PYG{p}{]}\PYG{o}{==}\PYG{n}{data}\PYG{p}{[}\PYG{l+s+s1}{\PYGZsq{}}\PYG{l+s+s1}{close}\PYG{l+s+s1}{\PYGZsq{}}\PYG{p}{]}\PYG{p}{)}\PYG{o}{*}\PYGZbs{}
    \PYG{p}{(}\PYG{n}{data}\PYG{p}{[}\PYG{l+s+s1}{\PYGZsq{}}\PYG{l+s+s1}{low}\PYG{l+s+s1}{\PYGZsq{}}\PYG{p}{]}\PYG{o}{==}\PYG{n}{data}\PYG{p}{[}\PYG{l+s+s1}{\PYGZsq{}}\PYG{l+s+s1}{open}\PYG{l+s+s1}{\PYGZsq{}}\PYG{p}{]}\PYG{p}{)}\PYG{o}{*}\PYG{p}{(}\PYG{n}{data}\PYG{p}{[}\PYG{l+s+s1}{\PYGZsq{}}\PYG{l+s+s1}{close}\PYG{l+s+s1}{\PYGZsq{}}\PYG{p}{]}\PYG{o}{/}\PYG{n}{data}\PYG{p}{[}\PYG{l+s+s1}{\PYGZsq{}}\PYG{l+s+s1}{open}\PYG{l+s+s1}{\PYGZsq{}}\PYG{p}{]} \PYG{o}{\PYGZgt{}} \PYG{l+m+mf}{1.2}\PYG{p}{)}\PYG{o}{.}\PYG{n}{astype}\PYG{p}{(}\PYG{n+nb}{int}\PYG{p}{)} \PYG{c+c1}{\PYGZsh{} 장대 양봉을 데이터로 표현}
    \PYG{n}{data}\PYG{p}{[}\PYG{l+s+s1}{\PYGZsq{}}\PYG{l+s+s1}{num\PYGZus{}long}\PYG{l+s+s1}{\PYGZsq{}}\PYG{p}{]} \PYG{o}{=}  \PYG{n}{data}\PYG{p}{[}\PYG{l+s+s1}{\PYGZsq{}}\PYG{l+s+s1}{long\PYGZus{}candle}\PYG{l+s+s1}{\PYGZsq{}}\PYG{p}{]}\PYG{o}{.}\PYG{n}{rolling}\PYG{p}{(}\PYG{l+m+mi}{60}\PYG{p}{)}\PYG{o}{.}\PYG{n}{sum}\PYG{p}{(}\PYG{p}{)} \PYG{c+c1}{\PYGZsh{} 지난 20 일 동안 장대양봉의 갯 수}
    
    
     \PYG{c+c1}{\PYGZsh{} 거래량이 종좀 터지며 매집의 흔적을 보인다   }
    \PYG{n}{data}\PYG{p}{[}\PYG{l+s+s1}{\PYGZsq{}}\PYG{l+s+s1}{volume\PYGZus{}mean}\PYG{l+s+s1}{\PYGZsq{}}\PYG{p}{]} \PYG{o}{=} \PYG{n}{data}\PYG{p}{[}\PYG{l+s+s1}{\PYGZsq{}}\PYG{l+s+s1}{volume}\PYG{l+s+s1}{\PYGZsq{}}\PYG{p}{]}\PYG{o}{.}\PYG{n}{rolling}\PYG{p}{(}\PYG{l+m+mi}{60}\PYG{p}{)}\PYG{o}{.}\PYG{n}{mean}\PYG{p}{(}\PYG{p}{)}
    \PYG{n}{data}\PYG{p}{[}\PYG{l+s+s1}{\PYGZsq{}}\PYG{l+s+s1}{volume\PYGZus{}std}\PYG{l+s+s1}{\PYGZsq{}}\PYG{p}{]} \PYG{o}{=} \PYG{n}{data}\PYG{p}{[}\PYG{l+s+s1}{\PYGZsq{}}\PYG{l+s+s1}{volume}\PYG{l+s+s1}{\PYGZsq{}}\PYG{p}{]}\PYG{o}{.}\PYG{n}{rolling}\PYG{p}{(}\PYG{l+m+mi}{60}\PYG{p}{)}\PYG{o}{.}\PYG{n}{std}\PYG{p}{(}\PYG{p}{)}
    \PYG{n}{data}\PYG{p}{[}\PYG{l+s+s1}{\PYGZsq{}}\PYG{l+s+s1}{volume\PYGZus{}z}\PYG{l+s+s1}{\PYGZsq{}}\PYG{p}{]} \PYG{o}{=} \PYG{p}{(}\PYG{n}{data}\PYG{p}{[}\PYG{l+s+s1}{\PYGZsq{}}\PYG{l+s+s1}{volume}\PYG{l+s+s1}{\PYGZsq{}}\PYG{p}{]} \PYG{o}{\PYGZhy{}} \PYG{n}{data}\PYG{p}{[}\PYG{l+s+s1}{\PYGZsq{}}\PYG{l+s+s1}{volume\PYGZus{}mean}\PYG{l+s+s1}{\PYGZsq{}}\PYG{p}{]}\PYG{p}{)}\PYG{o}{/}\PYG{n}{data}\PYG{p}{[}\PYG{l+s+s1}{\PYGZsq{}}\PYG{l+s+s1}{volume\PYGZus{}std}\PYG{l+s+s1}{\PYGZsq{}}\PYG{p}{]} \PYG{c+c1}{\PYGZsh{} 거래량은 종목과 주가에 따라 다르기 떄문에 표준화한 값이 필요함}
    \PYG{n}{data}\PYG{p}{[}\PYG{l+s+s1}{\PYGZsq{}}\PYG{l+s+s1}{z\PYGZgt{}1.96}\PYG{l+s+s1}{\PYGZsq{}}\PYG{p}{]} \PYG{o}{=} \PYG{p}{(}\PYG{n}{data}\PYG{p}{[}\PYG{l+s+s1}{\PYGZsq{}}\PYG{l+s+s1}{close}\PYG{l+s+s1}{\PYGZsq{}}\PYG{p}{]} \PYG{o}{\PYGZgt{}} \PYG{n}{data}\PYG{p}{[}\PYG{l+s+s1}{\PYGZsq{}}\PYG{l+s+s1}{open}\PYG{l+s+s1}{\PYGZsq{}}\PYG{p}{]}\PYG{p}{)}\PYG{o}{*}\PYG{p}{(}\PYG{n}{data}\PYG{p}{[}\PYG{l+s+s1}{\PYGZsq{}}\PYG{l+s+s1}{volume\PYGZus{}z}\PYG{l+s+s1}{\PYGZsq{}}\PYG{p}{]} \PYG{o}{\PYGZgt{}} \PYG{l+m+mf}{1.65}\PYG{p}{)}\PYG{o}{.}\PYG{n}{astype}\PYG{p}{(}\PYG{n+nb}{int}\PYG{p}{)} \PYG{c+c1}{\PYGZsh{} 양봉이면서 거래량이 90\PYGZpc{}신뢰구간을 벗어난 날}
    \PYG{n}{data}\PYG{p}{[}\PYG{l+s+s1}{\PYGZsq{}}\PYG{l+s+s1}{num\PYGZus{}z\PYGZgt{}1.96}\PYG{l+s+s1}{\PYGZsq{}}\PYG{p}{]} \PYG{o}{=}  \PYG{n}{data}\PYG{p}{[}\PYG{l+s+s1}{\PYGZsq{}}\PYG{l+s+s1}{z\PYGZgt{}1.96}\PYG{l+s+s1}{\PYGZsq{}}\PYG{p}{]}\PYG{o}{.}\PYG{n}{rolling}\PYG{p}{(}\PYG{l+m+mi}{60}\PYG{p}{)}\PYG{o}{.}\PYG{n}{sum}\PYG{p}{(}\PYG{p}{)}  \PYG{c+c1}{\PYGZsh{} 양봉이면서 거래량이 90\PYGZpc{}신뢰구간을 벗어난 날을 카운트}
    
    \PYG{c+c1}{\PYGZsh{} 주가지수보다 더 좋은 수익율을 보여준다}
    \PYG{n}{data}\PYG{p}{[}\PYG{l+s+s1}{\PYGZsq{}}\PYG{l+s+s1}{num\PYGZus{}win\PYGZus{}market}\PYG{l+s+s1}{\PYGZsq{}}\PYG{p}{]} \PYG{o}{=} \PYG{n}{data}\PYG{p}{[}\PYG{l+s+s1}{\PYGZsq{}}\PYG{l+s+s1}{win\PYGZus{}market}\PYG{l+s+s1}{\PYGZsq{}}\PYG{p}{]}\PYG{o}{.}\PYG{n}{rolling}\PYG{p}{(}\PYG{l+m+mi}{60}\PYG{p}{)}\PYG{o}{.}\PYG{n}{sum}\PYG{p}{(}\PYG{p}{)} \PYG{c+c1}{\PYGZsh{} 주가지수 수익율이 1 보다 작을 때, 종목 수익율이 1 보다 큰 날 수}
    \PYG{n}{data}\PYG{p}{[}\PYG{l+s+s1}{\PYGZsq{}}\PYG{l+s+s1}{pct\PYGZus{}win\PYGZus{}market}\PYG{l+s+s1}{\PYGZsq{}}\PYG{p}{]} \PYG{o}{=} \PYG{p}{(}\PYG{n}{data}\PYG{p}{[}\PYG{l+s+s1}{\PYGZsq{}}\PYG{l+s+s1}{return}\PYG{l+s+s1}{\PYGZsq{}}\PYG{p}{]}\PYG{o}{/}\PYG{n}{data}\PYG{p}{[}\PYG{l+s+s1}{\PYGZsq{}}\PYG{l+s+s1}{kosdaq\PYGZus{}return}\PYG{l+s+s1}{\PYGZsq{}}\PYG{p}{]}\PYG{p}{)}\PYG{o}{.}\PYG{n}{rolling}\PYG{p}{(}\PYG{l+m+mi}{60}\PYG{p}{)}\PYG{o}{.}\PYG{n}{mean}\PYG{p}{(}\PYG{p}{)} \PYG{c+c1}{\PYGZsh{} 주가지수 수익율 대비 종목 수익율}
    
    
    \PYG{c+c1}{\PYGZsh{} 동종업체 수익률보다 더 좋은 수익율을 보여준다.           }
    \PYG{n}{data}\PYG{p}{[}\PYG{l+s+s1}{\PYGZsq{}}\PYG{l+s+s1}{return\PYGZus{}mean}\PYG{l+s+s1}{\PYGZsq{}}\PYG{p}{]} \PYG{o}{=} \PYG{n}{data}\PYG{p}{[}\PYG{l+s+s1}{\PYGZsq{}}\PYG{l+s+s1}{return}\PYG{l+s+s1}{\PYGZsq{}}\PYG{p}{]}\PYG{o}{.}\PYG{n}{rolling}\PYG{p}{(}\PYG{l+m+mi}{60}\PYG{p}{)}\PYG{o}{.}\PYG{n}{mean}\PYG{p}{(}\PYG{p}{)} \PYG{c+c1}{\PYGZsh{} 종목별 최근 60 일 수익율의 평균}
    \PYG{n}{data}\PYG{p}{[}\PYG{l+s+s1}{\PYGZsq{}}\PYG{l+s+s1}{sector}\PYG{l+s+s1}{\PYGZsq{}}\PYG{p}{]} \PYG{o}{=} \PYG{n}{sector} 
       
    \PYG{n}{data}\PYG{p}{[}\PYG{l+s+s1}{\PYGZsq{}}\PYG{l+s+s1}{max\PYGZus{}close}\PYG{l+s+s1}{\PYGZsq{}}\PYG{p}{]}  \PYG{o}{=} \PYG{n}{data}\PYG{p}{[}\PYG{p}{[}\PYG{l+s+s1}{\PYGZsq{}}\PYG{l+s+s1}{close\PYGZus{}r1}\PYG{l+s+s1}{\PYGZsq{}}\PYG{p}{,}\PYG{l+s+s1}{\PYGZsq{}}\PYG{l+s+s1}{close\PYGZus{}r2}\PYG{l+s+s1}{\PYGZsq{}}\PYG{p}{,}\PYG{l+s+s1}{\PYGZsq{}}\PYG{l+s+s1}{close\PYGZus{}r3}\PYG{l+s+s1}{\PYGZsq{}}\PYG{p}{,}\PYG{l+s+s1}{\PYGZsq{}}\PYG{l+s+s1}{close\PYGZus{}r4}\PYG{l+s+s1}{\PYGZsq{}}\PYG{p}{,}\PYG{l+s+s1}{\PYGZsq{}}\PYG{l+s+s1}{close\PYGZus{}r5}\PYG{l+s+s1}{\PYGZsq{}}\PYG{p}{]}\PYG{p}{]}\PYG{o}{.}\PYG{n}{max}\PYG{p}{(}\PYG{n}{axis}\PYG{o}{=}\PYG{l+m+mi}{1}\PYG{p}{)} \PYG{c+c1}{\PYGZsh{} 5 영업일 종가 수익율 중 최고 값   }
    \PYG{n}{data}\PYG{p}{[}\PYG{l+s+s1}{\PYGZsq{}}\PYG{l+s+s1}{mean\PYGZus{}close}\PYG{l+s+s1}{\PYGZsq{}}\PYG{p}{]}  \PYG{o}{=} \PYG{n}{data}\PYG{p}{[}\PYG{p}{[}\PYG{l+s+s1}{\PYGZsq{}}\PYG{l+s+s1}{close\PYGZus{}r1}\PYG{l+s+s1}{\PYGZsq{}}\PYG{p}{,}\PYG{l+s+s1}{\PYGZsq{}}\PYG{l+s+s1}{close\PYGZus{}r2}\PYG{l+s+s1}{\PYGZsq{}}\PYG{p}{,}\PYG{l+s+s1}{\PYGZsq{}}\PYG{l+s+s1}{close\PYGZus{}r3}\PYG{l+s+s1}{\PYGZsq{}}\PYG{p}{,}\PYG{l+s+s1}{\PYGZsq{}}\PYG{l+s+s1}{close\PYGZus{}r4}\PYG{l+s+s1}{\PYGZsq{}}\PYG{p}{,}\PYG{l+s+s1}{\PYGZsq{}}\PYG{l+s+s1}{close\PYGZus{}r5}\PYG{l+s+s1}{\PYGZsq{}}\PYG{p}{]}\PYG{p}{]}\PYG{o}{.}\PYG{n}{mean}\PYG{p}{(}\PYG{n}{axis}\PYG{o}{=}\PYG{l+m+mi}{1}\PYG{p}{)} \PYG{c+c1}{\PYGZsh{} 5 영업일 종가 수익율 중 최고 값   }
    \PYG{n}{data}\PYG{p}{[}\PYG{l+s+s1}{\PYGZsq{}}\PYG{l+s+s1}{min\PYGZus{}close}\PYG{l+s+s1}{\PYGZsq{}}\PYG{p}{]}  \PYG{o}{=} \PYG{n}{data}\PYG{p}{[}\PYG{p}{[}\PYG{l+s+s1}{\PYGZsq{}}\PYG{l+s+s1}{close\PYGZus{}r1}\PYG{l+s+s1}{\PYGZsq{}}\PYG{p}{,}\PYG{l+s+s1}{\PYGZsq{}}\PYG{l+s+s1}{close\PYGZus{}r2}\PYG{l+s+s1}{\PYGZsq{}}\PYG{p}{,}\PYG{l+s+s1}{\PYGZsq{}}\PYG{l+s+s1}{close\PYGZus{}r3}\PYG{l+s+s1}{\PYGZsq{}}\PYG{p}{,}\PYG{l+s+s1}{\PYGZsq{}}\PYG{l+s+s1}{close\PYGZus{}r4}\PYG{l+s+s1}{\PYGZsq{}}\PYG{p}{,}\PYG{l+s+s1}{\PYGZsq{}}\PYG{l+s+s1}{close\PYGZus{}r5}\PYG{l+s+s1}{\PYGZsq{}}\PYG{p}{]}\PYG{p}{]}\PYG{o}{.}\PYG{n}{min}\PYG{p}{(}\PYG{n}{axis}\PYG{o}{=}\PYG{l+m+mi}{1}\PYG{p}{)} \PYG{c+c1}{\PYGZsh{} 5 영업일 종가 수익율 중 최저 값   }

    \PYG{n}{data} \PYG{o}{=} \PYG{n}{data}\PYG{p}{[}\PYG{p}{(}\PYG{n}{data}\PYG{p}{[}\PYG{l+s+s1}{\PYGZsq{}}\PYG{l+s+s1}{price\PYGZus{}std}\PYG{l+s+s1}{\PYGZsq{}}\PYG{p}{]}\PYG{o}{!=}\PYG{l+m+mi}{0}\PYG{p}{)} \PYG{o}{\PYGZam{}} \PYG{p}{(}\PYG{n}{data}\PYG{p}{[}\PYG{l+s+s1}{\PYGZsq{}}\PYG{l+s+s1}{volume\PYGZus{}std}\PYG{l+s+s1}{\PYGZsq{}}\PYG{p}{]}\PYG{o}{!=}\PYG{l+m+mi}{0}\PYG{p}{)}\PYG{p}{]} 
    
    \PYG{n}{feature\PYGZus{}all} \PYG{o}{=} \PYG{n}{pd}\PYG{o}{.}\PYG{n}{concat}\PYG{p}{(}\PYG{p}{[}\PYG{n}{data}\PYG{p}{,} \PYG{n}{feature\PYGZus{}all}\PYG{p}{]}\PYG{p}{,} \PYG{n}{axis}\PYG{o}{=}\PYG{l+m+mi}{0}\PYG{p}{)}
    

\PYG{n}{feature\PYGZus{}all}\PYG{p}{[}\PYG{l+s+s1}{\PYGZsq{}}\PYG{l+s+s1}{sector\PYGZus{}return}\PYG{l+s+s1}{\PYGZsq{}}\PYG{p}{]} \PYG{o}{=} \PYG{n}{feature\PYGZus{}all}\PYG{o}{.}\PYG{n}{groupby}\PYG{p}{(}\PYG{p}{[}\PYG{l+s+s1}{\PYGZsq{}}\PYG{l+s+s1}{sector}\PYG{l+s+s1}{\PYGZsq{}}\PYG{p}{,} \PYG{n}{feature\PYGZus{}all}\PYG{o}{.}\PYG{n}{index}\PYG{p}{]}\PYG{p}{)}\PYG{p}{[}\PYG{l+s+s1}{\PYGZsq{}}\PYG{l+s+s1}{return}\PYG{l+s+s1}{\PYGZsq{}}\PYG{p}{]}\PYG{o}{.}\PYG{n}{transform}\PYG{p}{(}\PYG{k}{lambda} \PYG{n}{x}\PYG{p}{:} \PYG{n}{x}\PYG{o}{.}\PYG{n}{mean}\PYG{p}{(}\PYG{p}{)}\PYG{p}{)} \PYG{c+c1}{\PYGZsh{} 섹터의 평균 수익율 계산}
\PYG{n}{feature\PYGZus{}all}\PYG{p}{[}\PYG{l+s+s1}{\PYGZsq{}}\PYG{l+s+s1}{return over sector}\PYG{l+s+s1}{\PYGZsq{}}\PYG{p}{]} \PYG{o}{=} \PYG{p}{(}\PYG{n}{feature\PYGZus{}all}\PYG{p}{[}\PYG{l+s+s1}{\PYGZsq{}}\PYG{l+s+s1}{return}\PYG{l+s+s1}{\PYGZsq{}}\PYG{p}{]}\PYG{o}{/}\PYG{n}{feature\PYGZus{}all}\PYG{p}{[}\PYG{l+s+s1}{\PYGZsq{}}\PYG{l+s+s1}{sector\PYGZus{}return}\PYG{l+s+s1}{\PYGZsq{}}\PYG{p}{]}\PYG{p}{)} \PYG{c+c1}{\PYGZsh{} 섹터 평균 수익률 대비 종목 수익률 계산}
\PYG{n}{feature\PYGZus{}all}\PYG{o}{.}\PYG{n}{dropna}\PYG{p}{(}\PYG{n}{inplace}\PYG{o}{=}\PYG{k+kc}{True}\PYG{p}{)} \PYG{c+c1}{\PYGZsh{} Missing 값 있는 행 모두 제거}


\PYG{c+c1}{\PYGZsh{} 최종 피처 및 수익률 데이터만으로 구성}
\PYG{n}{feature\PYGZus{}all} \PYG{o}{=} \PYG{n}{feature\PYGZus{}all}\PYG{p}{[}\PYG{p}{[}\PYG{l+s+s1}{\PYGZsq{}}\PYG{l+s+s1}{code}\PYG{l+s+s1}{\PYGZsq{}}\PYG{p}{,} \PYG{l+s+s1}{\PYGZsq{}}\PYG{l+s+s1}{sector}\PYG{l+s+s1}{\PYGZsq{}}\PYG{p}{,}\PYG{l+s+s1}{\PYGZsq{}}\PYG{l+s+s1}{return}\PYG{l+s+s1}{\PYGZsq{}}\PYG{p}{,}\PYG{l+s+s1}{\PYGZsq{}}\PYG{l+s+s1}{kosdaq\PYGZus{}return}\PYG{l+s+s1}{\PYGZsq{}}\PYG{p}{,}\PYG{l+s+s1}{\PYGZsq{}}\PYG{l+s+s1}{price\PYGZus{}z}\PYG{l+s+s1}{\PYGZsq{}}\PYG{p}{,}\PYG{l+s+s1}{\PYGZsq{}}\PYG{l+s+s1}{volume\PYGZus{}z}\PYG{l+s+s1}{\PYGZsq{}}\PYG{p}{,}\PYG{l+s+s1}{\PYGZsq{}}\PYG{l+s+s1}{num\PYGZus{}high/close}\PYG{l+s+s1}{\PYGZsq{}}\PYG{p}{,}\PYG{l+s+s1}{\PYGZsq{}}\PYG{l+s+s1}{num\PYGZus{}long}\PYG{l+s+s1}{\PYGZsq{}}\PYG{p}{,}\PYG{l+s+s1}{\PYGZsq{}}\PYG{l+s+s1}{num\PYGZus{}z\PYGZgt{}1.96}\PYG{l+s+s1}{\PYGZsq{}}\PYG{p}{,}\PYG{l+s+s1}{\PYGZsq{}}\PYG{l+s+s1}{num\PYGZus{}win\PYGZus{}market}\PYG{l+s+s1}{\PYGZsq{}}\PYG{p}{,}\PYG{l+s+s1}{\PYGZsq{}}\PYG{l+s+s1}{pct\PYGZus{}win\PYGZus{}market}\PYG{l+s+s1}{\PYGZsq{}}\PYG{p}{,}\PYG{l+s+s1}{\PYGZsq{}}\PYG{l+s+s1}{return over sector}\PYG{l+s+s1}{\PYGZsq{}}\PYG{p}{,}\PYG{l+s+s1}{\PYGZsq{}}\PYG{l+s+s1}{max\PYGZus{}close}\PYG{l+s+s1}{\PYGZsq{}}\PYG{p}{,}\PYG{l+s+s1}{\PYGZsq{}}\PYG{l+s+s1}{mean\PYGZus{}close}\PYG{l+s+s1}{\PYGZsq{}}\PYG{p}{,}\PYG{l+s+s1}{\PYGZsq{}}\PYG{l+s+s1}{min\PYGZus{}close}\PYG{l+s+s1}{\PYGZsq{}}\PYG{p}{]}\PYG{p}{]}
\PYG{n}{feature\PYGZus{}all}\PYG{o}{.}\PYG{n}{to\PYGZus{}pickle}\PYG{p}{(}\PYG{l+s+s1}{\PYGZsq{}}\PYG{l+s+s1}{feature\PYGZus{}all.pkl}\PYG{l+s+s1}{\PYGZsq{}}\PYG{p}{)}  
\end{sphinxVerbatim}

\end{sphinxuseclass}\end{sphinxVerbatimInput}

\end{sphinxuseclass}
\sphinxAtStartPar
 이제 모델링을 위한 데이터 준비가 끝났습니다. 간단한 프로파일을 뽑아봅니다. 평균과 표준편차 값을 보고, 피처들이 제대로 생성되었는 지 확인합니다. 그리고 price\_z 와 volum\_z 는 같이 분석했을 때 유의미했다는 사실을 기억하면 좋겠습니다.

\begin{sphinxuseclass}{cell}\begin{sphinxVerbatimInput}

\begin{sphinxuseclass}{cell_input}
\begin{sphinxVerbatim}[commandchars=\\\{\}]
\PYG{n}{feature\PYGZus{}all} \PYG{o}{=} \PYG{n}{pd}\PYG{o}{.}\PYG{n}{read\PYGZus{}pickle}\PYG{p}{(}\PYG{l+s+s1}{\PYGZsq{}}\PYG{l+s+s1}{feature\PYGZus{}all.pkl}\PYG{l+s+s1}{\PYGZsq{}}\PYG{p}{)} 
\PYG{n}{feature\PYGZus{}all}\PYG{o}{.}\PYG{n}{describe}\PYG{p}{(}\PYG{n}{percentiles}\PYG{o}{=}\PYG{p}{[}\PYG{l+m+mf}{0.05}\PYG{p}{,} \PYG{l+m+mf}{0.1}\PYG{p}{,} \PYG{l+m+mf}{0.9}\PYG{p}{,} \PYG{l+m+mf}{0.95}\PYG{p}{]}\PYG{p}{)}\PYG{o}{.}\PYG{n}{style}\PYG{o}{.}\PYG{n}{set\PYGZus{}table\PYGZus{}attributes}\PYG{p}{(}\PYG{l+s+s1}{\PYGZsq{}}\PYG{l+s+s1}{style=}\PYG{l+s+s1}{\PYGZdq{}}\PYG{l+s+s1}{font\PYGZhy{}size: 12px}\PYG{l+s+s1}{\PYGZdq{}}\PYG{l+s+s1}{\PYGZsq{}}\PYG{p}{)}\PYG{o}{.}\PYG{n}{format}\PYG{p}{(}\PYG{n}{precision}\PYG{o}{=}\PYG{l+m+mi}{3}\PYG{p}{)}
\end{sphinxVerbatim}

\end{sphinxuseclass}\end{sphinxVerbatimInput}
\begin{sphinxVerbatimOutput}

\begin{sphinxuseclass}{cell_output}
\begin{sphinxVerbatim}[commandchars=\\\{\}]
\PYGZlt{}pandas.io.formats.style.Styler at 0x19b5bda8130\PYGZgt{}
\end{sphinxVerbatim}

\end{sphinxuseclass}\end{sphinxVerbatimOutput}

\end{sphinxuseclass}
\begin{sphinxuseclass}{cell}\begin{sphinxVerbatimInput}

\begin{sphinxuseclass}{cell_input}
\begin{sphinxVerbatim}[commandchars=\\\{\}]
\PYG{k+kn}{import} \PYG{n+nn}{FinanceDataReader} \PYG{k}{as} \PYG{n+nn}{fdr}
\PYG{o}{\PYGZpc{}}\PYG{k}{matplotlib} inline
\PYG{k+kn}{import} \PYG{n+nn}{matplotlib}\PYG{n+nn}{.}\PYG{n+nn}{pyplot} \PYG{k}{as} \PYG{n+nn}{plt}
\PYG{k+kn}{import} \PYG{n+nn}{pandas} \PYG{k}{as} \PYG{n+nn}{pd}
\PYG{k+kn}{import} \PYG{n+nn}{numpy} \PYG{k}{as} \PYG{n+nn}{np}
\PYG{k+kn}{import} \PYG{n+nn}{warnings}
\PYG{n}{warnings}\PYG{o}{.}\PYG{n}{filterwarnings}\PYG{p}{(}\PYG{l+s+s1}{\PYGZsq{}}\PYG{l+s+s1}{ignore}\PYG{l+s+s1}{\PYGZsq{}}\PYG{p}{)}
\PYG{n}{pd}\PYG{o}{.}\PYG{n}{options}\PYG{o}{.}\PYG{n}{display}\PYG{o}{.}\PYG{n}{float\PYGZus{}format} \PYG{o}{=} \PYG{l+s+s1}{\PYGZsq{}}\PYG{l+s+si}{\PYGZob{}:,.3f\PYGZcb{}}\PYG{l+s+s1}{\PYGZsq{}}\PYG{o}{.}\PYG{n}{format}
\end{sphinxVerbatim}

\end{sphinxuseclass}\end{sphinxVerbatimInput}

\end{sphinxuseclass}

\section{모델링 라이브러리 소개}
\label{\detokenize{chapter5/5.2.2_Modeling_Library:id1}}\label{\detokenize{chapter5/5.2.2_Modeling_Library::doc}}
\sphinxAtStartPar
모델은 가설을 활용하여 타겟변수를 예측하는 알고리즘을 만드는 것이라고 생각하시면 될 것 같습니다. 파이썬에서는 모델링을 위하여 여러개의 라이브러리(패키지)를 제공하고 있습니다. 대표적인 모델개발 라이브러리는 Statsmodels, Scikit\sphinxhyphen{}Learn, Keras 등이 있습니다. 책에서 종목 추천으로 사용할 모델은 일반화 가법 모형(Generalized Additive Model) 입니다. 일반화가법모형은 Statsmodels 에서도 구현할 수 있으나, pyGAM 패키지를 사용하면 더 편리합니다.

\sphinxAtStartPar
Statsmodels 는 전통적인 통계모델에 특화되어 있고, Scikit\sphinxhyphen{}Learn 는 머신러닝모델, Keras 는 딥러닝 모델을 개발할 때 활용할 수 있습니다. 라이브러리는 모델을 만드는 패키지가 들어가 있으므로 호출해서 사용하면 됩니다. Statsmodels 은 주로 일반화 선형모형을 구현할 때 주로 사용합니다. 통계 선형모형의 장점은 해석이 가능한 모델을 만들 수 있다는 것입니다. 예를 들면 변수 X 가 1 단위 증가하면 타겟변수는 얼마나 증가하느냐? 등의 해석을 할 수 있습니다..

\sphinxAtStartPar
Scikit\sphinxhyphen{}Learn 은 주로 머신러닝 모델을 만들 때 활용하는 라이브러리입니다. 머신러닝 모델은 각 피쳐의 해석보다는 예측력을 최우선으로 합니다. 특히 트리(Tree) 기반 모델은 변수간의 상호작용을 고려하므로 입력 변수사이에 상호작용이 많을 때 효과가 좋습니다. 머신러닝 모델 중에는 앙상블 모델이 인기인데요. 앙상블도 Bias 를 줄이는데 집중하는 Boosting  모델(예 Ada Boost) 계열과 Variance 를 줄이는데 집중하는 Bagging 모델(예 Random Forest) 계열로 나눌 수 있습니다. Bias 랑 Variance 는 하나를 내리면 하나는 올라가는 특징이 있습니다. 두 명의 양궁선수가 있습니다. 한 명은 과녁 중앙에 골고루 퍼지게 활을 쏘는 능력이 있고, 한 명은 일단 처음 쏜 화살에 근처에 집중에서 쏘는 능력이 있다고 하면 누구를 선택하시겠습니까? 첫 번째 양궁선수는 과녁근방에 골고루 쏘는 분이므로 큰 점수는 못 얻어도 항상 기본점수 이상은 획득하는 안전함이 있습니다. 즉 Variance 가 낮음에 해당합니다. 두 번째 양궁선수는  일단 첫 화살에 중앙에 명중하면, 나머지도 10점을 얻을 수 있습니다. 하지만, 처음 화살이 빗나가면 나머지도 다 빗나갑니다. 따라서 첫 화살이 중요합니다. Bias 가 낮기 때문에 overfitting (과대적합) 을 주의해야 합니다. 운이 안 좋아 중앙에서 먼거리에 첫 화살이 명중했다면, 나머지 화살도 그  근방으로 가므로, 우리가 원하는 해답이 아닌 곳으로 모델학습이 이루어지게 되는 것입니다.

\sphinxAtStartPar
Keras 는 딥러닝을 위한 라이브러리입니다. 데이터 수가 많지 않고,  피쳐의 디멘젼이 5 개(시종고저, 거래량) 라면 데이터 복잡성도 높지 않습니다. 구현하고자 하는 예측모델은 딥러닝이 적절하지는 않아보입니다. 굳이 일봉 데이터가 요약된 피쳐로 뉴럴네트워크 모델을 구현한다면  Multi\sphinxhyphen{}Layer Perceptron (다층 퍼셉트론) 모델을 생각해 볼 수 있습니다. MLP 는 비선형관계를 표현하기 위해서 Activation Function (활성화함수) 를 이용하고,  Activation 함수에서 나온 값을 다시 다음 층의 입력변수로 넣는 형태입니다. 이렇게 함으로써 변수간의 상호작용과 비선형관계를 동시에 표현할 수 있습니다. 사실, 활성화 함수가 Sigmoid  함수인 뉴럴네트워크 모델은 Logistic Regression.모델을 가로 세로층으로 중첩한 것과 동일한 구조가 됩니다. 즉, Logistic Regression 의 확장형으로도 생각할 수 있습니다.  뉴럴네트워크 계통의 모델은 Loss Function (손실함수) 를 만들고 Loss Function 를 최소화하는 네트워크의 가중치를 찾도록 훈련합니다. 많이 쓰는 훈련방식은 오류 역전파(BackPropagation) 입니다. 이런 식의 접근 법은 과대적합이 항상 문제가 됩니다. 따라서 과대적합을 피하기 위해 다양한 기법이 개발 되고 있습니다.

\sphinxAtStartPar
이번절에서는 Statsmodel 과 Scikit\sphinxhyphen{}Learn 라이브러리가 모델 개발에 어떻게 활용되는지 경험해 보는 시간입니다.

\sphinxAtStartPar
모델링을 위해 준비한 데이터를 읽습니다. 그리고 모델의 오버피팅을 최소화하기 위하여 타겟변수를 0 과 1 로 치환합니다. 예를 들어, 5\% 익절의 데이터 표현은 \sphinxhyphen{} ‘max\_close’ 가 5\% 이상일 때 1, 아니면 0 이 됩니다. ‘max\_close’ 가 1 인 비율을 보니, 약 24\% 입니다. 10000 개 샘플을 뽑아 예측모델을 만들고 나머지로 데이터로 테스트(혹은 백테스팅)를 하겠습니다.

\begin{sphinxuseclass}{cell}\begin{sphinxVerbatimInput}

\begin{sphinxuseclass}{cell_input}
\begin{sphinxVerbatim}[commandchars=\\\{\}]
\PYG{n}{feature\PYGZus{}all} \PYG{o}{=} \PYG{n}{pd}\PYG{o}{.}\PYG{n}{read\PYGZus{}pickle}\PYG{p}{(}\PYG{l+s+s1}{\PYGZsq{}}\PYG{l+s+s1}{feature\PYGZus{}all.pkl}\PYG{l+s+s1}{\PYGZsq{}}\PYG{p}{)} 
\PYG{n}{feature\PYGZus{}all}\PYG{p}{[}\PYG{l+s+s1}{\PYGZsq{}}\PYG{l+s+s1}{target}\PYG{l+s+s1}{\PYGZsq{}}\PYG{p}{]} \PYG{o}{=} \PYG{n}{np}\PYG{o}{.}\PYG{n}{where}\PYG{p}{(}\PYG{n}{feature\PYGZus{}all}\PYG{p}{[}\PYG{l+s+s1}{\PYGZsq{}}\PYG{l+s+s1}{max\PYGZus{}close}\PYG{l+s+s1}{\PYGZsq{}}\PYG{p}{]}\PYG{o}{\PYGZgt{}}\PYG{o}{=} \PYG{l+m+mf}{1.05}\PYG{p}{,} \PYG{l+m+mi}{1}\PYG{p}{,} \PYG{l+m+mi}{0}\PYG{p}{)}
\PYG{n}{target} \PYG{o}{=} \PYG{n}{feature\PYGZus{}all}\PYG{p}{[}\PYG{l+s+s1}{\PYGZsq{}}\PYG{l+s+s1}{target}\PYG{l+s+s1}{\PYGZsq{}}\PYG{p}{]}\PYG{o}{.}\PYG{n}{mean}\PYG{p}{(}\PYG{p}{)}
\PYG{n+nb}{print}\PYG{p}{(}\PYG{l+s+sa}{f}\PYG{l+s+s1}{\PYGZsq{}}\PYG{l+s+s1}{\PYGZpc{} of target:}\PYG{l+s+si}{\PYGZob{}}\PYG{n}{target}\PYG{l+s+si}{:}\PYG{l+s+s1}{ 5.1\PYGZpc{}}\PYG{l+s+si}{\PYGZcb{}}\PYG{l+s+s1}{\PYGZsq{}}\PYG{p}{)}

\PYG{n}{mdl\PYGZus{}all} \PYG{o}{=} \PYG{n}{feature\PYGZus{}all}\PYG{o}{.}\PYG{n}{set\PYGZus{}index}\PYG{p}{(}\PYG{p}{[}\PYG{n}{feature\PYGZus{}all}\PYG{o}{.}\PYG{n}{index}\PYG{p}{,}\PYG{l+s+s1}{\PYGZsq{}}\PYG{l+s+s1}{code}\PYG{l+s+s1}{\PYGZsq{}}\PYG{p}{]}\PYG{p}{)}

\PYG{n}{train} \PYG{o}{=} \PYG{n}{mdl\PYGZus{}all}\PYG{o}{.}\PYG{n}{sample}\PYG{p}{(}\PYG{l+m+mi}{10000}\PYG{p}{,} \PYG{n}{random\PYGZus{}state}\PYG{o}{=}\PYG{l+m+mi}{124}\PYG{p}{)}
\PYG{n}{test} \PYG{o}{=} \PYG{n}{mdl\PYGZus{}all}\PYG{o}{.}\PYG{n}{loc}\PYG{p}{[}\PYG{o}{\PYGZti{}}\PYG{n}{mdl\PYGZus{}all}\PYG{o}{.}\PYG{n}{index}\PYG{o}{.}\PYG{n}{isin}\PYG{p}{(}\PYG{n}{train}\PYG{o}{.}\PYG{n}{index}\PYG{p}{)}\PYG{p}{]}
\end{sphinxVerbatim}

\end{sphinxuseclass}\end{sphinxVerbatimInput}
\begin{sphinxVerbatimOutput}

\begin{sphinxuseclass}{cell_output}
\begin{sphinxVerbatim}[commandchars=\\\{\}]
\PYGZpc{} of target: 24.3\PYGZpc{}
\end{sphinxVerbatim}

\end{sphinxuseclass}\end{sphinxVerbatimOutput}

\end{sphinxuseclass}

\section{Statsmodels \sphinxhyphen{} Logistic Regression}
\label{\detokenize{chapter5/5.2.2_Modeling_Library:statsmodels-logistic-regression}}
\sphinxAtStartPar
아래 코드는 Statsmodels 라이브러리에 대한 이해가 목적입니다. Statsmodels 는 전통적인 통계모델을 구현하는데 주로 활용하는데요. 통계모델의 장점은 변수의 해석이 가능하다는 것입니다. 아래 코드는 랜덤해게 뽑은 5천개의 샘플로 모델을 만들고, 나머지 데이터로 모델 성능을 테스트하는 과정입니다. 모델 개발은 여기서부터 시작입니다.  결과를 보면 P Value(P>|z|) 가 0.01(유의수준) 보다 큰 변수가 많은데요. P Value(P>|z|) 가 유의수준보다 크다는 이야기는 coefficient 가 0 일 가능성이 높다는 말이고, Coefficient 가 0 이라는 말은 예측에 도움을 안 준다는 말입니다. 이런 변수들은 적절한 변형을 통하여 유의미하게 만들거나 제거해야 합니다. 가장 대표적인 방법이 Binning 입니다. 이 절은 라이브러리를 소개하는 것이 목적이라, 모델 완성을 위한 나머지 과정은 생략하도록 하겠습니다. 제가 통계모델의 장점으로 해석을 언급했는데요. 아직 모델이 완성되지 않았지만, 변수 ‘volume\_z’ 를 해석해 보도록 하겠습니다. ‘volume\_z’ 는 과거 20일대비 당일 거래량이 얼마나 많은 지를 의미하는 변수입니다. ‘volume\_z’ 가 1 증가하면 log(odds) 는 그 변수의 계수 0.1765 만큼 증가하게 됩니다. 즉, odds 는 exp(0.1765) 증가하게 됩니다. 풀어서 이야기하면, 전일 20일 대비 당일 거래량 표준화 값 z 가  1 증가할 때마다, 5\% 로 익절할 odds(=p/1\sphinxhyphen{}p)는 exp(0.1765) 증가한다고 말할 수 있습니다.

\sphinxAtStartPar
모델을 완성하기까지 필요한 나머지 절차는 아래와 같습니다.
\begin{enumerate}
\sphinxsetlistlabels{\arabic}{enumi}{enumii}{}{.}%
\item {} 
\sphinxAtStartPar
각 설명변수와 타겟변수와 관계를 분석합니다 (변수간에 상호작용 강한 지 체크)

\item {} 
\sphinxAtStartPar
선형적인 관계가 없는 변수는 binning 등을 통해 문제를 해결합니다. 혹은 제곱근, 제곱, 로그 등의 변형으로 선형적으로 만들 수 도 있습니다.

\item {} 
\sphinxAtStartPar
다중 공선성이 의심되는 변수는 제거하거나 새로운 변수로 대체합니다. (다중 공선성이 높은 모델은 변수의 해석이 부정확함)

\item {} 
\sphinxAtStartPar
테스트 데이터셋과 예측성능을 비교합니다 (오버피팅 여부 확인).

\item {} 
\sphinxAtStartPar
변수를 해석하고 예측값을 만듭니다.

\end{enumerate}

\begin{sphinxuseclass}{cell}\begin{sphinxVerbatimInput}

\begin{sphinxuseclass}{cell_input}
\begin{sphinxVerbatim}[commandchars=\\\{\}]
\PYG{k+kn}{import} \PYG{n+nn}{statsmodels}\PYG{n+nn}{.}\PYG{n+nn}{api} \PYG{k}{as} \PYG{n+nn}{sm}

\PYG{n}{feature\PYGZus{}list} \PYG{o}{=} \PYG{p}{[}\PYG{l+s+s1}{\PYGZsq{}}\PYG{l+s+s1}{price\PYGZus{}z}\PYG{l+s+s1}{\PYGZsq{}}\PYG{p}{,} \PYG{l+s+s1}{\PYGZsq{}}\PYG{l+s+s1}{volume\PYGZus{}z}\PYG{l+s+s1}{\PYGZsq{}}\PYG{p}{,} \PYG{l+s+s1}{\PYGZsq{}}\PYG{l+s+s1}{num\PYGZus{}high/close}\PYG{l+s+s1}{\PYGZsq{}}\PYG{p}{,} \PYG{l+s+s1}{\PYGZsq{}}\PYG{l+s+s1}{num\PYGZus{}long}\PYG{l+s+s1}{\PYGZsq{}}\PYG{p}{,} \PYG{l+s+s1}{\PYGZsq{}}\PYG{l+s+s1}{num\PYGZus{}z\PYGZgt{}1.96}\PYG{l+s+s1}{\PYGZsq{}}\PYG{p}{,} \PYG{l+s+s1}{\PYGZsq{}}\PYG{l+s+s1}{num\PYGZus{}win\PYGZus{}market}\PYG{l+s+s1}{\PYGZsq{}}\PYG{p}{,} \PYG{l+s+s1}{\PYGZsq{}}\PYG{l+s+s1}{pct\PYGZus{}win\PYGZus{}market}\PYG{l+s+s1}{\PYGZsq{}}\PYG{p}{,} \PYG{l+s+s1}{\PYGZsq{}}\PYG{l+s+s1}{return over sector}\PYG{l+s+s1}{\PYGZsq{}}\PYG{p}{]}

\PYG{n}{X} \PYG{o}{=} \PYG{n}{train}\PYG{p}{[}\PYG{n}{feature\PYGZus{}list}\PYG{p}{]}
\PYG{n}{y} \PYG{o}{=} \PYG{n}{train}\PYG{p}{[}\PYG{l+s+s1}{\PYGZsq{}}\PYG{l+s+s1}{target}\PYG{l+s+s1}{\PYGZsq{}}\PYG{p}{]}

\PYG{n}{X} \PYG{o}{=} \PYG{n}{sm}\PYG{o}{.}\PYG{n}{add\PYGZus{}constant}\PYG{p}{(}\PYG{n}{X}\PYG{p}{)}
\PYG{n}{model} \PYG{o}{=} \PYG{n}{sm}\PYG{o}{.}\PYG{n}{Logit}\PYG{p}{(}\PYG{n}{y}\PYG{p}{,} \PYG{n}{X}\PYG{p}{)}
\PYG{n}{results} \PYG{o}{=} \PYG{n}{model}\PYG{o}{.}\PYG{n}{fit}\PYG{p}{(}\PYG{p}{)}
\PYG{n+nb}{print}\PYG{p}{(}\PYG{n}{results}\PYG{o}{.}\PYG{n}{summary}\PYG{p}{(}\PYG{p}{)}\PYG{p}{)}
\PYG{n}{yhat} \PYG{o}{=} \PYG{n}{results}\PYG{o}{.}\PYG{n}{predict}\PYG{p}{(}\PYG{n}{X}\PYG{p}{)}
\PYG{n}{yhat} \PYG{o}{=} \PYG{n}{pd}\PYG{o}{.}\PYG{n}{Series}\PYG{p}{(}\PYG{n}{yhat}\PYG{p}{,} \PYG{n}{name}\PYG{o}{=}\PYG{l+s+s1}{\PYGZsq{}}\PYG{l+s+s1}{yhat}\PYG{l+s+s1}{\PYGZsq{}}\PYG{p}{)}

\PYG{n}{X\PYGZus{}test} \PYG{o}{=} \PYG{n}{test}\PYG{p}{[}\PYG{n}{feature\PYGZus{}list}\PYG{p}{]}
\PYG{n}{y\PYGZus{}test} \PYG{o}{=} \PYG{n}{test}\PYG{p}{[}\PYG{l+s+s1}{\PYGZsq{}}\PYG{l+s+s1}{target}\PYG{l+s+s1}{\PYGZsq{}}\PYG{p}{]}
\PYG{n}{X\PYGZus{}test} \PYG{o}{=} \PYG{n}{sm}\PYG{o}{.}\PYG{n}{add\PYGZus{}constant}\PYG{p}{(}\PYG{n}{X\PYGZus{}test}\PYG{p}{)}
\PYG{n}{yhat\PYGZus{}test} \PYG{o}{=} \PYG{n}{results}\PYG{o}{.}\PYG{n}{predict}\PYG{p}{(}\PYG{n}{X\PYGZus{}test}\PYG{p}{)}
\PYG{n}{yhat\PYGZus{}test} \PYG{o}{=} \PYG{n}{pd}\PYG{o}{.}\PYG{n}{Series}\PYG{p}{(}\PYG{n}{yhat\PYGZus{}test}\PYG{p}{,} \PYG{n}{name}\PYG{o}{=}\PYG{l+s+s1}{\PYGZsq{}}\PYG{l+s+s1}{yhat}\PYG{l+s+s1}{\PYGZsq{}}\PYG{p}{)}
\end{sphinxVerbatim}

\end{sphinxuseclass}\end{sphinxVerbatimInput}
\begin{sphinxVerbatimOutput}

\begin{sphinxuseclass}{cell_output}
\begin{sphinxVerbatim}[commandchars=\\\{\}]
Optimization terminated successfully.
         Current function value: 0.549822
         Iterations 6
                           Logit Regression Results                           
==============================================================================
Dep. Variable:                 target   No. Observations:                10000
Model:                          Logit   Df Residuals:                     9991
Method:                           MLE   Df Model:                            8
Date:                Sun, 26 Jun 2022   Pseudo R\PYGZhy{}squ.:                 0.01089
Time:                        05:30:08   Log\PYGZhy{}Likelihood:                \PYGZhy{}5498.2
converged:                       True   LL\PYGZhy{}Null:                       \PYGZhy{}5558.7
Covariance Type:            nonrobust   LLR p\PYGZhy{}value:                 2.047e\PYGZhy{}22
======================================================================================
                         coef    std err          z      P\PYGZgt{}|z|      [0.025      0.975]
\PYGZhy{}\PYGZhy{}\PYGZhy{}\PYGZhy{}\PYGZhy{}\PYGZhy{}\PYGZhy{}\PYGZhy{}\PYGZhy{}\PYGZhy{}\PYGZhy{}\PYGZhy{}\PYGZhy{}\PYGZhy{}\PYGZhy{}\PYGZhy{}\PYGZhy{}\PYGZhy{}\PYGZhy{}\PYGZhy{}\PYGZhy{}\PYGZhy{}\PYGZhy{}\PYGZhy{}\PYGZhy{}\PYGZhy{}\PYGZhy{}\PYGZhy{}\PYGZhy{}\PYGZhy{}\PYGZhy{}\PYGZhy{}\PYGZhy{}\PYGZhy{}\PYGZhy{}\PYGZhy{}\PYGZhy{}\PYGZhy{}\PYGZhy{}\PYGZhy{}\PYGZhy{}\PYGZhy{}\PYGZhy{}\PYGZhy{}\PYGZhy{}\PYGZhy{}\PYGZhy{}\PYGZhy{}\PYGZhy{}\PYGZhy{}\PYGZhy{}\PYGZhy{}\PYGZhy{}\PYGZhy{}\PYGZhy{}\PYGZhy{}\PYGZhy{}\PYGZhy{}\PYGZhy{}\PYGZhy{}\PYGZhy{}\PYGZhy{}\PYGZhy{}\PYGZhy{}\PYGZhy{}\PYGZhy{}\PYGZhy{}\PYGZhy{}\PYGZhy{}\PYGZhy{}\PYGZhy{}\PYGZhy{}\PYGZhy{}\PYGZhy{}\PYGZhy{}\PYGZhy{}\PYGZhy{}\PYGZhy{}\PYGZhy{}\PYGZhy{}\PYGZhy{}\PYGZhy{}\PYGZhy{}\PYGZhy{}\PYGZhy{}\PYGZhy{}
const                \PYGZhy{}24.3868      6.812     \PYGZhy{}3.580      0.000     \PYGZhy{}37.739     \PYGZhy{}11.035
price\PYGZus{}z               \PYGZhy{}0.1266      0.021     \PYGZhy{}6.156      0.000      \PYGZhy{}0.167      \PYGZhy{}0.086
volume\PYGZus{}z               0.1491      0.024      6.255      0.000       0.102       0.196
num\PYGZus{}high/close         0.1396      0.058      2.411      0.016       0.026       0.253
num\PYGZus{}long              \PYGZhy{}0.1293      0.172     \PYGZhy{}0.752      0.452      \PYGZhy{}0.466       0.208
num\PYGZus{}z\PYGZgt{}1.96             0.0257      0.013      1.982      0.047       0.000       0.051
num\PYGZus{}win\PYGZus{}market         0.0251      0.009      2.722      0.006       0.007       0.043
pct\PYGZus{}win\PYGZus{}market        23.8495      6.797      3.509      0.000      10.527      37.172
return over sector    \PYGZhy{}0.8839      0.835     \PYGZhy{}1.058      0.290      \PYGZhy{}2.521       0.753
======================================================================================
\end{sphinxVerbatim}

\end{sphinxuseclass}\end{sphinxVerbatimOutput}

\end{sphinxuseclass}
\sphinxAtStartPar
 개발 데이터가 아니라, 테스트데이터에서도 좋은 성능을 보이는지 확인해봅니다. 쉽게 확인하는 방법은 Decile 분석입니다. 예측값의 변별력을 알기 위해서 정렬된 예측값을 10 개 구간으로 나누고, 각 구간에서 ‘target’의 평균값을 찍어봅니다. 파란색이 개발데이터, 주황색이 테스트 데이터입니다. 모델이 예측력이 좋다면, 예측값의 십분위 수가 증가하면 5\%로 익절할 확률도 같이 증가하는 형태를 보이게 됩니다. 아래 결과에서 완성되지 않은 모델이지만 단조증가하는 좋은 흐름을 보여주고 있습니다. 테스트 결과에서 제 1 십분위수(첫 번째 구간) 에서 종목을 선택한다면 19.7\% 로 익절할 확률이 있지만, 제 10 분위수(마지막 구간)에서 종목을 선택한다면 34.9\% 로 익절할 확률이 생깁니다.

\begin{sphinxuseclass}{cell}\begin{sphinxVerbatimInput}

\begin{sphinxuseclass}{cell_input}
\begin{sphinxVerbatim}[commandchars=\\\{\}]
\PYG{k}{def} \PYG{n+nf}{perf}\PYG{p}{(}\PYG{n}{y}\PYG{p}{,} \PYG{n}{yhat}\PYG{p}{)}\PYG{p}{:} \PYG{c+c1}{\PYGZsh{} Decile 분석 함수}
    \PYG{n}{combined} \PYG{o}{=} \PYG{n}{pd}\PYG{o}{.}\PYG{n}{concat}\PYG{p}{(}\PYG{p}{[}\PYG{n}{y}\PYG{p}{,} \PYG{n}{yhat}\PYG{p}{]}\PYG{p}{,} \PYG{n}{axis}\PYG{o}{=}\PYG{l+m+mi}{1}\PYG{p}{)}
    \PYG{n}{ranks} \PYG{o}{=} \PYG{n}{pd}\PYG{o}{.}\PYG{n}{qcut}\PYG{p}{(}\PYG{n}{combined}\PYG{p}{[}\PYG{l+s+s1}{\PYGZsq{}}\PYG{l+s+s1}{yhat}\PYG{l+s+s1}{\PYGZsq{}}\PYG{p}{]}\PYG{p}{,} \PYG{n}{q}\PYG{o}{=}\PYG{l+m+mi}{10}\PYG{p}{)}
    \PYG{n+nb}{print}\PYG{p}{(}\PYG{n}{combined}\PYG{o}{.}\PYG{n}{groupby}\PYG{p}{(}\PYG{n}{ranks}\PYG{p}{)}\PYG{p}{[}\PYG{l+s+s1}{\PYGZsq{}}\PYG{l+s+s1}{target}\PYG{l+s+s1}{\PYGZsq{}}\PYG{p}{]}\PYG{o}{.}\PYG{n}{agg}\PYG{p}{(}\PYG{p}{[}\PYG{l+s+s1}{\PYGZsq{}}\PYG{l+s+s1}{count}\PYG{l+s+s1}{\PYGZsq{}}\PYG{p}{,}\PYG{l+s+s1}{\PYGZsq{}}\PYG{l+s+s1}{mean}\PYG{l+s+s1}{\PYGZsq{}}\PYG{p}{]}\PYG{p}{)}\PYG{p}{)}
    \PYG{n}{combined}\PYG{o}{.}\PYG{n}{groupby}\PYG{p}{(}\PYG{n}{ranks}\PYG{p}{)}\PYG{p}{[}\PYG{l+s+s1}{\PYGZsq{}}\PYG{l+s+s1}{target}\PYG{l+s+s1}{\PYGZsq{}}\PYG{p}{]}\PYG{o}{.}\PYG{n}{mean}\PYG{p}{(}\PYG{p}{)}\PYG{o}{.}\PYG{n}{plot}\PYG{p}{(}\PYG{p}{)}

\PYG{n}{perf}\PYG{p}{(}\PYG{n}{y}\PYG{p}{,} \PYG{n}{yhat}\PYG{p}{)}
\PYG{n}{perf}\PYG{p}{(}\PYG{n}{y\PYGZus{}test}\PYG{p}{,} \PYG{n}{yhat\PYGZus{}test}\PYG{p}{)}
\end{sphinxVerbatim}

\end{sphinxuseclass}\end{sphinxVerbatimInput}
\begin{sphinxVerbatimOutput}

\begin{sphinxuseclass}{cell_output}
\begin{sphinxVerbatim}[commandchars=\\\{\}]
                count  mean
yhat                       
(0.138, 0.192]   1000 0.204
(0.192, 0.206]   1000 0.173
(0.206, 0.218]   1000 0.236
(0.218, 0.227]   1000 0.225
(0.227, 0.237]   1000 0.208
(0.237, 0.248]   1000 0.228
(0.248, 0.259]   1000 0.240
(0.259, 0.275]   1000 0.266
(0.275, 0.302]   1000 0.303
(0.302, 0.637]   1000 0.359
                              count  mean
yhat                                     
(0.11699999999999999, 0.192]  31931 0.195
(0.192, 0.205]                31931 0.198
(0.205, 0.216]                31930 0.212
(0.216, 0.226]                31931 0.221
(0.226, 0.236]                31931 0.225
(0.236, 0.247]                31930 0.230
(0.247, 0.259]                31931 0.248
(0.259, 0.275]                31930 0.260
(0.275, 0.302]                31931 0.292
(0.302, 0.728]                31931 0.349
\end{sphinxVerbatim}

\noindent\sphinxincludegraphics{{5.2.2_Modeling_Library_7_1}.png}

\end{sphinxuseclass}\end{sphinxVerbatimOutput}

\end{sphinxuseclass}

\section{SK\sphinxhyphen{}Learn \sphinxhyphen{} Logistic Regression}
\label{\detokenize{chapter5/5.2.2_Modeling_Library:sk-learn-logistic-regression}}
\sphinxAtStartPar
Scikit\sphinxhyphen{}Learn 에서도 Logistic Regression 을 지원합니다. 하지만 계수를 추정하는 방식이 Statsmodels 과는 다른데요. Scikit\sphinxhyphen{}Learn Logistic Regression 은 loss 함수를 만들고, 과대적합을 해결하기 위해 penalty term (L1/L2) 도 추가합니다. 이런 방식으로 Penalty Term 이 있는 Loss 함수를 최소화하는 방식을 계수를 찾을 때는 입력 피처의 스케일이 동일해야 의미가 있습니다. 아래 코드에서 입력 피쳐를 Scaling 하는 부분이 반드시 들어가야 합니다. 아래 Test 데이터의 결과가 Train 데이터보다 모델성능의 차이가 크지는 않습니다. 즉 overfitting(과대적합)이 심하지는 않아 보입니다.

\sphinxAtStartPar
SK\sphinxhyphen{}Learn 으로 만드는 모델은 절차가 아래와 같습니다.
\begin{enumerate}
\sphinxsetlistlabels{\arabic}{enumi}{enumii}{}{.}%
\item {} 
\sphinxAtStartPar
입력 피처 스케일링 (Loss 함수 + Penalty Term 로 훈련하는 모델만 해당)

\item {} 
\sphinxAtStartPar
하이퍼파라미터 튜닝 (성능을 최고로 만드는 하이퍼파라미터을 찾기)

\item {} 
\sphinxAtStartPar
테스트 데이터셋과 예측성능을 비교합니다 (오버피팅 여부 확인)

\item {} 
\sphinxAtStartPar
변수의 중요도 파악과 이해

\item {} 
\sphinxAtStartPar
예측값 만들기

\end{enumerate}

\begin{sphinxuseclass}{cell}\begin{sphinxVerbatimInput}

\begin{sphinxuseclass}{cell_input}
\begin{sphinxVerbatim}[commandchars=\\\{\}]
\PYG{k+kn}{from} \PYG{n+nn}{sklearn}\PYG{n+nn}{.}\PYG{n+nn}{linear\PYGZus{}model} \PYG{k+kn}{import} \PYG{n}{LogisticRegression}
\PYG{k+kn}{from} \PYG{n+nn}{sklearn}\PYG{n+nn}{.}\PYG{n+nn}{preprocessing} \PYG{k+kn}{import} \PYG{n}{StandardScaler}

\PYG{n}{feature\PYGZus{}list} \PYG{o}{=} \PYG{p}{[}\PYG{l+s+s1}{\PYGZsq{}}\PYG{l+s+s1}{price\PYGZus{}z}\PYG{l+s+s1}{\PYGZsq{}}\PYG{p}{,} \PYG{l+s+s1}{\PYGZsq{}}\PYG{l+s+s1}{volume\PYGZus{}z}\PYG{l+s+s1}{\PYGZsq{}}\PYG{p}{,} \PYG{l+s+s1}{\PYGZsq{}}\PYG{l+s+s1}{num\PYGZus{}high/close}\PYG{l+s+s1}{\PYGZsq{}}\PYG{p}{,} \PYG{l+s+s1}{\PYGZsq{}}\PYG{l+s+s1}{num\PYGZus{}long}\PYG{l+s+s1}{\PYGZsq{}}\PYG{p}{,} \PYG{l+s+s1}{\PYGZsq{}}\PYG{l+s+s1}{num\PYGZus{}z\PYGZgt{}1.96}\PYG{l+s+s1}{\PYGZsq{}}\PYG{p}{,} \PYG{l+s+s1}{\PYGZsq{}}\PYG{l+s+s1}{num\PYGZus{}win\PYGZus{}market}\PYG{l+s+s1}{\PYGZsq{}}\PYG{p}{,} \PYG{l+s+s1}{\PYGZsq{}}\PYG{l+s+s1}{pct\PYGZus{}win\PYGZus{}market}\PYG{l+s+s1}{\PYGZsq{}}\PYG{p}{,} \PYG{l+s+s1}{\PYGZsq{}}\PYG{l+s+s1}{return over sector}\PYG{l+s+s1}{\PYGZsq{}}\PYG{p}{]}

\PYG{n}{X} \PYG{o}{=} \PYG{n}{train}\PYG{p}{[}\PYG{n}{feature\PYGZus{}list}\PYG{p}{]}
\PYG{n}{y} \PYG{o}{=} \PYG{n}{train}\PYG{p}{[}\PYG{l+s+s1}{\PYGZsq{}}\PYG{l+s+s1}{target}\PYG{l+s+s1}{\PYGZsq{}}\PYG{p}{]}

\PYG{n}{X\PYGZus{}test} \PYG{o}{=} \PYG{n}{test}\PYG{p}{[}\PYG{n}{feature\PYGZus{}list}\PYG{p}{]}
\PYG{n}{y\PYGZus{}test} \PYG{o}{=} \PYG{n}{test}\PYG{p}{[}\PYG{l+s+s1}{\PYGZsq{}}\PYG{l+s+s1}{target}\PYG{l+s+s1}{\PYGZsq{}}\PYG{p}{]}

\PYG{c+c1}{\PYGZsh{} 입력 피처 표준화}
\PYG{n}{scaler} \PYG{o}{=} \PYG{n}{StandardScaler}\PYG{p}{(}\PYG{p}{)} \PYG{c+c1}{\PYGZsh{} 평균이 0 이고 편차가 1 가 되도록 피처 표준화}
\PYG{n}{scaler}\PYG{o}{.}\PYG{n}{fit\PYGZus{}transform}\PYG{p}{(}\PYG{n}{X}\PYG{p}{)}
\PYG{n}{scaler}\PYG{o}{.}\PYG{n}{transform}\PYG{p}{(}\PYG{n}{X\PYGZus{}test}\PYG{p}{)}

\PYG{n}{lr} \PYG{o}{=} \PYG{n}{LogisticRegression}\PYG{p}{(}\PYG{n}{fit\PYGZus{}intercept}\PYG{o}{=}\PYG{k+kc}{True}\PYG{p}{,} \PYG{n}{C}\PYG{o}{=}\PYG{l+m+mi}{1}\PYG{p}{)} \PYG{c+c1}{\PYGZsh{} 정해진 하이퍼파라미터를 가진 객체를 생성, C 값은 다중 공선성을 제거하기 위한 페널티의 가중치이며 디폴트는 L2(Ridge) 페널티}
\PYG{n}{lr}\PYG{o}{.}\PYG{n}{fit}\PYG{p}{(}\PYG{n}{X}\PYG{p}{,} \PYG{n}{y}\PYG{p}{)}
\PYG{n+nb}{print}\PYG{p}{(}\PYG{n}{lr}\PYG{o}{.}\PYG{n}{coef\PYGZus{}}\PYG{p}{)}
\PYG{n}{yhat} \PYG{o}{=} \PYG{n}{lr}\PYG{o}{.}\PYG{n}{predict\PYGZus{}proba}\PYG{p}{(}\PYG{n}{X}\PYG{p}{)}\PYG{p}{[}\PYG{p}{:}\PYG{p}{,}\PYG{l+m+mi}{1}\PYG{p}{]}
\PYG{n}{yhat\PYGZus{}test} \PYG{o}{=} \PYG{n}{lr}\PYG{o}{.}\PYG{n}{predict\PYGZus{}proba}\PYG{p}{(}\PYG{n}{X\PYGZus{}test}\PYG{p}{)}\PYG{p}{[}\PYG{p}{:}\PYG{p}{,}\PYG{l+m+mi}{1}\PYG{p}{]}

\PYG{n}{yhat} \PYG{o}{=} \PYG{n}{pd}\PYG{o}{.}\PYG{n}{Series}\PYG{p}{(}\PYG{n}{yhat}\PYG{p}{,} \PYG{n}{name}\PYG{o}{=}\PYG{l+s+s1}{\PYGZsq{}}\PYG{l+s+s1}{yhat}\PYG{l+s+s1}{\PYGZsq{}}\PYG{p}{,} \PYG{n}{index}\PYG{o}{=}\PYG{n}{y}\PYG{o}{.}\PYG{n}{index}\PYG{p}{)} 
\PYG{n}{yhat\PYGZus{}test} \PYG{o}{=} \PYG{n}{pd}\PYG{o}{.}\PYG{n}{Series}\PYG{p}{(}\PYG{n}{yhat\PYGZus{}test}\PYG{p}{,} \PYG{n}{name}\PYG{o}{=}\PYG{l+s+s1}{\PYGZsq{}}\PYG{l+s+s1}{yhat}\PYG{l+s+s1}{\PYGZsq{}}\PYG{p}{,} \PYG{n}{index}\PYG{o}{=}\PYG{n}{y\PYGZus{}test}\PYG{o}{.}\PYG{n}{index}\PYG{p}{)} 
\end{sphinxVerbatim}

\end{sphinxuseclass}\end{sphinxVerbatimInput}
\begin{sphinxVerbatimOutput}

\begin{sphinxuseclass}{cell_output}
\begin{sphinxVerbatim}[commandchars=\\\{\}]
[[\PYGZhy{}0.10997133  0.14023527  0.15290205 \PYGZhy{}0.03634754  0.04334864  0.03748485
   0.51234537 \PYGZhy{}0.51150234]]
\end{sphinxVerbatim}

\end{sphinxuseclass}\end{sphinxVerbatimOutput}

\end{sphinxuseclass}
\begin{sphinxuseclass}{cell}\begin{sphinxVerbatimInput}

\begin{sphinxuseclass}{cell_input}
\begin{sphinxVerbatim}[commandchars=\\\{\}]
\PYG{n}{perf}\PYG{p}{(}\PYG{n}{y}\PYG{p}{,} \PYG{n}{yhat}\PYG{p}{)}
\PYG{n}{perf}\PYG{p}{(}\PYG{n}{y\PYGZus{}test}\PYG{p}{,} \PYG{n}{yhat\PYGZus{}test}\PYG{p}{)}
\PYG{n}{plt}\PYG{o}{.}\PYG{n}{show}\PYG{p}{(}\PYG{p}{)}
\end{sphinxVerbatim}

\end{sphinxuseclass}\end{sphinxVerbatimInput}
\begin{sphinxVerbatimOutput}

\begin{sphinxuseclass}{cell_output}
\begin{sphinxVerbatim}[commandchars=\\\{\}]
                count  mean
yhat                       
(0.142, 0.194]   1000 0.175
(0.194, 0.208]   1000 0.201
(0.208, 0.219]   1000 0.224
(0.219, 0.228]   1000 0.236
(0.228, 0.238]   1000 0.217
(0.238, 0.248]   1000 0.234
(0.248, 0.26]    1000 0.270
(0.26, 0.275]    1000 0.249
(0.275, 0.3]     1000 0.294
(0.3, 0.632]     1000 0.342
                count  mean
yhat                       
(0.126, 0.194]  31931 0.187
(0.194, 0.207]  31931 0.207
(0.207, 0.218]  31930 0.217
(0.218, 0.228]  31931 0.218
(0.228, 0.237]  31931 0.226
(0.237, 0.247]  31930 0.235
(0.247, 0.259]  31931 0.254
(0.259, 0.274]  31930 0.258
(0.274, 0.299]  31931 0.289
(0.299, 0.652]  31931 0.338
\end{sphinxVerbatim}

\noindent\sphinxincludegraphics{{5.2.2_Modeling_Library_10_1}.png}

\end{sphinxuseclass}\end{sphinxVerbatimOutput}

\end{sphinxuseclass}

\section{SK\sphinxhyphen{}learn \sphinxhyphen{} Random Forerst}
\label{\detokenize{chapter5/5.2.2_Modeling_Library:sk-learn-random-forerst}}
\sphinxAtStartPar
Random Forest 는 Scikit\sphinxhyphen{}Learn 에서 인기있는 모델인데요. Decision\sphinxhyphen{}Tree(의사결정나무)의 문제점을 보완하고자 나온 개념입니다. 모델을 훈련시키기 위한 loss 함수와 Penalty term 이 없기 때문에 피쳐 스케일링이 불필요해서 쉽게 모델을 만들어 볼 수 있습니다. 보통 모델의 최소성능을 파악하기 위해 먼저 만들어보는 모델입니다.
sklearn 의 ensemble(앙상블) 모델군에서 RandomForestClassifier 을 불러옵니다. 그 다음 정해진 하이퍼파라미터(hyperparameter)를 가진 객체를 하나 생성합니다. 여기서 어떤 하이퍼파라미터로 객체를 생성하는가에 따라 모델의 성능이 결정되므로, 하이퍼파라미터 튜닝이라 절차가 필요합니다. 보통 최적의 하이퍼파라미터는 Grid Search 로 찾습니다. 그리고 fit 를 이용해서 데이터를 적용하면 됩니다. 예측값 생성은 predict 함수나 predict\_proba 함수로 할 수 있는데요. predict 함수는 0/1 값을 리턴하고, predict\_proba 함수는 ‘0 일 확률’/’1 일 확률’을 리턴합니다. 각 피처의 중요도를 그래프로 파악해보겠습니다. 이전 모델들에 비해 예측성능이 좋습니다.

\begin{sphinxuseclass}{cell}\begin{sphinxVerbatimInput}

\begin{sphinxuseclass}{cell_input}
\begin{sphinxVerbatim}[commandchars=\\\{\}]
\PYG{k+kn}{from} \PYG{n+nn}{sklearn}\PYG{n+nn}{.}\PYG{n+nn}{ensemble} \PYG{k+kn}{import} \PYG{n}{RandomForestClassifier}
\PYG{n}{rf} \PYG{o}{=} \PYG{n}{RandomForestClassifier}\PYG{p}{(}\PYG{n}{max\PYGZus{}depth}\PYG{o}{=}\PYG{l+m+mi}{4}\PYG{p}{,} \PYG{n}{min\PYGZus{}samples\PYGZus{}leaf}\PYG{o}{=}\PYG{l+m+mi}{30}\PYG{p}{)} \PYG{c+c1}{\PYGZsh{} 정해진 하이퍼파라미터를 가진 객체를 생성}

\PYG{n}{feature\PYGZus{}list} \PYG{o}{=} \PYG{p}{[}\PYG{l+s+s1}{\PYGZsq{}}\PYG{l+s+s1}{price\PYGZus{}z}\PYG{l+s+s1}{\PYGZsq{}}\PYG{p}{,} \PYG{l+s+s1}{\PYGZsq{}}\PYG{l+s+s1}{volume\PYGZus{}z}\PYG{l+s+s1}{\PYGZsq{}}\PYG{p}{,} \PYG{l+s+s1}{\PYGZsq{}}\PYG{l+s+s1}{num\PYGZus{}high/close}\PYG{l+s+s1}{\PYGZsq{}}\PYG{p}{,} \PYG{l+s+s1}{\PYGZsq{}}\PYG{l+s+s1}{num\PYGZus{}long}\PYG{l+s+s1}{\PYGZsq{}}\PYG{p}{,} \PYG{l+s+s1}{\PYGZsq{}}\PYG{l+s+s1}{num\PYGZus{}z\PYGZgt{}1.96}\PYG{l+s+s1}{\PYGZsq{}}\PYG{p}{,} \PYG{l+s+s1}{\PYGZsq{}}\PYG{l+s+s1}{num\PYGZus{}win\PYGZus{}market}\PYG{l+s+s1}{\PYGZsq{}}\PYG{p}{,} \PYG{l+s+s1}{\PYGZsq{}}\PYG{l+s+s1}{pct\PYGZus{}win\PYGZus{}market}\PYG{l+s+s1}{\PYGZsq{}}\PYG{p}{,} \PYG{l+s+s1}{\PYGZsq{}}\PYG{l+s+s1}{return over sector}\PYG{l+s+s1}{\PYGZsq{}}\PYG{p}{]}

\PYG{n}{X} \PYG{o}{=} \PYG{n}{train}\PYG{p}{[}\PYG{n}{feature\PYGZus{}list}\PYG{p}{]}
\PYG{n}{y} \PYG{o}{=} \PYG{n}{train}\PYG{p}{[}\PYG{l+s+s1}{\PYGZsq{}}\PYG{l+s+s1}{target}\PYG{l+s+s1}{\PYGZsq{}}\PYG{p}{]}
\PYG{n}{rf}\PYG{o}{.}\PYG{n}{fit}\PYG{p}{(}\PYG{n}{X}\PYG{p}{,} \PYG{n}{y}\PYG{p}{)}
\PYG{n}{yhat} \PYG{o}{=} \PYG{n}{rf}\PYG{o}{.}\PYG{n}{predict\PYGZus{}proba}\PYG{p}{(}\PYG{n}{X}\PYG{p}{)}\PYG{p}{[}\PYG{p}{:}\PYG{p}{,}\PYG{l+m+mi}{1}\PYG{p}{]} \PYG{c+c1}{\PYGZsh{} 첫번째 열은 0일 확률, 두번째 열은 1 일 확률 \PYGZhy{}\PYGZgt{} 1 일 확률을 저장}
\PYG{n}{yhat} \PYG{o}{=} \PYG{n}{pd}\PYG{o}{.}\PYG{n}{Series}\PYG{p}{(}\PYG{n}{yhat}\PYG{p}{,} \PYG{n}{name}\PYG{o}{=}\PYG{l+s+s1}{\PYGZsq{}}\PYG{l+s+s1}{yhat}\PYG{l+s+s1}{\PYGZsq{}}\PYG{p}{,} \PYG{n}{index}\PYG{o}{=}\PYG{n}{y}\PYG{o}{.}\PYG{n}{index}\PYG{p}{)} 

\PYG{n}{X\PYGZus{}test} \PYG{o}{=} \PYG{n}{test}\PYG{p}{[}\PYG{n}{feature\PYGZus{}list}\PYG{p}{]}
\PYG{n}{y\PYGZus{}test} \PYG{o}{=} \PYG{n}{test}\PYG{p}{[}\PYG{l+s+s1}{\PYGZsq{}}\PYG{l+s+s1}{target}\PYG{l+s+s1}{\PYGZsq{}}\PYG{p}{]}
\PYG{n}{yhat\PYGZus{}test} \PYG{o}{=} \PYG{n}{rf}\PYG{o}{.}\PYG{n}{predict\PYGZus{}proba}\PYG{p}{(}\PYG{n}{X\PYGZus{}test}\PYG{p}{)}\PYG{p}{[}\PYG{p}{:}\PYG{p}{,}\PYG{l+m+mi}{1}\PYG{p}{]} \PYG{c+c1}{\PYGZsh{} 첫번째 열은 0일 확률, 두번째 열은 1 일 확률 \PYGZhy{}\PYGZgt{} 1 일 확률을 저장}
\PYG{n}{yhat\PYGZus{}test} \PYG{o}{=} \PYG{n}{pd}\PYG{o}{.}\PYG{n}{Series}\PYG{p}{(}\PYG{n}{yhat\PYGZus{}test}\PYG{p}{,} \PYG{n}{name}\PYG{o}{=}\PYG{l+s+s1}{\PYGZsq{}}\PYG{l+s+s1}{yhat}\PYG{l+s+s1}{\PYGZsq{}}\PYG{p}{,} \PYG{n}{index}\PYG{o}{=}\PYG{n}{y\PYGZus{}test}\PYG{o}{.}\PYG{n}{index}\PYG{p}{)} 

\PYG{n}{importances} \PYG{o}{=} \PYG{n}{rf}\PYG{o}{.}\PYG{n}{feature\PYGZus{}importances\PYGZus{}}
\PYG{n}{sorted\PYGZus{}indices} \PYG{o}{=} \PYG{n}{np}\PYG{o}{.}\PYG{n}{argsort}\PYG{p}{(}\PYG{n}{importances}\PYG{p}{)}\PYG{p}{[}\PYG{p}{:}\PYG{p}{:}\PYG{o}{\PYGZhy{}}\PYG{l+m+mi}{1}\PYG{p}{]}

\PYG{k+kn}{import} \PYG{n+nn}{matplotlib}\PYG{n+nn}{.}\PYG{n+nn}{pyplot} \PYG{k}{as} \PYG{n+nn}{plt}
 
\PYG{n}{plt}\PYG{o}{.}\PYG{n}{title}\PYG{p}{(}\PYG{l+s+s1}{\PYGZsq{}}\PYG{l+s+s1}{Feature Importance}\PYG{l+s+s1}{\PYGZsq{}}\PYG{p}{)}
\PYG{n}{plt}\PYG{o}{.}\PYG{n}{bar}\PYG{p}{(}\PYG{n+nb}{range}\PYG{p}{(}\PYG{n}{X}\PYG{o}{.}\PYG{n}{shape}\PYG{p}{[}\PYG{l+m+mi}{1}\PYG{p}{]}\PYG{p}{)}\PYG{p}{,} \PYG{n}{importances}\PYG{p}{[}\PYG{n}{sorted\PYGZus{}indices}\PYG{p}{]}\PYG{p}{,} \PYG{n}{align}\PYG{o}{=}\PYG{l+s+s1}{\PYGZsq{}}\PYG{l+s+s1}{center}\PYG{l+s+s1}{\PYGZsq{}}\PYG{p}{)}
\PYG{n}{plt}\PYG{o}{.}\PYG{n}{xticks}\PYG{p}{(}\PYG{n+nb}{range}\PYG{p}{(}\PYG{n}{X}\PYG{o}{.}\PYG{n}{shape}\PYG{p}{[}\PYG{l+m+mi}{1}\PYG{p}{]}\PYG{p}{)}\PYG{p}{,} \PYG{n}{X}\PYG{o}{.}\PYG{n}{columns}\PYG{p}{[}\PYG{n}{sorted\PYGZus{}indices}\PYG{p}{]}\PYG{p}{,} \PYG{n}{rotation}\PYG{o}{=}\PYG{l+m+mi}{90}\PYG{p}{)}
\PYG{n}{plt}\PYG{o}{.}\PYG{n}{show}\PYG{p}{(}\PYG{p}{)}
\end{sphinxVerbatim}

\end{sphinxuseclass}\end{sphinxVerbatimInput}
\begin{sphinxVerbatimOutput}

\begin{sphinxuseclass}{cell_output}
\noindent\sphinxincludegraphics{{5.2.2_Modeling_Library_12_0}.png}

\end{sphinxuseclass}\end{sphinxVerbatimOutput}

\end{sphinxuseclass}
\begin{sphinxuseclass}{cell}\begin{sphinxVerbatimInput}

\begin{sphinxuseclass}{cell_input}
\begin{sphinxVerbatim}[commandchars=\\\{\}]
\PYG{n}{perf}\PYG{p}{(}\PYG{n}{y}\PYG{p}{,} \PYG{n}{yhat}\PYG{p}{)}
\PYG{n}{perf}\PYG{p}{(}\PYG{n}{y\PYGZus{}test}\PYG{p}{,} \PYG{n}{yhat\PYGZus{}test}\PYG{p}{)}
\PYG{n}{plt}\PYG{o}{.}\PYG{n}{show}\PYG{p}{(}\PYG{p}{)}
\end{sphinxVerbatim}

\end{sphinxuseclass}\end{sphinxVerbatimInput}
\begin{sphinxVerbatimOutput}

\begin{sphinxuseclass}{cell_output}
\begin{sphinxVerbatim}[commandchars=\\\{\}]
                count  mean
yhat                       
(0.167, 0.203]   1000 0.131
(0.203, 0.212]   1000 0.152
(0.212, 0.218]   1000 0.182
(0.218, 0.226]   1000 0.209
(0.226, 0.233]   1000 0.221
(0.233, 0.242]   1000 0.241
(0.242, 0.255]   1000 0.263
(0.255, 0.274]   1000 0.295
(0.274, 0.304]   1000 0.313
(0.304, 0.469]   1000 0.435
                count  mean
yhat                       
(0.164, 0.203]  31932 0.155
(0.203, 0.212]  31930 0.180
(0.212, 0.218]  31930 0.187
(0.218, 0.225]  31931 0.204
(0.225, 0.233]  31931 0.225
(0.233, 0.241]  31931 0.238
(0.241, 0.254]  31930 0.268
(0.254, 0.274]  31930 0.287
(0.274, 0.304]  31931 0.308
(0.304, 0.487]  31931 0.378
\end{sphinxVerbatim}

\noindent\sphinxincludegraphics{{5.2.2_Modeling_Library_13_1}.png}

\end{sphinxuseclass}\end{sphinxVerbatimOutput}

\end{sphinxuseclass}
\begin{sphinxuseclass}{cell}\begin{sphinxVerbatimInput}

\begin{sphinxuseclass}{cell_input}
\begin{sphinxVerbatim}[commandchars=\\\{\}]
\PYG{k+kn}{import} \PYG{n+nn}{FinanceDataReader} \PYG{k}{as} \PYG{n+nn}{fdr}
\PYG{o}{\PYGZpc{}}\PYG{k}{matplotlib} inline
\PYG{k+kn}{import} \PYG{n+nn}{matplotlib}\PYG{n+nn}{.}\PYG{n+nn}{pyplot} \PYG{k}{as} \PYG{n+nn}{plt}
\PYG{k+kn}{import} \PYG{n+nn}{pandas} \PYG{k}{as} \PYG{n+nn}{pd}
\PYG{k+kn}{import} \PYG{n+nn}{numpy} \PYG{k}{as} \PYG{n+nn}{np}
\PYG{k+kn}{import} \PYG{n+nn}{warnings}
\PYG{k+kn}{import} \PYG{n+nn}{pickle}
\PYG{n}{warnings}\PYG{o}{.}\PYG{n}{filterwarnings}\PYG{p}{(}\PYG{l+s+s1}{\PYGZsq{}}\PYG{l+s+s1}{ignore}\PYG{l+s+s1}{\PYGZsq{}}\PYG{p}{)}
\PYG{n}{pd}\PYG{o}{.}\PYG{n}{options}\PYG{o}{.}\PYG{n}{display}\PYG{o}{.}\PYG{n}{float\PYGZus{}format} \PYG{o}{=} \PYG{l+s+s1}{\PYGZsq{}}\PYG{l+s+si}{\PYGZob{}:,.3f\PYGZcb{}}\PYG{l+s+s1}{\PYGZsq{}}\PYG{o}{.}\PYG{n}{format}
\end{sphinxVerbatim}

\end{sphinxuseclass}\end{sphinxVerbatimInput}

\end{sphinxuseclass}

\section{종목 선정 모델 개발}
\label{\detokenize{chapter5/5.2.3_GAM:id1}}\label{\detokenize{chapter5/5.2.3_GAM::doc}}
\sphinxAtStartPar
선형모델에 대한 중요한 가정과 설명은 다음 절에서 추가로 설명드리겠으나,  수익률에 따라 단조 증가나 감소의 형태를 보이지 않는 피쳐(설명변수)는 변형을 해야 선형모형에서 더 유의미하게 사용될 수 있습니다.  주로 Binning (오름차순으로 정렬 후, 여러개 구간으로 분리) 을 통하여 이런 비선형적인 관계를 선형적으로 변경합니다. 2차 함수나 로그함수 등을 이용해 선형적으로 변경할 수 도 있습니다. 우리는 앞서 수익율과 피쳐사이에 선형적인 관계를 가지지 않는 가설(예: 섹터의 평균 수익률 대비 종목 수익률)들 이 있었습니다. 이런 피처들에 대하여 Binning 없이 적합할 수 있는 모델이 일반화가법모형(Generalized Additive Model) 입니다. 또한 가설 검정에서는 5 영업일 동안의 최대 수익률을 예측변수로 이용했으나, 모델의 overfitting (과대적합) 문제를 최소화하기 위하여, 예측값을 이진값(0/1)으로 치환한 후, 로지스틱 일반화가법모형(Logistic Generalized Additive Model) 을 구현합니다. 로지스틱 회귀모형은 \(log(odds) = a0 + a1*x1 + a2*x2 …\)  으로 표현할 수 있는데요. 여기서 X 를 여러개의 spline 로 함수로 만든 후, 다시 합하여 X 와 \(log(odds)\) 의  비선형적관계를 표현할 수 있도록 한 것이  Logistic GAM 입니다.  이 모델의 구현은 Statsmodels 에서 가능합니다만, pyGAM 패키지는 자동으로 하이퍼파라미터를 찾는 기능이 있어 편리합니다. GAM 을 선택한 다른 이유는 피처사이에 상호작용이 크지 않을 것이라는 가정이 있습니다. 무엇보다도 좋은 점은 모델이 왜 이 종목을 선택했는지에 대한 해석이 가능합니다. 향후, 모델의 예측력이 저하되는 경우 어떤 피처가 원인인지도 파악이 가능합니다.

\sphinxAtStartPar
단순히, 스코어가 높은 모든 종목을 매수하는 것이 아니라,  오늘의 종가 수익률과 주가를 고려하여 기본적인 필터링을 합니다. 분석결과 종가 수익률은 높고, 최근 20일 대비 가격이 낮은 종목을 매수하면 리스크가 적은 것으로 판단됩니다.

\sphinxAtStartPar
이번절에서는 책에서 종목선정을 위해 사용할 GAM 모델을 개발하겠습니다. 아나콘다 프롬프트에서 conda install \sphinxhyphen{}c conda\sphinxhyphen{}forge pygam 로 설치를 해 줍니다. 관련 링크 \sphinxurl{https://anaconda.org/conda-forge/pygam}

\sphinxAtStartPar
모델링을 위해 준비한 데이터를 읽습니다. 그리고 모델의 오버피팅을 최소화하기 위하여 타겟변수를 0 과 1 로 치환합니다. 5\% 익절은 다음과 같이 데이터로 표현할 수 있습니다. \sphinxhyphen{} ‘max\_close’ 가 5\% 이상일 때 1, 아니면 0. 파이썬 코드는 아래와 같습니다.

\begin{sphinxVerbatim}[commandchars=\\\{\}]
\PYG{n}{np}\PYG{o}{.}\PYG{n}{where}\PYG{p}{(}\PYG{n}{feature\PYGZus{}all}\PYG{p}{[}\PYG{l+s+s1}{\PYGZsq{}}\PYG{l+s+s1}{max\PYGZus{}close}\PYG{l+s+s1}{\PYGZsq{}}\PYG{p}{]}\PYG{o}{\PYGZgt{}}\PYG{o}{=} \PYG{l+m+mf}{1.05}\PYG{p}{,} \PYG{l+m+mi}{1}\PYG{p}{,} \PYG{l+m+mi}{0}\PYG{p}{)}
\end{sphinxVerbatim}

\sphinxAtStartPar
타겟 변수 \sphinxhyphen{} ‘target’ 값이 1 인 비율을 보니, 약 24\% 입니다. 타겟변수의 비율이 너무 적으면 모델 트레이닝이 어렵습니다.

\begin{sphinxuseclass}{cell}\begin{sphinxVerbatimInput}

\begin{sphinxuseclass}{cell_input}
\begin{sphinxVerbatim}[commandchars=\\\{\}]
\PYG{n}{feature\PYGZus{}all} \PYG{o}{=} \PYG{n}{pd}\PYG{o}{.}\PYG{n}{read\PYGZus{}pickle}\PYG{p}{(}\PYG{l+s+s1}{\PYGZsq{}}\PYG{l+s+s1}{feature\PYGZus{}all.pkl}\PYG{l+s+s1}{\PYGZsq{}}\PYG{p}{)} 
\PYG{n}{feature\PYGZus{}all}\PYG{p}{[}\PYG{l+s+s1}{\PYGZsq{}}\PYG{l+s+s1}{target}\PYG{l+s+s1}{\PYGZsq{}}\PYG{p}{]} \PYG{o}{=} \PYG{n}{np}\PYG{o}{.}\PYG{n}{where}\PYG{p}{(}\PYG{n}{feature\PYGZus{}all}\PYG{p}{[}\PYG{l+s+s1}{\PYGZsq{}}\PYG{l+s+s1}{max\PYGZus{}close}\PYG{l+s+s1}{\PYGZsq{}}\PYG{p}{]}\PYG{o}{\PYGZgt{}}\PYG{o}{=} \PYG{l+m+mf}{1.05}\PYG{p}{,} \PYG{l+m+mi}{1}\PYG{p}{,} \PYG{l+m+mi}{0}\PYG{p}{)}
\PYG{n}{target} \PYG{o}{=} \PYG{n}{feature\PYGZus{}all}\PYG{p}{[}\PYG{l+s+s1}{\PYGZsq{}}\PYG{l+s+s1}{target}\PYG{l+s+s1}{\PYGZsq{}}\PYG{p}{]}\PYG{o}{.}\PYG{n}{mean}\PYG{p}{(}\PYG{p}{)}
\PYG{n+nb}{print}\PYG{p}{(}\PYG{l+s+sa}{f}\PYG{l+s+s1}{\PYGZsq{}}\PYG{l+s+s1}{\PYGZpc{} of target:}\PYG{l+s+si}{\PYGZob{}}\PYG{n}{target}\PYG{l+s+si}{:}\PYG{l+s+s1}{ 5.1\PYGZpc{}}\PYG{l+s+si}{\PYGZcb{}}\PYG{l+s+s1}{\PYGZsq{}}\PYG{p}{)}
\end{sphinxVerbatim}

\end{sphinxuseclass}\end{sphinxVerbatimInput}
\begin{sphinxVerbatimOutput}

\begin{sphinxuseclass}{cell_output}
\begin{sphinxVerbatim}[commandchars=\\\{\}]
\PYGZpc{} of target: 24.3\PYGZpc{}
\end{sphinxVerbatim}

\end{sphinxuseclass}\end{sphinxVerbatimOutput}

\end{sphinxuseclass}
\sphinxAtStartPar
 날짜와 종목은 모델의 입력피처가 아닙니다. 편의를 위해 제거하거나 인덱스로 처리합니다. 모델 트레이닝 용도로 10,000 개 샘플을 뽑아 예측모델을 만들고, 나머지 데이터는 테스트(혹은 백테스팅)를 하겠습니다.

\begin{sphinxuseclass}{cell}\begin{sphinxVerbatimInput}

\begin{sphinxuseclass}{cell_input}
\begin{sphinxVerbatim}[commandchars=\\\{\}]
\PYG{n}{mdl\PYGZus{}all} \PYG{o}{=} \PYG{n}{feature\PYGZus{}all}\PYG{o}{.}\PYG{n}{set\PYGZus{}index}\PYG{p}{(}\PYG{p}{[}\PYG{n}{feature\PYGZus{}all}\PYG{o}{.}\PYG{n}{index}\PYG{p}{,}\PYG{l+s+s1}{\PYGZsq{}}\PYG{l+s+s1}{code}\PYG{l+s+s1}{\PYGZsq{}}\PYG{p}{]}\PYG{p}{)}

\PYG{n}{train} \PYG{o}{=} \PYG{n}{mdl\PYGZus{}all}\PYG{o}{.}\PYG{n}{sample}\PYG{p}{(}\PYG{l+m+mi}{10000}\PYG{p}{,} \PYG{n}{random\PYGZus{}state}\PYG{o}{=}\PYG{l+m+mi}{124}\PYG{p}{)}
\PYG{n}{test} \PYG{o}{=} \PYG{n}{mdl\PYGZus{}all}\PYG{o}{.}\PYG{n}{loc}\PYG{p}{[}\PYG{o}{\PYGZti{}}\PYG{n}{mdl\PYGZus{}all}\PYG{o}{.}\PYG{n}{index}\PYG{o}{.}\PYG{n}{isin}\PYG{p}{(}\PYG{n}{train}\PYG{o}{.}\PYG{n}{index}\PYG{p}{)}\PYG{p}{]}
\PYG{n+nb}{print}\PYG{p}{(}\PYG{n+nb}{len}\PYG{p}{(}\PYG{n}{train}\PYG{p}{)}\PYG{p}{,} \PYG{n+nb}{len}\PYG{p}{(}\PYG{n}{test}\PYG{p}{)}\PYG{p}{)}
\end{sphinxVerbatim}

\end{sphinxuseclass}\end{sphinxVerbatimInput}
\begin{sphinxVerbatimOutput}

\begin{sphinxuseclass}{cell_output}
\begin{sphinxVerbatim}[commandchars=\\\{\}]
10000 319307
\end{sphinxVerbatim}

\end{sphinxuseclass}\end{sphinxVerbatimOutput}

\end{sphinxuseclass}
\sphinxAtStartPar
입력 피처의 갯수와 데이터타입을 확인합니다.

\begin{sphinxuseclass}{cell}\begin{sphinxVerbatimInput}

\begin{sphinxuseclass}{cell_input}
\begin{sphinxVerbatim}[commandchars=\\\{\}]
\PYG{n}{train}\PYG{o}{.}\PYG{n}{info}\PYG{p}{(}\PYG{p}{)}
\end{sphinxVerbatim}

\end{sphinxuseclass}\end{sphinxVerbatimInput}
\begin{sphinxVerbatimOutput}

\begin{sphinxuseclass}{cell_output}
\begin{sphinxVerbatim}[commandchars=\\\{\}]
\PYGZlt{}class \PYGZsq{}pandas.core.frame.DataFrame\PYGZsq{}\PYGZgt{}
MultiIndex: 10000 entries, (\PYGZsq{}2021\PYGZhy{}10\PYGZhy{}22\PYGZsq{}, \PYGZsq{}312610\PYGZsq{}) to (\PYGZsq{}2021\PYGZhy{}04\PYGZhy{}28\PYGZsq{}, \PYGZsq{}011320\PYGZsq{})
Data columns (total 15 columns):
 \PYGZsh{}   Column              Non\PYGZhy{}Null Count  Dtype  
\PYGZhy{}\PYGZhy{}\PYGZhy{}  \PYGZhy{}\PYGZhy{}\PYGZhy{}\PYGZhy{}\PYGZhy{}\PYGZhy{}              \PYGZhy{}\PYGZhy{}\PYGZhy{}\PYGZhy{}\PYGZhy{}\PYGZhy{}\PYGZhy{}\PYGZhy{}\PYGZhy{}\PYGZhy{}\PYGZhy{}\PYGZhy{}\PYGZhy{}\PYGZhy{}  \PYGZhy{}\PYGZhy{}\PYGZhy{}\PYGZhy{}\PYGZhy{}  
 0   sector              10000 non\PYGZhy{}null  object 
 1   return              10000 non\PYGZhy{}null  float64
 2   kosdaq\PYGZus{}return       10000 non\PYGZhy{}null  float64
 3   price\PYGZus{}z             10000 non\PYGZhy{}null  float64
 4   volume\PYGZus{}z            10000 non\PYGZhy{}null  float64
 5   num\PYGZus{}high/close      10000 non\PYGZhy{}null  float64
 6   num\PYGZus{}long            10000 non\PYGZhy{}null  float64
 7   num\PYGZus{}z\PYGZgt{}1.96          10000 non\PYGZhy{}null  float64
 8   num\PYGZus{}win\PYGZus{}market      10000 non\PYGZhy{}null  float64
 9   pct\PYGZus{}win\PYGZus{}market      10000 non\PYGZhy{}null  float64
 10  return over sector  10000 non\PYGZhy{}null  float64
 11  max\PYGZus{}close           10000 non\PYGZhy{}null  float64
 12  mean\PYGZus{}close          10000 non\PYGZhy{}null  float64
 13  min\PYGZus{}close           10000 non\PYGZhy{}null  float64
 14  target              10000 non\PYGZhy{}null  int32  
dtypes: float64(13), int32(1), object(1)
memory usage: 1.2+ MB
\end{sphinxVerbatim}

\end{sphinxuseclass}\end{sphinxVerbatimOutput}

\end{sphinxuseclass}
\sphinxAtStartPar
 각 변수별로 다른 ‘lambda’ (Wiggliness Penalty Weight) 을 적용해서 grid Search 를 합니다. spline 수는 20 이 default 값입니다. spline 수는 고정하고 lambda의 최적 조합을 찾거나, lambda 를 고정하고, spline 수의 최적 조합을 찾는 것이 현실적이고, 두 하이퍼파라미터를 동시에 조합하여 grid Search 하는 것은 시간이 많이 걸립니다. 다양한 시도를 통하여 더 좋은 모델을 구현할 수 있겠으나, 이 책에서는 grid search 로 변수별 최적의 lambda 를 찾는 것으로 모델을 완성합니다. P value 가 크게 나타나는 입력변수는 제거하는 것이 좋겠습니다.

\begin{sphinxuseclass}{cell}\begin{sphinxVerbatimInput}

\begin{sphinxuseclass}{cell_input}
\begin{sphinxVerbatim}[commandchars=\\\{\}]
\PYG{k+kn}{from} \PYG{n+nn}{pygam} \PYG{k+kn}{import} \PYG{n}{LogisticGAM}\PYG{p}{,} \PYG{n}{s}\PYG{p}{,} \PYG{n}{f}\PYG{p}{,} \PYG{n}{te}\PYG{p}{,} \PYG{n}{l}
\PYG{k+kn}{from} \PYG{n+nn}{sklearn}\PYG{n+nn}{.}\PYG{n+nn}{metrics} \PYG{k+kn}{import} \PYG{n}{accuracy\PYGZus{}score}
\PYG{k+kn}{from} \PYG{n+nn}{sklearn}\PYG{n+nn}{.}\PYG{n+nn}{metrics} \PYG{k+kn}{import} \PYG{n}{log\PYGZus{}loss}

\PYG{n}{feature\PYGZus{}list} \PYG{o}{=} \PYG{p}{[}\PYG{l+s+s1}{\PYGZsq{}}\PYG{l+s+s1}{price\PYGZus{}z}\PYG{l+s+s1}{\PYGZsq{}}\PYG{p}{,}\PYG{l+s+s1}{\PYGZsq{}}\PYG{l+s+s1}{volume\PYGZus{}z}\PYG{l+s+s1}{\PYGZsq{}}\PYG{p}{,}\PYG{l+s+s1}{\PYGZsq{}}\PYG{l+s+s1}{num\PYGZus{}high/close}\PYG{l+s+s1}{\PYGZsq{}}\PYG{p}{,}\PYG{l+s+s1}{\PYGZsq{}}\PYG{l+s+s1}{num\PYGZus{}win\PYGZus{}market}\PYG{l+s+s1}{\PYGZsq{}}\PYG{p}{,}\PYG{l+s+s1}{\PYGZsq{}}\PYG{l+s+s1}{pct\PYGZus{}win\PYGZus{}market}\PYG{l+s+s1}{\PYGZsq{}}\PYG{p}{,}\PYG{l+s+s1}{\PYGZsq{}}\PYG{l+s+s1}{return over sector}\PYG{l+s+s1}{\PYGZsq{}}\PYG{p}{]}
\PYG{n}{X} \PYG{o}{=} \PYG{n}{train}\PYG{p}{[}\PYG{n}{feature\PYGZus{}list}\PYG{p}{]}
\PYG{n}{y} \PYG{o}{=} \PYG{n}{train}\PYG{p}{[}\PYG{l+s+s1}{\PYGZsq{}}\PYG{l+s+s1}{target}\PYG{l+s+s1}{\PYGZsq{}}\PYG{p}{]}
\PYG{n}{X\PYGZus{}test} \PYG{o}{=} \PYG{n}{test}\PYG{p}{[}\PYG{n}{feature\PYGZus{}list}\PYG{p}{]}
\PYG{n}{y\PYGZus{}test} \PYG{o}{=} \PYG{n}{test}\PYG{p}{[}\PYG{l+s+s1}{\PYGZsq{}}\PYG{l+s+s1}{target}\PYG{l+s+s1}{\PYGZsq{}}\PYG{p}{]}

\PYG{c+c1}{\PYGZsh{} 하이퍼파라미터 설정 N 개의 변수면 (M x N) 개의 리스트로 생성함으로써 변수별로 다른 하이퍼파라미터 테스트 가능. }
\PYG{c+c1}{\PYGZsh{} M 개만 1D 리스트를 만들면 동일한 lambda 른 모든 변수에 적용함.}
\PYG{n}{lam\PYGZus{}list} \PYG{o}{=} \PYG{p}{[}\PYG{n}{np}\PYG{o}{.}\PYG{n}{logspace}\PYG{p}{(}\PYG{l+m+mi}{0}\PYG{p}{,} \PYG{l+m+mi}{3}\PYG{p}{,} \PYG{l+m+mi}{2}\PYG{p}{)}\PYG{p}{]}\PYG{o}{*}\PYG{l+m+mi}{8}
   
\PYG{n}{gam} \PYG{o}{=} \PYG{n}{LogisticGAM}\PYG{p}{(}\PYG{n}{te}\PYG{p}{(}\PYG{l+m+mi}{0}\PYG{p}{,} \PYG{l+m+mi}{1}\PYG{p}{,} \PYG{n}{n\PYGZus{}splines}\PYG{o}{=}\PYG{l+m+mi}{5}\PYG{p}{)} \PYG{o}{+} \PYG{n}{s}\PYG{p}{(}\PYG{l+m+mi}{1}\PYG{p}{)} \PYG{o}{+} \PYG{n}{s}\PYG{p}{(}\PYG{l+m+mi}{2}\PYG{p}{)} \PYG{o}{+} \PYG{n}{s}\PYG{p}{(}\PYG{l+m+mi}{3}\PYG{p}{)} \PYG{o}{+} \PYG{n}{s}\PYG{p}{(}\PYG{l+m+mi}{4}\PYG{p}{)} \PYG{o}{+} \PYG{n}{te}\PYG{p}{(}\PYG{l+m+mi}{4}\PYG{p}{,} \PYG{l+m+mi}{5}\PYG{p}{,} \PYG{n}{n\PYGZus{}splines}\PYG{o}{=}\PYG{l+m+mi}{5}\PYG{p}{)}\PYG{p}{)}\PYG{o}{.}\PYG{n}{gridsearch}\PYG{p}{(}\PYG{n}{X}\PYG{o}{.}\PYG{n}{to\PYGZus{}numpy}\PYG{p}{(}\PYG{p}{)}\PYG{p}{,} \PYG{n}{y}\PYG{o}{.}\PYG{n}{to\PYGZus{}numpy}\PYG{p}{(}\PYG{p}{)}\PYG{p}{,} \PYG{n}{lam}\PYG{o}{=}\PYG{n}{lam\PYGZus{}list}\PYG{p}{)} 

\PYG{n+nb}{print}\PYG{p}{(}\PYG{n}{gam}\PYG{o}{.}\PYG{n}{summary}\PYG{p}{(}\PYG{p}{)}\PYG{p}{)}
\PYG{n+nb}{print}\PYG{p}{(}\PYG{n}{gam}\PYG{o}{.}\PYG{n}{accuracy}\PYG{p}{(}\PYG{n}{X\PYGZus{}test}\PYG{p}{,} \PYG{n}{y\PYGZus{}test}\PYG{p}{)}\PYG{p}{)}
\end{sphinxVerbatim}

\end{sphinxuseclass}\end{sphinxVerbatimInput}
\begin{sphinxVerbatimOutput}

\begin{sphinxuseclass}{cell_output}
\begin{sphinxVerbatim}[commandchars=\\\{\}]
100\PYGZpc{} (256 of 256) |\PYGZsh{}\PYGZsh{}\PYGZsh{}\PYGZsh{}\PYGZsh{}\PYGZsh{}\PYGZsh{}\PYGZsh{}\PYGZsh{}\PYGZsh{}\PYGZsh{}\PYGZsh{}\PYGZsh{}\PYGZsh{}\PYGZsh{}\PYGZsh{}\PYGZsh{}\PYGZsh{}\PYGZsh{}\PYGZsh{}\PYGZsh{}\PYGZsh{}| Elapsed Time: 0:03:32 Time:  0:03:32
\end{sphinxVerbatim}

\begin{sphinxVerbatim}[commandchars=\\\{\}]
LogisticGAM                                                                                               
=============================================== ==========================================================
Distribution:                      BinomialDist Effective DoF:                                     21.7127
Link Function:                        LogitLink Log Likelihood:                                 \PYGZhy{}5473.5193
Number of Samples:                        10000 AIC:                                            10990.4639
                                                AICc:                                           10990.5719
                                                UBRE:                                               3.1008
                                                Scale:                                                 1.0
                                                Pseudo R\PYGZhy{}Squared:                                   0.0197
==========================================================================================================
Feature Function                  Lambda               Rank         EDoF         P \PYGZgt{} x        Sig. Code   
================================= ==================== ============ ============ ============ ============
te(0, 1)                          [1. 1.]              25           6.2          1.11e\PYGZhy{}15     ***         
s(1)                              [1000.]              20           0.1          1.39e\PYGZhy{}04     ***         
s(2)                              [1000.]              20           1.8          1.65e\PYGZhy{}01                 
s(3)                              [1000.]              20           2.7          4.35e\PYGZhy{}02     *           
s(4)                              [1.]                 20           8.9          1.11e\PYGZhy{}11     ***         
te(4, 5)                          [1. 1.]              25           2.0          1.80e\PYGZhy{}01                 
intercept                                              1            0.0          2.07e\PYGZhy{}01                 
==========================================================================================================
Significance codes:  0 \PYGZsq{}***\PYGZsq{} 0.001 \PYGZsq{}**\PYGZsq{} 0.01 \PYGZsq{}*\PYGZsq{} 0.05 \PYGZsq{}.\PYGZsq{} 0.1 \PYGZsq{} \PYGZsq{} 1

WARNING: Fitting splines and a linear function to a feature introduces a model identifiability problem
         which can cause p\PYGZhy{}values to appear significant when they are not.

WARNING: p\PYGZhy{}values calculated in this manner behave correctly for un\PYGZhy{}penalized models or models with
         known smoothing parameters, but when smoothing parameters have been estimated, the p\PYGZhy{}values
         are typically lower than they should be, meaning that the tests reject the null too readily.
None
0.7571130530379218
\end{sphinxVerbatim}

\end{sphinxuseclass}\end{sphinxVerbatimOutput}

\end{sphinxuseclass}
\begin{sphinxuseclass}{cell}\begin{sphinxVerbatimInput}

\begin{sphinxuseclass}{cell_input}
\begin{sphinxVerbatim}[commandchars=\\\{\}]
\PYG{k}{for} \PYG{n}{i}\PYG{p}{,} \PYG{n}{term} \PYG{o+ow}{in} \PYG{n+nb}{enumerate}\PYG{p}{(}\PYG{n}{gam}\PYG{o}{.}\PYG{n}{terms}\PYG{p}{)}\PYG{p}{:}
    \PYG{n+nb}{print}\PYG{p}{(}\PYG{n}{i}\PYG{p}{,} \PYG{n}{term}\PYG{p}{)}
\end{sphinxVerbatim}

\end{sphinxuseclass}\end{sphinxVerbatimInput}
\begin{sphinxVerbatimOutput}

\begin{sphinxuseclass}{cell_output}
\begin{sphinxVerbatim}[commandchars=\\\{\}]
0 tensor\PYGZus{}term
1 spline\PYGZus{}term
2 spline\PYGZus{}term
3 spline\PYGZus{}term
4 spline\PYGZus{}term
5 tensor\PYGZus{}term
6 intercept\PYGZus{}term
\end{sphinxVerbatim}

\end{sphinxuseclass}\end{sphinxVerbatimOutput}

\end{sphinxuseclass}
\begin{sphinxuseclass}{cell}\begin{sphinxVerbatimInput}

\begin{sphinxuseclass}{cell_input}
\begin{sphinxVerbatim}[commandchars=\\\{\}]
\PYG{n}{feature\PYGZus{}list} \PYG{o}{=}  \PYG{p}{[}\PYG{l+s+s1}{\PYGZsq{}}\PYG{l+s+s1}{price\PYGZus{}z}\PYG{l+s+s1}{\PYGZsq{}}\PYG{p}{,}\PYG{l+s+s1}{\PYGZsq{}}\PYG{l+s+s1}{volume\PYGZus{}z}\PYG{l+s+s1}{\PYGZsq{}}\PYG{p}{,}\PYG{l+s+s1}{\PYGZsq{}}\PYG{l+s+s1}{num\PYGZus{}high/close}\PYG{l+s+s1}{\PYGZsq{}}\PYG{p}{,}\PYG{l+s+s1}{\PYGZsq{}}\PYG{l+s+s1}{num\PYGZus{}win\PYGZus{}market}\PYG{l+s+s1}{\PYGZsq{}}\PYG{p}{,}\PYG{l+s+s1}{\PYGZsq{}}\PYG{l+s+s1}{pct\PYGZus{}win\PYGZus{}market}\PYG{l+s+s1}{\PYGZsq{}}\PYG{p}{,}\PYG{l+s+s1}{\PYGZsq{}}\PYG{l+s+s1}{return over sector}\PYG{l+s+s1}{\PYGZsq{}}\PYG{p}{]}
\PYG{k}{for} \PYG{n}{i}\PYG{p}{,} \PYG{n}{term} \PYG{o+ow}{in} \PYG{n+nb}{enumerate}\PYG{p}{(}\PYG{n}{gam}\PYG{o}{.}\PYG{n}{terms}\PYG{p}{)}\PYG{p}{:}
    
    \PYG{k}{if} \PYG{n}{i}\PYG{o}{\PYGZgt{}}\PYG{o}{=}\PYG{l+m+mi}{1} \PYG{o+ow}{and} \PYG{n}{i}\PYG{o}{\PYGZlt{}}\PYG{o}{=}\PYG{l+m+mi}{4}\PYG{p}{:}

        \PYG{n}{XX} \PYG{o}{=} \PYG{n}{gam}\PYG{o}{.}\PYG{n}{generate\PYGZus{}X\PYGZus{}grid}\PYG{p}{(}\PYG{n}{term}\PYG{o}{=}\PYG{n}{i}\PYG{p}{)}
        \PYG{n}{pdep}\PYG{p}{,} \PYG{n}{confi} \PYG{o}{=} \PYG{n}{gam}\PYG{o}{.}\PYG{n}{partial\PYGZus{}dependence}\PYG{p}{(}\PYG{n}{term}\PYG{o}{=}\PYG{n}{i}\PYG{p}{,} \PYG{n}{X}\PYG{o}{=}\PYG{n}{XX}\PYG{p}{,} \PYG{n}{width}\PYG{o}{=}\PYG{l+m+mf}{0.95}\PYG{p}{)}

        \PYG{n}{plt}\PYG{o}{.}\PYG{n}{figure}\PYG{p}{(}\PYG{p}{)}
        \PYG{n}{plt}\PYG{o}{.}\PYG{n}{plot}\PYG{p}{(}\PYG{n}{XX}\PYG{p}{[}\PYG{p}{:}\PYG{p}{,} \PYG{n}{term}\PYG{o}{.}\PYG{n}{feature}\PYG{p}{]}\PYG{p}{,} \PYG{n}{pdep}\PYG{p}{)}
        \PYG{n}{plt}\PYG{o}{.}\PYG{n}{plot}\PYG{p}{(}\PYG{n}{XX}\PYG{p}{[}\PYG{p}{:}\PYG{p}{,} \PYG{n}{term}\PYG{o}{.}\PYG{n}{feature}\PYG{p}{]}\PYG{p}{,} \PYG{n}{confi}\PYG{p}{,} \PYG{n}{c}\PYG{o}{=}\PYG{l+s+s1}{\PYGZsq{}}\PYG{l+s+s1}{r}\PYG{l+s+s1}{\PYGZsq{}}\PYG{p}{,} \PYG{n}{ls}\PYG{o}{=}\PYG{l+s+s1}{\PYGZsq{}}\PYG{l+s+s1}{\PYGZhy{}\PYGZhy{}}\PYG{l+s+s1}{\PYGZsq{}}\PYG{p}{)}
        \PYG{n}{plt}\PYG{o}{.}\PYG{n}{title}\PYG{p}{(}\PYG{n}{feature\PYGZus{}list}\PYG{p}{[}\PYG{n}{i}\PYG{p}{]}\PYG{p}{)}
        \PYG{n}{plt}\PYG{o}{.}\PYG{n}{show}\PYG{p}{(}\PYG{p}{)}
\end{sphinxVerbatim}

\end{sphinxuseclass}\end{sphinxVerbatimInput}
\begin{sphinxVerbatimOutput}

\begin{sphinxuseclass}{cell_output}
\noindent\sphinxincludegraphics{{5.2.3_GAM_11_0}.png}

\noindent\sphinxincludegraphics{{5.2.3_GAM_11_1}.png}

\noindent\sphinxincludegraphics{{5.2.3_GAM_11_2}.png}

\noindent\sphinxincludegraphics{{5.2.3_GAM_11_3}.png}

\end{sphinxuseclass}\end{sphinxVerbatimOutput}

\end{sphinxuseclass}
\sphinxAtStartPar
 완성된 모델을 pickle 로 binary 파일로 저장합니다.

\begin{sphinxuseclass}{cell}\begin{sphinxVerbatimInput}

\begin{sphinxuseclass}{cell_input}
\begin{sphinxVerbatim}[commandchars=\\\{\}]
\PYG{k+kn}{import} \PYG{n+nn}{pickle}
\PYG{k}{with} \PYG{n+nb}{open}\PYG{p}{(}\PYG{l+s+s2}{\PYGZdq{}}\PYG{l+s+s2}{gam.pkl}\PYG{l+s+s2}{\PYGZdq{}}\PYG{p}{,} \PYG{l+s+s2}{\PYGZdq{}}\PYG{l+s+s2}{wb}\PYG{l+s+s2}{\PYGZdq{}}\PYG{p}{)} \PYG{k}{as} \PYG{n}{file}\PYG{p}{:}
    \PYG{n}{pickle}\PYG{o}{.}\PYG{n}{dump}\PYG{p}{(}\PYG{n}{gam}\PYG{p}{,} \PYG{n}{file}\PYG{p}{)}    
\end{sphinxVerbatim}

\end{sphinxuseclass}\end{sphinxVerbatimInput}

\end{sphinxuseclass}
\begin{sphinxuseclass}{cell}\begin{sphinxVerbatimInput}

\begin{sphinxuseclass}{cell_input}
\begin{sphinxVerbatim}[commandchars=\\\{\}]
\PYG{k}{with} \PYG{n+nb}{open}\PYG{p}{(}\PYG{l+s+s2}{\PYGZdq{}}\PYG{l+s+s2}{gam.pkl}\PYG{l+s+s2}{\PYGZdq{}}\PYG{p}{,} \PYG{l+s+s2}{\PYGZdq{}}\PYG{l+s+s2}{rb}\PYG{l+s+s2}{\PYGZdq{}}\PYG{p}{)} \PYG{k}{as} \PYG{n}{file}\PYG{p}{:}
    \PYG{n}{gam} \PYG{o}{=} \PYG{n}{pickle}\PYG{o}{.}\PYG{n}{load}\PYG{p}{(}\PYG{n}{file}\PYG{p}{)} 
\end{sphinxVerbatim}

\end{sphinxuseclass}\end{sphinxVerbatimInput}

\end{sphinxuseclass}
\begin{sphinxuseclass}{cell}\begin{sphinxVerbatimInput}

\begin{sphinxuseclass}{cell_input}
\begin{sphinxVerbatim}[commandchars=\\\{\}]
\PYG{n+nb}{print}\PYG{p}{(}\PYG{n}{gam}\PYG{o}{.}\PYG{n}{get\PYGZus{}params}\PYG{p}{(}\PYG{p}{)}\PYG{p}{)}
\PYG{n+nb}{print}\PYG{p}{(}\PYG{n}{gam}\PYG{o}{.}\PYG{n}{coef\PYGZus{}}\PYG{o}{.}\PYG{n}{shape}\PYG{p}{)}
\end{sphinxVerbatim}

\end{sphinxuseclass}\end{sphinxVerbatimInput}
\begin{sphinxVerbatimOutput}

\begin{sphinxuseclass}{cell_output}
\begin{sphinxVerbatim}[commandchars=\\\{\}]
\PYGZob{}\PYGZsq{}max\PYGZus{}iter\PYGZsq{}: 100, \PYGZsq{}tol\PYGZsq{}: 0.0001, \PYGZsq{}callbacks\PYGZsq{}: [Deviance(), Diffs(), Accuracy()], \PYGZsq{}verbose\PYGZsq{}: False, \PYGZsq{}terms\PYGZsq{}: te(0, 1) + s(1) + s(2) + s(3) + s(4) + te(4, 5) + intercept, \PYGZsq{}fit\PYGZus{}intercept\PYGZsq{}: True\PYGZcb{}
(131,)
\end{sphinxVerbatim}

\end{sphinxuseclass}\end{sphinxVerbatimOutput}

\end{sphinxuseclass}
\begin{sphinxuseclass}{cell}\begin{sphinxVerbatimInput}

\begin{sphinxuseclass}{cell_input}
\begin{sphinxVerbatim}[commandchars=\\\{\}]
\PYG{k}{for} \PYG{n}{i} \PYG{o+ow}{in} \PYG{n+nb}{range}\PYG{p}{(}\PYG{l+m+mi}{6}\PYG{p}{)}\PYG{p}{:}
    \PYG{n+nb}{print}\PYG{p}{(}\PYG{l+s+sa}{f}\PYG{l+s+s1}{\PYGZsq{}}\PYG{l+s+si}{\PYGZob{}}\PYG{n}{i}\PYG{l+s+si}{\PYGZcb{}}\PYG{l+s+s1}{: }\PYG{l+s+si}{\PYGZob{}}\PYG{n}{gam}\PYG{o}{.}\PYG{n}{\PYGZus{}compute\PYGZus{}p\PYGZus{}value}\PYG{p}{(}\PYG{n}{i}\PYG{p}{)}\PYG{l+s+si}{:}\PYG{l+s+s1}{ 5.3f}\PYG{l+s+si}{\PYGZcb{}}\PYG{l+s+s1}{ }\PYG{l+s+si}{\PYGZob{}}\PYG{n}{gam}\PYG{o}{.}\PYG{n}{generate\PYGZus{}X\PYGZus{}grid}\PYG{p}{(}\PYG{n}{term}\PYG{o}{=}\PYG{n}{i}\PYG{p}{)}\PYG{o}{.}\PYG{n}{shape}\PYG{l+s+si}{\PYGZcb{}}\PYG{l+s+s1}{\PYGZsq{}}\PYG{p}{)}
\end{sphinxVerbatim}

\end{sphinxuseclass}\end{sphinxVerbatimInput}
\begin{sphinxVerbatimOutput}

\begin{sphinxuseclass}{cell_output}
\begin{sphinxVerbatim}[commandchars=\\\{\}]
0:  0.000 (10000, 6)
1:  0.000 (100, 6)
2:  0.165 (100, 6)
3:  0.043 (100, 6)
4:  0.000 (100, 6)
5:  0.180 (10000, 6)
\end{sphinxVerbatim}

\end{sphinxuseclass}\end{sphinxVerbatimOutput}

\end{sphinxuseclass}
\sphinxAtStartPar
 간단하게 십분위수 분석을 하고, 성능을 평가합니다. 안정적인 모델을 만들었습니다. 이론적으로는 마지막 Decile(제 10 십분위 수)에서 랜덤하게 종목을 골라 동일한 금액으로 매수를 한다면, 5 영업일이내 5\% 익절할 확률이 37\% 가 됩니다.  100\% 만족스럽지는 않지만, 생성된 GAM 모델을 이용하여 종목 추천을 받도록 하겠습니다.

\begin{sphinxuseclass}{cell}\begin{sphinxVerbatimInput}

\begin{sphinxuseclass}{cell_input}
\begin{sphinxVerbatim}[commandchars=\\\{\}]
\PYG{n}{feature\PYGZus{}list} \PYG{o}{=} \PYG{p}{[}\PYG{l+s+s1}{\PYGZsq{}}\PYG{l+s+s1}{price\PYGZus{}z}\PYG{l+s+s1}{\PYGZsq{}}\PYG{p}{,}\PYG{l+s+s1}{\PYGZsq{}}\PYG{l+s+s1}{volume\PYGZus{}z}\PYG{l+s+s1}{\PYGZsq{}}\PYG{p}{,}\PYG{l+s+s1}{\PYGZsq{}}\PYG{l+s+s1}{num\PYGZus{}high/close}\PYG{l+s+s1}{\PYGZsq{}}\PYG{p}{,}\PYG{l+s+s1}{\PYGZsq{}}\PYG{l+s+s1}{num\PYGZus{}win\PYGZus{}market}\PYG{l+s+s1}{\PYGZsq{}}\PYG{p}{,}\PYG{l+s+s1}{\PYGZsq{}}\PYG{l+s+s1}{pct\PYGZus{}win\PYGZus{}market}\PYG{l+s+s1}{\PYGZsq{}}\PYG{p}{,}\PYG{l+s+s1}{\PYGZsq{}}\PYG{l+s+s1}{return over sector}\PYG{l+s+s1}{\PYGZsq{}}\PYG{p}{]}
\PYG{n}{X} \PYG{o}{=} \PYG{n}{train}\PYG{p}{[}\PYG{n}{feature\PYGZus{}list}\PYG{p}{]}
\PYG{n}{y} \PYG{o}{=} \PYG{n}{train}\PYG{p}{[}\PYG{l+s+s1}{\PYGZsq{}}\PYG{l+s+s1}{target}\PYG{l+s+s1}{\PYGZsq{}}\PYG{p}{]}
\PYG{n}{X\PYGZus{}test} \PYG{o}{=} \PYG{n}{test}\PYG{p}{[}\PYG{n}{feature\PYGZus{}list}\PYG{p}{]}
\PYG{n}{y\PYGZus{}test} \PYG{o}{=} \PYG{n}{test}\PYG{p}{[}\PYG{l+s+s1}{\PYGZsq{}}\PYG{l+s+s1}{target}\PYG{l+s+s1}{\PYGZsq{}}\PYG{p}{]}

\PYG{n}{yhat} \PYG{o}{=} \PYG{n}{gam}\PYG{o}{.}\PYG{n}{predict\PYGZus{}proba}\PYG{p}{(}\PYG{n}{X}\PYG{o}{.}\PYG{n}{to\PYGZus{}numpy}\PYG{p}{(}\PYG{p}{)}\PYG{p}{)}
\PYG{n}{yhat} \PYG{o}{=} \PYG{n}{pd}\PYG{o}{.}\PYG{n}{Series}\PYG{p}{(}\PYG{n}{yhat}\PYG{p}{,} \PYG{n}{name}\PYG{o}{=}\PYG{l+s+s1}{\PYGZsq{}}\PYG{l+s+s1}{yhat}\PYG{l+s+s1}{\PYGZsq{}}\PYG{p}{,} \PYG{n}{index}\PYG{o}{=}\PYG{n}{y}\PYG{o}{.}\PYG{n}{index}\PYG{p}{)}

\PYG{n}{yhat\PYGZus{}test} \PYG{o}{=} \PYG{n}{gam}\PYG{o}{.}\PYG{n}{predict\PYGZus{}proba}\PYG{p}{(}\PYG{n}{X\PYGZus{}test}\PYG{o}{.}\PYG{n}{to\PYGZus{}numpy}\PYG{p}{(}\PYG{p}{)}\PYG{p}{)}
\PYG{n}{yhat\PYGZus{}test} \PYG{o}{=} \PYG{n}{pd}\PYG{o}{.}\PYG{n}{Series}\PYG{p}{(}\PYG{n}{yhat\PYGZus{}test}\PYG{p}{,} \PYG{n}{name}\PYG{o}{=}\PYG{l+s+s1}{\PYGZsq{}}\PYG{l+s+s1}{yhat}\PYG{l+s+s1}{\PYGZsq{}}\PYG{p}{,} \PYG{n}{index}\PYG{o}{=}\PYG{n}{y\PYGZus{}test}\PYG{o}{.}\PYG{n}{index}\PYG{p}{)}
\end{sphinxVerbatim}

\end{sphinxuseclass}\end{sphinxVerbatimInput}

\end{sphinxuseclass}
\begin{sphinxuseclass}{cell}\begin{sphinxVerbatimInput}

\begin{sphinxuseclass}{cell_input}
\begin{sphinxVerbatim}[commandchars=\\\{\}]
\PYG{k}{def} \PYG{n+nf}{perf}\PYG{p}{(}\PYG{n}{y}\PYG{p}{,} \PYG{n}{yhat}\PYG{p}{)}\PYG{p}{:} \PYG{c+c1}{\PYGZsh{} Decile 분석 함수}
    \PYG{n}{combined} \PYG{o}{=} \PYG{n}{pd}\PYG{o}{.}\PYG{n}{concat}\PYG{p}{(}\PYG{p}{[}\PYG{n}{y}\PYG{p}{,} \PYG{n}{yhat}\PYG{p}{]}\PYG{p}{,} \PYG{n}{axis}\PYG{o}{=}\PYG{l+m+mi}{1}\PYG{p}{)}
    \PYG{n}{ranks} \PYG{o}{=} \PYG{n}{pd}\PYG{o}{.}\PYG{n}{qcut}\PYG{p}{(}\PYG{n}{combined}\PYG{p}{[}\PYG{l+s+s1}{\PYGZsq{}}\PYG{l+s+s1}{yhat}\PYG{l+s+s1}{\PYGZsq{}}\PYG{p}{]}\PYG{p}{,} \PYG{n}{q}\PYG{o}{=}\PYG{l+m+mi}{10}\PYG{p}{)}
    \PYG{n+nb}{print}\PYG{p}{(}\PYG{n}{combined}\PYG{o}{.}\PYG{n}{groupby}\PYG{p}{(}\PYG{n}{ranks}\PYG{p}{)}\PYG{p}{[}\PYG{l+s+s1}{\PYGZsq{}}\PYG{l+s+s1}{target}\PYG{l+s+s1}{\PYGZsq{}}\PYG{p}{]}\PYG{o}{.}\PYG{n}{agg}\PYG{p}{(}\PYG{p}{[}\PYG{l+s+s1}{\PYGZsq{}}\PYG{l+s+s1}{count}\PYG{l+s+s1}{\PYGZsq{}}\PYG{p}{,}\PYG{l+s+s1}{\PYGZsq{}}\PYG{l+s+s1}{mean}\PYG{l+s+s1}{\PYGZsq{}}\PYG{p}{]}\PYG{p}{)}\PYG{p}{)}
    \PYG{n}{combined}\PYG{o}{.}\PYG{n}{groupby}\PYG{p}{(}\PYG{n}{ranks}\PYG{p}{)}\PYG{p}{[}\PYG{l+s+s1}{\PYGZsq{}}\PYG{l+s+s1}{target}\PYG{l+s+s1}{\PYGZsq{}}\PYG{p}{]}\PYG{o}{.}\PYG{n}{mean}\PYG{p}{(}\PYG{p}{)}\PYG{o}{.}\PYG{n}{plot}\PYG{p}{(}\PYG{n}{figsize}\PYG{o}{=}\PYG{p}{(}\PYG{l+m+mi}{8}\PYG{p}{,}\PYG{l+m+mi}{5}\PYG{p}{)}\PYG{p}{)}

\PYG{n}{perf}\PYG{p}{(}\PYG{n}{y}\PYG{p}{,} \PYG{n}{yhat}\PYG{p}{)}
\PYG{n}{perf}\PYG{p}{(}\PYG{n}{y\PYGZus{}test}\PYG{p}{,} \PYG{n}{yhat\PYGZus{}test}\PYG{p}{)}
\end{sphinxVerbatim}

\end{sphinxuseclass}\end{sphinxVerbatimInput}
\begin{sphinxVerbatimOutput}

\begin{sphinxuseclass}{cell_output}
\begin{sphinxVerbatim}[commandchars=\\\{\}]
                count  mean
yhat                       
(0.121, 0.184]   1000 0.144
(0.184, 0.198]   1000 0.183
(0.198, 0.21]    1000 0.195
(0.21, 0.222]    1000 0.193
(0.222, 0.235]   1000 0.229
(0.235, 0.251]   1000 0.231
(0.251, 0.268]   1000 0.294
(0.268, 0.291]   1000 0.281
(0.291, 0.326]   1000 0.310
(0.326, 0.688]   1000 0.382
                 count  mean
yhat                        
(0.0342, 0.184]  31931 0.156
(0.184, 0.198]   31931 0.182
(0.198, 0.209]   31930 0.192
(0.209, 0.221]   31931 0.203
(0.221, 0.234]   31931 0.224
(0.234, 0.249]   31930 0.238
(0.249, 0.267]   31931 0.259
(0.267, 0.292]   31930 0.283
(0.292, 0.327]   31931 0.320
(0.327, 0.783]   31931 0.373
\end{sphinxVerbatim}

\noindent\sphinxincludegraphics{{5.2.3_GAM_19_1}.png}

\end{sphinxuseclass}\end{sphinxVerbatimOutput}

\end{sphinxuseclass}

\section{Basic Filtering}
\label{\detokenize{chapter5/5.2.3_GAM:basic-filtering}}
\sphinxAtStartPar
단순히 스코어가 높다고 무조건 매수했다가 큰 낙폭으로 손해를 볼 수도 있기 때문에 기본적인 필터링이 필요합니다. 오늘 종가 수익률과 가격 변동성으로 기본적인 필터를 만들어 보겠습니다.

\begin{sphinxuseclass}{cell}\begin{sphinxVerbatimInput}

\begin{sphinxuseclass}{cell_input}
\begin{sphinxVerbatim}[commandchars=\\\{\}]
\PYG{n}{test}\PYG{p}{[}\PYG{l+s+s1}{\PYGZsq{}}\PYG{l+s+s1}{yhat}\PYG{l+s+s1}{\PYGZsq{}}\PYG{p}{]} \PYG{o}{=} \PYG{n}{yhat\PYGZus{}test}
\PYG{n}{test}\PYG{p}{[}\PYG{l+s+s1}{\PYGZsq{}}\PYG{l+s+s1}{yhat\PYGZus{}rank}\PYG{l+s+s1}{\PYGZsq{}}\PYG{p}{]} \PYG{o}{=} \PYG{n}{pd}\PYG{o}{.}\PYG{n}{qcut}\PYG{p}{(}\PYG{n}{test}\PYG{p}{[}\PYG{l+s+s1}{\PYGZsq{}}\PYG{l+s+s1}{yhat}\PYG{l+s+s1}{\PYGZsq{}}\PYG{p}{]}\PYG{p}{,} \PYG{n}{q}\PYG{o}{=}\PYG{l+m+mi}{10}\PYG{p}{)}
\PYG{n}{test}\PYG{o}{.}\PYG{n}{groupby}\PYG{p}{(}\PYG{l+s+s1}{\PYGZsq{}}\PYG{l+s+s1}{yhat\PYGZus{}rank}\PYG{l+s+s1}{\PYGZsq{}}\PYG{p}{)}\PYG{p}{[}\PYG{l+s+s1}{\PYGZsq{}}\PYG{l+s+s1}{target}\PYG{l+s+s1}{\PYGZsq{}}\PYG{p}{]}\PYG{o}{.}\PYG{n}{mean}\PYG{p}{(}\PYG{p}{)}
\end{sphinxVerbatim}

\end{sphinxuseclass}\end{sphinxVerbatimInput}
\begin{sphinxVerbatimOutput}

\begin{sphinxuseclass}{cell_output}
\begin{sphinxVerbatim}[commandchars=\\\{\}]
yhat\PYGZus{}rank
(0.0342, 0.184]   0.156
(0.184, 0.198]    0.182
(0.198, 0.209]    0.192
(0.209, 0.221]    0.203
(0.221, 0.234]    0.224
(0.234, 0.249]    0.238
(0.249, 0.267]    0.259
(0.267, 0.292]    0.283
(0.292, 0.327]    0.320
(0.327, 0.783]    0.373
Name: target, dtype: float64
\end{sphinxVerbatim}

\end{sphinxuseclass}\end{sphinxVerbatimOutput}

\end{sphinxuseclass}
\sphinxAtStartPar
 종목선정은 상위 스코어 구간에서 할 것이므로 상위 구간에서 대하여 수익률 및 표준화 가격 구간으로 분리해서 미래 수익률을 보겠습니다. 표준화된 가격이 낮고 당일 수익율이 높은 경우 미래 수익률이 높을 것으로 예상됩니다.

\begin{sphinxuseclass}{cell}\begin{sphinxVerbatimInput}

\begin{sphinxuseclass}{cell_input}
\begin{sphinxVerbatim}[commandchars=\\\{\}]
\PYG{n}{tops} \PYG{o}{=} \PYG{n}{test}\PYG{p}{[}\PYG{n}{test}\PYG{p}{[}\PYG{l+s+s1}{\PYGZsq{}}\PYG{l+s+s1}{yhat}\PYG{l+s+s1}{\PYGZsq{}}\PYG{p}{]} \PYG{o}{\PYGZgt{}} \PYG{l+m+mf}{0.3}\PYG{p}{]}\PYG{o}{.}\PYG{n}{copy}\PYG{p}{(}\PYG{p}{)}

\PYG{n}{tops}\PYG{p}{[}\PYG{l+s+s1}{\PYGZsq{}}\PYG{l+s+s1}{return\PYGZus{}rank}\PYG{l+s+s1}{\PYGZsq{}}\PYG{p}{]}  \PYG{o}{=} \PYG{n}{pd}\PYG{o}{.}\PYG{n}{qcut}\PYG{p}{(}\PYG{n}{tops}\PYG{p}{[}\PYG{l+s+s1}{\PYGZsq{}}\PYG{l+s+s1}{return}\PYG{l+s+s1}{\PYGZsq{}}\PYG{p}{]}\PYG{p}{,} \PYG{n}{q}\PYG{o}{=}\PYG{l+m+mi}{5}\PYG{p}{)} \PYG{c+c1}{\PYGZsh{} 종가 수익률}
\PYG{n}{tops}\PYG{p}{[}\PYG{l+s+s1}{\PYGZsq{}}\PYG{l+s+s1}{price\PYGZus{}rank}\PYG{l+s+s1}{\PYGZsq{}}\PYG{p}{]}  \PYG{o}{=} \PYG{n}{pd}\PYG{o}{.}\PYG{n}{qcut}\PYG{p}{(}\PYG{n}{tops}\PYG{p}{[}\PYG{l+s+s1}{\PYGZsq{}}\PYG{l+s+s1}{price\PYGZus{}z}\PYG{l+s+s1}{\PYGZsq{}}\PYG{p}{]}\PYG{p}{,} \PYG{n}{q}\PYG{o}{=}\PYG{l+m+mi}{5}\PYG{p}{)} \PYG{c+c1}{\PYGZsh{} 가격 변동성}
\PYG{n}{tops}\PYG{o}{.}\PYG{n}{groupby}\PYG{p}{(}\PYG{p}{[}\PYG{l+s+s1}{\PYGZsq{}}\PYG{l+s+s1}{return\PYGZus{}rank}\PYG{l+s+s1}{\PYGZsq{}}\PYG{p}{,}\PYG{l+s+s1}{\PYGZsq{}}\PYG{l+s+s1}{price\PYGZus{}rank}\PYG{l+s+s1}{\PYGZsq{}}\PYG{p}{]}\PYG{p}{)}\PYG{p}{[}\PYG{l+s+s1}{\PYGZsq{}}\PYG{l+s+s1}{target}\PYG{l+s+s1}{\PYGZsq{}}\PYG{p}{]}\PYG{o}{.}\PYG{n}{mean}\PYG{p}{(}\PYG{p}{)}\PYG{o}{.}\PYG{n}{unstack}\PYG{p}{(}\PYG{p}{)}\PYG{o}{.}\PYG{n}{style}\PYG{o}{.}\PYG{n}{set\PYGZus{}table\PYGZus{}attributes}\PYG{p}{(}\PYG{l+s+s1}{\PYGZsq{}}\PYG{l+s+s1}{style=}\PYG{l+s+s1}{\PYGZdq{}}\PYG{l+s+s1}{font\PYGZhy{}size: 12px}\PYG{l+s+s1}{\PYGZdq{}}\PYG{l+s+s1}{\PYGZsq{}}\PYG{p}{)}
\end{sphinxVerbatim}

\end{sphinxuseclass}\end{sphinxVerbatimInput}
\begin{sphinxVerbatimOutput}

\begin{sphinxuseclass}{cell_output}
\begin{sphinxVerbatim}[commandchars=\\\{\}]
\PYGZlt{}pandas.io.formats.style.Styler at 0x1d8d6e270a0\PYGZgt{}
\end{sphinxVerbatim}

\end{sphinxuseclass}\end{sphinxVerbatimOutput}

\end{sphinxuseclass}
\sphinxAtStartPar
 참고로 groupby 로 데이터를 요약하는 방법은 직관적이나, 각 행과 열의 총계는 보여주지 않는다는 단점이 있습니다. 총계가 보고 싶을 때는 pivot\_table 에서 ‘margins=True’ 를 인수로 넣어주면 총계를 볼 수 있습니다.

\begin{sphinxuseclass}{cell}\begin{sphinxVerbatimInput}

\begin{sphinxuseclass}{cell_input}
\begin{sphinxVerbatim}[commandchars=\\\{\}]
\PYG{n}{pd}\PYG{o}{.}\PYG{n}{pivot\PYGZus{}table}\PYG{p}{(}\PYG{n}{data} \PYG{o}{=} \PYG{n}{tops}\PYG{p}{,} \PYG{n}{index} \PYG{o}{=} \PYG{l+s+s1}{\PYGZsq{}}\PYG{l+s+s1}{return\PYGZus{}rank}\PYG{l+s+s1}{\PYGZsq{}}\PYG{p}{,} \PYG{n}{columns} \PYG{o}{=} \PYG{l+s+s1}{\PYGZsq{}}\PYG{l+s+s1}{price\PYGZus{}rank}\PYG{l+s+s1}{\PYGZsq{}}\PYG{p}{,} \PYG{n}{values} \PYG{o}{=} \PYG{l+s+s1}{\PYGZsq{}}\PYG{l+s+s1}{target}\PYG{l+s+s1}{\PYGZsq{}}\PYG{p}{,} \PYG{n}{aggfunc}\PYG{o}{=}\PYG{l+s+s1}{\PYGZsq{}}\PYG{l+s+s1}{mean}\PYG{l+s+s1}{\PYGZsq{}}\PYG{p}{,} \PYG{n}{margins}\PYG{o}{=}\PYG{k+kc}{True}\PYG{p}{)}\PYG{o}{.}\PYG{n}{style}\PYG{o}{.}\PYG{n}{set\PYGZus{}table\PYGZus{}attributes}\PYG{p}{(}\PYG{l+s+s1}{\PYGZsq{}}\PYG{l+s+s1}{style=}\PYG{l+s+s1}{\PYGZdq{}}\PYG{l+s+s1}{font\PYGZhy{}size: 12px}\PYG{l+s+s1}{\PYGZdq{}}\PYG{l+s+s1}{\PYGZsq{}}\PYG{p}{)}
\end{sphinxVerbatim}

\end{sphinxuseclass}\end{sphinxVerbatimInput}
\begin{sphinxVerbatimOutput}

\begin{sphinxuseclass}{cell_output}
\begin{sphinxVerbatim}[commandchars=\\\{\}]
\PYGZlt{}pandas.io.formats.style.Styler at 0x1d8d6e44af0\PYGZgt{}
\end{sphinxVerbatim}

\end{sphinxuseclass}\end{sphinxVerbatimOutput}

\end{sphinxuseclass}
\sphinxAtStartPar
 최저 수익률(리스크)도 조사합니다. 당일 수익률 높고, 표준화 된 주가가 낮은 좌하단 부분의 리스크가 낮습니다.

\begin{sphinxuseclass}{cell}\begin{sphinxVerbatimInput}

\begin{sphinxuseclass}{cell_input}
\begin{sphinxVerbatim}[commandchars=\\\{\}]
\PYG{n}{pd}\PYG{o}{.}\PYG{n}{pivot\PYGZus{}table}\PYG{p}{(}\PYG{n}{data} \PYG{o}{=} \PYG{n}{tops}\PYG{p}{,} \PYG{n}{index} \PYG{o}{=} \PYG{l+s+s1}{\PYGZsq{}}\PYG{l+s+s1}{return\PYGZus{}rank}\PYG{l+s+s1}{\PYGZsq{}}\PYG{p}{,} \PYG{n}{columns} \PYG{o}{=} \PYG{l+s+s1}{\PYGZsq{}}\PYG{l+s+s1}{price\PYGZus{}rank}\PYG{l+s+s1}{\PYGZsq{}}\PYG{p}{,} \PYG{n}{values} \PYG{o}{=} \PYG{l+s+s1}{\PYGZsq{}}\PYG{l+s+s1}{min\PYGZus{}close}\PYG{l+s+s1}{\PYGZsq{}}\PYG{p}{,} \PYG{n}{aggfunc}\PYG{o}{=}\PYG{l+s+s1}{\PYGZsq{}}\PYG{l+s+s1}{mean}\PYG{l+s+s1}{\PYGZsq{}}\PYG{p}{,} \PYG{n}{margins}\PYG{o}{=}\PYG{k+kc}{True}\PYG{p}{)}\PYG{o}{.}\PYG{n}{style}\PYG{o}{.}\PYG{n}{set\PYGZus{}table\PYGZus{}attributes}\PYG{p}{(}\PYG{l+s+s1}{\PYGZsq{}}\PYG{l+s+s1}{style=}\PYG{l+s+s1}{\PYGZdq{}}\PYG{l+s+s1}{font\PYGZhy{}size: 12px}\PYG{l+s+s1}{\PYGZdq{}}\PYG{l+s+s1}{\PYGZsq{}}\PYG{p}{)}
\end{sphinxVerbatim}

\end{sphinxuseclass}\end{sphinxVerbatimInput}
\begin{sphinxVerbatimOutput}

\begin{sphinxuseclass}{cell_output}
\begin{sphinxVerbatim}[commandchars=\\\{\}]
\PYGZlt{}pandas.io.formats.style.Styler at 0x1d8d432a310\PYGZgt{}
\end{sphinxVerbatim}

\end{sphinxuseclass}\end{sphinxVerbatimOutput}

\end{sphinxuseclass}
\sphinxAtStartPar
 위 결과를 종합하면 당일 종가 수익률은 높고, 최근 20일 대비 가격이 낮은 종목을 매수하면 리스크가 적을 것으로 판단됩니다. ‘return’ 은 1.03 보다 크고, ‘price\_z’ 는 0 보다 작은 종목만을 고르겠습니다.

\begin{sphinxuseclass}{cell}\begin{sphinxVerbatimInput}

\begin{sphinxuseclass}{cell_input}
\begin{sphinxVerbatim}[commandchars=\\\{\}]
\PYG{n}{tops}\PYG{p}{[} \PYG{p}{(}\PYG{n}{tops}\PYG{p}{[}\PYG{l+s+s1}{\PYGZsq{}}\PYG{l+s+s1}{return}\PYG{l+s+s1}{\PYGZsq{}}\PYG{p}{]} \PYG{o}{\PYGZgt{}} \PYG{l+m+mf}{1.03}\PYG{p}{)} \PYG{o}{\PYGZam{}} \PYG{p}{(}\PYG{n}{tops}\PYG{p}{[}\PYG{l+s+s1}{\PYGZsq{}}\PYG{l+s+s1}{price\PYGZus{}z}\PYG{l+s+s1}{\PYGZsq{}}\PYG{p}{]} \PYG{o}{\PYGZlt{}} \PYG{l+m+mi}{0}\PYG{p}{)}\PYG{p}{]}\PYG{p}{[}\PYG{p}{[}\PYG{l+s+s1}{\PYGZsq{}}\PYG{l+s+s1}{return}\PYG{l+s+s1}{\PYGZsq{}}\PYG{p}{,}\PYG{l+s+s1}{\PYGZsq{}}\PYG{l+s+s1}{price\PYGZus{}z}\PYG{l+s+s1}{\PYGZsq{}}\PYG{p}{]}\PYG{p}{]}\PYG{o}{.}\PYG{n}{head}\PYG{p}{(}\PYG{p}{)}\PYG{o}{.}\PYG{n}{style}\PYG{o}{.}\PYG{n}{set\PYGZus{}table\PYGZus{}attributes}\PYG{p}{(}\PYG{l+s+s1}{\PYGZsq{}}\PYG{l+s+s1}{style=}\PYG{l+s+s1}{\PYGZdq{}}\PYG{l+s+s1}{font\PYGZhy{}size: 12px}\PYG{l+s+s1}{\PYGZdq{}}\PYG{l+s+s1}{\PYGZsq{}}\PYG{p}{)}
\end{sphinxVerbatim}

\end{sphinxuseclass}\end{sphinxVerbatimInput}
\begin{sphinxVerbatimOutput}

\begin{sphinxuseclass}{cell_output}
\begin{sphinxVerbatim}[commandchars=\\\{\}]
\PYGZlt{}pandas.io.formats.style.Styler at 0x1d8d7617a00\PYGZgt{}
\end{sphinxVerbatim}

\end{sphinxuseclass}\end{sphinxVerbatimOutput}

\end{sphinxuseclass}

\section{선형모델 가정에 대한 이해}
\label{\detokenize{chapter5/5.2.4_LM_Assumptions:id1}}\label{\detokenize{chapter5/5.2.4_LM_Assumptions::doc}}
\sphinxAtStartPar
왜 매수결정을 했는지에 대한 이유를 구체적으로 설명하기 유리한 모델은 Linear Model 입니다. 그 중 다변량 회귀모델 (Multivariate Linear Regression) 은 데이터분석을 배울 때, 가장 기초적으로 다루는 예측모델입니다. 예측하고자 하는 종속변수 Y (레이블 혹은 타겟 변수) 가 연속형이고, 이것을 설명 혹은 예측하는 독립변수 X (입력피쳐 혹은 입력변수) 들의 선형조합 Z 로 Fitting 을 하는 것인데, 충분한 이해없이 사용하면, 잘못된 결론을 내기 쉽습니다. 다변량 회귀분석 모델이 의미가 있을려면, 데이터가 상당히 강한 Assumptions 를 만족해야 합니다. 중요한 4 가지는 다음과 같습니다.
\begin{enumerate}
\sphinxsetlistlabels{\arabic}{enumi}{enumii}{}{.}%
\item {} 
\sphinxAtStartPar
Normality \sphinxhyphen{} 에러(실제값 \sphinxhyphen{} 예측값)가 정규분포를 따라야한다. 사실 이건 Y 가 정규분포를 따라야 한다는 것과 크게 다르지 않습니다.

\item {} 
\sphinxAtStartPar
Weak Heteroscedasticity \sphinxhyphen{} 에러가 등분산성을 만족해야 한다. 즉 에러의 분산이 예측 값의 크기에 따라서 크게 변화하지 않아야 한다.

\item {} 
\sphinxAtStartPar
Linearity \sphinxhyphen{} 선형성. 이것은 추정된 베타값이 X 값의 크기에 따라서 변화하지 않아야 한다. 예를 들어, 소득을 추정하는데, 카드 사용량이 변수라면 카드 월 사용량이 백만원일 때 추정된 계수(coefficient) 가 50 이라면, 카드 사용량이 천 만원일때도 베타 계수가 50이여야 한다는 말입니다.

\item {} 
\sphinxAtStartPar
Weak Multicollinearity \sphinxhyphen{} (다중 공선성)이 크지 않아야 한다. 쉽게 이야기 하면 어떤 여러개의 X 가 Y 를 설명하는데 있어서 X 들이 같은 방향으로 움직이면 안 된다고 이해하면 될 것 같습니다. 다중 공선성이 큰 경우는 계수 값이 정확하지 않아서, 계수에 대한 해석이 불가능하게 됩니다. 아주 심한 경우는 다른 변수의 영향으로 양의 계수가 음의 계수로 바뀌게 됩니다.

\end{enumerate}

\sphinxAtStartPar
위 가정 1 번과 2 번을 만족하지 않아도 Regression 을 할 수 있게 일반화 한 것이 일반화 회귀모형(Generalized Linear Model) 입니다. GLM 에서는 Y 가 갯수(count), 비율(proportion), 이진(0과 1) 등 같이 연속형 변수가 아니고 정규 분포를 따르지 않아도 선형모델을 구현할 수 있습니다. 물론 등분산성을 만족하지 않아도 됩니다. 대신에 Y 에 대한 명확한 분포 설정과 Y 에 대한 Link Function 필요합니다. 가장 많이 쓰이는 것이 Log Link 입니다. 이 부분을 쉽게 이해하기 위해서는 이렇게 생각하면 됩니다. X 의 선형조합 Z 는 음수의 값도 갖게 되는데, 비율이나, 갯수는 항상 양수입니다. 따라서 Y 에 Log 를 씌워서 음수를 갖게 할 수 있습니다. 반대로 EXP( a0 + a1\sphinxstyleemphasis{x1 + a2}x2 …) 로 항상 양수인 Y 를 Fitting 한다고 보시면 될 것 같습니다. 많이 다루는 로지스틱 회귀 모델은 Log(odds) 를 X 의 선형조합으로 Fitting 을 하는 일반화 선형모형의 한 예로 볼 수 있습니다. 데이터상으로는 Y 가 이항분포(Bernoulli 분포 혹은 0 과 1) 이므로 Link Function 가 Logit Link 즉, log(p/1\sphinxhyphen{}p) 로 하는 일반화 선형모형과 동일한 의미가 됩니다.  Y 가 0 과 1 이므로 이것을 가장 잘 근사하게 따라갈 수 있는 변형은 Logit Link 인 것입니다. logit Link 를 풀면 Y = exp(z) / 1 + exp(z) 가 됩니다. 즉 Y 를 설명하기에 좋은 형태로 변경이 되는 것입니다. Y 가  개 수(count) 인 경우는 포아송 회귀분석 (Poisson regression) 입니다. 주어진 시간 혹은 범위에서 뽑은 count 샘플은 포아송 분포를 따라간다는 것이 알려져 있습니다. 예를 들면 인구 만명당 암 발생 환자 수 등이 예가 될 것 같습니다. 포아송 분포의 평균과 분산은 같습니다. 즉 평균이 증가하면 분산이 증가하는 분포입니다. 따라서 등분산성을 만족하지 않아도 Y 를 fitting 할 수 있습니다. 이 경우, Link 는 log 입니다. 즉, X 의 선형조합인 Z 에 Exponential 를 씌워서 양수가 되도록 합니다. Y 가 비율(Proportion) 인 경우도 있습니다. 그럼 비율은 어떤 분포일지 궁금합니다. 비율은 항상 0 과 1 사이 양수이므로 Link 함수는 log link 를 쓰면 될 것 같습니다. 일반적으로 비율은 분자의 특성에 따라 분포가 바뀔 수 있습니다. 위에 예시한 인구 만명 당 암환자의 비율은 GLM 으로 Fitting (Y \textasciitilde{} Normal 분포, Log link) 할 수 있습니다. 하지만 더 Fitting 을 잘 하려면 분자를 Y 로 하고 분모인 인구 수를 exposure 요인으로 처리하는 것입니다. 이 경우 당연히 Y 는 포아송분포가 됩니다.  log(암 환수/인구 수) = Z(X 선형조합)  형태의 모델을 (암 환자수) = exp(Z)*(인구수) 이렇게 변경하는 것과 동일합니다. 그럼 여기서 인구수가 exposure 가 되고, 인구수를 고려하여 Z 에 계수값을 추정하게 됩니다. Proportion 을 Y 로  fitting 하는 것보다 훨씬 좋은 결과가 나옵니다.

\sphinxAtStartPar
마지막으로 3 번째 가정이 선형성을 만족하지 않아도 쓸 수 있는 Linear Model 이 있습니다.  Generalized Additive Model (GAM) 인데, 이경우는 spline 함수를 이용하여 각 X 를 곡선으로 만들어 Y 와 fitting 합니다. 예를 들어 Y 가 그랜저를 살 확률이고, X 가 소득이라고 할 때, 소득이 증가함에 따라 그랜저를 살 확률은 증가하다가 어느 순간 다시 감소할 것 입니다. 그럼 2 차원 곡선이 되는데요. 이런 경우도 소득을 spline 함수(곡선형태)로 만들면 Y 를 잘 Fitting 할 수 있습니다.

\sphinxAtStartPar
마지막 4 번째 가정은 선형모형의 구조상 피할 수 가 없습니다. 공선성을 일으키는 입력 변수를 빼거나, 주성분등으로 공선성을 완전히 제거해야 합니다. 기본적으로 Linear 모델이라는 것은 X 의 합으로 연결이 되어 있습니다. 따라서 Fitting 된 모델에서 X1 이 1 증가할 때,  X2 도 1 증가하는 구조라면, X1 와 X2 의 계수의 추정은 해석하기 어렵게 됩니다. 하지만, 이런 구조이기 때문에 잘 fitting 된 선형모델에서는 X 변화에 따른 Y 의 변화를 이해할 수 있는 장점으로 작용합니다. 요즘 관심을 받고 있는 해석가능한 모델이 되는 것입니다.


\part{chapter 6}


\chapter{\sphinxstylestrong{해결책의 효과 측정}}
\label{\detokenize{chapter6/6.0.0_Effectiveness_Testing:id1}}\label{\detokenize{chapter6/6.0.0_Effectiveness_Testing::doc}}
\sphinxAtStartPar
앞서 입증된 가설을 활용하며 예측모델을 구현했습니다. 이제 완성된 모델을 이용하여 종목 추천을 받는 전체 프로세스를 만들어보겠습니다.
오늘이 2022년 4월 1일라고 가정하고 어떤 종목들이 추천되는 지 보겠습니다. 4월1일 장 마감 후 프로그램을 돌려 추천 종목을 받고, 익일(4월 2일) 날 4월 1일의 종가에 매수를 하는 전략입니다.


\section{종목 추천 프로세스}
\label{\detokenize{chapter6/6.1.1_Stock_Selection:id1}}\label{\detokenize{chapter6/6.1.1_Stock_Selection::doc}}
\sphinxAtStartPar
완성된 모델을 이용하여 종목 추천을 받는 프로세스를 순서대로 만들어보겠습니다.
오늘이 2022년 4월 1일라고 가정하고 어떤 종목들이 추천되는 지 보겠습니다. 4월1일 장 마감 후 프로그램을 돌려 추천 종목을 받고, 익일(4월 2일) 날 4월 1일의 종가에 매수를 하는 전략입니다.

\begin{sphinxuseclass}{cell}\begin{sphinxVerbatimInput}

\begin{sphinxuseclass}{cell_input}
\begin{sphinxVerbatim}[commandchars=\\\{\}]
\PYG{k+kn}{import} \PYG{n+nn}{FinanceDataReader} \PYG{k}{as} \PYG{n+nn}{fdr}
\PYG{k+kn}{import} \PYG{n+nn}{matplotlib}\PYG{n+nn}{.}\PYG{n+nn}{pyplot} \PYG{k}{as} \PYG{n+nn}{plt}
\PYG{o}{\PYGZpc{}}\PYG{k}{matplotlib} inline

\PYG{k+kn}{import} \PYG{n+nn}{pandas} \PYG{k}{as} \PYG{n+nn}{pd}
\PYG{k+kn}{import} \PYG{n+nn}{numpy} \PYG{k}{as} \PYG{n+nn}{np}
\PYG{k+kn}{import} \PYG{n+nn}{requests}
\PYG{k+kn}{import} \PYG{n+nn}{bs4}

\PYG{n}{pd}\PYG{o}{.}\PYG{n}{options}\PYG{o}{.}\PYG{n}{display}\PYG{o}{.}\PYG{n}{float\PYGZus{}format} \PYG{o}{=} \PYG{l+s+s1}{\PYGZsq{}}\PYG{l+s+si}{\PYGZob{}:,.3f\PYGZcb{}}\PYG{l+s+s1}{\PYGZsq{}}\PYG{o}{.}\PYG{n}{format}
\end{sphinxVerbatim}

\end{sphinxuseclass}\end{sphinxVerbatimInput}

\end{sphinxuseclass}
\sphinxAtStartPar
 오늘이 2022년 4월 1일라고 가정하고 어떤 종목들이 추천되는 지 보겠습니다. 먼저 오늘 기준으로 100 일전 날짜를 timedelta 를 이용해 찾습니다.

\begin{sphinxuseclass}{cell}\begin{sphinxVerbatimInput}

\begin{sphinxuseclass}{cell_input}
\begin{sphinxVerbatim}[commandchars=\\\{\}]
\PYG{k+kn}{import} \PYG{n+nn}{datetime}
\PYG{n}{today\PYGZus{}dt} \PYG{o}{=} \PYG{l+s+s1}{\PYGZsq{}}\PYG{l+s+s1}{2022\PYGZhy{}04\PYGZhy{}01}\PYG{l+s+s1}{\PYGZsq{}}
\PYG{n}{today} \PYG{o}{=} \PYG{n}{datetime}\PYG{o}{.}\PYG{n}{datetime}\PYG{o}{.}\PYG{n}{strptime}\PYG{p}{(}\PYG{n}{today\PYGZus{}dt}\PYG{p}{,} \PYG{l+s+s1}{\PYGZsq{}}\PYG{l+s+s1}{\PYGZpc{}}\PYG{l+s+s1}{Y\PYGZhy{}}\PYG{l+s+s1}{\PYGZpc{}}\PYG{l+s+s1}{m\PYGZhy{}}\PYG{l+s+si}{\PYGZpc{}d}\PYG{l+s+s1}{\PYGZsq{}}\PYG{p}{)}
\PYG{n}{start\PYGZus{}dt} \PYG{o}{=} \PYG{n}{today} \PYG{o}{\PYGZhy{}} \PYG{n}{datetime}\PYG{o}{.}\PYG{n}{timedelta}\PYG{p}{(}\PYG{n}{days}\PYG{o}{=}\PYG{l+m+mi}{100}\PYG{p}{)} \PYG{c+c1}{\PYGZsh{} 100 일전 데이터 부터 시작 \PYGZhy{} 피쳐 엔지니어링은 최소 60 개의 일봉이 필요함}
\PYG{n+nb}{print}\PYG{p}{(}\PYG{n}{start\PYGZus{}dt}\PYG{p}{,} \PYG{n}{today\PYGZus{}dt}\PYG{p}{)}
\end{sphinxVerbatim}

\end{sphinxuseclass}\end{sphinxVerbatimInput}
\begin{sphinxVerbatimOutput}

\begin{sphinxuseclass}{cell_output}
\begin{sphinxVerbatim}[commandchars=\\\{\}]
2021\PYGZhy{}12\PYGZhy{}22 00:00:00 2022\PYGZhy{}04\PYGZhy{}01
\end{sphinxVerbatim}

\end{sphinxuseclass}\end{sphinxVerbatimOutput}

\end{sphinxuseclass}
\sphinxAtStartPar
 위 코드에서 찾은 시작일부터 오늘까지 종목별로 일봉을 가져와서 데이터셋을 구성합니다. 총 67 개의 일봉이 있습니다. 입력 피처를 생성하기 위해서는 최소한 60일의 데이터가 필요합니다.

\begin{sphinxuseclass}{cell}\begin{sphinxVerbatimInput}

\begin{sphinxuseclass}{cell_input}
\begin{sphinxVerbatim}[commandchars=\\\{\}]
\PYG{n}{kosdaq\PYGZus{}list} \PYG{o}{=} \PYG{n}{pd}\PYG{o}{.}\PYG{n}{read\PYGZus{}pickle}\PYG{p}{(}\PYG{l+s+s1}{\PYGZsq{}}\PYG{l+s+s1}{kosdaq\PYGZus{}list.pkl}\PYG{l+s+s1}{\PYGZsq{}}\PYG{p}{)}

\PYG{n}{price\PYGZus{}data} \PYG{o}{=} \PYG{n}{pd}\PYG{o}{.}\PYG{n}{DataFrame}\PYG{p}{(}\PYG{p}{)}

\PYG{k}{for} \PYG{n}{code}\PYG{p}{,} \PYG{n}{name} \PYG{o+ow}{in} \PYG{n+nb}{zip}\PYG{p}{(}\PYG{n}{kosdaq\PYGZus{}list}\PYG{p}{[}\PYG{l+s+s1}{\PYGZsq{}}\PYG{l+s+s1}{code}\PYG{l+s+s1}{\PYGZsq{}}\PYG{p}{]}\PYG{p}{,} \PYG{n}{kosdaq\PYGZus{}list}\PYG{p}{[}\PYG{l+s+s1}{\PYGZsq{}}\PYG{l+s+s1}{name}\PYG{l+s+s1}{\PYGZsq{}}\PYG{p}{]}\PYG{p}{)}\PYG{p}{:}  \PYG{c+c1}{\PYGZsh{} 코스닥 모든 종목에서 대하여 반복}
    \PYG{n}{daily\PYGZus{}price} \PYG{o}{=} \PYG{n}{fdr}\PYG{o}{.}\PYG{n}{DataReader}\PYG{p}{(}\PYG{n}{code}\PYG{p}{,}  \PYG{n}{start} \PYG{o}{=} \PYG{n}{start\PYGZus{}dt}\PYG{p}{,} \PYG{n}{end} \PYG{o}{=} \PYG{n}{today\PYGZus{}dt}\PYG{p}{)} \PYG{c+c1}{\PYGZsh{} 종목, 일봉, 데이터 갯수}
    \PYG{n}{daily\PYGZus{}price}\PYG{p}{[}\PYG{l+s+s1}{\PYGZsq{}}\PYG{l+s+s1}{code}\PYG{l+s+s1}{\PYGZsq{}}\PYG{p}{]} \PYG{o}{=} \PYG{n}{code}
    \PYG{n}{daily\PYGZus{}price}\PYG{p}{[}\PYG{l+s+s1}{\PYGZsq{}}\PYG{l+s+s1}{name}\PYG{l+s+s1}{\PYGZsq{}}\PYG{p}{]} \PYG{o}{=} \PYG{n}{name}
    \PYG{n}{price\PYGZus{}data} \PYG{o}{=} \PYG{n}{pd}\PYG{o}{.}\PYG{n}{concat}\PYG{p}{(}\PYG{p}{[}\PYG{n}{price\PYGZus{}data}\PYG{p}{,} \PYG{n}{daily\PYGZus{}price}\PYG{p}{]}\PYG{p}{,} \PYG{n}{axis}\PYG{o}{=}\PYG{l+m+mi}{0}\PYG{p}{)}   

\PYG{n}{price\PYGZus{}data}\PYG{o}{.}\PYG{n}{index}\PYG{o}{.}\PYG{n}{name} \PYG{o}{=} \PYG{l+s+s1}{\PYGZsq{}}\PYG{l+s+s1}{date}\PYG{l+s+s1}{\PYGZsq{}}
\PYG{n}{price\PYGZus{}data}\PYG{o}{.}\PYG{n}{columns}\PYG{o}{=} \PYG{n}{price\PYGZus{}data}\PYG{o}{.}\PYG{n}{columns}\PYG{o}{.}\PYG{n}{str}\PYG{o}{.}\PYG{n}{lower}\PYG{p}{(}\PYG{p}{)} \PYG{c+c1}{\PYGZsh{} 컬럼 이름 소문자로 변경}

\PYG{n+nb}{print}\PYG{p}{(}\PYG{n}{price\PYGZus{}data}\PYG{o}{.}\PYG{n}{index}\PYG{o}{.}\PYG{n}{nunique}\PYG{p}{(}\PYG{p}{)}\PYG{p}{)}
\end{sphinxVerbatim}

\end{sphinxuseclass}\end{sphinxVerbatimInput}
\begin{sphinxVerbatimOutput}

\begin{sphinxuseclass}{cell_output}
\begin{sphinxVerbatim}[commandchars=\\\{\}]
67
\end{sphinxVerbatim}

\end{sphinxuseclass}\end{sphinxVerbatimOutput}

\end{sphinxuseclass}
\sphinxAtStartPar
 주가지수 데이터를 가져오고, 일봉데이터에 추가합니다. 그리고 결과물을 merge 라는 이름으로 저장합니다.

\begin{sphinxuseclass}{cell}\begin{sphinxVerbatimInput}

\begin{sphinxuseclass}{cell_input}
\begin{sphinxVerbatim}[commandchars=\\\{\}]
\PYG{n}{kosdaq\PYGZus{}index} \PYG{o}{=} \PYG{n}{fdr}\PYG{o}{.}\PYG{n}{DataReader}\PYG{p}{(}\PYG{l+s+s1}{\PYGZsq{}}\PYG{l+s+s1}{KQ11}\PYG{l+s+s1}{\PYGZsq{}}\PYG{p}{,} \PYG{n}{start} \PYG{o}{=} \PYG{n}{start\PYGZus{}dt}\PYG{p}{,} \PYG{n}{end} \PYG{o}{=} \PYG{n}{today\PYGZus{}dt}\PYG{p}{)} \PYG{c+c1}{\PYGZsh{} 데이터 호출}
\PYG{n}{kosdaq\PYGZus{}index}\PYG{o}{.}\PYG{n}{columns} \PYG{o}{=} \PYG{p}{[}\PYG{l+s+s1}{\PYGZsq{}}\PYG{l+s+s1}{close}\PYG{l+s+s1}{\PYGZsq{}}\PYG{p}{,}\PYG{l+s+s1}{\PYGZsq{}}\PYG{l+s+s1}{open}\PYG{l+s+s1}{\PYGZsq{}}\PYG{p}{,}\PYG{l+s+s1}{\PYGZsq{}}\PYG{l+s+s1}{high}\PYG{l+s+s1}{\PYGZsq{}}\PYG{p}{,}\PYG{l+s+s1}{\PYGZsq{}}\PYG{l+s+s1}{low}\PYG{l+s+s1}{\PYGZsq{}}\PYG{p}{,}\PYG{l+s+s1}{\PYGZsq{}}\PYG{l+s+s1}{volume}\PYG{l+s+s1}{\PYGZsq{}}\PYG{p}{,}\PYG{l+s+s1}{\PYGZsq{}}\PYG{l+s+s1}{change}\PYG{l+s+s1}{\PYGZsq{}}\PYG{p}{]} \PYG{c+c1}{\PYGZsh{} 컬럼명 변경}
\PYG{n}{kosdaq\PYGZus{}index}\PYG{o}{.}\PYG{n}{index}\PYG{o}{.}\PYG{n}{name}\PYG{o}{=}\PYG{l+s+s1}{\PYGZsq{}}\PYG{l+s+s1}{date}\PYG{l+s+s1}{\PYGZsq{}} \PYG{c+c1}{\PYGZsh{} 인덱스 이름 생성}
\PYG{n}{kosdaq\PYGZus{}index}\PYG{o}{.}\PYG{n}{sort\PYGZus{}index}\PYG{p}{(}\PYG{n}{inplace}\PYG{o}{=}\PYG{k+kc}{True}\PYG{p}{)} \PYG{c+c1}{\PYGZsh{} 인덱스(날짜) 로 정렬 }
\PYG{n}{kosdaq\PYGZus{}index}\PYG{p}{[}\PYG{l+s+s1}{\PYGZsq{}}\PYG{l+s+s1}{kosdaq\PYGZus{}return}\PYG{l+s+s1}{\PYGZsq{}}\PYG{p}{]} \PYG{o}{=} \PYG{n}{kosdaq\PYGZus{}index}\PYG{p}{[}\PYG{l+s+s1}{\PYGZsq{}}\PYG{l+s+s1}{close}\PYG{l+s+s1}{\PYGZsq{}}\PYG{p}{]}\PYG{o}{/}\PYG{n}{kosdaq\PYGZus{}index}\PYG{p}{[}\PYG{l+s+s1}{\PYGZsq{}}\PYG{l+s+s1}{close}\PYG{l+s+s1}{\PYGZsq{}}\PYG{p}{]}\PYG{o}{.}\PYG{n}{shift}\PYG{p}{(}\PYG{l+m+mi}{1}\PYG{p}{)} \PYG{c+c1}{\PYGZsh{} 수익율 : 전 날 종가대비 당일 종가}

\PYG{n}{merged} \PYG{o}{=} \PYG{n}{price\PYGZus{}data}\PYG{o}{.}\PYG{n}{merge}\PYG{p}{(}\PYG{n}{kosdaq\PYGZus{}index}\PYG{p}{[}\PYG{l+s+s1}{\PYGZsq{}}\PYG{l+s+s1}{kosdaq\PYGZus{}return}\PYG{l+s+s1}{\PYGZsq{}}\PYG{p}{]}\PYG{p}{,} \PYG{n}{left\PYGZus{}index}\PYG{o}{=}\PYG{k+kc}{True}\PYG{p}{,} \PYG{n}{right\PYGZus{}index}\PYG{o}{=}\PYG{k+kc}{True}\PYG{p}{,} \PYG{n}{how}\PYG{o}{=}\PYG{l+s+s1}{\PYGZsq{}}\PYG{l+s+s1}{left}\PYG{l+s+s1}{\PYGZsq{}}\PYG{p}{)}
\end{sphinxVerbatim}

\end{sphinxuseclass}\end{sphinxVerbatimInput}

\end{sphinxuseclass}
\begin{sphinxuseclass}{cell}\begin{sphinxVerbatimInput}

\begin{sphinxuseclass}{cell_input}
\begin{sphinxVerbatim}[commandchars=\\\{\}]
\PYG{n}{merged}\PYG{o}{.}\PYG{n}{to\PYGZus{}pickle}\PYG{p}{(}\PYG{l+s+s1}{\PYGZsq{}}\PYG{l+s+s1}{merged.pkl}\PYG{l+s+s1}{\PYGZsq{}}\PYG{p}{)}
\end{sphinxVerbatim}

\end{sphinxuseclass}\end{sphinxVerbatimInput}

\end{sphinxuseclass}
\sphinxAtStartPar
 주가 지수 수익률과 종목별 수익율을 비교한 결과를 win\_market 이라는 변수에 담습니다.

\begin{sphinxuseclass}{cell}\begin{sphinxVerbatimInput}

\begin{sphinxuseclass}{cell_input}
\begin{sphinxVerbatim}[commandchars=\\\{\}]
\PYG{n}{merged} \PYG{o}{=} \PYG{n}{pd}\PYG{o}{.}\PYG{n}{read\PYGZus{}pickle}\PYG{p}{(}\PYG{l+s+s1}{\PYGZsq{}}\PYG{l+s+s1}{merged.pkl}\PYG{l+s+s1}{\PYGZsq{}}\PYG{p}{)}

\PYG{n}{return\PYGZus{}all} \PYG{o}{=} \PYG{n}{pd}\PYG{o}{.}\PYG{n}{DataFrame}\PYG{p}{(}\PYG{p}{)}

\PYG{k}{for} \PYG{n}{code} \PYG{o+ow}{in} \PYG{n}{kosdaq\PYGZus{}list}\PYG{p}{[}\PYG{l+s+s1}{\PYGZsq{}}\PYG{l+s+s1}{code}\PYG{l+s+s1}{\PYGZsq{}}\PYG{p}{]}\PYG{p}{:}  
    
    \PYG{n}{stock\PYGZus{}return} \PYG{o}{=} \PYG{n}{merged}\PYG{p}{[}\PYG{n}{merged}\PYG{p}{[}\PYG{l+s+s1}{\PYGZsq{}}\PYG{l+s+s1}{code}\PYG{l+s+s1}{\PYGZsq{}}\PYG{p}{]}\PYG{o}{==}\PYG{n}{code}\PYG{p}{]}\PYG{o}{.}\PYG{n}{sort\PYGZus{}index}\PYG{p}{(}\PYG{p}{)}
    \PYG{n}{stock\PYGZus{}return}\PYG{p}{[}\PYG{l+s+s1}{\PYGZsq{}}\PYG{l+s+s1}{return}\PYG{l+s+s1}{\PYGZsq{}}\PYG{p}{]} \PYG{o}{=} \PYG{n}{stock\PYGZus{}return}\PYG{p}{[}\PYG{l+s+s1}{\PYGZsq{}}\PYG{l+s+s1}{close}\PYG{l+s+s1}{\PYGZsq{}}\PYG{p}{]}\PYG{o}{/}\PYG{n}{stock\PYGZus{}return}\PYG{p}{[}\PYG{l+s+s1}{\PYGZsq{}}\PYG{l+s+s1}{close}\PYG{l+s+s1}{\PYGZsq{}}\PYG{p}{]}\PYG{o}{.}\PYG{n}{shift}\PYG{p}{(}\PYG{l+m+mi}{1}\PYG{p}{)} \PYG{c+c1}{\PYGZsh{} 종목별 전일 종가 대비 당일 종가 수익율}
    \PYG{n}{c1} \PYG{o}{=} \PYG{p}{(}\PYG{n}{stock\PYGZus{}return}\PYG{p}{[}\PYG{l+s+s1}{\PYGZsq{}}\PYG{l+s+s1}{kosdaq\PYGZus{}return}\PYG{l+s+s1}{\PYGZsq{}}\PYG{p}{]} \PYG{o}{\PYGZlt{}} \PYG{l+m+mi}{1}\PYG{p}{)} \PYG{c+c1}{\PYGZsh{} 수익율 1 보다 작음. 당일 종가가 전일 종가보다 낮음 (코스닥 지표)}
    \PYG{n}{c2} \PYG{o}{=} \PYG{p}{(}\PYG{n}{stock\PYGZus{}return}\PYG{p}{[}\PYG{l+s+s1}{\PYGZsq{}}\PYG{l+s+s1}{return}\PYG{l+s+s1}{\PYGZsq{}}\PYG{p}{]} \PYG{o}{\PYGZgt{}} \PYG{l+m+mi}{1}\PYG{p}{)} \PYG{c+c1}{\PYGZsh{} 수익율 1 보다 큼. 당일 종가가 전일 종가보다 큼 (개별 종목)}
    \PYG{n}{stock\PYGZus{}return}\PYG{p}{[}\PYG{l+s+s1}{\PYGZsq{}}\PYG{l+s+s1}{win\PYGZus{}market}\PYG{l+s+s1}{\PYGZsq{}}\PYG{p}{]} \PYG{o}{=} \PYG{n}{np}\PYG{o}{.}\PYG{n}{where}\PYG{p}{(}\PYG{p}{(}\PYG{n}{c1}\PYG{o}{\PYGZam{}}\PYG{n}{c2}\PYG{p}{)}\PYG{p}{,} \PYG{l+m+mi}{1}\PYG{p}{,} \PYG{l+m+mi}{0}\PYG{p}{)} \PYG{c+c1}{\PYGZsh{} C1 과 C2 조건을 동시에 만족하면 1, 아니면 0}
    \PYG{n}{return\PYGZus{}all} \PYG{o}{=} \PYG{n}{pd}\PYG{o}{.}\PYG{n}{concat}\PYG{p}{(}\PYG{p}{[}\PYG{n}{return\PYGZus{}all}\PYG{p}{,} \PYG{n}{stock\PYGZus{}return}\PYG{p}{]}\PYG{p}{,} \PYG{n}{axis}\PYG{o}{=}\PYG{l+m+mi}{0}\PYG{p}{)} 
    
\PYG{n}{return\PYGZus{}all}\PYG{o}{.}\PYG{n}{dropna}\PYG{p}{(}\PYG{n}{inplace}\PYG{o}{=}\PYG{k+kc}{True}\PYG{p}{)}    
\end{sphinxVerbatim}

\end{sphinxuseclass}\end{sphinxVerbatimInput}

\end{sphinxuseclass}
\sphinxAtStartPar
 데이터가 잘 생성되었는 지 확인해 봅니다.

\begin{sphinxuseclass}{cell}\begin{sphinxVerbatimInput}

\begin{sphinxuseclass}{cell_input}
\begin{sphinxVerbatim}[commandchars=\\\{\}]
\PYG{n}{return\PYGZus{}all}\PYG{o}{.}\PYG{n}{head}\PYG{p}{(}\PYG{p}{)}\PYG{o}{.}\PYG{n}{style}\PYG{o}{.}\PYG{n}{set\PYGZus{}table\PYGZus{}attributes}\PYG{p}{(}\PYG{l+s+s1}{\PYGZsq{}}\PYG{l+s+s1}{style=}\PYG{l+s+s1}{\PYGZdq{}}\PYG{l+s+s1}{font\PYGZhy{}size: 12px}\PYG{l+s+s1}{\PYGZdq{}}\PYG{l+s+s1}{\PYGZsq{}}\PYG{p}{)}\PYG{o}{.}\PYG{n}{format}\PYG{p}{(}\PYG{n}{precision}\PYG{o}{=}\PYG{l+m+mi}{3}\PYG{p}{)}
\end{sphinxVerbatim}

\end{sphinxuseclass}\end{sphinxVerbatimInput}
\begin{sphinxVerbatimOutput}

\begin{sphinxuseclass}{cell_output}
\begin{sphinxVerbatim}[commandchars=\\\{\}]
\PYGZlt{}pandas.io.formats.style.Styler at 0x2029278cee0\PYGZgt{}
\end{sphinxVerbatim}

\end{sphinxuseclass}\end{sphinxVerbatimOutput}

\end{sphinxuseclass}
\sphinxAtStartPar
  모델에 입력할 변수를 생성합니다.

\begin{sphinxuseclass}{cell}\begin{sphinxVerbatimInput}

\begin{sphinxuseclass}{cell_input}
\begin{sphinxVerbatim}[commandchars=\\\{\}]
\PYG{n}{model\PYGZus{}inputs} \PYG{o}{=} \PYG{n}{pd}\PYG{o}{.}\PYG{n}{DataFrame}\PYG{p}{(}\PYG{p}{)}

\PYG{k}{for} \PYG{n}{code}\PYG{p}{,} \PYG{n}{name}\PYG{p}{,} \PYG{n}{sector} \PYG{o+ow}{in} \PYG{n+nb}{zip}\PYG{p}{(}\PYG{n}{kosdaq\PYGZus{}list}\PYG{p}{[}\PYG{l+s+s1}{\PYGZsq{}}\PYG{l+s+s1}{code}\PYG{l+s+s1}{\PYGZsq{}}\PYG{p}{]}\PYG{p}{,} \PYG{n}{kosdaq\PYGZus{}list}\PYG{p}{[}\PYG{l+s+s1}{\PYGZsq{}}\PYG{l+s+s1}{name}\PYG{l+s+s1}{\PYGZsq{}}\PYG{p}{]}\PYG{p}{,} \PYG{n}{kosdaq\PYGZus{}list}\PYG{p}{[}\PYG{l+s+s1}{\PYGZsq{}}\PYG{l+s+s1}{sector}\PYG{l+s+s1}{\PYGZsq{}}\PYG{p}{]}\PYG{p}{)}\PYG{p}{:}

    \PYG{n}{data} \PYG{o}{=} \PYG{n}{return\PYGZus{}all}\PYG{p}{[}\PYG{n}{return\PYGZus{}all}\PYG{p}{[}\PYG{l+s+s1}{\PYGZsq{}}\PYG{l+s+s1}{code}\PYG{l+s+s1}{\PYGZsq{}}\PYG{p}{]}\PYG{o}{==}\PYG{n}{code}\PYG{p}{]}\PYG{o}{.}\PYG{n}{sort\PYGZus{}index}\PYG{p}{(}\PYG{p}{)}\PYG{o}{.}\PYG{n}{copy}\PYG{p}{(}\PYG{p}{)}    
    
    \PYG{c+c1}{\PYGZsh{} 가격변동성이 크고, 거래량이 몰린 종목이 주가가 상승한다}
    \PYG{n}{data}\PYG{p}{[}\PYG{l+s+s1}{\PYGZsq{}}\PYG{l+s+s1}{price\PYGZus{}mean}\PYG{l+s+s1}{\PYGZsq{}}\PYG{p}{]} \PYG{o}{=} \PYG{n}{data}\PYG{p}{[}\PYG{l+s+s1}{\PYGZsq{}}\PYG{l+s+s1}{close}\PYG{l+s+s1}{\PYGZsq{}}\PYG{p}{]}\PYG{o}{.}\PYG{n}{rolling}\PYG{p}{(}\PYG{l+m+mi}{20}\PYG{p}{)}\PYG{o}{.}\PYG{n}{mean}\PYG{p}{(}\PYG{p}{)}
    \PYG{n}{data}\PYG{p}{[}\PYG{l+s+s1}{\PYGZsq{}}\PYG{l+s+s1}{price\PYGZus{}std}\PYG{l+s+s1}{\PYGZsq{}}\PYG{p}{]} \PYG{o}{=} \PYG{n}{data}\PYG{p}{[}\PYG{l+s+s1}{\PYGZsq{}}\PYG{l+s+s1}{close}\PYG{l+s+s1}{\PYGZsq{}}\PYG{p}{]}\PYG{o}{.}\PYG{n}{rolling}\PYG{p}{(}\PYG{l+m+mi}{20}\PYG{p}{)}\PYG{o}{.}\PYG{n}{std}\PYG{p}{(}\PYG{n}{ddof}\PYG{o}{=}\PYG{l+m+mi}{0}\PYG{p}{)}
    \PYG{n}{data}\PYG{p}{[}\PYG{l+s+s1}{\PYGZsq{}}\PYG{l+s+s1}{price\PYGZus{}z}\PYG{l+s+s1}{\PYGZsq{}}\PYG{p}{]} \PYG{o}{=} \PYG{p}{(}\PYG{n}{data}\PYG{p}{[}\PYG{l+s+s1}{\PYGZsq{}}\PYG{l+s+s1}{close}\PYG{l+s+s1}{\PYGZsq{}}\PYG{p}{]} \PYG{o}{\PYGZhy{}} \PYG{n}{data}\PYG{p}{[}\PYG{l+s+s1}{\PYGZsq{}}\PYG{l+s+s1}{price\PYGZus{}mean}\PYG{l+s+s1}{\PYGZsq{}}\PYG{p}{]}\PYG{p}{)}\PYG{o}{/}\PYG{n}{data}\PYG{p}{[}\PYG{l+s+s1}{\PYGZsq{}}\PYG{l+s+s1}{price\PYGZus{}std}\PYG{l+s+s1}{\PYGZsq{}}\PYG{p}{]}    
    \PYG{n}{data}\PYG{p}{[}\PYG{l+s+s1}{\PYGZsq{}}\PYG{l+s+s1}{volume\PYGZus{}mean}\PYG{l+s+s1}{\PYGZsq{}}\PYG{p}{]} \PYG{o}{=} \PYG{n}{data}\PYG{p}{[}\PYG{l+s+s1}{\PYGZsq{}}\PYG{l+s+s1}{volume}\PYG{l+s+s1}{\PYGZsq{}}\PYG{p}{]}\PYG{o}{.}\PYG{n}{rolling}\PYG{p}{(}\PYG{l+m+mi}{20}\PYG{p}{)}\PYG{o}{.}\PYG{n}{mean}\PYG{p}{(}\PYG{p}{)}
    \PYG{n}{data}\PYG{p}{[}\PYG{l+s+s1}{\PYGZsq{}}\PYG{l+s+s1}{volume\PYGZus{}std}\PYG{l+s+s1}{\PYGZsq{}}\PYG{p}{]} \PYG{o}{=} \PYG{n}{data}\PYG{p}{[}\PYG{l+s+s1}{\PYGZsq{}}\PYG{l+s+s1}{volume}\PYG{l+s+s1}{\PYGZsq{}}\PYG{p}{]}\PYG{o}{.}\PYG{n}{rolling}\PYG{p}{(}\PYG{l+m+mi}{20}\PYG{p}{)}\PYG{o}{.}\PYG{n}{std}\PYG{p}{(}\PYG{n}{ddof}\PYG{o}{=}\PYG{l+m+mi}{0}\PYG{p}{)}
    \PYG{n}{data}\PYG{p}{[}\PYG{l+s+s1}{\PYGZsq{}}\PYG{l+s+s1}{volume\PYGZus{}z}\PYG{l+s+s1}{\PYGZsq{}}\PYG{p}{]} \PYG{o}{=} \PYG{p}{(}\PYG{n}{data}\PYG{p}{[}\PYG{l+s+s1}{\PYGZsq{}}\PYG{l+s+s1}{volume}\PYG{l+s+s1}{\PYGZsq{}}\PYG{p}{]} \PYG{o}{\PYGZhy{}} \PYG{n}{data}\PYG{p}{[}\PYG{l+s+s1}{\PYGZsq{}}\PYG{l+s+s1}{volume\PYGZus{}mean}\PYG{l+s+s1}{\PYGZsq{}}\PYG{p}{]}\PYG{p}{)}\PYG{o}{/}\PYG{n}{data}\PYG{p}{[}\PYG{l+s+s1}{\PYGZsq{}}\PYG{l+s+s1}{volume\PYGZus{}std}\PYG{l+s+s1}{\PYGZsq{}}\PYG{p}{]}
    
    \PYG{c+c1}{\PYGZsh{} 위꼬리가 긴 양봉이 자주발생한다.}
    \PYG{n}{data}\PYG{p}{[}\PYG{l+s+s1}{\PYGZsq{}}\PYG{l+s+s1}{positive\PYGZus{}candle}\PYG{l+s+s1}{\PYGZsq{}}\PYG{p}{]} \PYG{o}{=} \PYG{p}{(}\PYG{n}{data}\PYG{p}{[}\PYG{l+s+s1}{\PYGZsq{}}\PYG{l+s+s1}{close}\PYG{l+s+s1}{\PYGZsq{}}\PYG{p}{]} \PYG{o}{\PYGZgt{}} \PYG{n}{data}\PYG{p}{[}\PYG{l+s+s1}{\PYGZsq{}}\PYG{l+s+s1}{open}\PYG{l+s+s1}{\PYGZsq{}}\PYG{p}{]}\PYG{p}{)}\PYG{o}{.}\PYG{n}{astype}\PYG{p}{(}\PYG{n+nb}{int}\PYG{p}{)} \PYG{c+c1}{\PYGZsh{} 양봉}
    \PYG{n}{data}\PYG{p}{[}\PYG{l+s+s1}{\PYGZsq{}}\PYG{l+s+s1}{high/close}\PYG{l+s+s1}{\PYGZsq{}}\PYG{p}{]} \PYG{o}{=} \PYG{p}{(}\PYG{n}{data}\PYG{p}{[}\PYG{l+s+s1}{\PYGZsq{}}\PYG{l+s+s1}{positive\PYGZus{}candle}\PYG{l+s+s1}{\PYGZsq{}}\PYG{p}{]}\PYG{o}{==}\PYG{l+m+mi}{1}\PYG{p}{)}\PYG{o}{*}\PYG{p}{(}\PYG{n}{data}\PYG{p}{[}\PYG{l+s+s1}{\PYGZsq{}}\PYG{l+s+s1}{high}\PYG{l+s+s1}{\PYGZsq{}}\PYG{p}{]}\PYG{o}{/}\PYG{n}{data}\PYG{p}{[}\PYG{l+s+s1}{\PYGZsq{}}\PYG{l+s+s1}{close}\PYG{l+s+s1}{\PYGZsq{}}\PYG{p}{]} \PYG{o}{\PYGZgt{}} \PYG{l+m+mf}{1.1}\PYG{p}{)}\PYG{o}{.}\PYG{n}{astype}\PYG{p}{(}\PYG{n+nb}{int}\PYG{p}{)} \PYG{c+c1}{\PYGZsh{} 양봉이면서 고가가 종가보다 높게 위치}
    \PYG{n}{data}\PYG{p}{[}\PYG{l+s+s1}{\PYGZsq{}}\PYG{l+s+s1}{num\PYGZus{}high/close}\PYG{l+s+s1}{\PYGZsq{}}\PYG{p}{]} \PYG{o}{=}  \PYG{n}{data}\PYG{p}{[}\PYG{l+s+s1}{\PYGZsq{}}\PYG{l+s+s1}{high/close}\PYG{l+s+s1}{\PYGZsq{}}\PYG{p}{]}\PYG{o}{.}\PYG{n}{rolling}\PYG{p}{(}\PYG{l+m+mi}{20}\PYG{p}{)}\PYG{o}{.}\PYG{n}{sum}\PYG{p}{(}\PYG{p}{)}
    \PYG{n}{data}\PYG{p}{[}\PYG{l+s+s1}{\PYGZsq{}}\PYG{l+s+s1}{long\PYGZus{}candle}\PYG{l+s+s1}{\PYGZsq{}}\PYG{p}{]} \PYG{o}{=} \PYG{p}{(}\PYG{n}{data}\PYG{p}{[}\PYG{l+s+s1}{\PYGZsq{}}\PYG{l+s+s1}{positive\PYGZus{}candle}\PYG{l+s+s1}{\PYGZsq{}}\PYG{p}{]}\PYG{o}{==}\PYG{l+m+mi}{1}\PYG{p}{)}\PYG{o}{*}\PYG{p}{(}\PYG{n}{data}\PYG{p}{[}\PYG{l+s+s1}{\PYGZsq{}}\PYG{l+s+s1}{high}\PYG{l+s+s1}{\PYGZsq{}}\PYG{p}{]}\PYG{o}{==}\PYG{n}{data}\PYG{p}{[}\PYG{l+s+s1}{\PYGZsq{}}\PYG{l+s+s1}{close}\PYG{l+s+s1}{\PYGZsq{}}\PYG{p}{]}\PYG{p}{)}\PYG{o}{*}\PYGZbs{}
    \PYG{p}{(}\PYG{n}{data}\PYG{p}{[}\PYG{l+s+s1}{\PYGZsq{}}\PYG{l+s+s1}{low}\PYG{l+s+s1}{\PYGZsq{}}\PYG{p}{]}\PYG{o}{==}\PYG{n}{data}\PYG{p}{[}\PYG{l+s+s1}{\PYGZsq{}}\PYG{l+s+s1}{open}\PYG{l+s+s1}{\PYGZsq{}}\PYG{p}{]}\PYG{p}{)}\PYG{o}{*}\PYG{p}{(}\PYG{n}{data}\PYG{p}{[}\PYG{l+s+s1}{\PYGZsq{}}\PYG{l+s+s1}{close}\PYG{l+s+s1}{\PYGZsq{}}\PYG{p}{]}\PYG{o}{/}\PYG{n}{data}\PYG{p}{[}\PYG{l+s+s1}{\PYGZsq{}}\PYG{l+s+s1}{open}\PYG{l+s+s1}{\PYGZsq{}}\PYG{p}{]} \PYG{o}{\PYGZgt{}} \PYG{l+m+mf}{1.2}\PYG{p}{)}\PYG{o}{.}\PYG{n}{astype}\PYG{p}{(}\PYG{n+nb}{int}\PYG{p}{)} \PYG{c+c1}{\PYGZsh{} 장대 양봉을 데이터로 표현}
    \PYG{n}{data}\PYG{p}{[}\PYG{l+s+s1}{\PYGZsq{}}\PYG{l+s+s1}{num\PYGZus{}long}\PYG{l+s+s1}{\PYGZsq{}}\PYG{p}{]} \PYG{o}{=}  \PYG{n}{data}\PYG{p}{[}\PYG{l+s+s1}{\PYGZsq{}}\PYG{l+s+s1}{long\PYGZus{}candle}\PYG{l+s+s1}{\PYGZsq{}}\PYG{p}{]}\PYG{o}{.}\PYG{n}{rolling}\PYG{p}{(}\PYG{l+m+mi}{60}\PYG{p}{)}\PYG{o}{.}\PYG{n}{sum}\PYG{p}{(}\PYG{p}{)} \PYG{c+c1}{\PYGZsh{} 지난 60 일 동안 장대양봉의 갯 수}
    
    
     \PYG{c+c1}{\PYGZsh{} 거래량이 종좀 터지며 매집의 흔적을 보인다   }
    \PYG{n}{data}\PYG{p}{[}\PYG{l+s+s1}{\PYGZsq{}}\PYG{l+s+s1}{volume\PYGZus{}mean}\PYG{l+s+s1}{\PYGZsq{}}\PYG{p}{]} \PYG{o}{=} \PYG{n}{data}\PYG{p}{[}\PYG{l+s+s1}{\PYGZsq{}}\PYG{l+s+s1}{volume}\PYG{l+s+s1}{\PYGZsq{}}\PYG{p}{]}\PYG{o}{.}\PYG{n}{rolling}\PYG{p}{(}\PYG{l+m+mi}{60}\PYG{p}{)}\PYG{o}{.}\PYG{n}{mean}\PYG{p}{(}\PYG{p}{)}
    \PYG{n}{data}\PYG{p}{[}\PYG{l+s+s1}{\PYGZsq{}}\PYG{l+s+s1}{volume\PYGZus{}std}\PYG{l+s+s1}{\PYGZsq{}}\PYG{p}{]} \PYG{o}{=} \PYG{n}{data}\PYG{p}{[}\PYG{l+s+s1}{\PYGZsq{}}\PYG{l+s+s1}{volume}\PYG{l+s+s1}{\PYGZsq{}}\PYG{p}{]}\PYG{o}{.}\PYG{n}{rolling}\PYG{p}{(}\PYG{l+m+mi}{60}\PYG{p}{)}\PYG{o}{.}\PYG{n}{std}\PYG{p}{(}\PYG{p}{)}
    \PYG{n}{data}\PYG{p}{[}\PYG{l+s+s1}{\PYGZsq{}}\PYG{l+s+s1}{volume\PYGZus{}z}\PYG{l+s+s1}{\PYGZsq{}}\PYG{p}{]} \PYG{o}{=} \PYG{p}{(}\PYG{n}{data}\PYG{p}{[}\PYG{l+s+s1}{\PYGZsq{}}\PYG{l+s+s1}{volume}\PYG{l+s+s1}{\PYGZsq{}}\PYG{p}{]} \PYG{o}{\PYGZhy{}} \PYG{n}{data}\PYG{p}{[}\PYG{l+s+s1}{\PYGZsq{}}\PYG{l+s+s1}{volume\PYGZus{}mean}\PYG{l+s+s1}{\PYGZsq{}}\PYG{p}{]}\PYG{p}{)}\PYG{o}{/}\PYG{n}{data}\PYG{p}{[}\PYG{l+s+s1}{\PYGZsq{}}\PYG{l+s+s1}{volume\PYGZus{}std}\PYG{l+s+s1}{\PYGZsq{}}\PYG{p}{]} \PYG{c+c1}{\PYGZsh{} 거래량은 종목과 주가에 따라 다르기 떄문에 표준화한 값이 필요함}
    \PYG{n}{data}\PYG{p}{[}\PYG{l+s+s1}{\PYGZsq{}}\PYG{l+s+s1}{z\PYGZgt{}1.96}\PYG{l+s+s1}{\PYGZsq{}}\PYG{p}{]} \PYG{o}{=} \PYG{p}{(}\PYG{n}{data}\PYG{p}{[}\PYG{l+s+s1}{\PYGZsq{}}\PYG{l+s+s1}{close}\PYG{l+s+s1}{\PYGZsq{}}\PYG{p}{]} \PYG{o}{\PYGZgt{}} \PYG{n}{data}\PYG{p}{[}\PYG{l+s+s1}{\PYGZsq{}}\PYG{l+s+s1}{open}\PYG{l+s+s1}{\PYGZsq{}}\PYG{p}{]}\PYG{p}{)}\PYG{o}{*}\PYG{p}{(}\PYG{n}{data}\PYG{p}{[}\PYG{l+s+s1}{\PYGZsq{}}\PYG{l+s+s1}{volume\PYGZus{}z}\PYG{l+s+s1}{\PYGZsq{}}\PYG{p}{]} \PYG{o}{\PYGZgt{}} \PYG{l+m+mf}{1.65}\PYG{p}{)}\PYG{o}{.}\PYG{n}{astype}\PYG{p}{(}\PYG{n+nb}{int}\PYG{p}{)} \PYG{c+c1}{\PYGZsh{} 양봉이면서 거래량이 90\PYGZpc{}신뢰구간을 벗어난 날}
    \PYG{n}{data}\PYG{p}{[}\PYG{l+s+s1}{\PYGZsq{}}\PYG{l+s+s1}{num\PYGZus{}z\PYGZgt{}1.96}\PYG{l+s+s1}{\PYGZsq{}}\PYG{p}{]} \PYG{o}{=}  \PYG{n}{data}\PYG{p}{[}\PYG{l+s+s1}{\PYGZsq{}}\PYG{l+s+s1}{z\PYGZgt{}1.96}\PYG{l+s+s1}{\PYGZsq{}}\PYG{p}{]}\PYG{o}{.}\PYG{n}{rolling}\PYG{p}{(}\PYG{l+m+mi}{60}\PYG{p}{)}\PYG{o}{.}\PYG{n}{sum}\PYG{p}{(}\PYG{p}{)}  \PYG{c+c1}{\PYGZsh{} 양봉이면서 거래량이 90\PYGZpc{} 신뢰구간을 벗어난 날을 카운트}
    
    \PYG{c+c1}{\PYGZsh{} 주가지수보다 더 좋은 수익율을 보여준다}
    \PYG{n}{data}\PYG{p}{[}\PYG{l+s+s1}{\PYGZsq{}}\PYG{l+s+s1}{num\PYGZus{}win\PYGZus{}market}\PYG{l+s+s1}{\PYGZsq{}}\PYG{p}{]} \PYG{o}{=} \PYG{n}{data}\PYG{p}{[}\PYG{l+s+s1}{\PYGZsq{}}\PYG{l+s+s1}{win\PYGZus{}market}\PYG{l+s+s1}{\PYGZsq{}}\PYG{p}{]}\PYG{o}{.}\PYG{n}{rolling}\PYG{p}{(}\PYG{l+m+mi}{60}\PYG{p}{)}\PYG{o}{.}\PYG{n}{sum}\PYG{p}{(}\PYG{p}{)} \PYG{c+c1}{\PYGZsh{} 주가지수 수익율이 1 보다 작을 때, 종목 수익율이 1 보다 큰 날 수}
    \PYG{n}{data}\PYG{p}{[}\PYG{l+s+s1}{\PYGZsq{}}\PYG{l+s+s1}{pct\PYGZus{}win\PYGZus{}market}\PYG{l+s+s1}{\PYGZsq{}}\PYG{p}{]} \PYG{o}{=} \PYG{p}{(}\PYG{n}{data}\PYG{p}{[}\PYG{l+s+s1}{\PYGZsq{}}\PYG{l+s+s1}{return}\PYG{l+s+s1}{\PYGZsq{}}\PYG{p}{]}\PYG{o}{/}\PYG{n}{data}\PYG{p}{[}\PYG{l+s+s1}{\PYGZsq{}}\PYG{l+s+s1}{kosdaq\PYGZus{}return}\PYG{l+s+s1}{\PYGZsq{}}\PYG{p}{]}\PYG{p}{)}\PYG{o}{.}\PYG{n}{rolling}\PYG{p}{(}\PYG{l+m+mi}{60}\PYG{p}{)}\PYG{o}{.}\PYG{n}{mean}\PYG{p}{(}\PYG{p}{)} \PYG{c+c1}{\PYGZsh{} 주가지수 수익율 대비 종목 수익율}
    
    
    \PYG{c+c1}{\PYGZsh{} 동종업체 수익률보다 더 좋은 수익율을 보여준다.           }
    \PYG{n}{data}\PYG{p}{[}\PYG{l+s+s1}{\PYGZsq{}}\PYG{l+s+s1}{return\PYGZus{}mean}\PYG{l+s+s1}{\PYGZsq{}}\PYG{p}{]} \PYG{o}{=} \PYG{n}{data}\PYG{p}{[}\PYG{l+s+s1}{\PYGZsq{}}\PYG{l+s+s1}{return}\PYG{l+s+s1}{\PYGZsq{}}\PYG{p}{]}\PYG{o}{.}\PYG{n}{rolling}\PYG{p}{(}\PYG{l+m+mi}{60}\PYG{p}{)}\PYG{o}{.}\PYG{n}{mean}\PYG{p}{(}\PYG{p}{)} \PYG{c+c1}{\PYGZsh{} 종목별 최근 60 일 수익율의 평균}
    \PYG{n}{data}\PYG{p}{[}\PYG{l+s+s1}{\PYGZsq{}}\PYG{l+s+s1}{sector}\PYG{l+s+s1}{\PYGZsq{}}\PYG{p}{]} \PYG{o}{=} \PYG{n}{sector}    
    \PYG{n}{data}\PYG{p}{[}\PYG{l+s+s1}{\PYGZsq{}}\PYG{l+s+s1}{name}\PYG{l+s+s1}{\PYGZsq{}}\PYG{p}{]} \PYG{o}{=} \PYG{n}{name}
     
    \PYG{n}{data} \PYG{o}{=} \PYG{n}{data}\PYG{p}{[}\PYG{p}{(}\PYG{n}{data}\PYG{p}{[}\PYG{l+s+s1}{\PYGZsq{}}\PYG{l+s+s1}{price\PYGZus{}std}\PYG{l+s+s1}{\PYGZsq{}}\PYG{p}{]}\PYG{o}{!=}\PYG{l+m+mi}{0}\PYG{p}{)} \PYG{o}{\PYGZam{}} \PYG{p}{(}\PYG{n}{data}\PYG{p}{[}\PYG{l+s+s1}{\PYGZsq{}}\PYG{l+s+s1}{volume\PYGZus{}std}\PYG{l+s+s1}{\PYGZsq{}}\PYG{p}{]}\PYG{o}{!=}\PYG{l+m+mi}{0}\PYG{p}{)}\PYG{p}{]}    

    \PYG{n}{model\PYGZus{}inputs} \PYG{o}{=} \PYG{n}{pd}\PYG{o}{.}\PYG{n}{concat}\PYG{p}{(}\PYG{p}{[}\PYG{n}{data}\PYG{p}{,} \PYG{n}{model\PYGZus{}inputs}\PYG{p}{]}\PYG{p}{,} \PYG{n}{axis}\PYG{o}{=}\PYG{l+m+mi}{0}\PYG{p}{)}

\PYG{n}{model\PYGZus{}inputs}\PYG{p}{[}\PYG{l+s+s1}{\PYGZsq{}}\PYG{l+s+s1}{sector\PYGZus{}return}\PYG{l+s+s1}{\PYGZsq{}}\PYG{p}{]} \PYG{o}{=} \PYG{n}{model\PYGZus{}inputs}\PYG{o}{.}\PYG{n}{groupby}\PYG{p}{(}\PYG{p}{[}\PYG{l+s+s1}{\PYGZsq{}}\PYG{l+s+s1}{sector}\PYG{l+s+s1}{\PYGZsq{}}\PYG{p}{,} \PYG{n}{model\PYGZus{}inputs}\PYG{o}{.}\PYG{n}{index}\PYG{p}{]}\PYG{p}{)}\PYG{p}{[}\PYG{l+s+s1}{\PYGZsq{}}\PYG{l+s+s1}{return}\PYG{l+s+s1}{\PYGZsq{}}\PYG{p}{]}\PYG{o}{.}\PYG{n}{transform}\PYG{p}{(}\PYG{k}{lambda} \PYG{n}{x}\PYG{p}{:} \PYG{n}{x}\PYG{o}{.}\PYG{n}{mean}\PYG{p}{(}\PYG{p}{)}\PYG{p}{)} \PYG{c+c1}{\PYGZsh{} 섹터의 평균 수익율 계산}
\PYG{n}{model\PYGZus{}inputs}\PYG{p}{[}\PYG{l+s+s1}{\PYGZsq{}}\PYG{l+s+s1}{return over sector}\PYG{l+s+s1}{\PYGZsq{}}\PYG{p}{]} \PYG{o}{=} \PYG{p}{(}\PYG{n}{model\PYGZus{}inputs}\PYG{p}{[}\PYG{l+s+s1}{\PYGZsq{}}\PYG{l+s+s1}{return}\PYG{l+s+s1}{\PYGZsq{}}\PYG{p}{]}\PYG{o}{/}\PYG{n}{model\PYGZus{}inputs}\PYG{p}{[}\PYG{l+s+s1}{\PYGZsq{}}\PYG{l+s+s1}{sector\PYGZus{}return}\PYG{l+s+s1}{\PYGZsq{}}\PYG{p}{]}\PYG{p}{)} \PYG{c+c1}{\PYGZsh{} 섹터 평균 수익률 대비 종목 수익률 계산}
\PYG{n}{model\PYGZus{}inputs}\PYG{o}{.}\PYG{n}{dropna}\PYG{p}{(}\PYG{n}{inplace}\PYG{o}{=}\PYG{k+kc}{True}\PYG{p}{)} \PYG{c+c1}{\PYGZsh{} Missing 값 있는 행 모두 제거}

\PYG{n}{model\PYGZus{}inputs}\PYG{o}{.}\PYG{n}{to\PYGZus{}pickle}\PYG{p}{(}\PYG{l+s+s1}{\PYGZsq{}}\PYG{l+s+s1}{model\PYGZus{}inputs.pkl}\PYG{l+s+s1}{\PYGZsq{}}\PYG{p}{)}
\end{sphinxVerbatim}

\end{sphinxuseclass}\end{sphinxVerbatimInput}

\end{sphinxuseclass}
\sphinxAtStartPar
 모델에 입력할 변수를 생성하고 X 에 담습니다.

\begin{sphinxuseclass}{cell}\begin{sphinxVerbatimInput}

\begin{sphinxuseclass}{cell_input}
\begin{sphinxVerbatim}[commandchars=\\\{\}]
\PYG{c+c1}{\PYGZsh{} 최종 피처만으로 구성}
\PYG{n}{model\PYGZus{}inputs} \PYG{o}{=} \PYG{n}{pd}\PYG{o}{.}\PYG{n}{read\PYGZus{}pickle}\PYG{p}{(}\PYG{l+s+s1}{\PYGZsq{}}\PYG{l+s+s1}{model\PYGZus{}inputs.pkl}\PYG{l+s+s1}{\PYGZsq{}}\PYG{p}{)}
\PYG{n}{feature\PYGZus{}list} \PYG{o}{=} \PYG{p}{[}\PYG{l+s+s1}{\PYGZsq{}}\PYG{l+s+s1}{price\PYGZus{}z}\PYG{l+s+s1}{\PYGZsq{}}\PYG{p}{,}\PYG{l+s+s1}{\PYGZsq{}}\PYG{l+s+s1}{volume\PYGZus{}z}\PYG{l+s+s1}{\PYGZsq{}}\PYG{p}{,}\PYG{l+s+s1}{\PYGZsq{}}\PYG{l+s+s1}{num\PYGZus{}high/close}\PYG{l+s+s1}{\PYGZsq{}}\PYG{p}{,}\PYG{l+s+s1}{\PYGZsq{}}\PYG{l+s+s1}{num\PYGZus{}win\PYGZus{}market}\PYG{l+s+s1}{\PYGZsq{}}\PYG{p}{,}\PYG{l+s+s1}{\PYGZsq{}}\PYG{l+s+s1}{pct\PYGZus{}win\PYGZus{}market}\PYG{l+s+s1}{\PYGZsq{}}\PYG{p}{,}\PYG{l+s+s1}{\PYGZsq{}}\PYG{l+s+s1}{return over sector}\PYG{l+s+s1}{\PYGZsq{}}\PYG{p}{]}

\PYG{n}{X} \PYG{o}{=} \PYG{n}{model\PYGZus{}inputs}\PYG{o}{.}\PYG{n}{loc}\PYG{p}{[}\PYG{n}{today\PYGZus{}dt}\PYG{p}{]}\PYG{p}{[}\PYG{p}{[}\PYG{l+s+s1}{\PYGZsq{}}\PYG{l+s+s1}{code}\PYG{l+s+s1}{\PYGZsq{}}\PYG{p}{,}\PYG{l+s+s1}{\PYGZsq{}}\PYG{l+s+s1}{name}\PYG{l+s+s1}{\PYGZsq{}}\PYG{p}{,}\PYG{l+s+s1}{\PYGZsq{}}\PYG{l+s+s1}{return}\PYG{l+s+s1}{\PYGZsq{}}\PYG{p}{]} \PYG{o}{+} \PYG{n}{feature\PYGZus{}list}\PYG{p}{]}\PYG{o}{.}\PYG{n}{set\PYGZus{}index}\PYG{p}{(}\PYG{l+s+s1}{\PYGZsq{}}\PYG{l+s+s1}{code}\PYG{l+s+s1}{\PYGZsq{}}\PYG{p}{)} \PYG{c+c1}{\PYGZsh{} 오늘 날짜 2022년 4월 1일 데이터만}
\PYG{n}{X}\PYG{o}{.}\PYG{n}{head}\PYG{p}{(}\PYG{p}{)}\PYG{o}{.}\PYG{n}{style}\PYG{o}{.}\PYG{n}{set\PYGZus{}table\PYGZus{}attributes}\PYG{p}{(}\PYG{l+s+s1}{\PYGZsq{}}\PYG{l+s+s1}{style=}\PYG{l+s+s1}{\PYGZdq{}}\PYG{l+s+s1}{font\PYGZhy{}size: 12px}\PYG{l+s+s1}{\PYGZdq{}}\PYG{l+s+s1}{\PYGZsq{}}\PYG{p}{)}\PYG{o}{.}\PYG{n}{format}\PYG{p}{(}\PYG{n}{precision}\PYG{o}{=}\PYG{l+m+mi}{3}\PYG{p}{)}
\end{sphinxVerbatim}

\end{sphinxuseclass}\end{sphinxVerbatimInput}
\begin{sphinxVerbatimOutput}

\begin{sphinxuseclass}{cell_output}
\begin{sphinxVerbatim}[commandchars=\\\{\}]
\PYGZlt{}pandas.io.formats.style.Styler at 0x202979d5490\PYGZgt{}
\end{sphinxVerbatim}

\end{sphinxuseclass}\end{sphinxVerbatimOutput}

\end{sphinxuseclass}
\sphinxAtStartPar
 저장한 GAM 모델을 불러 읽고, 입력변수를 넣어 예측값을 생성합니다. 입력변수의 순서는 모델에 사용한 입력변수와 동일해야 합니다. X 라는 데이터 프레임에 예측값 yhat 이 추가되었습니다.

\begin{sphinxuseclass}{cell}\begin{sphinxVerbatimInput}

\begin{sphinxuseclass}{cell_input}
\begin{sphinxVerbatim}[commandchars=\\\{\}]
\PYG{k+kn}{import} \PYG{n+nn}{pickle}
\PYG{k}{with} \PYG{n+nb}{open}\PYG{p}{(}\PYG{l+s+s2}{\PYGZdq{}}\PYG{l+s+s2}{gam.pkl}\PYG{l+s+s2}{\PYGZdq{}}\PYG{p}{,} \PYG{l+s+s2}{\PYGZdq{}}\PYG{l+s+s2}{rb}\PYG{l+s+s2}{\PYGZdq{}}\PYG{p}{)} \PYG{k}{as} \PYG{n}{file}\PYG{p}{:}
    \PYG{n}{gam} \PYG{o}{=} \PYG{n}{pickle}\PYG{o}{.}\PYG{n}{load}\PYG{p}{(}\PYG{n}{file}\PYG{p}{)}     
    
\PYG{n}{yhat} \PYG{o}{=} \PYG{n}{gam}\PYG{o}{.}\PYG{n}{predict\PYGZus{}proba}\PYG{p}{(}\PYG{n}{X}\PYG{p}{[}\PYG{n}{feature\PYGZus{}list}\PYG{p}{]}\PYG{p}{)}
\PYG{n}{X}\PYG{p}{[}\PYG{l+s+s1}{\PYGZsq{}}\PYG{l+s+s1}{yhat}\PYG{l+s+s1}{\PYGZsq{}}\PYG{p}{]} \PYG{o}{=} \PYG{n}{yhat}
\PYG{n}{X}\PYG{o}{.}\PYG{n}{head}\PYG{p}{(}\PYG{p}{)}\PYG{o}{.}\PYG{n}{style}\PYG{o}{.}\PYG{n}{set\PYGZus{}table\PYGZus{}attributes}\PYG{p}{(}\PYG{l+s+s1}{\PYGZsq{}}\PYG{l+s+s1}{style=}\PYG{l+s+s1}{\PYGZdq{}}\PYG{l+s+s1}{font\PYGZhy{}size: 12px}\PYG{l+s+s1}{\PYGZdq{}}\PYG{l+s+s1}{\PYGZsq{}}\PYG{p}{)}\PYG{o}{.}\PYG{n}{format}\PYG{p}{(}\PYG{n}{precision}\PYG{o}{=}\PYG{l+m+mi}{3}\PYG{p}{)}
\end{sphinxVerbatim}

\end{sphinxuseclass}\end{sphinxVerbatimInput}
\begin{sphinxVerbatimOutput}

\begin{sphinxuseclass}{cell_output}
\begin{sphinxVerbatim}[commandchars=\\\{\}]
\PYGZlt{}pandas.io.formats.style.Styler at 0x202979af070\PYGZgt{}
\end{sphinxVerbatim}

\end{sphinxuseclass}\end{sphinxVerbatimOutput}

\end{sphinxuseclass}
\sphinxAtStartPar
 어떤 종목이 높은 스코어를 받았는지 궁금합니다. 스코어의 내림차순 정렬한 후 종목을 확인해 봅니다.

\begin{sphinxuseclass}{cell}\begin{sphinxVerbatimInput}

\begin{sphinxuseclass}{cell_input}
\begin{sphinxVerbatim}[commandchars=\\\{\}]
\PYG{n}{X}\PYG{o}{.}\PYG{n}{sort\PYGZus{}values}\PYG{p}{(}\PYG{n}{by}\PYG{o}{=}\PYG{l+s+s1}{\PYGZsq{}}\PYG{l+s+s1}{yhat}\PYG{l+s+s1}{\PYGZsq{}}\PYG{p}{,} \PYG{n}{ascending}\PYG{o}{=}\PYG{k+kc}{False}\PYG{p}{)}\PYG{o}{.}\PYG{n}{head}\PYG{p}{(}\PYG{l+m+mi}{5}\PYG{p}{)}\PYG{o}{.}\PYG{n}{style}\PYG{o}{.}\PYG{n}{set\PYGZus{}table\PYGZus{}attributes}\PYG{p}{(}\PYG{l+s+s1}{\PYGZsq{}}\PYG{l+s+s1}{style=}\PYG{l+s+s1}{\PYGZdq{}}\PYG{l+s+s1}{font\PYGZhy{}size: 12px}\PYG{l+s+s1}{\PYGZdq{}}\PYG{l+s+s1}{\PYGZsq{}}\PYG{p}{)}\PYG{o}{.}\PYG{n}{format}\PYG{p}{(}\PYG{n}{precision}\PYG{o}{=}\PYG{l+m+mi}{3}\PYG{p}{)}
\end{sphinxVerbatim}

\end{sphinxuseclass}\end{sphinxVerbatimInput}
\begin{sphinxVerbatimOutput}

\begin{sphinxuseclass}{cell_output}
\begin{sphinxVerbatim}[commandchars=\\\{\}]
\PYGZlt{}pandas.io.formats.style.Styler at 0x202979a7f40\PYGZgt{}
\end{sphinxVerbatim}

\end{sphinxuseclass}\end{sphinxVerbatimOutput}

\end{sphinxuseclass}
\sphinxAtStartPar
 그리고 필터링을 적용해서 최종 종목을 선정합니다. 최종적으로 5 개의 종목이 선정되었습니다. 우리는 4월 1일 이후에 주가 흐름을 알고 있습니다. 4월 2일이후 데이터를 추가하여 선택된 종목들이 유의미한지 점검해 보겠습니다.

\begin{sphinxuseclass}{cell}\begin{sphinxVerbatimInput}

\begin{sphinxuseclass}{cell_input}
\begin{sphinxVerbatim}[commandchars=\\\{\}]
\PYG{n}{tops} \PYG{o}{=} \PYG{n}{X}\PYG{p}{[}\PYG{n}{X}\PYG{p}{[}\PYG{l+s+s1}{\PYGZsq{}}\PYG{l+s+s1}{yhat}\PYG{l+s+s1}{\PYGZsq{}}\PYG{p}{]} \PYG{o}{\PYGZgt{}}\PYG{o}{=} \PYG{l+m+mf}{0.3}\PYG{p}{]}\PYG{o}{.}\PYG{n}{copy}\PYG{p}{(}\PYG{p}{)} \PYG{c+c1}{\PYGZsh{} 스코어 0.3 이상 종목만 }
\PYG{n+nb}{print}\PYG{p}{(}\PYG{n+nb}{len}\PYG{p}{(}\PYG{n}{tops}\PYG{p}{)}\PYG{p}{)}
\PYG{n}{select\PYGZus{}tops} \PYG{o}{=} \PYG{n}{tops}\PYG{p}{[}\PYG{p}{(}\PYG{n}{tops}\PYG{p}{[}\PYG{l+s+s1}{\PYGZsq{}}\PYG{l+s+s1}{return}\PYG{l+s+s1}{\PYGZsq{}}\PYG{p}{]} \PYG{o}{\PYGZgt{}} \PYG{l+m+mf}{1.03}\PYG{p}{)} \PYG{o}{\PYGZam{}} \PYG{p}{(}\PYG{n}{tops}\PYG{p}{[}\PYG{l+s+s1}{\PYGZsq{}}\PYG{l+s+s1}{price\PYGZus{}z}\PYG{l+s+s1}{\PYGZsq{}}\PYG{p}{]} \PYG{o}{\PYGZlt{}} \PYG{l+m+mi}{0}\PYG{p}{)}\PYG{p}{]}\PYG{p}{[}\PYG{p}{[}\PYG{l+s+s1}{\PYGZsq{}}\PYG{l+s+s1}{name}\PYG{l+s+s1}{\PYGZsq{}}\PYG{p}{,}\PYG{l+s+s1}{\PYGZsq{}}\PYG{l+s+s1}{return}\PYG{l+s+s1}{\PYGZsq{}}\PYG{p}{,}\PYG{l+s+s1}{\PYGZsq{}}\PYG{l+s+s1}{price\PYGZus{}z}\PYG{l+s+s1}{\PYGZsq{}}\PYG{p}{,}\PYG{l+s+s1}{\PYGZsq{}}\PYG{l+s+s1}{yhat}\PYG{l+s+s1}{\PYGZsq{}}\PYG{p}{,}\PYG{l+s+s1}{\PYGZsq{}}\PYG{l+s+s1}{return}\PYG{l+s+s1}{\PYGZsq{}}\PYG{p}{]}\PYG{p}{]}          
\PYG{n}{select\PYGZus{}tops}\PYG{o}{.}\PYG{n}{style}\PYG{o}{.}\PYG{n}{set\PYGZus{}table\PYGZus{}attributes}\PYG{p}{(}\PYG{l+s+s1}{\PYGZsq{}}\PYG{l+s+s1}{style=}\PYG{l+s+s1}{\PYGZdq{}}\PYG{l+s+s1}{font\PYGZhy{}size: 12px}\PYG{l+s+s1}{\PYGZdq{}}\PYG{l+s+s1}{\PYGZsq{}}\PYG{p}{)}\PYG{o}{.}\PYG{n}{format}\PYG{p}{(}\PYG{n}{precision}\PYG{o}{=}\PYG{l+m+mi}{3}\PYG{p}{)}
\end{sphinxVerbatim}

\end{sphinxuseclass}\end{sphinxVerbatimInput}
\begin{sphinxVerbatimOutput}

\begin{sphinxuseclass}{cell_output}
\begin{sphinxVerbatim}[commandchars=\\\{\}]
204
\end{sphinxVerbatim}

\begin{sphinxVerbatim}[commandchars=\\\{\}]
\PYGZlt{}pandas.io.formats.style.Styler at 0x202925e1e20\PYGZgt{}
\end{sphinxVerbatim}

\end{sphinxuseclass}\end{sphinxVerbatimOutput}

\end{sphinxuseclass}
\begin{sphinxuseclass}{cell}\begin{sphinxVerbatimInput}

\begin{sphinxuseclass}{cell_input}
\begin{sphinxVerbatim}[commandchars=\\\{\}]
\PYG{n}{outcome\PYGZus{}data} \PYG{o}{=} \PYG{n}{pd}\PYG{o}{.}\PYG{n}{DataFrame}\PYG{p}{(}\PYG{p}{)}

\PYG{n}{today\PYGZus{}dt} \PYG{o}{=} \PYG{l+s+s1}{\PYGZsq{}}\PYG{l+s+s1}{2022\PYGZhy{}04\PYGZhy{}01}\PYG{l+s+s1}{\PYGZsq{}}
\PYG{n}{end\PYGZus{}dt} \PYG{o}{=} \PYG{l+s+s1}{\PYGZsq{}}\PYG{l+s+s1}{2022\PYGZhy{}04\PYGZhy{}08}\PYG{l+s+s1}{\PYGZsq{}}

\PYG{k}{for} \PYG{n}{code} \PYG{o+ow}{in} \PYG{n+nb}{list}\PYG{p}{(}\PYG{n}{select\PYGZus{}tops}\PYG{o}{.}\PYG{n}{index}\PYG{p}{)}\PYG{p}{:}  \PYG{c+c1}{\PYGZsh{} 스코어가 생성된 모든 종목에서 대하여 반복}
    \PYG{n}{daily\PYGZus{}price} \PYG{o}{=} \PYG{n}{fdr}\PYG{o}{.}\PYG{n}{DataReader}\PYG{p}{(}\PYG{n}{code}\PYG{p}{,}  \PYG{n}{start} \PYG{o}{=} \PYG{n}{today\PYGZus{}dt}\PYG{p}{,} \PYG{n}{end} \PYG{o}{=} \PYG{n}{end\PYGZus{}dt}\PYG{p}{)} \PYG{c+c1}{\PYGZsh{} 종목, 일봉, 데이터 갯수}
    \PYG{n}{daily\PYGZus{}price}\PYG{p}{[}\PYG{l+s+s1}{\PYGZsq{}}\PYG{l+s+s1}{code}\PYG{l+s+s1}{\PYGZsq{}}\PYG{p}{]} \PYG{o}{=} \PYG{n}{code}
  
    
    \PYG{n}{daily\PYGZus{}price}\PYG{p}{[}\PYG{l+s+s1}{\PYGZsq{}}\PYG{l+s+s1}{close\PYGZus{}r1}\PYG{l+s+s1}{\PYGZsq{}}\PYG{p}{]} \PYG{o}{=} \PYG{n}{daily\PYGZus{}price}\PYG{p}{[}\PYG{l+s+s1}{\PYGZsq{}}\PYG{l+s+s1}{Close}\PYG{l+s+s1}{\PYGZsq{}}\PYG{p}{]}\PYG{o}{.}\PYG{n}{shift}\PYG{p}{(}\PYG{o}{\PYGZhy{}}\PYG{l+m+mi}{1}\PYG{p}{)}\PYG{o}{/}\PYG{n}{daily\PYGZus{}price}\PYG{p}{[}\PYG{l+s+s1}{\PYGZsq{}}\PYG{l+s+s1}{Close}\PYG{l+s+s1}{\PYGZsq{}}\PYG{p}{]}   \PYG{c+c1}{\PYGZsh{} 4월 1일 종가 매수한 후, 4월 4일 수익율}
    \PYG{n}{daily\PYGZus{}price}\PYG{p}{[}\PYG{l+s+s1}{\PYGZsq{}}\PYG{l+s+s1}{close\PYGZus{}r2}\PYG{l+s+s1}{\PYGZsq{}}\PYG{p}{]} \PYG{o}{=} \PYG{n}{daily\PYGZus{}price}\PYG{p}{[}\PYG{l+s+s1}{\PYGZsq{}}\PYG{l+s+s1}{Close}\PYG{l+s+s1}{\PYGZsq{}}\PYG{p}{]}\PYG{o}{.}\PYG{n}{shift}\PYG{p}{(}\PYG{o}{\PYGZhy{}}\PYG{l+m+mi}{2}\PYG{p}{)}\PYG{o}{/}\PYG{n}{daily\PYGZus{}price}\PYG{p}{[}\PYG{l+s+s1}{\PYGZsq{}}\PYG{l+s+s1}{Close}\PYG{l+s+s1}{\PYGZsq{}}\PYG{p}{]}   \PYG{c+c1}{\PYGZsh{} 4월 1일 종가 매수한 후, 4월 5일 수익율}
    \PYG{n}{daily\PYGZus{}price}\PYG{p}{[}\PYG{l+s+s1}{\PYGZsq{}}\PYG{l+s+s1}{close\PYGZus{}r3}\PYG{l+s+s1}{\PYGZsq{}}\PYG{p}{]} \PYG{o}{=} \PYG{n}{daily\PYGZus{}price}\PYG{p}{[}\PYG{l+s+s1}{\PYGZsq{}}\PYG{l+s+s1}{Close}\PYG{l+s+s1}{\PYGZsq{}}\PYG{p}{]}\PYG{o}{.}\PYG{n}{shift}\PYG{p}{(}\PYG{o}{\PYGZhy{}}\PYG{l+m+mi}{3}\PYG{p}{)}\PYG{o}{/}\PYG{n}{daily\PYGZus{}price}\PYG{p}{[}\PYG{l+s+s1}{\PYGZsq{}}\PYG{l+s+s1}{Close}\PYG{l+s+s1}{\PYGZsq{}}\PYG{p}{]}   \PYG{c+c1}{\PYGZsh{} 4월 1일 종가 매수한 후, 4월 6일 수익율}
    \PYG{n}{daily\PYGZus{}price}\PYG{p}{[}\PYG{l+s+s1}{\PYGZsq{}}\PYG{l+s+s1}{close\PYGZus{}r4}\PYG{l+s+s1}{\PYGZsq{}}\PYG{p}{]} \PYG{o}{=} \PYG{n}{daily\PYGZus{}price}\PYG{p}{[}\PYG{l+s+s1}{\PYGZsq{}}\PYG{l+s+s1}{Close}\PYG{l+s+s1}{\PYGZsq{}}\PYG{p}{]}\PYG{o}{.}\PYG{n}{shift}\PYG{p}{(}\PYG{o}{\PYGZhy{}}\PYG{l+m+mi}{4}\PYG{p}{)}\PYG{o}{/}\PYG{n}{daily\PYGZus{}price}\PYG{p}{[}\PYG{l+s+s1}{\PYGZsq{}}\PYG{l+s+s1}{Close}\PYG{l+s+s1}{\PYGZsq{}}\PYG{p}{]}   \PYG{c+c1}{\PYGZsh{} 4월 1일 종가 매수한 후, 4월 7일 수익율}
    \PYG{n}{daily\PYGZus{}price}\PYG{p}{[}\PYG{l+s+s1}{\PYGZsq{}}\PYG{l+s+s1}{close\PYGZus{}r5}\PYG{l+s+s1}{\PYGZsq{}}\PYG{p}{]} \PYG{o}{=} \PYG{n}{daily\PYGZus{}price}\PYG{p}{[}\PYG{l+s+s1}{\PYGZsq{}}\PYG{l+s+s1}{Close}\PYG{l+s+s1}{\PYGZsq{}}\PYG{p}{]}\PYG{o}{.}\PYG{n}{shift}\PYG{p}{(}\PYG{o}{\PYGZhy{}}\PYG{l+m+mi}{5}\PYG{p}{)}\PYG{o}{/}\PYG{n}{daily\PYGZus{}price}\PYG{p}{[}\PYG{l+s+s1}{\PYGZsq{}}\PYG{l+s+s1}{Close}\PYG{l+s+s1}{\PYGZsq{}}\PYG{p}{]}   \PYG{c+c1}{\PYGZsh{} 4월 1일 종가 매수한 후, 4월 8일 수익율}

    \PYG{n}{daily\PYGZus{}price}\PYG{p}{[}\PYG{l+s+s1}{\PYGZsq{}}\PYG{l+s+s1}{max\PYGZus{}close}\PYG{l+s+s1}{\PYGZsq{}}\PYG{p}{]} \PYG{o}{=} \PYG{n}{daily\PYGZus{}price}\PYG{p}{[}\PYG{p}{[}\PYG{l+s+s1}{\PYGZsq{}}\PYG{l+s+s1}{close\PYGZus{}r1}\PYG{l+s+s1}{\PYGZsq{}}\PYG{p}{,}\PYG{l+s+s1}{\PYGZsq{}}\PYG{l+s+s1}{close\PYGZus{}r2}\PYG{l+s+s1}{\PYGZsq{}}\PYG{p}{,}\PYG{l+s+s1}{\PYGZsq{}}\PYG{l+s+s1}{close\PYGZus{}r3}\PYG{l+s+s1}{\PYGZsq{}}\PYG{p}{,}\PYG{l+s+s1}{\PYGZsq{}}\PYG{l+s+s1}{close\PYGZus{}r4}\PYG{l+s+s1}{\PYGZsq{}}\PYG{p}{,}\PYG{l+s+s1}{\PYGZsq{}}\PYG{l+s+s1}{close\PYGZus{}r5}\PYG{l+s+s1}{\PYGZsq{}}\PYG{p}{]}\PYG{p}{]}\PYG{o}{.}\PYG{n}{max}\PYG{p}{(}\PYG{n}{axis}\PYG{o}{=}\PYG{l+m+mi}{1}\PYG{p}{)}
    \PYG{n}{daily\PYGZus{}price}\PYG{p}{[}\PYG{l+s+s1}{\PYGZsq{}}\PYG{l+s+s1}{mean\PYGZus{}close}\PYG{l+s+s1}{\PYGZsq{}}\PYG{p}{]} \PYG{o}{=} \PYG{n}{daily\PYGZus{}price}\PYG{p}{[}\PYG{p}{[}\PYG{l+s+s1}{\PYGZsq{}}\PYG{l+s+s1}{close\PYGZus{}r1}\PYG{l+s+s1}{\PYGZsq{}}\PYG{p}{,}\PYG{l+s+s1}{\PYGZsq{}}\PYG{l+s+s1}{close\PYGZus{}r2}\PYG{l+s+s1}{\PYGZsq{}}\PYG{p}{,}\PYG{l+s+s1}{\PYGZsq{}}\PYG{l+s+s1}{close\PYGZus{}r3}\PYG{l+s+s1}{\PYGZsq{}}\PYG{p}{,}\PYG{l+s+s1}{\PYGZsq{}}\PYG{l+s+s1}{close\PYGZus{}r4}\PYG{l+s+s1}{\PYGZsq{}}\PYG{p}{,}\PYG{l+s+s1}{\PYGZsq{}}\PYG{l+s+s1}{close\PYGZus{}r5}\PYG{l+s+s1}{\PYGZsq{}}\PYG{p}{]}\PYG{p}{]}\PYG{o}{.}\PYG{n}{mean}\PYG{p}{(}\PYG{n}{axis}\PYG{o}{=}\PYG{l+m+mi}{1}\PYG{p}{)}
    \PYG{n}{daily\PYGZus{}price}\PYG{p}{[}\PYG{l+s+s1}{\PYGZsq{}}\PYG{l+s+s1}{min\PYGZus{}close}\PYG{l+s+s1}{\PYGZsq{}}\PYG{p}{]} \PYG{o}{=} \PYG{n}{daily\PYGZus{}price}\PYG{p}{[}\PYG{p}{[}\PYG{l+s+s1}{\PYGZsq{}}\PYG{l+s+s1}{close\PYGZus{}r1}\PYG{l+s+s1}{\PYGZsq{}}\PYG{p}{,}\PYG{l+s+s1}{\PYGZsq{}}\PYG{l+s+s1}{close\PYGZus{}r2}\PYG{l+s+s1}{\PYGZsq{}}\PYG{p}{,}\PYG{l+s+s1}{\PYGZsq{}}\PYG{l+s+s1}{close\PYGZus{}r3}\PYG{l+s+s1}{\PYGZsq{}}\PYG{p}{,}\PYG{l+s+s1}{\PYGZsq{}}\PYG{l+s+s1}{close\PYGZus{}r4}\PYG{l+s+s1}{\PYGZsq{}}\PYG{p}{,}\PYG{l+s+s1}{\PYGZsq{}}\PYG{l+s+s1}{close\PYGZus{}r5}\PYG{l+s+s1}{\PYGZsq{}}\PYG{p}{]}\PYG{p}{]}\PYG{o}{.}\PYG{n}{min}\PYG{p}{(}\PYG{n}{axis}\PYG{o}{=}\PYG{l+m+mi}{1}\PYG{p}{)}

    \PYG{n}{daily\PYGZus{}price}\PYG{p}{[}\PYG{l+s+s1}{\PYGZsq{}}\PYG{l+s+s1}{buy\PYGZus{}price}\PYG{l+s+s1}{\PYGZsq{}}\PYG{p}{]} \PYG{o}{=} \PYG{n}{daily\PYGZus{}price}\PYG{p}{[}\PYG{l+s+s1}{\PYGZsq{}}\PYG{l+s+s1}{Close}\PYG{l+s+s1}{\PYGZsq{}}\PYG{p}{]}
    \PYG{n}{daily\PYGZus{}price}\PYG{p}{[}\PYG{l+s+s1}{\PYGZsq{}}\PYG{l+s+s1}{buy\PYGZus{}low}\PYG{l+s+s1}{\PYGZsq{}}\PYG{p}{]} \PYG{o}{=} \PYG{n}{daily\PYGZus{}price}\PYG{p}{[}\PYG{l+s+s1}{\PYGZsq{}}\PYG{l+s+s1}{Low}\PYG{l+s+s1}{\PYGZsq{}}\PYG{p}{]}\PYG{o}{.}\PYG{n}{shift}\PYG{p}{(}\PYG{o}{\PYGZhy{}}\PYG{l+m+mi}{1}\PYG{p}{)} 
    \PYG{n}{daily\PYGZus{}price}\PYG{p}{[}\PYG{l+s+s1}{\PYGZsq{}}\PYG{l+s+s1}{buy\PYGZus{}high}\PYG{l+s+s1}{\PYGZsq{}}\PYG{p}{]} \PYG{o}{=} \PYG{n}{daily\PYGZus{}price}\PYG{p}{[}\PYG{l+s+s1}{\PYGZsq{}}\PYG{l+s+s1}{High}\PYG{l+s+s1}{\PYGZsq{}}\PYG{p}{]}\PYG{o}{.}\PYG{n}{shift}\PYG{p}{(}\PYG{o}{\PYGZhy{}}\PYG{l+m+mi}{1}\PYG{p}{)}

    \PYG{n}{daily\PYGZus{}price}\PYG{p}{[}\PYG{l+s+s1}{\PYGZsq{}}\PYG{l+s+s1}{buy}\PYG{l+s+s1}{\PYGZsq{}}\PYG{p}{]} \PYG{o}{=} \PYG{n}{np}\PYG{o}{.}\PYG{n}{where}\PYG{p}{(}\PYG{p}{(}\PYG{n}{daily\PYGZus{}price}\PYG{p}{[}\PYG{l+s+s1}{\PYGZsq{}}\PYG{l+s+s1}{buy\PYGZus{}price}\PYG{l+s+s1}{\PYGZsq{}}\PYG{p}{]}\PYG{o}{.}\PYG{n}{between}\PYG{p}{(}\PYG{n}{daily\PYGZus{}price}\PYG{p}{[}\PYG{l+s+s1}{\PYGZsq{}}\PYG{l+s+s1}{buy\PYGZus{}low}\PYG{l+s+s1}{\PYGZsq{}}\PYG{p}{]}\PYG{p}{,} \PYG{n}{daily\PYGZus{}price}\PYG{p}{[}\PYG{l+s+s1}{\PYGZsq{}}\PYG{l+s+s1}{buy\PYGZus{}high}\PYG{l+s+s1}{\PYGZsq{}}\PYG{p}{]}\PYG{p}{)}\PYG{p}{)}\PYG{p}{,} \PYG{l+m+mi}{1}\PYG{p}{,} \PYG{l+m+mi}{0}\PYG{p}{)} \PYG{c+c1}{\PYGZsh{} 4월 2일 매수일, 4월 1일 종가에 살 수 있는 지 여부}
    \PYG{n}{daily\PYGZus{}price}\PYG{p}{[}\PYG{l+s+s1}{\PYGZsq{}}\PYG{l+s+s1}{target}\PYG{l+s+s1}{\PYGZsq{}}\PYG{p}{]} \PYG{o}{=} \PYG{n}{np}\PYG{o}{.}\PYG{n}{where}\PYG{p}{(}\PYG{n}{daily\PYGZus{}price}\PYG{p}{[}\PYG{l+s+s1}{\PYGZsq{}}\PYG{l+s+s1}{max\PYGZus{}close}\PYG{l+s+s1}{\PYGZsq{}}\PYG{p}{]}\PYG{o}{\PYGZgt{}}\PYG{o}{=}\PYG{l+m+mf}{1.05}\PYG{p}{,} \PYG{l+m+mi}{1}\PYG{p}{,} \PYG{l+m+mi}{0}\PYG{p}{)}    
    
    \PYG{n}{outcome\PYGZus{}data} \PYG{o}{=} \PYG{n}{pd}\PYG{o}{.}\PYG{n}{concat}\PYG{p}{(}\PYG{p}{[}\PYG{n}{outcome\PYGZus{}data}\PYG{p}{,} \PYG{n}{daily\PYGZus{}price}\PYG{p}{]}\PYG{p}{,} \PYG{n}{axis}\PYG{o}{=}\PYG{l+m+mi}{0}\PYG{p}{)}  
\end{sphinxVerbatim}

\end{sphinxuseclass}\end{sphinxVerbatimInput}

\end{sphinxuseclass}
\sphinxAtStartPar
 최종 선정된 종목들의 결과가 궁금합니다. 선정된 종목 데이터에 결과 데이터를 병합합니다. 두 데이터셋의 인덱스는 종목이어야 병합이 가능합니다. 5\% 익절할 확률은 83.3\% 로 높게 나왔습니다. 최저 수익률의 평균은 .98 로 리스크도 비교적 낮은 것으로 보입니다. 2022년 4월 1일 매수한 종목은 수익권으로 예상이 됩니다. 물론 모든 날짜에 대하여 동일한 결과가 나오지는 않습니다.

\begin{sphinxuseclass}{cell}\begin{sphinxVerbatimInput}

\begin{sphinxuseclass}{cell_input}
\begin{sphinxVerbatim}[commandchars=\\\{\}]
\PYG{n}{outcome} \PYG{o}{=} \PYG{n}{outcome\PYGZus{}data}\PYG{o}{.}\PYG{n}{loc}\PYG{p}{[}\PYG{n}{today\PYGZus{}dt}\PYG{p}{]}\PYG{p}{[}\PYG{p}{[}\PYG{l+s+s1}{\PYGZsq{}}\PYG{l+s+s1}{code}\PYG{l+s+s1}{\PYGZsq{}}\PYG{p}{,}\PYG{l+s+s1}{\PYGZsq{}}\PYG{l+s+s1}{buy}\PYG{l+s+s1}{\PYGZsq{}}\PYG{p}{,}\PYG{l+s+s1}{\PYGZsq{}}\PYG{l+s+s1}{buy\PYGZus{}price}\PYG{l+s+s1}{\PYGZsq{}}\PYG{p}{,}\PYG{l+s+s1}{\PYGZsq{}}\PYG{l+s+s1}{buy\PYGZus{}low}\PYG{l+s+s1}{\PYGZsq{}}\PYG{p}{,}\PYG{l+s+s1}{\PYGZsq{}}\PYG{l+s+s1}{buy\PYGZus{}high}\PYG{l+s+s1}{\PYGZsq{}}\PYG{p}{,}\PYG{l+s+s1}{\PYGZsq{}}\PYG{l+s+s1}{max\PYGZus{}close}\PYG{l+s+s1}{\PYGZsq{}}\PYG{p}{,}\PYG{l+s+s1}{\PYGZsq{}}\PYG{l+s+s1}{mean\PYGZus{}close}\PYG{l+s+s1}{\PYGZsq{}}\PYG{p}{,}\PYG{l+s+s1}{\PYGZsq{}}\PYG{l+s+s1}{min\PYGZus{}close}\PYG{l+s+s1}{\PYGZsq{}}\PYG{p}{,}\PYG{l+s+s1}{\PYGZsq{}}\PYG{l+s+s1}{target}\PYG{l+s+s1}{\PYGZsq{}}\PYG{p}{]}\PYG{p}{]}\PYG{o}{.}\PYG{n}{set\PYGZus{}index}\PYG{p}{(}\PYG{l+s+s1}{\PYGZsq{}}\PYG{l+s+s1}{code}\PYG{l+s+s1}{\PYGZsq{}}\PYG{p}{)}
\PYG{n}{select\PYGZus{}outcome} \PYG{o}{=} \PYG{n}{tops}\PYG{o}{.}\PYG{n}{merge}\PYG{p}{(}\PYG{n}{outcome}\PYG{p}{,} \PYG{n}{left\PYGZus{}index}\PYG{o}{=}\PYG{k+kc}{True}\PYG{p}{,} \PYG{n}{right\PYGZus{}index}\PYG{o}{=}\PYG{k+kc}{True}\PYG{p}{,} \PYG{n}{how}\PYG{o}{=}\PYG{l+s+s1}{\PYGZsq{}}\PYG{l+s+s1}{inner}\PYG{l+s+s1}{\PYGZsq{}}\PYG{p}{)}
\PYG{n}{select\PYGZus{}outcome}\PYG{p}{[}\PYG{p}{[}\PYG{l+s+s1}{\PYGZsq{}}\PYG{l+s+s1}{yhat}\PYG{l+s+s1}{\PYGZsq{}}\PYG{p}{,}\PYG{l+s+s1}{\PYGZsq{}}\PYG{l+s+s1}{buy}\PYG{l+s+s1}{\PYGZsq{}}\PYG{p}{,}\PYG{l+s+s1}{\PYGZsq{}}\PYG{l+s+s1}{max\PYGZus{}close}\PYG{l+s+s1}{\PYGZsq{}}\PYG{p}{,}\PYG{l+s+s1}{\PYGZsq{}}\PYG{l+s+s1}{mean\PYGZus{}close}\PYG{l+s+s1}{\PYGZsq{}}\PYG{p}{,}\PYG{l+s+s1}{\PYGZsq{}}\PYG{l+s+s1}{min\PYGZus{}close}\PYG{l+s+s1}{\PYGZsq{}}\PYG{p}{]}\PYG{p}{]}\PYG{o}{.}\PYG{n}{mean}\PYG{p}{(}\PYG{p}{)}
\end{sphinxVerbatim}

\end{sphinxuseclass}\end{sphinxVerbatimInput}
\begin{sphinxVerbatimOutput}

\begin{sphinxuseclass}{cell_output}
\begin{sphinxVerbatim}[commandchars=\\\{\}]
yhat         0.375
buy          0.800
max\PYGZus{}close    1.127
mean\PYGZus{}close   1.055
min\PYGZus{}close    0.983
dtype: float64
\end{sphinxVerbatim}

\end{sphinxuseclass}\end{sphinxVerbatimOutput}

\end{sphinxuseclass}
\sphinxAtStartPar
 buy 는 4월 1일 종가에 4월 2일 매수할 수 있는 기회가 있는 지를 알려주는 Flag 입니다. CSA 코스믹은 4월 2일 갭상승으로 시작했습니다. 4월 1일 종가에 살 수 있는 기회가 없습니다.

\begin{sphinxuseclass}{cell}\begin{sphinxVerbatimInput}

\begin{sphinxuseclass}{cell_input}
\begin{sphinxVerbatim}[commandchars=\\\{\}]
\PYG{n}{select\PYGZus{}outcome}\PYG{p}{[}\PYG{p}{[}\PYG{l+s+s1}{\PYGZsq{}}\PYG{l+s+s1}{name}\PYG{l+s+s1}{\PYGZsq{}}\PYG{p}{,}\PYG{l+s+s1}{\PYGZsq{}}\PYG{l+s+s1}{buy}\PYG{l+s+s1}{\PYGZsq{}}\PYG{p}{,}\PYG{l+s+s1}{\PYGZsq{}}\PYG{l+s+s1}{buy\PYGZus{}price}\PYG{l+s+s1}{\PYGZsq{}}\PYG{p}{,} \PYG{l+s+s1}{\PYGZsq{}}\PYG{l+s+s1}{buy\PYGZus{}low}\PYG{l+s+s1}{\PYGZsq{}}\PYG{p}{,}\PYG{l+s+s1}{\PYGZsq{}}\PYG{l+s+s1}{buy\PYGZus{}high}\PYG{l+s+s1}{\PYGZsq{}}\PYG{p}{,}\PYG{l+s+s1}{\PYGZsq{}}\PYG{l+s+s1}{yhat}\PYG{l+s+s1}{\PYGZsq{}}\PYG{p}{,}\PYG{l+s+s1}{\PYGZsq{}}\PYG{l+s+s1}{max\PYGZus{}close}\PYG{l+s+s1}{\PYGZsq{}}\PYG{p}{,}\PYG{l+s+s1}{\PYGZsq{}}\PYG{l+s+s1}{mean\PYGZus{}close}\PYG{l+s+s1}{\PYGZsq{}}\PYG{p}{,}\PYG{l+s+s1}{\PYGZsq{}}\PYG{l+s+s1}{min\PYGZus{}close}\PYG{l+s+s1}{\PYGZsq{}}\PYG{p}{]}\PYG{p}{]}\PYG{o}{.}\PYG{n}{style}\PYG{o}{.}\PYG{n}{set\PYGZus{}table\PYGZus{}attributes}\PYG{p}{(}\PYG{l+s+s1}{\PYGZsq{}}\PYG{l+s+s1}{style=}\PYG{l+s+s1}{\PYGZdq{}}\PYG{l+s+s1}{font\PYGZhy{}size: 12px}\PYG{l+s+s1}{\PYGZdq{}}\PYG{l+s+s1}{\PYGZsq{}}\PYG{p}{)}\PYG{o}{.}\PYG{n}{format}\PYG{p}{(}\PYG{n}{precision}\PYG{o}{=}\PYG{l+m+mi}{3}\PYG{p}{)}
\end{sphinxVerbatim}

\end{sphinxuseclass}\end{sphinxVerbatimInput}
\begin{sphinxVerbatimOutput}

\begin{sphinxuseclass}{cell_output}
\begin{sphinxVerbatim}[commandchars=\\\{\}]
\PYGZlt{}pandas.io.formats.style.Styler at 0x202925b2550\PYGZgt{}
\end{sphinxVerbatim}

\end{sphinxuseclass}\end{sphinxVerbatimOutput}

\end{sphinxuseclass}
\sphinxAtStartPar
2022년 4월 1일 추천받은 종목들의 일봉 차트를 보겠습니다. CSA 코스믹은 전일 종가로 당일 매수가 불가능합니다. 2022년 4월 2일 갭상승으로 시작을 했습니다.  에디슨 INNO 는 4월 2일 매수 후 익절할 기회를 제공하고 있습니다.

\sphinxAtStartPar
 \sphinxstylestrong{한일단조}

\sphinxAtStartPar
\sphinxstylestrong{장원테크}

\sphinxAtStartPar
\sphinxstylestrong{에디슨INNO}

\sphinxAtStartPar
\sphinxstylestrong{서진오토모티브}

\sphinxAtStartPar
\sphinxstylestrong{CSA 코스믹}


\section{해결책 테스트}
\label{\detokenize{chapter6/6.1.2_Stock_Selection_Function:id1}}\label{\detokenize{chapter6/6.1.2_Stock_Selection_Function::doc}}
\sphinxAtStartPar
오늘 날짜만 입력하면 내일 매수할 종목이 추천되도록 각 프로세스를 통합하여 함수를 구현합니다. 임의 날짜를 넣어서 테스트 해 봅니다. 이 책에서는 2022년 4월 1일, 2022년 4월 18일,  2022년 5월 2일, 2022년 5월 9일, 2022년 5월 25일,  2022년 6월 2일, 6월 16일에 대하여 종목 선정 및 결과 수익률을 테스트 해 보았습니다.  모델을 개발하는데 사용한 날짜는 모델 검증 용도로 적절하지 않습니다. 왜냐하면 개발에서 사용한 데이터는 모델이 좋은 성과가 나오도록 최적화되어 있기 때문입니다.  참고로 모델 개발은 2021년 1월 4일부터 2022년 3월 24일까지 데이터가 사용되었습니다.

\sphinxAtStartPar
이제 종목을 추천하는 프로세스를 완성했습니다. 장 마감 후 종목 추천을 받아 익일 증권사 API  를 이용해서 자동매매를 구현하고 한 달 동안의 수익이 어떤지 검증해 보겠습니다. 홈트레이딩 시스템에도 자동매매가 가능합니다. 책에서 구현할 자동매매는 홈트레이딩 감시 매매 설정으로도 충분히 가능합니다.

\sphinxAtStartPar
실전에서는 HTS 에서 제공하는 예약 매수기능과 매도 감시기능을 이용하는 것리 편리합니다. HTS 를 활용하여 자동으로 매수 매도가 가능합니다.

\begin{sphinxuseclass}{cell}\begin{sphinxVerbatimInput}

\begin{sphinxuseclass}{cell_input}
\begin{sphinxVerbatim}[commandchars=\\\{\}]
\PYG{k+kn}{import} \PYG{n+nn}{FinanceDataReader} \PYG{k}{as} \PYG{n+nn}{fdr}
\PYG{k+kn}{import} \PYG{n+nn}{matplotlib}\PYG{n+nn}{.}\PYG{n+nn}{pyplot} \PYG{k}{as} \PYG{n+nn}{plt}
\PYG{o}{\PYGZpc{}}\PYG{k}{matplotlib} inline
\PYG{k+kn}{import} \PYG{n+nn}{pandas} \PYG{k}{as} \PYG{n+nn}{pd}
\PYG{k+kn}{import} \PYG{n+nn}{numpy} \PYG{k}{as} \PYG{n+nn}{np}
\PYG{k+kn}{import} \PYG{n+nn}{datetime}
\PYG{k+kn}{import} \PYG{n+nn}{pickle}
\PYG{k+kn}{import} \PYG{n+nn}{glob}
\end{sphinxVerbatim}

\end{sphinxuseclass}\end{sphinxVerbatimInput}

\end{sphinxuseclass}
\sphinxAtStartPar
 추전 종목을 만드는 여러 개의 프로세스를 하나의 함수로 만들었습니다.

\begin{sphinxuseclass}{cell}\begin{sphinxVerbatimInput}

\begin{sphinxuseclass}{cell_input}
\begin{sphinxVerbatim}[commandchars=\\\{\}]
\PYG{k}{def} \PYG{n+nf}{select\PYGZus{}stocks}\PYG{p}{(}\PYG{n}{today\PYGZus{}dt}\PYG{p}{)}\PYG{p}{:}
    
    \PYG{n}{today} \PYG{o}{=} \PYG{n}{datetime}\PYG{o}{.}\PYG{n}{datetime}\PYG{o}{.}\PYG{n}{strptime}\PYG{p}{(}\PYG{n}{today\PYGZus{}dt}\PYG{p}{,} \PYG{l+s+s1}{\PYGZsq{}}\PYG{l+s+s1}{\PYGZpc{}}\PYG{l+s+s1}{Y\PYGZhy{}}\PYG{l+s+s1}{\PYGZpc{}}\PYG{l+s+s1}{m\PYGZhy{}}\PYG{l+s+si}{\PYGZpc{}d}\PYG{l+s+s1}{\PYGZsq{}}\PYG{p}{)}
    \PYG{n}{start\PYGZus{}dt} \PYG{o}{=} \PYG{n}{today} \PYG{o}{\PYGZhy{}} \PYG{n}{datetime}\PYG{o}{.}\PYG{n}{timedelta}\PYG{p}{(}\PYG{n}{days}\PYG{o}{=}\PYG{l+m+mi}{100}\PYG{p}{)} \PYG{c+c1}{\PYGZsh{} 100 일전 데이터 부터 시작 \PYGZhy{} 피쳐 엔지니어링은 최소 60 개의 일봉이 필요함}
    \PYG{n+nb}{print}\PYG{p}{(}\PYG{n}{start\PYGZus{}dt}\PYG{p}{,} \PYG{n}{today\PYGZus{}dt}\PYG{p}{)}

    \PYG{n}{kosdaq\PYGZus{}list} \PYG{o}{=} \PYG{n}{pd}\PYG{o}{.}\PYG{n}{read\PYGZus{}pickle}\PYG{p}{(}\PYG{l+s+s1}{\PYGZsq{}}\PYG{l+s+s1}{kosdaq\PYGZus{}list.pkl}\PYG{l+s+s1}{\PYGZsq{}}\PYG{p}{)}

    \PYG{n}{price\PYGZus{}data} \PYG{o}{=} \PYG{n}{pd}\PYG{o}{.}\PYG{n}{DataFrame}\PYG{p}{(}\PYG{p}{)}

    \PYG{k}{for} \PYG{n}{code}\PYG{p}{,} \PYG{n}{name} \PYG{o+ow}{in} \PYG{n+nb}{zip}\PYG{p}{(}\PYG{n}{kosdaq\PYGZus{}list}\PYG{p}{[}\PYG{l+s+s1}{\PYGZsq{}}\PYG{l+s+s1}{code}\PYG{l+s+s1}{\PYGZsq{}}\PYG{p}{]}\PYG{p}{,} \PYG{n}{kosdaq\PYGZus{}list}\PYG{p}{[}\PYG{l+s+s1}{\PYGZsq{}}\PYG{l+s+s1}{name}\PYG{l+s+s1}{\PYGZsq{}}\PYG{p}{]}\PYG{p}{)}\PYG{p}{:}  \PYG{c+c1}{\PYGZsh{} 코스닥 모든 종목에서 대하여 반복}
        \PYG{n}{daily\PYGZus{}price} \PYG{o}{=} \PYG{n}{fdr}\PYG{o}{.}\PYG{n}{DataReader}\PYG{p}{(}\PYG{n}{code}\PYG{p}{,} \PYG{n}{start} \PYG{o}{=} \PYG{n}{start\PYGZus{}dt}\PYG{p}{,} \PYG{n}{end} \PYG{o}{=} \PYG{n}{today\PYGZus{}dt}\PYG{p}{)} \PYG{c+c1}{\PYGZsh{} 종목, 일봉, 데이터 갯수}

        \PYG{n}{daily\PYGZus{}price}\PYG{p}{[}\PYG{l+s+s1}{\PYGZsq{}}\PYG{l+s+s1}{code}\PYG{l+s+s1}{\PYGZsq{}}\PYG{p}{]} \PYG{o}{=} \PYG{n}{code}
        \PYG{n}{daily\PYGZus{}price}\PYG{p}{[}\PYG{l+s+s1}{\PYGZsq{}}\PYG{l+s+s1}{name}\PYG{l+s+s1}{\PYGZsq{}}\PYG{p}{]} \PYG{o}{=} \PYG{n}{name}
        \PYG{n}{price\PYGZus{}data} \PYG{o}{=} \PYG{n}{pd}\PYG{o}{.}\PYG{n}{concat}\PYG{p}{(}\PYG{p}{[}\PYG{n}{price\PYGZus{}data}\PYG{p}{,} \PYG{n}{daily\PYGZus{}price}\PYG{p}{]}\PYG{p}{,} \PYG{n}{axis}\PYG{o}{=}\PYG{l+m+mi}{0}\PYG{p}{)}   

    \PYG{n}{price\PYGZus{}data}\PYG{o}{.}\PYG{n}{index}\PYG{o}{.}\PYG{n}{name} \PYG{o}{=} \PYG{l+s+s1}{\PYGZsq{}}\PYG{l+s+s1}{date}\PYG{l+s+s1}{\PYGZsq{}}
    \PYG{n}{price\PYGZus{}data}\PYG{o}{.}\PYG{n}{columns}\PYG{o}{=} \PYG{n}{price\PYGZus{}data}\PYG{o}{.}\PYG{n}{columns}\PYG{o}{.}\PYG{n}{str}\PYG{o}{.}\PYG{n}{lower}\PYG{p}{(}\PYG{p}{)} \PYG{c+c1}{\PYGZsh{} 컬럼 이름 소문자로 변경}

    \PYG{n}{kosdaq\PYGZus{}index} \PYG{o}{=} \PYG{n}{fdr}\PYG{o}{.}\PYG{n}{DataReader}\PYG{p}{(}\PYG{l+s+s1}{\PYGZsq{}}\PYG{l+s+s1}{KQ11}\PYG{l+s+s1}{\PYGZsq{}}\PYG{p}{,} \PYG{n}{start} \PYG{o}{=} \PYG{n}{start\PYGZus{}dt}\PYG{p}{,} \PYG{n}{end} \PYG{o}{=} \PYG{n}{today\PYGZus{}dt}\PYG{p}{)} \PYG{c+c1}{\PYGZsh{} 데이터 호출}
    \PYG{n}{kosdaq\PYGZus{}index}\PYG{o}{.}\PYG{n}{columns} \PYG{o}{=} \PYG{p}{[}\PYG{l+s+s1}{\PYGZsq{}}\PYG{l+s+s1}{close}\PYG{l+s+s1}{\PYGZsq{}}\PYG{p}{,}\PYG{l+s+s1}{\PYGZsq{}}\PYG{l+s+s1}{open}\PYG{l+s+s1}{\PYGZsq{}}\PYG{p}{,}\PYG{l+s+s1}{\PYGZsq{}}\PYG{l+s+s1}{high}\PYG{l+s+s1}{\PYGZsq{}}\PYG{p}{,}\PYG{l+s+s1}{\PYGZsq{}}\PYG{l+s+s1}{low}\PYG{l+s+s1}{\PYGZsq{}}\PYG{p}{,}\PYG{l+s+s1}{\PYGZsq{}}\PYG{l+s+s1}{volume}\PYG{l+s+s1}{\PYGZsq{}}\PYG{p}{,}\PYG{l+s+s1}{\PYGZsq{}}\PYG{l+s+s1}{change}\PYG{l+s+s1}{\PYGZsq{}}\PYG{p}{]} \PYG{c+c1}{\PYGZsh{} 컬럼명 변경}
    \PYG{n}{kosdaq\PYGZus{}index}\PYG{o}{.}\PYG{n}{index}\PYG{o}{.}\PYG{n}{name}\PYG{o}{=}\PYG{l+s+s1}{\PYGZsq{}}\PYG{l+s+s1}{date}\PYG{l+s+s1}{\PYGZsq{}} \PYG{c+c1}{\PYGZsh{} 인덱스 이름 생성}
    \PYG{n}{kosdaq\PYGZus{}index}\PYG{o}{.}\PYG{n}{sort\PYGZus{}index}\PYG{p}{(}\PYG{n}{inplace}\PYG{o}{=}\PYG{k+kc}{True}\PYG{p}{)} \PYG{c+c1}{\PYGZsh{} 인덱스(날짜) 로 정렬 }
    \PYG{n}{kosdaq\PYGZus{}index}\PYG{p}{[}\PYG{l+s+s1}{\PYGZsq{}}\PYG{l+s+s1}{kosdaq\PYGZus{}return}\PYG{l+s+s1}{\PYGZsq{}}\PYG{p}{]} \PYG{o}{=} \PYG{n}{kosdaq\PYGZus{}index}\PYG{p}{[}\PYG{l+s+s1}{\PYGZsq{}}\PYG{l+s+s1}{close}\PYG{l+s+s1}{\PYGZsq{}}\PYG{p}{]}\PYG{o}{/}\PYG{n}{kosdaq\PYGZus{}index}\PYG{p}{[}\PYG{l+s+s1}{\PYGZsq{}}\PYG{l+s+s1}{close}\PYG{l+s+s1}{\PYGZsq{}}\PYG{p}{]}\PYG{o}{.}\PYG{n}{shift}\PYG{p}{(}\PYG{l+m+mi}{1}\PYG{p}{)} \PYG{c+c1}{\PYGZsh{} 수익율 : 전 날 종가대비 당일 종가}

    \PYG{n}{merged} \PYG{o}{=} \PYG{n}{price\PYGZus{}data}\PYG{o}{.}\PYG{n}{merge}\PYG{p}{(}\PYG{n}{kosdaq\PYGZus{}index}\PYG{p}{[}\PYG{l+s+s1}{\PYGZsq{}}\PYG{l+s+s1}{kosdaq\PYGZus{}return}\PYG{l+s+s1}{\PYGZsq{}}\PYG{p}{]}\PYG{p}{,} \PYG{n}{left\PYGZus{}index}\PYG{o}{=}\PYG{k+kc}{True}\PYG{p}{,} \PYG{n}{right\PYGZus{}index}\PYG{o}{=}\PYG{k+kc}{True}\PYG{p}{,} \PYG{n}{how}\PYG{o}{=}\PYG{l+s+s1}{\PYGZsq{}}\PYG{l+s+s1}{left}\PYG{l+s+s1}{\PYGZsq{}}\PYG{p}{)}

    \PYG{n}{return\PYGZus{}all} \PYG{o}{=} \PYG{n}{pd}\PYG{o}{.}\PYG{n}{DataFrame}\PYG{p}{(}\PYG{p}{)}

    \PYG{k}{for} \PYG{n}{code} \PYG{o+ow}{in} \PYG{n}{kosdaq\PYGZus{}list}\PYG{p}{[}\PYG{l+s+s1}{\PYGZsq{}}\PYG{l+s+s1}{code}\PYG{l+s+s1}{\PYGZsq{}}\PYG{p}{]}\PYG{p}{:}  

        \PYG{n}{stock\PYGZus{}return} \PYG{o}{=} \PYG{n}{merged}\PYG{p}{[}\PYG{n}{merged}\PYG{p}{[}\PYG{l+s+s1}{\PYGZsq{}}\PYG{l+s+s1}{code}\PYG{l+s+s1}{\PYGZsq{}}\PYG{p}{]}\PYG{o}{==}\PYG{n}{code}\PYG{p}{]}\PYG{o}{.}\PYG{n}{sort\PYGZus{}index}\PYG{p}{(}\PYG{p}{)}
        \PYG{n}{stock\PYGZus{}return}\PYG{p}{[}\PYG{l+s+s1}{\PYGZsq{}}\PYG{l+s+s1}{return}\PYG{l+s+s1}{\PYGZsq{}}\PYG{p}{]} \PYG{o}{=} \PYG{n}{stock\PYGZus{}return}\PYG{p}{[}\PYG{l+s+s1}{\PYGZsq{}}\PYG{l+s+s1}{close}\PYG{l+s+s1}{\PYGZsq{}}\PYG{p}{]}\PYG{o}{/}\PYG{n}{stock\PYGZus{}return}\PYG{p}{[}\PYG{l+s+s1}{\PYGZsq{}}\PYG{l+s+s1}{close}\PYG{l+s+s1}{\PYGZsq{}}\PYG{p}{]}\PYG{o}{.}\PYG{n}{shift}\PYG{p}{(}\PYG{l+m+mi}{1}\PYG{p}{)} \PYG{c+c1}{\PYGZsh{} 종목별 전일 종가 대비 당일 종가 수익율}
        \PYG{n}{c1} \PYG{o}{=} \PYG{p}{(}\PYG{n}{stock\PYGZus{}return}\PYG{p}{[}\PYG{l+s+s1}{\PYGZsq{}}\PYG{l+s+s1}{kosdaq\PYGZus{}return}\PYG{l+s+s1}{\PYGZsq{}}\PYG{p}{]} \PYG{o}{\PYGZlt{}} \PYG{l+m+mi}{1}\PYG{p}{)} \PYG{c+c1}{\PYGZsh{} 수익율 1 보다 작음. 당일 종가가 전일 종가보다 낮음 (코스닥 지표)}
        \PYG{n}{c2} \PYG{o}{=} \PYG{p}{(}\PYG{n}{stock\PYGZus{}return}\PYG{p}{[}\PYG{l+s+s1}{\PYGZsq{}}\PYG{l+s+s1}{return}\PYG{l+s+s1}{\PYGZsq{}}\PYG{p}{]} \PYG{o}{\PYGZgt{}} \PYG{l+m+mi}{1}\PYG{p}{)} \PYG{c+c1}{\PYGZsh{} 수익율 1 보다 큼. 당일 종가가 전일 종가보다 큼 (개별 종목)}
        \PYG{n}{stock\PYGZus{}return}\PYG{p}{[}\PYG{l+s+s1}{\PYGZsq{}}\PYG{l+s+s1}{win\PYGZus{}market}\PYG{l+s+s1}{\PYGZsq{}}\PYG{p}{]} \PYG{o}{=} \PYG{n}{np}\PYG{o}{.}\PYG{n}{where}\PYG{p}{(}\PYG{p}{(}\PYG{n}{c1}\PYG{o}{\PYGZam{}}\PYG{n}{c2}\PYG{p}{)}\PYG{p}{,} \PYG{l+m+mi}{1}\PYG{p}{,} \PYG{l+m+mi}{0}\PYG{p}{)} \PYG{c+c1}{\PYGZsh{} C1 과 C2 조건을 동시에 만족하면 1, 아니면 0}
        \PYG{n}{return\PYGZus{}all} \PYG{o}{=} \PYG{n}{pd}\PYG{o}{.}\PYG{n}{concat}\PYG{p}{(}\PYG{p}{[}\PYG{n}{return\PYGZus{}all}\PYG{p}{,} \PYG{n}{stock\PYGZus{}return}\PYG{p}{]}\PYG{p}{,} \PYG{n}{axis}\PYG{o}{=}\PYG{l+m+mi}{0}\PYG{p}{)} 

    \PYG{n}{return\PYGZus{}all}\PYG{o}{.}\PYG{n}{dropna}\PYG{p}{(}\PYG{n}{inplace}\PYG{o}{=}\PYG{k+kc}{True}\PYG{p}{)}    

    \PYG{n}{model\PYGZus{}inputs} \PYG{o}{=} \PYG{n}{pd}\PYG{o}{.}\PYG{n}{DataFrame}\PYG{p}{(}\PYG{p}{)}

    \PYG{k}{for} \PYG{n}{code}\PYG{p}{,} \PYG{n}{name}\PYG{p}{,} \PYG{n}{sector} \PYG{o+ow}{in} \PYG{n+nb}{zip}\PYG{p}{(}\PYG{n}{kosdaq\PYGZus{}list}\PYG{p}{[}\PYG{l+s+s1}{\PYGZsq{}}\PYG{l+s+s1}{code}\PYG{l+s+s1}{\PYGZsq{}}\PYG{p}{]}\PYG{p}{,} \PYG{n}{kosdaq\PYGZus{}list}\PYG{p}{[}\PYG{l+s+s1}{\PYGZsq{}}\PYG{l+s+s1}{name}\PYG{l+s+s1}{\PYGZsq{}}\PYG{p}{]}\PYG{p}{,} \PYG{n}{kosdaq\PYGZus{}list}\PYG{p}{[}\PYG{l+s+s1}{\PYGZsq{}}\PYG{l+s+s1}{sector}\PYG{l+s+s1}{\PYGZsq{}}\PYG{p}{]}\PYG{p}{)}\PYG{p}{:}

        \PYG{n}{data} \PYG{o}{=} \PYG{n}{return\PYGZus{}all}\PYG{p}{[}\PYG{n}{return\PYGZus{}all}\PYG{p}{[}\PYG{l+s+s1}{\PYGZsq{}}\PYG{l+s+s1}{code}\PYG{l+s+s1}{\PYGZsq{}}\PYG{p}{]}\PYG{o}{==}\PYG{n}{code}\PYG{p}{]}\PYG{o}{.}\PYG{n}{sort\PYGZus{}index}\PYG{p}{(}\PYG{p}{)}\PYG{o}{.}\PYG{n}{copy}\PYG{p}{(}\PYG{p}{)}    

        \PYG{c+c1}{\PYGZsh{} 가격변동성이 크고, 거래량이 몰린 종목이 주가가 상승한다}
        \PYG{n}{data}\PYG{p}{[}\PYG{l+s+s1}{\PYGZsq{}}\PYG{l+s+s1}{price\PYGZus{}mean}\PYG{l+s+s1}{\PYGZsq{}}\PYG{p}{]} \PYG{o}{=} \PYG{n}{data}\PYG{p}{[}\PYG{l+s+s1}{\PYGZsq{}}\PYG{l+s+s1}{close}\PYG{l+s+s1}{\PYGZsq{}}\PYG{p}{]}\PYG{o}{.}\PYG{n}{rolling}\PYG{p}{(}\PYG{l+m+mi}{20}\PYG{p}{)}\PYG{o}{.}\PYG{n}{mean}\PYG{p}{(}\PYG{p}{)}
        \PYG{n}{data}\PYG{p}{[}\PYG{l+s+s1}{\PYGZsq{}}\PYG{l+s+s1}{price\PYGZus{}std}\PYG{l+s+s1}{\PYGZsq{}}\PYG{p}{]} \PYG{o}{=} \PYG{n}{data}\PYG{p}{[}\PYG{l+s+s1}{\PYGZsq{}}\PYG{l+s+s1}{close}\PYG{l+s+s1}{\PYGZsq{}}\PYG{p}{]}\PYG{o}{.}\PYG{n}{rolling}\PYG{p}{(}\PYG{l+m+mi}{20}\PYG{p}{)}\PYG{o}{.}\PYG{n}{std}\PYG{p}{(}\PYG{n}{ddof}\PYG{o}{=}\PYG{l+m+mi}{0}\PYG{p}{)}
        \PYG{n}{data}\PYG{p}{[}\PYG{l+s+s1}{\PYGZsq{}}\PYG{l+s+s1}{price\PYGZus{}z}\PYG{l+s+s1}{\PYGZsq{}}\PYG{p}{]} \PYG{o}{=} \PYG{p}{(}\PYG{n}{data}\PYG{p}{[}\PYG{l+s+s1}{\PYGZsq{}}\PYG{l+s+s1}{close}\PYG{l+s+s1}{\PYGZsq{}}\PYG{p}{]} \PYG{o}{\PYGZhy{}} \PYG{n}{data}\PYG{p}{[}\PYG{l+s+s1}{\PYGZsq{}}\PYG{l+s+s1}{price\PYGZus{}mean}\PYG{l+s+s1}{\PYGZsq{}}\PYG{p}{]}\PYG{p}{)}\PYG{o}{/}\PYG{n}{data}\PYG{p}{[}\PYG{l+s+s1}{\PYGZsq{}}\PYG{l+s+s1}{price\PYGZus{}std}\PYG{l+s+s1}{\PYGZsq{}}\PYG{p}{]}    
        \PYG{n}{data}\PYG{p}{[}\PYG{l+s+s1}{\PYGZsq{}}\PYG{l+s+s1}{volume\PYGZus{}mean}\PYG{l+s+s1}{\PYGZsq{}}\PYG{p}{]} \PYG{o}{=} \PYG{n}{data}\PYG{p}{[}\PYG{l+s+s1}{\PYGZsq{}}\PYG{l+s+s1}{volume}\PYG{l+s+s1}{\PYGZsq{}}\PYG{p}{]}\PYG{o}{.}\PYG{n}{rolling}\PYG{p}{(}\PYG{l+m+mi}{20}\PYG{p}{)}\PYG{o}{.}\PYG{n}{mean}\PYG{p}{(}\PYG{p}{)}
        \PYG{n}{data}\PYG{p}{[}\PYG{l+s+s1}{\PYGZsq{}}\PYG{l+s+s1}{volume\PYGZus{}std}\PYG{l+s+s1}{\PYGZsq{}}\PYG{p}{]} \PYG{o}{=} \PYG{n}{data}\PYG{p}{[}\PYG{l+s+s1}{\PYGZsq{}}\PYG{l+s+s1}{volume}\PYG{l+s+s1}{\PYGZsq{}}\PYG{p}{]}\PYG{o}{.}\PYG{n}{rolling}\PYG{p}{(}\PYG{l+m+mi}{20}\PYG{p}{)}\PYG{o}{.}\PYG{n}{std}\PYG{p}{(}\PYG{n}{ddof}\PYG{o}{=}\PYG{l+m+mi}{0}\PYG{p}{)}
        \PYG{n}{data}\PYG{p}{[}\PYG{l+s+s1}{\PYGZsq{}}\PYG{l+s+s1}{volume\PYGZus{}z}\PYG{l+s+s1}{\PYGZsq{}}\PYG{p}{]} \PYG{o}{=} \PYG{p}{(}\PYG{n}{data}\PYG{p}{[}\PYG{l+s+s1}{\PYGZsq{}}\PYG{l+s+s1}{volume}\PYG{l+s+s1}{\PYGZsq{}}\PYG{p}{]} \PYG{o}{\PYGZhy{}} \PYG{n}{data}\PYG{p}{[}\PYG{l+s+s1}{\PYGZsq{}}\PYG{l+s+s1}{volume\PYGZus{}mean}\PYG{l+s+s1}{\PYGZsq{}}\PYG{p}{]}\PYG{p}{)}\PYG{o}{/}\PYG{n}{data}\PYG{p}{[}\PYG{l+s+s1}{\PYGZsq{}}\PYG{l+s+s1}{volume\PYGZus{}std}\PYG{l+s+s1}{\PYGZsq{}}\PYG{p}{]}

        \PYG{c+c1}{\PYGZsh{} 위꼬리가 긴 양봉이 자주발생한다.}
        \PYG{n}{data}\PYG{p}{[}\PYG{l+s+s1}{\PYGZsq{}}\PYG{l+s+s1}{positive\PYGZus{}candle}\PYG{l+s+s1}{\PYGZsq{}}\PYG{p}{]} \PYG{o}{=} \PYG{p}{(}\PYG{n}{data}\PYG{p}{[}\PYG{l+s+s1}{\PYGZsq{}}\PYG{l+s+s1}{close}\PYG{l+s+s1}{\PYGZsq{}}\PYG{p}{]} \PYG{o}{\PYGZgt{}} \PYG{n}{data}\PYG{p}{[}\PYG{l+s+s1}{\PYGZsq{}}\PYG{l+s+s1}{open}\PYG{l+s+s1}{\PYGZsq{}}\PYG{p}{]}\PYG{p}{)}\PYG{o}{.}\PYG{n}{astype}\PYG{p}{(}\PYG{n+nb}{int}\PYG{p}{)} \PYG{c+c1}{\PYGZsh{} 양봉}
        \PYG{n}{data}\PYG{p}{[}\PYG{l+s+s1}{\PYGZsq{}}\PYG{l+s+s1}{high/close}\PYG{l+s+s1}{\PYGZsq{}}\PYG{p}{]} \PYG{o}{=} \PYG{p}{(}\PYG{n}{data}\PYG{p}{[}\PYG{l+s+s1}{\PYGZsq{}}\PYG{l+s+s1}{positive\PYGZus{}candle}\PYG{l+s+s1}{\PYGZsq{}}\PYG{p}{]}\PYG{o}{==}\PYG{l+m+mi}{1}\PYG{p}{)}\PYG{o}{*}\PYG{p}{(}\PYG{n}{data}\PYG{p}{[}\PYG{l+s+s1}{\PYGZsq{}}\PYG{l+s+s1}{high}\PYG{l+s+s1}{\PYGZsq{}}\PYG{p}{]}\PYG{o}{/}\PYG{n}{data}\PYG{p}{[}\PYG{l+s+s1}{\PYGZsq{}}\PYG{l+s+s1}{close}\PYG{l+s+s1}{\PYGZsq{}}\PYG{p}{]} \PYG{o}{\PYGZgt{}} \PYG{l+m+mf}{1.1}\PYG{p}{)}\PYG{o}{.}\PYG{n}{astype}\PYG{p}{(}\PYG{n+nb}{int}\PYG{p}{)} \PYG{c+c1}{\PYGZsh{} 양봉이면서 고가가 종가보다 높게 위치}
        \PYG{n}{data}\PYG{p}{[}\PYG{l+s+s1}{\PYGZsq{}}\PYG{l+s+s1}{num\PYGZus{}high/close}\PYG{l+s+s1}{\PYGZsq{}}\PYG{p}{]} \PYG{o}{=}  \PYG{n}{data}\PYG{p}{[}\PYG{l+s+s1}{\PYGZsq{}}\PYG{l+s+s1}{high/close}\PYG{l+s+s1}{\PYGZsq{}}\PYG{p}{]}\PYG{o}{.}\PYG{n}{rolling}\PYG{p}{(}\PYG{l+m+mi}{20}\PYG{p}{)}\PYG{o}{.}\PYG{n}{sum}\PYG{p}{(}\PYG{p}{)}
        \PYG{n}{data}\PYG{p}{[}\PYG{l+s+s1}{\PYGZsq{}}\PYG{l+s+s1}{long\PYGZus{}candle}\PYG{l+s+s1}{\PYGZsq{}}\PYG{p}{]} \PYG{o}{=} \PYG{p}{(}\PYG{n}{data}\PYG{p}{[}\PYG{l+s+s1}{\PYGZsq{}}\PYG{l+s+s1}{positive\PYGZus{}candle}\PYG{l+s+s1}{\PYGZsq{}}\PYG{p}{]}\PYG{o}{==}\PYG{l+m+mi}{1}\PYG{p}{)}\PYG{o}{*}\PYG{p}{(}\PYG{n}{data}\PYG{p}{[}\PYG{l+s+s1}{\PYGZsq{}}\PYG{l+s+s1}{high}\PYG{l+s+s1}{\PYGZsq{}}\PYG{p}{]}\PYG{o}{==}\PYG{n}{data}\PYG{p}{[}\PYG{l+s+s1}{\PYGZsq{}}\PYG{l+s+s1}{close}\PYG{l+s+s1}{\PYGZsq{}}\PYG{p}{]}\PYG{p}{)}\PYG{o}{*}\PYGZbs{}
        \PYG{p}{(}\PYG{n}{data}\PYG{p}{[}\PYG{l+s+s1}{\PYGZsq{}}\PYG{l+s+s1}{low}\PYG{l+s+s1}{\PYGZsq{}}\PYG{p}{]}\PYG{o}{==}\PYG{n}{data}\PYG{p}{[}\PYG{l+s+s1}{\PYGZsq{}}\PYG{l+s+s1}{open}\PYG{l+s+s1}{\PYGZsq{}}\PYG{p}{]}\PYG{p}{)}\PYG{o}{*}\PYG{p}{(}\PYG{n}{data}\PYG{p}{[}\PYG{l+s+s1}{\PYGZsq{}}\PYG{l+s+s1}{close}\PYG{l+s+s1}{\PYGZsq{}}\PYG{p}{]}\PYG{o}{/}\PYG{n}{data}\PYG{p}{[}\PYG{l+s+s1}{\PYGZsq{}}\PYG{l+s+s1}{open}\PYG{l+s+s1}{\PYGZsq{}}\PYG{p}{]} \PYG{o}{\PYGZgt{}} \PYG{l+m+mf}{1.2}\PYG{p}{)}\PYG{o}{.}\PYG{n}{astype}\PYG{p}{(}\PYG{n+nb}{int}\PYG{p}{)} \PYG{c+c1}{\PYGZsh{} 장대 양봉을 데이터로 표현}
        \PYG{n}{data}\PYG{p}{[}\PYG{l+s+s1}{\PYGZsq{}}\PYG{l+s+s1}{num\PYGZus{}long}\PYG{l+s+s1}{\PYGZsq{}}\PYG{p}{]} \PYG{o}{=}  \PYG{n}{data}\PYG{p}{[}\PYG{l+s+s1}{\PYGZsq{}}\PYG{l+s+s1}{long\PYGZus{}candle}\PYG{l+s+s1}{\PYGZsq{}}\PYG{p}{]}\PYG{o}{.}\PYG{n}{rolling}\PYG{p}{(}\PYG{l+m+mi}{60}\PYG{p}{)}\PYG{o}{.}\PYG{n}{sum}\PYG{p}{(}\PYG{p}{)} \PYG{c+c1}{\PYGZsh{} 지난 20 일 동안 장대양봉의 갯 수}


         \PYG{c+c1}{\PYGZsh{} 거래량이 종좀 터지며 매집의 흔적을 보인다   }
        \PYG{n}{data}\PYG{p}{[}\PYG{l+s+s1}{\PYGZsq{}}\PYG{l+s+s1}{volume\PYGZus{}mean}\PYG{l+s+s1}{\PYGZsq{}}\PYG{p}{]} \PYG{o}{=} \PYG{n}{data}\PYG{p}{[}\PYG{l+s+s1}{\PYGZsq{}}\PYG{l+s+s1}{volume}\PYG{l+s+s1}{\PYGZsq{}}\PYG{p}{]}\PYG{o}{.}\PYG{n}{rolling}\PYG{p}{(}\PYG{l+m+mi}{60}\PYG{p}{)}\PYG{o}{.}\PYG{n}{mean}\PYG{p}{(}\PYG{p}{)}
        \PYG{n}{data}\PYG{p}{[}\PYG{l+s+s1}{\PYGZsq{}}\PYG{l+s+s1}{volume\PYGZus{}std}\PYG{l+s+s1}{\PYGZsq{}}\PYG{p}{]} \PYG{o}{=} \PYG{n}{data}\PYG{p}{[}\PYG{l+s+s1}{\PYGZsq{}}\PYG{l+s+s1}{volume}\PYG{l+s+s1}{\PYGZsq{}}\PYG{p}{]}\PYG{o}{.}\PYG{n}{rolling}\PYG{p}{(}\PYG{l+m+mi}{60}\PYG{p}{)}\PYG{o}{.}\PYG{n}{std}\PYG{p}{(}\PYG{p}{)}
        \PYG{n}{data}\PYG{p}{[}\PYG{l+s+s1}{\PYGZsq{}}\PYG{l+s+s1}{volume\PYGZus{}z}\PYG{l+s+s1}{\PYGZsq{}}\PYG{p}{]} \PYG{o}{=} \PYG{p}{(}\PYG{n}{data}\PYG{p}{[}\PYG{l+s+s1}{\PYGZsq{}}\PYG{l+s+s1}{volume}\PYG{l+s+s1}{\PYGZsq{}}\PYG{p}{]} \PYG{o}{\PYGZhy{}} \PYG{n}{data}\PYG{p}{[}\PYG{l+s+s1}{\PYGZsq{}}\PYG{l+s+s1}{volume\PYGZus{}mean}\PYG{l+s+s1}{\PYGZsq{}}\PYG{p}{]}\PYG{p}{)}\PYG{o}{/}\PYG{n}{data}\PYG{p}{[}\PYG{l+s+s1}{\PYGZsq{}}\PYG{l+s+s1}{volume\PYGZus{}std}\PYG{l+s+s1}{\PYGZsq{}}\PYG{p}{]} \PYG{c+c1}{\PYGZsh{} 거래량은 종목과 주가에 따라 다르기 떄문에 표준화한 값이 필요함}
        \PYG{n}{data}\PYG{p}{[}\PYG{l+s+s1}{\PYGZsq{}}\PYG{l+s+s1}{z\PYGZgt{}1.96}\PYG{l+s+s1}{\PYGZsq{}}\PYG{p}{]} \PYG{o}{=} \PYG{p}{(}\PYG{n}{data}\PYG{p}{[}\PYG{l+s+s1}{\PYGZsq{}}\PYG{l+s+s1}{close}\PYG{l+s+s1}{\PYGZsq{}}\PYG{p}{]} \PYG{o}{\PYGZgt{}} \PYG{n}{data}\PYG{p}{[}\PYG{l+s+s1}{\PYGZsq{}}\PYG{l+s+s1}{open}\PYG{l+s+s1}{\PYGZsq{}}\PYG{p}{]}\PYG{p}{)}\PYG{o}{*}\PYG{p}{(}\PYG{n}{data}\PYG{p}{[}\PYG{l+s+s1}{\PYGZsq{}}\PYG{l+s+s1}{volume\PYGZus{}z}\PYG{l+s+s1}{\PYGZsq{}}\PYG{p}{]} \PYG{o}{\PYGZgt{}} \PYG{l+m+mf}{1.65}\PYG{p}{)}\PYG{o}{.}\PYG{n}{astype}\PYG{p}{(}\PYG{n+nb}{int}\PYG{p}{)} \PYG{c+c1}{\PYGZsh{} 양봉이면서 거래량이 90\PYGZpc{}신뢰구간을 벗어난 날}
        \PYG{n}{data}\PYG{p}{[}\PYG{l+s+s1}{\PYGZsq{}}\PYG{l+s+s1}{num\PYGZus{}z\PYGZgt{}1.96}\PYG{l+s+s1}{\PYGZsq{}}\PYG{p}{]} \PYG{o}{=}  \PYG{n}{data}\PYG{p}{[}\PYG{l+s+s1}{\PYGZsq{}}\PYG{l+s+s1}{z\PYGZgt{}1.96}\PYG{l+s+s1}{\PYGZsq{}}\PYG{p}{]}\PYG{o}{.}\PYG{n}{rolling}\PYG{p}{(}\PYG{l+m+mi}{60}\PYG{p}{)}\PYG{o}{.}\PYG{n}{sum}\PYG{p}{(}\PYG{p}{)}  \PYG{c+c1}{\PYGZsh{} 양봉이면서 거래량이 90\PYGZpc{} 신뢰구간을 벗어난 날을 카운트}

        \PYG{c+c1}{\PYGZsh{} 주가지수보다 더 좋은 수익율을 보여준다}
        \PYG{n}{data}\PYG{p}{[}\PYG{l+s+s1}{\PYGZsq{}}\PYG{l+s+s1}{num\PYGZus{}win\PYGZus{}market}\PYG{l+s+s1}{\PYGZsq{}}\PYG{p}{]} \PYG{o}{=} \PYG{n}{data}\PYG{p}{[}\PYG{l+s+s1}{\PYGZsq{}}\PYG{l+s+s1}{win\PYGZus{}market}\PYG{l+s+s1}{\PYGZsq{}}\PYG{p}{]}\PYG{o}{.}\PYG{n}{rolling}\PYG{p}{(}\PYG{l+m+mi}{60}\PYG{p}{)}\PYG{o}{.}\PYG{n}{sum}\PYG{p}{(}\PYG{p}{)} \PYG{c+c1}{\PYGZsh{} 주가지수 수익율이 1 보다 작을 때, 종목 수익율이 1 보다 큰 날 수}
        \PYG{n}{data}\PYG{p}{[}\PYG{l+s+s1}{\PYGZsq{}}\PYG{l+s+s1}{pct\PYGZus{}win\PYGZus{}market}\PYG{l+s+s1}{\PYGZsq{}}\PYG{p}{]} \PYG{o}{=} \PYG{p}{(}\PYG{n}{data}\PYG{p}{[}\PYG{l+s+s1}{\PYGZsq{}}\PYG{l+s+s1}{return}\PYG{l+s+s1}{\PYGZsq{}}\PYG{p}{]}\PYG{o}{/}\PYG{n}{data}\PYG{p}{[}\PYG{l+s+s1}{\PYGZsq{}}\PYG{l+s+s1}{kosdaq\PYGZus{}return}\PYG{l+s+s1}{\PYGZsq{}}\PYG{p}{]}\PYG{p}{)}\PYG{o}{.}\PYG{n}{rolling}\PYG{p}{(}\PYG{l+m+mi}{60}\PYG{p}{)}\PYG{o}{.}\PYG{n}{mean}\PYG{p}{(}\PYG{p}{)} \PYG{c+c1}{\PYGZsh{} 주가지수 수익율 대비 종목 수익율}


        \PYG{c+c1}{\PYGZsh{} 동종업체 수익률보다 더 좋은 수익율을 보여준다.           }
        \PYG{n}{data}\PYG{p}{[}\PYG{l+s+s1}{\PYGZsq{}}\PYG{l+s+s1}{return\PYGZus{}mean}\PYG{l+s+s1}{\PYGZsq{}}\PYG{p}{]} \PYG{o}{=} \PYG{n}{data}\PYG{p}{[}\PYG{l+s+s1}{\PYGZsq{}}\PYG{l+s+s1}{return}\PYG{l+s+s1}{\PYGZsq{}}\PYG{p}{]}\PYG{o}{.}\PYG{n}{rolling}\PYG{p}{(}\PYG{l+m+mi}{60}\PYG{p}{)}\PYG{o}{.}\PYG{n}{mean}\PYG{p}{(}\PYG{p}{)} \PYG{c+c1}{\PYGZsh{} 종목별 최근 60 일 수익율의 평균}
        \PYG{n}{data}\PYG{p}{[}\PYG{l+s+s1}{\PYGZsq{}}\PYG{l+s+s1}{sector}\PYG{l+s+s1}{\PYGZsq{}}\PYG{p}{]} \PYG{o}{=} \PYG{n}{sector}    
        \PYG{n}{data}\PYG{p}{[}\PYG{l+s+s1}{\PYGZsq{}}\PYG{l+s+s1}{name}\PYG{l+s+s1}{\PYGZsq{}}\PYG{p}{]} \PYG{o}{=} \PYG{n}{name}

        \PYG{n}{data} \PYG{o}{=} \PYG{n}{data}\PYG{p}{[}\PYG{p}{(}\PYG{n}{data}\PYG{p}{[}\PYG{l+s+s1}{\PYGZsq{}}\PYG{l+s+s1}{price\PYGZus{}std}\PYG{l+s+s1}{\PYGZsq{}}\PYG{p}{]}\PYG{o}{!=}\PYG{l+m+mi}{0}\PYG{p}{)} \PYG{o}{\PYGZam{}} \PYG{p}{(}\PYG{n}{data}\PYG{p}{[}\PYG{l+s+s1}{\PYGZsq{}}\PYG{l+s+s1}{volume\PYGZus{}std}\PYG{l+s+s1}{\PYGZsq{}}\PYG{p}{]}\PYG{o}{!=}\PYG{l+m+mi}{0}\PYG{p}{)}\PYG{p}{]}    

        \PYG{n}{model\PYGZus{}inputs} \PYG{o}{=} \PYG{n}{pd}\PYG{o}{.}\PYG{n}{concat}\PYG{p}{(}\PYG{p}{[}\PYG{n}{data}\PYG{p}{,} \PYG{n}{model\PYGZus{}inputs}\PYG{p}{]}\PYG{p}{,} \PYG{n}{axis}\PYG{o}{=}\PYG{l+m+mi}{0}\PYG{p}{)}

    \PYG{n}{model\PYGZus{}inputs}\PYG{p}{[}\PYG{l+s+s1}{\PYGZsq{}}\PYG{l+s+s1}{sector\PYGZus{}return}\PYG{l+s+s1}{\PYGZsq{}}\PYG{p}{]} \PYG{o}{=} \PYG{n}{model\PYGZus{}inputs}\PYG{o}{.}\PYG{n}{groupby}\PYG{p}{(}\PYG{p}{[}\PYG{l+s+s1}{\PYGZsq{}}\PYG{l+s+s1}{sector}\PYG{l+s+s1}{\PYGZsq{}}\PYG{p}{,} \PYG{n}{model\PYGZus{}inputs}\PYG{o}{.}\PYG{n}{index}\PYG{p}{]}\PYG{p}{)}\PYG{p}{[}\PYG{l+s+s1}{\PYGZsq{}}\PYG{l+s+s1}{return}\PYG{l+s+s1}{\PYGZsq{}}\PYG{p}{]}\PYG{o}{.}\PYG{n}{transform}\PYG{p}{(}\PYG{k}{lambda} \PYG{n}{x}\PYG{p}{:} \PYG{n}{x}\PYG{o}{.}\PYG{n}{mean}\PYG{p}{(}\PYG{p}{)}\PYG{p}{)} \PYG{c+c1}{\PYGZsh{} 섹터의 평균 수익율 계산}
    \PYG{n}{model\PYGZus{}inputs}\PYG{p}{[}\PYG{l+s+s1}{\PYGZsq{}}\PYG{l+s+s1}{return over sector}\PYG{l+s+s1}{\PYGZsq{}}\PYG{p}{]} \PYG{o}{=} \PYG{p}{(}\PYG{n}{model\PYGZus{}inputs}\PYG{p}{[}\PYG{l+s+s1}{\PYGZsq{}}\PYG{l+s+s1}{return}\PYG{l+s+s1}{\PYGZsq{}}\PYG{p}{]}\PYG{o}{/}\PYG{n}{model\PYGZus{}inputs}\PYG{p}{[}\PYG{l+s+s1}{\PYGZsq{}}\PYG{l+s+s1}{sector\PYGZus{}return}\PYG{l+s+s1}{\PYGZsq{}}\PYG{p}{]}\PYG{p}{)} \PYG{c+c1}{\PYGZsh{} 섹터 평균 수익률 대비 종목 수익률 계산}
    \PYG{n}{model\PYGZus{}inputs}\PYG{o}{.}\PYG{n}{dropna}\PYG{p}{(}\PYG{n}{inplace}\PYG{o}{=}\PYG{k+kc}{True}\PYG{p}{)} \PYG{c+c1}{\PYGZsh{} Missing 값 있는 행 모두 제거}


    \PYG{n}{feature\PYGZus{}list} \PYG{o}{=} \PYG{p}{[}\PYG{l+s+s1}{\PYGZsq{}}\PYG{l+s+s1}{price\PYGZus{}z}\PYG{l+s+s1}{\PYGZsq{}}\PYG{p}{,}\PYG{l+s+s1}{\PYGZsq{}}\PYG{l+s+s1}{volume\PYGZus{}z}\PYG{l+s+s1}{\PYGZsq{}}\PYG{p}{,}\PYG{l+s+s1}{\PYGZsq{}}\PYG{l+s+s1}{num\PYGZus{}high/close}\PYG{l+s+s1}{\PYGZsq{}}\PYG{p}{,}\PYG{l+s+s1}{\PYGZsq{}}\PYG{l+s+s1}{num\PYGZus{}win\PYGZus{}market}\PYG{l+s+s1}{\PYGZsq{}}\PYG{p}{,}\PYG{l+s+s1}{\PYGZsq{}}\PYG{l+s+s1}{pct\PYGZus{}win\PYGZus{}market}\PYG{l+s+s1}{\PYGZsq{}}\PYG{p}{,}\PYG{l+s+s1}{\PYGZsq{}}\PYG{l+s+s1}{return over sector}\PYG{l+s+s1}{\PYGZsq{}}\PYG{p}{]}

    \PYG{n}{X} \PYG{o}{=} \PYG{n}{model\PYGZus{}inputs}\PYG{o}{.}\PYG{n}{loc}\PYG{p}{[}\PYG{n}{today\PYGZus{}dt}\PYG{p}{]}\PYG{p}{[}\PYG{p}{[}\PYG{l+s+s1}{\PYGZsq{}}\PYG{l+s+s1}{code}\PYG{l+s+s1}{\PYGZsq{}}\PYG{p}{,}\PYG{l+s+s1}{\PYGZsq{}}\PYG{l+s+s1}{name}\PYG{l+s+s1}{\PYGZsq{}}\PYG{p}{,}\PYG{l+s+s1}{\PYGZsq{}}\PYG{l+s+s1}{return}\PYG{l+s+s1}{\PYGZsq{}}\PYG{p}{,}\PYG{l+s+s1}{\PYGZsq{}}\PYG{l+s+s1}{kosdaq\PYGZus{}return}\PYG{l+s+s1}{\PYGZsq{}}\PYG{p}{,}\PYG{l+s+s1}{\PYGZsq{}}\PYG{l+s+s1}{close}\PYG{l+s+s1}{\PYGZsq{}}\PYG{p}{]} \PYG{o}{+} \PYG{n}{feature\PYGZus{}list}\PYG{p}{]}\PYG{o}{.}\PYG{n}{set\PYGZus{}index}\PYG{p}{(}\PYG{l+s+s1}{\PYGZsq{}}\PYG{l+s+s1}{code}\PYG{l+s+s1}{\PYGZsq{}}\PYG{p}{)} 

    \PYG{k}{with} \PYG{n+nb}{open}\PYG{p}{(}\PYG{l+s+s2}{\PYGZdq{}}\PYG{l+s+s2}{gam.pkl}\PYG{l+s+s2}{\PYGZdq{}}\PYG{p}{,} \PYG{l+s+s2}{\PYGZdq{}}\PYG{l+s+s2}{rb}\PYG{l+s+s2}{\PYGZdq{}}\PYG{p}{)} \PYG{k}{as} \PYG{n}{file}\PYG{p}{:}
        \PYG{n}{gam} \PYG{o}{=} \PYG{n}{pickle}\PYG{o}{.}\PYG{n}{load}\PYG{p}{(}\PYG{n}{file}\PYG{p}{)}     

    \PYG{n}{yhat} \PYG{o}{=} \PYG{n}{gam}\PYG{o}{.}\PYG{n}{predict\PYGZus{}proba}\PYG{p}{(}\PYG{n}{X}\PYG{p}{[}\PYG{n}{feature\PYGZus{}list}\PYG{p}{]}\PYG{p}{)}
    \PYG{n}{X}\PYG{p}{[}\PYG{l+s+s1}{\PYGZsq{}}\PYG{l+s+s1}{yhat}\PYG{l+s+s1}{\PYGZsq{}}\PYG{p}{]} \PYG{o}{=} \PYG{n}{yhat}

    \PYG{n}{tops} \PYG{o}{=} \PYG{n}{X}\PYG{p}{[}\PYG{n}{X}\PYG{p}{[}\PYG{l+s+s1}{\PYGZsq{}}\PYG{l+s+s1}{yhat}\PYG{l+s+s1}{\PYGZsq{}}\PYG{p}{]} \PYG{o}{\PYGZgt{}}\PYG{o}{=} \PYG{l+m+mf}{0.3}\PYG{p}{]}\PYG{o}{.}\PYG{n}{sort\PYGZus{}values}\PYG{p}{(}\PYG{n}{by}\PYG{o}{=}\PYG{l+s+s1}{\PYGZsq{}}\PYG{l+s+s1}{yhat}\PYG{l+s+s1}{\PYGZsq{}}\PYG{p}{,} \PYG{n}{ascending}\PYG{o}{=}\PYG{k+kc}{False}\PYG{p}{)} \PYG{c+c1}{\PYGZsh{} 스코어 0.3 이상 종목만 }
    \PYG{n+nb}{print}\PYG{p}{(}\PYG{n+nb}{len}\PYG{p}{(}\PYG{n}{tops}\PYG{p}{)}\PYG{p}{)}    
     
    \PYG{n}{select\PYGZus{}tops} \PYG{o}{=} \PYG{n}{tops}\PYG{p}{[}\PYG{p}{(}\PYG{n}{tops}\PYG{p}{[}\PYG{l+s+s1}{\PYGZsq{}}\PYG{l+s+s1}{return}\PYG{l+s+s1}{\PYGZsq{}}\PYG{p}{]} \PYG{o}{\PYGZgt{}} \PYG{l+m+mf}{1.03}\PYG{p}{)} \PYG{o}{\PYGZam{}} \PYG{p}{(}\PYG{n}{tops}\PYG{p}{[}\PYG{l+s+s1}{\PYGZsq{}}\PYG{l+s+s1}{price\PYGZus{}z}\PYG{l+s+s1}{\PYGZsq{}}\PYG{p}{]} \PYG{o}{\PYGZlt{}} \PYG{l+m+mi}{0}\PYG{p}{)}\PYG{p}{]}\PYG{p}{[}\PYG{p}{[}\PYG{l+s+s1}{\PYGZsq{}}\PYG{l+s+s1}{name}\PYG{l+s+s1}{\PYGZsq{}}\PYG{p}{,}\PYG{l+s+s1}{\PYGZsq{}}\PYG{l+s+s1}{return}\PYG{l+s+s1}{\PYGZsq{}}\PYG{p}{,}\PYG{l+s+s1}{\PYGZsq{}}\PYG{l+s+s1}{price\PYGZus{}z}\PYG{l+s+s1}{\PYGZsq{}}\PYG{p}{,}\PYG{l+s+s1}{\PYGZsq{}}\PYG{l+s+s1}{yhat}\PYG{l+s+s1}{\PYGZsq{}}\PYG{p}{,}\PYG{l+s+s1}{\PYGZsq{}}\PYG{l+s+s1}{return}\PYG{l+s+s1}{\PYGZsq{}}\PYG{p}{,} \PYG{l+s+s1}{\PYGZsq{}}\PYG{l+s+s1}{kosdaq\PYGZus{}return}\PYG{l+s+s1}{\PYGZsq{}}\PYG{p}{,}\PYG{l+s+s1}{\PYGZsq{}}\PYG{l+s+s1}{close}\PYG{l+s+s1}{\PYGZsq{}}\PYG{p}{]}\PYG{p}{]}  \PYG{c+c1}{\PYGZsh{} 기본 필터링 조건   }
      
    \PYG{k}{if} \PYG{n+nb}{len}\PYG{p}{(}\PYG{n}{select\PYGZus{}tops}\PYG{p}{)} \PYG{o}{\PYGZgt{}} \PYG{l+m+mi}{1}\PYG{p}{:} \PYG{c+c1}{\PYGZsh{} 최소한 2개 종목 \PYGZhy{} 추천 리스크 분산        }
        \PYG{k}{return} \PYG{n}{select\PYGZus{}tops}    
    
    \PYG{k}{else}\PYG{p}{:}
        \PYG{k}{return} \PYG{k+kc}{None}
\end{sphinxVerbatim}

\end{sphinxuseclass}\end{sphinxVerbatimInput}

\end{sphinxuseclass}
\sphinxAtStartPar
 수익률 검정하는 프로세스도 하나의 함수로 구현합니다.

\begin{sphinxuseclass}{cell}\begin{sphinxVerbatimInput}

\begin{sphinxuseclass}{cell_input}
\begin{sphinxVerbatim}[commandchars=\\\{\}]
\PYG{k}{def} \PYG{n+nf}{outcome\PYGZus{}tops}\PYG{p}{(}\PYG{n}{select\PYGZus{}tops}\PYG{p}{,} \PYG{n}{today\PYGZus{}dt}\PYG{p}{,} \PYG{n}{end\PYGZus{}dt}\PYG{p}{)}\PYG{p}{:}   

    \PYG{n}{outcome\PYGZus{}data} \PYG{o}{=} \PYG{n}{pd}\PYG{o}{.}\PYG{n}{DataFrame}\PYG{p}{(}\PYG{p}{)}

    \PYG{k}{for} \PYG{n}{code} \PYG{o+ow}{in} \PYG{n+nb}{list}\PYG{p}{(}\PYG{n}{select\PYGZus{}tops}\PYG{o}{.}\PYG{n}{index}\PYG{p}{)}\PYG{p}{:}  \PYG{c+c1}{\PYGZsh{} 스코어가 생성된 모든 종목에서 대하여 반복}
        \PYG{n}{daily\PYGZus{}price} \PYG{o}{=} \PYG{n}{fdr}\PYG{o}{.}\PYG{n}{DataReader}\PYG{p}{(}\PYG{n}{code}\PYG{p}{,}  \PYG{n}{start} \PYG{o}{=} \PYG{n}{today\PYGZus{}dt}\PYG{p}{,} \PYG{n}{end} \PYG{o}{=} \PYG{n}{end\PYGZus{}dt}\PYG{p}{)} \PYG{c+c1}{\PYGZsh{} 종목, 일봉, 데이터 갯수}
        \PYG{n}{daily\PYGZus{}price}\PYG{p}{[}\PYG{l+s+s1}{\PYGZsq{}}\PYG{l+s+s1}{code}\PYG{l+s+s1}{\PYGZsq{}}\PYG{p}{]} \PYG{o}{=} \PYG{n}{code}  

        \PYG{n}{daily\PYGZus{}price}\PYG{p}{[}\PYG{l+s+s1}{\PYGZsq{}}\PYG{l+s+s1}{close\PYGZus{}r1}\PYG{l+s+s1}{\PYGZsq{}}\PYG{p}{]} \PYG{o}{=} \PYG{n}{daily\PYGZus{}price}\PYG{p}{[}\PYG{l+s+s1}{\PYGZsq{}}\PYG{l+s+s1}{Close}\PYG{l+s+s1}{\PYGZsq{}}\PYG{p}{]}\PYG{o}{.}\PYG{n}{shift}\PYG{p}{(}\PYG{o}{\PYGZhy{}}\PYG{l+m+mi}{1}\PYG{p}{)}\PYG{o}{/}\PYG{n}{daily\PYGZus{}price}\PYG{p}{[}\PYG{l+s+s1}{\PYGZsq{}}\PYG{l+s+s1}{Close}\PYG{l+s+s1}{\PYGZsq{}}\PYG{p}{]}   
        \PYG{n}{daily\PYGZus{}price}\PYG{p}{[}\PYG{l+s+s1}{\PYGZsq{}}\PYG{l+s+s1}{close\PYGZus{}r2}\PYG{l+s+s1}{\PYGZsq{}}\PYG{p}{]} \PYG{o}{=} \PYG{n}{daily\PYGZus{}price}\PYG{p}{[}\PYG{l+s+s1}{\PYGZsq{}}\PYG{l+s+s1}{Close}\PYG{l+s+s1}{\PYGZsq{}}\PYG{p}{]}\PYG{o}{.}\PYG{n}{shift}\PYG{p}{(}\PYG{o}{\PYGZhy{}}\PYG{l+m+mi}{2}\PYG{p}{)}\PYG{o}{/}\PYG{n}{daily\PYGZus{}price}\PYG{p}{[}\PYG{l+s+s1}{\PYGZsq{}}\PYG{l+s+s1}{Close}\PYG{l+s+s1}{\PYGZsq{}}\PYG{p}{]}  
        \PYG{n}{daily\PYGZus{}price}\PYG{p}{[}\PYG{l+s+s1}{\PYGZsq{}}\PYG{l+s+s1}{close\PYGZus{}r3}\PYG{l+s+s1}{\PYGZsq{}}\PYG{p}{]} \PYG{o}{=} \PYG{n}{daily\PYGZus{}price}\PYG{p}{[}\PYG{l+s+s1}{\PYGZsq{}}\PYG{l+s+s1}{Close}\PYG{l+s+s1}{\PYGZsq{}}\PYG{p}{]}\PYG{o}{.}\PYG{n}{shift}\PYG{p}{(}\PYG{o}{\PYGZhy{}}\PYG{l+m+mi}{3}\PYG{p}{)}\PYG{o}{/}\PYG{n}{daily\PYGZus{}price}\PYG{p}{[}\PYG{l+s+s1}{\PYGZsq{}}\PYG{l+s+s1}{Close}\PYG{l+s+s1}{\PYGZsq{}}\PYG{p}{]}   
        \PYG{n}{daily\PYGZus{}price}\PYG{p}{[}\PYG{l+s+s1}{\PYGZsq{}}\PYG{l+s+s1}{close\PYGZus{}r4}\PYG{l+s+s1}{\PYGZsq{}}\PYG{p}{]} \PYG{o}{=} \PYG{n}{daily\PYGZus{}price}\PYG{p}{[}\PYG{l+s+s1}{\PYGZsq{}}\PYG{l+s+s1}{Close}\PYG{l+s+s1}{\PYGZsq{}}\PYG{p}{]}\PYG{o}{.}\PYG{n}{shift}\PYG{p}{(}\PYG{o}{\PYGZhy{}}\PYG{l+m+mi}{4}\PYG{p}{)}\PYG{o}{/}\PYG{n}{daily\PYGZus{}price}\PYG{p}{[}\PYG{l+s+s1}{\PYGZsq{}}\PYG{l+s+s1}{Close}\PYG{l+s+s1}{\PYGZsq{}}\PYG{p}{]}   
        \PYG{n}{daily\PYGZus{}price}\PYG{p}{[}\PYG{l+s+s1}{\PYGZsq{}}\PYG{l+s+s1}{close\PYGZus{}r5}\PYG{l+s+s1}{\PYGZsq{}}\PYG{p}{]} \PYG{o}{=} \PYG{n}{daily\PYGZus{}price}\PYG{p}{[}\PYG{l+s+s1}{\PYGZsq{}}\PYG{l+s+s1}{Close}\PYG{l+s+s1}{\PYGZsq{}}\PYG{p}{]}\PYG{o}{.}\PYG{n}{shift}\PYG{p}{(}\PYG{o}{\PYGZhy{}}\PYG{l+m+mi}{5}\PYG{p}{)}\PYG{o}{/}\PYG{n}{daily\PYGZus{}price}\PYG{p}{[}\PYG{l+s+s1}{\PYGZsq{}}\PYG{l+s+s1}{Close}\PYG{l+s+s1}{\PYGZsq{}}\PYG{p}{]}   

        \PYG{n}{daily\PYGZus{}price}\PYG{p}{[}\PYG{l+s+s1}{\PYGZsq{}}\PYG{l+s+s1}{max\PYGZus{}close}\PYG{l+s+s1}{\PYGZsq{}}\PYG{p}{]} \PYG{o}{=} \PYG{n}{daily\PYGZus{}price}\PYG{p}{[}\PYG{p}{[}\PYG{l+s+s1}{\PYGZsq{}}\PYG{l+s+s1}{close\PYGZus{}r1}\PYG{l+s+s1}{\PYGZsq{}}\PYG{p}{,}\PYG{l+s+s1}{\PYGZsq{}}\PYG{l+s+s1}{close\PYGZus{}r2}\PYG{l+s+s1}{\PYGZsq{}}\PYG{p}{,}\PYG{l+s+s1}{\PYGZsq{}}\PYG{l+s+s1}{close\PYGZus{}r3}\PYG{l+s+s1}{\PYGZsq{}}\PYG{p}{,}\PYG{l+s+s1}{\PYGZsq{}}\PYG{l+s+s1}{close\PYGZus{}r4}\PYG{l+s+s1}{\PYGZsq{}}\PYG{p}{,}\PYG{l+s+s1}{\PYGZsq{}}\PYG{l+s+s1}{close\PYGZus{}r5}\PYG{l+s+s1}{\PYGZsq{}}\PYG{p}{]}\PYG{p}{]}\PYG{o}{.}\PYG{n}{max}\PYG{p}{(}\PYG{n}{axis}\PYG{o}{=}\PYG{l+m+mi}{1}\PYG{p}{)}
        \PYG{n}{daily\PYGZus{}price}\PYG{p}{[}\PYG{l+s+s1}{\PYGZsq{}}\PYG{l+s+s1}{mean\PYGZus{}close}\PYG{l+s+s1}{\PYGZsq{}}\PYG{p}{]} \PYG{o}{=} \PYG{n}{daily\PYGZus{}price}\PYG{p}{[}\PYG{p}{[}\PYG{l+s+s1}{\PYGZsq{}}\PYG{l+s+s1}{close\PYGZus{}r1}\PYG{l+s+s1}{\PYGZsq{}}\PYG{p}{,}\PYG{l+s+s1}{\PYGZsq{}}\PYG{l+s+s1}{close\PYGZus{}r2}\PYG{l+s+s1}{\PYGZsq{}}\PYG{p}{,}\PYG{l+s+s1}{\PYGZsq{}}\PYG{l+s+s1}{close\PYGZus{}r3}\PYG{l+s+s1}{\PYGZsq{}}\PYG{p}{,}\PYG{l+s+s1}{\PYGZsq{}}\PYG{l+s+s1}{close\PYGZus{}r4}\PYG{l+s+s1}{\PYGZsq{}}\PYG{p}{,}\PYG{l+s+s1}{\PYGZsq{}}\PYG{l+s+s1}{close\PYGZus{}r5}\PYG{l+s+s1}{\PYGZsq{}}\PYG{p}{]}\PYG{p}{]}\PYG{o}{.}\PYG{n}{mean}\PYG{p}{(}\PYG{n}{axis}\PYG{o}{=}\PYG{l+m+mi}{1}\PYG{p}{)}
        \PYG{n}{daily\PYGZus{}price}\PYG{p}{[}\PYG{l+s+s1}{\PYGZsq{}}\PYG{l+s+s1}{min\PYGZus{}close}\PYG{l+s+s1}{\PYGZsq{}}\PYG{p}{]} \PYG{o}{=} \PYG{n}{daily\PYGZus{}price}\PYG{p}{[}\PYG{p}{[}\PYG{l+s+s1}{\PYGZsq{}}\PYG{l+s+s1}{close\PYGZus{}r1}\PYG{l+s+s1}{\PYGZsq{}}\PYG{p}{,}\PYG{l+s+s1}{\PYGZsq{}}\PYG{l+s+s1}{close\PYGZus{}r2}\PYG{l+s+s1}{\PYGZsq{}}\PYG{p}{,}\PYG{l+s+s1}{\PYGZsq{}}\PYG{l+s+s1}{close\PYGZus{}r3}\PYG{l+s+s1}{\PYGZsq{}}\PYG{p}{,}\PYG{l+s+s1}{\PYGZsq{}}\PYG{l+s+s1}{close\PYGZus{}r4}\PYG{l+s+s1}{\PYGZsq{}}\PYG{p}{,}\PYG{l+s+s1}{\PYGZsq{}}\PYG{l+s+s1}{close\PYGZus{}r5}\PYG{l+s+s1}{\PYGZsq{}}\PYG{p}{]}\PYG{p}{]}\PYG{o}{.}\PYG{n}{min}\PYG{p}{(}\PYG{n}{axis}\PYG{o}{=}\PYG{l+m+mi}{1}\PYG{p}{)}

        \PYG{n}{daily\PYGZus{}price}\PYG{p}{[}\PYG{l+s+s1}{\PYGZsq{}}\PYG{l+s+s1}{buy\PYGZus{}price}\PYG{l+s+s1}{\PYGZsq{}}\PYG{p}{]} \PYG{o}{=} \PYG{n}{daily\PYGZus{}price}\PYG{p}{[}\PYG{l+s+s1}{\PYGZsq{}}\PYG{l+s+s1}{Close}\PYG{l+s+s1}{\PYGZsq{}}\PYG{p}{]}
        \PYG{n}{daily\PYGZus{}price}\PYG{p}{[}\PYG{l+s+s1}{\PYGZsq{}}\PYG{l+s+s1}{buy\PYGZus{}low}\PYG{l+s+s1}{\PYGZsq{}}\PYG{p}{]} \PYG{o}{=} \PYG{n}{daily\PYGZus{}price}\PYG{p}{[}\PYG{l+s+s1}{\PYGZsq{}}\PYG{l+s+s1}{Low}\PYG{l+s+s1}{\PYGZsq{}}\PYG{p}{]}\PYG{o}{.}\PYG{n}{shift}\PYG{p}{(}\PYG{o}{\PYGZhy{}}\PYG{l+m+mi}{1}\PYG{p}{)} \PYG{c+c1}{\PYGZsh{} 익일 저가}
        \PYG{n}{daily\PYGZus{}price}\PYG{p}{[}\PYG{l+s+s1}{\PYGZsq{}}\PYG{l+s+s1}{buy\PYGZus{}high}\PYG{l+s+s1}{\PYGZsq{}}\PYG{p}{]} \PYG{o}{=} \PYG{n}{daily\PYGZus{}price}\PYG{p}{[}\PYG{l+s+s1}{\PYGZsq{}}\PYG{l+s+s1}{High}\PYG{l+s+s1}{\PYGZsq{}}\PYG{p}{]}\PYG{o}{.}\PYG{n}{shift}\PYG{p}{(}\PYG{o}{\PYGZhy{}}\PYG{l+m+mi}{1}\PYG{p}{)} \PYG{c+c1}{\PYGZsh{} 익일 고가}

        \PYG{n}{daily\PYGZus{}price}\PYG{p}{[}\PYG{l+s+s1}{\PYGZsq{}}\PYG{l+s+s1}{buy}\PYG{l+s+s1}{\PYGZsq{}}\PYG{p}{]} \PYG{o}{=} \PYG{n}{np}\PYG{o}{.}\PYG{n}{where}\PYG{p}{(}\PYG{p}{(}\PYG{n}{daily\PYGZus{}price}\PYG{p}{[}\PYG{l+s+s1}{\PYGZsq{}}\PYG{l+s+s1}{buy\PYGZus{}price}\PYG{l+s+s1}{\PYGZsq{}}\PYG{p}{]}\PYG{o}{.}\PYG{n}{between}\PYG{p}{(}\PYG{n}{daily\PYGZus{}price}\PYG{p}{[}\PYG{l+s+s1}{\PYGZsq{}}\PYG{l+s+s1}{buy\PYGZus{}low}\PYG{l+s+s1}{\PYGZsq{}}\PYG{p}{]}\PYG{p}{,} \PYG{n}{daily\PYGZus{}price}\PYG{p}{[}\PYG{l+s+s1}{\PYGZsq{}}\PYG{l+s+s1}{buy\PYGZus{}high}\PYG{l+s+s1}{\PYGZsq{}}\PYG{p}{]}\PYG{p}{)}\PYG{p}{)}\PYG{p}{,} \PYG{l+m+mi}{1}\PYG{p}{,} \PYG{l+m+mi}{0}\PYG{p}{)}  \PYG{c+c1}{\PYGZsh{} 당일 종가로 익일 매수 가능한지 여부}

        \PYG{n}{outcome\PYGZus{}data} \PYG{o}{=} \PYG{n}{pd}\PYG{o}{.}\PYG{n}{concat}\PYG{p}{(}\PYG{p}{[}\PYG{n}{outcome\PYGZus{}data}\PYG{p}{,} \PYG{n}{daily\PYGZus{}price}\PYG{p}{]}\PYG{p}{,} \PYG{n}{axis}\PYG{o}{=}\PYG{l+m+mi}{0}\PYG{p}{)}

    \PYG{n}{outcome} \PYG{o}{=} \PYG{n}{outcome\PYGZus{}data}\PYG{o}{.}\PYG{n}{loc}\PYG{p}{[}\PYG{n}{today\PYGZus{}dt}\PYG{p}{]}\PYG{p}{[}\PYG{p}{[}\PYG{l+s+s1}{\PYGZsq{}}\PYG{l+s+s1}{code}\PYG{l+s+s1}{\PYGZsq{}}\PYG{p}{,}\PYG{l+s+s1}{\PYGZsq{}}\PYG{l+s+s1}{buy}\PYG{l+s+s1}{\PYGZsq{}}\PYG{p}{,}\PYG{l+s+s1}{\PYGZsq{}}\PYG{l+s+s1}{buy\PYGZus{}price}\PYG{l+s+s1}{\PYGZsq{}}\PYG{p}{,}\PYG{l+s+s1}{\PYGZsq{}}\PYG{l+s+s1}{buy\PYGZus{}low}\PYG{l+s+s1}{\PYGZsq{}}\PYG{p}{,}\PYG{l+s+s1}{\PYGZsq{}}\PYG{l+s+s1}{buy\PYGZus{}high}\PYG{l+s+s1}{\PYGZsq{}}\PYG{p}{,}\PYG{l+s+s1}{\PYGZsq{}}\PYG{l+s+s1}{max\PYGZus{}close}\PYG{l+s+s1}{\PYGZsq{}}\PYG{p}{,}\PYG{l+s+s1}{\PYGZsq{}}\PYG{l+s+s1}{mean\PYGZus{}close}\PYG{l+s+s1}{\PYGZsq{}}\PYG{p}{,}\PYG{l+s+s1}{\PYGZsq{}}\PYG{l+s+s1}{min\PYGZus{}close}\PYG{l+s+s1}{\PYGZsq{}}\PYG{p}{]}\PYG{p}{]}\PYG{o}{.}\PYG{n}{set\PYGZus{}index}\PYG{p}{(}\PYG{l+s+s1}{\PYGZsq{}}\PYG{l+s+s1}{code}\PYG{l+s+s1}{\PYGZsq{}}\PYG{p}{)}
    \PYG{n}{select\PYGZus{}outcome} \PYG{o}{=} \PYG{n}{select\PYGZus{}tops}\PYG{o}{.}\PYG{n}{merge}\PYG{p}{(}\PYG{n}{outcome}\PYG{p}{,} \PYG{n}{left\PYGZus{}index}\PYG{o}{=}\PYG{k+kc}{True}\PYG{p}{,} \PYG{n}{right\PYGZus{}index}\PYG{o}{=}\PYG{k+kc}{True}\PYG{p}{,} \PYG{n}{how}\PYG{o}{=}\PYG{l+s+s1}{\PYGZsq{}}\PYG{l+s+s1}{inner}\PYG{l+s+s1}{\PYGZsq{}}\PYG{p}{)}

    \PYG{k}{return} \PYG{n}{select\PYGZus{}outcome}\PYG{p}{[}\PYG{p}{[}\PYG{l+s+s1}{\PYGZsq{}}\PYG{l+s+s1}{name}\PYG{l+s+s1}{\PYGZsq{}}\PYG{p}{,}\PYG{l+s+s1}{\PYGZsq{}}\PYG{l+s+s1}{buy}\PYG{l+s+s1}{\PYGZsq{}}\PYG{p}{,}\PYG{l+s+s1}{\PYGZsq{}}\PYG{l+s+s1}{buy\PYGZus{}price}\PYG{l+s+s1}{\PYGZsq{}}\PYG{p}{,} \PYG{l+s+s1}{\PYGZsq{}}\PYG{l+s+s1}{buy\PYGZus{}low}\PYG{l+s+s1}{\PYGZsq{}}\PYG{p}{,}\PYG{l+s+s1}{\PYGZsq{}}\PYG{l+s+s1}{buy\PYGZus{}high}\PYG{l+s+s1}{\PYGZsq{}}\PYG{p}{,}\PYG{l+s+s1}{\PYGZsq{}}\PYG{l+s+s1}{yhat}\PYG{l+s+s1}{\PYGZsq{}}\PYG{p}{,}\PYG{l+s+s1}{\PYGZsq{}}\PYG{l+s+s1}{max\PYGZus{}close}\PYG{l+s+s1}{\PYGZsq{}}\PYG{p}{,}\PYG{l+s+s1}{\PYGZsq{}}\PYG{l+s+s1}{mean\PYGZus{}close}\PYG{l+s+s1}{\PYGZsq{}}\PYG{p}{,}\PYG{l+s+s1}{\PYGZsq{}}\PYG{l+s+s1}{min\PYGZus{}close}\PYG{l+s+s1}{\PYGZsq{}}\PYG{p}{]}\PYG{p}{]}
\end{sphinxVerbatim}

\end{sphinxuseclass}\end{sphinxVerbatimInput}

\end{sphinxuseclass}
\sphinxAtStartPar
 \sphinxstylestrong{2022년 4월 1일 \sphinxhyphen{} 종목 선정 및 수익률 테스트}상당이 고무적입니다. 모든 종목이 익절이 가능합니다. 단 CSA 코스믹은 전일 종가로 당일 매수가 불가능합니다. 2022년 4월 2일 갭상승으로 시작을 했습니다.

\begin{sphinxuseclass}{cell}\begin{sphinxVerbatimInput}

\begin{sphinxuseclass}{cell_input}
\begin{sphinxVerbatim}[commandchars=\\\{\}]
\PYG{n}{select\PYGZus{}tops} \PYG{o}{=} \PYG{n}{select\PYGZus{}stocks}\PYG{p}{(}\PYG{l+s+s1}{\PYGZsq{}}\PYG{l+s+s1}{2022\PYGZhy{}04\PYGZhy{}01}\PYG{l+s+s1}{\PYGZsq{}}\PYG{p}{)}

\PYG{k}{if} \PYG{n}{select\PYGZus{}tops} \PYG{o+ow}{is} \PYG{o+ow}{not} \PYG{k+kc}{None}\PYG{p}{:}
    \PYG{n}{results} \PYG{o}{=} \PYG{n}{outcome\PYGZus{}tops}\PYG{p}{(}\PYG{n}{select\PYGZus{}tops}\PYG{p}{,} \PYG{l+s+s1}{\PYGZsq{}}\PYG{l+s+s1}{2022\PYGZhy{}04\PYGZhy{}01}\PYG{l+s+s1}{\PYGZsq{}}\PYG{p}{,} \PYG{l+s+s1}{\PYGZsq{}}\PYG{l+s+s1}{2022\PYGZhy{}04\PYGZhy{}08}\PYG{l+s+s1}{\PYGZsq{}}\PYG{p}{)} \PYG{c+c1}{\PYGZsh{} 5 영업일}
\PYG{n}{results}\PYG{o}{.}\PYG{n}{sort\PYGZus{}values}\PYG{p}{(}\PYG{n}{by}\PYG{o}{=}\PYG{l+s+s1}{\PYGZsq{}}\PYG{l+s+s1}{buy}\PYG{l+s+s1}{\PYGZsq{}}\PYG{p}{)}\PYG{o}{.}\PYG{n}{style}\PYG{o}{.}\PYG{n}{set\PYGZus{}table\PYGZus{}attributes}\PYG{p}{(}\PYG{l+s+s1}{\PYGZsq{}}\PYG{l+s+s1}{style=}\PYG{l+s+s1}{\PYGZdq{}}\PYG{l+s+s1}{font\PYGZhy{}size: 12px}\PYG{l+s+s1}{\PYGZdq{}}\PYG{l+s+s1}{\PYGZsq{}}\PYG{p}{)}\PYG{o}{.}\PYG{n}{format}\PYG{p}{(}\PYG{n}{precision}\PYG{o}{=}\PYG{l+m+mi}{3}\PYG{p}{)}
\end{sphinxVerbatim}

\end{sphinxuseclass}\end{sphinxVerbatimInput}
\begin{sphinxVerbatimOutput}

\begin{sphinxuseclass}{cell_output}
\begin{sphinxVerbatim}[commandchars=\\\{\}]
2021\PYGZhy{}12\PYGZhy{}22 00:00:00 2022\PYGZhy{}04\PYGZhy{}01
204
\end{sphinxVerbatim}

\begin{sphinxVerbatim}[commandchars=\\\{\}]
\PYGZlt{}pandas.io.formats.style.Styler at 0x172a4423ee0\PYGZgt{}
\end{sphinxVerbatim}

\end{sphinxuseclass}\end{sphinxVerbatimOutput}

\end{sphinxuseclass}
\sphinxAtStartPar
 \sphinxstylestrong{2022년 4월 18일 \sphinxhyphen{} 종목 선정 및 수익률 테스트}4 월 18일은 인성정보는 수익권, 웨이버스는 손절로 대응이 필요합니다.

\begin{sphinxuseclass}{cell}\begin{sphinxVerbatimInput}

\begin{sphinxuseclass}{cell_input}
\begin{sphinxVerbatim}[commandchars=\\\{\}]
\PYG{n}{select\PYGZus{}tops} \PYG{o}{=} \PYG{n}{select\PYGZus{}stocks}\PYG{p}{(}\PYG{l+s+s1}{\PYGZsq{}}\PYG{l+s+s1}{2022\PYGZhy{}04\PYGZhy{}18}\PYG{l+s+s1}{\PYGZsq{}}\PYG{p}{)}

\PYG{k}{if} \PYG{n}{select\PYGZus{}tops} \PYG{o+ow}{is} \PYG{o+ow}{not} \PYG{k+kc}{None}\PYG{p}{:}
    \PYG{n}{results} \PYG{o}{=} \PYG{n}{outcome\PYGZus{}tops}\PYG{p}{(}\PYG{n}{select\PYGZus{}tops}\PYG{p}{,} \PYG{l+s+s1}{\PYGZsq{}}\PYG{l+s+s1}{2022\PYGZhy{}04\PYGZhy{}18}\PYG{l+s+s1}{\PYGZsq{}}\PYG{p}{,} \PYG{l+s+s1}{\PYGZsq{}}\PYG{l+s+s1}{2022\PYGZhy{}04\PYGZhy{}25}\PYG{l+s+s1}{\PYGZsq{}}\PYG{p}{)} \PYG{c+c1}{\PYGZsh{} 5 영업일}
\PYG{n}{results}\PYG{o}{.}\PYG{n}{sort\PYGZus{}values}\PYG{p}{(}\PYG{n}{by}\PYG{o}{=}\PYG{l+s+s1}{\PYGZsq{}}\PYG{l+s+s1}{buy}\PYG{l+s+s1}{\PYGZsq{}}\PYG{p}{)}\PYG{o}{.}\PYG{n}{style}\PYG{o}{.}\PYG{n}{set\PYGZus{}table\PYGZus{}attributes}\PYG{p}{(}\PYG{l+s+s1}{\PYGZsq{}}\PYG{l+s+s1}{style=}\PYG{l+s+s1}{\PYGZdq{}}\PYG{l+s+s1}{font\PYGZhy{}size: 12px}\PYG{l+s+s1}{\PYGZdq{}}\PYG{l+s+s1}{\PYGZsq{}}\PYG{p}{)}\PYG{o}{.}\PYG{n}{format}\PYG{p}{(}\PYG{n}{precision}\PYG{o}{=}\PYG{l+m+mi}{3}\PYG{p}{)}
\end{sphinxVerbatim}

\end{sphinxuseclass}\end{sphinxVerbatimInput}
\begin{sphinxVerbatimOutput}

\begin{sphinxuseclass}{cell_output}
\begin{sphinxVerbatim}[commandchars=\\\{\}]
2022\PYGZhy{}01\PYGZhy{}08 00:00:00 2022\PYGZhy{}04\PYGZhy{}18
180
\end{sphinxVerbatim}

\begin{sphinxVerbatim}[commandchars=\\\{\}]
\PYGZlt{}pandas.io.formats.style.Styler at 0x172a83ace50\PYGZgt{}
\end{sphinxVerbatim}

\end{sphinxuseclass}\end{sphinxVerbatimOutput}

\end{sphinxuseclass}
\sphinxAtStartPar
 \sphinxstylestrong{2022년 5월 2일 \sphinxhyphen{} 종목 선정 및 수익률 테스트}미래생명자원은 매수 후, 주가가 하락하는 것으로 나왔습니다. 다행이 급락 종목은 아니여서 손절로 대응하는 것이 좋을 것으로 판단됩니다.

\begin{sphinxuseclass}{cell}\begin{sphinxVerbatimInput}

\begin{sphinxuseclass}{cell_input}
\begin{sphinxVerbatim}[commandchars=\\\{\}]
\PYG{n}{select\PYGZus{}tops} \PYG{o}{=} \PYG{n}{select\PYGZus{}stocks}\PYG{p}{(}\PYG{l+s+s1}{\PYGZsq{}}\PYG{l+s+s1}{2022\PYGZhy{}05\PYGZhy{}02}\PYG{l+s+s1}{\PYGZsq{}}\PYG{p}{)}

\PYG{k}{if} \PYG{n}{select\PYGZus{}tops} \PYG{o+ow}{is} \PYG{o+ow}{not} \PYG{k+kc}{None}\PYG{p}{:}
    \PYG{n}{results} \PYG{o}{=} \PYG{n}{outcome\PYGZus{}tops}\PYG{p}{(}\PYG{n}{select\PYGZus{}tops}\PYG{p}{,} \PYG{l+s+s1}{\PYGZsq{}}\PYG{l+s+s1}{2022\PYGZhy{}05\PYGZhy{}02}\PYG{l+s+s1}{\PYGZsq{}}\PYG{p}{,} \PYG{l+s+s1}{\PYGZsq{}}\PYG{l+s+s1}{2022\PYGZhy{}05\PYGZhy{}10}\PYG{l+s+s1}{\PYGZsq{}}\PYG{p}{)} \PYG{c+c1}{\PYGZsh{} 5 영업일 (5월 5일 어린이날)}
    
\PYG{n}{results}\PYG{o}{.}\PYG{n}{sort\PYGZus{}values}\PYG{p}{(}\PYG{n}{by}\PYG{o}{=}\PYG{l+s+s1}{\PYGZsq{}}\PYG{l+s+s1}{buy}\PYG{l+s+s1}{\PYGZsq{}}\PYG{p}{)}\PYG{o}{.}\PYG{n}{style}\PYG{o}{.}\PYG{n}{set\PYGZus{}table\PYGZus{}attributes}\PYG{p}{(}\PYG{l+s+s1}{\PYGZsq{}}\PYG{l+s+s1}{style=}\PYG{l+s+s1}{\PYGZdq{}}\PYG{l+s+s1}{font\PYGZhy{}size: 12px}\PYG{l+s+s1}{\PYGZdq{}}\PYG{l+s+s1}{\PYGZsq{}}\PYG{p}{)}\PYG{o}{.}\PYG{n}{format}\PYG{p}{(}\PYG{n}{precision}\PYG{o}{=}\PYG{l+m+mi}{3}\PYG{p}{)}
\end{sphinxVerbatim}

\end{sphinxuseclass}\end{sphinxVerbatimInput}
\begin{sphinxVerbatimOutput}

\begin{sphinxuseclass}{cell_output}
\begin{sphinxVerbatim}[commandchars=\\\{\}]
2022\PYGZhy{}01\PYGZhy{}22 00:00:00 2022\PYGZhy{}05\PYGZhy{}02
169
\end{sphinxVerbatim}

\begin{sphinxVerbatim}[commandchars=\\\{\}]
\PYGZlt{}pandas.io.formats.style.Styler at 0x172a6a01d60\PYGZgt{}
\end{sphinxVerbatim}

\end{sphinxuseclass}\end{sphinxVerbatimOutput}

\end{sphinxuseclass}
\sphinxAtStartPar
 \sphinxstylestrong{2022년 5월 9일 \sphinxhyphen{} 종목 선정 및 수익률 테스트}5월 9일은 추천종목이 없습니다.

\begin{sphinxuseclass}{cell}\begin{sphinxVerbatimInput}

\begin{sphinxuseclass}{cell_input}
\begin{sphinxVerbatim}[commandchars=\\\{\}]
\PYG{n}{select\PYGZus{}tops} \PYG{o}{=} \PYG{n}{select\PYGZus{}stocks}\PYG{p}{(}\PYG{l+s+s1}{\PYGZsq{}}\PYG{l+s+s1}{2022\PYGZhy{}05\PYGZhy{}09}\PYG{l+s+s1}{\PYGZsq{}}\PYG{p}{)}

\PYG{k}{if} \PYG{n}{select\PYGZus{}tops} \PYG{o+ow}{is} \PYG{o+ow}{not} \PYG{k+kc}{None}\PYG{p}{:}
    \PYG{n}{results} \PYG{o}{=} \PYG{n}{outcome\PYGZus{}tops}\PYG{p}{(}\PYG{n}{select\PYGZus{}tops}\PYG{p}{,} \PYG{l+s+s1}{\PYGZsq{}}\PYG{l+s+s1}{2022\PYGZhy{}05\PYGZhy{}09}\PYG{l+s+s1}{\PYGZsq{}}\PYG{p}{,} \PYG{l+s+s1}{\PYGZsq{}}\PYG{l+s+s1}{2022\PYGZhy{}05\PYGZhy{}16}\PYG{l+s+s1}{\PYGZsq{}}\PYG{p}{)} \PYG{c+c1}{\PYGZsh{} 5 영업일 (5월 5일 어린이날)}
    
\PYG{n}{results}\PYG{o}{.}\PYG{n}{sort\PYGZus{}values}\PYG{p}{(}\PYG{n}{by}\PYG{o}{=}\PYG{l+s+s1}{\PYGZsq{}}\PYG{l+s+s1}{buy}\PYG{l+s+s1}{\PYGZsq{}}\PYG{p}{)}\PYG{o}{.}\PYG{n}{style}\PYG{o}{.}\PYG{n}{set\PYGZus{}table\PYGZus{}attributes}\PYG{p}{(}\PYG{l+s+s1}{\PYGZsq{}}\PYG{l+s+s1}{style=}\PYG{l+s+s1}{\PYGZdq{}}\PYG{l+s+s1}{font\PYGZhy{}size: 12px}\PYG{l+s+s1}{\PYGZdq{}}\PYG{l+s+s1}{\PYGZsq{}}\PYG{p}{)}\PYG{o}{.}\PYG{n}{format}\PYG{p}{(}\PYG{n}{precision}\PYG{o}{=}\PYG{l+m+mi}{3}\PYG{p}{)}
\end{sphinxVerbatim}

\end{sphinxuseclass}\end{sphinxVerbatimInput}
\begin{sphinxVerbatimOutput}

\begin{sphinxuseclass}{cell_output}
\begin{sphinxVerbatim}[commandchars=\\\{\}]
2022\PYGZhy{}01\PYGZhy{}29 00:00:00 2022\PYGZhy{}05\PYGZhy{}09
348
\end{sphinxVerbatim}

\begin{sphinxVerbatim}[commandchars=\\\{\}]
\PYGZlt{}pandas.io.formats.style.Styler at 0x172a800ffd0\PYGZgt{}
\end{sphinxVerbatim}

\end{sphinxuseclass}\end{sphinxVerbatimOutput}

\end{sphinxuseclass}
\sphinxAtStartPar
 \sphinxstylestrong{2022년 5월 25일 \sphinxhyphen{} 종목 선정 및 수익률 테스트}지더블유바이텍과 아이에스이커머스는 5영업일이내 익절이 가능할 것으로 보입니다. 조이시티와 상지카일룸은 대응이 필요합니다.

\begin{sphinxuseclass}{cell}\begin{sphinxVerbatimInput}

\begin{sphinxuseclass}{cell_input}
\begin{sphinxVerbatim}[commandchars=\\\{\}]
\PYG{n}{select\PYGZus{}tops} \PYG{o}{=} \PYG{n}{select\PYGZus{}stocks}\PYG{p}{(}\PYG{l+s+s1}{\PYGZsq{}}\PYG{l+s+s1}{2022\PYGZhy{}05\PYGZhy{}25}\PYG{l+s+s1}{\PYGZsq{}}\PYG{p}{)}

\PYG{k}{if} \PYG{n}{select\PYGZus{}tops} \PYG{o+ow}{is} \PYG{o+ow}{not} \PYG{k+kc}{None}\PYG{p}{:}
    \PYG{n}{results} \PYG{o}{=} \PYG{n}{outcome\PYGZus{}tops}\PYG{p}{(}\PYG{n}{select\PYGZus{}tops}\PYG{p}{,} \PYG{l+s+s1}{\PYGZsq{}}\PYG{l+s+s1}{2022\PYGZhy{}05\PYGZhy{}25}\PYG{l+s+s1}{\PYGZsq{}}\PYG{p}{,} \PYG{l+s+s1}{\PYGZsq{}}\PYG{l+s+s1}{2022\PYGZhy{}06\PYGZhy{}02}\PYG{l+s+s1}{\PYGZsq{}}\PYG{p}{)} \PYG{c+c1}{\PYGZsh{} 5 영업일 (6월 1일 지방선거)          }
    
\PYG{n}{results}\PYG{o}{.}\PYG{n}{sort\PYGZus{}values}\PYG{p}{(}\PYG{n}{by}\PYG{o}{=}\PYG{l+s+s1}{\PYGZsq{}}\PYG{l+s+s1}{buy}\PYG{l+s+s1}{\PYGZsq{}}\PYG{p}{)}\PYG{o}{.}\PYG{n}{style}\PYG{o}{.}\PYG{n}{set\PYGZus{}table\PYGZus{}attributes}\PYG{p}{(}\PYG{l+s+s1}{\PYGZsq{}}\PYG{l+s+s1}{style=}\PYG{l+s+s1}{\PYGZdq{}}\PYG{l+s+s1}{font\PYGZhy{}size: 12px}\PYG{l+s+s1}{\PYGZdq{}}\PYG{l+s+s1}{\PYGZsq{}}\PYG{p}{)}\PYG{o}{.}\PYG{n}{format}\PYG{p}{(}\PYG{n}{precision}\PYG{o}{=}\PYG{l+m+mi}{3}\PYG{p}{)}
\end{sphinxVerbatim}

\end{sphinxuseclass}\end{sphinxVerbatimInput}
\begin{sphinxVerbatimOutput}

\begin{sphinxuseclass}{cell_output}
\begin{sphinxVerbatim}[commandchars=\\\{\}]
2022\PYGZhy{}02\PYGZhy{}14 00:00:00 2022\PYGZhy{}05\PYGZhy{}25
144
\end{sphinxVerbatim}

\begin{sphinxVerbatim}[commandchars=\\\{\}]
\PYGZlt{}pandas.io.formats.style.Styler at 0x172a7a17760\PYGZgt{}
\end{sphinxVerbatim}

\end{sphinxuseclass}\end{sphinxVerbatimOutput}

\end{sphinxuseclass}
\sphinxAtStartPar
 \sphinxstylestrong{2022년 6월 2일 \sphinxhyphen{} 종목 선정 및 수익률 테스트}토탈소프트를 제외한 모든 종목이 익절이 가능할 것으로 보입니다.

\begin{sphinxuseclass}{cell}\begin{sphinxVerbatimInput}

\begin{sphinxuseclass}{cell_input}
\begin{sphinxVerbatim}[commandchars=\\\{\}]
\PYG{n}{select\PYGZus{}tops} \PYG{o}{=} \PYG{n}{select\PYGZus{}stocks}\PYG{p}{(}\PYG{l+s+s1}{\PYGZsq{}}\PYG{l+s+s1}{2022\PYGZhy{}06\PYGZhy{}02}\PYG{l+s+s1}{\PYGZsq{}}\PYG{p}{)}

\PYG{k}{if} \PYG{n}{select\PYGZus{}tops} \PYG{o+ow}{is} \PYG{o+ow}{not} \PYG{k+kc}{None}\PYG{p}{:}
    \PYG{n}{results} \PYG{o}{=} \PYG{n}{outcome\PYGZus{}tops}\PYG{p}{(}\PYG{n}{select\PYGZus{}tops}\PYG{p}{,} \PYG{l+s+s1}{\PYGZsq{}}\PYG{l+s+s1}{2022\PYGZhy{}06\PYGZhy{}02}\PYG{l+s+s1}{\PYGZsq{}}\PYG{p}{,} \PYG{l+s+s1}{\PYGZsq{}}\PYG{l+s+s1}{2022\PYGZhy{}06\PYGZhy{}10}\PYG{l+s+s1}{\PYGZsq{}}\PYG{p}{)} \PYG{c+c1}{\PYGZsh{} 5 영업일 (6월 6일 현충일)}
    
\PYG{n}{results}\PYG{o}{.}\PYG{n}{sort\PYGZus{}values}\PYG{p}{(}\PYG{n}{by}\PYG{o}{=}\PYG{l+s+s1}{\PYGZsq{}}\PYG{l+s+s1}{buy}\PYG{l+s+s1}{\PYGZsq{}}\PYG{p}{)}\PYG{o}{.}\PYG{n}{style}\PYG{o}{.}\PYG{n}{set\PYGZus{}table\PYGZus{}attributes}\PYG{p}{(}\PYG{l+s+s1}{\PYGZsq{}}\PYG{l+s+s1}{style=}\PYG{l+s+s1}{\PYGZdq{}}\PYG{l+s+s1}{font\PYGZhy{}size: 12px}\PYG{l+s+s1}{\PYGZdq{}}\PYG{l+s+s1}{\PYGZsq{}}\PYG{p}{)}\PYG{o}{.}\PYG{n}{format}\PYG{p}{(}\PYG{n}{precision}\PYG{o}{=}\PYG{l+m+mi}{3}\PYG{p}{)}
\end{sphinxVerbatim}

\end{sphinxuseclass}\end{sphinxVerbatimInput}
\begin{sphinxVerbatimOutput}

\begin{sphinxuseclass}{cell_output}
\begin{sphinxVerbatim}[commandchars=\\\{\}]
2022\PYGZhy{}02\PYGZhy{}22 00:00:00 2022\PYGZhy{}06\PYGZhy{}02
125
\end{sphinxVerbatim}

\begin{sphinxVerbatim}[commandchars=\\\{\}]
\PYGZlt{}pandas.io.formats.style.Styler at 0x172a8141cd0\PYGZgt{}
\end{sphinxVerbatim}

\end{sphinxuseclass}\end{sphinxVerbatimOutput}

\end{sphinxuseclass}
\sphinxAtStartPar
 \sphinxstylestrong{2022년 6월 16일 \sphinxhyphen{} 종목 선정 및 수익률 테스트}2022년 6월 16일 추천종목은 20 종목이 넘습니다. 종목은 모델 스코어가 높은 5 종목만 선택하도록 하겠습니다. 한탑, 에이에프더블류, 베셀이 매수가 가능했습니다. 익절 가능할 것으로 예상됩니다.

\begin{sphinxuseclass}{cell}\begin{sphinxVerbatimInput}

\begin{sphinxuseclass}{cell_input}
\begin{sphinxVerbatim}[commandchars=\\\{\}]
\PYG{n}{select\PYGZus{}tops} \PYG{o}{=} \PYG{n}{select\PYGZus{}stocks}\PYG{p}{(}\PYG{l+s+s1}{\PYGZsq{}}\PYG{l+s+s1}{2022\PYGZhy{}06\PYGZhy{}16}\PYG{l+s+s1}{\PYGZsq{}}\PYG{p}{)}

\PYG{k}{if} \PYG{n}{select\PYGZus{}tops} \PYG{o+ow}{is} \PYG{o+ow}{not} \PYG{k+kc}{None}\PYG{p}{:}
    \PYG{n}{results} \PYG{o}{=} \PYG{n}{outcome\PYGZus{}tops}\PYG{p}{(}\PYG{n}{select\PYGZus{}tops}\PYG{p}{,} \PYG{l+s+s1}{\PYGZsq{}}\PYG{l+s+s1}{2022\PYGZhy{}06\PYGZhy{}16}\PYG{l+s+s1}{\PYGZsq{}}\PYG{p}{,} \PYG{l+s+s1}{\PYGZsq{}}\PYG{l+s+s1}{2022\PYGZhy{}06\PYGZhy{}23}\PYG{l+s+s1}{\PYGZsq{}}\PYG{p}{)} \PYG{c+c1}{\PYGZsh{} 5 영업일 (6월 6일 현충일)}
    
\PYG{n}{results}\PYG{o}{.}\PYG{n}{sort\PYGZus{}values}\PYG{p}{(}\PYG{n}{by}\PYG{o}{=}\PYG{p}{[}\PYG{l+s+s1}{\PYGZsq{}}\PYG{l+s+s1}{buy}\PYG{l+s+s1}{\PYGZsq{}}\PYG{p}{,}\PYG{l+s+s1}{\PYGZsq{}}\PYG{l+s+s1}{yhat}\PYG{l+s+s1}{\PYGZsq{}}\PYG{p}{]}\PYG{p}{,} \PYG{n}{ascending}\PYG{o}{=}\PYG{k+kc}{False}\PYG{p}{)}\PYG{o}{.}\PYG{n}{head}\PYG{p}{(}\PYG{l+m+mi}{5}\PYG{p}{)}\PYG{o}{.}\PYG{n}{style}\PYG{o}{.}\PYG{n}{set\PYGZus{}table\PYGZus{}attributes}\PYG{p}{(}\PYG{l+s+s1}{\PYGZsq{}}\PYG{l+s+s1}{style=}\PYG{l+s+s1}{\PYGZdq{}}\PYG{l+s+s1}{font\PYGZhy{}size: 12px}\PYG{l+s+s1}{\PYGZdq{}}\PYG{l+s+s1}{\PYGZsq{}}\PYG{p}{)}\PYG{o}{.}\PYG{n}{format}\PYG{p}{(}\PYG{n}{precision}\PYG{o}{=}\PYG{l+m+mi}{3}\PYG{p}{)}
\end{sphinxVerbatim}

\end{sphinxuseclass}\end{sphinxVerbatimInput}
\begin{sphinxVerbatimOutput}

\begin{sphinxuseclass}{cell_output}
\begin{sphinxVerbatim}[commandchars=\\\{\}]
2022\PYGZhy{}03\PYGZhy{}08 00:00:00 2022\PYGZhy{}06\PYGZhy{}16
391
\end{sphinxVerbatim}

\begin{sphinxVerbatim}[commandchars=\\\{\}]
\PYGZlt{}pandas.io.formats.style.Styler at 0x172a52cb910\PYGZgt{}
\end{sphinxVerbatim}

\end{sphinxuseclass}\end{sphinxVerbatimOutput}

\end{sphinxuseclass}
\begin{sphinxuseclass}{cell}\begin{sphinxVerbatimInput}

\begin{sphinxuseclass}{cell_input}
\begin{sphinxVerbatim}[commandchars=\\\{\}]
\PYG{k+kn}{import} \PYG{n+nn}{FinanceDataReader} \PYG{k}{as} \PYG{n+nn}{fdr}
\PYG{k+kn}{import} \PYG{n+nn}{matplotlib}\PYG{n+nn}{.}\PYG{n+nn}{pyplot} \PYG{k}{as} \PYG{n+nn}{plt}
\PYG{o}{\PYGZpc{}}\PYG{k}{matplotlib} inline
\PYG{k+kn}{import} \PYG{n+nn}{pandas} \PYG{k}{as} \PYG{n+nn}{pd}
\PYG{k+kn}{import} \PYG{n+nn}{numpy} \PYG{k}{as} \PYG{n+nn}{np}
\PYG{k+kn}{import} \PYG{n+nn}{datetime}
\PYG{k+kn}{import} \PYG{n+nn}{pickle}
\PYG{k+kn}{import} \PYG{n+nn}{glob}
\end{sphinxVerbatim}

\end{sphinxuseclass}\end{sphinxVerbatimInput}

\end{sphinxuseclass}

\chapter{\sphinxstylestrong{익절/손절 라인}}
\label{\detokenize{chapter6/6.2.1_Profit_Loss_Sell:id1}}\label{\detokenize{chapter6/6.2.1_Profit_Loss_Sell::doc}}
\sphinxAtStartPar
 이번에는 익절/손절라인을 결정할 수 있는 모델을 만들어 보겠습니다. 먼저 피쳐가 있는 데이터를 불러옵니다.

\begin{sphinxuseclass}{cell}\begin{sphinxVerbatimInput}

\begin{sphinxuseclass}{cell_input}
\begin{sphinxVerbatim}[commandchars=\\\{\}]
\PYG{n}{feature\PYGZus{}all} \PYG{o}{=} \PYG{n}{pd}\PYG{o}{.}\PYG{n}{read\PYGZus{}pickle}\PYG{p}{(}\PYG{l+s+s1}{\PYGZsq{}}\PYG{l+s+s1}{feature\PYGZus{}all.pkl}\PYG{l+s+s1}{\PYGZsq{}}\PYG{p}{)} 
\PYG{n}{feature\PYGZus{}all}\PYG{p}{[}\PYG{l+s+s1}{\PYGZsq{}}\PYG{l+s+s1}{target}\PYG{l+s+s1}{\PYGZsq{}}\PYG{p}{]} \PYG{o}{=} \PYG{n}{np}\PYG{o}{.}\PYG{n}{where}\PYG{p}{(}\PYG{n}{feature\PYGZus{}all}\PYG{p}{[}\PYG{l+s+s1}{\PYGZsq{}}\PYG{l+s+s1}{max\PYGZus{}close}\PYG{l+s+s1}{\PYGZsq{}}\PYG{p}{]}\PYG{o}{\PYGZgt{}}\PYG{o}{=} \PYG{l+m+mf}{1.05}\PYG{p}{,} \PYG{l+m+mi}{1}\PYG{p}{,} \PYG{l+m+mi}{0}\PYG{p}{)}
\PYG{n}{target} \PYG{o}{=} \PYG{n}{feature\PYGZus{}all}\PYG{p}{[}\PYG{l+s+s1}{\PYGZsq{}}\PYG{l+s+s1}{target}\PYG{l+s+s1}{\PYGZsq{}}\PYG{p}{]}\PYG{o}{.}\PYG{n}{mean}\PYG{p}{(}\PYG{p}{)}
\PYG{n+nb}{print}\PYG{p}{(}\PYG{l+s+sa}{f}\PYG{l+s+s1}{\PYGZsq{}}\PYG{l+s+s1}{\PYGZpc{} of target:}\PYG{l+s+si}{\PYGZob{}}\PYG{n}{target}\PYG{l+s+si}{:}\PYG{l+s+s1}{ 5.1\PYGZpc{}}\PYG{l+s+si}{\PYGZcb{}}\PYG{l+s+s1}{\PYGZsq{}}\PYG{p}{)}
\end{sphinxVerbatim}

\end{sphinxuseclass}\end{sphinxVerbatimInput}
\begin{sphinxVerbatimOutput}

\begin{sphinxuseclass}{cell_output}
\begin{sphinxVerbatim}[commandchars=\\\{\}]
\PYGZpc{} of target: 24.3\PYGZpc{}
\end{sphinxVerbatim}

\end{sphinxuseclass}\end{sphinxVerbatimOutput}

\end{sphinxuseclass}
\sphinxAtStartPar
 날짜와 종목을 인덱스로 설정합니다. 데이터에 예측모델을 적용하고 매수 대상 종목을 select\_top 이라는 DataFrame 에 저장합니다.

\begin{sphinxuseclass}{cell}\begin{sphinxVerbatimInput}

\begin{sphinxuseclass}{cell_input}
\begin{sphinxVerbatim}[commandchars=\\\{\}]
\PYG{n}{mdl\PYGZus{}all} \PYG{o}{=} \PYG{n}{feature\PYGZus{}all}\PYG{o}{.}\PYG{n}{set\PYGZus{}index}\PYG{p}{(}\PYG{p}{[}\PYG{n}{feature\PYGZus{}all}\PYG{o}{.}\PYG{n}{index}\PYG{p}{,}\PYG{l+s+s1}{\PYGZsq{}}\PYG{l+s+s1}{code}\PYG{l+s+s1}{\PYGZsq{}}\PYG{p}{]}\PYG{p}{)}

\PYG{k}{with} \PYG{n+nb}{open}\PYG{p}{(}\PYG{l+s+s2}{\PYGZdq{}}\PYG{l+s+s2}{gam.pkl}\PYG{l+s+s2}{\PYGZdq{}}\PYG{p}{,} \PYG{l+s+s2}{\PYGZdq{}}\PYG{l+s+s2}{rb}\PYG{l+s+s2}{\PYGZdq{}}\PYG{p}{)} \PYG{k}{as} \PYG{n}{file}\PYG{p}{:}
    \PYG{n}{gam} \PYG{o}{=} \PYG{n}{pickle}\PYG{o}{.}\PYG{n}{load}\PYG{p}{(}\PYG{n}{file}\PYG{p}{)} 

\PYG{n}{feature\PYGZus{}list} \PYG{o}{=} \PYG{p}{[}\PYG{l+s+s1}{\PYGZsq{}}\PYG{l+s+s1}{price\PYGZus{}z}\PYG{l+s+s1}{\PYGZsq{}}\PYG{p}{,}\PYG{l+s+s1}{\PYGZsq{}}\PYG{l+s+s1}{volume\PYGZus{}z}\PYG{l+s+s1}{\PYGZsq{}}\PYG{p}{,}\PYG{l+s+s1}{\PYGZsq{}}\PYG{l+s+s1}{num\PYGZus{}high/close}\PYG{l+s+s1}{\PYGZsq{}}\PYG{p}{,}\PYG{l+s+s1}{\PYGZsq{}}\PYG{l+s+s1}{num\PYGZus{}win\PYGZus{}market}\PYG{l+s+s1}{\PYGZsq{}}\PYG{p}{,}\PYG{l+s+s1}{\PYGZsq{}}\PYG{l+s+s1}{pct\PYGZus{}win\PYGZus{}market}\PYG{l+s+s1}{\PYGZsq{}}\PYG{p}{,}\PYG{l+s+s1}{\PYGZsq{}}\PYG{l+s+s1}{return over sector}\PYG{l+s+s1}{\PYGZsq{}}\PYG{p}{]}
\PYG{n}{X} \PYG{o}{=} \PYG{n}{mdl\PYGZus{}all}\PYG{p}{[}\PYG{n}{feature\PYGZus{}list}\PYG{p}{]}
\PYG{n}{y} \PYG{o}{=} \PYG{n}{mdl\PYGZus{}all}\PYG{p}{[}\PYG{l+s+s1}{\PYGZsq{}}\PYG{l+s+s1}{target}\PYG{l+s+s1}{\PYGZsq{}}\PYG{p}{]}

\PYG{n}{yhat} \PYG{o}{=} \PYG{n}{gam}\PYG{o}{.}\PYG{n}{predict\PYGZus{}proba}\PYG{p}{(}\PYG{n}{X}\PYG{o}{.}\PYG{n}{to\PYGZus{}numpy}\PYG{p}{(}\PYG{p}{)}\PYG{p}{)}
\PYG{n}{yhat} \PYG{o}{=} \PYG{n}{pd}\PYG{o}{.}\PYG{n}{Series}\PYG{p}{(}\PYG{n}{yhat}\PYG{p}{,} \PYG{n}{name}\PYG{o}{=}\PYG{l+s+s1}{\PYGZsq{}}\PYG{l+s+s1}{yhat}\PYG{l+s+s1}{\PYGZsq{}}\PYG{p}{,} \PYG{n}{index}\PYG{o}{=}\PYG{n}{y}\PYG{o}{.}\PYG{n}{index}\PYG{p}{)}

\PYG{n}{mdl\PYGZus{}all}\PYG{p}{[}\PYG{l+s+s1}{\PYGZsq{}}\PYG{l+s+s1}{yhat}\PYG{l+s+s1}{\PYGZsq{}}\PYG{p}{]} \PYG{o}{=} \PYG{n}{yhat}

\PYG{n}{tops} \PYG{o}{=} \PYG{n}{mdl\PYGZus{}all}\PYG{p}{[}\PYG{n}{mdl\PYGZus{}all}\PYG{p}{[}\PYG{l+s+s1}{\PYGZsq{}}\PYG{l+s+s1}{yhat}\PYG{l+s+s1}{\PYGZsq{}}\PYG{p}{]} \PYG{o}{\PYGZgt{}} \PYG{l+m+mf}{0.3}\PYG{p}{]}\PYG{o}{.}\PYG{n}{copy}\PYG{p}{(}\PYG{p}{)}

\PYG{n}{select\PYGZus{}tops} \PYG{o}{=} \PYG{n}{tops}\PYG{p}{[}\PYG{p}{(}\PYG{n}{tops}\PYG{p}{[}\PYG{l+s+s1}{\PYGZsq{}}\PYG{l+s+s1}{return}\PYG{l+s+s1}{\PYGZsq{}}\PYG{p}{]} \PYG{o}{\PYGZgt{}} \PYG{l+m+mf}{1.03}\PYG{p}{)} \PYG{o}{\PYGZam{}} \PYG{p}{(}\PYG{n}{tops}\PYG{p}{[}\PYG{l+s+s1}{\PYGZsq{}}\PYG{l+s+s1}{price\PYGZus{}z}\PYG{l+s+s1}{\PYGZsq{}}\PYG{p}{]} \PYG{o}{\PYGZlt{}} \PYG{l+m+mi}{0}\PYG{p}{)}\PYG{p}{]}
\PYG{n+nb}{print}\PYG{p}{(}\PYG{n+nb}{len}\PYG{p}{(}\PYG{n}{select\PYGZus{}tops}\PYG{p}{)}\PYG{p}{)}
\end{sphinxVerbatim}

\end{sphinxuseclass}\end{sphinxVerbatimInput}
\begin{sphinxVerbatimOutput}

\begin{sphinxuseclass}{cell_output}
\begin{sphinxVerbatim}[commandchars=\\\{\}]
1985
\end{sphinxVerbatim}

\end{sphinxuseclass}\end{sphinxVerbatimOutput}

\end{sphinxuseclass}
\sphinxAtStartPar
 최저 기대 수익율과 피쳐와의 상관계수를 조사합니다. 예상하지 못햇던 사실은 5 영업일 동안 최저 기대 수익률은 종목보다는 지수 수익률과 더 상관관계가 높습니다.

\begin{sphinxuseclass}{cell}\begin{sphinxVerbatimInput}

\begin{sphinxuseclass}{cell_input}
\begin{sphinxVerbatim}[commandchars=\\\{\}]
\PYG{n}{select\PYGZus{}tops}\PYG{p}{[}\PYG{p}{[}\PYG{l+s+s1}{\PYGZsq{}}\PYG{l+s+s1}{return}\PYG{l+s+s1}{\PYGZsq{}}\PYG{p}{,}\PYG{l+s+s1}{\PYGZsq{}}\PYG{l+s+s1}{kosdaq\PYGZus{}return}\PYG{l+s+s1}{\PYGZsq{}}\PYG{p}{,}\PYG{l+s+s1}{\PYGZsq{}}\PYG{l+s+s1}{min\PYGZus{}close}\PYG{l+s+s1}{\PYGZsq{}}\PYG{p}{]}\PYG{p}{]}\PYG{o}{.}\PYG{n}{corr}\PYG{p}{(}\PYG{p}{)}\PYG{p}{[}\PYG{l+s+s1}{\PYGZsq{}}\PYG{l+s+s1}{min\PYGZus{}close}\PYG{l+s+s1}{\PYGZsq{}}\PYG{p}{]}
\end{sphinxVerbatim}

\end{sphinxuseclass}\end{sphinxVerbatimInput}
\begin{sphinxVerbatimOutput}

\begin{sphinxuseclass}{cell_output}
\begin{sphinxVerbatim}[commandchars=\\\{\}]
return          \PYGZhy{}0.052747
kosdaq\PYGZus{}return    0.152224
min\PYGZus{}close        1.000000
Name: min\PYGZus{}close, dtype: float64
\end{sphinxVerbatim}

\end{sphinxuseclass}\end{sphinxVerbatimOutput}

\end{sphinxuseclass}
\sphinxAtStartPar
 ‘kosdaq\_return’ 에 따른 최저 기대 수익률의 평균을 구해봅니다. 그래프를 보니 ‘kosdaq\_return’(코스닥 지수 수익률)이 최저 기대수익률과 양의 상관관계가 높은 것으로 나타납니다.

\begin{sphinxuseclass}{cell}\begin{sphinxVerbatimInput}

\begin{sphinxuseclass}{cell_input}
\begin{sphinxVerbatim}[commandchars=\\\{\}]
\PYG{n}{ranks} \PYG{o}{=} \PYG{n}{pd}\PYG{o}{.}\PYG{n}{qcut}\PYG{p}{(}\PYG{n}{select\PYGZus{}tops}\PYG{p}{[}\PYG{l+s+s1}{\PYGZsq{}}\PYG{l+s+s1}{kosdaq\PYGZus{}return}\PYG{l+s+s1}{\PYGZsq{}}\PYG{p}{]}\PYG{p}{,} \PYG{n}{q}\PYG{o}{=}\PYG{l+m+mi}{5}\PYG{p}{)}
\PYG{n+nb}{print}\PYG{p}{(}\PYG{n}{select\PYGZus{}tops}\PYG{o}{.}\PYG{n}{groupby}\PYG{p}{(}\PYG{n}{ranks}\PYG{p}{)}\PYG{p}{[}\PYG{l+s+s1}{\PYGZsq{}}\PYG{l+s+s1}{min\PYGZus{}close}\PYG{l+s+s1}{\PYGZsq{}}\PYG{p}{]}\PYG{o}{.}\PYG{n}{mean}\PYG{p}{(}\PYG{p}{)}\PYG{p}{)}
\PYG{n}{select\PYGZus{}tops}\PYG{o}{.}\PYG{n}{groupby}\PYG{p}{(}\PYG{n}{ranks}\PYG{p}{)}\PYG{p}{[}\PYG{l+s+s1}{\PYGZsq{}}\PYG{l+s+s1}{min\PYGZus{}close}\PYG{l+s+s1}{\PYGZsq{}}\PYG{p}{]}\PYG{o}{.}\PYG{n}{mean}\PYG{p}{(}\PYG{p}{)}\PYG{o}{.}\PYG{n}{plot}\PYG{p}{(}\PYG{p}{)}
\end{sphinxVerbatim}

\end{sphinxuseclass}\end{sphinxVerbatimInput}
\begin{sphinxVerbatimOutput}

\begin{sphinxuseclass}{cell_output}
\begin{sphinxVerbatim}[commandchars=\\\{\}]
kosdaq\PYGZus{}return
(0.962, 0.999]    0.943243
(0.999, 1.005]    0.957895
(1.005, 1.012]    0.962220
(1.012, 1.026]    0.978050
(1.026, 1.046]    0.974586
Name: min\PYGZus{}close, dtype: float64
\end{sphinxVerbatim}

\begin{sphinxVerbatim}[commandchars=\\\{\}]
\PYGZlt{}AxesSubplot:xlabel=\PYGZsq{}kosdaq\PYGZus{}return\PYGZsq{}\PYGZgt{}
\end{sphinxVerbatim}

\noindent\sphinxincludegraphics{{6.2.1_Profit_Loss_Sell_9_2}.png}

\end{sphinxuseclass}\end{sphinxVerbatimOutput}

\end{sphinxuseclass}
\sphinxAtStartPar
 ‘kosdaq\_return’ 값에 따라 아래와 같이 익절/손절 라인을 변동할 수 있도록 합니다. 기본적인 아이디어는 장이 좋을 때는 익절/손절 범위를 늘리고, 안 좋은 날은 익절/손절 범위를 좁히는 것입니다. 아래 함수는 익절 수익률과 손절 수익률을 딕셔너리로 반환합니다.

\begin{sphinxuseclass}{cell}\begin{sphinxVerbatimInput}

\begin{sphinxuseclass}{cell_input}
\begin{sphinxVerbatim}[commandchars=\\\{\}]
\PYG{k}{def} \PYG{n+nf}{profit\PYGZus{}loss\PYGZus{}cut}\PYG{p}{(}\PYG{n}{x}\PYG{p}{)}\PYG{p}{:}
    
    \PYG{k}{if} \PYG{n}{x} \PYG{o}{\PYGZlt{}}\PYG{o}{=} \PYG{l+m+mf}{1.00}\PYG{p}{:} \PYG{c+c1}{\PYGZsh{} 익절 손절 범위 축소}
        \PYG{k}{return} \PYG{l+m+mf}{1.04}\PYG{p}{,} \PYG{l+m+mf}{0.98} 
    
    \PYG{k}{elif} \PYG{n}{x} \PYG{o}{\PYGZlt{}}\PYG{o}{=} \PYG{l+m+mf}{1.02}\PYG{p}{:} 
        \PYG{k}{return} \PYG{l+m+mf}{1.05}\PYG{p}{,} \PYG{l+m+mf}{0.97}
    
    \PYG{k}{else}\PYG{p}{:} \PYG{c+c1}{\PYGZsh{} 익절/손절범위 확대}
        \PYG{k}{return} \PYG{l+m+mf}{1.06}\PYG{p}{,} \PYG{l+m+mf}{0.96}
\end{sphinxVerbatim}

\end{sphinxuseclass}\end{sphinxVerbatimInput}

\end{sphinxuseclass}
\sphinxAtStartPar
 종목 선정 및 익절/손절라인 반환하는 전체 프로세스를 함수를 만듭니다.

\begin{sphinxuseclass}{cell}\begin{sphinxVerbatimInput}

\begin{sphinxuseclass}{cell_input}
\begin{sphinxVerbatim}[commandchars=\\\{\}]
\PYG{k}{def} \PYG{n+nf}{select\PYGZus{}stocks\PYGZus{}sell}\PYG{p}{(}\PYG{n}{today\PYGZus{}dt}\PYG{p}{)}\PYG{p}{:}
    
    \PYG{n}{today} \PYG{o}{=} \PYG{n}{datetime}\PYG{o}{.}\PYG{n}{datetime}\PYG{o}{.}\PYG{n}{strptime}\PYG{p}{(}\PYG{n}{today\PYGZus{}dt}\PYG{p}{,} \PYG{l+s+s1}{\PYGZsq{}}\PYG{l+s+s1}{\PYGZpc{}}\PYG{l+s+s1}{Y\PYGZhy{}}\PYG{l+s+s1}{\PYGZpc{}}\PYG{l+s+s1}{m\PYGZhy{}}\PYG{l+s+si}{\PYGZpc{}d}\PYG{l+s+s1}{\PYGZsq{}}\PYG{p}{)}
    \PYG{n}{start\PYGZus{}dt} \PYG{o}{=} \PYG{n}{today} \PYG{o}{\PYGZhy{}} \PYG{n}{datetime}\PYG{o}{.}\PYG{n}{timedelta}\PYG{p}{(}\PYG{n}{days}\PYG{o}{=}\PYG{l+m+mi}{100}\PYG{p}{)} \PYG{c+c1}{\PYGZsh{} 100 일전 데이터 부터 시작 \PYGZhy{} 피쳐 엔지니어링은 최소 60 개의 일봉이 필요함}
    \PYG{n+nb}{print}\PYG{p}{(}\PYG{n}{start\PYGZus{}dt}\PYG{p}{,} \PYG{n}{today\PYGZus{}dt}\PYG{p}{)}

    \PYG{n}{kosdaq\PYGZus{}list} \PYG{o}{=} \PYG{n}{pd}\PYG{o}{.}\PYG{n}{read\PYGZus{}pickle}\PYG{p}{(}\PYG{l+s+s1}{\PYGZsq{}}\PYG{l+s+s1}{kosdaq\PYGZus{}list.pkl}\PYG{l+s+s1}{\PYGZsq{}}\PYG{p}{)}

    \PYG{n}{price\PYGZus{}data} \PYG{o}{=} \PYG{n}{pd}\PYG{o}{.}\PYG{n}{DataFrame}\PYG{p}{(}\PYG{p}{)}

    \PYG{k}{for} \PYG{n}{code}\PYG{p}{,} \PYG{n}{name} \PYG{o+ow}{in} \PYG{n+nb}{zip}\PYG{p}{(}\PYG{n}{kosdaq\PYGZus{}list}\PYG{p}{[}\PYG{l+s+s1}{\PYGZsq{}}\PYG{l+s+s1}{code}\PYG{l+s+s1}{\PYGZsq{}}\PYG{p}{]}\PYG{p}{,} \PYG{n}{kosdaq\PYGZus{}list}\PYG{p}{[}\PYG{l+s+s1}{\PYGZsq{}}\PYG{l+s+s1}{name}\PYG{l+s+s1}{\PYGZsq{}}\PYG{p}{]}\PYG{p}{)}\PYG{p}{:}  \PYG{c+c1}{\PYGZsh{} 코스닥 모든 종목에서 대하여 반복}
        \PYG{n}{daily\PYGZus{}price} \PYG{o}{=} \PYG{n}{fdr}\PYG{o}{.}\PYG{n}{DataReader}\PYG{p}{(}\PYG{n}{code}\PYG{p}{,} \PYG{n}{start} \PYG{o}{=} \PYG{n}{start\PYGZus{}dt}\PYG{p}{,} \PYG{n}{end} \PYG{o}{=} \PYG{n}{today\PYGZus{}dt}\PYG{p}{)} \PYG{c+c1}{\PYGZsh{} 종목, 일봉, 데이터 갯수}

        \PYG{n}{daily\PYGZus{}price}\PYG{p}{[}\PYG{l+s+s1}{\PYGZsq{}}\PYG{l+s+s1}{code}\PYG{l+s+s1}{\PYGZsq{}}\PYG{p}{]} \PYG{o}{=} \PYG{n}{code}
        \PYG{n}{daily\PYGZus{}price}\PYG{p}{[}\PYG{l+s+s1}{\PYGZsq{}}\PYG{l+s+s1}{name}\PYG{l+s+s1}{\PYGZsq{}}\PYG{p}{]} \PYG{o}{=} \PYG{n}{name}
        \PYG{n}{price\PYGZus{}data} \PYG{o}{=} \PYG{n}{pd}\PYG{o}{.}\PYG{n}{concat}\PYG{p}{(}\PYG{p}{[}\PYG{n}{price\PYGZus{}data}\PYG{p}{,} \PYG{n}{daily\PYGZus{}price}\PYG{p}{]}\PYG{p}{,} \PYG{n}{axis}\PYG{o}{=}\PYG{l+m+mi}{0}\PYG{p}{)}   

    \PYG{n}{price\PYGZus{}data}\PYG{o}{.}\PYG{n}{index}\PYG{o}{.}\PYG{n}{name} \PYG{o}{=} \PYG{l+s+s1}{\PYGZsq{}}\PYG{l+s+s1}{date}\PYG{l+s+s1}{\PYGZsq{}}
    \PYG{n}{price\PYGZus{}data}\PYG{o}{.}\PYG{n}{columns}\PYG{o}{=} \PYG{n}{price\PYGZus{}data}\PYG{o}{.}\PYG{n}{columns}\PYG{o}{.}\PYG{n}{str}\PYG{o}{.}\PYG{n}{lower}\PYG{p}{(}\PYG{p}{)} \PYG{c+c1}{\PYGZsh{} 컬럼 이름 소문자로 변경}

    \PYG{n}{kosdaq\PYGZus{}index} \PYG{o}{=} \PYG{n}{fdr}\PYG{o}{.}\PYG{n}{DataReader}\PYG{p}{(}\PYG{l+s+s1}{\PYGZsq{}}\PYG{l+s+s1}{KQ11}\PYG{l+s+s1}{\PYGZsq{}}\PYG{p}{,} \PYG{n}{start} \PYG{o}{=} \PYG{n}{start\PYGZus{}dt}\PYG{p}{,} \PYG{n}{end} \PYG{o}{=} \PYG{n}{today\PYGZus{}dt}\PYG{p}{)} \PYG{c+c1}{\PYGZsh{} 데이터 호출}
    \PYG{n}{kosdaq\PYGZus{}index}\PYG{o}{.}\PYG{n}{columns} \PYG{o}{=} \PYG{p}{[}\PYG{l+s+s1}{\PYGZsq{}}\PYG{l+s+s1}{close}\PYG{l+s+s1}{\PYGZsq{}}\PYG{p}{,}\PYG{l+s+s1}{\PYGZsq{}}\PYG{l+s+s1}{open}\PYG{l+s+s1}{\PYGZsq{}}\PYG{p}{,}\PYG{l+s+s1}{\PYGZsq{}}\PYG{l+s+s1}{high}\PYG{l+s+s1}{\PYGZsq{}}\PYG{p}{,}\PYG{l+s+s1}{\PYGZsq{}}\PYG{l+s+s1}{low}\PYG{l+s+s1}{\PYGZsq{}}\PYG{p}{,}\PYG{l+s+s1}{\PYGZsq{}}\PYG{l+s+s1}{volume}\PYG{l+s+s1}{\PYGZsq{}}\PYG{p}{,}\PYG{l+s+s1}{\PYGZsq{}}\PYG{l+s+s1}{change}\PYG{l+s+s1}{\PYGZsq{}}\PYG{p}{]} \PYG{c+c1}{\PYGZsh{} 컬럼명 변경}
    \PYG{n}{kosdaq\PYGZus{}index}\PYG{o}{.}\PYG{n}{index}\PYG{o}{.}\PYG{n}{name}\PYG{o}{=}\PYG{l+s+s1}{\PYGZsq{}}\PYG{l+s+s1}{date}\PYG{l+s+s1}{\PYGZsq{}} \PYG{c+c1}{\PYGZsh{} 인덱스 이름 생성}
    \PYG{n}{kosdaq\PYGZus{}index}\PYG{o}{.}\PYG{n}{sort\PYGZus{}index}\PYG{p}{(}\PYG{n}{inplace}\PYG{o}{=}\PYG{k+kc}{True}\PYG{p}{)} \PYG{c+c1}{\PYGZsh{} 인덱스(날짜) 로 정렬 }
    \PYG{n}{kosdaq\PYGZus{}index}\PYG{p}{[}\PYG{l+s+s1}{\PYGZsq{}}\PYG{l+s+s1}{kosdaq\PYGZus{}return}\PYG{l+s+s1}{\PYGZsq{}}\PYG{p}{]} \PYG{o}{=} \PYG{n}{kosdaq\PYGZus{}index}\PYG{p}{[}\PYG{l+s+s1}{\PYGZsq{}}\PYG{l+s+s1}{close}\PYG{l+s+s1}{\PYGZsq{}}\PYG{p}{]}\PYG{o}{/}\PYG{n}{kosdaq\PYGZus{}index}\PYG{p}{[}\PYG{l+s+s1}{\PYGZsq{}}\PYG{l+s+s1}{close}\PYG{l+s+s1}{\PYGZsq{}}\PYG{p}{]}\PYG{o}{.}\PYG{n}{shift}\PYG{p}{(}\PYG{l+m+mi}{1}\PYG{p}{)} \PYG{c+c1}{\PYGZsh{} 수익율 : 전 날 종가대비 당일 종가}

    \PYG{n}{merged} \PYG{o}{=} \PYG{n}{price\PYGZus{}data}\PYG{o}{.}\PYG{n}{merge}\PYG{p}{(}\PYG{n}{kosdaq\PYGZus{}index}\PYG{p}{[}\PYG{l+s+s1}{\PYGZsq{}}\PYG{l+s+s1}{kosdaq\PYGZus{}return}\PYG{l+s+s1}{\PYGZsq{}}\PYG{p}{]}\PYG{p}{,} \PYG{n}{left\PYGZus{}index}\PYG{o}{=}\PYG{k+kc}{True}\PYG{p}{,} \PYG{n}{right\PYGZus{}index}\PYG{o}{=}\PYG{k+kc}{True}\PYG{p}{,} \PYG{n}{how}\PYG{o}{=}\PYG{l+s+s1}{\PYGZsq{}}\PYG{l+s+s1}{left}\PYG{l+s+s1}{\PYGZsq{}}\PYG{p}{)}

    \PYG{n}{return\PYGZus{}all} \PYG{o}{=} \PYG{n}{pd}\PYG{o}{.}\PYG{n}{DataFrame}\PYG{p}{(}\PYG{p}{)}

    \PYG{k}{for} \PYG{n}{code} \PYG{o+ow}{in} \PYG{n}{kosdaq\PYGZus{}list}\PYG{p}{[}\PYG{l+s+s1}{\PYGZsq{}}\PYG{l+s+s1}{code}\PYG{l+s+s1}{\PYGZsq{}}\PYG{p}{]}\PYG{p}{:}  

        \PYG{n}{stock\PYGZus{}return} \PYG{o}{=} \PYG{n}{merged}\PYG{p}{[}\PYG{n}{merged}\PYG{p}{[}\PYG{l+s+s1}{\PYGZsq{}}\PYG{l+s+s1}{code}\PYG{l+s+s1}{\PYGZsq{}}\PYG{p}{]}\PYG{o}{==}\PYG{n}{code}\PYG{p}{]}\PYG{o}{.}\PYG{n}{sort\PYGZus{}index}\PYG{p}{(}\PYG{p}{)}
        \PYG{n}{stock\PYGZus{}return}\PYG{p}{[}\PYG{l+s+s1}{\PYGZsq{}}\PYG{l+s+s1}{return}\PYG{l+s+s1}{\PYGZsq{}}\PYG{p}{]} \PYG{o}{=} \PYG{n}{stock\PYGZus{}return}\PYG{p}{[}\PYG{l+s+s1}{\PYGZsq{}}\PYG{l+s+s1}{close}\PYG{l+s+s1}{\PYGZsq{}}\PYG{p}{]}\PYG{o}{/}\PYG{n}{stock\PYGZus{}return}\PYG{p}{[}\PYG{l+s+s1}{\PYGZsq{}}\PYG{l+s+s1}{close}\PYG{l+s+s1}{\PYGZsq{}}\PYG{p}{]}\PYG{o}{.}\PYG{n}{shift}\PYG{p}{(}\PYG{l+m+mi}{1}\PYG{p}{)} \PYG{c+c1}{\PYGZsh{} 종목별 전일 종가 대비 당일 종가 수익율}
        \PYG{n}{c1} \PYG{o}{=} \PYG{p}{(}\PYG{n}{stock\PYGZus{}return}\PYG{p}{[}\PYG{l+s+s1}{\PYGZsq{}}\PYG{l+s+s1}{kosdaq\PYGZus{}return}\PYG{l+s+s1}{\PYGZsq{}}\PYG{p}{]} \PYG{o}{\PYGZlt{}} \PYG{l+m+mi}{1}\PYG{p}{)} \PYG{c+c1}{\PYGZsh{} 수익율 1 보다 작음. 당일 종가가 전일 종가보다 낮음 (코스닥 지표)}
        \PYG{n}{c2} \PYG{o}{=} \PYG{p}{(}\PYG{n}{stock\PYGZus{}return}\PYG{p}{[}\PYG{l+s+s1}{\PYGZsq{}}\PYG{l+s+s1}{return}\PYG{l+s+s1}{\PYGZsq{}}\PYG{p}{]} \PYG{o}{\PYGZgt{}} \PYG{l+m+mi}{1}\PYG{p}{)} \PYG{c+c1}{\PYGZsh{} 수익율 1 보다 큼. 당일 종가가 전일 종가보다 큼 (개별 종목)}
        \PYG{n}{stock\PYGZus{}return}\PYG{p}{[}\PYG{l+s+s1}{\PYGZsq{}}\PYG{l+s+s1}{win\PYGZus{}market}\PYG{l+s+s1}{\PYGZsq{}}\PYG{p}{]} \PYG{o}{=} \PYG{n}{np}\PYG{o}{.}\PYG{n}{where}\PYG{p}{(}\PYG{p}{(}\PYG{n}{c1}\PYG{o}{\PYGZam{}}\PYG{n}{c2}\PYG{p}{)}\PYG{p}{,} \PYG{l+m+mi}{1}\PYG{p}{,} \PYG{l+m+mi}{0}\PYG{p}{)} \PYG{c+c1}{\PYGZsh{} C1 과 C2 조건을 동시에 만족하면 1, 아니면 0}
        \PYG{n}{return\PYGZus{}all} \PYG{o}{=} \PYG{n}{pd}\PYG{o}{.}\PYG{n}{concat}\PYG{p}{(}\PYG{p}{[}\PYG{n}{return\PYGZus{}all}\PYG{p}{,} \PYG{n}{stock\PYGZus{}return}\PYG{p}{]}\PYG{p}{,} \PYG{n}{axis}\PYG{o}{=}\PYG{l+m+mi}{0}\PYG{p}{)} 

    \PYG{n}{return\PYGZus{}all}\PYG{o}{.}\PYG{n}{dropna}\PYG{p}{(}\PYG{n}{inplace}\PYG{o}{=}\PYG{k+kc}{True}\PYG{p}{)}    

    \PYG{n}{model\PYGZus{}inputs} \PYG{o}{=} \PYG{n}{pd}\PYG{o}{.}\PYG{n}{DataFrame}\PYG{p}{(}\PYG{p}{)}

    \PYG{k}{for} \PYG{n}{code}\PYG{p}{,} \PYG{n}{name}\PYG{p}{,} \PYG{n}{sector} \PYG{o+ow}{in} \PYG{n+nb}{zip}\PYG{p}{(}\PYG{n}{kosdaq\PYGZus{}list}\PYG{p}{[}\PYG{l+s+s1}{\PYGZsq{}}\PYG{l+s+s1}{code}\PYG{l+s+s1}{\PYGZsq{}}\PYG{p}{]}\PYG{p}{,} \PYG{n}{kosdaq\PYGZus{}list}\PYG{p}{[}\PYG{l+s+s1}{\PYGZsq{}}\PYG{l+s+s1}{name}\PYG{l+s+s1}{\PYGZsq{}}\PYG{p}{]}\PYG{p}{,} \PYG{n}{kosdaq\PYGZus{}list}\PYG{p}{[}\PYG{l+s+s1}{\PYGZsq{}}\PYG{l+s+s1}{sector}\PYG{l+s+s1}{\PYGZsq{}}\PYG{p}{]}\PYG{p}{)}\PYG{p}{:}

        \PYG{n}{data} \PYG{o}{=} \PYG{n}{return\PYGZus{}all}\PYG{p}{[}\PYG{n}{return\PYGZus{}all}\PYG{p}{[}\PYG{l+s+s1}{\PYGZsq{}}\PYG{l+s+s1}{code}\PYG{l+s+s1}{\PYGZsq{}}\PYG{p}{]}\PYG{o}{==}\PYG{n}{code}\PYG{p}{]}\PYG{o}{.}\PYG{n}{sort\PYGZus{}index}\PYG{p}{(}\PYG{p}{)}\PYG{o}{.}\PYG{n}{copy}\PYG{p}{(}\PYG{p}{)}    

        \PYG{c+c1}{\PYGZsh{} 가격변동성이 크고, 거래량이 몰린 종목이 주가가 상승한다}
        \PYG{n}{data}\PYG{p}{[}\PYG{l+s+s1}{\PYGZsq{}}\PYG{l+s+s1}{price\PYGZus{}mean}\PYG{l+s+s1}{\PYGZsq{}}\PYG{p}{]} \PYG{o}{=} \PYG{n}{data}\PYG{p}{[}\PYG{l+s+s1}{\PYGZsq{}}\PYG{l+s+s1}{close}\PYG{l+s+s1}{\PYGZsq{}}\PYG{p}{]}\PYG{o}{.}\PYG{n}{rolling}\PYG{p}{(}\PYG{l+m+mi}{20}\PYG{p}{)}\PYG{o}{.}\PYG{n}{mean}\PYG{p}{(}\PYG{p}{)}
        \PYG{n}{data}\PYG{p}{[}\PYG{l+s+s1}{\PYGZsq{}}\PYG{l+s+s1}{price\PYGZus{}std}\PYG{l+s+s1}{\PYGZsq{}}\PYG{p}{]} \PYG{o}{=} \PYG{n}{data}\PYG{p}{[}\PYG{l+s+s1}{\PYGZsq{}}\PYG{l+s+s1}{close}\PYG{l+s+s1}{\PYGZsq{}}\PYG{p}{]}\PYG{o}{.}\PYG{n}{rolling}\PYG{p}{(}\PYG{l+m+mi}{20}\PYG{p}{)}\PYG{o}{.}\PYG{n}{std}\PYG{p}{(}\PYG{n}{ddof}\PYG{o}{=}\PYG{l+m+mi}{0}\PYG{p}{)}
        \PYG{n}{data}\PYG{p}{[}\PYG{l+s+s1}{\PYGZsq{}}\PYG{l+s+s1}{price\PYGZus{}z}\PYG{l+s+s1}{\PYGZsq{}}\PYG{p}{]} \PYG{o}{=} \PYG{p}{(}\PYG{n}{data}\PYG{p}{[}\PYG{l+s+s1}{\PYGZsq{}}\PYG{l+s+s1}{close}\PYG{l+s+s1}{\PYGZsq{}}\PYG{p}{]} \PYG{o}{\PYGZhy{}} \PYG{n}{data}\PYG{p}{[}\PYG{l+s+s1}{\PYGZsq{}}\PYG{l+s+s1}{price\PYGZus{}mean}\PYG{l+s+s1}{\PYGZsq{}}\PYG{p}{]}\PYG{p}{)}\PYG{o}{/}\PYG{n}{data}\PYG{p}{[}\PYG{l+s+s1}{\PYGZsq{}}\PYG{l+s+s1}{price\PYGZus{}std}\PYG{l+s+s1}{\PYGZsq{}}\PYG{p}{]}    
        \PYG{n}{data}\PYG{p}{[}\PYG{l+s+s1}{\PYGZsq{}}\PYG{l+s+s1}{volume\PYGZus{}mean}\PYG{l+s+s1}{\PYGZsq{}}\PYG{p}{]} \PYG{o}{=} \PYG{n}{data}\PYG{p}{[}\PYG{l+s+s1}{\PYGZsq{}}\PYG{l+s+s1}{volume}\PYG{l+s+s1}{\PYGZsq{}}\PYG{p}{]}\PYG{o}{.}\PYG{n}{rolling}\PYG{p}{(}\PYG{l+m+mi}{20}\PYG{p}{)}\PYG{o}{.}\PYG{n}{mean}\PYG{p}{(}\PYG{p}{)}
        \PYG{n}{data}\PYG{p}{[}\PYG{l+s+s1}{\PYGZsq{}}\PYG{l+s+s1}{volume\PYGZus{}std}\PYG{l+s+s1}{\PYGZsq{}}\PYG{p}{]} \PYG{o}{=} \PYG{n}{data}\PYG{p}{[}\PYG{l+s+s1}{\PYGZsq{}}\PYG{l+s+s1}{volume}\PYG{l+s+s1}{\PYGZsq{}}\PYG{p}{]}\PYG{o}{.}\PYG{n}{rolling}\PYG{p}{(}\PYG{l+m+mi}{20}\PYG{p}{)}\PYG{o}{.}\PYG{n}{std}\PYG{p}{(}\PYG{n}{ddof}\PYG{o}{=}\PYG{l+m+mi}{0}\PYG{p}{)}
        \PYG{n}{data}\PYG{p}{[}\PYG{l+s+s1}{\PYGZsq{}}\PYG{l+s+s1}{volume\PYGZus{}z}\PYG{l+s+s1}{\PYGZsq{}}\PYG{p}{]} \PYG{o}{=} \PYG{p}{(}\PYG{n}{data}\PYG{p}{[}\PYG{l+s+s1}{\PYGZsq{}}\PYG{l+s+s1}{volume}\PYG{l+s+s1}{\PYGZsq{}}\PYG{p}{]} \PYG{o}{\PYGZhy{}} \PYG{n}{data}\PYG{p}{[}\PYG{l+s+s1}{\PYGZsq{}}\PYG{l+s+s1}{volume\PYGZus{}mean}\PYG{l+s+s1}{\PYGZsq{}}\PYG{p}{]}\PYG{p}{)}\PYG{o}{/}\PYG{n}{data}\PYG{p}{[}\PYG{l+s+s1}{\PYGZsq{}}\PYG{l+s+s1}{volume\PYGZus{}std}\PYG{l+s+s1}{\PYGZsq{}}\PYG{p}{]}

        \PYG{c+c1}{\PYGZsh{} 위꼬리가 긴 양봉이 자주발생한다.}
        \PYG{n}{data}\PYG{p}{[}\PYG{l+s+s1}{\PYGZsq{}}\PYG{l+s+s1}{positive\PYGZus{}candle}\PYG{l+s+s1}{\PYGZsq{}}\PYG{p}{]} \PYG{o}{=} \PYG{p}{(}\PYG{n}{data}\PYG{p}{[}\PYG{l+s+s1}{\PYGZsq{}}\PYG{l+s+s1}{close}\PYG{l+s+s1}{\PYGZsq{}}\PYG{p}{]} \PYG{o}{\PYGZgt{}} \PYG{n}{data}\PYG{p}{[}\PYG{l+s+s1}{\PYGZsq{}}\PYG{l+s+s1}{open}\PYG{l+s+s1}{\PYGZsq{}}\PYG{p}{]}\PYG{p}{)}\PYG{o}{.}\PYG{n}{astype}\PYG{p}{(}\PYG{n+nb}{int}\PYG{p}{)} \PYG{c+c1}{\PYGZsh{} 양봉}
        \PYG{n}{data}\PYG{p}{[}\PYG{l+s+s1}{\PYGZsq{}}\PYG{l+s+s1}{high/close}\PYG{l+s+s1}{\PYGZsq{}}\PYG{p}{]} \PYG{o}{=} \PYG{p}{(}\PYG{n}{data}\PYG{p}{[}\PYG{l+s+s1}{\PYGZsq{}}\PYG{l+s+s1}{positive\PYGZus{}candle}\PYG{l+s+s1}{\PYGZsq{}}\PYG{p}{]}\PYG{o}{==}\PYG{l+m+mi}{1}\PYG{p}{)}\PYG{o}{*}\PYG{p}{(}\PYG{n}{data}\PYG{p}{[}\PYG{l+s+s1}{\PYGZsq{}}\PYG{l+s+s1}{high}\PYG{l+s+s1}{\PYGZsq{}}\PYG{p}{]}\PYG{o}{/}\PYG{n}{data}\PYG{p}{[}\PYG{l+s+s1}{\PYGZsq{}}\PYG{l+s+s1}{close}\PYG{l+s+s1}{\PYGZsq{}}\PYG{p}{]} \PYG{o}{\PYGZgt{}} \PYG{l+m+mf}{1.1}\PYG{p}{)}\PYG{o}{.}\PYG{n}{astype}\PYG{p}{(}\PYG{n+nb}{int}\PYG{p}{)} \PYG{c+c1}{\PYGZsh{} 양봉이면서 고가가 종가보다 높게 위치}
        \PYG{n}{data}\PYG{p}{[}\PYG{l+s+s1}{\PYGZsq{}}\PYG{l+s+s1}{num\PYGZus{}high/close}\PYG{l+s+s1}{\PYGZsq{}}\PYG{p}{]} \PYG{o}{=}  \PYG{n}{data}\PYG{p}{[}\PYG{l+s+s1}{\PYGZsq{}}\PYG{l+s+s1}{high/close}\PYG{l+s+s1}{\PYGZsq{}}\PYG{p}{]}\PYG{o}{.}\PYG{n}{rolling}\PYG{p}{(}\PYG{l+m+mi}{20}\PYG{p}{)}\PYG{o}{.}\PYG{n}{sum}\PYG{p}{(}\PYG{p}{)}
        \PYG{n}{data}\PYG{p}{[}\PYG{l+s+s1}{\PYGZsq{}}\PYG{l+s+s1}{long\PYGZus{}candle}\PYG{l+s+s1}{\PYGZsq{}}\PYG{p}{]} \PYG{o}{=} \PYG{p}{(}\PYG{n}{data}\PYG{p}{[}\PYG{l+s+s1}{\PYGZsq{}}\PYG{l+s+s1}{positive\PYGZus{}candle}\PYG{l+s+s1}{\PYGZsq{}}\PYG{p}{]}\PYG{o}{==}\PYG{l+m+mi}{1}\PYG{p}{)}\PYG{o}{*}\PYG{p}{(}\PYG{n}{data}\PYG{p}{[}\PYG{l+s+s1}{\PYGZsq{}}\PYG{l+s+s1}{high}\PYG{l+s+s1}{\PYGZsq{}}\PYG{p}{]}\PYG{o}{==}\PYG{n}{data}\PYG{p}{[}\PYG{l+s+s1}{\PYGZsq{}}\PYG{l+s+s1}{close}\PYG{l+s+s1}{\PYGZsq{}}\PYG{p}{]}\PYG{p}{)}\PYG{o}{*}\PYGZbs{}
        \PYG{p}{(}\PYG{n}{data}\PYG{p}{[}\PYG{l+s+s1}{\PYGZsq{}}\PYG{l+s+s1}{low}\PYG{l+s+s1}{\PYGZsq{}}\PYG{p}{]}\PYG{o}{==}\PYG{n}{data}\PYG{p}{[}\PYG{l+s+s1}{\PYGZsq{}}\PYG{l+s+s1}{open}\PYG{l+s+s1}{\PYGZsq{}}\PYG{p}{]}\PYG{p}{)}\PYG{o}{*}\PYG{p}{(}\PYG{n}{data}\PYG{p}{[}\PYG{l+s+s1}{\PYGZsq{}}\PYG{l+s+s1}{close}\PYG{l+s+s1}{\PYGZsq{}}\PYG{p}{]}\PYG{o}{/}\PYG{n}{data}\PYG{p}{[}\PYG{l+s+s1}{\PYGZsq{}}\PYG{l+s+s1}{open}\PYG{l+s+s1}{\PYGZsq{}}\PYG{p}{]} \PYG{o}{\PYGZgt{}} \PYG{l+m+mf}{1.2}\PYG{p}{)}\PYG{o}{.}\PYG{n}{astype}\PYG{p}{(}\PYG{n+nb}{int}\PYG{p}{)} \PYG{c+c1}{\PYGZsh{} 장대 양봉을 데이터로 표현}
        \PYG{n}{data}\PYG{p}{[}\PYG{l+s+s1}{\PYGZsq{}}\PYG{l+s+s1}{num\PYGZus{}long}\PYG{l+s+s1}{\PYGZsq{}}\PYG{p}{]} \PYG{o}{=}  \PYG{n}{data}\PYG{p}{[}\PYG{l+s+s1}{\PYGZsq{}}\PYG{l+s+s1}{long\PYGZus{}candle}\PYG{l+s+s1}{\PYGZsq{}}\PYG{p}{]}\PYG{o}{.}\PYG{n}{rolling}\PYG{p}{(}\PYG{l+m+mi}{60}\PYG{p}{)}\PYG{o}{.}\PYG{n}{sum}\PYG{p}{(}\PYG{p}{)} \PYG{c+c1}{\PYGZsh{} 지난 20 일 동안 장대양봉의 갯 수}


         \PYG{c+c1}{\PYGZsh{} 거래량이 종좀 터지며 매집의 흔적을 보인다   }
        \PYG{n}{data}\PYG{p}{[}\PYG{l+s+s1}{\PYGZsq{}}\PYG{l+s+s1}{volume\PYGZus{}mean}\PYG{l+s+s1}{\PYGZsq{}}\PYG{p}{]} \PYG{o}{=} \PYG{n}{data}\PYG{p}{[}\PYG{l+s+s1}{\PYGZsq{}}\PYG{l+s+s1}{volume}\PYG{l+s+s1}{\PYGZsq{}}\PYG{p}{]}\PYG{o}{.}\PYG{n}{rolling}\PYG{p}{(}\PYG{l+m+mi}{60}\PYG{p}{)}\PYG{o}{.}\PYG{n}{mean}\PYG{p}{(}\PYG{p}{)}
        \PYG{n}{data}\PYG{p}{[}\PYG{l+s+s1}{\PYGZsq{}}\PYG{l+s+s1}{volume\PYGZus{}std}\PYG{l+s+s1}{\PYGZsq{}}\PYG{p}{]} \PYG{o}{=} \PYG{n}{data}\PYG{p}{[}\PYG{l+s+s1}{\PYGZsq{}}\PYG{l+s+s1}{volume}\PYG{l+s+s1}{\PYGZsq{}}\PYG{p}{]}\PYG{o}{.}\PYG{n}{rolling}\PYG{p}{(}\PYG{l+m+mi}{60}\PYG{p}{)}\PYG{o}{.}\PYG{n}{std}\PYG{p}{(}\PYG{p}{)}
        \PYG{n}{data}\PYG{p}{[}\PYG{l+s+s1}{\PYGZsq{}}\PYG{l+s+s1}{volume\PYGZus{}z}\PYG{l+s+s1}{\PYGZsq{}}\PYG{p}{]} \PYG{o}{=} \PYG{p}{(}\PYG{n}{data}\PYG{p}{[}\PYG{l+s+s1}{\PYGZsq{}}\PYG{l+s+s1}{volume}\PYG{l+s+s1}{\PYGZsq{}}\PYG{p}{]} \PYG{o}{\PYGZhy{}} \PYG{n}{data}\PYG{p}{[}\PYG{l+s+s1}{\PYGZsq{}}\PYG{l+s+s1}{volume\PYGZus{}mean}\PYG{l+s+s1}{\PYGZsq{}}\PYG{p}{]}\PYG{p}{)}\PYG{o}{/}\PYG{n}{data}\PYG{p}{[}\PYG{l+s+s1}{\PYGZsq{}}\PYG{l+s+s1}{volume\PYGZus{}std}\PYG{l+s+s1}{\PYGZsq{}}\PYG{p}{]} \PYG{c+c1}{\PYGZsh{} 거래량은 종목과 주가에 따라 다르기 떄문에 표준화한 값이 필요함}
        \PYG{n}{data}\PYG{p}{[}\PYG{l+s+s1}{\PYGZsq{}}\PYG{l+s+s1}{z\PYGZgt{}1.96}\PYG{l+s+s1}{\PYGZsq{}}\PYG{p}{]} \PYG{o}{=} \PYG{p}{(}\PYG{n}{data}\PYG{p}{[}\PYG{l+s+s1}{\PYGZsq{}}\PYG{l+s+s1}{close}\PYG{l+s+s1}{\PYGZsq{}}\PYG{p}{]} \PYG{o}{\PYGZgt{}} \PYG{n}{data}\PYG{p}{[}\PYG{l+s+s1}{\PYGZsq{}}\PYG{l+s+s1}{open}\PYG{l+s+s1}{\PYGZsq{}}\PYG{p}{]}\PYG{p}{)}\PYG{o}{*}\PYG{p}{(}\PYG{n}{data}\PYG{p}{[}\PYG{l+s+s1}{\PYGZsq{}}\PYG{l+s+s1}{volume\PYGZus{}z}\PYG{l+s+s1}{\PYGZsq{}}\PYG{p}{]} \PYG{o}{\PYGZgt{}} \PYG{l+m+mf}{1.65}\PYG{p}{)}\PYG{o}{.}\PYG{n}{astype}\PYG{p}{(}\PYG{n+nb}{int}\PYG{p}{)} \PYG{c+c1}{\PYGZsh{} 양봉이면서 거래량이 90\PYGZpc{}신뢰구간을 벗어난 날}
        \PYG{n}{data}\PYG{p}{[}\PYG{l+s+s1}{\PYGZsq{}}\PYG{l+s+s1}{num\PYGZus{}z\PYGZgt{}1.96}\PYG{l+s+s1}{\PYGZsq{}}\PYG{p}{]} \PYG{o}{=}  \PYG{n}{data}\PYG{p}{[}\PYG{l+s+s1}{\PYGZsq{}}\PYG{l+s+s1}{z\PYGZgt{}1.96}\PYG{l+s+s1}{\PYGZsq{}}\PYG{p}{]}\PYG{o}{.}\PYG{n}{rolling}\PYG{p}{(}\PYG{l+m+mi}{60}\PYG{p}{)}\PYG{o}{.}\PYG{n}{sum}\PYG{p}{(}\PYG{p}{)}  \PYG{c+c1}{\PYGZsh{} 양봉이면서 거래량이 90\PYGZpc{} 신뢰구간을 벗어난 날을 카운트}

        \PYG{c+c1}{\PYGZsh{} 주가지수보다 더 좋은 수익율을 보여준다}
        \PYG{n}{data}\PYG{p}{[}\PYG{l+s+s1}{\PYGZsq{}}\PYG{l+s+s1}{num\PYGZus{}win\PYGZus{}market}\PYG{l+s+s1}{\PYGZsq{}}\PYG{p}{]} \PYG{o}{=} \PYG{n}{data}\PYG{p}{[}\PYG{l+s+s1}{\PYGZsq{}}\PYG{l+s+s1}{win\PYGZus{}market}\PYG{l+s+s1}{\PYGZsq{}}\PYG{p}{]}\PYG{o}{.}\PYG{n}{rolling}\PYG{p}{(}\PYG{l+m+mi}{60}\PYG{p}{)}\PYG{o}{.}\PYG{n}{sum}\PYG{p}{(}\PYG{p}{)} \PYG{c+c1}{\PYGZsh{} 주가지수 수익율이 1 보다 작을 때, 종목 수익율이 1 보다 큰 날 수}
        \PYG{n}{data}\PYG{p}{[}\PYG{l+s+s1}{\PYGZsq{}}\PYG{l+s+s1}{pct\PYGZus{}win\PYGZus{}market}\PYG{l+s+s1}{\PYGZsq{}}\PYG{p}{]} \PYG{o}{=} \PYG{p}{(}\PYG{n}{data}\PYG{p}{[}\PYG{l+s+s1}{\PYGZsq{}}\PYG{l+s+s1}{return}\PYG{l+s+s1}{\PYGZsq{}}\PYG{p}{]}\PYG{o}{/}\PYG{n}{data}\PYG{p}{[}\PYG{l+s+s1}{\PYGZsq{}}\PYG{l+s+s1}{kosdaq\PYGZus{}return}\PYG{l+s+s1}{\PYGZsq{}}\PYG{p}{]}\PYG{p}{)}\PYG{o}{.}\PYG{n}{rolling}\PYG{p}{(}\PYG{l+m+mi}{60}\PYG{p}{)}\PYG{o}{.}\PYG{n}{mean}\PYG{p}{(}\PYG{p}{)} \PYG{c+c1}{\PYGZsh{} 주가지수 수익율 대비 종목 수익율}


        \PYG{c+c1}{\PYGZsh{} 동종업체 수익률보다 더 좋은 수익율을 보여준다.           }
        \PYG{n}{data}\PYG{p}{[}\PYG{l+s+s1}{\PYGZsq{}}\PYG{l+s+s1}{return\PYGZus{}mean}\PYG{l+s+s1}{\PYGZsq{}}\PYG{p}{]} \PYG{o}{=} \PYG{n}{data}\PYG{p}{[}\PYG{l+s+s1}{\PYGZsq{}}\PYG{l+s+s1}{return}\PYG{l+s+s1}{\PYGZsq{}}\PYG{p}{]}\PYG{o}{.}\PYG{n}{rolling}\PYG{p}{(}\PYG{l+m+mi}{60}\PYG{p}{)}\PYG{o}{.}\PYG{n}{mean}\PYG{p}{(}\PYG{p}{)} \PYG{c+c1}{\PYGZsh{} 종목별 최근 60 일 수익율의 평균}
        \PYG{n}{data}\PYG{p}{[}\PYG{l+s+s1}{\PYGZsq{}}\PYG{l+s+s1}{sector}\PYG{l+s+s1}{\PYGZsq{}}\PYG{p}{]} \PYG{o}{=} \PYG{n}{sector}    
        \PYG{n}{data}\PYG{p}{[}\PYG{l+s+s1}{\PYGZsq{}}\PYG{l+s+s1}{name}\PYG{l+s+s1}{\PYGZsq{}}\PYG{p}{]} \PYG{o}{=} \PYG{n}{name}

        \PYG{n}{data} \PYG{o}{=} \PYG{n}{data}\PYG{p}{[}\PYG{p}{(}\PYG{n}{data}\PYG{p}{[}\PYG{l+s+s1}{\PYGZsq{}}\PYG{l+s+s1}{price\PYGZus{}std}\PYG{l+s+s1}{\PYGZsq{}}\PYG{p}{]}\PYG{o}{!=}\PYG{l+m+mi}{0}\PYG{p}{)} \PYG{o}{\PYGZam{}} \PYG{p}{(}\PYG{n}{data}\PYG{p}{[}\PYG{l+s+s1}{\PYGZsq{}}\PYG{l+s+s1}{volume\PYGZus{}std}\PYG{l+s+s1}{\PYGZsq{}}\PYG{p}{]}\PYG{o}{!=}\PYG{l+m+mi}{0}\PYG{p}{)}\PYG{p}{]}    

        \PYG{n}{model\PYGZus{}inputs} \PYG{o}{=} \PYG{n}{pd}\PYG{o}{.}\PYG{n}{concat}\PYG{p}{(}\PYG{p}{[}\PYG{n}{data}\PYG{p}{,} \PYG{n}{model\PYGZus{}inputs}\PYG{p}{]}\PYG{p}{,} \PYG{n}{axis}\PYG{o}{=}\PYG{l+m+mi}{0}\PYG{p}{)}

    \PYG{n}{model\PYGZus{}inputs}\PYG{p}{[}\PYG{l+s+s1}{\PYGZsq{}}\PYG{l+s+s1}{sector\PYGZus{}return}\PYG{l+s+s1}{\PYGZsq{}}\PYG{p}{]} \PYG{o}{=} \PYG{n}{model\PYGZus{}inputs}\PYG{o}{.}\PYG{n}{groupby}\PYG{p}{(}\PYG{p}{[}\PYG{l+s+s1}{\PYGZsq{}}\PYG{l+s+s1}{sector}\PYG{l+s+s1}{\PYGZsq{}}\PYG{p}{,} \PYG{n}{model\PYGZus{}inputs}\PYG{o}{.}\PYG{n}{index}\PYG{p}{]}\PYG{p}{)}\PYG{p}{[}\PYG{l+s+s1}{\PYGZsq{}}\PYG{l+s+s1}{return}\PYG{l+s+s1}{\PYGZsq{}}\PYG{p}{]}\PYG{o}{.}\PYG{n}{transform}\PYG{p}{(}\PYG{k}{lambda} \PYG{n}{x}\PYG{p}{:} \PYG{n}{x}\PYG{o}{.}\PYG{n}{mean}\PYG{p}{(}\PYG{p}{)}\PYG{p}{)} \PYG{c+c1}{\PYGZsh{} 섹터의 평균 수익율 계산}
    \PYG{n}{model\PYGZus{}inputs}\PYG{p}{[}\PYG{l+s+s1}{\PYGZsq{}}\PYG{l+s+s1}{return over sector}\PYG{l+s+s1}{\PYGZsq{}}\PYG{p}{]} \PYG{o}{=} \PYG{p}{(}\PYG{n}{model\PYGZus{}inputs}\PYG{p}{[}\PYG{l+s+s1}{\PYGZsq{}}\PYG{l+s+s1}{return}\PYG{l+s+s1}{\PYGZsq{}}\PYG{p}{]}\PYG{o}{/}\PYG{n}{model\PYGZus{}inputs}\PYG{p}{[}\PYG{l+s+s1}{\PYGZsq{}}\PYG{l+s+s1}{sector\PYGZus{}return}\PYG{l+s+s1}{\PYGZsq{}}\PYG{p}{]}\PYG{p}{)} \PYG{c+c1}{\PYGZsh{} 섹터 평균 수익률 대비 종목 수익률 계산}
    \PYG{n}{model\PYGZus{}inputs}\PYG{o}{.}\PYG{n}{dropna}\PYG{p}{(}\PYG{n}{inplace}\PYG{o}{=}\PYG{k+kc}{True}\PYG{p}{)} \PYG{c+c1}{\PYGZsh{} Missing 값 있는 행 모두 제거}


    \PYG{n}{feature\PYGZus{}list} \PYG{o}{=} \PYG{p}{[}\PYG{l+s+s1}{\PYGZsq{}}\PYG{l+s+s1}{price\PYGZus{}z}\PYG{l+s+s1}{\PYGZsq{}}\PYG{p}{,}\PYG{l+s+s1}{\PYGZsq{}}\PYG{l+s+s1}{volume\PYGZus{}z}\PYG{l+s+s1}{\PYGZsq{}}\PYG{p}{,}\PYG{l+s+s1}{\PYGZsq{}}\PYG{l+s+s1}{num\PYGZus{}high/close}\PYG{l+s+s1}{\PYGZsq{}}\PYG{p}{,}\PYG{l+s+s1}{\PYGZsq{}}\PYG{l+s+s1}{num\PYGZus{}win\PYGZus{}market}\PYG{l+s+s1}{\PYGZsq{}}\PYG{p}{,}\PYG{l+s+s1}{\PYGZsq{}}\PYG{l+s+s1}{pct\PYGZus{}win\PYGZus{}market}\PYG{l+s+s1}{\PYGZsq{}}\PYG{p}{,}\PYG{l+s+s1}{\PYGZsq{}}\PYG{l+s+s1}{return over sector}\PYG{l+s+s1}{\PYGZsq{}}\PYG{p}{]}

    \PYG{n}{X} \PYG{o}{=} \PYG{n}{model\PYGZus{}inputs}\PYG{o}{.}\PYG{n}{loc}\PYG{p}{[}\PYG{n}{today\PYGZus{}dt}\PYG{p}{]}\PYG{p}{[}\PYG{p}{[}\PYG{l+s+s1}{\PYGZsq{}}\PYG{l+s+s1}{code}\PYG{l+s+s1}{\PYGZsq{}}\PYG{p}{,}\PYG{l+s+s1}{\PYGZsq{}}\PYG{l+s+s1}{name}\PYG{l+s+s1}{\PYGZsq{}}\PYG{p}{,}\PYG{l+s+s1}{\PYGZsq{}}\PYG{l+s+s1}{return}\PYG{l+s+s1}{\PYGZsq{}}\PYG{p}{,}\PYG{l+s+s1}{\PYGZsq{}}\PYG{l+s+s1}{kosdaq\PYGZus{}return}\PYG{l+s+s1}{\PYGZsq{}}\PYG{p}{,}\PYG{l+s+s1}{\PYGZsq{}}\PYG{l+s+s1}{close}\PYG{l+s+s1}{\PYGZsq{}}\PYG{p}{]} \PYG{o}{+} \PYG{n}{feature\PYGZus{}list}\PYG{p}{]}\PYG{o}{.}\PYG{n}{set\PYGZus{}index}\PYG{p}{(}\PYG{l+s+s1}{\PYGZsq{}}\PYG{l+s+s1}{code}\PYG{l+s+s1}{\PYGZsq{}}\PYG{p}{)} 

    \PYG{k}{with} \PYG{n+nb}{open}\PYG{p}{(}\PYG{l+s+s2}{\PYGZdq{}}\PYG{l+s+s2}{gam.pkl}\PYG{l+s+s2}{\PYGZdq{}}\PYG{p}{,} \PYG{l+s+s2}{\PYGZdq{}}\PYG{l+s+s2}{rb}\PYG{l+s+s2}{\PYGZdq{}}\PYG{p}{)} \PYG{k}{as} \PYG{n}{file}\PYG{p}{:}
        \PYG{n}{gam} \PYG{o}{=} \PYG{n}{pickle}\PYG{o}{.}\PYG{n}{load}\PYG{p}{(}\PYG{n}{file}\PYG{p}{)}     

    \PYG{n}{yhat} \PYG{o}{=} \PYG{n}{gam}\PYG{o}{.}\PYG{n}{predict\PYGZus{}proba}\PYG{p}{(}\PYG{n}{X}\PYG{p}{[}\PYG{n}{feature\PYGZus{}list}\PYG{p}{]}\PYG{p}{)}
    \PYG{n}{X}\PYG{p}{[}\PYG{l+s+s1}{\PYGZsq{}}\PYG{l+s+s1}{yhat}\PYG{l+s+s1}{\PYGZsq{}}\PYG{p}{]} \PYG{o}{=} \PYG{n}{yhat}

    \PYG{n}{tops} \PYG{o}{=} \PYG{n}{X}\PYG{p}{[}\PYG{n}{X}\PYG{p}{[}\PYG{l+s+s1}{\PYGZsq{}}\PYG{l+s+s1}{yhat}\PYG{l+s+s1}{\PYGZsq{}}\PYG{p}{]} \PYG{o}{\PYGZgt{}}\PYG{o}{=} \PYG{l+m+mf}{0.3}\PYG{p}{]}\PYG{o}{.}\PYG{n}{sort\PYGZus{}values}\PYG{p}{(}\PYG{n}{by}\PYG{o}{=}\PYG{l+s+s1}{\PYGZsq{}}\PYG{l+s+s1}{yhat}\PYG{l+s+s1}{\PYGZsq{}}\PYG{p}{,} \PYG{n}{ascending}\PYG{o}{=}\PYG{k+kc}{False}\PYG{p}{)} \PYG{c+c1}{\PYGZsh{} 스코어 0.3 이상 종목만 }
    \PYG{n+nb}{print}\PYG{p}{(}\PYG{n+nb}{len}\PYG{p}{(}\PYG{n}{tops}\PYG{p}{)}\PYG{p}{)}    
  
    \PYG{n}{select\PYGZus{}tops} \PYG{o}{=} \PYG{n}{tops}\PYG{p}{[}\PYG{p}{(}\PYG{n}{tops}\PYG{p}{[}\PYG{l+s+s1}{\PYGZsq{}}\PYG{l+s+s1}{return}\PYG{l+s+s1}{\PYGZsq{}}\PYG{p}{]} \PYG{o}{\PYGZgt{}} \PYG{l+m+mf}{1.03}\PYG{p}{)} \PYG{o}{\PYGZam{}} \PYG{p}{(}\PYG{n}{tops}\PYG{p}{[}\PYG{l+s+s1}{\PYGZsq{}}\PYG{l+s+s1}{price\PYGZus{}z}\PYG{l+s+s1}{\PYGZsq{}}\PYG{p}{]} \PYG{o}{\PYGZlt{}} \PYG{l+m+mi}{0}\PYG{p}{)}\PYG{p}{]}\PYG{p}{[}\PYG{p}{[}\PYG{l+s+s1}{\PYGZsq{}}\PYG{l+s+s1}{name}\PYG{l+s+s1}{\PYGZsq{}}\PYG{p}{,}\PYG{l+s+s1}{\PYGZsq{}}\PYG{l+s+s1}{return}\PYG{l+s+s1}{\PYGZsq{}}\PYG{p}{,}\PYG{l+s+s1}{\PYGZsq{}}\PYG{l+s+s1}{price\PYGZus{}z}\PYG{l+s+s1}{\PYGZsq{}}\PYG{p}{,}\PYG{l+s+s1}{\PYGZsq{}}\PYG{l+s+s1}{yhat}\PYG{l+s+s1}{\PYGZsq{}}\PYG{p}{,}\PYG{l+s+s1}{\PYGZsq{}}\PYG{l+s+s1}{return}\PYG{l+s+s1}{\PYGZsq{}}\PYG{p}{,} \PYG{l+s+s1}{\PYGZsq{}}\PYG{l+s+s1}{kosdaq\PYGZus{}return}\PYG{l+s+s1}{\PYGZsq{}}\PYG{p}{,}\PYG{l+s+s1}{\PYGZsq{}}\PYG{l+s+s1}{close}\PYG{l+s+s1}{\PYGZsq{}}\PYG{p}{]}\PYG{p}{]}       
    
    \PYG{c+c1}{\PYGZsh{}  코스닥 지수에 따라 익절/손절 라인 변경    }
    \PYG{n}{cuts} \PYG{o}{=} \PYG{n}{select\PYGZus{}tops}\PYG{p}{[}\PYG{l+s+s1}{\PYGZsq{}}\PYG{l+s+s1}{kosdaq\PYGZus{}return}\PYG{l+s+s1}{\PYGZsq{}}\PYG{p}{]}\PYG{o}{.}\PYG{n}{apply}\PYG{p}{(}\PYG{n}{profit\PYGZus{}loss\PYGZus{}cut}\PYG{p}{)}
    
    \PYG{n}{select\PYGZus{}tops}\PYG{p}{[}\PYG{l+s+s1}{\PYGZsq{}}\PYG{l+s+s1}{profit\PYGZus{}cut}\PYG{l+s+s1}{\PYGZsq{}}\PYG{p}{]} \PYG{o}{=} \PYG{p}{[}\PYG{n}{c}\PYG{p}{[}\PYG{l+m+mi}{0}\PYG{p}{]} \PYG{k}{for} \PYG{n}{c} \PYG{o+ow}{in} \PYG{n}{cuts}\PYG{p}{]}
    \PYG{n}{select\PYGZus{}tops}\PYG{p}{[}\PYG{l+s+s1}{\PYGZsq{}}\PYG{l+s+s1}{loss\PYGZus{}cut}\PYG{l+s+s1}{\PYGZsq{}}\PYG{p}{]}   \PYG{o}{=} \PYG{p}{[}\PYG{n}{c}\PYG{p}{[}\PYG{l+m+mi}{1}\PYG{p}{]} \PYG{k}{for} \PYG{n}{c} \PYG{o+ow}{in} \PYG{n}{cuts}\PYG{p}{]}    
    
    \PYG{k}{if} \PYG{n+nb}{len}\PYG{p}{(}\PYG{n}{select\PYGZus{}tops}\PYG{p}{)} \PYG{o}{\PYGZgt{}} \PYG{l+m+mi}{1}\PYG{p}{:} \PYG{c+c1}{\PYGZsh{} 최소한 2개 종목 \PYGZhy{} 추천 리스크 분산        }
        \PYG{k}{return} \PYG{n}{select\PYGZus{}tops}    
    
    \PYG{k}{else}\PYG{p}{:}
        \PYG{k}{return} \PYG{k+kc}{None}
\end{sphinxVerbatim}

\end{sphinxuseclass}\end{sphinxVerbatimInput}

\end{sphinxuseclass}
\sphinxAtStartPar
 6월 16일 추천종목을 리스트를 추출합니다.

\begin{sphinxuseclass}{cell}\begin{sphinxVerbatimInput}

\begin{sphinxuseclass}{cell_input}
\begin{sphinxVerbatim}[commandchars=\\\{\}]
\PYG{n}{results} \PYG{o}{=} \PYG{n}{select\PYGZus{}stocks\PYGZus{}sell}\PYG{p}{(}\PYG{l+s+s1}{\PYGZsq{}}\PYG{l+s+s1}{2022\PYGZhy{}06\PYGZhy{}16}\PYG{l+s+s1}{\PYGZsq{}}\PYG{p}{)}
\PYG{n}{results}\PYG{o}{.}\PYG{n}{style}\PYG{o}{.}\PYG{n}{set\PYGZus{}table\PYGZus{}attributes}\PYG{p}{(}\PYG{l+s+s1}{\PYGZsq{}}\PYG{l+s+s1}{style=}\PYG{l+s+s1}{\PYGZdq{}}\PYG{l+s+s1}{font\PYGZhy{}size: 12px}\PYG{l+s+s1}{\PYGZdq{}}\PYG{l+s+s1}{\PYGZsq{}}\PYG{p}{)}\PYG{o}{.}\PYG{n}{format}\PYG{p}{(}\PYG{n}{precision}\PYG{o}{=}\PYG{l+m+mi}{3}\PYG{p}{)}
\end{sphinxVerbatim}

\end{sphinxuseclass}\end{sphinxVerbatimInput}
\begin{sphinxVerbatimOutput}

\begin{sphinxuseclass}{cell_output}
\begin{sphinxVerbatim}[commandchars=\\\{\}]
2022\PYGZhy{}03\PYGZhy{}08 00:00:00 2022\PYGZhy{}06\PYGZhy{}16
391
\end{sphinxVerbatim}

\begin{sphinxVerbatim}[commandchars=\\\{\}]
\PYGZlt{}pandas.io.formats.style.Styler at 0x17f8c052130\PYGZgt{}
\end{sphinxVerbatim}

\end{sphinxuseclass}\end{sphinxVerbatimOutput}

\end{sphinxuseclass}
\sphinxAtStartPar
 예측 스코어가 높은 상위 5 개 종목만 선택합니다.

\begin{sphinxuseclass}{cell}\begin{sphinxVerbatimInput}

\begin{sphinxuseclass}{cell_input}
\begin{sphinxVerbatim}[commandchars=\\\{\}]
\PYG{n}{results2} \PYG{o}{=} \PYG{n}{results}\PYG{o}{.}\PYG{n}{head}\PYG{p}{(}\PYG{l+m+mi}{5}\PYG{p}{)}\PYG{o}{.}\PYG{n}{style}\PYG{o}{.}\PYG{n}{set\PYGZus{}table\PYGZus{}attributes}\PYG{p}{(}\PYG{l+s+s1}{\PYGZsq{}}\PYG{l+s+s1}{style=}\PYG{l+s+s1}{\PYGZdq{}}\PYG{l+s+s1}{font\PYGZhy{}size: 12px}\PYG{l+s+s1}{\PYGZdq{}}\PYG{l+s+s1}{\PYGZsq{}}\PYG{p}{)}\PYG{o}{.}\PYG{n}{format}\PYG{p}{(}\PYG{n}{precision}\PYG{o}{=}\PYG{l+m+mi}{3}\PYG{p}{)}
\PYG{n}{results2}
\end{sphinxVerbatim}

\end{sphinxuseclass}\end{sphinxVerbatimInput}
\begin{sphinxVerbatimOutput}

\begin{sphinxuseclass}{cell_output}
\begin{sphinxVerbatim}[commandchars=\\\{\}]
\PYGZlt{}pandas.io.formats.style.Styler at 0x17f8c052880\PYGZgt{}
\end{sphinxVerbatim}

\end{sphinxuseclass}\end{sphinxVerbatimOutput}

\end{sphinxuseclass}
\sphinxAtStartPar
 선택된 종목에 익절/손절 라인 값을 딕셔너리로 반환합니다.

\begin{sphinxuseclass}{cell}\begin{sphinxVerbatimInput}

\begin{sphinxuseclass}{cell_input}
\begin{sphinxVerbatim}[commandchars=\\\{\}]
\PYG{n}{select\PYGZus{}dict} \PYG{o}{=} \PYG{p}{\PYGZob{}}\PYG{p}{\PYGZcb{}}
\PYG{k}{for} \PYG{n}{code} \PYG{o+ow}{in} \PYG{n+nb}{list}\PYG{p}{(}\PYG{n}{results2}\PYG{o}{.}\PYG{n}{index}\PYG{p}{)}\PYG{p}{:}
    \PYG{n}{s} \PYG{o}{=} \PYG{n}{results}\PYG{o}{.}\PYG{n}{loc}\PYG{p}{[}\PYG{n}{code}\PYG{p}{]}
    \PYG{n}{select\PYGZus{}dict}\PYG{p}{[}\PYG{n}{code}\PYG{p}{]} \PYG{o}{=} \PYG{p}{[}\PYG{n}{s}\PYG{p}{[}\PYG{l+s+s1}{\PYGZsq{}}\PYG{l+s+s1}{name}\PYG{l+s+s1}{\PYGZsq{}}\PYG{p}{]}\PYG{p}{,} \PYG{n}{s}\PYG{p}{[}\PYG{l+s+s1}{\PYGZsq{}}\PYG{l+s+s1}{close}\PYG{l+s+s1}{\PYGZsq{}}\PYG{p}{]}\PYG{p}{,} \PYG{n}{s}\PYG{p}{[}\PYG{l+s+s1}{\PYGZsq{}}\PYG{l+s+s1}{profit\PYGZus{}cut}\PYG{l+s+s1}{\PYGZsq{}}\PYG{p}{]}\PYG{p}{,} \PYG{n}{s}\PYG{p}{[}\PYG{l+s+s1}{\PYGZsq{}}\PYG{l+s+s1}{loss\PYGZus{}cut}\PYG{l+s+s1}{\PYGZsq{}}\PYG{p}{]}\PYG{p}{]}    
\end{sphinxVerbatim}

\end{sphinxuseclass}\end{sphinxVerbatimInput}

\end{sphinxuseclass}
\begin{sphinxuseclass}{cell}\begin{sphinxVerbatimInput}

\begin{sphinxuseclass}{cell_input}
\begin{sphinxVerbatim}[commandchars=\\\{\}]
\PYG{n}{select\PYGZus{}dict}
\end{sphinxVerbatim}

\end{sphinxuseclass}\end{sphinxVerbatimInput}
\begin{sphinxVerbatimOutput}

\begin{sphinxuseclass}{cell_output}
\begin{sphinxVerbatim}[commandchars=\\\{\}]
\PYGZob{}\PYGZsq{}002680\PYGZsq{}: [\PYGZsq{}한탑\PYGZsq{}, 3155, 1.05, 0.97],
 \PYGZsq{}312610\PYGZsq{}: [\PYGZsq{}에이에프더블류\PYGZsq{}, 3505, 1.05, 0.97],
 \PYGZsq{}177350\PYGZsq{}: [\PYGZsq{}베셀\PYGZsq{}, 7630, 1.05, 0.97],
 \PYGZsq{}311390\PYGZsq{}: [\PYGZsq{}네오크레마\PYGZsq{}, 13450, 1.05, 0.97],
 \PYGZsq{}192250\PYGZsq{}: [\PYGZsq{}케이사인\PYGZsq{}, 1890, 1.05, 0.97]\PYGZcb{}
\end{sphinxVerbatim}

\end{sphinxuseclass}\end{sphinxVerbatimOutput}

\end{sphinxuseclass}

\part{chapter 7}


\chapter{\sphinxstylestrong{자동매매를 해보자}}
\label{\detokenize{chapter7/7.0.0_Algo_Trading:id1}}\label{\detokenize{chapter7/7.0.0_Algo_Trading::doc}}
\sphinxAtStartPar
이베스트의 Xing API 를 이용하여 개발한 알고리즘에 따라 자동매매을 해 보겠습니다.


\part{chapter 8}


\chapter{\sphinxstylestrong{WebApp을 만들어보자}}
\label{\detokenize{chapter8/8.0.0_Web_Application:webapp}}\label{\detokenize{chapter8/8.0.0_Web_Application::doc}}
\sphinxAtStartPar
Streamlit 는 머신러닝 결과를 손쉽게 배포할 수 있는 패키지입니다. 일반적으로 웹앱은 HTML, CSS, JavaScript 등의 기술이 있어야 원하는 웹앱을 만들 수 있습니다. 파이썬만으로 웹앱을 만들 수 있게 해 주는 패키지가 Streamlit 입니다.


\part{chapter 9}


\chapter{\sphinxstylestrong{계좌인증}}
\label{\detokenize{chapter9/9.0.0_POC:id1}}\label{\detokenize{chapter9/9.0.0_POC::doc}}
\sphinxAtStartPar
계좌인증







\renewcommand{\indexname}{Index}
\printindex
\end{document}