%% Generated by Sphinx.
\def\sphinxdocclass{jupyterBook}
\documentclass[letterpaper,10pt,english]{jupyterBook}
\ifdefined\pdfpxdimen
   \let\sphinxpxdimen\pdfpxdimen\else\newdimen\sphinxpxdimen
\fi \sphinxpxdimen=.75bp\relax
\ifdefined\pdfimageresolution
    \pdfimageresolution= \numexpr \dimexpr1in\relax/\sphinxpxdimen\relax
\fi
%% let collapsible pdf bookmarks panel have high depth per default
\PassOptionsToPackage{bookmarksdepth=5}{hyperref}
%% turn off hyperref patch of \index as sphinx.xdy xindy module takes care of
%% suitable \hyperpage mark-up, working around hyperref-xindy incompatibility
\PassOptionsToPackage{hyperindex=false}{hyperref}
%% memoir class requires extra handling
\makeatletter\@ifclassloaded{memoir}
{\ifdefined\memhyperindexfalse\memhyperindexfalse\fi}{}\makeatother

\PassOptionsToPackage{warn}{textcomp}

\catcode`^^^^00a0\active\protected\def^^^^00a0{\leavevmode\nobreak\ }
\usepackage{cmap}
\usepackage{fontspec}
\defaultfontfeatures[\rmfamily,\sffamily,\ttfamily]{}
\usepackage{amsmath,amssymb,amstext}
\usepackage{polyglossia}
\setmainlanguage{english}



\setmainfont{FreeSerif}[
  Extension      = .otf,
  UprightFont    = *,
  ItalicFont     = *Italic,
  BoldFont       = *Bold,
  BoldItalicFont = *BoldItalic
]
\setsansfont{FreeSans}[
  Extension      = .otf,
  UprightFont    = *,
  ItalicFont     = *Oblique,
  BoldFont       = *Bold,
  BoldItalicFont = *BoldOblique,
]
\setmonofont{FreeMono}[
  Extension      = .otf,
  UprightFont    = *,
  ItalicFont     = *Oblique,
  BoldFont       = *Bold,
  BoldItalicFont = *BoldOblique,
]



\usepackage[Bjarne]{fncychap}
\usepackage[,numfigreset=1,mathnumfig]{sphinx}

\fvset{fontsize=\small}
\usepackage{geometry}


% Include hyperref last.
\usepackage{hyperref}
% Fix anchor placement for figures with captions.
\usepackage{hypcap}% it must be loaded after hyperref.
% Set up styles of URL: it should be placed after hyperref.
\urlstyle{same}


\usepackage{sphinxmessages}



        % Start of preamble defined in sphinx-jupyterbook-latex %
         \usepackage[Latin,Greek]{ucharclasses}
        \usepackage{unicode-math}
        % fixing title of the toc
        \addto\captionsenglish{\renewcommand{\contentsname}{Contents}}
        \hypersetup{
            pdfencoding=auto,
            psdextra
        }
        % End of preamble defined in sphinx-jupyterbook-latex %
        

\title{If Condition}
\date{Jul 02, 2022}
\release{}
\author{KHS}
\newcommand{\sphinxlogo}{\vbox{}}
\renewcommand{\releasename}{}
\makeindex
\begin{document}

\pagestyle{empty}
\sphinxmaketitle
\pagestyle{plain}
\sphinxtableofcontents
\pagestyle{normal}
\phantomsection\label{\detokenize{chapter2/2.1.4_Python_Basics::doc}}


\sphinxAtStartPar
파이썬의 조건문 if \textasciitilde{} else 형식으로 다른 컴퓨터 언어와 다르지 않습니다. 단지 else if 부분은 줄여서 elif 로 씁니다. 아래 예제를 보시면 쉽게 이해가 되실 것으로 생각합니다.

\begin{sphinxuseclass}{cell}\begin{sphinxVerbatimInput}

\begin{sphinxuseclass}{cell_input}
\begin{sphinxVerbatim}[commandchars=\\\{\}]
\PYG{n}{a} \PYG{o}{=} \PYG{l+m+mi}{3}
\PYG{n}{b} \PYG{o}{=} \PYG{l+m+mi}{2}
\PYG{k}{if} \PYG{n}{a} \PYG{o}{\PYGZgt{}} \PYG{n}{b}\PYG{p}{:}
    \PYG{n+nb}{print}\PYG{p}{(}\PYG{l+s+s1}{\PYGZsq{}}\PYG{l+s+s1}{a \PYGZgt{} b}\PYG{l+s+s1}{\PYGZsq{}}\PYG{p}{)}
\PYG{k}{else}\PYG{p}{:}
    \PYG{n+nb}{print}\PYG{p}{(}\PYG{l+s+s1}{\PYGZsq{}}\PYG{l+s+s1}{a \PYGZlt{}= b}\PYG{l+s+s1}{\PYGZsq{}}\PYG{p}{)}
\end{sphinxVerbatim}

\end{sphinxuseclass}\end{sphinxVerbatimInput}
\begin{sphinxVerbatimOutput}

\begin{sphinxuseclass}{cell_output}
\begin{sphinxVerbatim}[commandchars=\\\{\}]
a \PYGZgt{} b
\end{sphinxVerbatim}

\end{sphinxuseclass}\end{sphinxVerbatimOutput}

\end{sphinxuseclass}
\begin{sphinxuseclass}{cell}\begin{sphinxVerbatimInput}

\begin{sphinxuseclass}{cell_input}
\begin{sphinxVerbatim}[commandchars=\\\{\}]
\PYG{n}{num\PYGZus{}list} \PYG{o}{=} \PYG{p}{[}\PYG{l+m+mi}{1}\PYG{p}{,}\PYG{l+m+mi}{2}\PYG{p}{,}\PYG{l+m+mi}{3}\PYG{p}{,}\PYG{l+m+mi}{4}\PYG{p}{,}\PYG{l+m+mi}{5}\PYG{p}{,}\PYG{l+m+mi}{6}\PYG{p}{]}

\PYG{k}{for} \PYG{n}{i} \PYG{o+ow}{in} \PYG{n}{num\PYGZus{}list}\PYG{p}{:}
    \PYG{k}{if} \PYG{n}{i} \PYG{o}{\PYGZlt{}} \PYG{l+m+mi}{3}\PYG{p}{:}
        \PYG{n+nb}{print}\PYG{p}{(}\PYG{n}{i}\PYG{p}{,} \PYG{l+s+s1}{\PYGZsq{}}\PYG{l+s+s1}{the number is less than 3}\PYG{l+s+s1}{\PYGZsq{}}\PYG{p}{)}
    \PYG{k}{elif} \PYG{n}{i} \PYG{o}{\PYGZgt{}} \PYG{l+m+mi}{3}\PYG{p}{:}
        \PYG{n+nb}{print}\PYG{p}{(}\PYG{n}{i}\PYG{p}{,} \PYG{l+s+s1}{\PYGZsq{}}\PYG{l+s+s1}{The number is greater than 3}\PYG{l+s+s1}{\PYGZsq{}}\PYG{p}{)}
    \PYG{k}{else}\PYG{p}{:}
        \PYG{n+nb}{print}\PYG{p}{(}\PYG{n}{i}\PYG{p}{,} \PYG{l+s+s1}{\PYGZsq{}}\PYG{l+s+s1}{The number is 3}\PYG{l+s+s1}{\PYGZsq{}}\PYG{p}{)}
\end{sphinxVerbatim}

\end{sphinxuseclass}\end{sphinxVerbatimInput}
\begin{sphinxVerbatimOutput}

\begin{sphinxuseclass}{cell_output}
\begin{sphinxVerbatim}[commandchars=\\\{\}]
1 the number is less than 3
2 the number is less than 3
3 The number is 3
4 The number is greater than 3
5 The number is greater than 3
6 The number is greater than 3
\end{sphinxVerbatim}

\end{sphinxuseclass}\end{sphinxVerbatimOutput}

\end{sphinxuseclass}
\begin{sphinxuseclass}{cell}\begin{sphinxVerbatimInput}

\begin{sphinxuseclass}{cell_input}
\begin{sphinxVerbatim}[commandchars=\\\{\}]
\PYG{k}{for} \PYG{n}{i} \PYG{o+ow}{in} \PYG{n}{num\PYGZus{}list}\PYG{p}{:}
    \PYG{k}{if} \PYG{n}{i} \PYG{o}{\PYGZlt{}} \PYG{l+m+mi}{3}\PYG{p}{:}
        \PYG{n+nb}{print}\PYG{p}{(}\PYG{n}{i}\PYG{p}{,} \PYG{l+s+s1}{\PYGZsq{}}\PYG{l+s+s1}{the number is less than 3}\PYG{l+s+s1}{\PYGZsq{}}\PYG{p}{)}
    \PYG{k}{elif} \PYG{n}{i} \PYG{o}{\PYGZgt{}} \PYG{l+m+mi}{3}\PYG{p}{:}
        \PYG{k}{pass}  \PYG{c+c1}{\PYGZsh{} 아무런 처리를 하지 않음}
    \PYG{k}{else}\PYG{p}{:}
        \PYG{n+nb}{print}\PYG{p}{(}\PYG{n}{i}\PYG{p}{,} \PYG{l+s+s1}{\PYGZsq{}}\PYG{l+s+s1}{The number is 3}\PYG{l+s+s1}{\PYGZsq{}}\PYG{p}{)}
\end{sphinxVerbatim}

\end{sphinxuseclass}\end{sphinxVerbatimInput}
\begin{sphinxVerbatimOutput}

\begin{sphinxuseclass}{cell_output}
\begin{sphinxVerbatim}[commandchars=\\\{\}]
1 the number is less than 3
2 the number is less than 3
3 The number is 3
\end{sphinxVerbatim}

\end{sphinxuseclass}\end{sphinxVerbatimOutput}

\end{sphinxuseclass}






\renewcommand{\indexname}{Index}
\printindex
\end{document}