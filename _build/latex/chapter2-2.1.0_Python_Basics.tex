%% Generated by Sphinx.
\def\sphinxdocclass{jupyterBook}
\documentclass[letterpaper,10pt,english]{jupyterBook}
\ifdefined\pdfpxdimen
   \let\sphinxpxdimen\pdfpxdimen\else\newdimen\sphinxpxdimen
\fi \sphinxpxdimen=.75bp\relax
\ifdefined\pdfimageresolution
    \pdfimageresolution= \numexpr \dimexpr1in\relax/\sphinxpxdimen\relax
\fi
%% let collapsible pdf bookmarks panel have high depth per default
\PassOptionsToPackage{bookmarksdepth=5}{hyperref}
%% turn off hyperref patch of \index as sphinx.xdy xindy module takes care of
%% suitable \hyperpage mark-up, working around hyperref-xindy incompatibility
\PassOptionsToPackage{hyperindex=false}{hyperref}
%% memoir class requires extra handling
\makeatletter\@ifclassloaded{memoir}
{\ifdefined\memhyperindexfalse\memhyperindexfalse\fi}{}\makeatother

\PassOptionsToPackage{warn}{textcomp}

\catcode`^^^^00a0\active\protected\def^^^^00a0{\leavevmode\nobreak\ }
\usepackage{cmap}
\usepackage{fontspec}
\defaultfontfeatures[\rmfamily,\sffamily,\ttfamily]{}
\usepackage{amsmath,amssymb,amstext}
\usepackage{polyglossia}
\setmainlanguage{english}



\setmainfont{FreeSerif}[
  Extension      = .otf,
  UprightFont    = *,
  ItalicFont     = *Italic,
  BoldFont       = *Bold,
  BoldItalicFont = *BoldItalic
]
\setsansfont{FreeSans}[
  Extension      = .otf,
  UprightFont    = *,
  ItalicFont     = *Oblique,
  BoldFont       = *Bold,
  BoldItalicFont = *BoldOblique,
]
\setmonofont{FreeMono}[
  Extension      = .otf,
  UprightFont    = *,
  ItalicFont     = *Oblique,
  BoldFont       = *Bold,
  BoldItalicFont = *BoldOblique,
]



\usepackage[Bjarne]{fncychap}
\usepackage[,numfigreset=1,mathnumfig]{sphinx}

\fvset{fontsize=\small}
\usepackage{geometry}


% Include hyperref last.
\usepackage{hyperref}
% Fix anchor placement for figures with captions.
\usepackage{hypcap}% it must be loaded after hyperref.
% Set up styles of URL: it should be placed after hyperref.
\urlstyle{same}


\usepackage{sphinxmessages}



        % Start of preamble defined in sphinx-jupyterbook-latex %
         \usepackage[Latin,Greek]{ucharclasses}
        \usepackage{unicode-math}
        % fixing title of the toc
        \addto\captionsenglish{\renewcommand{\contentsname}{Contents}}
        \hypersetup{
            pdfencoding=auto,
            psdextra
        }
        % End of preamble defined in sphinx-jupyterbook-latex %
        

\title{파이썬 데이터분석 기초}
\date{Jul 02, 2022}
\release{}
\author{KHS}
\newcommand{\sphinxlogo}{\vbox{}}
\renewcommand{\releasename}{}
\makeindex
\begin{document}

\pagestyle{empty}
\sphinxmaketitle
\pagestyle{plain}
\sphinxtableofcontents
\pagestyle{normal}
\phantomsection\label{\detokenize{chapter2/2.1.0_Python_Basics::doc}}


\sphinxAtStartPar
이번 장에서는 데이터 분석에 꼭 필요한 부분만 다룰 수 있도록 하겠습니다. 파이썬 기초 학습은 유튜브에서 좋은 강의를 쉽게 찾으실 수 있습니다. 이 장에서는 주가데이터 분석에 필요한 기술만 우선적으로 익히고, 배운 기술을 이용하여 래리 윌리암스의 변동성 돌파전략을 구현해 보겠습니다. 알고리즘 개발을 위한 파이썬 코드는 쥬피터노트북 환경에서 작성하고 실행을 하였습니다. 주피터노트북은 대화형으로 코딩을 할 수 있는 파이썬 에디터이기 때문에 데이터분석가 가장 선호하는 에디터입니다. 아나콘다를 설치하면 데이터분석을 위한 여러 패키지와 함께 주피터노트북이 설치됩니다. 아나콘다 설치 관련하여 다양한 유투브 동영상과 온라인 자료가 있습니다. 본서에는 7장(자동매매)에서 아나콘다 설치를 위한 가이드를 제공합니다. 아나콘다를 설치 못하신 독자는 아나콘다 설치 완료하시고 진행하시면 되겠습니다.


\part{Data Type}
\label{\detokenize{chapter2/2.1.1_Python_Basics:data-type}}\label{\detokenize{chapter2/2.1.1_Python_Basics::doc}}
\sphinxAtStartPar
먼저 데이터 타입에 대한 이해가 필요합니다. 주식 데이터 분석에서 활용할 데이터 타입은 숫자(Number),  문자열(String),  날짜(Date), 딕셔너리(Dictionary), 리스트(List), 시리즈(Series), 데이터프레임(DataFrame) 등이 있습니다. 각 타입의 형식은 아래와 같습니다.

\begin{sphinxuseclass}{cell}\begin{sphinxVerbatimInput}

\begin{sphinxuseclass}{cell_input}
\begin{sphinxVerbatim}[commandchars=\\\{\}]
\PYG{c+c1}{\PYGZsh{} Number}
\PYG{n}{n1} \PYG{o}{=} \PYG{l+m+mi}{123}
\PYG{n}{n2} \PYG{o}{=} \PYG{l+m+mi}{234}

\PYG{c+c1}{\PYGZsh{} String }
\PYG{n}{s1} \PYG{o}{=} \PYG{l+s+s1}{\PYGZsq{}}\PYG{l+s+s1}{string}\PYG{l+s+s1}{\PYGZsq{}}
\PYG{n}{s2} \PYG{o}{=} \PYG{l+s+s1}{\PYGZsq{}}\PYG{l+s+s1}{I am Tom}\PYG{l+s+s1}{\PYGZsq{}}

\PYG{c+c1}{\PYGZsh{} Date}
\PYG{k+kn}{import} \PYG{n+nn}{datetime}
\PYG{n}{d1} \PYG{o}{=} \PYG{n}{datetime}\PYG{o}{.}\PYG{n}{datetime}\PYG{p}{(}\PYG{l+m+mi}{2021}\PYG{p}{,} \PYG{l+m+mi}{1}\PYG{p}{,} \PYG{l+m+mi}{3}\PYG{p}{,} \PYG{l+m+mi}{0}\PYG{p}{,} \PYG{l+m+mi}{0}\PYG{p}{)}
\PYG{n}{yymmdd} \PYG{o}{=} \PYG{l+s+s1}{\PYGZsq{}}\PYG{l+s+s1}{2021\PYGZhy{}01\PYGZhy{}03}\PYG{l+s+s1}{\PYGZsq{}}
\PYG{n}{d2} \PYG{o}{=} \PYG{n}{datetime}\PYG{o}{.}\PYG{n}{datetime}\PYG{o}{.}\PYG{n}{strptime}\PYG{p}{(}\PYG{n}{yymmdd}\PYG{p}{,} \PYG{l+s+s1}{\PYGZsq{}}\PYG{l+s+s1}{\PYGZpc{}}\PYG{l+s+s1}{Y\PYGZhy{}}\PYG{l+s+s1}{\PYGZpc{}}\PYG{l+s+s1}{m\PYGZhy{}}\PYG{l+s+si}{\PYGZpc{}d}\PYG{l+s+s1}{\PYGZsq{}}\PYG{p}{)}

\PYG{c+c1}{\PYGZsh{} Dictionary}
\PYG{n}{dic1} \PYG{o}{=} \PYG{p}{\PYGZob{}}\PYG{l+s+s1}{\PYGZsq{}}\PYG{l+s+s1}{a}\PYG{l+s+s1}{\PYGZsq{}}\PYG{p}{:}\PYG{l+m+mi}{11}\PYG{p}{,} \PYG{l+s+s1}{\PYGZsq{}}\PYG{l+s+s1}{b}\PYG{l+s+s1}{\PYGZsq{}}\PYG{p}{:}\PYG{l+m+mi}{12}\PYG{p}{,} \PYG{l+s+s1}{\PYGZsq{}}\PYG{l+s+s1}{c}\PYG{l+s+s1}{\PYGZsq{}}\PYG{p}{:}\PYG{l+m+mi}{13}\PYG{p}{\PYGZcb{}}

\PYG{c+c1}{\PYGZsh{} List}
\PYG{n}{l1} \PYG{o}{=} \PYG{p}{[}\PYG{l+m+mi}{1}\PYG{p}{,}\PYG{l+m+mi}{2}\PYG{p}{,}\PYG{l+m+mi}{3}\PYG{p}{]}
\PYG{n}{l2} \PYG{o}{=}  \PYG{p}{[}\PYG{l+s+s1}{\PYGZsq{}}\PYG{l+s+s1}{a}\PYG{l+s+s1}{\PYGZsq{}}\PYG{p}{,}\PYG{l+s+s1}{\PYGZsq{}}\PYG{l+s+s1}{b}\PYG{l+s+s1}{\PYGZsq{}}\PYG{p}{,}\PYG{l+s+s1}{\PYGZsq{}}\PYG{l+s+s1}{c}\PYG{l+s+s1}{\PYGZsq{}}\PYG{p}{]}

\PYG{c+c1}{\PYGZsh{} Series}
\PYG{k+kn}{import} \PYG{n+nn}{pandas} \PYG{k}{as} \PYG{n+nn}{pd}
\PYG{n}{ss1} \PYG{o}{=} \PYG{p}{[}\PYG{l+m+mi}{11}\PYG{p}{,}\PYG{l+m+mi}{12}\PYG{p}{,}\PYG{l+m+mi}{13}\PYG{p}{,}\PYG{l+m+mi}{14}\PYG{p}{,}\PYG{l+m+mi}{15}\PYG{p}{]}
\PYG{n}{ss2} \PYG{o}{=} \PYG{n}{pd}\PYG{o}{.}\PYG{n}{Series}\PYG{p}{(}\PYG{n}{ss1}\PYG{p}{)}

\PYG{c+c1}{\PYGZsh{} DataFrame}
\PYG{n}{c\PYGZus{}1} \PYG{o}{=} \PYG{p}{[}\PYG{l+m+mi}{1}\PYG{p}{,}\PYG{l+m+mi}{2}\PYG{p}{,}\PYG{l+m+mi}{3}\PYG{p}{]}
\PYG{n}{c\PYGZus{}2} \PYG{o}{=}  \PYG{p}{[}\PYG{l+s+s1}{\PYGZsq{}}\PYG{l+s+s1}{a}\PYG{l+s+s1}{\PYGZsq{}}\PYG{p}{,}\PYG{l+s+s1}{\PYGZsq{}}\PYG{l+s+s1}{b}\PYG{l+s+s1}{\PYGZsq{}}\PYG{p}{,}\PYG{l+s+s1}{\PYGZsq{}}\PYG{l+s+s1}{c}\PYG{l+s+s1}{\PYGZsq{}}\PYG{p}{]}
\PYG{n}{df1} \PYG{o}{=} \PYG{n}{pd}\PYG{o}{.}\PYG{n}{DataFrame}\PYG{p}{(}\PYG{p}{\PYGZob{}}\PYG{l+s+s1}{\PYGZsq{}}\PYG{l+s+s1}{col1}\PYG{l+s+s1}{\PYGZsq{}}\PYG{p}{:} \PYG{n}{c\PYGZus{}1}\PYG{p}{,} \PYG{l+s+s1}{\PYGZsq{}}\PYG{l+s+s1}{col2}\PYG{l+s+s1}{\PYGZsq{}}\PYG{p}{:} \PYG{n}{c\PYGZus{}2}\PYG{p}{\PYGZcb{}}\PYG{p}{)}
\end{sphinxVerbatim}

\end{sphinxuseclass}\end{sphinxVerbatimInput}

\end{sphinxuseclass}

\part{String}
\label{\detokenize{chapter2/2.1.1_Python_Basics:string}}
\sphinxAtStartPar
문자열의 첫 글자부터 0, 1, 2 번째 문자 해당합니다. 예를 들어, ‘String’ 이란 문자열을 s1 이란 변수에서 저장한 경우, s1{[}0{]} 은 ‘s’ 에 해당하고, s1{[}1{]} 은 ‘t’ 에 해당합니다.

\begin{sphinxuseclass}{cell}\begin{sphinxVerbatimInput}

\begin{sphinxuseclass}{cell_input}
\begin{sphinxVerbatim}[commandchars=\\\{\}]
\PYG{c+c1}{\PYGZsh{} String }
\PYG{n}{s1} \PYG{o}{=} \PYG{l+s+s1}{\PYGZsq{}}\PYG{l+s+s1}{string}\PYG{l+s+s1}{\PYGZsq{}}
\PYG{n}{s2} \PYG{o}{=} \PYG{l+s+s1}{\PYGZsq{}}\PYG{l+s+s1}{I am Tom}\PYG{l+s+s1}{\PYGZsq{}}

\PYG{n+nb}{print}\PYG{p}{(}\PYG{n}{s1}\PYG{p}{)}
\PYG{n+nb}{print}\PYG{p}{(}\PYG{n}{s2}\PYG{p}{[}\PYG{l+m+mi}{0}\PYG{p}{]}\PYG{p}{,} \PYG{n}{s2}\PYG{p}{[}\PYG{l+m+mi}{5}\PYG{p}{]}\PYG{p}{)} \PYG{c+c1}{\PYGZsh{} I am Tom 의 첫번재[0], 6번째[5] 글자 반환. 대부분의 컴푸터 언어는 0 부터 시작.}
\end{sphinxVerbatim}

\end{sphinxuseclass}\end{sphinxVerbatimInput}
\begin{sphinxVerbatimOutput}

\begin{sphinxuseclass}{cell_output}
\begin{sphinxVerbatim}[commandchars=\\\{\}]
string
I T
\end{sphinxVerbatim}

\end{sphinxuseclass}\end{sphinxVerbatimOutput}

\end{sphinxuseclass}

\part{List}
\label{\detokenize{chapter2/2.1.1_Python_Basics:list}}
\sphinxAtStartPar
리스트는 여러 개의 원소를 대괄호 {[}{]} 에 넣은 형태입니다. 리스트도 문자열과 동일하게 0 부터 시작합니다.
아래 l1 에서 0 번째 원소는 1 이고, l2 에서는 ‘a’ 입니다. 그리고 {[}start\_point:end\_point{]} 를 형식으로 원소의 일부만 가져올 수 있습니다. 단, {[}start\_point:end\_point{]} 에서 원소는 end\_point 전까지 가져오는 사실만 유의하시면 됩니다.

\begin{sphinxuseclass}{cell}\begin{sphinxVerbatimInput}

\begin{sphinxuseclass}{cell_input}
\begin{sphinxVerbatim}[commandchars=\\\{\}]
\PYG{c+c1}{\PYGZsh{} List}
\PYG{n}{l1} \PYG{o}{=} \PYG{p}{[}\PYG{l+m+mi}{1}\PYG{p}{,}\PYG{l+m+mi}{2}\PYG{p}{,}\PYG{l+m+mi}{3}\PYG{p}{]}
\PYG{n}{l2} \PYG{o}{=}  \PYG{p}{[}\PYG{l+s+s1}{\PYGZsq{}}\PYG{l+s+s1}{a}\PYG{l+s+s1}{\PYGZsq{}}\PYG{p}{,}\PYG{l+s+s1}{\PYGZsq{}}\PYG{l+s+s1}{b}\PYG{l+s+s1}{\PYGZsq{}}\PYG{p}{,}\PYG{l+s+s1}{\PYGZsq{}}\PYG{l+s+s1}{c}\PYG{l+s+s1}{\PYGZsq{}}\PYG{p}{,}\PYG{l+s+s1}{\PYGZsq{}}\PYG{l+s+s1}{d}\PYG{l+s+s1}{\PYGZsq{}}\PYG{p}{]}

\PYG{n+nb}{print}\PYG{p}{(}\PYG{n}{l1}\PYG{p}{[}\PYG{l+m+mi}{0}\PYG{p}{]}\PYG{p}{)}
\PYG{n+nb}{print}\PYG{p}{(}\PYG{n}{l2}\PYG{p}{[}\PYG{l+m+mi}{0}\PYG{p}{]}\PYG{p}{)}
\PYG{n+nb}{print}\PYG{p}{(}\PYG{n}{l2}\PYG{p}{[}\PYG{p}{:}\PYG{l+m+mi}{2}\PYG{p}{]}\PYG{p}{)} \PYG{c+c1}{\PYGZsh{} 0 \PYGZti{} 1 번째 문자를 가져옴. 2 번째 포함되지 않음.}
\PYG{n+nb}{print}\PYG{p}{(}\PYG{n}{l2}\PYG{p}{[}\PYG{l+m+mi}{1}\PYG{p}{:}\PYG{l+m+mi}{3}\PYG{p}{]}\PYG{p}{)} \PYG{c+c1}{\PYGZsh{} 1 \PYGZti{} 2 번째 문자를 가져옴. 3 번째는 포함되지 않음}
\end{sphinxVerbatim}

\end{sphinxuseclass}\end{sphinxVerbatimInput}
\begin{sphinxVerbatimOutput}

\begin{sphinxuseclass}{cell_output}
\begin{sphinxVerbatim}[commandchars=\\\{\}]
1
a
[\PYGZsq{}a\PYGZsq{}, \PYGZsq{}b\PYGZsq{}]
[\PYGZsq{}b\PYGZsq{}, \PYGZsq{}c\PYGZsq{}]
\end{sphinxVerbatim}

\end{sphinxuseclass}\end{sphinxVerbatimOutput}

\end{sphinxuseclass}

\part{Number}
\label{\detokenize{chapter2/2.1.1_Python_Basics:number}}
\sphinxAtStartPar
숫자형은 정수형, 소숫점형으로 나눌 수 있으나, 아래와 같이 곧바로 사칙연산이 가능합니다.

\begin{sphinxuseclass}{cell}\begin{sphinxVerbatimInput}

\begin{sphinxuseclass}{cell_input}
\begin{sphinxVerbatim}[commandchars=\\\{\}]
\PYG{c+c1}{\PYGZsh{} Number}
\PYG{n}{n1} \PYG{o}{=} \PYG{l+m+mi}{123}
\PYG{n}{n2} \PYG{o}{=} \PYG{l+m+mi}{234}
\PYG{n+nb}{print}\PYG{p}{(}\PYG{n}{n1}\PYG{o}{*}\PYG{n}{n2}\PYG{p}{)}
\end{sphinxVerbatim}

\end{sphinxuseclass}\end{sphinxVerbatimInput}
\begin{sphinxVerbatimOutput}

\begin{sphinxuseclass}{cell_output}
\begin{sphinxVerbatim}[commandchars=\\\{\}]
28782
\end{sphinxVerbatim}

\end{sphinxuseclass}\end{sphinxVerbatimOutput}

\end{sphinxuseclass}

\part{Date}
\label{\detokenize{chapter2/2.1.1_Python_Basics:date}}
\sphinxAtStartPar
날짜형 데이터는 주식데이터 분석에 중요한 데이터형식입니다. 활용도가 아주 높습니다. 날짜 데이터를 활용하기 위해서는 먼저 datetime 패키지를 import 합니다. datetime 는 날짜 데이터를 다루기 위한 여러 메소드를 가지고 있습니다. 첫 번째 d1 변수에서 datetime.datetime 함수는 년, 월, 시 등의 숫자를 파이썬 날짜로 변형할 수 있게 해 줍니다. d2 변수는 문자열로 되어 있는 날짜를 파이썬 날짜로 변형하는 한 후 저장을 합니다. strptime 은 문자열을 파이썬 날짜로 변경해주는 메소드입니다. strptime 은 두 개의 인수가 필요한데요. 이 함수의 첫 번째 인자는 날짜 형태의 문자열, 두번 째 인자는 형식 포맷입니다. 문자열 날짜 포맷이 \%Y\sphinxhyphen{}\%m\sphinxhyphen{}\%d 형태라는 것을 인수로 알려줍니다. d3 는 다른 날짜 형식으로 되어 있는 문자열을 파이썬 날짜로 변경하는 법을 보여줍니다. d4 는 반대로 파이썬 날짜를 다시 문자열로 변경하는 법을 보여주고 있습니다. 이 때 함수는 strpftime 입니다. 마지막으로 timedelta 를 알아보겠습니다. timedelta 는 일정한 시간을 뒤로 이동한 결과를 반환합니다. 예를 들어 d1 에 hours=5 를 추가하면 2021년 1월 3일 0 시에서 2021년 1월 3일 5시로 변경됩니다. 아래 예시는 d1 에서 5 시간을 더 했을 때, 2 일 더했을 때 결과를 보여줍니다.

\begin{sphinxuseclass}{cell}\begin{sphinxVerbatimInput}

\begin{sphinxuseclass}{cell_input}
\begin{sphinxVerbatim}[commandchars=\\\{\}]
\PYG{c+c1}{\PYGZsh{} Date}
\PYG{k+kn}{import} \PYG{n+nn}{datetime}

\PYG{n}{d1} \PYG{o}{=} \PYG{n}{datetime}\PYG{o}{.}\PYG{n}{datetime}\PYG{p}{(}\PYG{l+m+mi}{2021}\PYG{p}{,} \PYG{l+m+mi}{1}\PYG{p}{,} \PYG{l+m+mi}{3}\PYG{p}{,} \PYG{l+m+mi}{0}\PYG{p}{,} \PYG{l+m+mi}{0}\PYG{p}{)}

\PYG{n}{yymmdd} \PYG{o}{=} \PYG{l+s+s1}{\PYGZsq{}}\PYG{l+s+s1}{2021\PYGZhy{}01\PYGZhy{}03}\PYG{l+s+s1}{\PYGZsq{}}
\PYG{n}{d2} \PYG{o}{=} \PYG{n}{datetime}\PYG{o}{.}\PYG{n}{datetime}\PYG{o}{.}\PYG{n}{strptime}\PYG{p}{(}\PYG{n}{yymmdd}\PYG{p}{,} \PYG{l+s+s1}{\PYGZsq{}}\PYG{l+s+s1}{\PYGZpc{}}\PYG{l+s+s1}{Y\PYGZhy{}}\PYG{l+s+s1}{\PYGZpc{}}\PYG{l+s+s1}{m\PYGZhy{}}\PYG{l+s+si}{\PYGZpc{}d}\PYG{l+s+s1}{\PYGZsq{}}\PYG{p}{)}

\PYG{n}{time\PYGZus{}point} \PYG{o}{=} \PYG{l+s+s1}{\PYGZsq{}}\PYG{l+s+s1}{2021/01/03 19:15:32}\PYG{l+s+s1}{\PYGZsq{}}
\PYG{n}{d3} \PYG{o}{=}  \PYG{n}{datetime}\PYG{o}{.}\PYG{n}{datetime}\PYG{o}{.}\PYG{n}{strptime}\PYG{p}{(}\PYG{n}{time\PYGZus{}point}\PYG{p}{,} \PYG{l+s+s1}{\PYGZsq{}}\PYG{l+s+s1}{\PYGZpc{}}\PYG{l+s+s1}{Y/}\PYG{l+s+s1}{\PYGZpc{}}\PYG{l+s+s1}{m/}\PYG{l+s+si}{\PYGZpc{}d}\PYG{l+s+s1}{ }\PYG{l+s+s1}{\PYGZpc{}}\PYG{l+s+s1}{H:}\PYG{l+s+s1}{\PYGZpc{}}\PYG{l+s+s1}{M:}\PYG{l+s+s1}{\PYGZpc{}}\PYG{l+s+s1}{S}\PYG{l+s+s1}{\PYGZsq{}}\PYG{p}{)}

\PYG{n}{d4} \PYG{o}{=} \PYG{n}{datetime}\PYG{o}{.}\PYG{n}{datetime}\PYG{o}{.}\PYG{n}{strftime}\PYG{p}{(}\PYG{n}{d3}\PYG{p}{,} \PYG{l+s+s1}{\PYGZsq{}}\PYG{l+s+s1}{\PYGZpc{}}\PYG{l+s+s1}{Y\PYGZhy{}}\PYG{l+s+s1}{\PYGZpc{}}\PYG{l+s+s1}{m\PYGZhy{}}\PYG{l+s+si}{\PYGZpc{}d}\PYG{l+s+s1}{\PYGZsq{}}\PYG{p}{)}

\PYG{n+nb}{print}\PYG{p}{(}\PYG{n}{d1}\PYG{p}{)}
\PYG{n+nb}{print}\PYG{p}{(}\PYG{n}{d2}\PYG{p}{)}
\PYG{n+nb}{print}\PYG{p}{(}\PYG{n}{d3}\PYG{p}{,} \PYG{n+nb}{type}\PYG{p}{(}\PYG{n}{d3}\PYG{p}{)}\PYG{p}{)}
\PYG{n+nb}{print}\PYG{p}{(}\PYG{n}{d4}\PYG{p}{,} \PYG{n+nb}{type}\PYG{p}{(}\PYG{n}{d4}\PYG{p}{)}\PYG{p}{)}    

\PYG{n+nb}{print}\PYG{p}{(}\PYG{n}{d1} \PYG{o}{+} \PYG{n}{datetime}\PYG{o}{.}\PYG{n}{timedelta}\PYG{p}{(}\PYG{n}{hours}\PYG{o}{=}\PYG{l+m+mi}{5}\PYG{p}{)}\PYG{p}{)}
\PYG{n+nb}{print}\PYG{p}{(}\PYG{n}{d1} \PYG{o}{+} \PYG{n}{datetime}\PYG{o}{.}\PYG{n}{timedelta}\PYG{p}{(}\PYG{n}{days}\PYG{o}{=}\PYG{l+m+mi}{2}\PYG{p}{)}\PYG{p}{)}
\end{sphinxVerbatim}

\end{sphinxuseclass}\end{sphinxVerbatimInput}
\begin{sphinxVerbatimOutput}

\begin{sphinxuseclass}{cell_output}
\begin{sphinxVerbatim}[commandchars=\\\{\}]
2021\PYGZhy{}01\PYGZhy{}03 00:00:00
2021\PYGZhy{}01\PYGZhy{}03 00:00:00
2021\PYGZhy{}01\PYGZhy{}03 19:15:32 \PYGZlt{}class \PYGZsq{}datetime.datetime\PYGZsq{}\PYGZgt{}
2021\PYGZhy{}01\PYGZhy{}03 \PYGZlt{}class \PYGZsq{}str\PYGZsq{}\PYGZgt{}
2021\PYGZhy{}01\PYGZhy{}03 05:00:00
2021\PYGZhy{}01\PYGZhy{}05 00:00:00
\end{sphinxVerbatim}

\end{sphinxuseclass}\end{sphinxVerbatimOutput}

\end{sphinxuseclass}

\part{Dictionary}
\label{\detokenize{chapter2/2.1.1_Python_Basics:dictionary}}
\sphinxAtStartPar
디셔너리는 key 와 value 가 있는 짝으로 있는 형태입니다. key 를 이용하여 원하는 값을 찾을 수 있어서, 프로그램에서 참조 값을 저장해 둘 때 아주 유용합니다. value 대신에 List 나 DataFrame 형식의 데이터도 넣을 수 도 있습니다. 아래 예시에서는 dic1 에서 키값 ‘a’ 에 해당하는 값은 11 입니다. 새로운 key 와 value 를 넣어서 기존의 Dictionary 에 추가할 수 있습니다. 아래 두 번째 예시는 ‘d’라는 key 에 value 14 를 추가하는 방법입니다. Dictionary 가 너무 커서 어떤 key 값들이 있는 지 알고 싶을 때는 key() 메소드를 사용합니다.

\begin{sphinxuseclass}{cell}\begin{sphinxVerbatimInput}

\begin{sphinxuseclass}{cell_input}
\begin{sphinxVerbatim}[commandchars=\\\{\}]
\PYG{c+c1}{\PYGZsh{} Dictionary}
\PYG{n}{dic1} \PYG{o}{=} \PYG{p}{\PYGZob{}}\PYG{l+s+s1}{\PYGZsq{}}\PYG{l+s+s1}{a}\PYG{l+s+s1}{\PYGZsq{}}\PYG{p}{:}\PYG{l+m+mi}{11}\PYG{p}{,} \PYG{l+s+s1}{\PYGZsq{}}\PYG{l+s+s1}{b}\PYG{l+s+s1}{\PYGZsq{}}\PYG{p}{:}\PYG{l+m+mi}{12}\PYG{p}{,} \PYG{l+s+s1}{\PYGZsq{}}\PYG{l+s+s1}{c}\PYG{l+s+s1}{\PYGZsq{}}\PYG{p}{:}\PYG{l+m+mi}{13}\PYG{p}{\PYGZcb{}}
\PYG{n+nb}{print}\PYG{p}{(}\PYG{n}{dic1}\PYG{p}{[}\PYG{l+s+s1}{\PYGZsq{}}\PYG{l+s+s1}{a}\PYG{l+s+s1}{\PYGZsq{}}\PYG{p}{]}\PYG{p}{)}

\PYG{n}{dic1}\PYG{p}{[}\PYG{l+s+s1}{\PYGZsq{}}\PYG{l+s+s1}{d}\PYG{l+s+s1}{\PYGZsq{}}\PYG{p}{]} \PYG{o}{=} \PYG{l+m+mi}{14}
\PYG{n+nb}{print}\PYG{p}{(}\PYG{n}{dic1}\PYG{p}{)}

\PYG{n+nb}{print}\PYG{p}{(}\PYG{n}{dic1}\PYG{o}{.}\PYG{n}{keys}\PYG{p}{(}\PYG{p}{)}\PYG{p}{)}
\end{sphinxVerbatim}

\end{sphinxuseclass}\end{sphinxVerbatimInput}
\begin{sphinxVerbatimOutput}

\begin{sphinxuseclass}{cell_output}
\begin{sphinxVerbatim}[commandchars=\\\{\}]
11
\PYGZob{}\PYGZsq{}a\PYGZsq{}: 11, \PYGZsq{}b\PYGZsq{}: 12, \PYGZsq{}c\PYGZsq{}: 13, \PYGZsq{}d\PYGZsq{}: 14\PYGZcb{}
dict\PYGZus{}keys([\PYGZsq{}a\PYGZsq{}, \PYGZsq{}b\PYGZsq{}, \PYGZsq{}c\PYGZsq{}, \PYGZsq{}d\PYGZsq{}])
\end{sphinxVerbatim}

\end{sphinxuseclass}\end{sphinxVerbatimOutput}

\end{sphinxuseclass}

\part{Series}
\label{\detokenize{chapter2/2.1.1_Python_Basics:series}}
\sphinxAtStartPar
Series 라는 데이터 타입을 이용하기 위해서는 Pandas 패키지를 이용합니다. Pandas 는 테이블 형태의 데이터를 다루는데 정말 강력한 패키지입니다. 먼저 ss1 이라는 리스트를 생성해 보겠습니다. ss1 이라는 리스트에 어떤 메소드를 사용할 수 있는 지 알기 위해서는 dir() 를 이용합니다.  dir() 를 하면 Built\sphinxhyphen{}in 함수 전체를 알 수 있습니다. append 부터 sort 까지 총 11 개의 함수가 나옵니다. 이 11 개 함수가 List 에서 쓸 수 있는 메소드입니다. 이번에는 ss1 를 Series 로 변경한 후, ss2 에 저장하겠습니다. 그리고 dir() 함수로 호출해 보겠습니다. 많은 메소스가 나열됩니다. 예를 들어 List 값의 평균을 알고 싶은데, List 는 평균을 구하는 메소드가 없습니다. 하지만 Series 에는 mean() 으로 평균값을 구할 수 있습니다.

\begin{sphinxuseclass}{cell}\begin{sphinxVerbatimInput}

\begin{sphinxuseclass}{cell_input}
\begin{sphinxVerbatim}[commandchars=\\\{\}]
\PYG{k+kn}{import} \PYG{n+nn}{pandas} \PYG{k}{as} \PYG{n+nn}{pd}

\PYG{n}{ss1} \PYG{o}{=} \PYG{p}{[}\PYG{l+m+mi}{11}\PYG{p}{,}\PYG{l+m+mi}{12}\PYG{p}{,}\PYG{l+m+mi}{13}\PYG{p}{,}\PYG{l+m+mi}{14}\PYG{p}{,}\PYG{l+m+mi}{15}\PYG{p}{]}
\PYG{n}{ss2} \PYG{o}{=} \PYG{n}{pd}\PYG{o}{.}\PYG{n}{Series}\PYG{p}{(}\PYG{n}{ss1}\PYG{p}{)}

\PYG{n+nb}{print}\PYG{p}{(}\PYG{n+nb}{dir}\PYG{p}{(}\PYG{n}{ss1}\PYG{p}{)}\PYG{p}{)} \PYG{c+c1}{\PYGZsh{} \PYGZsq{}append\PYGZsq{}, \PYGZsq{}clear\PYGZsq{}, \PYGZsq{}copy\PYGZsq{}, \PYGZsq{}count\PYGZsq{}, \PYGZsq{}extend\PYGZsq{}, \PYGZsq{}index\PYGZsq{}, \PYGZsq{}insert\PYGZsq{}, \PYGZsq{}pop\PYGZsq{}, \PYGZsq{}remove\PYGZsq{}, \PYGZsq{}reverse\PYGZsq{}, \PYGZsq{}sort\PYGZsq{} 등 사용가능}
\PYG{n+nb}{print}\PYG{p}{(}\PYG{l+s+s1}{\PYGZsq{}}\PYG{l+s+se}{\PYGZbs{}n}\PYG{l+s+s1}{\PYGZsq{}}\PYG{p}{)}

\PYG{n+nb}{print}\PYG{p}{(}\PYG{n+nb}{dir}\PYG{p}{(}\PYG{n}{ss2}\PYG{p}{)}\PYG{p}{)}
\PYG{n+nb}{print}\PYG{p}{(}\PYG{l+s+s1}{\PYGZsq{}}\PYG{l+s+se}{\PYGZbs{}n}\PYG{l+s+s1}{\PYGZsq{}}\PYG{p}{)}

\PYG{l+s+sd}{\PYGZsq{}\PYGZsq{}\PYGZsq{} ss1.mean() \PYGZhy{}\PYGZhy{}\PYGZgt{} 에러발생 \PYGZsq{}\PYGZsq{}\PYGZsq{}}
\PYG{n+nb}{print}\PYG{p}{(}\PYG{n}{ss2}\PYG{o}{.}\PYG{n}{mean}\PYG{p}{(}\PYG{p}{)}\PYG{p}{)} \PYG{c+c1}{\PYGZsh{} 평균값 13 반환}
\end{sphinxVerbatim}

\end{sphinxuseclass}\end{sphinxVerbatimInput}
\begin{sphinxVerbatimOutput}

\begin{sphinxuseclass}{cell_output}
\begin{sphinxVerbatim}[commandchars=\\\{\}]
[\PYGZsq{}\PYGZus{}\PYGZus{}add\PYGZus{}\PYGZus{}\PYGZsq{}, \PYGZsq{}\PYGZus{}\PYGZus{}class\PYGZus{}\PYGZus{}\PYGZsq{}, \PYGZsq{}\PYGZus{}\PYGZus{}class\PYGZus{}getitem\PYGZus{}\PYGZus{}\PYGZsq{}, \PYGZsq{}\PYGZus{}\PYGZus{}contains\PYGZus{}\PYGZus{}\PYGZsq{}, \PYGZsq{}\PYGZus{}\PYGZus{}delattr\PYGZus{}\PYGZus{}\PYGZsq{}, \PYGZsq{}\PYGZus{}\PYGZus{}delitem\PYGZus{}\PYGZus{}\PYGZsq{}, \PYGZsq{}\PYGZus{}\PYGZus{}dir\PYGZus{}\PYGZus{}\PYGZsq{}, \PYGZsq{}\PYGZus{}\PYGZus{}doc\PYGZus{}\PYGZus{}\PYGZsq{}, \PYGZsq{}\PYGZus{}\PYGZus{}eq\PYGZus{}\PYGZus{}\PYGZsq{}, \PYGZsq{}\PYGZus{}\PYGZus{}format\PYGZus{}\PYGZus{}\PYGZsq{}, \PYGZsq{}\PYGZus{}\PYGZus{}ge\PYGZus{}\PYGZus{}\PYGZsq{}, \PYGZsq{}\PYGZus{}\PYGZus{}getattribute\PYGZus{}\PYGZus{}\PYGZsq{}, \PYGZsq{}\PYGZus{}\PYGZus{}getitem\PYGZus{}\PYGZus{}\PYGZsq{}, \PYGZsq{}\PYGZus{}\PYGZus{}gt\PYGZus{}\PYGZus{}\PYGZsq{}, \PYGZsq{}\PYGZus{}\PYGZus{}hash\PYGZus{}\PYGZus{}\PYGZsq{}, \PYGZsq{}\PYGZus{}\PYGZus{}iadd\PYGZus{}\PYGZus{}\PYGZsq{}, \PYGZsq{}\PYGZus{}\PYGZus{}imul\PYGZus{}\PYGZus{}\PYGZsq{}, \PYGZsq{}\PYGZus{}\PYGZus{}init\PYGZus{}\PYGZus{}\PYGZsq{}, \PYGZsq{}\PYGZus{}\PYGZus{}init\PYGZus{}subclass\PYGZus{}\PYGZus{}\PYGZsq{}, \PYGZsq{}\PYGZus{}\PYGZus{}iter\PYGZus{}\PYGZus{}\PYGZsq{}, \PYGZsq{}\PYGZus{}\PYGZus{}le\PYGZus{}\PYGZus{}\PYGZsq{}, \PYGZsq{}\PYGZus{}\PYGZus{}len\PYGZus{}\PYGZus{}\PYGZsq{}, \PYGZsq{}\PYGZus{}\PYGZus{}lt\PYGZus{}\PYGZus{}\PYGZsq{}, \PYGZsq{}\PYGZus{}\PYGZus{}mul\PYGZus{}\PYGZus{}\PYGZsq{}, \PYGZsq{}\PYGZus{}\PYGZus{}ne\PYGZus{}\PYGZus{}\PYGZsq{}, \PYGZsq{}\PYGZus{}\PYGZus{}new\PYGZus{}\PYGZus{}\PYGZsq{}, \PYGZsq{}\PYGZus{}\PYGZus{}reduce\PYGZus{}\PYGZus{}\PYGZsq{}, \PYGZsq{}\PYGZus{}\PYGZus{}reduce\PYGZus{}ex\PYGZus{}\PYGZus{}\PYGZsq{}, \PYGZsq{}\PYGZus{}\PYGZus{}repr\PYGZus{}\PYGZus{}\PYGZsq{}, \PYGZsq{}\PYGZus{}\PYGZus{}reversed\PYGZus{}\PYGZus{}\PYGZsq{}, \PYGZsq{}\PYGZus{}\PYGZus{}rmul\PYGZus{}\PYGZus{}\PYGZsq{}, \PYGZsq{}\PYGZus{}\PYGZus{}setattr\PYGZus{}\PYGZus{}\PYGZsq{}, \PYGZsq{}\PYGZus{}\PYGZus{}setitem\PYGZus{}\PYGZus{}\PYGZsq{}, \PYGZsq{}\PYGZus{}\PYGZus{}sizeof\PYGZus{}\PYGZus{}\PYGZsq{}, \PYGZsq{}\PYGZus{}\PYGZus{}str\PYGZus{}\PYGZus{}\PYGZsq{}, \PYGZsq{}\PYGZus{}\PYGZus{}subclasshook\PYGZus{}\PYGZus{}\PYGZsq{}, \PYGZsq{}append\PYGZsq{}, \PYGZsq{}clear\PYGZsq{}, \PYGZsq{}copy\PYGZsq{}, \PYGZsq{}count\PYGZsq{}, \PYGZsq{}extend\PYGZsq{}, \PYGZsq{}index\PYGZsq{}, \PYGZsq{}insert\PYGZsq{}, \PYGZsq{}pop\PYGZsq{}, \PYGZsq{}remove\PYGZsq{}, \PYGZsq{}reverse\PYGZsq{}, \PYGZsq{}sort\PYGZsq{}]


[\PYGZsq{}T\PYGZsq{}, \PYGZsq{}\PYGZus{}AXIS\PYGZus{}LEN\PYGZsq{}, \PYGZsq{}\PYGZus{}AXIS\PYGZus{}ORDERS\PYGZsq{}, \PYGZsq{}\PYGZus{}AXIS\PYGZus{}REVERSED\PYGZsq{}, \PYGZsq{}\PYGZus{}AXIS\PYGZus{}TO\PYGZus{}AXIS\PYGZus{}NUMBER\PYGZsq{}, \PYGZsq{}\PYGZus{}HANDLED\PYGZus{}TYPES\PYGZsq{}, \PYGZsq{}\PYGZus{}\PYGZus{}abs\PYGZus{}\PYGZus{}\PYGZsq{}, \PYGZsq{}\PYGZus{}\PYGZus{}add\PYGZus{}\PYGZus{}\PYGZsq{}, \PYGZsq{}\PYGZus{}\PYGZus{}and\PYGZus{}\PYGZus{}\PYGZsq{}, \PYGZsq{}\PYGZus{}\PYGZus{}annotations\PYGZus{}\PYGZus{}\PYGZsq{}, \PYGZsq{}\PYGZus{}\PYGZus{}array\PYGZus{}\PYGZus{}\PYGZsq{}, \PYGZsq{}\PYGZus{}\PYGZus{}array\PYGZus{}priority\PYGZus{}\PYGZus{}\PYGZsq{}, \PYGZsq{}\PYGZus{}\PYGZus{}array\PYGZus{}ufunc\PYGZus{}\PYGZus{}\PYGZsq{}, \PYGZsq{}\PYGZus{}\PYGZus{}array\PYGZus{}wrap\PYGZus{}\PYGZus{}\PYGZsq{}, \PYGZsq{}\PYGZus{}\PYGZus{}bool\PYGZus{}\PYGZus{}\PYGZsq{}, \PYGZsq{}\PYGZus{}\PYGZus{}class\PYGZus{}\PYGZus{}\PYGZsq{}, \PYGZsq{}\PYGZus{}\PYGZus{}contains\PYGZus{}\PYGZus{}\PYGZsq{}, \PYGZsq{}\PYGZus{}\PYGZus{}copy\PYGZus{}\PYGZus{}\PYGZsq{}, \PYGZsq{}\PYGZus{}\PYGZus{}deepcopy\PYGZus{}\PYGZus{}\PYGZsq{}, \PYGZsq{}\PYGZus{}\PYGZus{}delattr\PYGZus{}\PYGZus{}\PYGZsq{}, \PYGZsq{}\PYGZus{}\PYGZus{}delitem\PYGZus{}\PYGZus{}\PYGZsq{}, \PYGZsq{}\PYGZus{}\PYGZus{}dict\PYGZus{}\PYGZus{}\PYGZsq{}, \PYGZsq{}\PYGZus{}\PYGZus{}dir\PYGZus{}\PYGZus{}\PYGZsq{}, \PYGZsq{}\PYGZus{}\PYGZus{}divmod\PYGZus{}\PYGZus{}\PYGZsq{}, \PYGZsq{}\PYGZus{}\PYGZus{}doc\PYGZus{}\PYGZus{}\PYGZsq{}, \PYGZsq{}\PYGZus{}\PYGZus{}eq\PYGZus{}\PYGZus{}\PYGZsq{}, \PYGZsq{}\PYGZus{}\PYGZus{}finalize\PYGZus{}\PYGZus{}\PYGZsq{}, \PYGZsq{}\PYGZus{}\PYGZus{}float\PYGZus{}\PYGZus{}\PYGZsq{}, \PYGZsq{}\PYGZus{}\PYGZus{}floordiv\PYGZus{}\PYGZus{}\PYGZsq{}, \PYGZsq{}\PYGZus{}\PYGZus{}format\PYGZus{}\PYGZus{}\PYGZsq{}, \PYGZsq{}\PYGZus{}\PYGZus{}ge\PYGZus{}\PYGZus{}\PYGZsq{}, \PYGZsq{}\PYGZus{}\PYGZus{}getattr\PYGZus{}\PYGZus{}\PYGZsq{}, \PYGZsq{}\PYGZus{}\PYGZus{}getattribute\PYGZus{}\PYGZus{}\PYGZsq{}, \PYGZsq{}\PYGZus{}\PYGZus{}getitem\PYGZus{}\PYGZus{}\PYGZsq{}, \PYGZsq{}\PYGZus{}\PYGZus{}getstate\PYGZus{}\PYGZus{}\PYGZsq{}, \PYGZsq{}\PYGZus{}\PYGZus{}gt\PYGZus{}\PYGZus{}\PYGZsq{}, \PYGZsq{}\PYGZus{}\PYGZus{}hash\PYGZus{}\PYGZus{}\PYGZsq{}, \PYGZsq{}\PYGZus{}\PYGZus{}iadd\PYGZus{}\PYGZus{}\PYGZsq{}, \PYGZsq{}\PYGZus{}\PYGZus{}iand\PYGZus{}\PYGZus{}\PYGZsq{}, \PYGZsq{}\PYGZus{}\PYGZus{}ifloordiv\PYGZus{}\PYGZus{}\PYGZsq{}, \PYGZsq{}\PYGZus{}\PYGZus{}imod\PYGZus{}\PYGZus{}\PYGZsq{}, \PYGZsq{}\PYGZus{}\PYGZus{}imul\PYGZus{}\PYGZus{}\PYGZsq{}, \PYGZsq{}\PYGZus{}\PYGZus{}init\PYGZus{}\PYGZus{}\PYGZsq{}, \PYGZsq{}\PYGZus{}\PYGZus{}init\PYGZus{}subclass\PYGZus{}\PYGZus{}\PYGZsq{}, \PYGZsq{}\PYGZus{}\PYGZus{}int\PYGZus{}\PYGZus{}\PYGZsq{}, \PYGZsq{}\PYGZus{}\PYGZus{}invert\PYGZus{}\PYGZus{}\PYGZsq{}, \PYGZsq{}\PYGZus{}\PYGZus{}ior\PYGZus{}\PYGZus{}\PYGZsq{}, \PYGZsq{}\PYGZus{}\PYGZus{}ipow\PYGZus{}\PYGZus{}\PYGZsq{}, \PYGZsq{}\PYGZus{}\PYGZus{}isub\PYGZus{}\PYGZus{}\PYGZsq{}, \PYGZsq{}\PYGZus{}\PYGZus{}iter\PYGZus{}\PYGZus{}\PYGZsq{}, \PYGZsq{}\PYGZus{}\PYGZus{}itruediv\PYGZus{}\PYGZus{}\PYGZsq{}, \PYGZsq{}\PYGZus{}\PYGZus{}ixor\PYGZus{}\PYGZus{}\PYGZsq{}, \PYGZsq{}\PYGZus{}\PYGZus{}le\PYGZus{}\PYGZus{}\PYGZsq{}, \PYGZsq{}\PYGZus{}\PYGZus{}len\PYGZus{}\PYGZus{}\PYGZsq{}, \PYGZsq{}\PYGZus{}\PYGZus{}long\PYGZus{}\PYGZus{}\PYGZsq{}, \PYGZsq{}\PYGZus{}\PYGZus{}lt\PYGZus{}\PYGZus{}\PYGZsq{}, \PYGZsq{}\PYGZus{}\PYGZus{}matmul\PYGZus{}\PYGZus{}\PYGZsq{}, \PYGZsq{}\PYGZus{}\PYGZus{}mod\PYGZus{}\PYGZus{}\PYGZsq{}, \PYGZsq{}\PYGZus{}\PYGZus{}module\PYGZus{}\PYGZus{}\PYGZsq{}, \PYGZsq{}\PYGZus{}\PYGZus{}mul\PYGZus{}\PYGZus{}\PYGZsq{}, \PYGZsq{}\PYGZus{}\PYGZus{}ne\PYGZus{}\PYGZus{}\PYGZsq{}, \PYGZsq{}\PYGZus{}\PYGZus{}neg\PYGZus{}\PYGZus{}\PYGZsq{}, \PYGZsq{}\PYGZus{}\PYGZus{}new\PYGZus{}\PYGZus{}\PYGZsq{}, \PYGZsq{}\PYGZus{}\PYGZus{}nonzero\PYGZus{}\PYGZus{}\PYGZsq{}, \PYGZsq{}\PYGZus{}\PYGZus{}or\PYGZus{}\PYGZus{}\PYGZsq{}, \PYGZsq{}\PYGZus{}\PYGZus{}pos\PYGZus{}\PYGZus{}\PYGZsq{}, \PYGZsq{}\PYGZus{}\PYGZus{}pow\PYGZus{}\PYGZus{}\PYGZsq{}, \PYGZsq{}\PYGZus{}\PYGZus{}radd\PYGZus{}\PYGZus{}\PYGZsq{}, \PYGZsq{}\PYGZus{}\PYGZus{}rand\PYGZus{}\PYGZus{}\PYGZsq{}, \PYGZsq{}\PYGZus{}\PYGZus{}rdivmod\PYGZus{}\PYGZus{}\PYGZsq{}, \PYGZsq{}\PYGZus{}\PYGZus{}reduce\PYGZus{}\PYGZus{}\PYGZsq{}, \PYGZsq{}\PYGZus{}\PYGZus{}reduce\PYGZus{}ex\PYGZus{}\PYGZus{}\PYGZsq{}, \PYGZsq{}\PYGZus{}\PYGZus{}repr\PYGZus{}\PYGZus{}\PYGZsq{}, \PYGZsq{}\PYGZus{}\PYGZus{}rfloordiv\PYGZus{}\PYGZus{}\PYGZsq{}, \PYGZsq{}\PYGZus{}\PYGZus{}rmatmul\PYGZus{}\PYGZus{}\PYGZsq{}, \PYGZsq{}\PYGZus{}\PYGZus{}rmod\PYGZus{}\PYGZus{}\PYGZsq{}, \PYGZsq{}\PYGZus{}\PYGZus{}rmul\PYGZus{}\PYGZus{}\PYGZsq{}, \PYGZsq{}\PYGZus{}\PYGZus{}ror\PYGZus{}\PYGZus{}\PYGZsq{}, \PYGZsq{}\PYGZus{}\PYGZus{}round\PYGZus{}\PYGZus{}\PYGZsq{}, \PYGZsq{}\PYGZus{}\PYGZus{}rpow\PYGZus{}\PYGZus{}\PYGZsq{}, \PYGZsq{}\PYGZus{}\PYGZus{}rsub\PYGZus{}\PYGZus{}\PYGZsq{}, \PYGZsq{}\PYGZus{}\PYGZus{}rtruediv\PYGZus{}\PYGZus{}\PYGZsq{}, \PYGZsq{}\PYGZus{}\PYGZus{}rxor\PYGZus{}\PYGZus{}\PYGZsq{}, \PYGZsq{}\PYGZus{}\PYGZus{}setattr\PYGZus{}\PYGZus{}\PYGZsq{}, \PYGZsq{}\PYGZus{}\PYGZus{}setitem\PYGZus{}\PYGZus{}\PYGZsq{}, \PYGZsq{}\PYGZus{}\PYGZus{}setstate\PYGZus{}\PYGZus{}\PYGZsq{}, \PYGZsq{}\PYGZus{}\PYGZus{}sizeof\PYGZus{}\PYGZus{}\PYGZsq{}, \PYGZsq{}\PYGZus{}\PYGZus{}str\PYGZus{}\PYGZus{}\PYGZsq{}, \PYGZsq{}\PYGZus{}\PYGZus{}sub\PYGZus{}\PYGZus{}\PYGZsq{}, \PYGZsq{}\PYGZus{}\PYGZus{}subclasshook\PYGZus{}\PYGZus{}\PYGZsq{}, \PYGZsq{}\PYGZus{}\PYGZus{}truediv\PYGZus{}\PYGZus{}\PYGZsq{}, \PYGZsq{}\PYGZus{}\PYGZus{}weakref\PYGZus{}\PYGZus{}\PYGZsq{}, \PYGZsq{}\PYGZus{}\PYGZus{}xor\PYGZus{}\PYGZus{}\PYGZsq{}, \PYGZsq{}\PYGZus{}accessors\PYGZsq{}, \PYGZsq{}\PYGZus{}accum\PYGZus{}func\PYGZsq{}, \PYGZsq{}\PYGZus{}add\PYGZus{}numeric\PYGZus{}operations\PYGZsq{}, \PYGZsq{}\PYGZus{}agg\PYGZus{}by\PYGZus{}level\PYGZsq{}, \PYGZsq{}\PYGZus{}agg\PYGZus{}examples\PYGZus{}doc\PYGZsq{}, \PYGZsq{}\PYGZus{}agg\PYGZus{}see\PYGZus{}also\PYGZus{}doc\PYGZsq{}, \PYGZsq{}\PYGZus{}align\PYGZus{}frame\PYGZsq{}, \PYGZsq{}\PYGZus{}align\PYGZus{}series\PYGZsq{}, \PYGZsq{}\PYGZus{}arith\PYGZus{}method\PYGZsq{}, \PYGZsq{}\PYGZus{}as\PYGZus{}manager\PYGZsq{}, \PYGZsq{}\PYGZus{}attrs\PYGZsq{}, \PYGZsq{}\PYGZus{}binop\PYGZsq{}, \PYGZsq{}\PYGZus{}can\PYGZus{}hold\PYGZus{}na\PYGZsq{}, \PYGZsq{}\PYGZus{}check\PYGZus{}inplace\PYGZus{}and\PYGZus{}allows\PYGZus{}duplicate\PYGZus{}labels\PYGZsq{}, \PYGZsq{}\PYGZus{}check\PYGZus{}inplace\PYGZus{}setting\PYGZsq{}, \PYGZsq{}\PYGZus{}check\PYGZus{}is\PYGZus{}chained\PYGZus{}assignment\PYGZus{}possible\PYGZsq{}, \PYGZsq{}\PYGZus{}check\PYGZus{}label\PYGZus{}or\PYGZus{}level\PYGZus{}ambiguity\PYGZsq{}, \PYGZsq{}\PYGZus{}check\PYGZus{}setitem\PYGZus{}copy\PYGZsq{}, \PYGZsq{}\PYGZus{}clear\PYGZus{}item\PYGZus{}cache\PYGZsq{}, \PYGZsq{}\PYGZus{}clip\PYGZus{}with\PYGZus{}one\PYGZus{}bound\PYGZsq{}, \PYGZsq{}\PYGZus{}clip\PYGZus{}with\PYGZus{}scalar\PYGZsq{}, \PYGZsq{}\PYGZus{}cmp\PYGZus{}method\PYGZsq{}, \PYGZsq{}\PYGZus{}consolidate\PYGZsq{}, \PYGZsq{}\PYGZus{}consolidate\PYGZus{}inplace\PYGZsq{}, \PYGZsq{}\PYGZus{}construct\PYGZus{}axes\PYGZus{}dict\PYGZsq{}, \PYGZsq{}\PYGZus{}construct\PYGZus{}axes\PYGZus{}from\PYGZus{}arguments\PYGZsq{}, \PYGZsq{}\PYGZus{}construct\PYGZus{}result\PYGZsq{}, \PYGZsq{}\PYGZus{}constructor\PYGZsq{}, \PYGZsq{}\PYGZus{}constructor\PYGZus{}expanddim\PYGZsq{}, \PYGZsq{}\PYGZus{}convert\PYGZsq{}, \PYGZsq{}\PYGZus{}convert\PYGZus{}dtypes\PYGZsq{}, \PYGZsq{}\PYGZus{}data\PYGZsq{}, \PYGZsq{}\PYGZus{}dir\PYGZus{}additions\PYGZsq{}, \PYGZsq{}\PYGZus{}dir\PYGZus{}deletions\PYGZsq{}, \PYGZsq{}\PYGZus{}drop\PYGZus{}axis\PYGZsq{}, \PYGZsq{}\PYGZus{}drop\PYGZus{}labels\PYGZus{}or\PYGZus{}levels\PYGZsq{}, \PYGZsq{}\PYGZus{}duplicated\PYGZsq{}, \PYGZsq{}\PYGZus{}find\PYGZus{}valid\PYGZus{}index\PYGZsq{}, \PYGZsq{}\PYGZus{}flags\PYGZsq{}, \PYGZsq{}\PYGZus{}from\PYGZus{}mgr\PYGZsq{}, \PYGZsq{}\PYGZus{}get\PYGZus{}axis\PYGZsq{}, \PYGZsq{}\PYGZus{}get\PYGZus{}axis\PYGZus{}name\PYGZsq{}, \PYGZsq{}\PYGZus{}get\PYGZus{}axis\PYGZus{}number\PYGZsq{}, \PYGZsq{}\PYGZus{}get\PYGZus{}axis\PYGZus{}resolvers\PYGZsq{}, \PYGZsq{}\PYGZus{}get\PYGZus{}block\PYGZus{}manager\PYGZus{}axis\PYGZsq{}, \PYGZsq{}\PYGZus{}get\PYGZus{}bool\PYGZus{}data\PYGZsq{}, \PYGZsq{}\PYGZus{}get\PYGZus{}cacher\PYGZsq{}, \PYGZsq{}\PYGZus{}get\PYGZus{}cleaned\PYGZus{}column\PYGZus{}resolvers\PYGZsq{}, \PYGZsq{}\PYGZus{}get\PYGZus{}index\PYGZus{}resolvers\PYGZsq{}, \PYGZsq{}\PYGZus{}get\PYGZus{}label\PYGZus{}or\PYGZus{}level\PYGZus{}values\PYGZsq{}, \PYGZsq{}\PYGZus{}get\PYGZus{}numeric\PYGZus{}data\PYGZsq{}, \PYGZsq{}\PYGZus{}get\PYGZus{}value\PYGZsq{}, \PYGZsq{}\PYGZus{}get\PYGZus{}values\PYGZsq{}, \PYGZsq{}\PYGZus{}get\PYGZus{}values\PYGZus{}tuple\PYGZsq{}, \PYGZsq{}\PYGZus{}get\PYGZus{}with\PYGZsq{}, \PYGZsq{}\PYGZus{}gotitem\PYGZsq{}, \PYGZsq{}\PYGZus{}hidden\PYGZus{}attrs\PYGZsq{}, \PYGZsq{}\PYGZus{}index\PYGZsq{}, \PYGZsq{}\PYGZus{}indexed\PYGZus{}same\PYGZsq{}, \PYGZsq{}\PYGZus{}info\PYGZus{}axis\PYGZsq{}, \PYGZsq{}\PYGZus{}info\PYGZus{}axis\PYGZus{}name\PYGZsq{}, \PYGZsq{}\PYGZus{}info\PYGZus{}axis\PYGZus{}number\PYGZsq{}, \PYGZsq{}\PYGZus{}init\PYGZus{}dict\PYGZsq{}, \PYGZsq{}\PYGZus{}init\PYGZus{}mgr\PYGZsq{}, \PYGZsq{}\PYGZus{}inplace\PYGZus{}method\PYGZsq{}, \PYGZsq{}\PYGZus{}internal\PYGZus{}names\PYGZsq{}, \PYGZsq{}\PYGZus{}internal\PYGZus{}names\PYGZus{}set\PYGZsq{}, \PYGZsq{}\PYGZus{}is\PYGZus{}cached\PYGZsq{}, \PYGZsq{}\PYGZus{}is\PYGZus{}copy\PYGZsq{}, \PYGZsq{}\PYGZus{}is\PYGZus{}label\PYGZus{}or\PYGZus{}level\PYGZus{}reference\PYGZsq{}, \PYGZsq{}\PYGZus{}is\PYGZus{}label\PYGZus{}reference\PYGZsq{}, \PYGZsq{}\PYGZus{}is\PYGZus{}level\PYGZus{}reference\PYGZsq{}, \PYGZsq{}\PYGZus{}is\PYGZus{}mixed\PYGZus{}type\PYGZsq{}, \PYGZsq{}\PYGZus{}is\PYGZus{}view\PYGZsq{}, \PYGZsq{}\PYGZus{}item\PYGZus{}cache\PYGZsq{}, \PYGZsq{}\PYGZus{}ixs\PYGZsq{}, \PYGZsq{}\PYGZus{}logical\PYGZus{}func\PYGZsq{}, \PYGZsq{}\PYGZus{}logical\PYGZus{}method\PYGZsq{}, \PYGZsq{}\PYGZus{}map\PYGZus{}values\PYGZsq{}, \PYGZsq{}\PYGZus{}maybe\PYGZus{}update\PYGZus{}cacher\PYGZsq{}, \PYGZsq{}\PYGZus{}memory\PYGZus{}usage\PYGZsq{}, \PYGZsq{}\PYGZus{}metadata\PYGZsq{}, \PYGZsq{}\PYGZus{}mgr\PYGZsq{}, \PYGZsq{}\PYGZus{}min\PYGZus{}count\PYGZus{}stat\PYGZus{}function\PYGZsq{}, \PYGZsq{}\PYGZus{}name\PYGZsq{}, \PYGZsq{}\PYGZus{}needs\PYGZus{}reindex\PYGZus{}multi\PYGZsq{}, \PYGZsq{}\PYGZus{}protect\PYGZus{}consolidate\PYGZsq{}, \PYGZsq{}\PYGZus{}reduce\PYGZsq{}, \PYGZsq{}\PYGZus{}reindex\PYGZus{}axes\PYGZsq{}, \PYGZsq{}\PYGZus{}reindex\PYGZus{}indexer\PYGZsq{}, \PYGZsq{}\PYGZus{}reindex\PYGZus{}multi\PYGZsq{}, \PYGZsq{}\PYGZus{}reindex\PYGZus{}with\PYGZus{}indexers\PYGZsq{}, \PYGZsq{}\PYGZus{}replace\PYGZus{}single\PYGZsq{}, \PYGZsq{}\PYGZus{}repr\PYGZus{}data\PYGZus{}resource\PYGZus{}\PYGZsq{}, \PYGZsq{}\PYGZus{}repr\PYGZus{}latex\PYGZus{}\PYGZsq{}, \PYGZsq{}\PYGZus{}reset\PYGZus{}cache\PYGZsq{}, \PYGZsq{}\PYGZus{}reset\PYGZus{}cacher\PYGZsq{}, \PYGZsq{}\PYGZus{}set\PYGZus{}as\PYGZus{}cached\PYGZsq{}, \PYGZsq{}\PYGZus{}set\PYGZus{}axis\PYGZsq{}, \PYGZsq{}\PYGZus{}set\PYGZus{}axis\PYGZus{}name\PYGZsq{}, \PYGZsq{}\PYGZus{}set\PYGZus{}axis\PYGZus{}nocheck\PYGZsq{}, \PYGZsq{}\PYGZus{}set\PYGZus{}is\PYGZus{}copy\PYGZsq{}, \PYGZsq{}\PYGZus{}set\PYGZus{}labels\PYGZsq{}, \PYGZsq{}\PYGZus{}set\PYGZus{}name\PYGZsq{}, \PYGZsq{}\PYGZus{}set\PYGZus{}value\PYGZsq{}, \PYGZsq{}\PYGZus{}set\PYGZus{}values\PYGZsq{}, \PYGZsq{}\PYGZus{}set\PYGZus{}with\PYGZsq{}, \PYGZsq{}\PYGZus{}set\PYGZus{}with\PYGZus{}engine\PYGZsq{}, \PYGZsq{}\PYGZus{}slice\PYGZsq{}, \PYGZsq{}\PYGZus{}stat\PYGZus{}axis\PYGZsq{}, \PYGZsq{}\PYGZus{}stat\PYGZus{}axis\PYGZus{}name\PYGZsq{}, \PYGZsq{}\PYGZus{}stat\PYGZus{}axis\PYGZus{}number\PYGZsq{}, \PYGZsq{}\PYGZus{}stat\PYGZus{}function\PYGZsq{}, \PYGZsq{}\PYGZus{}stat\PYGZus{}function\PYGZus{}ddof\PYGZsq{}, \PYGZsq{}\PYGZus{}take\PYGZus{}with\PYGZus{}is\PYGZus{}copy\PYGZsq{}, \PYGZsq{}\PYGZus{}typ\PYGZsq{}, \PYGZsq{}\PYGZus{}update\PYGZus{}inplace\PYGZsq{}, \PYGZsq{}\PYGZus{}validate\PYGZus{}dtype\PYGZsq{}, \PYGZsq{}\PYGZus{}values\PYGZsq{}, \PYGZsq{}\PYGZus{}where\PYGZsq{}, \PYGZsq{}abs\PYGZsq{}, \PYGZsq{}add\PYGZsq{}, \PYGZsq{}add\PYGZus{}prefix\PYGZsq{}, \PYGZsq{}add\PYGZus{}suffix\PYGZsq{}, \PYGZsq{}agg\PYGZsq{}, \PYGZsq{}aggregate\PYGZsq{}, \PYGZsq{}align\PYGZsq{}, \PYGZsq{}all\PYGZsq{}, \PYGZsq{}any\PYGZsq{}, \PYGZsq{}append\PYGZsq{}, \PYGZsq{}apply\PYGZsq{}, \PYGZsq{}argmax\PYGZsq{}, \PYGZsq{}argmin\PYGZsq{}, \PYGZsq{}argsort\PYGZsq{}, \PYGZsq{}array\PYGZsq{}, \PYGZsq{}asfreq\PYGZsq{}, \PYGZsq{}asof\PYGZsq{}, \PYGZsq{}astype\PYGZsq{}, \PYGZsq{}at\PYGZsq{}, \PYGZsq{}at\PYGZus{}time\PYGZsq{}, \PYGZsq{}attrs\PYGZsq{}, \PYGZsq{}autocorr\PYGZsq{}, \PYGZsq{}axes\PYGZsq{}, \PYGZsq{}backfill\PYGZsq{}, \PYGZsq{}between\PYGZsq{}, \PYGZsq{}between\PYGZus{}time\PYGZsq{}, \PYGZsq{}bfill\PYGZsq{}, \PYGZsq{}bool\PYGZsq{}, \PYGZsq{}clip\PYGZsq{}, \PYGZsq{}combine\PYGZsq{}, \PYGZsq{}combine\PYGZus{}first\PYGZsq{}, \PYGZsq{}compare\PYGZsq{}, \PYGZsq{}convert\PYGZus{}dtypes\PYGZsq{}, \PYGZsq{}copy\PYGZsq{}, \PYGZsq{}corr\PYGZsq{}, \PYGZsq{}count\PYGZsq{}, \PYGZsq{}cov\PYGZsq{}, \PYGZsq{}cummax\PYGZsq{}, \PYGZsq{}cummin\PYGZsq{}, \PYGZsq{}cumprod\PYGZsq{}, \PYGZsq{}cumsum\PYGZsq{}, \PYGZsq{}describe\PYGZsq{}, \PYGZsq{}diff\PYGZsq{}, \PYGZsq{}div\PYGZsq{}, \PYGZsq{}divide\PYGZsq{}, \PYGZsq{}divmod\PYGZsq{}, \PYGZsq{}dot\PYGZsq{}, \PYGZsq{}drop\PYGZsq{}, \PYGZsq{}drop\PYGZus{}duplicates\PYGZsq{}, \PYGZsq{}droplevel\PYGZsq{}, \PYGZsq{}dropna\PYGZsq{}, \PYGZsq{}dtype\PYGZsq{}, \PYGZsq{}dtypes\PYGZsq{}, \PYGZsq{}duplicated\PYGZsq{}, \PYGZsq{}empty\PYGZsq{}, \PYGZsq{}eq\PYGZsq{}, \PYGZsq{}equals\PYGZsq{}, \PYGZsq{}ewm\PYGZsq{}, \PYGZsq{}expanding\PYGZsq{}, \PYGZsq{}explode\PYGZsq{}, \PYGZsq{}factorize\PYGZsq{}, \PYGZsq{}ffill\PYGZsq{}, \PYGZsq{}fillna\PYGZsq{}, \PYGZsq{}filter\PYGZsq{}, \PYGZsq{}first\PYGZsq{}, \PYGZsq{}first\PYGZus{}valid\PYGZus{}index\PYGZsq{}, \PYGZsq{}flags\PYGZsq{}, \PYGZsq{}floordiv\PYGZsq{}, \PYGZsq{}ge\PYGZsq{}, \PYGZsq{}get\PYGZsq{}, \PYGZsq{}groupby\PYGZsq{}, \PYGZsq{}gt\PYGZsq{}, \PYGZsq{}hasnans\PYGZsq{}, \PYGZsq{}head\PYGZsq{}, \PYGZsq{}hist\PYGZsq{}, \PYGZsq{}iat\PYGZsq{}, \PYGZsq{}idxmax\PYGZsq{}, \PYGZsq{}idxmin\PYGZsq{}, \PYGZsq{}iloc\PYGZsq{}, \PYGZsq{}index\PYGZsq{}, \PYGZsq{}infer\PYGZus{}objects\PYGZsq{}, \PYGZsq{}interpolate\PYGZsq{}, \PYGZsq{}is\PYGZus{}monotonic\PYGZsq{}, \PYGZsq{}is\PYGZus{}monotonic\PYGZus{}decreasing\PYGZsq{}, \PYGZsq{}is\PYGZus{}monotonic\PYGZus{}increasing\PYGZsq{}, \PYGZsq{}is\PYGZus{}unique\PYGZsq{}, \PYGZsq{}isin\PYGZsq{}, \PYGZsq{}isna\PYGZsq{}, \PYGZsq{}isnull\PYGZsq{}, \PYGZsq{}item\PYGZsq{}, \PYGZsq{}items\PYGZsq{}, \PYGZsq{}iteritems\PYGZsq{}, \PYGZsq{}keys\PYGZsq{}, \PYGZsq{}kurt\PYGZsq{}, \PYGZsq{}kurtosis\PYGZsq{}, \PYGZsq{}last\PYGZsq{}, \PYGZsq{}last\PYGZus{}valid\PYGZus{}index\PYGZsq{}, \PYGZsq{}le\PYGZsq{}, \PYGZsq{}loc\PYGZsq{}, \PYGZsq{}lt\PYGZsq{}, \PYGZsq{}mad\PYGZsq{}, \PYGZsq{}map\PYGZsq{}, \PYGZsq{}mask\PYGZsq{}, \PYGZsq{}max\PYGZsq{}, \PYGZsq{}mean\PYGZsq{}, \PYGZsq{}median\PYGZsq{}, \PYGZsq{}memory\PYGZus{}usage\PYGZsq{}, \PYGZsq{}min\PYGZsq{}, \PYGZsq{}mod\PYGZsq{}, \PYGZsq{}mode\PYGZsq{}, \PYGZsq{}mul\PYGZsq{}, \PYGZsq{}multiply\PYGZsq{}, \PYGZsq{}name\PYGZsq{}, \PYGZsq{}nbytes\PYGZsq{}, \PYGZsq{}ndim\PYGZsq{}, \PYGZsq{}ne\PYGZsq{}, \PYGZsq{}nlargest\PYGZsq{}, \PYGZsq{}notna\PYGZsq{}, \PYGZsq{}notnull\PYGZsq{}, \PYGZsq{}nsmallest\PYGZsq{}, \PYGZsq{}nunique\PYGZsq{}, \PYGZsq{}pad\PYGZsq{}, \PYGZsq{}pct\PYGZus{}change\PYGZsq{}, \PYGZsq{}pipe\PYGZsq{}, \PYGZsq{}plot\PYGZsq{}, \PYGZsq{}pop\PYGZsq{}, \PYGZsq{}pow\PYGZsq{}, \PYGZsq{}prod\PYGZsq{}, \PYGZsq{}product\PYGZsq{}, \PYGZsq{}quantile\PYGZsq{}, \PYGZsq{}radd\PYGZsq{}, \PYGZsq{}rank\PYGZsq{}, \PYGZsq{}ravel\PYGZsq{}, \PYGZsq{}rdiv\PYGZsq{}, \PYGZsq{}rdivmod\PYGZsq{}, \PYGZsq{}reindex\PYGZsq{}, \PYGZsq{}reindex\PYGZus{}like\PYGZsq{}, \PYGZsq{}rename\PYGZsq{}, \PYGZsq{}rename\PYGZus{}axis\PYGZsq{}, \PYGZsq{}reorder\PYGZus{}levels\PYGZsq{}, \PYGZsq{}repeat\PYGZsq{}, \PYGZsq{}replace\PYGZsq{}, \PYGZsq{}resample\PYGZsq{}, \PYGZsq{}reset\PYGZus{}index\PYGZsq{}, \PYGZsq{}rfloordiv\PYGZsq{}, \PYGZsq{}rmod\PYGZsq{}, \PYGZsq{}rmul\PYGZsq{}, \PYGZsq{}rolling\PYGZsq{}, \PYGZsq{}round\PYGZsq{}, \PYGZsq{}rpow\PYGZsq{}, \PYGZsq{}rsub\PYGZsq{}, \PYGZsq{}rtruediv\PYGZsq{}, \PYGZsq{}sample\PYGZsq{}, \PYGZsq{}searchsorted\PYGZsq{}, \PYGZsq{}sem\PYGZsq{}, \PYGZsq{}set\PYGZus{}axis\PYGZsq{}, \PYGZsq{}set\PYGZus{}flags\PYGZsq{}, \PYGZsq{}shape\PYGZsq{}, \PYGZsq{}shift\PYGZsq{}, \PYGZsq{}size\PYGZsq{}, \PYGZsq{}skew\PYGZsq{}, \PYGZsq{}slice\PYGZus{}shift\PYGZsq{}, \PYGZsq{}sort\PYGZus{}index\PYGZsq{}, \PYGZsq{}sort\PYGZus{}values\PYGZsq{}, \PYGZsq{}squeeze\PYGZsq{}, \PYGZsq{}std\PYGZsq{}, \PYGZsq{}sub\PYGZsq{}, \PYGZsq{}subtract\PYGZsq{}, \PYGZsq{}sum\PYGZsq{}, \PYGZsq{}swapaxes\PYGZsq{}, \PYGZsq{}swaplevel\PYGZsq{}, \PYGZsq{}tail\PYGZsq{}, \PYGZsq{}take\PYGZsq{}, \PYGZsq{}to\PYGZus{}clipboard\PYGZsq{}, \PYGZsq{}to\PYGZus{}csv\PYGZsq{}, \PYGZsq{}to\PYGZus{}dict\PYGZsq{}, \PYGZsq{}to\PYGZus{}excel\PYGZsq{}, \PYGZsq{}to\PYGZus{}frame\PYGZsq{}, \PYGZsq{}to\PYGZus{}hdf\PYGZsq{}, \PYGZsq{}to\PYGZus{}json\PYGZsq{}, \PYGZsq{}to\PYGZus{}latex\PYGZsq{}, \PYGZsq{}to\PYGZus{}list\PYGZsq{}, \PYGZsq{}to\PYGZus{}markdown\PYGZsq{}, \PYGZsq{}to\PYGZus{}numpy\PYGZsq{}, \PYGZsq{}to\PYGZus{}period\PYGZsq{}, \PYGZsq{}to\PYGZus{}pickle\PYGZsq{}, \PYGZsq{}to\PYGZus{}sql\PYGZsq{}, \PYGZsq{}to\PYGZus{}string\PYGZsq{}, \PYGZsq{}to\PYGZus{}timestamp\PYGZsq{}, \PYGZsq{}to\PYGZus{}xarray\PYGZsq{}, \PYGZsq{}transform\PYGZsq{}, \PYGZsq{}transpose\PYGZsq{}, \PYGZsq{}truediv\PYGZsq{}, \PYGZsq{}truncate\PYGZsq{}, \PYGZsq{}tz\PYGZus{}convert\PYGZsq{}, \PYGZsq{}tz\PYGZus{}localize\PYGZsq{}, \PYGZsq{}unique\PYGZsq{}, \PYGZsq{}unstack\PYGZsq{}, \PYGZsq{}update\PYGZsq{}, \PYGZsq{}value\PYGZus{}counts\PYGZsq{}, \PYGZsq{}values\PYGZsq{}, \PYGZsq{}var\PYGZsq{}, \PYGZsq{}view\PYGZsq{}, \PYGZsq{}where\PYGZsq{}, \PYGZsq{}xs\PYGZsq{}]


13.0
\end{sphinxVerbatim}

\end{sphinxuseclass}\end{sphinxVerbatimOutput}

\end{sphinxuseclass}

\part{DataFrame}
\label{\detokenize{chapter2/2.1.1_Python_Basics:dataframe}}
\sphinxAtStartPar
DataFrame 은 Series의 확장으로, DataFrame 에서 한 개의 Series 는 하나의 Column 이 됩니다. DataFrame 은 여러 개 Column 이 모여 있는 테이블 형태의 데이터 형식입니다. 우선 Dictionary 로 DataFrame 을 만들어 보겠습니다. 두 개의 List \sphinxhyphen{} c1\_list 와 c2\_list 가 두 개의 key \sphinxhyphen{} ‘c1’, ‘c2’ 대응이 되는 Dictionary, dic\_c12 를 생성합니다. 그 다음 pd.DataFrame(dic\_c12) 와 같이 DataFrame 으로 데이터 타입을 변경합니다. 출력해보면 테이블 형태로 변경되었음을 알 수 있습니다. 그리고 이 DataFrame 을 df 라는 변수에 저장합니다. df 에서 한 컬럼만 자르면 다시 Series 로 변경됩니다. Series 보다 더 많은 메소드를 이용할 수 있습니다.

\begin{sphinxuseclass}{cell}\begin{sphinxVerbatimInput}

\begin{sphinxuseclass}{cell_input}
\begin{sphinxVerbatim}[commandchars=\\\{\}]
\PYG{n}{c1\PYGZus{}list} \PYG{o}{=} \PYG{p}{[}\PYG{l+m+mi}{11}\PYG{p}{,}\PYG{l+m+mi}{12}\PYG{p}{,}\PYG{l+m+mi}{13}\PYG{p}{,}\PYG{l+m+mi}{14}\PYG{p}{,}\PYG{l+m+mi}{15}\PYG{p}{]}
\PYG{n}{c2\PYGZus{}list} \PYG{o}{=} \PYG{p}{[}\PYG{l+s+s1}{\PYGZsq{}}\PYG{l+s+s1}{a}\PYG{l+s+s1}{\PYGZsq{}}\PYG{p}{,}\PYG{l+s+s1}{\PYGZsq{}}\PYG{l+s+s1}{b}\PYG{l+s+s1}{\PYGZsq{}}\PYG{p}{,}\PYG{l+s+s1}{\PYGZsq{}}\PYG{l+s+s1}{c}\PYG{l+s+s1}{\PYGZsq{}}\PYG{p}{,}\PYG{l+s+s1}{\PYGZsq{}}\PYG{l+s+s1}{d}\PYG{l+s+s1}{\PYGZsq{}}\PYG{p}{,}\PYG{l+s+s1}{\PYGZsq{}}\PYG{l+s+s1}{e}\PYG{l+s+s1}{\PYGZsq{}}\PYG{p}{]}

\PYG{n}{dic\PYGZus{}c12} \PYG{o}{=} \PYG{p}{\PYGZob{}}\PYG{l+s+s1}{\PYGZsq{}}\PYG{l+s+s1}{c1}\PYG{l+s+s1}{\PYGZsq{}}\PYG{p}{:} \PYG{n}{c1\PYGZus{}list}\PYG{p}{,} \PYG{l+s+s1}{\PYGZsq{}}\PYG{l+s+s1}{c2}\PYG{l+s+s1}{\PYGZsq{}}\PYG{p}{:} \PYG{n}{c2\PYGZus{}list}\PYG{p}{\PYGZcb{}}
\PYG{n}{df} \PYG{o}{=} \PYG{n}{pd}\PYG{o}{.}\PYG{n}{DataFrame}\PYG{p}{(}\PYG{n}{dic\PYGZus{}c12}\PYG{p}{)} \PYG{c+c1}{\PYGZsh{} DataFrame 으로 변경}

\PYG{n+nb}{print}\PYG{p}{(}\PYG{n}{df}\PYG{p}{)}
\PYG{n+nb}{print}\PYG{p}{(}\PYG{l+s+s1}{\PYGZsq{}}\PYG{l+s+se}{\PYGZbs{}n}\PYG{l+s+s1}{\PYGZsq{}}\PYG{p}{)}

\PYG{n+nb}{print}\PYG{p}{(}\PYG{n}{df}\PYG{p}{[}\PYG{l+s+s1}{\PYGZsq{}}\PYG{l+s+s1}{c1}\PYG{l+s+s1}{\PYGZsq{}}\PYG{p}{]}\PYG{p}{,} \PYG{n+nb}{type}\PYG{p}{(}\PYG{n}{df}\PYG{p}{[}\PYG{l+s+s1}{\PYGZsq{}}\PYG{l+s+s1}{c1}\PYG{l+s+s1}{\PYGZsq{}}\PYG{p}{]}\PYG{p}{)}\PYG{p}{)}
\end{sphinxVerbatim}

\end{sphinxuseclass}\end{sphinxVerbatimInput}
\begin{sphinxVerbatimOutput}

\begin{sphinxuseclass}{cell_output}
\begin{sphinxVerbatim}[commandchars=\\\{\}]
   c1 c2
0  11  a
1  12  b
2  13  c
3  14  d
4  15  e


0    11
1    12
2    13
3    14
4    15
Name: c1, dtype: int64 \PYGZlt{}class \PYGZsq{}pandas.core.series.Series\PYGZsq{}\PYGZgt{}
\end{sphinxVerbatim}

\end{sphinxuseclass}\end{sphinxVerbatimOutput}

\end{sphinxuseclass}

\part{Index}
\label{\detokenize{chapter2/2.1.2_Python_Basics:index}}\label{\detokenize{chapter2/2.1.2_Python_Basics::doc}}
\sphinxAtStartPar
데이터 처리에 중요한 역활을 하는 Index 에 대하여 알아보겠습니다. Index 는 우리말로 색인이라고 할 수 있을 것 같은데요. 색인은 무엇을 빨리 찾기 위해 순서대로 정리되어 있는 목록입니다. Index 는 색인처럼 어떤 값을 빨리 찾을 때도 필요하지만, 두 데이터를 어떤 값을 기준으로 결합하는데도 유용하게 쓰입니다. Index 는 Series 와 DataFrame 에 주로 활용됩니다. ss2 는 바로 이전 장에서 만든 Series 입니다. 출력을 해 보면 맨 왼쪽에 0 \textasciitilde{} 4 까지 값이 보이는데요. 이게 Index 입니다. 특별하게 지정하지 않으면 숫자 0 부터서 순서대로 들어가게 됩니다. 다음은 알파벳 Index 를 넣어서 ss3 를 생성하고 출력 해보겠습니다. 맨 왼쪽 index 값이 숫자가 아니라 알파벳으로 바뀌었습니다.

\begin{sphinxuseclass}{cell}\begin{sphinxVerbatimInput}

\begin{sphinxuseclass}{cell_input}
\begin{sphinxVerbatim}[commandchars=\\\{\}]
\PYG{k+kn}{import} \PYG{n+nn}{pandas} \PYG{k}{as} \PYG{n+nn}{pd}

\PYG{n}{ss1} \PYG{o}{=} \PYG{p}{[}\PYG{l+m+mi}{11}\PYG{p}{,}\PYG{l+m+mi}{12}\PYG{p}{,}\PYG{l+m+mi}{13}\PYG{p}{,}\PYG{l+m+mi}{14}\PYG{p}{,}\PYG{l+m+mi}{15}\PYG{p}{]}
\PYG{n}{ss2} \PYG{o}{=} \PYG{n}{pd}\PYG{o}{.}\PYG{n}{Series}\PYG{p}{(}\PYG{n}{ss1}\PYG{p}{)}
\PYG{n+nb}{print}\PYG{p}{(}\PYG{n}{ss2}\PYG{p}{)}

\PYG{n}{ss3} \PYG{o}{=} \PYG{n}{pd}\PYG{o}{.}\PYG{n}{Series}\PYG{p}{(}\PYG{n}{ss1}\PYG{p}{,} \PYG{n}{index}\PYG{o}{=}\PYG{p}{[}\PYG{l+s+s1}{\PYGZsq{}}\PYG{l+s+s1}{a}\PYG{l+s+s1}{\PYGZsq{}}\PYG{p}{,} \PYG{l+s+s1}{\PYGZsq{}}\PYG{l+s+s1}{b}\PYG{l+s+s1}{\PYGZsq{}}\PYG{p}{,} \PYG{l+s+s1}{\PYGZsq{}}\PYG{l+s+s1}{c}\PYG{l+s+s1}{\PYGZsq{}}\PYG{p}{,} \PYG{l+s+s1}{\PYGZsq{}}\PYG{l+s+s1}{d}\PYG{l+s+s1}{\PYGZsq{}}\PYG{p}{,} \PYG{l+s+s1}{\PYGZsq{}}\PYG{l+s+s1}{e}\PYG{l+s+s1}{\PYGZsq{}}\PYG{p}{]}\PYG{p}{)}
\PYG{n+nb}{print}\PYG{p}{(}\PYG{n}{ss3}\PYG{p}{)}
\end{sphinxVerbatim}

\end{sphinxuseclass}\end{sphinxVerbatimInput}
\begin{sphinxVerbatimOutput}

\begin{sphinxuseclass}{cell_output}
\begin{sphinxVerbatim}[commandchars=\\\{\}]
0    11
1    12
2    13
3    14
4    15
dtype: int64
a    11
b    12
c    13
d    14
e    15
dtype: int64
\end{sphinxVerbatim}

\end{sphinxuseclass}\end{sphinxVerbatimOutput}

\end{sphinxuseclass}

\part{Index 활용}
\label{\detokenize{chapter2/2.1.2_Python_Basics:id1}}
\sphinxAtStartPar
Index 의 본연의 기능은 찾기입니다. ss3.loc{[}인덱스값{]} 를 이용하여 원하는 값을 찾을 수 있습니다. 인덱스 ‘c’ 에 해당하는 값은 13입니다. ss3.loc{[}‘c’{]} 를 하면 13이 출력됩니다. 만약, 인덱스 ‘a’ 와 ‘c’ 를 다 찾고 싶으면 {[}‘a’, ‘c’{]} 와 같이 List 로 넣어주면 됩니다. loc 를 하지 않아도 같은 결과를 얻으시겠지만, loc 를 넣으면 ‘a’,’c’ 를 column 이 아니라 index 에서 찾는다는 것을 명확하게 해 줍니다.

\begin{sphinxuseclass}{cell}\begin{sphinxVerbatimInput}

\begin{sphinxuseclass}{cell_input}
\begin{sphinxVerbatim}[commandchars=\\\{\}]
\PYG{n+nb}{print}\PYG{p}{(}\PYG{n}{ss3}\PYG{o}{.}\PYG{n}{loc}\PYG{p}{[}\PYG{l+s+s1}{\PYGZsq{}}\PYG{l+s+s1}{a}\PYG{l+s+s1}{\PYGZsq{}}\PYG{p}{]}\PYG{p}{,} \PYG{n}{ss3}\PYG{p}{[}\PYG{l+s+s1}{\PYGZsq{}}\PYG{l+s+s1}{c}\PYG{l+s+s1}{\PYGZsq{}}\PYG{p}{]}\PYG{p}{)}
\PYG{n+nb}{print}\PYG{p}{(}\PYG{n}{ss3}\PYG{o}{.}\PYG{n}{loc}\PYG{p}{[}\PYG{p}{[}\PYG{l+s+s1}{\PYGZsq{}}\PYG{l+s+s1}{a}\PYG{l+s+s1}{\PYGZsq{}}\PYG{p}{,}\PYG{l+s+s1}{\PYGZsq{}}\PYG{l+s+s1}{c}\PYG{l+s+s1}{\PYGZsq{}}\PYG{p}{]}\PYG{p}{]}\PYG{p}{)}
\end{sphinxVerbatim}

\end{sphinxuseclass}\end{sphinxVerbatimInput}
\begin{sphinxVerbatimOutput}

\begin{sphinxuseclass}{cell_output}
\begin{sphinxVerbatim}[commandchars=\\\{\}]
11 13
a    11
c    13
dtype: int64
\end{sphinxVerbatim}

\end{sphinxuseclass}\end{sphinxVerbatimOutput}

\end{sphinxuseclass}
\sphinxAtStartPar
 DataFrame 에서도 동일하게 활용가능합니다. 먼저 df1 이라는 DataFrame 을 생성하고 출력합니다. Default Index 인 숫자 0 \textasciitilde{} 4 로 되어 있음을 확인할 수 있습니다. 이제 원하는 인덱스 s1 \textasciitilde{} s5 를 할당하고 df2 에 저장합니다. 출력 결과를 보니 df2 의 인덱스가 바뀌었습니다.

\sphinxAtStartPar
이번에는 원하는 값을 찾아보겠습니다. df2 의 index 가 ‘s3’ 인 c1 컬럼값을 알고 싶다면 df2.loc{[}‘s3’{]}{[}‘c1’{]} 이라고 하면 됩니다. 만약, c1 과 c2 둘다 출력하고 싶으면  df2.loc{[}‘s3’{]}{[}{[}‘c1’,’c2’{]}{]} 형태로 리스트로 입력합니다. 실수로 df2.loc{[}‘s3’{]}{[}‘c1’,’c2’{]} 로 입력을 하면 Pandas 패키지는 ‘c1,’c2’ 가 하나의 column 이름이라고 착각하게 되어 에러가 발생합니다.

\begin{sphinxuseclass}{cell}\begin{sphinxVerbatimInput}

\begin{sphinxuseclass}{cell_input}
\begin{sphinxVerbatim}[commandchars=\\\{\}]
\PYG{c+c1}{\PYGZsh{} DataFrame 생성}
\PYG{n}{c1\PYGZus{}list} \PYG{o}{=} \PYG{p}{[}\PYG{l+m+mi}{11}\PYG{p}{,}\PYG{l+m+mi}{12}\PYG{p}{,}\PYG{l+m+mi}{13}\PYG{p}{,}\PYG{l+m+mi}{14}\PYG{p}{,}\PYG{l+m+mi}{15}\PYG{p}{]}
\PYG{n}{c2\PYGZus{}list} \PYG{o}{=} \PYG{p}{[}\PYG{l+s+s1}{\PYGZsq{}}\PYG{l+s+s1}{a}\PYG{l+s+s1}{\PYGZsq{}}\PYG{p}{,}\PYG{l+s+s1}{\PYGZsq{}}\PYG{l+s+s1}{b}\PYG{l+s+s1}{\PYGZsq{}}\PYG{p}{,}\PYG{l+s+s1}{\PYGZsq{}}\PYG{l+s+s1}{c}\PYG{l+s+s1}{\PYGZsq{}}\PYG{p}{,}\PYG{l+s+s1}{\PYGZsq{}}\PYG{l+s+s1}{d}\PYG{l+s+s1}{\PYGZsq{}}\PYG{p}{,}\PYG{l+s+s1}{\PYGZsq{}}\PYG{l+s+s1}{e}\PYG{l+s+s1}{\PYGZsq{}}\PYG{p}{]}
\PYG{n}{df1} \PYG{o}{=} \PYG{n}{pd}\PYG{o}{.}\PYG{n}{DataFrame}\PYG{p}{(}\PYG{p}{\PYGZob{}}\PYG{l+s+s1}{\PYGZsq{}}\PYG{l+s+s1}{c1}\PYG{l+s+s1}{\PYGZsq{}}\PYG{p}{:} \PYG{n}{c1\PYGZus{}list}\PYG{p}{,} \PYG{l+s+s1}{\PYGZsq{}}\PYG{l+s+s1}{c2}\PYG{l+s+s1}{\PYGZsq{}}\PYG{p}{:} \PYG{n}{c2\PYGZus{}list}\PYG{p}{\PYGZcb{}}\PYG{p}{)}
\PYG{n+nb}{print}\PYG{p}{(}\PYG{n}{df1}\PYG{p}{)}

\PYG{n+nb}{print}\PYG{p}{(}\PYG{l+s+s1}{\PYGZsq{}}\PYG{l+s+se}{\PYGZbs{}n}\PYG{l+s+s1}{\PYGZsq{}}\PYG{p}{)}
\PYG{n}{df2} \PYG{o}{=} \PYG{n}{pd}\PYG{o}{.}\PYG{n}{DataFrame}\PYG{p}{(}\PYG{p}{\PYGZob{}}\PYG{l+s+s1}{\PYGZsq{}}\PYG{l+s+s1}{c1}\PYG{l+s+s1}{\PYGZsq{}}\PYG{p}{:} \PYG{n}{c1\PYGZus{}list}\PYG{p}{,} \PYG{l+s+s1}{\PYGZsq{}}\PYG{l+s+s1}{c2}\PYG{l+s+s1}{\PYGZsq{}}\PYG{p}{:} \PYG{n}{c2\PYGZus{}list}\PYG{p}{\PYGZcb{}}\PYG{p}{,} \PYG{n}{index}\PYG{o}{=}\PYG{p}{[}\PYG{l+s+s1}{\PYGZsq{}}\PYG{l+s+s1}{s1}\PYG{l+s+s1}{\PYGZsq{}}\PYG{p}{,}\PYG{l+s+s1}{\PYGZsq{}}\PYG{l+s+s1}{s2}\PYG{l+s+s1}{\PYGZsq{}}\PYG{p}{,}\PYG{l+s+s1}{\PYGZsq{}}\PYG{l+s+s1}{s3}\PYG{l+s+s1}{\PYGZsq{}}\PYG{p}{,}\PYG{l+s+s1}{\PYGZsq{}}\PYG{l+s+s1}{s4}\PYG{l+s+s1}{\PYGZsq{}}\PYG{p}{,}\PYG{l+s+s1}{\PYGZsq{}}\PYG{l+s+s1}{s4}\PYG{l+s+s1}{\PYGZsq{}}\PYG{p}{]}\PYG{p}{)}
\PYG{n+nb}{print}\PYG{p}{(}\PYG{n}{df2}\PYG{p}{)}

\PYG{n+nb}{print}\PYG{p}{(}\PYG{l+s+s1}{\PYGZsq{}}\PYG{l+s+se}{\PYGZbs{}n}\PYG{l+s+s1}{\PYGZsq{}}\PYG{p}{)}
\PYG{n+nb}{print}\PYG{p}{(}\PYG{n}{df2}\PYG{o}{.}\PYG{n}{loc}\PYG{p}{[}\PYG{l+s+s1}{\PYGZsq{}}\PYG{l+s+s1}{s3}\PYG{l+s+s1}{\PYGZsq{}}\PYG{p}{]}\PYG{p}{[}\PYG{l+s+s1}{\PYGZsq{}}\PYG{l+s+s1}{c1}\PYG{l+s+s1}{\PYGZsq{}}\PYG{p}{]}\PYG{p}{)} \PYG{c+c1}{\PYGZsh{} 13 출력}
\PYG{n+nb}{print}\PYG{p}{(}\PYG{n}{df2}\PYG{o}{.}\PYG{n}{loc}\PYG{p}{[}\PYG{l+s+s1}{\PYGZsq{}}\PYG{l+s+s1}{s3}\PYG{l+s+s1}{\PYGZsq{}}\PYG{p}{]}\PYG{p}{[}\PYG{p}{[}\PYG{l+s+s1}{\PYGZsq{}}\PYG{l+s+s1}{c1}\PYG{l+s+s1}{\PYGZsq{}}\PYG{p}{,}\PYG{l+s+s1}{\PYGZsq{}}\PYG{l+s+s1}{c2}\PYG{l+s+s1}{\PYGZsq{}}\PYG{p}{]}\PYG{p}{]}\PYG{p}{)} \PYG{c+c1}{\PYGZsh{} 13 과 c 출력}
\end{sphinxVerbatim}

\end{sphinxuseclass}\end{sphinxVerbatimInput}
\begin{sphinxVerbatimOutput}

\begin{sphinxuseclass}{cell_output}
\begin{sphinxVerbatim}[commandchars=\\\{\}]
   c1 c2
0  11  a
1  12  b
2  13  c
3  14  d
4  15  e


    c1 c2
s1  11  a
s2  12  b
s3  13  c
s4  14  d
s4  15  e


13
c1    13
c2     c
Name: s3, dtype: object
\end{sphinxVerbatim}

\end{sphinxuseclass}\end{sphinxVerbatimOutput}

\end{sphinxuseclass}

\part{Index 생성 및 추출}
\label{\detokenize{chapter2/2.1.2_Python_Basics:id2}}
\sphinxAtStartPar
set\_index 메소드로 기존의 column 을 index 로 만들 수 있습니다. set\_index(‘c2’) 처리 후, df2 를 출력하시면 df1 의 ‘c2’ 컬럼이 index 로 되어 있음을 확인할 수 있습니다.
이제 df2 의 index 값을 변경해 보겠습니다. 아래와 같이 DataFrame 의 Index를 호출하여 원하는 Index 로 교체도 가능합니다. 참고로 아래 df2 는 column 하나지만 현재 Series 가 아닌 DataFrame 입니다.

\begin{sphinxuseclass}{cell}\begin{sphinxVerbatimInput}

\begin{sphinxuseclass}{cell_input}
\begin{sphinxVerbatim}[commandchars=\\\{\}]
\PYG{n}{c1\PYGZus{}list} \PYG{o}{=} \PYG{p}{[}\PYG{l+m+mi}{11}\PYG{p}{,}\PYG{l+m+mi}{12}\PYG{p}{,}\PYG{l+m+mi}{13}\PYG{p}{,}\PYG{l+m+mi}{14}\PYG{p}{,}\PYG{l+m+mi}{15}\PYG{p}{]}
\PYG{n}{c2\PYGZus{}list} \PYG{o}{=} \PYG{p}{[}\PYG{l+s+s1}{\PYGZsq{}}\PYG{l+s+s1}{a}\PYG{l+s+s1}{\PYGZsq{}}\PYG{p}{,}\PYG{l+s+s1}{\PYGZsq{}}\PYG{l+s+s1}{b}\PYG{l+s+s1}{\PYGZsq{}}\PYG{p}{,}\PYG{l+s+s1}{\PYGZsq{}}\PYG{l+s+s1}{c}\PYG{l+s+s1}{\PYGZsq{}}\PYG{p}{,}\PYG{l+s+s1}{\PYGZsq{}}\PYG{l+s+s1}{d}\PYG{l+s+s1}{\PYGZsq{}}\PYG{p}{,}\PYG{l+s+s1}{\PYGZsq{}}\PYG{l+s+s1}{e}\PYG{l+s+s1}{\PYGZsq{}}\PYG{p}{]}
\PYG{n}{df1} \PYG{o}{=} \PYG{n}{pd}\PYG{o}{.}\PYG{n}{DataFrame}\PYG{p}{(}\PYG{p}{\PYGZob{}}\PYG{l+s+s1}{\PYGZsq{}}\PYG{l+s+s1}{c1}\PYG{l+s+s1}{\PYGZsq{}}\PYG{p}{:} \PYG{n}{c1\PYGZus{}list}\PYG{p}{,} \PYG{l+s+s1}{\PYGZsq{}}\PYG{l+s+s1}{c2}\PYG{l+s+s1}{\PYGZsq{}}\PYG{p}{:} \PYG{n}{c2\PYGZus{}list}\PYG{p}{\PYGZcb{}}\PYG{p}{)}
\PYG{n+nb}{print}\PYG{p}{(}\PYG{n}{df1}\PYG{p}{)}        

\PYG{n}{df2} \PYG{o}{=} \PYG{n}{df1}\PYG{o}{.}\PYG{n}{set\PYGZus{}index}\PYG{p}{(}\PYG{l+s+s1}{\PYGZsq{}}\PYG{l+s+s1}{c2}\PYG{l+s+s1}{\PYGZsq{}}\PYG{p}{)} \PYG{c+c1}{\PYGZsh{} c1 를 index 로 변경}
\PYG{n+nb}{print}\PYG{p}{(}\PYG{n}{df2}\PYG{p}{)}

\PYG{n+nb}{print}\PYG{p}{(}\PYG{l+s+s1}{\PYGZsq{}}\PYG{l+s+se}{\PYGZbs{}n}\PYG{l+s+s1}{\PYGZsq{}}\PYG{p}{)}
\PYG{n}{df2}\PYG{o}{.}\PYG{n}{index} \PYG{o}{=} \PYG{p}{[}\PYG{l+s+s1}{\PYGZsq{}}\PYG{l+s+s1}{ss1}\PYG{l+s+s1}{\PYGZsq{}}\PYG{p}{,} \PYG{l+s+s1}{\PYGZsq{}}\PYG{l+s+s1}{ss2}\PYG{l+s+s1}{\PYGZsq{}}\PYG{p}{,} \PYG{l+s+s1}{\PYGZsq{}}\PYG{l+s+s1}{ss3}\PYG{l+s+s1}{\PYGZsq{}}\PYG{p}{,} \PYG{l+s+s1}{\PYGZsq{}}\PYG{l+s+s1}{ss4}\PYG{l+s+s1}{\PYGZsq{}}\PYG{p}{,} \PYG{l+s+s1}{\PYGZsq{}}\PYG{l+s+s1}{ss5}\PYG{l+s+s1}{\PYGZsq{}}\PYG{p}{]}
\PYG{n+nb}{print}\PYG{p}{(}\PYG{n}{df2}\PYG{p}{,} \PYG{n+nb}{type}\PYG{p}{(}\PYG{n}{df2}\PYG{p}{)}\PYG{p}{)}
\end{sphinxVerbatim}

\end{sphinxuseclass}\end{sphinxVerbatimInput}
\begin{sphinxVerbatimOutput}

\begin{sphinxuseclass}{cell_output}
\begin{sphinxVerbatim}[commandchars=\\\{\}]
   c1 c2
0  11  a
1  12  b
2  13  c
3  14  d
4  15  e
    c1
c2    
a   11
b   12
c   13
d   14
e   15


     c1
ss1  11
ss2  12
ss3  13
ss4  14
ss5  15 \PYGZlt{}class \PYGZsq{}pandas.core.frame.DataFrame\PYGZsq{}\PYGZgt{}
\end{sphinxVerbatim}

\end{sphinxuseclass}\end{sphinxVerbatimOutput}

\end{sphinxuseclass}
\sphinxAtStartPar
 항상 두 데이터셋을 index 로 병합할 때는 index 에 중복이 있는지 확인을 할 필요가 있습니다. index 가 중복 여부를 체크하는 인수는 verify\_integriry 입니다. 아래는 중복이 있는 경우 에러를 발생시킵니다.

\begin{sphinxuseclass}{cell}\begin{sphinxVerbatimInput}

\begin{sphinxuseclass}{cell_input}
\begin{sphinxVerbatim}[commandchars=\\\{\}]
\PYG{n}{c1\PYGZus{}list} \PYG{o}{=} \PYG{p}{[}\PYG{l+m+mi}{11}\PYG{p}{,}\PYG{l+m+mi}{12}\PYG{p}{,}\PYG{l+m+mi}{13}\PYG{p}{,}\PYG{l+m+mi}{14}\PYG{p}{,}\PYG{l+m+mi}{15}\PYG{p}{]}
\PYG{n}{c2\PYGZus{}list} \PYG{o}{=} \PYG{p}{[}\PYG{l+s+s1}{\PYGZsq{}}\PYG{l+s+s1}{a}\PYG{l+s+s1}{\PYGZsq{}}\PYG{p}{,}\PYG{l+s+s1}{\PYGZsq{}}\PYG{l+s+s1}{a}\PYG{l+s+s1}{\PYGZsq{}}\PYG{p}{,}\PYG{l+s+s1}{\PYGZsq{}}\PYG{l+s+s1}{b}\PYG{l+s+s1}{\PYGZsq{}}\PYG{p}{,}\PYG{l+s+s1}{\PYGZsq{}}\PYG{l+s+s1}{c}\PYG{l+s+s1}{\PYGZsq{}}\PYG{p}{,}\PYG{l+s+s1}{\PYGZsq{}}\PYG{l+s+s1}{d}\PYG{l+s+s1}{\PYGZsq{}}\PYG{p}{]} \PYG{c+c1}{\PYGZsh{} 값에 중복이 있음}
\PYG{n}{df} \PYG{o}{=} \PYG{n}{pd}\PYG{o}{.}\PYG{n}{DataFrame}\PYG{p}{(}\PYG{p}{\PYGZob{}}\PYG{l+s+s1}{\PYGZsq{}}\PYG{l+s+s1}{c1}\PYG{l+s+s1}{\PYGZsq{}}\PYG{p}{:} \PYG{n}{c1\PYGZus{}list}\PYG{p}{,} \PYG{l+s+s1}{\PYGZsq{}}\PYG{l+s+s1}{c2}\PYG{l+s+s1}{\PYGZsq{}}\PYG{p}{:} \PYG{n}{c2\PYGZus{}list}\PYG{p}{\PYGZcb{}}\PYG{p}{)}
\PYG{n}{df}\PYG{o}{.}\PYG{n}{set\PYGZus{}index}\PYG{p}{(}\PYG{l+s+s1}{\PYGZsq{}}\PYG{l+s+s1}{c2}\PYG{l+s+s1}{\PYGZsq{}}\PYG{p}{,} \PYG{n}{verify\PYGZus{}integrity}\PYG{o}{=}\PYG{k+kc}{True}\PYG{p}{)} \PYG{c+c1}{\PYGZsh{} index 중복여부를 체크}
\end{sphinxVerbatim}

\end{sphinxuseclass}\end{sphinxVerbatimInput}
\begin{sphinxVerbatimOutput}

\begin{sphinxuseclass}{cell_output}
\begin{sphinxVerbatim}[commandchars=\\\{\}]
\PYG{g+gt}{\PYGZhy{}\PYGZhy{}\PYGZhy{}\PYGZhy{}\PYGZhy{}\PYGZhy{}\PYGZhy{}\PYGZhy{}\PYGZhy{}\PYGZhy{}\PYGZhy{}\PYGZhy{}\PYGZhy{}\PYGZhy{}\PYGZhy{}\PYGZhy{}\PYGZhy{}\PYGZhy{}\PYGZhy{}\PYGZhy{}\PYGZhy{}\PYGZhy{}\PYGZhy{}\PYGZhy{}\PYGZhy{}\PYGZhy{}\PYGZhy{}\PYGZhy{}\PYGZhy{}\PYGZhy{}\PYGZhy{}\PYGZhy{}\PYGZhy{}\PYGZhy{}\PYGZhy{}\PYGZhy{}\PYGZhy{}\PYGZhy{}\PYGZhy{}\PYGZhy{}\PYGZhy{}\PYGZhy{}\PYGZhy{}\PYGZhy{}\PYGZhy{}\PYGZhy{}\PYGZhy{}\PYGZhy{}\PYGZhy{}\PYGZhy{}\PYGZhy{}\PYGZhy{}\PYGZhy{}\PYGZhy{}\PYGZhy{}\PYGZhy{}\PYGZhy{}\PYGZhy{}\PYGZhy{}\PYGZhy{}\PYGZhy{}\PYGZhy{}\PYGZhy{}\PYGZhy{}\PYGZhy{}\PYGZhy{}\PYGZhy{}\PYGZhy{}\PYGZhy{}\PYGZhy{}\PYGZhy{}\PYGZhy{}\PYGZhy{}\PYGZhy{}\PYGZhy{}}
\PYG{n+ne}{ValueError}\PYG{g+gWhitespace}{                                }Traceback (most recent call last)
\PYG{o}{\PYGZti{}}\PYGZbs{}\PYG{n}{AppData}\PYGZbs{}\PYG{n}{Local}\PYGZbs{}\PYG{n}{Temp}\PYGZbs{}\PYG{n}{ipykernel\PYGZus{}3484}\PYGZbs{}\PYG{l+m+mf}{2078573242.}\PYG{n}{py} \PYG{o+ow}{in} \PYG{o}{\PYGZlt{}}\PYG{n}{module}\PYG{o}{\PYGZgt{}}
\PYG{g+gWhitespace}{      }\PYG{l+m+mi}{2} \PYG{n}{c2\PYGZus{}list} \PYG{o}{=} \PYG{p}{[}\PYG{l+s+s1}{\PYGZsq{}}\PYG{l+s+s1}{a}\PYG{l+s+s1}{\PYGZsq{}}\PYG{p}{,}\PYG{l+s+s1}{\PYGZsq{}}\PYG{l+s+s1}{a}\PYG{l+s+s1}{\PYGZsq{}}\PYG{p}{,}\PYG{l+s+s1}{\PYGZsq{}}\PYG{l+s+s1}{b}\PYG{l+s+s1}{\PYGZsq{}}\PYG{p}{,}\PYG{l+s+s1}{\PYGZsq{}}\PYG{l+s+s1}{c}\PYG{l+s+s1}{\PYGZsq{}}\PYG{p}{,}\PYG{l+s+s1}{\PYGZsq{}}\PYG{l+s+s1}{d}\PYG{l+s+s1}{\PYGZsq{}}\PYG{p}{]} \PYG{c+c1}{\PYGZsh{} 값에 중복이 있음}
\PYG{g+gWhitespace}{      }\PYG{l+m+mi}{3} \PYG{n}{df} \PYG{o}{=} \PYG{n}{pd}\PYG{o}{.}\PYG{n}{DataFrame}\PYG{p}{(}\PYG{p}{\PYGZob{}}\PYG{l+s+s1}{\PYGZsq{}}\PYG{l+s+s1}{c1}\PYG{l+s+s1}{\PYGZsq{}}\PYG{p}{:} \PYG{n}{c1\PYGZus{}list}\PYG{p}{,} \PYG{l+s+s1}{\PYGZsq{}}\PYG{l+s+s1}{c2}\PYG{l+s+s1}{\PYGZsq{}}\PYG{p}{:} \PYG{n}{c2\PYGZus{}list}\PYG{p}{\PYGZcb{}}\PYG{p}{)}
\PYG{n+ne}{\PYGZhy{}\PYGZhy{}\PYGZhy{}\PYGZhy{}\PYGZgt{} }\PYG{l+m+mi}{4} \PYG{n}{df}\PYG{o}{.}\PYG{n}{set\PYGZus{}index}\PYG{p}{(}\PYG{l+s+s1}{\PYGZsq{}}\PYG{l+s+s1}{c2}\PYG{l+s+s1}{\PYGZsq{}}\PYG{p}{,} \PYG{n}{verify\PYGZus{}integrity}\PYG{o}{=}\PYG{k+kc}{True}\PYG{p}{)} \PYG{c+c1}{\PYGZsh{} index 중복여부를 체크}

\PYG{n+nn}{\PYGZti{}\PYGZbs{}Anaconda3\PYGZbs{}lib\PYGZbs{}site\PYGZhy{}packages\PYGZbs{}pandas\PYGZbs{}util\PYGZbs{}\PYGZus{}decorators.py} in \PYG{n+ni}{wrapper}\PYG{n+nt}{(*args, **kwargs)}
\PYG{g+gWhitespace}{    }\PYG{l+m+mi}{309}                     \PYG{n}{stacklevel}\PYG{o}{=}\PYG{n}{stacklevel}\PYG{p}{,}
\PYG{g+gWhitespace}{    }\PYG{l+m+mi}{310}                 \PYG{p}{)}
\PYG{n+ne}{\PYGZhy{}\PYGZhy{}\PYGZgt{} }\PYG{l+m+mi}{311}             \PYG{k}{return} \PYG{n}{func}\PYG{p}{(}\PYG{o}{*}\PYG{n}{args}\PYG{p}{,} \PYG{o}{*}\PYG{o}{*}\PYG{n}{kwargs}\PYG{p}{)}
\PYG{g+gWhitespace}{    }\PYG{l+m+mi}{312} 
\PYG{g+gWhitespace}{    }\PYG{l+m+mi}{313}         \PYG{k}{return} \PYG{n}{wrapper}

\PYG{n+nn}{\PYGZti{}\PYGZbs{}Anaconda3\PYGZbs{}lib\PYGZbs{}site\PYGZhy{}packages\PYGZbs{}pandas\PYGZbs{}core\PYGZbs{}frame.py} in \PYG{n+ni}{set\PYGZus{}index}\PYG{n+nt}{(self, keys, drop, append, inplace, verify\PYGZus{}integrity)}
\PYG{g+gWhitespace}{   }\PYG{l+m+mi}{5508}         \PYG{k}{if} \PYG{n}{verify\PYGZus{}integrity} \PYG{o+ow}{and} \PYG{o+ow}{not} \PYG{n}{index}\PYG{o}{.}\PYG{n}{is\PYGZus{}unique}\PYG{p}{:}
\PYG{g+gWhitespace}{   }\PYG{l+m+mi}{5509}             \PYG{n}{duplicates} \PYG{o}{=} \PYG{n}{index}\PYG{p}{[}\PYG{n}{index}\PYG{o}{.}\PYG{n}{duplicated}\PYG{p}{(}\PYG{p}{)}\PYG{p}{]}\PYG{o}{.}\PYG{n}{unique}\PYG{p}{(}\PYG{p}{)}
\PYG{n+ne}{\PYGZhy{}\PYGZgt{} }\PYG{l+m+mi}{5510}             \PYG{k}{raise} \PYG{n+ne}{ValueError}\PYG{p}{(}\PYG{l+s+sa}{f}\PYG{l+s+s2}{\PYGZdq{}}\PYG{l+s+s2}{Index has duplicate keys: }\PYG{l+s+si}{\PYGZob{}}\PYG{n}{duplicates}\PYG{l+s+si}{\PYGZcb{}}\PYG{l+s+s2}{\PYGZdq{}}\PYG{p}{)}
\PYG{g+gWhitespace}{   }\PYG{l+m+mi}{5511} 
\PYG{g+gWhitespace}{   }\PYG{l+m+mi}{5512}         \PYG{c+c1}{\PYGZsh{} use set to handle duplicate column names gracefully in case of drop}

\PYG{n+ne}{ValueError}: Index has duplicate keys: Index([\PYGZsq{}a\PYGZsq{}], dtype=\PYGZsq{}object\PYGZsq{}, name=\PYGZsq{}c2\PYGZsq{})
\end{sphinxVerbatim}

\end{sphinxuseclass}\end{sphinxVerbatimOutput}

\end{sphinxuseclass}

\part{For Loop}
\label{\detokenize{chapter2/2.1.3_Python_Basics:for-loop}}\label{\detokenize{chapter2/2.1.3_Python_Basics::doc}}
\sphinxAtStartPar
컴퓨터를 잘 활용한다는 것의 컴퓨터의 3 가지 강점 \sphinxhyphen{} 기억, 반복, 계산을 잘 활용한다는 뜻입니다. 그 중에서도 인간보다 탁월한 능력이 바로 반복입니다. 컴퓨터는 수만번, 수천번의 반복도 금방 해 치웁니다. 이번에는 그 반복문을 배우겠습니다. 반복문 중에 for \textasciitilde{} in 구분이 가장 많이 활용됩니다. for \textasciitilde{} in 형식에서 in 다음에 List 를 넣으면 List 의 원소를 순서대로 꺼내어 처리합니다. 단순히 출력만 해보겠습니다. 다음에는 제곱한 값을 출력해 보겠습니다.

\begin{sphinxuseclass}{cell}\begin{sphinxVerbatimInput}

\begin{sphinxuseclass}{cell_input}
\begin{sphinxVerbatim}[commandchars=\\\{\}]
\PYG{n}{num\PYGZus{}list} \PYG{o}{=} \PYG{p}{[}\PYG{l+m+mi}{1}\PYG{p}{,}\PYG{l+m+mi}{2}\PYG{p}{,}\PYG{l+m+mi}{3}\PYG{p}{,}\PYG{l+m+mi}{4}\PYG{p}{,}\PYG{l+m+mi}{5}\PYG{p}{,}\PYG{l+m+mi}{6}\PYG{p}{]}
\PYG{k}{for} \PYG{n}{i} \PYG{o+ow}{in} \PYG{n}{num\PYGZus{}list}\PYG{p}{:}
    \PYG{n+nb}{print}\PYG{p}{(}\PYG{n}{i}\PYG{p}{)}

\PYG{n+nb}{print}\PYG{p}{(}\PYG{l+s+s1}{\PYGZsq{}}\PYG{l+s+se}{\PYGZbs{}n}\PYG{l+s+s1}{\PYGZsq{}}\PYG{p}{)}    
\PYG{k}{for} \PYG{n}{i} \PYG{o+ow}{in} \PYG{n}{num\PYGZus{}list}\PYG{p}{:}
    \PYG{n+nb}{print}\PYG{p}{(}\PYG{n}{i}\PYG{o}{*}\PYG{o}{*}\PYG{l+m+mi}{2}\PYG{p}{)}    
\end{sphinxVerbatim}

\end{sphinxuseclass}\end{sphinxVerbatimInput}
\begin{sphinxVerbatimOutput}

\begin{sphinxuseclass}{cell_output}
\begin{sphinxVerbatim}[commandchars=\\\{\}]
1
2
3
4
5
6


1
4
9
16
25
36
\end{sphinxVerbatim}

\end{sphinxuseclass}\end{sphinxVerbatimOutput}

\end{sphinxuseclass}


\begin{sphinxuseclass}{cell}\begin{sphinxVerbatimInput}

\begin{sphinxuseclass}{cell_input}
\begin{sphinxVerbatim}[commandchars=\\\{\}]
\PYG{c+c1}{\PYGZsh{} break}
\PYG{k}{for} \PYG{n}{i} \PYG{o+ow}{in} \PYG{n}{num\PYGZus{}list}\PYG{p}{:}
    \PYG{k}{if} \PYG{n}{i} \PYG{o}{==} \PYG{l+m+mi}{3}\PYG{p}{:}
        \PYG{k}{break}
    \PYG{n+nb}{print}\PYG{p}{(}\PYG{n}{i}\PYG{o}{*}\PYG{o}{*}\PYG{l+m+mi}{2}\PYG{p}{)}  
\end{sphinxVerbatim}

\end{sphinxuseclass}\end{sphinxVerbatimInput}
\begin{sphinxVerbatimOutput}

\begin{sphinxuseclass}{cell_output}
\begin{sphinxVerbatim}[commandchars=\\\{\}]
1
4
\end{sphinxVerbatim}

\end{sphinxuseclass}\end{sphinxVerbatimOutput}

\end{sphinxuseclass}
\begin{sphinxuseclass}{cell}\begin{sphinxVerbatimInput}

\begin{sphinxuseclass}{cell_input}
\begin{sphinxVerbatim}[commandchars=\\\{\}]
\PYG{c+c1}{\PYGZsh{} continue}
\PYG{k}{for} \PYG{n}{i} \PYG{o+ow}{in} \PYG{n}{num\PYGZus{}list}\PYG{p}{:}
    \PYG{k}{if} \PYG{n}{i} \PYG{o}{==} \PYG{l+m+mi}{3}\PYG{p}{:}
        \PYG{k}{continue}
    \PYG{n+nb}{print}\PYG{p}{(}\PYG{n}{i}\PYG{o}{*}\PYG{o}{*}\PYG{l+m+mi}{2}\PYG{p}{)}    
\end{sphinxVerbatim}

\end{sphinxuseclass}\end{sphinxVerbatimInput}
\begin{sphinxVerbatimOutput}

\begin{sphinxuseclass}{cell_output}
\begin{sphinxVerbatim}[commandchars=\\\{\}]
1
4
16
25
36
\end{sphinxVerbatim}

\end{sphinxuseclass}\end{sphinxVerbatimOutput}

\end{sphinxuseclass}

\part{While Loop}
\label{\detokenize{chapter2/2.1.3_Python_Basics:while-loop}}
\sphinxAtStartPar
While 반복문도 자주 활용됩니다. While 안의 조건이 만족하는 한, 계속 반복합니다. break 문으로 While Loop 를 빠져나올 수 있습니다.

\begin{sphinxuseclass}{cell}\begin{sphinxVerbatimInput}

\begin{sphinxuseclass}{cell_input}
\begin{sphinxVerbatim}[commandchars=\\\{\}]
\PYG{n}{i} \PYG{o}{=} \PYG{l+m+mi}{0}
\PYG{k}{while}\PYG{p}{(}\PYG{k+kc}{True}\PYG{p}{)}\PYG{p}{:}
    \PYG{n}{i} \PYG{o}{=} \PYG{n}{i} \PYG{o}{+} \PYG{l+m+mi}{1}
    \PYG{n+nb}{print}\PYG{p}{(}\PYG{n}{i}\PYG{o}{*}\PYG{o}{*}\PYG{l+m+mi}{2}\PYG{p}{)}
    \PYG{k}{if} \PYG{n}{i} \PYG{o}{==} \PYG{l+m+mi}{10}\PYG{p}{:}
        \PYG{k}{break}
\end{sphinxVerbatim}

\end{sphinxuseclass}\end{sphinxVerbatimInput}
\begin{sphinxVerbatimOutput}

\begin{sphinxuseclass}{cell_output}
\begin{sphinxVerbatim}[commandchars=\\\{\}]
1
4
9
16
25
36
49
64
81
100
\end{sphinxVerbatim}

\end{sphinxuseclass}\end{sphinxVerbatimOutput}

\end{sphinxuseclass}
\begin{sphinxuseclass}{cell}\begin{sphinxVerbatimInput}

\begin{sphinxuseclass}{cell_input}
\begin{sphinxVerbatim}[commandchars=\\\{\}]
\PYG{n}{i} \PYG{o}{=} \PYG{l+m+mi}{0}
\PYG{k}{while}\PYG{p}{(}\PYG{n}{i}\PYG{o}{\PYGZlt{}}\PYG{l+m+mi}{10}\PYG{p}{)}\PYG{p}{:}
    \PYG{n}{i} \PYG{o}{=} \PYG{n}{i} \PYG{o}{+} \PYG{l+m+mi}{1}
    \PYG{n+nb}{print}\PYG{p}{(}\PYG{n}{i}\PYG{o}{*}\PYG{o}{*}\PYG{l+m+mi}{2}\PYG{p}{)}
\end{sphinxVerbatim}

\end{sphinxuseclass}\end{sphinxVerbatimInput}
\begin{sphinxVerbatimOutput}

\begin{sphinxuseclass}{cell_output}
\begin{sphinxVerbatim}[commandchars=\\\{\}]
1
4
9
16
25
36
49
64
81
100
\end{sphinxVerbatim}

\end{sphinxuseclass}\end{sphinxVerbatimOutput}

\end{sphinxuseclass}

\part{If Condition}
\label{\detokenize{chapter2/2.1.4_Python_Basics:if-condition}}\label{\detokenize{chapter2/2.1.4_Python_Basics::doc}}
\sphinxAtStartPar
파이썬의 조건문 if \textasciitilde{} else 형식으로 다른 컴퓨터 언어와 다르지 않습니다. 단지 else if 부분은 줄여서 elif 로 씁니다. 아래 예제를 보시면 쉽게 이해가 되실 것으로 생각합니다.

\begin{sphinxuseclass}{cell}\begin{sphinxVerbatimInput}

\begin{sphinxuseclass}{cell_input}
\begin{sphinxVerbatim}[commandchars=\\\{\}]
\PYG{n}{a} \PYG{o}{=} \PYG{l+m+mi}{3}
\PYG{n}{b} \PYG{o}{=} \PYG{l+m+mi}{2}
\PYG{k}{if} \PYG{n}{a} \PYG{o}{\PYGZgt{}} \PYG{n}{b}\PYG{p}{:}
    \PYG{n+nb}{print}\PYG{p}{(}\PYG{l+s+s1}{\PYGZsq{}}\PYG{l+s+s1}{a \PYGZgt{} b}\PYG{l+s+s1}{\PYGZsq{}}\PYG{p}{)}
\PYG{k}{else}\PYG{p}{:}
    \PYG{n+nb}{print}\PYG{p}{(}\PYG{l+s+s1}{\PYGZsq{}}\PYG{l+s+s1}{a \PYGZlt{}= b}\PYG{l+s+s1}{\PYGZsq{}}\PYG{p}{)}
\end{sphinxVerbatim}

\end{sphinxuseclass}\end{sphinxVerbatimInput}
\begin{sphinxVerbatimOutput}

\begin{sphinxuseclass}{cell_output}
\begin{sphinxVerbatim}[commandchars=\\\{\}]
a \PYGZgt{} b
\end{sphinxVerbatim}

\end{sphinxuseclass}\end{sphinxVerbatimOutput}

\end{sphinxuseclass}
\begin{sphinxuseclass}{cell}\begin{sphinxVerbatimInput}

\begin{sphinxuseclass}{cell_input}
\begin{sphinxVerbatim}[commandchars=\\\{\}]
\PYG{n}{num\PYGZus{}list} \PYG{o}{=} \PYG{p}{[}\PYG{l+m+mi}{1}\PYG{p}{,}\PYG{l+m+mi}{2}\PYG{p}{,}\PYG{l+m+mi}{3}\PYG{p}{,}\PYG{l+m+mi}{4}\PYG{p}{,}\PYG{l+m+mi}{5}\PYG{p}{,}\PYG{l+m+mi}{6}\PYG{p}{]}

\PYG{k}{for} \PYG{n}{i} \PYG{o+ow}{in} \PYG{n}{num\PYGZus{}list}\PYG{p}{:}
    \PYG{k}{if} \PYG{n}{i} \PYG{o}{\PYGZlt{}} \PYG{l+m+mi}{3}\PYG{p}{:}
        \PYG{n+nb}{print}\PYG{p}{(}\PYG{n}{i}\PYG{p}{,} \PYG{l+s+s1}{\PYGZsq{}}\PYG{l+s+s1}{the number is less than 3}\PYG{l+s+s1}{\PYGZsq{}}\PYG{p}{)}
    \PYG{k}{elif} \PYG{n}{i} \PYG{o}{\PYGZgt{}} \PYG{l+m+mi}{3}\PYG{p}{:}
        \PYG{n+nb}{print}\PYG{p}{(}\PYG{n}{i}\PYG{p}{,} \PYG{l+s+s1}{\PYGZsq{}}\PYG{l+s+s1}{The number is greater than 3}\PYG{l+s+s1}{\PYGZsq{}}\PYG{p}{)}
    \PYG{k}{else}\PYG{p}{:}
        \PYG{n+nb}{print}\PYG{p}{(}\PYG{n}{i}\PYG{p}{,} \PYG{l+s+s1}{\PYGZsq{}}\PYG{l+s+s1}{The number is 3}\PYG{l+s+s1}{\PYGZsq{}}\PYG{p}{)}
\end{sphinxVerbatim}

\end{sphinxuseclass}\end{sphinxVerbatimInput}
\begin{sphinxVerbatimOutput}

\begin{sphinxuseclass}{cell_output}
\begin{sphinxVerbatim}[commandchars=\\\{\}]
1 the number is less than 3
2 the number is less than 3
3 The number is 3
4 The number is greater than 3
5 The number is greater than 3
6 The number is greater than 3
\end{sphinxVerbatim}

\end{sphinxuseclass}\end{sphinxVerbatimOutput}

\end{sphinxuseclass}
\begin{sphinxuseclass}{cell}\begin{sphinxVerbatimInput}

\begin{sphinxuseclass}{cell_input}
\begin{sphinxVerbatim}[commandchars=\\\{\}]
\PYG{k}{for} \PYG{n}{i} \PYG{o+ow}{in} \PYG{n}{num\PYGZus{}list}\PYG{p}{:}
    \PYG{k}{if} \PYG{n}{i} \PYG{o}{\PYGZlt{}} \PYG{l+m+mi}{3}\PYG{p}{:}
        \PYG{n+nb}{print}\PYG{p}{(}\PYG{n}{i}\PYG{p}{,} \PYG{l+s+s1}{\PYGZsq{}}\PYG{l+s+s1}{the number is less than 3}\PYG{l+s+s1}{\PYGZsq{}}\PYG{p}{)}
    \PYG{k}{elif} \PYG{n}{i} \PYG{o}{\PYGZgt{}} \PYG{l+m+mi}{3}\PYG{p}{:}
        \PYG{k}{pass}  \PYG{c+c1}{\PYGZsh{} 아무런 처리를 하지 않음}
    \PYG{k}{else}\PYG{p}{:}
        \PYG{n+nb}{print}\PYG{p}{(}\PYG{n}{i}\PYG{p}{,} \PYG{l+s+s1}{\PYGZsq{}}\PYG{l+s+s1}{The number is 3}\PYG{l+s+s1}{\PYGZsq{}}\PYG{p}{)}
\end{sphinxVerbatim}

\end{sphinxuseclass}\end{sphinxVerbatimInput}
\begin{sphinxVerbatimOutput}

\begin{sphinxuseclass}{cell_output}
\begin{sphinxVerbatim}[commandchars=\\\{\}]
1 the number is less than 3
2 the number is less than 3
3 The number is 3
\end{sphinxVerbatim}

\end{sphinxuseclass}\end{sphinxVerbatimOutput}

\end{sphinxuseclass}

\part{Functions}
\label{\detokenize{chapter2/2.1.5_Python_Basics:functions}}\label{\detokenize{chapter2/2.1.5_Python_Basics::doc}}
\sphinxAtStartPar
파이썬의 함수는 def 로 시작하고 결과값을 return 으로 반환합니다. 결과값의 반환은 여러 개도 가능합니다. 단, 함수 호출 후 결과 값을 받을 때, 함수가 return 하는 결과 값 갯수가 동일해야 합니다. 함수도 아래 예제를 보시면 쉽게 이해가 되시리라 생각합니다.

\begin{sphinxuseclass}{cell}\begin{sphinxVerbatimInput}

\begin{sphinxuseclass}{cell_input}
\begin{sphinxVerbatim}[commandchars=\\\{\}]
\PYG{k}{def} \PYG{n+nf}{cal}\PYG{p}{(}\PYG{n}{x}\PYG{p}{,} \PYG{n}{y}\PYG{p}{)}\PYG{p}{:}
    \PYG{n}{z} \PYG{o}{=} \PYG{n}{x} \PYG{o}{+} \PYG{n}{y}
    \PYG{k}{return} \PYG{n}{z}

\PYG{n}{result} \PYG{o}{=} \PYG{n}{cal}\PYG{p}{(}\PYG{l+m+mi}{2}\PYG{p}{,}\PYG{l+m+mi}{3}\PYG{p}{)}
\PYG{n+nb}{print}\PYG{p}{(}\PYG{n}{result}\PYG{p}{)}
\end{sphinxVerbatim}

\end{sphinxuseclass}\end{sphinxVerbatimInput}
\begin{sphinxVerbatimOutput}

\begin{sphinxuseclass}{cell_output}
\begin{sphinxVerbatim}[commandchars=\\\{\}]
5
\end{sphinxVerbatim}

\end{sphinxuseclass}\end{sphinxVerbatimOutput}

\end{sphinxuseclass}
\begin{sphinxuseclass}{cell}\begin{sphinxVerbatimInput}

\begin{sphinxuseclass}{cell_input}
\begin{sphinxVerbatim}[commandchars=\\\{\}]
\PYG{k}{def} \PYG{n+nf}{cal}\PYG{p}{(}\PYG{n}{x}\PYG{p}{,} \PYG{n}{y}\PYG{p}{)}\PYG{p}{:}
    \PYG{n}{z1} \PYG{o}{=} \PYG{n}{x} \PYG{o}{+} \PYG{n}{y}
    \PYG{n}{z2} \PYG{o}{=} \PYG{n}{x}\PYG{o}{*}\PYG{n}{y}
    \PYG{k}{return} \PYG{n}{z1}\PYG{p}{,} \PYG{n}{z2}

\PYG{n}{result1}\PYG{p}{,} \PYG{n}{result2} \PYG{o}{=} \PYG{n}{cal}\PYG{p}{(}\PYG{l+m+mi}{2}\PYG{p}{,}\PYG{l+m+mi}{3}\PYG{p}{)}
\PYG{n+nb}{print}\PYG{p}{(}\PYG{n}{result1}\PYG{p}{,} \PYG{n}{result2}\PYG{p}{)}
\end{sphinxVerbatim}

\end{sphinxuseclass}\end{sphinxVerbatimInput}
\begin{sphinxVerbatimOutput}

\begin{sphinxuseclass}{cell_output}
\begin{sphinxVerbatim}[commandchars=\\\{\}]
5 6
\end{sphinxVerbatim}

\end{sphinxuseclass}\end{sphinxVerbatimOutput}

\end{sphinxuseclass}
\begin{sphinxuseclass}{cell}\begin{sphinxVerbatimInput}

\begin{sphinxuseclass}{cell_input}
\begin{sphinxVerbatim}[commandchars=\\\{\}]
\PYG{k}{def} \PYG{n+nf}{cal}\PYG{p}{(}\PYG{n}{x}\PYG{p}{,} \PYG{n}{y}\PYG{p}{)}\PYG{p}{:}
    \PYG{k}{return}  \PYG{p}{(}\PYG{n}{x}\PYG{o}{+}\PYG{n}{y}\PYG{p}{)}\PYG{p}{,} \PYG{p}{(}\PYG{n}{x}\PYG{o}{*}\PYG{n}{y}\PYG{p}{)}\PYG{p}{,} \PYG{p}{(}\PYG{n}{x}\PYG{o}{*}\PYG{o}{*}\PYG{n}{y}\PYG{p}{)}

\PYG{n}{result1}\PYG{p}{,} \PYG{n}{result2}\PYG{p}{,} \PYG{n}{result3} \PYG{o}{=} \PYG{n}{cal}\PYG{p}{(}\PYG{l+m+mi}{2}\PYG{p}{,}\PYG{l+m+mi}{3}\PYG{p}{)}
\PYG{n+nb}{print}\PYG{p}{(}\PYG{n}{result1}\PYG{p}{,} \PYG{n}{result2}\PYG{p}{,} \PYG{n}{result3}\PYG{p}{)}
\end{sphinxVerbatim}

\end{sphinxuseclass}\end{sphinxVerbatimInput}
\begin{sphinxVerbatimOutput}

\begin{sphinxuseclass}{cell_output}
\begin{sphinxVerbatim}[commandchars=\\\{\}]
5 6 8
\end{sphinxVerbatim}

\end{sphinxuseclass}\end{sphinxVerbatimOutput}

\end{sphinxuseclass}






\renewcommand{\indexname}{Index}
\printindex
\end{document}