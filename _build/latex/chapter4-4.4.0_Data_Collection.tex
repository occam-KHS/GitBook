%% Generated by Sphinx.
\def\sphinxdocclass{jupyterBook}
\documentclass[letterpaper,10pt,english]{jupyterBook}
\ifdefined\pdfpxdimen
   \let\sphinxpxdimen\pdfpxdimen\else\newdimen\sphinxpxdimen
\fi \sphinxpxdimen=.75bp\relax
\ifdefined\pdfimageresolution
    \pdfimageresolution= \numexpr \dimexpr1in\relax/\sphinxpxdimen\relax
\fi
%% let collapsible pdf bookmarks panel have high depth per default
\PassOptionsToPackage{bookmarksdepth=5}{hyperref}
%% turn off hyperref patch of \index as sphinx.xdy xindy module takes care of
%% suitable \hyperpage mark-up, working around hyperref-xindy incompatibility
\PassOptionsToPackage{hyperindex=false}{hyperref}
%% memoir class requires extra handling
\makeatletter\@ifclassloaded{memoir}
{\ifdefined\memhyperindexfalse\memhyperindexfalse\fi}{}\makeatother

\PassOptionsToPackage{warn}{textcomp}

\catcode`^^^^00a0\active\protected\def^^^^00a0{\leavevmode\nobreak\ }
\usepackage{cmap}
\usepackage{fontspec}
\defaultfontfeatures[\rmfamily,\sffamily,\ttfamily]{}
\usepackage{amsmath,amssymb,amstext}
\usepackage{polyglossia}
\setmainlanguage{english}



\setmainfont{FreeSerif}[
  Extension      = .otf,
  UprightFont    = *,
  ItalicFont     = *Italic,
  BoldFont       = *Bold,
  BoldItalicFont = *BoldItalic
]
\setsansfont{FreeSans}[
  Extension      = .otf,
  UprightFont    = *,
  ItalicFont     = *Oblique,
  BoldFont       = *Bold,
  BoldItalicFont = *BoldOblique,
]
\setmonofont{FreeMono}[
  Extension      = .otf,
  UprightFont    = *,
  ItalicFont     = *Oblique,
  BoldFont       = *Bold,
  BoldItalicFont = *BoldOblique,
]



\usepackage[Bjarne]{fncychap}
\usepackage[,numfigreset=1,mathnumfig]{sphinx}

\fvset{fontsize=\small}
\usepackage{geometry}


% Include hyperref last.
\usepackage{hyperref}
% Fix anchor placement for figures with captions.
\usepackage{hypcap}% it must be loaded after hyperref.
% Set up styles of URL: it should be placed after hyperref.
\urlstyle{same}


\usepackage{sphinxmessages}



        % Start of preamble defined in sphinx-jupyterbook-latex %
         \usepackage[Latin,Greek]{ucharclasses}
        \usepackage{unicode-math}
        % fixing title of the toc
        \addto\captionsenglish{\renewcommand{\contentsname}{Contents}}
        \hypersetup{
            pdfencoding=auto,
            psdextra
        }
        % End of preamble defined in sphinx-jupyterbook-latex %
        

\title{데이터 수집 및 분석}
\date{Jul 02, 2022}
\release{}
\author{KHS}
\newcommand{\sphinxlogo}{\vbox{}}
\renewcommand{\releasename}{}
\makeindex
\begin{document}

\pagestyle{empty}
\sphinxmaketitle
\pagestyle{plain}
\sphinxtableofcontents
\pagestyle{normal}
\phantomsection\label{\detokenize{chapter4/4.4.0_Data_Collection::doc}}


\sphinxAtStartPar
이 부분이 데이터분석에서 가장 많은 시간이 필요한 부분입니다. 데이터를 수집해야하고 분석할 수 있는 모양으로 데이터 가공을 해야합니다. 다행이 일봉데이터는 데이터 크렌징은 필요없습니다. 하지만, 판다스 라이브러리를 이용하여 많은 가공이 필요합니다. 이 부분은 피쳐 엔지니어링이라고도 부릅니다.

\sphinxAtStartPar
일봉 데이터 수집은 웹크롤링으로 수집할 수 도 있고, 파이썬 라이브러리로도 수집이 가능합니다. 일봉 데이터에는 시종고저 값과 거래량 값이 기본적으로 제공됩니다. 우리는 이렇게 5개의 데이터를 이용하여 위 가설을 검정 해 볼 것 입니다.

\begin{sphinxuseclass}{cell}\begin{sphinxVerbatimInput}

\begin{sphinxuseclass}{cell_input}
\begin{sphinxVerbatim}[commandchars=\\\{\}]
\PYG{k+kn}{import} \PYG{n+nn}{FinanceDataReader} \PYG{k}{as} \PYG{n+nn}{fdr}
\PYG{k+kn}{import} \PYG{n+nn}{matplotlib}\PYG{n+nn}{.}\PYG{n+nn}{pyplot} \PYG{k}{as} \PYG{n+nn}{plt}
\PYG{o}{\PYGZpc{}}\PYG{k}{matplotlib} inline
\PYG{k+kn}{import} \PYG{n+nn}{os}
\PYG{k+kn}{import} \PYG{n+nn}{FinanceDataReader} \PYG{k}{as} \PYG{n+nn}{fdr} 
\PYG{k+kn}{import} \PYG{n+nn}{pandas} \PYG{k}{as} \PYG{n+nn}{pd}
\PYG{k+kn}{import} \PYG{n+nn}{numpy} \PYG{k}{as} \PYG{n+nn}{np}
\PYG{k+kn}{import} \PYG{n+nn}{requests}
\PYG{k+kn}{import} \PYG{n+nn}{bs4}

\PYG{n}{pd}\PYG{o}{.}\PYG{n}{options}\PYG{o}{.}\PYG{n}{display}\PYG{o}{.}\PYG{n}{float\PYGZus{}format} \PYG{o}{=} \PYG{l+s+s1}{\PYGZsq{}}\PYG{l+s+si}{\PYGZob{}:,.3f\PYGZcb{}}\PYG{l+s+s1}{\PYGZsq{}}\PYG{o}{.}\PYG{n}{format}
\end{sphinxVerbatim}

\end{sphinxuseclass}\end{sphinxVerbatimInput}

\end{sphinxuseclass}

\part{FinanceDataReader 로 일봉 데이터 가져오기}
\label{\detokenize{chapter4/4.4.1_Data_Collection:financedatareader}}\label{\detokenize{chapter4/4.4.1_Data_Collection::doc}}
\sphinxAtStartPar
가설 분석과 수익율 예측 모델링은 변동성이 큰 코스닥 종목만을 대상으로 하겠습니다.
가설검정을 위하여 과거 수 개월치의 일봉데이터가 필요합니다. 우선 데이터를 종목별로 가져오기 위해서 FinanceDataReader 의 Stocklisting 메소드에서 코스닥의 종목 코드와 정보를 불러옵니다.

\begin{sphinxuseclass}{cell}\begin{sphinxVerbatimInput}

\begin{sphinxuseclass}{cell_input}
\begin{sphinxVerbatim}[commandchars=\\\{\}]
\PYG{n}{kosdaq\PYGZus{}df} \PYG{o}{=} \PYG{n}{fdr}\PYG{o}{.}\PYG{n}{StockListing}\PYG{p}{(}\PYG{l+s+s1}{\PYGZsq{}}\PYG{l+s+s1}{KOSDAQ}\PYG{l+s+s1}{\PYGZsq{}}\PYG{p}{)}
\PYG{n}{kosdaq\PYGZus{}df}\PYG{o}{.}\PYG{n}{head}\PYG{p}{(}\PYG{p}{)}\PYG{o}{.}\PYG{n}{style}\PYG{o}{.}\PYG{n}{set\PYGZus{}table\PYGZus{}attributes}\PYG{p}{(}\PYG{l+s+s1}{\PYGZsq{}}\PYG{l+s+s1}{style=}\PYG{l+s+s1}{\PYGZdq{}}\PYG{l+s+s1}{font\PYGZhy{}size: 12px}\PYG{l+s+s1}{\PYGZdq{}}\PYG{l+s+s1}{\PYGZsq{}}\PYG{p}{)}
\end{sphinxVerbatim}

\end{sphinxuseclass}\end{sphinxVerbatimInput}
\begin{sphinxVerbatimOutput}

\begin{sphinxuseclass}{cell_output}
\begin{sphinxVerbatim}[commandchars=\\\{\}]
\PYGZlt{}pandas.io.formats.style.Styler at 0x199abda1f40\PYGZgt{}
\end{sphinxVerbatim}

\end{sphinxuseclass}\end{sphinxVerbatimOutput}

\end{sphinxuseclass}
\sphinxAtStartPar
 섹터가 정의되지 않은 종목과 2021년 1월 1일 이후 상장된 종목은 제외하겠습니다. 종 1422 개의 종목이 있습니다. 독자분이 책을 보시는 시점에는 종목 수가 바뀌어 있을 것입니다.kosdaq\_df 에서 필요한 컬럼 ‘Symbol’ 과 ‘Name’ 두 개만 kosdaq\_list 에 저장합니다. 그리고 종목코드 ‘Symbol’ 과 ‘Name’ 을 각 각 ‘code’ 외 ‘name’ 으로 바꿔줍니다. 그리고 나중을 위해서 결과물을 pickle 파일로 저장도 합니다.

\begin{sphinxuseclass}{cell}\begin{sphinxVerbatimInput}

\begin{sphinxuseclass}{cell_input}
\begin{sphinxVerbatim}[commandchars=\\\{\}]
\PYG{n+nb}{print}\PYG{p}{(}\PYG{n}{kosdaq\PYGZus{}df}\PYG{p}{[}\PYG{l+s+s1}{\PYGZsq{}}\PYG{l+s+s1}{Symbol}\PYG{l+s+s1}{\PYGZsq{}}\PYG{p}{]}\PYG{o}{.}\PYG{n}{nunique}\PYG{p}{(}\PYG{p}{)}\PYG{p}{)}

\PYG{n}{c1} \PYG{o}{=} \PYG{p}{(}\PYG{n}{kosdaq\PYGZus{}df}\PYG{p}{[}\PYG{l+s+s1}{\PYGZsq{}}\PYG{l+s+s1}{ListingDate}\PYG{l+s+s1}{\PYGZsq{}}\PYG{p}{]}\PYG{o}{\PYGZgt{}}\PYG{l+s+s1}{\PYGZsq{}}\PYG{l+s+s1}{2021\PYGZhy{}01\PYGZhy{}01}\PYG{l+s+s1}{\PYGZsq{}}\PYG{p}{)} \PYG{c+c1}{\PYGZsh{} 2021년 1월 1일 이후 상장된 종목}
\PYG{n}{c2} \PYG{o}{=} \PYG{p}{(}\PYG{n}{kosdaq\PYGZus{}df}\PYG{p}{[}\PYG{l+s+s1}{\PYGZsq{}}\PYG{l+s+s1}{Sector}\PYG{l+s+s1}{\PYGZsq{}}\PYG{p}{]}\PYG{o}{.}\PYG{n}{isnull}\PYG{p}{(}\PYG{p}{)}\PYG{p}{)} \PYG{c+c1}{\PYGZsh{} 섹터 값이 비어있음}
\PYG{n+nb}{print}\PYG{p}{(}\PYG{n}{kosdaq\PYGZus{}df}\PYG{p}{[}\PYG{o}{\PYGZti{}}\PYG{n}{c1} \PYG{o}{\PYGZam{}} \PYG{o}{\PYGZti{}}\PYG{n}{c2}\PYG{p}{]}\PYG{p}{[}\PYG{l+s+s1}{\PYGZsq{}}\PYG{l+s+s1}{Symbol}\PYG{l+s+s1}{\PYGZsq{}}\PYG{p}{]}\PYG{o}{.}\PYG{n}{nunique}\PYG{p}{(}\PYG{p}{)}\PYG{p}{)}  \PYG{c+c1}{\PYGZsh{} c1 이 아니고 c2 가 아닌 종목의 갯 수}

\PYG{n}{kosdaq\PYGZus{}list} \PYG{o}{=} \PYG{n}{kosdaq\PYGZus{}df}\PYG{p}{[}\PYG{o}{\PYGZti{}}\PYG{n}{c1} \PYG{o}{\PYGZam{}} \PYG{o}{\PYGZti{}}\PYG{n}{c2}\PYG{p}{]}\PYG{p}{[}\PYG{p}{[}\PYG{l+s+s1}{\PYGZsq{}}\PYG{l+s+s1}{Symbol}\PYG{l+s+s1}{\PYGZsq{}}\PYG{p}{,}\PYG{l+s+s1}{\PYGZsq{}}\PYG{l+s+s1}{Name}\PYG{l+s+s1}{\PYGZsq{}}\PYG{p}{,}\PYG{l+s+s1}{\PYGZsq{}}\PYG{l+s+s1}{Sector}\PYG{l+s+s1}{\PYGZsq{}}\PYG{p}{]}\PYG{p}{]}\PYG{o}{.}\PYG{n}{rename}\PYG{p}{(}\PYG{n}{columns}\PYG{o}{=}\PYG{p}{\PYGZob{}}\PYG{l+s+s1}{\PYGZsq{}}\PYG{l+s+s1}{Symbol}\PYG{l+s+s1}{\PYGZsq{}}\PYG{p}{:}\PYG{l+s+s1}{\PYGZsq{}}\PYG{l+s+s1}{code}\PYG{l+s+s1}{\PYGZsq{}}\PYG{p}{,}\PYG{l+s+s1}{\PYGZsq{}}\PYG{l+s+s1}{Name}\PYG{l+s+s1}{\PYGZsq{}}\PYG{p}{:}\PYG{l+s+s1}{\PYGZsq{}}\PYG{l+s+s1}{name}\PYG{l+s+s1}{\PYGZsq{}}\PYG{p}{,}\PYG{l+s+s1}{\PYGZsq{}}\PYG{l+s+s1}{Sector}\PYG{l+s+s1}{\PYGZsq{}}\PYG{p}{:}\PYG{l+s+s1}{\PYGZsq{}}\PYG{l+s+s1}{sector}\PYG{l+s+s1}{\PYGZsq{}}\PYG{p}{\PYGZcb{}}\PYG{p}{)}
\PYG{n}{kosdaq\PYGZus{}list}\PYG{o}{.}\PYG{n}{to\PYGZus{}pickle}\PYG{p}{(}\PYG{l+s+s1}{\PYGZsq{}}\PYG{l+s+s1}{kosdaq\PYGZus{}list.pkl}\PYG{l+s+s1}{\PYGZsq{}}\PYG{p}{)}
\end{sphinxVerbatim}

\end{sphinxuseclass}\end{sphinxVerbatimInput}
\begin{sphinxVerbatimOutput}

\begin{sphinxuseclass}{cell_output}
\begin{sphinxVerbatim}[commandchars=\\\{\}]
1579
1417
\end{sphinxVerbatim}

\end{sphinxuseclass}\end{sphinxVerbatimOutput}

\end{sphinxuseclass}
\sphinxAtStartPar
 저장한 pickle 파일을 읽고, sector 가 몇개나 있는 지 세어봅니다.

\begin{sphinxuseclass}{cell}\begin{sphinxVerbatimInput}

\begin{sphinxuseclass}{cell_input}
\begin{sphinxVerbatim}[commandchars=\\\{\}]
\PYG{n}{kosdaq\PYGZus{}list} \PYG{o}{=} \PYG{n}{pd}\PYG{o}{.}\PYG{n}{read\PYGZus{}pickle}\PYG{p}{(}\PYG{l+s+s1}{\PYGZsq{}}\PYG{l+s+s1}{kosdaq\PYGZus{}list.pkl}\PYG{l+s+s1}{\PYGZsq{}}\PYG{p}{)}
\PYG{n}{kosdaq\PYGZus{}list}\PYG{p}{[}\PYG{l+s+s1}{\PYGZsq{}}\PYG{l+s+s1}{sector}\PYG{l+s+s1}{\PYGZsq{}}\PYG{p}{]}\PYG{o}{.}\PYG{n}{nunique}\PYG{p}{(}\PYG{p}{)}
\end{sphinxVerbatim}

\end{sphinxuseclass}\end{sphinxVerbatimInput}
\begin{sphinxVerbatimOutput}

\begin{sphinxuseclass}{cell_output}
\begin{sphinxVerbatim}[commandchars=\\\{\}]
132
\end{sphinxVerbatim}

\end{sphinxuseclass}\end{sphinxVerbatimOutput}

\end{sphinxuseclass}
\sphinxAtStartPar
 For Loop 에서 kosdaq\_list 의 종목코드와 종목이름을 하나씩 불러서 DataReader 로 2021년 1월 3일부터 2022년 3월 31일까지 일봉데이터를 수집합니다.

\begin{sphinxuseclass}{cell}\begin{sphinxVerbatimInput}

\begin{sphinxuseclass}{cell_input}
\begin{sphinxVerbatim}[commandchars=\\\{\}]
\PYG{n}{price\PYGZus{}data} \PYG{o}{=} \PYG{n}{pd}\PYG{o}{.}\PYG{n}{DataFrame}\PYG{p}{(}\PYG{p}{)}

\PYG{k}{for} \PYG{n}{code}\PYG{p}{,} \PYG{n}{name} \PYG{o+ow}{in} \PYG{n+nb}{zip}\PYG{p}{(}\PYG{n}{kosdaq\PYGZus{}list}\PYG{p}{[}\PYG{l+s+s1}{\PYGZsq{}}\PYG{l+s+s1}{code}\PYG{l+s+s1}{\PYGZsq{}}\PYG{p}{]}\PYG{p}{,} \PYG{n}{kosdaq\PYGZus{}list}\PYG{p}{[}\PYG{l+s+s1}{\PYGZsq{}}\PYG{l+s+s1}{name}\PYG{l+s+s1}{\PYGZsq{}}\PYG{p}{]}\PYG{p}{)}\PYG{p}{:}  \PYG{c+c1}{\PYGZsh{} 코스닥 모든 종목에서 대하여 반복}
    \PYG{n}{daily\PYGZus{}price} \PYG{o}{=} \PYG{n}{fdr}\PYG{o}{.}\PYG{n}{DataReader}\PYG{p}{(}\PYG{n}{code}\PYG{p}{,}  \PYG{n}{start}\PYG{o}{=}\PYG{l+s+s1}{\PYGZsq{}}\PYG{l+s+s1}{2021\PYGZhy{}01\PYGZhy{}03}\PYG{l+s+s1}{\PYGZsq{}}\PYG{p}{,} \PYG{n}{end}\PYG{o}{=}\PYG{l+s+s1}{\PYGZsq{}}\PYG{l+s+s1}{2022\PYGZhy{}03\PYGZhy{}31}\PYG{l+s+s1}{\PYGZsq{}}\PYG{p}{)} \PYG{c+c1}{\PYGZsh{} 종목, 일봉, 데이터 갯수}
    \PYG{n}{daily\PYGZus{}price}\PYG{p}{[}\PYG{l+s+s1}{\PYGZsq{}}\PYG{l+s+s1}{code}\PYG{l+s+s1}{\PYGZsq{}}\PYG{p}{]} \PYG{o}{=} \PYG{n}{code}
    \PYG{n}{daily\PYGZus{}price}\PYG{p}{[}\PYG{l+s+s1}{\PYGZsq{}}\PYG{l+s+s1}{name}\PYG{l+s+s1}{\PYGZsq{}}\PYG{p}{]} \PYG{o}{=} \PYG{n}{name}
    \PYG{n}{price\PYGZus{}data} \PYG{o}{=} \PYG{n}{pd}\PYG{o}{.}\PYG{n}{concat}\PYG{p}{(}\PYG{p}{[}\PYG{n}{price\PYGZus{}data}\PYG{p}{,} \PYG{n}{daily\PYGZus{}price}\PYG{p}{]}\PYG{p}{,} \PYG{n}{axis}\PYG{o}{=}\PYG{l+m+mi}{0}\PYG{p}{)}   

\PYG{n}{price\PYGZus{}data}\PYG{o}{.}\PYG{n}{index}\PYG{o}{.}\PYG{n}{name} \PYG{o}{=} \PYG{l+s+s1}{\PYGZsq{}}\PYG{l+s+s1}{date}\PYG{l+s+s1}{\PYGZsq{}}
\PYG{n}{price\PYGZus{}data}\PYG{o}{.}\PYG{n}{columns}\PYG{o}{=} \PYG{n}{price\PYGZus{}data}\PYG{o}{.}\PYG{n}{columns}\PYG{o}{.}\PYG{n}{str}\PYG{o}{.}\PYG{n}{lower}\PYG{p}{(}\PYG{p}{)} \PYG{c+c1}{\PYGZsh{} 컬럼 이름 소문자로 변경}
\PYG{n}{price\PYGZus{}data}\PYG{o}{.}\PYG{n}{to\PYGZus{}pickle}\PYG{p}{(}\PYG{l+s+s1}{\PYGZsq{}}\PYG{l+s+s1}{stock\PYGZus{}data\PYGZus{}from\PYGZus{}fdr.pkl}\PYG{l+s+s1}{\PYGZsq{}}\PYG{p}{)}
\end{sphinxVerbatim}

\end{sphinxuseclass}\end{sphinxVerbatimInput}

\end{sphinxuseclass}
\sphinxAtStartPar
 저장한 pickle 파일을 다시 읽어 첫 5 라인을 head 메소드로 찍어보면 아래와 같습니다. 여기서 date가 인덱스로 처리되어 있다는 것을 기억해주시면 좋습니다. 타이핑 편의를 위해 컬럼이름을 소문자료 변경하겠습니다.

\begin{sphinxuseclass}{cell}\begin{sphinxVerbatimInput}

\begin{sphinxuseclass}{cell_input}
\begin{sphinxVerbatim}[commandchars=\\\{\}]
\PYG{n}{price\PYGZus{}data} \PYG{o}{=} \PYG{n}{pd}\PYG{o}{.}\PYG{n}{read\PYGZus{}pickle}\PYG{p}{(}\PYG{l+s+s1}{\PYGZsq{}}\PYG{l+s+s1}{stock\PYGZus{}data\PYGZus{}from\PYGZus{}fdr.pkl}\PYG{l+s+s1}{\PYGZsq{}}\PYG{p}{)}
\PYG{n}{price\PYGZus{}data}\PYG{o}{.}\PYG{n}{head}\PYG{p}{(}\PYG{p}{)}\PYG{o}{.}\PYG{n}{style}\PYG{o}{.}\PYG{n}{set\PYGZus{}table\PYGZus{}attributes}\PYG{p}{(}\PYG{l+s+s1}{\PYGZsq{}}\PYG{l+s+s1}{style=}\PYG{l+s+s1}{\PYGZdq{}}\PYG{l+s+s1}{font\PYGZhy{}size: 12px}\PYG{l+s+s1}{\PYGZdq{}}\PYG{l+s+s1}{\PYGZsq{}}\PYG{p}{)}
\end{sphinxVerbatim}

\end{sphinxuseclass}\end{sphinxVerbatimInput}
\begin{sphinxVerbatimOutput}

\begin{sphinxuseclass}{cell_output}
\begin{sphinxVerbatim}[commandchars=\\\{\}]
\PYGZlt{}pandas.io.formats.style.Styler at 0x199b0f463a0\PYGZgt{}
\end{sphinxVerbatim}

\end{sphinxuseclass}\end{sphinxVerbatimOutput}

\end{sphinxuseclass}
\sphinxAtStartPar
 몇 개의 종목이 있고, 각 종목별 일봉의 갯 수 가 몇 개인지 확인해 보겠습니다. 종목 수는 1417 개, 307 개의 일봉이 있습니다.

\begin{sphinxuseclass}{cell}\begin{sphinxVerbatimInput}

\begin{sphinxuseclass}{cell_input}
\begin{sphinxVerbatim}[commandchars=\\\{\}]
\PYG{n+nb}{print}\PYG{p}{(}\PYG{n}{price\PYGZus{}data}\PYG{p}{[}\PYG{l+s+s1}{\PYGZsq{}}\PYG{l+s+s1}{code}\PYG{l+s+s1}{\PYGZsq{}}\PYG{p}{]}\PYG{o}{.}\PYG{n}{nunique}\PYG{p}{(}\PYG{p}{)}\PYG{p}{)}
\PYG{n+nb}{print}\PYG{p}{(}\PYG{n}{price\PYGZus{}data}\PYG{o}{.}\PYG{n}{groupby}\PYG{p}{(}\PYG{l+s+s1}{\PYGZsq{}}\PYG{l+s+s1}{code}\PYG{l+s+s1}{\PYGZsq{}}\PYG{p}{)}\PYG{p}{[}\PYG{l+s+s1}{\PYGZsq{}}\PYG{l+s+s1}{close}\PYG{l+s+s1}{\PYGZsq{}}\PYG{p}{]}\PYG{o}{.}\PYG{n}{count}\PYG{p}{(}\PYG{p}{)}\PYG{o}{.}\PYG{n}{agg}\PYG{p}{(}\PYG{p}{[}\PYG{l+s+s1}{\PYGZsq{}}\PYG{l+s+s1}{min}\PYG{l+s+s1}{\PYGZsq{}}\PYG{p}{,}\PYG{l+s+s1}{\PYGZsq{}}\PYG{l+s+s1}{max}\PYG{l+s+s1}{\PYGZsq{}}\PYG{p}{]}\PYG{p}{)}\PYG{p}{)}
\end{sphinxVerbatim}

\end{sphinxuseclass}\end{sphinxVerbatimInput}
\begin{sphinxVerbatimOutput}

\begin{sphinxuseclass}{cell_output}
\begin{sphinxVerbatim}[commandchars=\\\{\}]
1417
min    307
max    307
Name: close, dtype: int64
\end{sphinxVerbatim}

\end{sphinxuseclass}\end{sphinxVerbatimOutput}

\end{sphinxuseclass}



\part{네이버 증권 웹크롤링으로 일봉 가져오기}
\label{\detokenize{chapter4/4.4.1_Data_Collection:id1}}
\sphinxAtStartPar
이 번에는 네이버 증권 차트 \sphinxstyleemphasis{(네이버 차트 예시 필요)} 에서 데이터를 가져오는 방법도 시도해 보겠습니다. 웹 크롤링은 코드가 복잡합니다. 첫 번째 방법인 FinanceDataReader 로 추출하는 방법을 추천드립니다.다시 pickle 파일을 읽습니다. make\_price\_data 함수는 ‘종목’, ‘추출단위’, ‘데이터 건수’ 를 인자로 네이버증권에서 데이터를 가져오는 함수입니다. 인자는 작은 따옴표에 넣어야 합니다. 셀트리온 헬스케어(091990) 의 일봉 데이터를 최근 300 일 가져오고 싶다면  make\_price\_data(‘091990’, ‘day’, ‘300’) 와 같이 호출합니다. 이 함수를 for 문을 이용해 모든 코스닥 종목에서 대하여 호출하고, 각 결과를 price\_data 라는 데이터프레임에 담습니다.
for 문을 돌리고 결과를 concat 함수로 연속으로 저장하는 방법은 자주 활용되는 기법입니다.

\begin{sphinxuseclass}{cell}\begin{sphinxVerbatimInput}

\begin{sphinxuseclass}{cell_input}
\begin{sphinxVerbatim}[commandchars=\\\{\}]
\PYG{c+c1}{\PYGZsh{} 네이버 증권 차트에서 데이터 크롤링}

\PYG{n}{kosdaq\PYGZus{}list} \PYG{o}{=} \PYG{n}{pd}\PYG{o}{.}\PYG{n}{read\PYGZus{}pickle}\PYG{p}{(}\PYG{l+s+s1}{\PYGZsq{}}\PYG{l+s+s1}{kosdaq\PYGZus{}list.pkl}\PYG{l+s+s1}{\PYGZsq{}}\PYG{p}{)}

\PYG{k}{def} \PYG{n+nf}{make\PYGZus{}price\PYGZus{}data}\PYG{p}{(}\PYG{n}{code}\PYG{p}{,} \PYG{n}{name}\PYG{p}{,} \PYG{n}{timeframe}\PYG{p}{,} \PYG{n}{count}\PYG{p}{)}\PYG{p}{:}
    \PYG{n}{url} \PYG{o}{=} \PYG{l+s+s1}{\PYGZsq{}}\PYG{l+s+s1}{https://fchart.stock.naver.com/sise.nhn?symbol=}\PYG{l+s+s1}{\PYGZsq{}} \PYG{o}{+} \PYG{n}{code} \PYG{o}{+} \PYG{l+s+s1}{\PYGZsq{}}\PYG{l+s+s1}{\PYGZam{}timeframe=}\PYG{l+s+s1}{\PYGZsq{}} \PYG{o}{+} \PYG{n}{timeframe} \PYG{o}{+} \PYG{l+s+s1}{\PYGZsq{}}\PYG{l+s+s1}{\PYGZam{}count=}\PYG{l+s+s1}{\PYGZsq{}} \PYG{o}{+} \PYG{n}{count} \PYG{o}{+} \PYG{l+s+s1}{\PYGZsq{}}\PYG{l+s+s1}{\PYGZam{}requestType=0}\PYG{l+s+s1}{\PYGZsq{}}
    \PYG{n}{price\PYGZus{}data} \PYG{o}{=} \PYG{n}{requests}\PYG{o}{.}\PYG{n}{get}\PYG{p}{(}\PYG{n}{url}\PYG{p}{)}
    \PYG{n}{price\PYGZus{}data\PYGZus{}bs} \PYG{o}{=} \PYG{n}{bs4}\PYG{o}{.}\PYG{n}{BeautifulSoup}\PYG{p}{(}\PYG{n}{price\PYGZus{}data}\PYG{o}{.}\PYG{n}{text}\PYG{p}{,} \PYG{l+s+s1}{\PYGZsq{}}\PYG{l+s+s1}{lxml}\PYG{l+s+s1}{\PYGZsq{}}\PYG{p}{)}
    \PYG{n}{item\PYGZus{}list} \PYG{o}{=} \PYG{n}{price\PYGZus{}data\PYGZus{}bs}\PYG{o}{.}\PYG{n}{find\PYGZus{}all}\PYG{p}{(}\PYG{l+s+s1}{\PYGZsq{}}\PYG{l+s+s1}{item}\PYG{l+s+s1}{\PYGZsq{}}\PYG{p}{)}

    \PYG{n}{date\PYGZus{}list} \PYG{o}{=} \PYG{p}{[}\PYG{p}{]} 
    \PYG{n}{open\PYGZus{}list} \PYG{o}{=} \PYG{p}{[}\PYG{p}{]}
    \PYG{n}{high\PYGZus{}list} \PYG{o}{=} \PYG{p}{[}\PYG{p}{]}
    \PYG{n}{low\PYGZus{}list} \PYG{o}{=} \PYG{p}{[}\PYG{p}{]}
    \PYG{n}{close\PYGZus{}list} \PYG{o}{=} \PYG{p}{[}\PYG{p}{]}
    \PYG{n}{trade\PYGZus{}list} \PYG{o}{=} \PYG{p}{[}\PYG{p}{]}

    \PYG{k}{for} \PYG{n}{item} \PYG{o+ow}{in} \PYG{n}{item\PYGZus{}list}\PYG{p}{:}
        \PYG{n}{data} \PYG{o}{=} \PYG{n}{item}\PYG{p}{[}\PYG{l+s+s1}{\PYGZsq{}}\PYG{l+s+s1}{data}\PYG{l+s+s1}{\PYGZsq{}}\PYG{p}{]}\PYG{o}{.}\PYG{n}{split}\PYG{p}{(}\PYG{l+s+s1}{\PYGZsq{}}\PYG{l+s+s1}{|}\PYG{l+s+s1}{\PYGZsq{}}\PYG{p}{)}
        \PYG{n}{date\PYGZus{}list}\PYG{o}{.}\PYG{n}{append}\PYG{p}{(}\PYG{n}{data}\PYG{p}{[}\PYG{l+m+mi}{0}\PYG{p}{]}\PYG{p}{)}
        \PYG{n}{open\PYGZus{}list}\PYG{o}{.}\PYG{n}{append}\PYG{p}{(}\PYG{n}{data}\PYG{p}{[}\PYG{l+m+mi}{1}\PYG{p}{]}\PYG{p}{)}
        \PYG{n}{high\PYGZus{}list}\PYG{o}{.}\PYG{n}{append}\PYG{p}{(}\PYG{n}{data}\PYG{p}{[}\PYG{l+m+mi}{2}\PYG{p}{]}\PYG{p}{)}
        \PYG{n}{low\PYGZus{}list}\PYG{o}{.}\PYG{n}{append}\PYG{p}{(}\PYG{n}{data}\PYG{p}{[}\PYG{l+m+mi}{3}\PYG{p}{]}\PYG{p}{)}
        \PYG{n}{close\PYGZus{}list}\PYG{o}{.}\PYG{n}{append}\PYG{p}{(}\PYG{n}{data}\PYG{p}{[}\PYG{l+m+mi}{4}\PYG{p}{]}\PYG{p}{)}
        \PYG{n}{trade\PYGZus{}list}\PYG{o}{.}\PYG{n}{append}\PYG{p}{(}\PYG{n}{data}\PYG{p}{[}\PYG{l+m+mi}{5}\PYG{p}{]}\PYG{p}{)}        

    \PYG{n}{price\PYGZus{}df} \PYG{o}{=} \PYG{n}{pd}\PYG{o}{.}\PYG{n}{DataFrame}\PYG{p}{(}\PYG{p}{\PYGZob{}}\PYG{l+s+s1}{\PYGZsq{}}\PYG{l+s+s1}{open}\PYG{l+s+s1}{\PYGZsq{}}\PYG{p}{:} \PYG{n}{open\PYGZus{}list}\PYG{p}{,} \PYG{l+s+s1}{\PYGZsq{}}\PYG{l+s+s1}{high}\PYG{l+s+s1}{\PYGZsq{}}\PYG{p}{:} \PYG{n}{high\PYGZus{}list}\PYG{p}{,} \PYG{l+s+s1}{\PYGZsq{}}\PYG{l+s+s1}{low}\PYG{l+s+s1}{\PYGZsq{}}\PYG{p}{:} \PYG{n}{low\PYGZus{}list}\PYG{p}{,} \PYG{l+s+s1}{\PYGZsq{}}\PYG{l+s+s1}{close}\PYG{l+s+s1}{\PYGZsq{}}\PYG{p}{:} \PYG{n}{close\PYGZus{}list}\PYG{p}{,} \PYG{l+s+s1}{\PYGZsq{}}\PYG{l+s+s1}{volume}\PYG{l+s+s1}{\PYGZsq{}}\PYG{p}{:} \PYG{n}{trade\PYGZus{}list}\PYG{p}{\PYGZcb{}}\PYG{p}{,} \PYG{n}{index}\PYG{o}{=}\PYG{n}{date\PYGZus{}list}\PYG{p}{)}            
    \PYG{n}{price\PYGZus{}df}\PYG{p}{[}\PYG{l+s+s1}{\PYGZsq{}}\PYG{l+s+s1}{code}\PYG{l+s+s1}{\PYGZsq{}}\PYG{p}{]} \PYG{o}{=} \PYG{n}{code}
    \PYG{n}{price\PYGZus{}df}\PYG{p}{[}\PYG{l+s+s1}{\PYGZsq{}}\PYG{l+s+s1}{name}\PYG{l+s+s1}{\PYGZsq{}}\PYG{p}{]} \PYG{o}{=} \PYG{n}{name}
    \PYG{n}{num\PYGZus{}vars} \PYG{o}{=} \PYG{p}{[}\PYG{l+s+s1}{\PYGZsq{}}\PYG{l+s+s1}{open}\PYG{l+s+s1}{\PYGZsq{}}\PYG{p}{,}\PYG{l+s+s1}{\PYGZsq{}}\PYG{l+s+s1}{high}\PYG{l+s+s1}{\PYGZsq{}}\PYG{p}{,}\PYG{l+s+s1}{\PYGZsq{}}\PYG{l+s+s1}{low}\PYG{l+s+s1}{\PYGZsq{}}\PYG{p}{,}\PYG{l+s+s1}{\PYGZsq{}}\PYG{l+s+s1}{close}\PYG{l+s+s1}{\PYGZsq{}}\PYG{p}{,}\PYG{l+s+s1}{\PYGZsq{}}\PYG{l+s+s1}{volume}\PYG{l+s+s1}{\PYGZsq{}}\PYG{p}{]}
    \PYG{n}{char\PYGZus{}vars} \PYG{o}{=} \PYG{p}{[}\PYG{l+s+s1}{\PYGZsq{}}\PYG{l+s+s1}{code}\PYG{l+s+s1}{\PYGZsq{}}\PYG{p}{,}\PYG{l+s+s1}{\PYGZsq{}}\PYG{l+s+s1}{name}\PYG{l+s+s1}{\PYGZsq{}}\PYG{p}{]}
    \PYG{n}{price\PYGZus{}df} \PYG{o}{=} \PYG{n}{price\PYGZus{}df}\PYG{o}{.}\PYG{n}{reindex}\PYG{p}{(}\PYG{n}{columns} \PYG{o}{=} \PYG{n}{char\PYGZus{}vars} \PYG{o}{+} \PYG{n}{num\PYGZus{}vars}\PYG{p}{)}

    \PYG{k}{for} \PYG{n}{var} \PYG{o+ow}{in} \PYG{n}{num\PYGZus{}vars}\PYG{p}{:}
        \PYG{n}{price\PYGZus{}df}\PYG{p}{[}\PYG{n}{var}\PYG{p}{]} \PYG{o}{=} \PYG{n}{pd}\PYG{o}{.}\PYG{n}{to\PYGZus{}numeric}\PYG{p}{(}\PYG{n}{price\PYGZus{}df}\PYG{p}{[}\PYG{n}{var}\PYG{p}{]}\PYG{p}{,} \PYG{n}{errors}\PYG{o}{=}\PYG{l+s+s1}{\PYGZsq{}}\PYG{l+s+s1}{coerce}\PYG{l+s+s1}{\PYGZsq{}}\PYG{p}{)}

    \PYG{n}{price\PYGZus{}df}\PYG{o}{.}\PYG{n}{index} \PYG{o}{=} \PYG{n}{pd}\PYG{o}{.}\PYG{n}{to\PYGZus{}datetime}\PYG{p}{(}\PYG{n}{price\PYGZus{}df}\PYG{o}{.}\PYG{n}{index}\PYG{p}{,} \PYG{n}{errors}\PYG{o}{=}\PYG{l+s+s1}{\PYGZsq{}}\PYG{l+s+s1}{coerce}\PYG{l+s+s1}{\PYGZsq{}}\PYG{p}{)}

    \PYG{k}{return} \PYG{n}{price\PYGZus{}df}

\PYG{n}{price\PYGZus{}data} \PYG{o}{=} \PYG{n}{pd}\PYG{o}{.}\PYG{n}{DataFrame}\PYG{p}{(}\PYG{p}{)}

\PYG{k}{for} \PYG{n}{code}\PYG{p}{,} \PYG{n}{name} \PYG{o+ow}{in} \PYG{n+nb}{zip}\PYG{p}{(}\PYG{n}{kosdaq\PYGZus{}list}\PYG{p}{[}\PYG{l+s+s1}{\PYGZsq{}}\PYG{l+s+s1}{code}\PYG{l+s+s1}{\PYGZsq{}}\PYG{p}{]}\PYG{p}{,} \PYG{n}{kosdaq\PYGZus{}list}\PYG{p}{[}\PYG{l+s+s1}{\PYGZsq{}}\PYG{l+s+s1}{name}\PYG{l+s+s1}{\PYGZsq{}}\PYG{p}{]}\PYG{p}{)}\PYG{p}{:}  \PYG{c+c1}{\PYGZsh{} 코스닥 모든 종목에서 대하여 반복}
    \PYG{n}{daily\PYGZus{}price} \PYG{o}{=} \PYG{n}{make\PYGZus{}price\PYGZus{}data}\PYG{p}{(}\PYG{n}{code}\PYG{p}{,} \PYG{n}{name}\PYG{p}{,} \PYG{l+s+s1}{\PYGZsq{}}\PYG{l+s+s1}{day}\PYG{l+s+s1}{\PYGZsq{}}\PYG{p}{,} \PYG{l+s+s1}{\PYGZsq{}}\PYG{l+s+s1}{307}\PYG{l+s+s1}{\PYGZsq{}}\PYG{p}{)} \PYG{c+c1}{\PYGZsh{} 종목, 일봉, 데이터 갯수}
    \PYG{n}{price\PYGZus{}data} \PYG{o}{=} \PYG{n}{pd}\PYG{o}{.}\PYG{n}{concat}\PYG{p}{(}\PYG{p}{[}\PYG{n}{price\PYGZus{}data}\PYG{p}{,} \PYG{n}{daily\PYGZus{}price}\PYG{p}{]}\PYG{p}{,} \PYG{n}{axis}\PYG{o}{=}\PYG{l+m+mi}{0}\PYG{p}{)}   

\PYG{n}{price\PYGZus{}data}\PYG{o}{.}\PYG{n}{index}\PYG{o}{.}\PYG{n}{name} \PYG{o}{=} \PYG{l+s+s1}{\PYGZsq{}}\PYG{l+s+s1}{date}\PYG{l+s+s1}{\PYGZsq{}}
\PYG{n}{price\PYGZus{}data}\PYG{o}{.}\PYG{n}{to\PYGZus{}pickle}\PYG{p}{(}\PYG{l+s+s1}{\PYGZsq{}}\PYG{l+s+s1}{stock\PYGZus{}data\PYGZus{}from\PYGZus{}naver.pkl}\PYG{l+s+s1}{\PYGZsq{}}\PYG{p}{)}
\end{sphinxVerbatim}

\end{sphinxuseclass}\end{sphinxVerbatimInput}

\end{sphinxuseclass}
\sphinxAtStartPar
 저장한 pickle 파일을 다시 읽어 첫 5 라인을 head 메소드로 찍어보면 아래와 같습니다. 여기서 date가 인덱스로 처리되어 있다는 것을 기억해주시면 좋습니다.

\begin{sphinxuseclass}{cell}\begin{sphinxVerbatimInput}

\begin{sphinxuseclass}{cell_input}
\begin{sphinxVerbatim}[commandchars=\\\{\}]
\PYG{n}{price\PYGZus{}data} \PYG{o}{=} \PYG{n}{pd}\PYG{o}{.}\PYG{n}{read\PYGZus{}pickle}\PYG{p}{(}\PYG{l+s+s1}{\PYGZsq{}}\PYG{l+s+s1}{stock\PYGZus{}data\PYGZus{}from\PYGZus{}naver.pkl}\PYG{l+s+s1}{\PYGZsq{}}\PYG{p}{)}
\PYG{n}{price\PYGZus{}data}\PYG{o}{.}\PYG{n}{head}\PYG{p}{(}\PYG{p}{)}\PYG{o}{.}\PYG{n}{style}\PYG{o}{.}\PYG{n}{set\PYGZus{}table\PYGZus{}attributes}\PYG{p}{(}\PYG{l+s+s1}{\PYGZsq{}}\PYG{l+s+s1}{style=}\PYG{l+s+s1}{\PYGZdq{}}\PYG{l+s+s1}{font\PYGZhy{}size: 12px}\PYG{l+s+s1}{\PYGZdq{}}\PYG{l+s+s1}{\PYGZsq{}}\PYG{p}{)}
\end{sphinxVerbatim}

\end{sphinxuseclass}\end{sphinxVerbatimInput}
\begin{sphinxVerbatimOutput}

\begin{sphinxuseclass}{cell_output}
\begin{sphinxVerbatim}[commandchars=\\\{\}]
\PYGZlt{}pandas.io.formats.style.Styler at 0x1c862384a90\PYGZgt{}
\end{sphinxVerbatim}

\end{sphinxuseclass}\end{sphinxVerbatimOutput}

\end{sphinxuseclass}
\sphinxAtStartPar
 몇 개의 종목이 있고, 각 종목별 일봉의 갯 수 가 몇 개인지 확인해 보겠습니다. 종목 수는 1422 개, 307 개의 일봉이 있습니다.

\begin{sphinxuseclass}{cell}\begin{sphinxVerbatimInput}

\begin{sphinxuseclass}{cell_input}
\begin{sphinxVerbatim}[commandchars=\\\{\}]
\PYG{n+nb}{print}\PYG{p}{(}\PYG{n}{price\PYGZus{}data}\PYG{p}{[}\PYG{l+s+s1}{\PYGZsq{}}\PYG{l+s+s1}{code}\PYG{l+s+s1}{\PYGZsq{}}\PYG{p}{]}\PYG{o}{.}\PYG{n}{nunique}\PYG{p}{(}\PYG{p}{)}\PYG{p}{)}
\PYG{n+nb}{print}\PYG{p}{(}\PYG{n}{price\PYGZus{}data}\PYG{o}{.}\PYG{n}{groupby}\PYG{p}{(}\PYG{l+s+s1}{\PYGZsq{}}\PYG{l+s+s1}{code}\PYG{l+s+s1}{\PYGZsq{}}\PYG{p}{)}\PYG{p}{[}\PYG{l+s+s1}{\PYGZsq{}}\PYG{l+s+s1}{close}\PYG{l+s+s1}{\PYGZsq{}}\PYG{p}{]}\PYG{o}{.}\PYG{n}{count}\PYG{p}{(}\PYG{p}{)}\PYG{o}{.}\PYG{n}{agg}\PYG{p}{(}\PYG{p}{[}\PYG{l+s+s1}{\PYGZsq{}}\PYG{l+s+s1}{min}\PYG{l+s+s1}{\PYGZsq{}}\PYG{p}{,}\PYG{l+s+s1}{\PYGZsq{}}\PYG{l+s+s1}{max}\PYG{l+s+s1}{\PYGZsq{}}\PYG{p}{]}\PYG{p}{)}\PYG{p}{)}
\end{sphinxVerbatim}

\end{sphinxuseclass}\end{sphinxVerbatimInput}
\begin{sphinxVerbatimOutput}

\begin{sphinxuseclass}{cell_output}
\begin{sphinxVerbatim}[commandchars=\\\{\}]
1417
min    307
max    307
Name: close, dtype: int64
\end{sphinxVerbatim}

\end{sphinxuseclass}\end{sphinxVerbatimOutput}

\end{sphinxuseclass}
\begin{sphinxuseclass}{cell}\begin{sphinxVerbatimInput}

\begin{sphinxuseclass}{cell_input}
\begin{sphinxVerbatim}[commandchars=\\\{\}]
\PYG{k+kn}{import} \PYG{n+nn}{FinanceDataReader} \PYG{k}{as} \PYG{n+nn}{fdr}
\PYG{k+kn}{import} \PYG{n+nn}{pandas} \PYG{k}{as} \PYG{n+nn}{pd}
\PYG{k+kn}{import} \PYG{n+nn}{matplotlib}\PYG{n+nn}{.}\PYG{n+nn}{pyplot} \PYG{k}{as} \PYG{n+nn}{plt}
\PYG{o}{\PYGZpc{}}\PYG{k}{matplotlib} inline
\PYG{n}{pd}\PYG{o}{.}\PYG{n}{options}\PYG{o}{.}\PYG{n}{display}\PYG{o}{.}\PYG{n}{float\PYGZus{}format} \PYG{o}{=} \PYG{l+s+s1}{\PYGZsq{}}\PYG{l+s+si}{\PYGZob{}:,.3f\PYGZcb{}}\PYG{l+s+s1}{\PYGZsq{}}\PYG{o}{.}\PYG{n}{format}
\end{sphinxVerbatim}

\end{sphinxuseclass}\end{sphinxVerbatimInput}

\end{sphinxuseclass}

\part{코스닥 인덱스 데이터}
\label{\detokenize{chapter4/4.4.2_Data_Collection:id1}}\label{\detokenize{chapter4/4.4.2_Data_Collection::doc}}
\sphinxAtStartPar
코스닥 인덱스 데이터는 FinanceDataReader 로 데이터를 수집해 보겠습니다.
사용법에 대한 설명은 아래 링크에 자세하게 되어 있습니다.
\sphinxurl{https://financedata.github.io/posts/finance-data-reader-users-guide.html}

\sphinxAtStartPar
FinanceDataReader 는 국내 주식 데이터 뿐만 아니라 해외 데이터도 수집이 가능합니다. 환율, 암호화폐 등의 데이터도 제공됩니다.
이성준님이 개발해서 무료로 제공해 주시는 파이썬 라이브러리입니다. 금융데이터를 쉽게 수집할 수 있게 해 주신 이성준님께 다시 한 번 깊은 감사를 드립니다.

\sphinxAtStartPar
FinanceDataReader 를 fdr 이름으로 import 하시고, fdr.DataReader 함수에서 KQ11 를 호출하시면 결과값을 얻을 수 있습니다.
fdr.DataReader(‘KQ11’, ‘2021’) 에서 ‘KQ11’ 는 코스닥 지수 종목을 의미하고 ‘2021’ 은 2021 년부터 데이터를 가져오라는 뜻 입니다.

\sphinxAtStartPar
Pandas 에서 제공하는 plot 를 이용하여 2021년 부터 코스닥 지수 시계열 데이터를 그려보았습니다.

\begin{sphinxuseclass}{cell}\begin{sphinxVerbatimInput}

\begin{sphinxuseclass}{cell_input}
\begin{sphinxVerbatim}[commandchars=\\\{\}]
\PYG{n}{kosdaq\PYGZus{}index} \PYG{o}{=} \PYG{n}{fdr}\PYG{o}{.}\PYG{n}{DataReader}\PYG{p}{(}\PYG{l+s+s1}{\PYGZsq{}}\PYG{l+s+s1}{KQ11}\PYG{l+s+s1}{\PYGZsq{}}\PYG{p}{,} \PYG{l+s+s1}{\PYGZsq{}}\PYG{l+s+s1}{2021}\PYG{l+s+s1}{\PYGZsq{}}\PYG{p}{)} \PYG{c+c1}{\PYGZsh{} 데이터 호출}
\PYG{n}{kosdaq\PYGZus{}index}\PYG{o}{.}\PYG{n}{columns} \PYG{o}{=} \PYG{p}{[}\PYG{l+s+s1}{\PYGZsq{}}\PYG{l+s+s1}{close}\PYG{l+s+s1}{\PYGZsq{}}\PYG{p}{,}\PYG{l+s+s1}{\PYGZsq{}}\PYG{l+s+s1}{open}\PYG{l+s+s1}{\PYGZsq{}}\PYG{p}{,}\PYG{l+s+s1}{\PYGZsq{}}\PYG{l+s+s1}{high}\PYG{l+s+s1}{\PYGZsq{}}\PYG{p}{,}\PYG{l+s+s1}{\PYGZsq{}}\PYG{l+s+s1}{low}\PYG{l+s+s1}{\PYGZsq{}}\PYG{p}{,}\PYG{l+s+s1}{\PYGZsq{}}\PYG{l+s+s1}{volume}\PYG{l+s+s1}{\PYGZsq{}}\PYG{p}{,}\PYG{l+s+s1}{\PYGZsq{}}\PYG{l+s+s1}{change}\PYG{l+s+s1}{\PYGZsq{}}\PYG{p}{]} \PYG{c+c1}{\PYGZsh{} 컬럼명 변경}
\PYG{n}{kosdaq\PYGZus{}index}\PYG{o}{.}\PYG{n}{index}\PYG{o}{.}\PYG{n}{name}\PYG{o}{=}\PYG{l+s+s1}{\PYGZsq{}}\PYG{l+s+s1}{date}\PYG{l+s+s1}{\PYGZsq{}} \PYG{c+c1}{\PYGZsh{} 인덱스 이름 생성}
\PYG{n}{kosdaq\PYGZus{}index}\PYG{o}{.}\PYG{n}{sort\PYGZus{}index}\PYG{p}{(}\PYG{n}{inplace}\PYG{o}{=}\PYG{k+kc}{True}\PYG{p}{)} \PYG{c+c1}{\PYGZsh{} 인덱스(날짜) 로 정렬 }
\PYG{n}{kosdaq\PYGZus{}index}\PYG{p}{[}\PYG{l+s+s1}{\PYGZsq{}}\PYG{l+s+s1}{kosdaq\PYGZus{}return}\PYG{l+s+s1}{\PYGZsq{}}\PYG{p}{]} \PYG{o}{=} \PYG{n}{kosdaq\PYGZus{}index}\PYG{p}{[}\PYG{l+s+s1}{\PYGZsq{}}\PYG{l+s+s1}{close}\PYG{l+s+s1}{\PYGZsq{}}\PYG{p}{]}\PYG{o}{/}\PYG{n}{kosdaq\PYGZus{}index}\PYG{p}{[}\PYG{l+s+s1}{\PYGZsq{}}\PYG{l+s+s1}{close}\PYG{l+s+s1}{\PYGZsq{}}\PYG{p}{]}\PYG{o}{.}\PYG{n}{shift}\PYG{p}{(}\PYG{l+m+mi}{1}\PYG{p}{)} \PYG{c+c1}{\PYGZsh{} 수익율 : 전 날 종가대비 당일 종가}
\PYG{n}{kosdaq\PYGZus{}index}\PYG{o}{.}\PYG{n}{to\PYGZus{}pickle}\PYG{p}{(}\PYG{l+s+s1}{\PYGZsq{}}\PYG{l+s+s1}{kosdaq\PYGZus{}index.pkl}\PYG{l+s+s1}{\PYGZsq{}}\PYG{p}{)} \PYG{c+c1}{\PYGZsh{} 피클로 저장}

\PYG{c+c1}{\PYGZsh{} 차트 생성}
\PYG{n}{kosdaq\PYGZus{}index}\PYG{p}{[}\PYG{l+s+s1}{\PYGZsq{}}\PYG{l+s+s1}{close}\PYG{l+s+s1}{\PYGZsq{}}\PYG{p}{]}\PYG{o}{.}\PYG{n}{plot}\PYG{p}{(}\PYG{n}{figsize}\PYG{o}{=}\PYG{p}{(}\PYG{l+m+mi}{20}\PYG{p}{,}\PYG{l+m+mi}{5}\PYG{p}{)}\PYG{p}{)}
\PYG{n}{plt}\PYG{o}{.}\PYG{n}{title}\PYG{p}{(}\PYG{l+s+s1}{\PYGZsq{}}\PYG{l+s+s1}{KOSDAQ Index}\PYG{l+s+s1}{\PYGZsq{}}\PYG{p}{)}
\end{sphinxVerbatim}

\end{sphinxuseclass}\end{sphinxVerbatimInput}
\begin{sphinxVerbatimOutput}

\begin{sphinxuseclass}{cell_output}
\begin{sphinxVerbatim}[commandchars=\\\{\}]
Text(0.5, 1.0, \PYGZsq{}KOSDAQ Index\PYGZsq{})
\end{sphinxVerbatim}

\noindent\sphinxincludegraphics{{4.4.2_Data_Collection_2_1}.png}

\end{sphinxuseclass}\end{sphinxVerbatimOutput}

\end{sphinxuseclass}
\sphinxAtStartPar
일별 수익율 그래프도 함 그려보겠습니다. 2021년 3월부터 2021년 8월까지는 수익율의 변동성이 비교적 적어보입니다.

\begin{sphinxuseclass}{cell}\begin{sphinxVerbatimInput}

\begin{sphinxuseclass}{cell_input}
\begin{sphinxVerbatim}[commandchars=\\\{\}]
\PYG{c+c1}{\PYGZsh{} 차트 생성}
\PYG{n}{kosdaq\PYGZus{}index}\PYG{p}{[}\PYG{l+s+s1}{\PYGZsq{}}\PYG{l+s+s1}{kosdaq\PYGZus{}return}\PYG{l+s+s1}{\PYGZsq{}}\PYG{p}{]}\PYG{o}{.}\PYG{n}{plot}\PYG{p}{(}\PYG{n}{figsize}\PYG{o}{=}\PYG{p}{(}\PYG{l+m+mi}{20}\PYG{p}{,}\PYG{l+m+mi}{5}\PYG{p}{)}\PYG{p}{,} \PYG{n}{color}\PYG{o}{=}\PYG{l+s+s1}{\PYGZsq{}}\PYG{l+s+s1}{orangered}\PYG{l+s+s1}{\PYGZsq{}}\PYG{p}{,} \PYG{n}{style}\PYG{o}{=}\PYG{l+s+s1}{\PYGZsq{}}\PYG{l+s+s1}{\PYGZhy{}\PYGZhy{}}\PYG{l+s+s1}{\PYGZsq{}}\PYG{p}{)}
\PYG{n}{plt}\PYG{o}{.}\PYG{n}{title}\PYG{p}{(}\PYG{l+s+s1}{\PYGZsq{}}\PYG{l+s+s1}{KOSDAQ Index Daily Return}\PYG{l+s+s1}{\PYGZsq{}}\PYG{p}{)}
\end{sphinxVerbatim}

\end{sphinxuseclass}\end{sphinxVerbatimInput}
\begin{sphinxVerbatimOutput}

\begin{sphinxuseclass}{cell_output}
\begin{sphinxVerbatim}[commandchars=\\\{\}]
Text(0.5, 1.0, \PYGZsq{}KOSDAQ Index Daily Return\PYGZsq{})
\end{sphinxVerbatim}

\noindent\sphinxincludegraphics{{4.4.2_Data_Collection_4_1}.png}

\end{sphinxuseclass}\end{sphinxVerbatimOutput}

\end{sphinxuseclass}
\sphinxAtStartPar
저장된 Pickle 파일을 읽어서 첫 5 행 출력해 봅니다.

\begin{sphinxuseclass}{cell}\begin{sphinxVerbatimInput}

\begin{sphinxuseclass}{cell_input}
\begin{sphinxVerbatim}[commandchars=\\\{\}]
\PYG{n}{kosdaq\PYGZus{}index} \PYG{o}{=} \PYG{n}{pd}\PYG{o}{.}\PYG{n}{read\PYGZus{}pickle}\PYG{p}{(}\PYG{l+s+s1}{\PYGZsq{}}\PYG{l+s+s1}{kosdaq\PYGZus{}index.pkl}\PYG{l+s+s1}{\PYGZsq{}}\PYG{p}{)} 
\PYG{n}{kosdaq\PYGZus{}index}\PYG{o}{.}\PYG{n}{head}\PYG{p}{(}\PYG{p}{)}\PYG{o}{.}\PYG{n}{style}\PYG{o}{.}\PYG{n}{set\PYGZus{}table\PYGZus{}attributes}\PYG{p}{(}\PYG{l+s+s1}{\PYGZsq{}}\PYG{l+s+s1}{style=}\PYG{l+s+s1}{\PYGZdq{}}\PYG{l+s+s1}{font\PYGZhy{}size: 12px}\PYG{l+s+s1}{\PYGZdq{}}\PYG{l+s+s1}{\PYGZsq{}}\PYG{p}{)}
\end{sphinxVerbatim}

\end{sphinxuseclass}\end{sphinxVerbatimInput}
\begin{sphinxVerbatimOutput}

\begin{sphinxuseclass}{cell_output}
\begin{sphinxVerbatim}[commandchars=\\\{\}]
\PYGZlt{}pandas.io.formats.style.Styler at 0x1de8d29b850\PYGZgt{}
\end{sphinxVerbatim}

\end{sphinxuseclass}\end{sphinxVerbatimOutput}

\end{sphinxuseclass}
\begin{sphinxuseclass}{cell}\begin{sphinxVerbatimInput}

\begin{sphinxuseclass}{cell_input}
\begin{sphinxVerbatim}[commandchars=\\\{\}]
\PYG{k+kn}{import} \PYG{n+nn}{FinanceDataReader} \PYG{k}{as} \PYG{n+nn}{fdr}
\PYG{o}{\PYGZpc{}}\PYG{k}{matplotlib} inline
\PYG{k+kn}{import} \PYG{n+nn}{matplotlib}\PYG{n+nn}{.}\PYG{n+nn}{pyplot} \PYG{k}{as} \PYG{n+nn}{plt}
\PYG{k+kn}{import} \PYG{n+nn}{pandas} \PYG{k}{as} \PYG{n+nn}{pd}
\PYG{k+kn}{import} \PYG{n+nn}{numpy} \PYG{k}{as} \PYG{n+nn}{np}
\PYG{n}{pd}\PYG{o}{.}\PYG{n}{options}\PYG{o}{.}\PYG{n}{display}\PYG{o}{.}\PYG{n}{float\PYGZus{}format} \PYG{o}{=} \PYG{l+s+s1}{\PYGZsq{}}\PYG{l+s+si}{\PYGZob{}:,.3f\PYGZcb{}}\PYG{l+s+s1}{\PYGZsq{}}\PYG{o}{.}\PYG{n}{format}
\end{sphinxVerbatim}

\end{sphinxuseclass}\end{sphinxVerbatimInput}

\end{sphinxuseclass}

\part{종목별 일봉 데이터와 코스피 지수 데이터와 결합}
\label{\detokenize{chapter4/4.4.3_Data_Collection:id1}}\label{\detokenize{chapter4/4.4.3_Data_Collection::doc}}
\sphinxAtStartPar
앞에서 저장한 종목 리스트, 코스닥 종목별 주가 데이터와 지수 데이터를 읽습니다. 인덱스(날짜) 의 최소값과 최대값을 확인해 봅니다.

\begin{sphinxuseclass}{cell}\begin{sphinxVerbatimInput}

\begin{sphinxuseclass}{cell_input}
\begin{sphinxVerbatim}[commandchars=\\\{\}]
\PYG{n}{price\PYGZus{}data} \PYG{o}{=} \PYG{n}{pd}\PYG{o}{.}\PYG{n}{read\PYGZus{}pickle}\PYG{p}{(}\PYG{l+s+s1}{\PYGZsq{}}\PYG{l+s+s1}{stock\PYGZus{}data\PYGZus{}from\PYGZus{}fdr.pkl}\PYG{l+s+s1}{\PYGZsq{}}\PYG{p}{)} \PYG{c+c1}{\PYGZsh{} 주가 정보}
\PYG{n}{kosdaq\PYGZus{}index} \PYG{o}{=} \PYG{n}{pd}\PYG{o}{.}\PYG{n}{read\PYGZus{}pickle}\PYG{p}{(}\PYG{l+s+s1}{\PYGZsq{}}\PYG{l+s+s1}{kosdaq\PYGZus{}index.pkl}\PYG{l+s+s1}{\PYGZsq{}}\PYG{p}{)} \PYG{c+c1}{\PYGZsh{} 지수 정보}
\PYG{n}{kosdaq\PYGZus{}list} \PYG{o}{=} \PYG{n}{pd}\PYG{o}{.}\PYG{n}{read\PYGZus{}pickle}\PYG{p}{(}\PYG{l+s+s1}{\PYGZsq{}}\PYG{l+s+s1}{kosdaq\PYGZus{}list.pkl}\PYG{l+s+s1}{\PYGZsq{}}\PYG{p}{)} \PYG{c+c1}{\PYGZsh{} 종목 정보}

\PYG{n+nb}{print}\PYG{p}{(}\PYG{n}{price\PYGZus{}data}\PYG{o}{.}\PYG{n}{index}\PYG{o}{.}\PYG{n}{min}\PYG{p}{(}\PYG{p}{)}\PYG{p}{,} \PYG{n}{price\PYGZus{}data}\PYG{o}{.}\PYG{n}{index}\PYG{o}{.}\PYG{n}{max}\PYG{p}{(}\PYG{p}{)}\PYG{p}{)}
\PYG{n+nb}{print}\PYG{p}{(}\PYG{n}{kosdaq\PYGZus{}index}\PYG{o}{.}\PYG{n}{index}\PYG{o}{.}\PYG{n}{min}\PYG{p}{(}\PYG{p}{)}\PYG{p}{,} \PYG{n}{kosdaq\PYGZus{}index}\PYG{o}{.}\PYG{n}{index}\PYG{o}{.}\PYG{n}{max}\PYG{p}{(}\PYG{p}{)}\PYG{p}{)}
\end{sphinxVerbatim}

\end{sphinxuseclass}\end{sphinxVerbatimInput}
\begin{sphinxVerbatimOutput}

\begin{sphinxuseclass}{cell_output}
\begin{sphinxVerbatim}[commandchars=\\\{\}]
2021\PYGZhy{}01\PYGZhy{}04 00:00:00 2022\PYGZhy{}03\PYGZhy{}31 00:00:00
2021\PYGZhy{}01\PYGZhy{}04 00:00:00 2022\PYGZhy{}06\PYGZhy{}24 00:00:00
\end{sphinxVerbatim}

\end{sphinxuseclass}\end{sphinxVerbatimOutput}

\end{sphinxuseclass}
\sphinxAtStartPar
 나중에 검정할 가설 중 하나가 “주가가 상승할 확률이 높은 종목은 마켓이 안 좋을 때(즉 지표가 빠질 때) 수익율이 좋았다” 입니다. 이 가설을 검증하기 위해 두 데이타셋을 병합합니다. 두 데이터를 종목별 날짜별로 병합을 해야 ‘종목 수익율’과 ‘코스닥 지수 수익율’을 비교할 수 있습니다.

\sphinxAtStartPar
price\_data 를 기준으로 kosdaq\_index 데이터의 지수 수익율을 추가합니다. price\_data 에 날짜를 Index 로 left merge 를 하면 주가지수 정보를 추가할 수 있습니다.

\begin{sphinxuseclass}{cell}\begin{sphinxVerbatimInput}

\begin{sphinxuseclass}{cell_input}
\begin{sphinxVerbatim}[commandchars=\\\{\}]
\PYG{n}{merged} \PYG{o}{=} \PYG{n}{price\PYGZus{}data}\PYG{o}{.}\PYG{n}{merge}\PYG{p}{(}\PYG{n}{kosdaq\PYGZus{}index}\PYG{p}{[}\PYG{l+s+s1}{\PYGZsq{}}\PYG{l+s+s1}{kosdaq\PYGZus{}return}\PYG{l+s+s1}{\PYGZsq{}}\PYG{p}{]}\PYG{p}{,} \PYG{n}{left\PYGZus{}index}\PYG{o}{=}\PYG{k+kc}{True}\PYG{p}{,} \PYG{n}{right\PYGZus{}index}\PYG{o}{=}\PYG{k+kc}{True}\PYG{p}{,} \PYG{n}{how}\PYG{o}{=}\PYG{l+s+s1}{\PYGZsq{}}\PYG{l+s+s1}{left}\PYG{l+s+s1}{\PYGZsq{}}\PYG{p}{)}
\PYG{n}{merged}\PYG{o}{.}\PYG{n}{head}\PYG{p}{(}\PYG{p}{)}\PYG{o}{.}\PYG{n}{style}\PYG{o}{.}\PYG{n}{set\PYGZus{}table\PYGZus{}attributes}\PYG{p}{(}\PYG{l+s+s1}{\PYGZsq{}}\PYG{l+s+s1}{style=}\PYG{l+s+s1}{\PYGZdq{}}\PYG{l+s+s1}{font\PYGZhy{}size: 12px}\PYG{l+s+s1}{\PYGZdq{}}\PYG{l+s+s1}{\PYGZsq{}}\PYG{p}{)}
\end{sphinxVerbatim}

\end{sphinxuseclass}\end{sphinxVerbatimInput}
\begin{sphinxVerbatimOutput}

\begin{sphinxuseclass}{cell_output}
\begin{sphinxVerbatim}[commandchars=\\\{\}]
\PYGZlt{}pandas.io.formats.style.Styler at 0x25487cdbf40\PYGZgt{}
\end{sphinxVerbatim}

\end{sphinxuseclass}\end{sphinxVerbatimOutput}

\end{sphinxuseclass}
\sphinxAtStartPar
가설 검정을 위해 미리 컬럼을 생성합니다. 코스닥 지수 수익율이 1 보다 적을 때, 종목의 수익율이 1 보다 크면 1, 아니면 0 을 생성합니다. 그 값을 win\_market 이라는 새로운 컬럼에 저장합니다. 아래오와 같이 np.where 구문을 사용했는데요.

\begin{sphinxVerbatim}[commandchars=\\\{\}]
\PYG{n}{stock\PYGZus{}return}\PYG{p}{[}\PYG{l+s+s1}{\PYGZsq{}}\PYG{l+s+s1}{win\PYGZus{}market}\PYG{l+s+s1}{\PYGZsq{}}\PYG{p}{]} \PYG{o}{=} \PYG{n}{np}\PYG{o}{.}\PYG{n}{where}\PYG{p}{(}\PYG{p}{(}\PYG{n}{c1}\PYG{o}{\PYGZam{}}\PYG{n}{c2}\PYG{p}{)}\PYG{p}{,} \PYG{l+m+mi}{1}\PYG{p}{,} \PYG{l+m+mi}{0}\PYG{p}{)}
\end{sphinxVerbatim}

\sphinxAtStartPar
이 메소드는 np.where(조건, 조건이 참일 때 값, 조건이 거짓일 때 값)와 같이 처리를 합니다.

\begin{sphinxuseclass}{cell}\begin{sphinxVerbatimInput}

\begin{sphinxuseclass}{cell_input}
\begin{sphinxVerbatim}[commandchars=\\\{\}]
\PYG{n}{return\PYGZus{}all} \PYG{o}{=} \PYG{n}{pd}\PYG{o}{.}\PYG{n}{DataFrame}\PYG{p}{(}\PYG{p}{)}

\PYG{k}{for} \PYG{n}{code} \PYG{o+ow}{in} \PYG{n}{kosdaq\PYGZus{}list}\PYG{p}{[}\PYG{l+s+s1}{\PYGZsq{}}\PYG{l+s+s1}{code}\PYG{l+s+s1}{\PYGZsq{}}\PYG{p}{]}\PYG{p}{:}  
    
    \PYG{n}{stock\PYGZus{}return} \PYG{o}{=} \PYG{n}{merged}\PYG{p}{[}\PYG{n}{merged}\PYG{p}{[}\PYG{l+s+s1}{\PYGZsq{}}\PYG{l+s+s1}{code}\PYG{l+s+s1}{\PYGZsq{}}\PYG{p}{]}\PYG{o}{==}\PYG{n}{code}\PYG{p}{]}\PYG{o}{.}\PYG{n}{sort\PYGZus{}index}\PYG{p}{(}\PYG{p}{)}
    \PYG{n}{stock\PYGZus{}return}\PYG{p}{[}\PYG{l+s+s1}{\PYGZsq{}}\PYG{l+s+s1}{return}\PYG{l+s+s1}{\PYGZsq{}}\PYG{p}{]} \PYG{o}{=} \PYG{n}{stock\PYGZus{}return}\PYG{p}{[}\PYG{l+s+s1}{\PYGZsq{}}\PYG{l+s+s1}{close}\PYG{l+s+s1}{\PYGZsq{}}\PYG{p}{]}\PYG{o}{/}\PYG{n}{stock\PYGZus{}return}\PYG{p}{[}\PYG{l+s+s1}{\PYGZsq{}}\PYG{l+s+s1}{close}\PYG{l+s+s1}{\PYGZsq{}}\PYG{p}{]}\PYG{o}{.}\PYG{n}{shift}\PYG{p}{(}\PYG{l+m+mi}{1}\PYG{p}{)} \PYG{c+c1}{\PYGZsh{} 종목별 전일 종가 대비 당일 종가 수익율}
    \PYG{n}{c1} \PYG{o}{=} \PYG{p}{(}\PYG{n}{stock\PYGZus{}return}\PYG{p}{[}\PYG{l+s+s1}{\PYGZsq{}}\PYG{l+s+s1}{kosdaq\PYGZus{}return}\PYG{l+s+s1}{\PYGZsq{}}\PYG{p}{]} \PYG{o}{\PYGZlt{}} \PYG{l+m+mi}{1}\PYG{p}{)} \PYG{c+c1}{\PYGZsh{} 수익율 1 보다 작음. 당일 종가가 전일 종가보다 낮음 (코스닥 지표)}
    \PYG{n}{c2} \PYG{o}{=} \PYG{p}{(}\PYG{n}{stock\PYGZus{}return}\PYG{p}{[}\PYG{l+s+s1}{\PYGZsq{}}\PYG{l+s+s1}{return}\PYG{l+s+s1}{\PYGZsq{}}\PYG{p}{]} \PYG{o}{\PYGZgt{}} \PYG{l+m+mi}{1}\PYG{p}{)} \PYG{c+c1}{\PYGZsh{} 수익율 1 보다 큼. 당일 종가가 전일 종가보다 큼 (개별 종목)}
    \PYG{n}{stock\PYGZus{}return}\PYG{p}{[}\PYG{l+s+s1}{\PYGZsq{}}\PYG{l+s+s1}{win\PYGZus{}market}\PYG{l+s+s1}{\PYGZsq{}}\PYG{p}{]} \PYG{o}{=} \PYG{n}{np}\PYG{o}{.}\PYG{n}{where}\PYG{p}{(}\PYG{p}{(}\PYG{n}{c1}\PYG{o}{\PYGZam{}}\PYG{n}{c2}\PYG{p}{)}\PYG{p}{,} \PYG{l+m+mi}{1}\PYG{p}{,} \PYG{l+m+mi}{0}\PYG{p}{)} \PYG{c+c1}{\PYGZsh{} C1 과 C2 조건을 동시에 만족하면 1, 아니면 0}
    \PYG{n}{return\PYGZus{}all} \PYG{o}{=} \PYG{n}{pd}\PYG{o}{.}\PYG{n}{concat}\PYG{p}{(}\PYG{p}{[}\PYG{n}{return\PYGZus{}all}\PYG{p}{,} \PYG{n}{stock\PYGZus{}return}\PYG{p}{]}\PYG{p}{,} \PYG{n}{axis}\PYG{o}{=}\PYG{l+m+mi}{0}\PYG{p}{)} 
    
\PYG{n}{return\PYGZus{}all}\PYG{o}{.}\PYG{n}{to\PYGZus{}pickle}\PYG{p}{(}\PYG{l+s+s1}{\PYGZsq{}}\PYG{l+s+s1}{return\PYGZus{}all.pkl}\PYG{l+s+s1}{\PYGZsq{}}\PYG{p}{)}       
\end{sphinxVerbatim}

\end{sphinxuseclass}\end{sphinxVerbatimInput}

\end{sphinxuseclass}
\sphinxAtStartPar
 값이 잘 들어갔는 지 head 메소드로 첫 번째 행 5 개를 출력해 봅니다.

\begin{sphinxuseclass}{cell}\begin{sphinxVerbatimInput}

\begin{sphinxuseclass}{cell_input}
\begin{sphinxVerbatim}[commandchars=\\\{\}]
\PYG{n}{return\PYGZus{}all} \PYG{o}{=} \PYG{n}{pd}\PYG{o}{.}\PYG{n}{read\PYGZus{}pickle}\PYG{p}{(}\PYG{l+s+s1}{\PYGZsq{}}\PYG{l+s+s1}{return\PYGZus{}all.pkl}\PYG{l+s+s1}{\PYGZsq{}}\PYG{p}{)}  
\PYG{n}{return\PYGZus{}all}\PYG{o}{.}\PYG{n}{head}\PYG{p}{(}\PYG{p}{)}\PYG{o}{.}\PYG{n}{style}\PYG{o}{.}\PYG{n}{set\PYGZus{}table\PYGZus{}attributes}\PYG{p}{(}\PYG{l+s+s1}{\PYGZsq{}}\PYG{l+s+s1}{style=}\PYG{l+s+s1}{\PYGZdq{}}\PYG{l+s+s1}{font\PYGZhy{}size: 12px}\PYG{l+s+s1}{\PYGZdq{}}\PYG{l+s+s1}{\PYGZsq{}}\PYG{p}{)}
\end{sphinxVerbatim}

\end{sphinxuseclass}\end{sphinxVerbatimInput}
\begin{sphinxVerbatimOutput}

\begin{sphinxuseclass}{cell_output}
\begin{sphinxVerbatim}[commandchars=\\\{\}]
\PYGZlt{}pandas.io.formats.style.Styler at 0x25489ff2850\PYGZgt{}
\end{sphinxVerbatim}

\end{sphinxuseclass}\end{sphinxVerbatimOutput}

\end{sphinxuseclass}
\sphinxAtStartPar
가설 검정 시 자세히 다루겠지만, win\_market 의 비율과 종목별 수익율과의 관계를 간단하게 조사하겠습니다. 이번에 scatter plot 를 함 그려보겠습니다. Scatter plot 에는 x 축의 값과 y 축의 값을 인수로 넣어주면 됩니다. 그래프를 보니 두 값 사이에 상관성이 높아 보입니다.

\begin{sphinxuseclass}{cell}\begin{sphinxVerbatimInput}

\begin{sphinxuseclass}{cell_input}
\begin{sphinxVerbatim}[commandchars=\\\{\}]
\PYG{n}{plt}\PYG{o}{.}\PYG{n}{figure}\PYG{p}{(}\PYG{n}{figsize}\PYG{o}{=}\PYG{p}{(}\PYG{l+m+mi}{10}\PYG{p}{,}\PYG{l+m+mi}{6}\PYG{p}{)}\PYG{p}{)}
\PYG{n}{x} \PYG{o}{=} \PYG{n}{return\PYGZus{}all}\PYG{o}{.}\PYG{n}{groupby}\PYG{p}{(}\PYG{l+s+s1}{\PYGZsq{}}\PYG{l+s+s1}{code}\PYG{l+s+s1}{\PYGZsq{}}\PYG{p}{)}\PYG{p}{[}\PYG{l+s+s1}{\PYGZsq{}}\PYG{l+s+s1}{win\PYGZus{}market}\PYG{l+s+s1}{\PYGZsq{}}\PYG{p}{]}\PYG{o}{.}\PYG{n}{mean}\PYG{p}{(}\PYG{p}{)} \PYG{c+c1}{\PYGZsh{} 종목별 win\PYGZus{}market의 비율}
\PYG{n}{y} \PYG{o}{=} \PYG{n}{return\PYGZus{}all}\PYG{o}{.}\PYG{n}{groupby}\PYG{p}{(}\PYG{l+s+s1}{\PYGZsq{}}\PYG{l+s+s1}{code}\PYG{l+s+s1}{\PYGZsq{}}\PYG{p}{)}\PYG{p}{[}\PYG{l+s+s1}{\PYGZsq{}}\PYG{l+s+s1}{return}\PYG{l+s+s1}{\PYGZsq{}}\PYG{p}{]}\PYG{o}{.}\PYG{n}{mean}\PYG{p}{(}\PYG{p}{)} \PYG{c+c1}{\PYGZsh{} 종목별 평균 수익율 }
\PYG{n}{plt}\PYG{o}{.}\PYG{n}{scatter}\PYG{p}{(}\PYG{n}{x} \PYG{o}{=} \PYG{n}{x} \PYG{p}{,} \PYG{n}{y}\PYG{o}{=} \PYG{n}{y}\PYG{p}{,} \PYG{n}{s}\PYG{o}{=}\PYG{l+m+mi}{1}\PYG{p}{,} \PYG{n}{color}\PYG{o}{=}\PYG{l+s+s1}{\PYGZsq{}}\PYG{l+s+s1}{red}\PYG{l+s+s1}{\PYGZsq{}}\PYG{p}{)}
\PYG{n}{plt}\PYG{o}{.}\PYG{n}{xlabel}\PYG{p}{(}\PYG{l+s+s1}{\PYGZsq{}}\PYG{l+s+si}{\PYGZpc{} o}\PYG{l+s+s1}{f Win Market}\PYG{l+s+s1}{\PYGZsq{}}\PYG{p}{)}
\PYG{n}{plt}\PYG{o}{.}\PYG{n}{ylabel}\PYG{p}{(}\PYG{l+s+s1}{\PYGZsq{}}\PYG{l+s+s1}{Avg. Return}\PYG{l+s+s1}{\PYGZsq{}}\PYG{p}{)}
\end{sphinxVerbatim}

\end{sphinxuseclass}\end{sphinxVerbatimInput}
\begin{sphinxVerbatimOutput}

\begin{sphinxuseclass}{cell_output}
\begin{sphinxVerbatim}[commandchars=\\\{\}]
Text(0, 0.5, \PYGZsq{}Avg. Return\PYGZsq{})
\end{sphinxVerbatim}

\noindent\sphinxincludegraphics{{4.4.3_Data_Collection_10_1}.png}

\end{sphinxuseclass}\end{sphinxVerbatimOutput}

\end{sphinxuseclass}
\begin{sphinxuseclass}{cell}\begin{sphinxVerbatimInput}

\begin{sphinxuseclass}{cell_input}
\begin{sphinxVerbatim}[commandchars=\\\{\}]
\PYG{k+kn}{import} \PYG{n+nn}{FinanceDataReader} \PYG{k}{as} \PYG{n+nn}{fdr}
\PYG{o}{\PYGZpc{}}\PYG{k}{matplotlib} inline
\PYG{k+kn}{import} \PYG{n+nn}{matplotlib}\PYG{n+nn}{.}\PYG{n+nn}{pyplot} \PYG{k}{as} \PYG{n+nn}{plt}
\PYG{k+kn}{import} \PYG{n+nn}{pandas} \PYG{k}{as} \PYG{n+nn}{pd}
\PYG{k+kn}{import} \PYG{n+nn}{numpy} \PYG{k}{as} \PYG{n+nn}{np}
\PYG{k+kn}{import} \PYG{n+nn}{datetime}
\PYG{n}{pd}\PYG{o}{.}\PYG{n}{options}\PYG{o}{.}\PYG{n}{display}\PYG{o}{.}\PYG{n}{float\PYGZus{}format} \PYG{o}{=} \PYG{l+s+s1}{\PYGZsq{}}\PYG{l+s+si}{\PYGZob{}:,.2f\PYGZcb{}}\PYG{l+s+s1}{\PYGZsq{}}\PYG{o}{.}\PYG{n}{format}
\PYG{n}{pd}\PYG{o}{.}\PYG{n}{set\PYGZus{}option}\PYG{p}{(}\PYG{l+s+s1}{\PYGZsq{}}\PYG{l+s+s1}{display.expand\PYGZus{}frame\PYGZus{}repr}\PYG{l+s+s1}{\PYGZsq{}}\PYG{p}{,} \PYG{k+kc}{False}\PYG{p}{)}
\end{sphinxVerbatim}

\end{sphinxuseclass}\end{sphinxVerbatimInput}

\end{sphinxuseclass}

\part{가설 검증을 위한 데이터 처리}
\label{\detokenize{chapter4/4.4.4_Data_Processing:id1}}\label{\detokenize{chapter4/4.4.4_Data_Processing::doc}}
\sphinxAtStartPar
앞서 만든 return\_all (주가 데이터에 지수데이터가 추가된 파일) 을 아래와 같이 로드하고, Missing Data 는 제거합니다.

\begin{sphinxuseclass}{cell}\begin{sphinxVerbatimInput}

\begin{sphinxuseclass}{cell_input}
\begin{sphinxVerbatim}[commandchars=\\\{\}]
\PYG{n}{return\PYGZus{}all} \PYG{o}{=} \PYG{n}{pd}\PYG{o}{.}\PYG{n}{read\PYGZus{}pickle}\PYG{p}{(}\PYG{l+s+s1}{\PYGZsq{}}\PYG{l+s+s1}{return\PYGZus{}all.pkl}\PYG{l+s+s1}{\PYGZsq{}}\PYG{p}{)}\PYG{o}{.}\PYG{n}{dropna}\PYG{p}{(}\PYG{p}{)}  
\PYG{n}{return\PYGZus{}all}\PYG{o}{.}\PYG{n}{index} \PYG{o}{=} \PYG{p}{[}\PYG{n}{datetime}\PYG{o}{.}\PYG{n}{datetime}\PYG{o}{.}\PYG{n}{strftime}\PYG{p}{(}\PYG{n}{dt}\PYG{p}{,} \PYG{l+s+s1}{\PYGZsq{}}\PYG{l+s+s1}{\PYGZpc{}}\PYG{l+s+s1}{Y\PYGZhy{}}\PYG{l+s+s1}{\PYGZpc{}}\PYG{l+s+s1}{m\PYGZhy{}}\PYG{l+s+si}{\PYGZpc{}d}\PYG{l+s+s1}{\PYGZsq{}}\PYG{p}{)} \PYG{k}{for} \PYG{n}{dt} \PYG{o+ow}{in} \PYG{n}{return\PYGZus{}all}\PYG{o}{.}\PYG{n}{index}\PYG{p}{]}
\end{sphinxVerbatim}

\end{sphinxuseclass}\end{sphinxVerbatimInput}

\end{sphinxuseclass}
\begin{sphinxuseclass}{cell}\begin{sphinxVerbatimInput}

\begin{sphinxuseclass}{cell_input}
\begin{sphinxVerbatim}[commandchars=\\\{\}]
\PYG{n}{return\PYGZus{}all}\PYG{o}{.}\PYG{n}{head}\PYG{p}{(}\PYG{p}{)}\PYG{o}{.}\PYG{n}{style}\PYG{o}{.}\PYG{n}{set\PYGZus{}table\PYGZus{}attributes}\PYG{p}{(}\PYG{l+s+s1}{\PYGZsq{}}\PYG{l+s+s1}{style=}\PYG{l+s+s1}{\PYGZdq{}}\PYG{l+s+s1}{font\PYGZhy{}size: 12px}\PYG{l+s+s1}{\PYGZdq{}}\PYG{l+s+s1}{\PYGZsq{}}\PYG{p}{)}
\end{sphinxVerbatim}

\end{sphinxuseclass}\end{sphinxVerbatimInput}
\begin{sphinxVerbatimOutput}

\begin{sphinxuseclass}{cell_output}
\begin{sphinxVerbatim}[commandchars=\\\{\}]
\PYGZlt{}pandas.io.formats.style.Styler at 0x218fd39b5b0\PYGZgt{}
\end{sphinxVerbatim}

\end{sphinxuseclass}\end{sphinxVerbatimOutput}

\end{sphinxuseclass}
\sphinxAtStartPar
일주일(5영업일)을 수익율의 관찰 기간으로 하고, 관찰 기간 동안 주가 상승이 있으면 저희가 세운 가설들을 유의미한 가설로 판단하겠습니다. 여기서 주가 상승의 기준은  “종가 매수 일부터 다음 5 영업일 동안 최고 종가 수익율” 하겠습니다.

\sphinxAtStartPar
첫 번째 종목 060310 에 대하여 처리를 먼저 해 보겠습니다. df{[}‘close’{]} * shift(\sphinxhyphen{}1) 은 다음 영업일의 종가 수익율을 참조하고, df{[}‘close’{]}*shift(\sphinxhyphen{}2) 은 그 다음의 영업일의 종가 수익율을 참조합니다. 따라서 매수 후 2 영업일 후, 종가 수익율은 \{ df{[}‘close’{]} * shift(\sphinxhyphen{}1) \} * \{ df{[}‘close’{]} * shift(\sphinxhyphen{}2) \} 로 계산됩니다. 이렇게 1 영업일, 2 영업일, 3 영업일, 4 영업일, 5 영업일 후 종가 수익율을 새로운 컬럼에 생성하고, 그 중에서 가장 큰 수익율을 고르면 됩니다. 생성된 컬럼 중 가장 큰 값은 max(axis=1) 로 찾습니다. 참고로 max() 에서는 axis=0 이 Default 라서 axis=1 로 정해주지 않으면 열에서 가장 큰 값을 찾게 됩니다. 이 부분을 유의해 주세요.

\begin{sphinxuseclass}{cell}\begin{sphinxVerbatimInput}

\begin{sphinxuseclass}{cell_input}
\begin{sphinxVerbatim}[commandchars=\\\{\}]
\PYG{n}{s} \PYG{o}{=} \PYG{l+s+s1}{\PYGZsq{}}\PYG{l+s+s1}{060310}\PYG{l+s+s1}{\PYGZsq{}}
\PYG{n}{df} \PYG{o}{=} \PYG{n}{return\PYGZus{}all}\PYG{p}{[}\PYG{n}{return\PYGZus{}all}\PYG{p}{[}\PYG{l+s+s1}{\PYGZsq{}}\PYG{l+s+s1}{code}\PYG{l+s+s1}{\PYGZsq{}}\PYG{p}{]}\PYG{o}{==}\PYG{n}{s}\PYG{p}{]}\PYG{o}{.}\PYG{n}{sort\PYGZus{}index}\PYG{p}{(}\PYG{p}{)}\PYG{o}{.}\PYG{n}{copy}\PYG{p}{(}\PYG{p}{)}

\PYG{n}{df}\PYG{p}{[}\PYG{l+s+s1}{\PYGZsq{}}\PYG{l+s+s1}{close\PYGZus{}r1}\PYG{l+s+s1}{\PYGZsq{}}\PYG{p}{]} \PYG{o}{=} \PYG{n}{df}\PYG{p}{[}\PYG{l+s+s1}{\PYGZsq{}}\PYG{l+s+s1}{close}\PYG{l+s+s1}{\PYGZsq{}}\PYG{p}{]}\PYG{o}{.}\PYG{n}{shift}\PYG{p}{(}\PYG{o}{\PYGZhy{}}\PYG{l+m+mi}{1}\PYG{p}{)}\PYG{o}{/}\PYG{n}{df}\PYG{p}{[}\PYG{l+s+s1}{\PYGZsq{}}\PYG{l+s+s1}{close}\PYG{l+s+s1}{\PYGZsq{}}\PYG{p}{]} \PYG{c+c1}{\PYGZsh{} 1 일후 종가 수익률}
\PYG{n}{df}\PYG{p}{[}\PYG{l+s+s1}{\PYGZsq{}}\PYG{l+s+s1}{close\PYGZus{}r2}\PYG{l+s+s1}{\PYGZsq{}}\PYG{p}{]} \PYG{o}{=} \PYG{n}{df}\PYG{p}{[}\PYG{l+s+s1}{\PYGZsq{}}\PYG{l+s+s1}{close}\PYG{l+s+s1}{\PYGZsq{}}\PYG{p}{]}\PYG{o}{.}\PYG{n}{shift}\PYG{p}{(}\PYG{o}{\PYGZhy{}}\PYG{l+m+mi}{2}\PYG{p}{)}\PYG{o}{/}\PYG{n}{df}\PYG{p}{[}\PYG{l+s+s1}{\PYGZsq{}}\PYG{l+s+s1}{close}\PYG{l+s+s1}{\PYGZsq{}}\PYG{p}{]} \PYG{c+c1}{\PYGZsh{} 2 일후 종가 수익률}
\PYG{n}{df}\PYG{p}{[}\PYG{l+s+s1}{\PYGZsq{}}\PYG{l+s+s1}{close\PYGZus{}r3}\PYG{l+s+s1}{\PYGZsq{}}\PYG{p}{]} \PYG{o}{=} \PYG{n}{df}\PYG{p}{[}\PYG{l+s+s1}{\PYGZsq{}}\PYG{l+s+s1}{close}\PYG{l+s+s1}{\PYGZsq{}}\PYG{p}{]}\PYG{o}{.}\PYG{n}{shift}\PYG{p}{(}\PYG{o}{\PYGZhy{}}\PYG{l+m+mi}{3}\PYG{p}{)}\PYG{o}{/}\PYG{n}{df}\PYG{p}{[}\PYG{l+s+s1}{\PYGZsq{}}\PYG{l+s+s1}{close}\PYG{l+s+s1}{\PYGZsq{}}\PYG{p}{]} \PYG{c+c1}{\PYGZsh{} 3 일후 종가 수익률}
\PYG{n}{df}\PYG{p}{[}\PYG{l+s+s1}{\PYGZsq{}}\PYG{l+s+s1}{close\PYGZus{}r4}\PYG{l+s+s1}{\PYGZsq{}}\PYG{p}{]} \PYG{o}{=} \PYG{n}{df}\PYG{p}{[}\PYG{l+s+s1}{\PYGZsq{}}\PYG{l+s+s1}{close}\PYG{l+s+s1}{\PYGZsq{}}\PYG{p}{]}\PYG{o}{.}\PYG{n}{shift}\PYG{p}{(}\PYG{o}{\PYGZhy{}}\PYG{l+m+mi}{4}\PYG{p}{)}\PYG{o}{/}\PYG{n}{df}\PYG{p}{[}\PYG{l+s+s1}{\PYGZsq{}}\PYG{l+s+s1}{close}\PYG{l+s+s1}{\PYGZsq{}}\PYG{p}{]} \PYG{c+c1}{\PYGZsh{} 4 일후 종가 수익률}
\PYG{n}{df}\PYG{p}{[}\PYG{l+s+s1}{\PYGZsq{}}\PYG{l+s+s1}{close\PYGZus{}r5}\PYG{l+s+s1}{\PYGZsq{}}\PYG{p}{]} \PYG{o}{=} \PYG{n}{df}\PYG{p}{[}\PYG{l+s+s1}{\PYGZsq{}}\PYG{l+s+s1}{close}\PYG{l+s+s1}{\PYGZsq{}}\PYG{p}{]}\PYG{o}{.}\PYG{n}{shift}\PYG{p}{(}\PYG{o}{\PYGZhy{}}\PYG{l+m+mi}{5}\PYG{p}{)}\PYG{o}{/}\PYG{n}{df}\PYG{p}{[}\PYG{l+s+s1}{\PYGZsq{}}\PYG{l+s+s1}{close}\PYG{l+s+s1}{\PYGZsq{}}\PYG{p}{]} \PYG{c+c1}{\PYGZsh{} 5 일후 종가 수익률}

\PYG{l+s+sd}{\PYGZsq{}\PYGZsq{}\PYGZsq{} 위 코드와 같은 결과}
\PYG{l+s+sd}{df[\PYGZsq{}return\PYGZus{}1\PYGZsq{}] = df[\PYGZsq{}return\PYGZsq{}].shift(\PYGZhy{}1)}
\PYG{l+s+sd}{df[\PYGZsq{}return\PYGZus{}2\PYGZsq{}] = df[\PYGZsq{}return\PYGZsq{}].shift(\PYGZhy{}2)*df[\PYGZsq{}return\PYGZsq{}].shift(\PYGZhy{}1)}
\PYG{l+s+sd}{df[\PYGZsq{}return\PYGZus{}3\PYGZsq{}] = df[\PYGZsq{}return\PYGZsq{}].shift(\PYGZhy{}3)*df[\PYGZsq{}return\PYGZsq{}].shift(\PYGZhy{}2)*df[\PYGZsq{}return\PYGZsq{}].shift(\PYGZhy{}1)}
\PYG{l+s+sd}{df[\PYGZsq{}return\PYGZus{}4\PYGZsq{}] = df[\PYGZsq{}return\PYGZsq{}].shift(\PYGZhy{}4)*df[\PYGZsq{}return\PYGZsq{}].shift(\PYGZhy{}3)*df[\PYGZsq{}return\PYGZsq{}].shift(\PYGZhy{}2)*df[\PYGZsq{}return\PYGZsq{}].shift(\PYGZhy{}1)}
\PYG{l+s+sd}{df[\PYGZsq{}return\PYGZus{}5\PYGZsq{}] = df[\PYGZsq{}return\PYGZsq{}].shift(\PYGZhy{}5)*df[\PYGZsq{}return\PYGZsq{}].shift(\PYGZhy{}4)*df[\PYGZsq{}return\PYGZsq{}].shift(\PYGZhy{}3)*df[\PYGZsq{}return\PYGZsq{}].shift(\PYGZhy{}2)*df[\PYGZsq{}return\PYGZsq{}].shift(\PYGZhy{}1)}
\PYG{l+s+sd}{\PYGZsq{}\PYGZsq{}\PYGZsq{}}

\PYG{n}{df}\PYG{p}{[}\PYG{l+s+s1}{\PYGZsq{}}\PYG{l+s+s1}{target}\PYG{l+s+s1}{\PYGZsq{}}\PYG{p}{]} \PYG{o}{=} \PYG{n}{df}\PYG{p}{[}\PYG{p}{[}\PYG{l+s+s1}{\PYGZsq{}}\PYG{l+s+s1}{close\PYGZus{}r1}\PYG{l+s+s1}{\PYGZsq{}}\PYG{p}{,}\PYG{l+s+s1}{\PYGZsq{}}\PYG{l+s+s1}{close\PYGZus{}r2}\PYG{l+s+s1}{\PYGZsq{}}\PYG{p}{,}\PYG{l+s+s1}{\PYGZsq{}}\PYG{l+s+s1}{close\PYGZus{}r3}\PYG{l+s+s1}{\PYGZsq{}}\PYG{p}{,}\PYG{l+s+s1}{\PYGZsq{}}\PYG{l+s+s1}{close\PYGZus{}r4}\PYG{l+s+s1}{\PYGZsq{}}\PYG{p}{,}\PYG{l+s+s1}{\PYGZsq{}}\PYG{l+s+s1}{close\PYGZus{}r5}\PYG{l+s+s1}{\PYGZsq{}}\PYG{p}{]}\PYG{p}{]}\PYG{o}{.}\PYG{n}{max}\PYG{p}{(}\PYG{n}{axis}\PYG{o}{=}\PYG{l+m+mi}{1}\PYG{p}{)} \PYG{c+c1}{\PYGZsh{} 주어지 컬럼에서 최대 값을 찾고 \PYGZsq{}target\PYGZsq{} 에 저장}
\PYG{n}{df}\PYG{o}{.}\PYG{n}{dropna}\PYG{p}{(}\PYG{n}{subset}\PYG{o}{=}\PYG{p}{[}\PYG{l+s+s1}{\PYGZsq{}}\PYG{l+s+s1}{close\PYGZus{}r1}\PYG{l+s+s1}{\PYGZsq{}}\PYG{p}{,}\PYG{l+s+s1}{\PYGZsq{}}\PYG{l+s+s1}{close\PYGZus{}r2}\PYG{l+s+s1}{\PYGZsq{}}\PYG{p}{,}\PYG{l+s+s1}{\PYGZsq{}}\PYG{l+s+s1}{close\PYGZus{}r3}\PYG{l+s+s1}{\PYGZsq{}}\PYG{p}{,}\PYG{l+s+s1}{\PYGZsq{}}\PYG{l+s+s1}{close\PYGZus{}r4}\PYG{l+s+s1}{\PYGZsq{}}\PYG{p}{,}\PYG{l+s+s1}{\PYGZsq{}}\PYG{l+s+s1}{close\PYGZus{}r5}\PYG{l+s+s1}{\PYGZsq{}}\PYG{p}{]}\PYG{p}{,} \PYG{n}{inplace}\PYG{o}{=}\PYG{k+kc}{True}\PYG{p}{)} \PYG{c+c1}{\PYGZsh{} 주어진 컬럼 중에 missing 값이 있으면 행을 제거(dropna)하고, 자신을 덮어 씀(inplace=True).}
\end{sphinxVerbatim}

\end{sphinxuseclass}\end{sphinxVerbatimInput}

\end{sphinxuseclass}
\begin{sphinxuseclass}{cell}\begin{sphinxVerbatimInput}

\begin{sphinxuseclass}{cell_input}
\begin{sphinxVerbatim}[commandchars=\\\{\}]
\PYG{n}{df}\PYG{o}{.}\PYG{n}{head}\PYG{p}{(}\PYG{l+m+mi}{10}\PYG{p}{)}\PYG{o}{.}\PYG{n}{style}\PYG{o}{.}\PYG{n}{set\PYGZus{}table\PYGZus{}attributes}\PYG{p}{(}\PYG{l+s+s1}{\PYGZsq{}}\PYG{l+s+s1}{style=}\PYG{l+s+s1}{\PYGZdq{}}\PYG{l+s+s1}{font\PYGZhy{}size: 12px}\PYG{l+s+s1}{\PYGZdq{}}\PYG{l+s+s1}{\PYGZsq{}}\PYG{p}{)}
\end{sphinxVerbatim}

\end{sphinxuseclass}\end{sphinxVerbatimInput}
\begin{sphinxVerbatimOutput}

\begin{sphinxuseclass}{cell_output}
\begin{sphinxVerbatim}[commandchars=\\\{\}]
\PYGZlt{}pandas.io.formats.style.Styler at 0x200186c8610\PYGZgt{}
\end{sphinxVerbatim}

\end{sphinxuseclass}\end{sphinxVerbatimOutput}

\end{sphinxuseclass}
\sphinxAtStartPar
이제 모든 종목에 대하여 For loop 로 매수 종가로 매도 시 수익율을 최대값을 생성합니다. ‘max\_close’ 의 분포를 보니 평균은 1.033, 최소값 0.326, 최대값 3.703 입니다. 단, max\_close 는 가설 검정으로 활용할 지표입니다. 매수 후, 몇 번 째 영업일이 최고 수익율인지 알 수 없기 때문에 기간 중 최고 수익율을 이용합니다.

\begin{sphinxuseclass}{cell}\begin{sphinxVerbatimInput}

\begin{sphinxuseclass}{cell_input}
\begin{sphinxVerbatim}[commandchars=\\\{\}]
\PYG{n}{kosdaq\PYGZus{}list} \PYG{o}{=} \PYG{n}{pd}\PYG{o}{.}\PYG{n}{read\PYGZus{}pickle}\PYG{p}{(}\PYG{l+s+s1}{\PYGZsq{}}\PYG{l+s+s1}{kosdaq\PYGZus{}list.pkl}\PYG{l+s+s1}{\PYGZsq{}}\PYG{p}{)}

\PYG{n}{mdl\PYGZus{}data} \PYG{o}{=} \PYG{n}{pd}\PYG{o}{.}\PYG{n}{DataFrame}\PYG{p}{(}\PYG{p}{)}

\PYG{k}{for} \PYG{n}{code} \PYG{o+ow}{in} \PYG{n}{kosdaq\PYGZus{}list}\PYG{p}{[}\PYG{l+s+s1}{\PYGZsq{}}\PYG{l+s+s1}{code}\PYG{l+s+s1}{\PYGZsq{}}\PYG{p}{]}\PYG{p}{:}
    \PYG{n}{df} \PYG{o}{=} \PYG{n}{return\PYGZus{}all}\PYG{p}{[}\PYG{n}{return\PYGZus{}all}\PYG{p}{[}\PYG{l+s+s1}{\PYGZsq{}}\PYG{l+s+s1}{code}\PYG{l+s+s1}{\PYGZsq{}}\PYG{p}{]}\PYG{o}{==}\PYG{n}{code}\PYG{p}{]}\PYG{o}{.}\PYG{n}{sort\PYGZus{}index}\PYG{p}{(}\PYG{p}{)}\PYG{o}{.}\PYG{n}{copy}\PYG{p}{(}\PYG{p}{)}

    \PYG{n}{df}\PYG{p}{[}\PYG{l+s+s1}{\PYGZsq{}}\PYG{l+s+s1}{close\PYGZus{}r1}\PYG{l+s+s1}{\PYGZsq{}}\PYG{p}{]} \PYG{o}{=} \PYG{n}{df}\PYG{p}{[}\PYG{l+s+s1}{\PYGZsq{}}\PYG{l+s+s1}{close}\PYG{l+s+s1}{\PYGZsq{}}\PYG{p}{]}\PYG{o}{.}\PYG{n}{shift}\PYG{p}{(}\PYG{o}{\PYGZhy{}}\PYG{l+m+mi}{1}\PYG{p}{)}\PYG{o}{/}\PYG{n}{df}\PYG{p}{[}\PYG{l+s+s1}{\PYGZsq{}}\PYG{l+s+s1}{close}\PYG{l+s+s1}{\PYGZsq{}}\PYG{p}{]}
    \PYG{n}{df}\PYG{p}{[}\PYG{l+s+s1}{\PYGZsq{}}\PYG{l+s+s1}{close\PYGZus{}r2}\PYG{l+s+s1}{\PYGZsq{}}\PYG{p}{]} \PYG{o}{=} \PYG{n}{df}\PYG{p}{[}\PYG{l+s+s1}{\PYGZsq{}}\PYG{l+s+s1}{close}\PYG{l+s+s1}{\PYGZsq{}}\PYG{p}{]}\PYG{o}{.}\PYG{n}{shift}\PYG{p}{(}\PYG{o}{\PYGZhy{}}\PYG{l+m+mi}{2}\PYG{p}{)}\PYG{o}{/}\PYG{n}{df}\PYG{p}{[}\PYG{l+s+s1}{\PYGZsq{}}\PYG{l+s+s1}{close}\PYG{l+s+s1}{\PYGZsq{}}\PYG{p}{]}
    \PYG{n}{df}\PYG{p}{[}\PYG{l+s+s1}{\PYGZsq{}}\PYG{l+s+s1}{close\PYGZus{}r3}\PYG{l+s+s1}{\PYGZsq{}}\PYG{p}{]} \PYG{o}{=} \PYG{n}{df}\PYG{p}{[}\PYG{l+s+s1}{\PYGZsq{}}\PYG{l+s+s1}{close}\PYG{l+s+s1}{\PYGZsq{}}\PYG{p}{]}\PYG{o}{.}\PYG{n}{shift}\PYG{p}{(}\PYG{o}{\PYGZhy{}}\PYG{l+m+mi}{3}\PYG{p}{)}\PYG{o}{/}\PYG{n}{df}\PYG{p}{[}\PYG{l+s+s1}{\PYGZsq{}}\PYG{l+s+s1}{close}\PYG{l+s+s1}{\PYGZsq{}}\PYG{p}{]}
    \PYG{n}{df}\PYG{p}{[}\PYG{l+s+s1}{\PYGZsq{}}\PYG{l+s+s1}{close\PYGZus{}r4}\PYG{l+s+s1}{\PYGZsq{}}\PYG{p}{]} \PYG{o}{=} \PYG{n}{df}\PYG{p}{[}\PYG{l+s+s1}{\PYGZsq{}}\PYG{l+s+s1}{close}\PYG{l+s+s1}{\PYGZsq{}}\PYG{p}{]}\PYG{o}{.}\PYG{n}{shift}\PYG{p}{(}\PYG{o}{\PYGZhy{}}\PYG{l+m+mi}{4}\PYG{p}{)}\PYG{o}{/}\PYG{n}{df}\PYG{p}{[}\PYG{l+s+s1}{\PYGZsq{}}\PYG{l+s+s1}{close}\PYG{l+s+s1}{\PYGZsq{}}\PYG{p}{]}
    \PYG{n}{df}\PYG{p}{[}\PYG{l+s+s1}{\PYGZsq{}}\PYG{l+s+s1}{close\PYGZus{}r5}\PYG{l+s+s1}{\PYGZsq{}}\PYG{p}{]} \PYG{o}{=} \PYG{n}{df}\PYG{p}{[}\PYG{l+s+s1}{\PYGZsq{}}\PYG{l+s+s1}{close}\PYG{l+s+s1}{\PYGZsq{}}\PYG{p}{]}\PYG{o}{.}\PYG{n}{shift}\PYG{p}{(}\PYG{o}{\PYGZhy{}}\PYG{l+m+mi}{5}\PYG{p}{)}\PYG{o}{/}\PYG{n}{df}\PYG{p}{[}\PYG{l+s+s1}{\PYGZsq{}}\PYG{l+s+s1}{close}\PYG{l+s+s1}{\PYGZsq{}}\PYG{p}{]}

    \PYG{n}{df}\PYG{p}{[}\PYG{l+s+s1}{\PYGZsq{}}\PYG{l+s+s1}{max\PYGZus{}close}\PYG{l+s+s1}{\PYGZsq{}}\PYG{p}{]} \PYG{o}{=} \PYG{n}{df}\PYG{p}{[}\PYG{p}{[}\PYG{l+s+s1}{\PYGZsq{}}\PYG{l+s+s1}{close\PYGZus{}r1}\PYG{l+s+s1}{\PYGZsq{}}\PYG{p}{,}\PYG{l+s+s1}{\PYGZsq{}}\PYG{l+s+s1}{close\PYGZus{}r2}\PYG{l+s+s1}{\PYGZsq{}}\PYG{p}{,}\PYG{l+s+s1}{\PYGZsq{}}\PYG{l+s+s1}{close\PYGZus{}r3}\PYG{l+s+s1}{\PYGZsq{}}\PYG{p}{,}\PYG{l+s+s1}{\PYGZsq{}}\PYG{l+s+s1}{close\PYGZus{}r4}\PYG{l+s+s1}{\PYGZsq{}}\PYG{p}{,}\PYG{l+s+s1}{\PYGZsq{}}\PYG{l+s+s1}{close\PYGZus{}r5}\PYG{l+s+s1}{\PYGZsq{}}\PYG{p}{]}\PYG{p}{]}\PYG{o}{.}\PYG{n}{max}\PYG{p}{(}\PYG{n}{axis}\PYG{o}{=}\PYG{l+m+mi}{1}\PYG{p}{)} \PYG{c+c1}{\PYGZsh{} 주어지 컬럼에서 최대 값을 찾음}
    \PYG{n}{df}\PYG{o}{.}\PYG{n}{dropna}\PYG{p}{(}\PYG{n}{subset}\PYG{o}{=}\PYG{p}{[}\PYG{l+s+s1}{\PYGZsq{}}\PYG{l+s+s1}{close\PYGZus{}r1}\PYG{l+s+s1}{\PYGZsq{}}\PYG{p}{,}\PYG{l+s+s1}{\PYGZsq{}}\PYG{l+s+s1}{close\PYGZus{}r2}\PYG{l+s+s1}{\PYGZsq{}}\PYG{p}{,}\PYG{l+s+s1}{\PYGZsq{}}\PYG{l+s+s1}{close\PYGZus{}r3}\PYG{l+s+s1}{\PYGZsq{}}\PYG{p}{,}\PYG{l+s+s1}{\PYGZsq{}}\PYG{l+s+s1}{close\PYGZus{}r4}\PYG{l+s+s1}{\PYGZsq{}}\PYG{p}{,}\PYG{l+s+s1}{\PYGZsq{}}\PYG{l+s+s1}{close\PYGZus{}r5}\PYG{l+s+s1}{\PYGZsq{}}\PYG{p}{]}\PYG{p}{,} \PYG{n}{inplace}\PYG{o}{=}\PYG{k+kc}{True}\PYG{p}{)} \PYG{c+c1}{\PYGZsh{} 주어진 컬럼 중에 missing 값이 있으면 행을 제거(dropna)하고, 자신을 덮어 씀(inplace=True).}
    
    \PYG{n}{mdl\PYGZus{}data} \PYG{o}{=} \PYG{n}{pd}\PYG{o}{.}\PYG{n}{concat}\PYG{p}{(}\PYG{p}{[}\PYG{n}{mdl\PYGZus{}data}\PYG{p}{,} \PYG{n}{df}\PYG{p}{]}\PYG{p}{,} \PYG{n}{axis}\PYG{o}{=}\PYG{l+m+mi}{0}\PYG{p}{)}    
    
\PYG{n}{mdl\PYGZus{}data}\PYG{o}{.}\PYG{n}{to\PYGZus{}pickle}\PYG{p}{(}\PYG{l+s+s1}{\PYGZsq{}}\PYG{l+s+s1}{mdl\PYGZus{}data.pkl}\PYG{l+s+s1}{\PYGZsq{}}\PYG{p}{)}
\end{sphinxVerbatim}

\end{sphinxuseclass}\end{sphinxVerbatimInput}

\end{sphinxuseclass}
\sphinxAtStartPar
 ‘max\_close’ 의 분포를 확인합니다.

\begin{sphinxuseclass}{cell}\begin{sphinxVerbatimInput}

\begin{sphinxuseclass}{cell_input}
\begin{sphinxVerbatim}[commandchars=\\\{\}]
\PYG{n}{mdl\PYGZus{}data} \PYG{o}{=} \PYG{n}{pd}\PYG{o}{.}\PYG{n}{read\PYGZus{}pickle}\PYG{p}{(}\PYG{l+s+s1}{\PYGZsq{}}\PYG{l+s+s1}{mdl\PYGZus{}data.pkl}\PYG{l+s+s1}{\PYGZsq{}}\PYG{p}{)}
\PYG{n+nb}{print}\PYG{p}{(}\PYG{n}{mdl\PYGZus{}data}\PYG{p}{[}\PYG{l+s+s1}{\PYGZsq{}}\PYG{l+s+s1}{max\PYGZus{}close}\PYG{l+s+s1}{\PYGZsq{}}\PYG{p}{]}\PYG{o}{.}\PYG{n}{describe}\PYG{p}{(}\PYG{n}{percentiles}\PYG{o}{=}\PYG{p}{[}\PYG{l+m+mf}{0.1}\PYG{p}{,} \PYG{l+m+mf}{0.2}\PYG{p}{,} \PYG{l+m+mf}{0.5}\PYG{p}{,} \PYG{l+m+mf}{0.8}\PYG{p}{,} \PYG{l+m+mf}{0.9}\PYG{p}{]}\PYG{p}{)}\PYG{p}{)}
\end{sphinxVerbatim}

\end{sphinxuseclass}\end{sphinxVerbatimInput}
\begin{sphinxVerbatimOutput}

\begin{sphinxuseclass}{cell_output}
\begin{sphinxVerbatim}[commandchars=\\\{\}]
count   426,517.00
mean          1.03
std           0.07
min           0.33
10\PYGZpc{}           0.98
20\PYGZpc{}           0.99
50\PYGZpc{}           1.02
80\PYGZpc{}           1.06
90\PYGZpc{}           1.10
max           3.70
Name: max\PYGZus{}close, dtype: float64
\end{sphinxVerbatim}

\end{sphinxuseclass}\end{sphinxVerbatimOutput}

\end{sphinxuseclass}

\part{ 매도 전략 데이터 프로세싱}
\label{\detokenize{chapter4/4.4.4_Data_Processing:br}}
\sphinxAtStartPar
모델 개발을 위해서는 매도 전략에 따는 수익을 계산을 할 수 있어야 합니다. 이번 장에서는 기본적인 몇 가지 전략의 수익율을 계산해보겠습니다. 저장해 둔 mdl\_data pickle 파일을 읽습니다. mdl\_data 는 수익률 결과값이 있는 데이터입니다.

\begin{sphinxuseclass}{cell}\begin{sphinxVerbatimInput}

\begin{sphinxuseclass}{cell_input}
\begin{sphinxVerbatim}[commandchars=\\\{\}]
\PYG{n}{mdl\PYGZus{}data} \PYG{o}{=} \PYG{n}{pd}\PYG{o}{.}\PYG{n}{read\PYGZus{}pickle}\PYG{p}{(}\PYG{l+s+s1}{\PYGZsq{}}\PYG{l+s+s1}{mdl\PYGZus{}data.pkl}\PYG{l+s+s1}{\PYGZsq{}}\PYG{p}{)}
\end{sphinxVerbatim}

\end{sphinxuseclass}\end{sphinxVerbatimInput}

\end{sphinxuseclass}
\sphinxAtStartPar
 \sphinxstylestrong{매도 전략 1 \sphinxhyphen{} 모든 종목 종가 매수 후, 5 영업일 기간 6\% 익절 매도}한가지 전략을 테스트 해 보겠습니다. 모든 종목을 같은 금액으로 매일 종가 매수합니다. 매수 후 5 영업일 동안 수익율이 6\% 이상되면 곧바로 익절합니다. 나머지 종목은 5 영업일에 전부 종가 매도하면 수익율은 어떻게 될까요?

\begin{sphinxuseclass}{cell}\begin{sphinxVerbatimInput}

\begin{sphinxuseclass}{cell_input}
\begin{sphinxVerbatim}[commandchars=\\\{\}]
\PYG{n}{kosdaq\PYGZus{}list} \PYG{o}{=} \PYG{n}{pd}\PYG{o}{.}\PYG{n}{read\PYGZus{}pickle}\PYG{p}{(}\PYG{l+s+s1}{\PYGZsq{}}\PYG{l+s+s1}{kosdaq\PYGZus{}list.pkl}\PYG{l+s+s1}{\PYGZsq{}}\PYG{p}{)}

\PYG{n}{data\PYGZus{}all\PYGZus{}5} \PYG{o}{=} \PYG{n}{pd}\PYG{o}{.}\PYG{n}{DataFrame}\PYG{p}{(}\PYG{p}{)}

\PYG{n}{ub} \PYG{o}{=} \PYG{l+m+mf}{1.06}

\PYG{k}{for} \PYG{n}{code} \PYG{o+ow}{in} \PYG{n}{kosdaq\PYGZus{}list}\PYG{p}{[}\PYG{l+s+s1}{\PYGZsq{}}\PYG{l+s+s1}{code}\PYG{l+s+s1}{\PYGZsq{}}\PYG{p}{]}\PYG{p}{:}
    
    \PYG{c+c1}{\PYGZsh{} 종목별 처리}
    \PYG{n}{data} \PYG{o}{=} \PYG{n}{mdl\PYGZus{}data}\PYG{p}{[}\PYG{n}{mdl\PYGZus{}data}\PYG{p}{[}\PYG{l+s+s1}{\PYGZsq{}}\PYG{l+s+s1}{code}\PYG{l+s+s1}{\PYGZsq{}}\PYG{p}{]}\PYG{o}{==}\PYG{n}{code}\PYG{p}{]}\PYG{o}{.}\PYG{n}{sort\PYGZus{}index}\PYG{p}{(}\PYG{p}{)}\PYG{o}{.}\PYG{n}{copy}\PYG{p}{(}\PYG{p}{)}
    
    \PYG{c+c1}{\PYGZsh{} 고가, 저가, 종가 수익율}
    \PYG{k}{for} \PYG{n}{i} \PYG{o+ow}{in} \PYG{p}{[}\PYG{l+m+mi}{1}\PYG{p}{,} \PYG{l+m+mi}{2}\PYG{p}{,} \PYG{l+m+mi}{3}\PYG{p}{,} \PYG{l+m+mi}{4}\PYG{p}{,} \PYG{l+m+mi}{5}\PYG{p}{]}\PYG{p}{:}

        \PYG{n}{data}\PYG{p}{[}\PYG{l+s+s1}{\PYGZsq{}}\PYG{l+s+s1}{high\PYGZus{}r}\PYG{l+s+s1}{\PYGZsq{}} \PYG{o}{+} \PYG{n+nb}{str}\PYG{p}{(}\PYG{n}{i}\PYG{p}{)}\PYG{p}{]} \PYG{o}{=} \PYG{n}{data}\PYG{p}{[}\PYG{l+s+s1}{\PYGZsq{}}\PYG{l+s+s1}{high}\PYG{l+s+s1}{\PYGZsq{}}\PYG{p}{]}\PYG{o}{.}\PYG{n}{shift}\PYG{p}{(}\PYG{o}{\PYGZhy{}}\PYG{l+m+mi}{1}\PYG{o}{*}\PYG{n}{i}\PYG{p}{)}\PYG{o}{/}\PYG{n}{data}\PYG{p}{[}\PYG{l+s+s1}{\PYGZsq{}}\PYG{l+s+s1}{close}\PYG{l+s+s1}{\PYGZsq{}}\PYG{p}{]}      
        \PYG{n}{data}\PYG{p}{[}\PYG{l+s+s1}{\PYGZsq{}}\PYG{l+s+s1}{low\PYGZus{}r}\PYG{l+s+s1}{\PYGZsq{}} \PYG{o}{+} \PYG{n+nb}{str}\PYG{p}{(}\PYG{n}{i}\PYG{p}{)}\PYG{p}{]} \PYG{o}{=} \PYG{n}{data}\PYG{p}{[}\PYG{l+s+s1}{\PYGZsq{}}\PYG{l+s+s1}{low}\PYG{l+s+s1}{\PYGZsq{}}\PYG{p}{]}\PYG{o}{.}\PYG{n}{shift}\PYG{p}{(}\PYG{o}{\PYGZhy{}}\PYG{l+m+mi}{1}\PYG{o}{*}\PYG{n}{i}\PYG{p}{)}\PYG{o}{/}\PYG{n}{data}\PYG{p}{[}\PYG{l+s+s1}{\PYGZsq{}}\PYG{l+s+s1}{close}\PYG{l+s+s1}{\PYGZsq{}}\PYG{p}{]}   
        \PYG{n}{data}\PYG{p}{[}\PYG{l+s+s1}{\PYGZsq{}}\PYG{l+s+s1}{close\PYGZus{}r}\PYG{l+s+s1}{\PYGZsq{}} \PYG{o}{+} \PYG{n+nb}{str}\PYG{p}{(}\PYG{n}{i}\PYG{p}{)}\PYG{p}{]} \PYG{o}{=} \PYG{n}{data}\PYG{p}{[}\PYG{l+s+s1}{\PYGZsq{}}\PYG{l+s+s1}{close}\PYG{l+s+s1}{\PYGZsq{}}\PYG{p}{]}\PYG{o}{.}\PYG{n}{shift}\PYG{p}{(}\PYG{o}{\PYGZhy{}}\PYG{l+m+mi}{1}\PYG{o}{*}\PYG{n}{i}\PYG{p}{)}\PYG{o}{/}\PYG{n}{data}\PYG{p}{[}\PYG{l+s+s1}{\PYGZsq{}}\PYG{l+s+s1}{close}\PYG{l+s+s1}{\PYGZsq{}}\PYG{p}{]}    
        
    \PYG{n}{data}\PYG{p}{[}\PYG{l+s+s1}{\PYGZsq{}}\PYG{l+s+s1}{max\PYGZus{}high}\PYG{l+s+s1}{\PYGZsq{}}\PYG{p}{]}  \PYG{o}{=} \PYG{p}{(}\PYG{n}{data}\PYG{p}{[}\PYG{p}{[}\PYG{l+s+s1}{\PYGZsq{}}\PYG{l+s+s1}{high\PYGZus{}r1}\PYG{l+s+s1}{\PYGZsq{}}\PYG{p}{,}\PYG{l+s+s1}{\PYGZsq{}}\PYG{l+s+s1}{high\PYGZus{}r2}\PYG{l+s+s1}{\PYGZsq{}}\PYG{p}{,}\PYG{l+s+s1}{\PYGZsq{}}\PYG{l+s+s1}{high\PYGZus{}r3}\PYG{l+s+s1}{\PYGZsq{}}\PYG{p}{,}\PYG{l+s+s1}{\PYGZsq{}}\PYG{l+s+s1}{high\PYGZus{}r4}\PYG{l+s+s1}{\PYGZsq{}}\PYG{p}{,}\PYG{l+s+s1}{\PYGZsq{}}\PYG{l+s+s1}{high\PYGZus{}r5}\PYG{l+s+s1}{\PYGZsq{}}\PYG{p}{]}\PYG{p}{]}\PYG{o}{.}\PYG{n}{max}\PYG{p}{(}\PYG{n}{axis}\PYG{o}{=}\PYG{l+m+mi}{1}\PYG{p}{)} \PYG{o}{\PYGZgt{}} \PYG{n}{ub}\PYG{p}{)}\PYG{o}{.}\PYG{n}{astype}\PYG{p}{(}\PYG{n+nb}{int}\PYG{p}{)} \PYG{c+c1}{\PYGZsh{} 5 영업일 최고가 중 최고가         }
    \PYG{n}{data}\PYG{p}{[}\PYG{l+s+s1}{\PYGZsq{}}\PYG{l+s+s1}{ub\PYGZus{}return}\PYG{l+s+s1}{\PYGZsq{}}\PYG{p}{]} \PYG{o}{=} \PYG{n}{np}\PYG{o}{.}\PYG{n}{where}\PYG{p}{(}\PYG{n}{data}\PYG{p}{[}\PYG{l+s+s1}{\PYGZsq{}}\PYG{l+s+s1}{max\PYGZus{}high}\PYG{l+s+s1}{\PYGZsq{}}\PYG{p}{]}\PYG{o}{==}\PYG{l+m+mi}{1}\PYG{p}{,} \PYG{n}{ub}\PYG{p}{,} \PYG{n}{data}\PYG{p}{[}\PYG{l+s+s1}{\PYGZsq{}}\PYG{l+s+s1}{close\PYGZus{}r5}\PYG{l+s+s1}{\PYGZsq{}}\PYG{p}{]}\PYG{p}{)} \PYG{c+c1}{\PYGZsh{} 종가 수익률이 6\PYGZpc{} 이면 매도, 아니면 마지막 5 영업일 수익률}
       
    \PYG{n}{data}\PYG{o}{.}\PYG{n}{dropna}\PYG{p}{(}\PYG{n}{subset}\PYG{o}{=}\PYG{p}{[}\PYG{l+s+s1}{\PYGZsq{}}\PYG{l+s+s1}{close\PYGZus{}r1}\PYG{l+s+s1}{\PYGZsq{}}\PYG{p}{,}\PYG{l+s+s1}{\PYGZsq{}}\PYG{l+s+s1}{close\PYGZus{}r2}\PYG{l+s+s1}{\PYGZsq{}}\PYG{p}{,}\PYG{l+s+s1}{\PYGZsq{}}\PYG{l+s+s1}{close\PYGZus{}r3}\PYG{l+s+s1}{\PYGZsq{}}\PYG{p}{,}\PYG{l+s+s1}{\PYGZsq{}}\PYG{l+s+s1}{close\PYGZus{}r4}\PYG{l+s+s1}{\PYGZsq{}}\PYG{p}{,}\PYG{l+s+s1}{\PYGZsq{}}\PYG{l+s+s1}{close\PYGZus{}r5}\PYG{l+s+s1}{\PYGZsq{}}\PYG{p}{]}\PYG{p}{,} \PYG{n}{inplace}\PYG{o}{=}\PYG{k+kc}{True}\PYG{p}{)}   
    \PYG{n}{data\PYGZus{}all\PYGZus{}5} \PYG{o}{=} \PYG{n}{pd}\PYG{o}{.}\PYG{n}{concat}\PYG{p}{(}\PYG{p}{[}\PYG{n}{data}\PYG{p}{,} \PYG{n}{data\PYGZus{}all\PYGZus{}5}\PYG{p}{]}\PYG{p}{,} \PYG{n}{axis}\PYG{o}{=}\PYG{l+m+mi}{0}\PYG{p}{)}

\PYG{n}{data\PYGZus{}all\PYGZus{}5}\PYG{o}{.}\PYG{n}{to\PYGZus{}pickle}\PYG{p}{(}\PYG{l+s+s1}{\PYGZsq{}}\PYG{l+s+s1}{data\PYGZus{}all\PYGZus{}5.pkl}\PYG{l+s+s1}{\PYGZsq{}}\PYG{p}{)} 
\end{sphinxVerbatim}

\end{sphinxuseclass}\end{sphinxVerbatimInput}

\end{sphinxuseclass}
\begin{sphinxuseclass}{cell}\begin{sphinxVerbatimInput}

\begin{sphinxuseclass}{cell_input}
\begin{sphinxVerbatim}[commandchars=\\\{\}]
\PYG{n}{data\PYGZus{}all\PYGZus{}5}\PYG{o}{.}\PYG{n}{head}\PYG{p}{(}\PYG{p}{)}\PYG{o}{.}\PYG{n}{style}\PYG{o}{.}\PYG{n}{set\PYGZus{}table\PYGZus{}attributes}\PYG{p}{(}\PYG{l+s+s1}{\PYGZsq{}}\PYG{l+s+s1}{style=}\PYG{l+s+s1}{\PYGZdq{}}\PYG{l+s+s1}{font\PYGZhy{}size: 12px}\PYG{l+s+s1}{\PYGZdq{}}\PYG{l+s+s1}{\PYGZsq{}}\PYG{p}{)}
\end{sphinxVerbatim}

\end{sphinxuseclass}\end{sphinxVerbatimInput}
\begin{sphinxVerbatimOutput}

\begin{sphinxuseclass}{cell_output}
\begin{sphinxVerbatim}[commandchars=\\\{\}]
\PYGZlt{}pandas.io.formats.style.Styler at 0x200186cb670\PYGZgt{}
\end{sphinxVerbatim}

\end{sphinxuseclass}\end{sphinxVerbatimOutput}

\end{sphinxuseclass}
\sphinxAtStartPar
 수익률의 분포를 확인합니다. 수익이 되는 전략은 아닙니다.

\begin{sphinxuseclass}{cell}\begin{sphinxVerbatimInput}

\begin{sphinxuseclass}{cell_input}
\begin{sphinxVerbatim}[commandchars=\\\{\}]
\PYG{n}{data\PYGZus{}all\PYGZus{}5} \PYG{o}{=} \PYG{n}{pd}\PYG{o}{.}\PYG{n}{read\PYGZus{}pickle}\PYG{p}{(}\PYG{l+s+s1}{\PYGZsq{}}\PYG{l+s+s1}{data\PYGZus{}all\PYGZus{}5.pkl}\PYG{l+s+s1}{\PYGZsq{}}\PYG{p}{)}
\PYG{n+nb}{print}\PYG{p}{(}\PYG{n}{data\PYGZus{}all\PYGZus{}5}\PYG{p}{[}\PYG{l+s+s1}{\PYGZsq{}}\PYG{l+s+s1}{ub\PYGZus{}return}\PYG{l+s+s1}{\PYGZsq{}}\PYG{p}{]}\PYG{o}{.}\PYG{n}{describe}\PYG{p}{(}\PYG{n}{percentiles}\PYG{o}{=}\PYG{p}{[}\PYG{l+m+mf}{0.01}\PYG{p}{,} \PYG{l+m+mf}{0.1}\PYG{p}{,} \PYG{l+m+mf}{0.5}\PYG{p}{,} \PYG{l+m+mf}{0.9}\PYG{p}{,} \PYG{l+m+mf}{0.99}\PYG{p}{]}\PYG{p}{)}\PYG{p}{)}
\PYG{n+nb}{print}\PYG{p}{(}\PYG{n}{data\PYGZus{}all\PYGZus{}5}\PYG{o}{.}\PYG{n}{groupby}\PYG{p}{(}\PYG{l+s+s1}{\PYGZsq{}}\PYG{l+s+s1}{max\PYGZus{}high}\PYG{l+s+s1}{\PYGZsq{}}\PYG{p}{)}\PYG{p}{[}\PYG{l+s+s1}{\PYGZsq{}}\PYG{l+s+s1}{ub\PYGZus{}return}\PYG{l+s+s1}{\PYGZsq{}}\PYG{p}{]}\PYG{o}{.}\PYG{n}{describe}\PYG{p}{(}\PYG{p}{)}\PYG{p}{)}
\end{sphinxVerbatim}

\end{sphinxuseclass}\end{sphinxVerbatimInput}
\begin{sphinxVerbatimOutput}

\begin{sphinxuseclass}{cell_output}
\begin{sphinxVerbatim}[commandchars=\\\{\}]
count   419,432.00
mean          1.00
std           0.06
min           0.29
1\PYGZpc{}            0.84
10\PYGZpc{}           0.93
50\PYGZpc{}           1.00
90\PYGZpc{}           1.06
99\PYGZpc{}           1.06
max           1.06
Name: ub\PYGZus{}return, dtype: float64
              count  mean  std  min  25\PYGZpc{}  50\PYGZpc{}  75\PYGZpc{}  max
max\PYGZus{}high                                               
0        280,145.00  0.97 0.05 0.29 0.95 0.98 1.00 1.06
1        139,287.00  1.06 0.00 1.06 1.06 1.06 1.06 1.06
\end{sphinxVerbatim}

\end{sphinxuseclass}\end{sphinxVerbatimOutput}

\end{sphinxuseclass}
\sphinxAtStartPar
 \sphinxstylestrong{매도 전략 2 \sphinxhyphen{} 모든 종목을 종가 매수 후, 아래와 같은 순서로 매도}
\begin{enumerate}
\sphinxsetlistlabels{\arabic}{enumi}{enumii}{}{.}%
\item {} 
\sphinxAtStartPar
익일 고가가 당일 고가 보다 크면 2 영업일 시가 매도

\item {} 
\sphinxAtStartPar
1 조건 만족하지 않으면 2 영업일 종가 매도

\end{enumerate}

\sphinxAtStartPar
위와 같은 매도 전략은 수익율이 어떻게 될까요?

\begin{sphinxuseclass}{cell}\begin{sphinxVerbatimInput}

\begin{sphinxuseclass}{cell_input}
\begin{sphinxVerbatim}[commandchars=\\\{\}]
\PYG{n}{kosdaq\PYGZus{}list} \PYG{o}{=} \PYG{n}{pd}\PYG{o}{.}\PYG{n}{read\PYGZus{}pickle}\PYG{p}{(}\PYG{l+s+s1}{\PYGZsq{}}\PYG{l+s+s1}{kosdaq\PYGZus{}list.pkl}\PYG{l+s+s1}{\PYGZsq{}}\PYG{p}{)}

\PYG{n}{data\PYGZus{}all\PYGZus{}5} \PYG{o}{=} \PYG{n}{pd}\PYG{o}{.}\PYG{n}{DataFrame}\PYG{p}{(}\PYG{p}{)}

\PYG{k}{def} \PYG{n+nf}{final\PYGZus{}r}\PYG{p}{(}\PYG{n}{x}\PYG{p}{)}\PYG{p}{:}
    
    \PYG{k}{if}   \PYG{n}{x}\PYG{p}{[}\PYG{l+s+s1}{\PYGZsq{}}\PYG{l+s+s1}{high\PYGZus{}r0}\PYG{l+s+s1}{\PYGZsq{}}\PYG{p}{]} \PYG{o}{\PYGZlt{}} \PYG{n}{x}\PYG{p}{[}\PYG{l+s+s1}{\PYGZsq{}}\PYG{l+s+s1}{high\PYGZus{}r1}\PYG{l+s+s1}{\PYGZsq{}}\PYG{p}{]}\PYG{p}{:}  \PYG{c+c1}{\PYGZsh{}  (당일 고가/매수 종가) 비율이 (익일 고가/매수 종가) 비율 값이 작으면 2 영업일 시가 매도     }
        \PYG{k}{return} \PYG{n}{x}\PYG{p}{[}\PYG{l+s+s1}{\PYGZsq{}}\PYG{l+s+s1}{open\PYGZus{}r2}\PYG{l+s+s1}{\PYGZsq{}}\PYG{p}{]}    
    
    \PYG{k}{else}\PYG{p}{:}
        \PYG{k}{return} \PYG{n}{x}\PYG{p}{[}\PYG{l+s+s1}{\PYGZsq{}}\PYG{l+s+s1}{close\PYGZus{}r2}\PYG{l+s+s1}{\PYGZsq{}}\PYG{p}{]} \PYG{c+c1}{\PYGZsh{} 매도 안된 종목은 전부 2 영업일 종가 매도         }
    
\PYG{k}{for} \PYG{n}{code} \PYG{o+ow}{in} \PYG{n}{kosdaq\PYGZus{}list}\PYG{p}{[}\PYG{l+s+s1}{\PYGZsq{}}\PYG{l+s+s1}{code}\PYG{l+s+s1}{\PYGZsq{}}\PYG{p}{]}\PYG{p}{:}    
    
    \PYG{c+c1}{\PYGZsh{} 종목별 처리}
    \PYG{n}{data} \PYG{o}{=} \PYG{n}{mdl\PYGZus{}data}\PYG{p}{[}\PYG{n}{mdl\PYGZus{}data}\PYG{p}{[}\PYG{l+s+s1}{\PYGZsq{}}\PYG{l+s+s1}{code}\PYG{l+s+s1}{\PYGZsq{}}\PYG{p}{]}\PYG{o}{==}\PYG{n}{code}\PYG{p}{]}\PYG{o}{.}\PYG{n}{sort\PYGZus{}index}\PYG{p}{(}\PYG{p}{)}\PYG{o}{.}\PYG{n}{copy}\PYG{p}{(}\PYG{p}{)}
    
    \PYG{c+c1}{\PYGZsh{} 최고/최저 수익율}
    \PYG{k}{for} \PYG{n}{i} \PYG{o+ow}{in} \PYG{p}{[}\PYG{l+m+mi}{0}\PYG{p}{,} \PYG{l+m+mi}{1}\PYG{p}{,} \PYG{l+m+mi}{2}\PYG{p}{]}\PYG{p}{:}

        \PYG{n}{data}\PYG{p}{[}\PYG{l+s+s1}{\PYGZsq{}}\PYG{l+s+s1}{high\PYGZus{}r}\PYG{l+s+s1}{\PYGZsq{}} \PYG{o}{+} \PYG{n+nb}{str}\PYG{p}{(}\PYG{n}{i}\PYG{p}{)}\PYG{p}{]} \PYG{o}{=} \PYG{n}{data}\PYG{p}{[}\PYG{l+s+s1}{\PYGZsq{}}\PYG{l+s+s1}{high}\PYG{l+s+s1}{\PYGZsq{}}\PYG{p}{]}\PYG{o}{.}\PYG{n}{shift}\PYG{p}{(}\PYG{o}{\PYGZhy{}}\PYG{l+m+mi}{1}\PYG{o}{*}\PYG{n}{i}\PYG{p}{)}\PYG{o}{/}\PYG{n}{data}\PYG{p}{[}\PYG{l+s+s1}{\PYGZsq{}}\PYG{l+s+s1}{close}\PYG{l+s+s1}{\PYGZsq{}}\PYG{p}{]}        
        \PYG{n}{data}\PYG{p}{[}\PYG{l+s+s1}{\PYGZsq{}}\PYG{l+s+s1}{close\PYGZus{}r}\PYG{l+s+s1}{\PYGZsq{}} \PYG{o}{+} \PYG{n+nb}{str}\PYG{p}{(}\PYG{n}{i}\PYG{p}{)}\PYG{p}{]} \PYG{o}{=} \PYG{n}{data}\PYG{p}{[}\PYG{l+s+s1}{\PYGZsq{}}\PYG{l+s+s1}{close}\PYG{l+s+s1}{\PYGZsq{}}\PYG{p}{]}\PYG{o}{.}\PYG{n}{shift}\PYG{p}{(}\PYG{o}{\PYGZhy{}}\PYG{l+m+mi}{1}\PYG{o}{*}\PYG{n}{i}\PYG{p}{)}\PYG{o}{/}\PYG{n}{data}\PYG{p}{[}\PYG{l+s+s1}{\PYGZsq{}}\PYG{l+s+s1}{close}\PYG{l+s+s1}{\PYGZsq{}}\PYG{p}{]}
        \PYG{n}{data}\PYG{p}{[}\PYG{l+s+s1}{\PYGZsq{}}\PYG{l+s+s1}{open\PYGZus{}r}\PYG{l+s+s1}{\PYGZsq{}} \PYG{o}{+} \PYG{n+nb}{str}\PYG{p}{(}\PYG{n}{i}\PYG{p}{)}\PYG{p}{]} \PYG{o}{=} \PYG{n}{data}\PYG{p}{[}\PYG{l+s+s1}{\PYGZsq{}}\PYG{l+s+s1}{open}\PYG{l+s+s1}{\PYGZsq{}}\PYG{p}{]}\PYG{o}{.}\PYG{n}{shift}\PYG{p}{(}\PYG{o}{\PYGZhy{}}\PYG{l+m+mi}{1}\PYG{o}{*}\PYG{n}{i}\PYG{p}{)}\PYG{o}{/}\PYG{n}{data}\PYG{p}{[}\PYG{l+s+s1}{\PYGZsq{}}\PYG{l+s+s1}{close}\PYG{l+s+s1}{\PYGZsq{}}\PYG{p}{]}
        
    \PYG{n}{data}\PYG{p}{[}\PYG{l+s+s1}{\PYGZsq{}}\PYG{l+s+s1}{final\PYGZus{}return}\PYG{l+s+s1}{\PYGZsq{}}\PYG{p}{]} \PYG{o}{=} \PYG{n}{data}\PYG{o}{.}\PYG{n}{apply}\PYG{p}{(}\PYG{n}{final\PYGZus{}r}\PYG{p}{,} \PYG{n}{axis}\PYG{o}{=}\PYG{l+m+mi}{1}\PYG{p}{)} \PYG{c+c1}{\PYGZsh{} 각 row 에 대하여 final\PYGZus{}r 함수를 적용}
                                                                                                                                                 
    \PYG{n}{data}\PYG{o}{.}\PYG{n}{dropna}\PYG{p}{(}\PYG{n}{subset}\PYG{o}{=}\PYG{p}{[}\PYG{l+s+s1}{\PYGZsq{}}\PYG{l+s+s1}{close\PYGZus{}r0}\PYG{l+s+s1}{\PYGZsq{}}\PYG{p}{,}\PYG{l+s+s1}{\PYGZsq{}}\PYG{l+s+s1}{close\PYGZus{}r1}\PYG{l+s+s1}{\PYGZsq{}}\PYG{p}{,} \PYG{l+s+s1}{\PYGZsq{}}\PYG{l+s+s1}{close\PYGZus{}r2}\PYG{l+s+s1}{\PYGZsq{}}\PYG{p}{,}\PYG{l+s+s1}{\PYGZsq{}}\PYG{l+s+s1}{high\PYGZus{}r0}\PYG{l+s+s1}{\PYGZsq{}}\PYG{p}{,} \PYG{l+s+s1}{\PYGZsq{}}\PYG{l+s+s1}{high\PYGZus{}r1}\PYG{l+s+s1}{\PYGZsq{}}\PYG{p}{,} \PYG{l+s+s1}{\PYGZsq{}}\PYG{l+s+s1}{open\PYGZus{}r2}\PYG{l+s+s1}{\PYGZsq{}}\PYG{p}{]}\PYG{p}{,} \PYG{n}{inplace}\PYG{o}{=}\PYG{k+kc}{True}\PYG{p}{)}   \PYG{c+c1}{\PYGZsh{} 데아터 처리 중 missing 값이 사용된 경우는 제거}
    \PYG{n}{data\PYGZus{}all\PYGZus{}5} \PYG{o}{=} \PYG{n}{pd}\PYG{o}{.}\PYG{n}{concat}\PYG{p}{(}\PYG{p}{[}\PYG{n}{data}\PYG{p}{,} \PYG{n}{data\PYGZus{}all\PYGZus{}5}\PYG{p}{]}\PYG{p}{,} \PYG{n}{axis}\PYG{o}{=}\PYG{l+m+mi}{0}\PYG{p}{)}

\PYG{n}{data\PYGZus{}all\PYGZus{}5}\PYG{o}{.}\PYG{n}{to\PYGZus{}pickle}\PYG{p}{(}\PYG{l+s+s1}{\PYGZsq{}}\PYG{l+s+s1}{data\PYGZus{}all\PYGZus{}5.pkl}\PYG{l+s+s1}{\PYGZsq{}}\PYG{p}{)}    
\end{sphinxVerbatim}

\end{sphinxuseclass}\end{sphinxVerbatimInput}

\end{sphinxuseclass}
\sphinxAtStartPar
 수익률을 확인합니다.

\begin{sphinxuseclass}{cell}\begin{sphinxVerbatimInput}

\begin{sphinxuseclass}{cell_input}
\begin{sphinxVerbatim}[commandchars=\\\{\}]
\PYG{n}{data\PYGZus{}all\PYGZus{}5} \PYG{o}{=} \PYG{n}{pd}\PYG{o}{.}\PYG{n}{read\PYGZus{}pickle}\PYG{p}{(}\PYG{l+s+s1}{\PYGZsq{}}\PYG{l+s+s1}{data\PYGZus{}all\PYGZus{}5.pkl}\PYG{l+s+s1}{\PYGZsq{}}\PYG{p}{)}  
\PYG{n}{data\PYGZus{}all\PYGZus{}5}\PYG{p}{[}\PYG{l+s+s1}{\PYGZsq{}}\PYG{l+s+s1}{final\PYGZus{}return}\PYG{l+s+s1}{\PYGZsq{}}\PYG{p}{]}\PYG{o}{.}\PYG{n}{describe}\PYG{p}{(}\PYG{n}{percentiles}\PYG{o}{=}\PYG{p}{[}\PYG{l+m+mf}{0.01}\PYG{p}{,} \PYG{l+m+mf}{0.1}\PYG{p}{,} \PYG{l+m+mf}{0.5}\PYG{p}{,} \PYG{l+m+mf}{0.9}\PYG{p}{,} \PYG{l+m+mf}{0.99}\PYG{p}{]}\PYG{p}{)}
\end{sphinxVerbatim}

\end{sphinxuseclass}\end{sphinxVerbatimInput}
\begin{sphinxVerbatimOutput}

\begin{sphinxuseclass}{cell_output}
\begin{sphinxVerbatim}[commandchars=\\\{\}]
count   423,683.00
mean          1.00
std           0.05
min           0.00
1\PYGZpc{}            0.89
10\PYGZpc{}           0.96
50\PYGZpc{}           1.00
90\PYGZpc{}           1.04
99\PYGZpc{}           1.15
max           1.69
Name: final\PYGZus{}return, dtype: float64
\end{sphinxVerbatim}

\end{sphinxuseclass}\end{sphinxVerbatimOutput}

\end{sphinxuseclass}

\part{ 매수 전략 데이터 프로세싱}
\label{\detokenize{chapter4/4.4.4_Data_Processing:id2}}
\sphinxAtStartPar
모델 개발을 위해서는 매수 전략에 따라 매수 종목을 결정할 수 있어야 합니다. 이번 장에서는 기본적인 매수 종목을 찾는 데이터처리를 진행해 보겠습니다. 결과 수익률 데이터가 있는
mdl\_data pickle 파일을 읽습니다.

\begin{sphinxuseclass}{cell}\begin{sphinxVerbatimInput}

\begin{sphinxuseclass}{cell_input}
\begin{sphinxVerbatim}[commandchars=\\\{\}]
\PYG{n}{mdl\PYGZus{}data} \PYG{o}{=} \PYG{n}{pd}\PYG{o}{.}\PYG{n}{read\PYGZus{}pickle}\PYG{p}{(}\PYG{l+s+s1}{\PYGZsq{}}\PYG{l+s+s1}{mdl\PYGZus{}data.pkl}\PYG{l+s+s1}{\PYGZsq{}}\PYG{p}{)}
\end{sphinxVerbatim}

\end{sphinxuseclass}\end{sphinxVerbatimInput}

\end{sphinxuseclass}
\sphinxAtStartPar
 \sphinxstylestrong{매수 전략 1 \sphinxhyphen{} 시장 수익율보다 더 좋은 수익율을 보인 종목을 매수}시장 수익율보다 더 좋은 수익율을 보인 종목을 알기 위해 4.4.5 절에 ‘win\_market’ 이라는 변수를 생성했습니다. 이것을 이용할 것인데요. 더 의미있는 지표를 생성하기 위해서 과거 60일 누적 합을 보겠습니다. 수익율은 max\_close(5 영업일 중 최고 종가 수익율) 이용하겠습니다.

\begin{sphinxuseclass}{cell}\begin{sphinxVerbatimInput}

\begin{sphinxuseclass}{cell_input}
\begin{sphinxVerbatim}[commandchars=\\\{\}]
\PYG{n}{kosdaq\PYGZus{}list} \PYG{o}{=} \PYG{n}{pd}\PYG{o}{.}\PYG{n}{read\PYGZus{}pickle}\PYG{p}{(}\PYG{l+s+s1}{\PYGZsq{}}\PYG{l+s+s1}{kosdaq\PYGZus{}list.pkl}\PYG{l+s+s1}{\PYGZsq{}}\PYG{p}{)}

\PYG{n}{data\PYGZus{}all\PYGZus{}6} \PYG{o}{=} \PYG{n}{pd}\PYG{o}{.}\PYG{n}{DataFrame}\PYG{p}{(}\PYG{p}{)}

\PYG{k}{for} \PYG{n}{code} \PYG{o+ow}{in} \PYG{n}{kosdaq\PYGZus{}list}\PYG{p}{[}\PYG{l+s+s1}{\PYGZsq{}}\PYG{l+s+s1}{code}\PYG{l+s+s1}{\PYGZsq{}}\PYG{p}{]}\PYG{p}{:}
    
    \PYG{c+c1}{\PYGZsh{} 종목별 처리}
    \PYG{n}{data} \PYG{o}{=} \PYG{n}{mdl\PYGZus{}data}\PYG{p}{[}\PYG{n}{mdl\PYGZus{}data}\PYG{p}{[}\PYG{l+s+s1}{\PYGZsq{}}\PYG{l+s+s1}{code}\PYG{l+s+s1}{\PYGZsq{}}\PYG{p}{]}\PYG{o}{==}\PYG{n}{code}\PYG{p}{]}\PYG{o}{.}\PYG{n}{sort\PYGZus{}index}\PYG{p}{(}\PYG{p}{)}\PYG{o}{.}\PYG{n}{copy}\PYG{p}{(}\PYG{p}{)}
    
    \PYG{c+c1}{\PYGZsh{} 과거 60일 win\PYGZus{}market 누적 합}
    \PYG{n}{data}\PYG{p}{[}\PYG{l+s+s1}{\PYGZsq{}}\PYG{l+s+s1}{win\PYGZus{}market\PYGZus{}sum}\PYG{l+s+s1}{\PYGZsq{}}\PYG{p}{]} \PYG{o}{=} \PYG{n}{data}\PYG{p}{[}\PYG{l+s+s1}{\PYGZsq{}}\PYG{l+s+s1}{win\PYGZus{}market}\PYG{l+s+s1}{\PYGZsq{}}\PYG{p}{]}\PYG{o}{.}\PYG{n}{rolling}\PYG{p}{(}\PYG{l+m+mi}{60}\PYG{p}{)}\PYG{o}{.}\PYG{n}{sum}\PYG{p}{(}\PYG{p}{)} \PYG{c+c1}{\PYGZsh{} 과거 60일 누적합}
    
    \PYG{c+c1}{\PYGZsh{} 고가, 저가, 종가 수익율}
    \PYG{k}{for} \PYG{n}{i} \PYG{o+ow}{in} \PYG{p}{[}\PYG{l+m+mi}{1}\PYG{p}{,}\PYG{l+m+mi}{2}\PYG{p}{,}\PYG{l+m+mi}{3}\PYG{p}{,}\PYG{l+m+mi}{4}\PYG{p}{,}\PYG{l+m+mi}{5}\PYG{p}{]}\PYG{p}{:}

        \PYG{n}{data}\PYG{p}{[}\PYG{l+s+s1}{\PYGZsq{}}\PYG{l+s+s1}{high\PYGZus{}r}\PYG{l+s+s1}{\PYGZsq{}} \PYG{o}{+} \PYG{n+nb}{str}\PYG{p}{(}\PYG{n}{i}\PYG{p}{)}\PYG{p}{]} \PYG{o}{=} \PYG{n}{data}\PYG{p}{[}\PYG{l+s+s1}{\PYGZsq{}}\PYG{l+s+s1}{high}\PYG{l+s+s1}{\PYGZsq{}}\PYG{p}{]}\PYG{o}{.}\PYG{n}{shift}\PYG{p}{(}\PYG{o}{\PYGZhy{}}\PYG{l+m+mi}{1}\PYG{o}{*}\PYG{n}{i}\PYG{p}{)}\PYG{o}{/}\PYG{n}{data}\PYG{p}{[}\PYG{l+s+s1}{\PYGZsq{}}\PYG{l+s+s1}{close}\PYG{l+s+s1}{\PYGZsq{}}\PYG{p}{]}      
        \PYG{n}{data}\PYG{p}{[}\PYG{l+s+s1}{\PYGZsq{}}\PYG{l+s+s1}{low\PYGZus{}r}\PYG{l+s+s1}{\PYGZsq{}} \PYG{o}{+} \PYG{n+nb}{str}\PYG{p}{(}\PYG{n}{i}\PYG{p}{)}\PYG{p}{]} \PYG{o}{=} \PYG{n}{data}\PYG{p}{[}\PYG{l+s+s1}{\PYGZsq{}}\PYG{l+s+s1}{low}\PYG{l+s+s1}{\PYGZsq{}}\PYG{p}{]}\PYG{o}{.}\PYG{n}{shift}\PYG{p}{(}\PYG{o}{\PYGZhy{}}\PYG{l+m+mi}{1}\PYG{o}{*}\PYG{n}{i}\PYG{p}{)}\PYG{o}{/}\PYG{n}{data}\PYG{p}{[}\PYG{l+s+s1}{\PYGZsq{}}\PYG{l+s+s1}{close}\PYG{l+s+s1}{\PYGZsq{}}\PYG{p}{]}   
        \PYG{n}{data}\PYG{p}{[}\PYG{l+s+s1}{\PYGZsq{}}\PYG{l+s+s1}{close\PYGZus{}r}\PYG{l+s+s1}{\PYGZsq{}} \PYG{o}{+} \PYG{n+nb}{str}\PYG{p}{(}\PYG{n}{i}\PYG{p}{)}\PYG{p}{]} \PYG{o}{=} \PYG{n}{data}\PYG{p}{[}\PYG{l+s+s1}{\PYGZsq{}}\PYG{l+s+s1}{close}\PYG{l+s+s1}{\PYGZsq{}}\PYG{p}{]}\PYG{o}{.}\PYG{n}{shift}\PYG{p}{(}\PYG{o}{\PYGZhy{}}\PYG{l+m+mi}{1}\PYG{o}{*}\PYG{n}{i}\PYG{p}{)}\PYG{o}{/}\PYG{n}{data}\PYG{p}{[}\PYG{l+s+s1}{\PYGZsq{}}\PYG{l+s+s1}{close}\PYG{l+s+s1}{\PYGZsq{}}\PYG{p}{]}    
        
    \PYG{n}{data}\PYG{p}{[}\PYG{l+s+s1}{\PYGZsq{}}\PYG{l+s+s1}{max\PYGZus{}close}\PYG{l+s+s1}{\PYGZsq{}}\PYG{p}{]}  \PYG{o}{=} \PYG{n}{data}\PYG{p}{[}\PYG{p}{[}\PYG{l+s+s1}{\PYGZsq{}}\PYG{l+s+s1}{close\PYGZus{}r1}\PYG{l+s+s1}{\PYGZsq{}}\PYG{p}{,}\PYG{l+s+s1}{\PYGZsq{}}\PYG{l+s+s1}{close\PYGZus{}r2}\PYG{l+s+s1}{\PYGZsq{}}\PYG{p}{,}\PYG{l+s+s1}{\PYGZsq{}}\PYG{l+s+s1}{close\PYGZus{}r3}\PYG{l+s+s1}{\PYGZsq{}}\PYG{p}{,}\PYG{l+s+s1}{\PYGZsq{}}\PYG{l+s+s1}{close\PYGZus{}r4}\PYG{l+s+s1}{\PYGZsq{}}\PYG{p}{,}\PYG{l+s+s1}{\PYGZsq{}}\PYG{l+s+s1}{close\PYGZus{}r5}\PYG{l+s+s1}{\PYGZsq{}}\PYG{p}{]}\PYG{p}{]}\PYG{o}{.}\PYG{n}{max}\PYG{p}{(}\PYG{n}{axis}\PYG{o}{=}\PYG{l+m+mi}{1}\PYG{p}{)} \PYG{c+c1}{\PYGZsh{} 5 영업일 종가 수익율 중 최고 값}
    \PYG{n}{data}\PYG{o}{.}\PYG{n}{dropna}\PYG{p}{(}\PYG{n}{subset}\PYG{o}{=}\PYG{p}{[}\PYG{l+s+s1}{\PYGZsq{}}\PYG{l+s+s1}{win\PYGZus{}market\PYGZus{}sum}\PYG{l+s+s1}{\PYGZsq{}}\PYG{p}{,}\PYG{l+s+s1}{\PYGZsq{}}\PYG{l+s+s1}{close\PYGZus{}r1}\PYG{l+s+s1}{\PYGZsq{}}\PYG{p}{,}\PYG{l+s+s1}{\PYGZsq{}}\PYG{l+s+s1}{close\PYGZus{}r2}\PYG{l+s+s1}{\PYGZsq{}}\PYG{p}{,}\PYG{l+s+s1}{\PYGZsq{}}\PYG{l+s+s1}{close\PYGZus{}r3}\PYG{l+s+s1}{\PYGZsq{}}\PYG{p}{,}\PYG{l+s+s1}{\PYGZsq{}}\PYG{l+s+s1}{close\PYGZus{}r4}\PYG{l+s+s1}{\PYGZsq{}}\PYG{p}{,}\PYG{l+s+s1}{\PYGZsq{}}\PYG{l+s+s1}{close\PYGZus{}r5}\PYG{l+s+s1}{\PYGZsq{}}\PYG{p}{]}\PYG{p}{,} \PYG{n}{inplace}\PYG{o}{=}\PYG{k+kc}{True}\PYG{p}{)} \PYG{c+c1}{\PYGZsh{} missing 이 있는 행은 제거   }
 
    \PYG{n}{data\PYGZus{}all\PYGZus{}6} \PYG{o}{=} \PYG{n}{pd}\PYG{o}{.}\PYG{n}{concat}\PYG{p}{(}\PYG{p}{[}\PYG{n}{data}\PYG{p}{,} \PYG{n}{data\PYGZus{}all\PYGZus{}6}\PYG{p}{]}\PYG{p}{,} \PYG{n}{axis}\PYG{o}{=}\PYG{l+m+mi}{0}\PYG{p}{)}

\PYG{n}{data\PYGZus{}all\PYGZus{}6}\PYG{o}{.}\PYG{n}{to\PYGZus{}pickle}\PYG{p}{(}\PYG{l+s+s1}{\PYGZsq{}}\PYG{l+s+s1}{data\PYGZus{}all\PYGZus{}6.pkl}\PYG{l+s+s1}{\PYGZsq{}}\PYG{p}{)}    
\PYG{n}{data\PYGZus{}all\PYGZus{}6}\PYG{o}{.}\PYG{n}{head}\PYG{p}{(}\PYG{p}{)}\PYG{o}{.}\PYG{n}{style}\PYG{o}{.}\PYG{n}{set\PYGZus{}table\PYGZus{}attributes}\PYG{p}{(}\PYG{l+s+s1}{\PYGZsq{}}\PYG{l+s+s1}{style=}\PYG{l+s+s1}{\PYGZdq{}}\PYG{l+s+s1}{font\PYGZhy{}size: 12px}\PYG{l+s+s1}{\PYGZdq{}}\PYG{l+s+s1}{\PYGZsq{}}\PYG{p}{)}
\end{sphinxVerbatim}

\end{sphinxuseclass}\end{sphinxVerbatimInput}
\begin{sphinxVerbatimOutput}

\begin{sphinxuseclass}{cell_output}
\begin{sphinxVerbatim}[commandchars=\\\{\}]
\PYGZlt{}pandas.io.formats.style.Styler at 0x2188364a1c0\PYGZgt{}
\end{sphinxVerbatim}

\end{sphinxuseclass}\end{sphinxVerbatimOutput}

\end{sphinxuseclass}
\sphinxAtStartPar
 win\_market\_sum 에 따른 수익률의 변화를 확인합니다. win\_market\_sum 이 클수록 수익률이 높아지는 경향이 있다는 것을 확인했습니다.

\begin{sphinxuseclass}{cell}\begin{sphinxVerbatimInput}

\begin{sphinxuseclass}{cell_input}
\begin{sphinxVerbatim}[commandchars=\\\{\}]
\PYG{n}{data\PYGZus{}all\PYGZus{}6} \PYG{o}{=} \PYG{n}{pd}\PYG{o}{.}\PYG{n}{read\PYGZus{}pickle}\PYG{p}{(}\PYG{l+s+s1}{\PYGZsq{}}\PYG{l+s+s1}{data\PYGZus{}all\PYGZus{}6.pkl}\PYG{l+s+s1}{\PYGZsq{}}\PYG{p}{)}    
\PYG{n}{ranks} \PYG{o}{=} \PYG{n}{pd}\PYG{o}{.}\PYG{n}{qcut}\PYG{p}{(}\PYG{n}{data\PYGZus{}all\PYGZus{}6}\PYG{p}{[}\PYG{l+s+s1}{\PYGZsq{}}\PYG{l+s+s1}{win\PYGZus{}market\PYGZus{}sum}\PYG{l+s+s1}{\PYGZsq{}}\PYG{p}{]}\PYG{p}{,} \PYG{n}{q}\PYG{o}{=}\PYG{l+m+mi}{8}\PYG{p}{)}
\PYG{n+nb}{print}\PYG{p}{(}\PYG{n}{data\PYGZus{}all\PYGZus{}6}\PYG{o}{.}\PYG{n}{groupby}\PYG{p}{(}\PYG{n}{ranks}\PYG{p}{)}\PYG{p}{[}\PYG{l+s+s1}{\PYGZsq{}}\PYG{l+s+s1}{max\PYGZus{}close}\PYG{l+s+s1}{\PYGZsq{}}\PYG{p}{]}\PYG{o}{.}\PYG{n}{mean}\PYG{p}{(}\PYG{p}{)}\PYG{p}{)}
\PYG{n}{data\PYGZus{}all\PYGZus{}6}\PYG{o}{.}\PYG{n}{groupby}\PYG{p}{(}\PYG{n}{ranks}\PYG{p}{)}\PYG{p}{[}\PYG{l+s+s1}{\PYGZsq{}}\PYG{l+s+s1}{max\PYGZus{}close}\PYG{l+s+s1}{\PYGZsq{}}\PYG{p}{]}\PYG{o}{.}\PYG{n}{mean}\PYG{p}{(}\PYG{p}{)}\PYG{o}{.}\PYG{n}{plot}\PYG{p}{(}\PYG{n}{figsize}\PYG{o}{=}\PYG{p}{(}\PYG{l+m+mi}{12}\PYG{p}{,}\PYG{l+m+mi}{5}\PYG{p}{)}\PYG{p}{)}
\end{sphinxVerbatim}

\end{sphinxuseclass}\end{sphinxVerbatimInput}
\begin{sphinxVerbatimOutput}

\begin{sphinxuseclass}{cell_output}
\begin{sphinxVerbatim}[commandchars=\\\{\}]
win\PYGZus{}market\PYGZus{}sum
(\PYGZhy{}0.001, 4.0]   1.02
(4.0, 5.0]      1.03
(5.0, 6.0]      1.03
(6.0, 7.0]      1.03
(7.0, 8.0]      1.03
(8.0, 9.0]      1.03
(9.0, 11.0]     1.04
(11.0, 22.0]    1.04
Name: max\PYGZus{}close, dtype: float64
\end{sphinxVerbatim}

\begin{sphinxVerbatim}[commandchars=\\\{\}]
\PYGZlt{}AxesSubplot:xlabel=\PYGZsq{}win\PYGZus{}market\PYGZus{}sum\PYGZsq{}\PYGZgt{}
\end{sphinxVerbatim}

\noindent\sphinxincludegraphics{{4.4.4_Data_Processing_27_2}.png}

\end{sphinxuseclass}\end{sphinxVerbatimOutput}

\end{sphinxuseclass}
\sphinxAtStartPar
 \sphinxstylestrong{매수 전략 2 \sphinxhyphen{} 섹터 평균 수익율보다 더 높은 수익율을 보인 종목을 매수}kosdaq\_list 에 있는 종목별 섹터 정보를 이용하겠습니다. 우선, 종목별 최근 60일 평균 수익율을 rolling 함수를 이용하여 으로 계산합니다. for Loop 을 이용하여 종목에 섹터 정보를 추가합니다.

\begin{sphinxuseclass}{cell}\begin{sphinxVerbatimInput}

\begin{sphinxuseclass}{cell_input}
\begin{sphinxVerbatim}[commandchars=\\\{\}]
\PYG{n}{kosdaq\PYGZus{}list} \PYG{o}{=} \PYG{n}{pd}\PYG{o}{.}\PYG{n}{read\PYGZus{}pickle}\PYG{p}{(}\PYG{l+s+s1}{\PYGZsq{}}\PYG{l+s+s1}{kosdaq\PYGZus{}list.pkl}\PYG{l+s+s1}{\PYGZsq{}}\PYG{p}{)}

\PYG{n}{data\PYGZus{}all\PYGZus{}6} \PYG{o}{=} \PYG{n}{pd}\PYG{o}{.}\PYG{n}{DataFrame}\PYG{p}{(}\PYG{p}{)}

\PYG{k}{for} \PYG{n}{code}\PYG{p}{,} \PYG{n}{sector} \PYG{o+ow}{in} \PYG{n+nb}{zip}\PYG{p}{(}\PYG{n}{kosdaq\PYGZus{}list}\PYG{p}{[}\PYG{l+s+s1}{\PYGZsq{}}\PYG{l+s+s1}{code}\PYG{l+s+s1}{\PYGZsq{}}\PYG{p}{]}\PYG{p}{,} \PYG{n}{kosdaq\PYGZus{}list}\PYG{p}{[}\PYG{l+s+s1}{\PYGZsq{}}\PYG{l+s+s1}{sector}\PYG{l+s+s1}{\PYGZsq{}}\PYG{p}{]}\PYG{p}{)}\PYG{p}{:}
    
    \PYG{c+c1}{\PYGZsh{} 종목별 처리}
    \PYG{n}{data} \PYG{o}{=} \PYG{n}{mdl\PYGZus{}data}\PYG{p}{[}\PYG{n}{mdl\PYGZus{}data}\PYG{p}{[}\PYG{l+s+s1}{\PYGZsq{}}\PYG{l+s+s1}{code}\PYG{l+s+s1}{\PYGZsq{}}\PYG{p}{]}\PYG{o}{==}\PYG{n}{code}\PYG{p}{]}\PYG{o}{.}\PYG{n}{sort\PYGZus{}index}\PYG{p}{(}\PYG{p}{)}\PYG{o}{.}\PYG{n}{copy}\PYG{p}{(}\PYG{p}{)}
    \PYG{n}{data}\PYG{o}{.}\PYG{n}{dropna}\PYG{p}{(}\PYG{n}{inplace}\PYG{o}{=}\PYG{k+kc}{True}\PYG{p}{)}
    
    \PYG{c+c1}{\PYGZsh{} 최근 60일 평균 수익율            }
    \PYG{n}{data}\PYG{p}{[}\PYG{l+s+s1}{\PYGZsq{}}\PYG{l+s+s1}{return\PYGZus{}mean}\PYG{l+s+s1}{\PYGZsq{}}\PYG{p}{]} \PYG{o}{=} \PYG{n}{data}\PYG{p}{[}\PYG{l+s+s1}{\PYGZsq{}}\PYG{l+s+s1}{return}\PYG{l+s+s1}{\PYGZsq{}}\PYG{p}{]}\PYG{o}{.}\PYG{n}{rolling}\PYG{p}{(}\PYG{l+m+mi}{60}\PYG{p}{)}\PYG{o}{.}\PYG{n}{mean}\PYG{p}{(}\PYG{p}{)} \PYG{c+c1}{\PYGZsh{} 종목별 최근 60 일 수익율의 평균}
    \PYG{n}{data}\PYG{p}{[}\PYG{l+s+s1}{\PYGZsq{}}\PYG{l+s+s1}{sector}\PYG{l+s+s1}{\PYGZsq{}}\PYG{p}{]} \PYG{o}{=} \PYG{n}{sector}     
  
    \PYG{n}{data}\PYG{o}{.}\PYG{n}{dropna}\PYG{p}{(}\PYG{n}{subset}\PYG{o}{=}\PYG{p}{[}\PYG{l+s+s1}{\PYGZsq{}}\PYG{l+s+s1}{return\PYGZus{}mean}\PYG{l+s+s1}{\PYGZsq{}}\PYG{p}{]}\PYG{p}{,} \PYG{n}{inplace}\PYG{o}{=}\PYG{k+kc}{True}\PYG{p}{)}    
    \PYG{n}{data\PYGZus{}all\PYGZus{}6} \PYG{o}{=} \PYG{n}{pd}\PYG{o}{.}\PYG{n}{concat}\PYG{p}{(}\PYG{p}{[}\PYG{n}{data}\PYG{p}{,} \PYG{n}{data\PYGZus{}all\PYGZus{}6}\PYG{p}{]}\PYG{p}{,} \PYG{n}{axis}\PYG{o}{=}\PYG{l+m+mi}{0}\PYG{p}{)}

\PYG{n}{data\PYGZus{}all\PYGZus{}6}\PYG{o}{.}\PYG{n}{to\PYGZus{}pickle}\PYG{p}{(}\PYG{l+s+s1}{\PYGZsq{}}\PYG{l+s+s1}{data\PYGZus{}all\PYGZus{}6.pkl}\PYG{l+s+s1}{\PYGZsq{}}\PYG{p}{)}   
\end{sphinxVerbatim}

\end{sphinxuseclass}\end{sphinxVerbatimInput}

\end{sphinxuseclass}
\begin{sphinxuseclass}{cell}\begin{sphinxVerbatimInput}

\begin{sphinxuseclass}{cell_input}
\begin{sphinxVerbatim}[commandchars=\\\{\}]
\PYG{n}{data\PYGZus{}all\PYGZus{}6} \PYG{o}{=} \PYG{n}{pd}\PYG{o}{.}\PYG{n}{read\PYGZus{}pickle}\PYG{p}{(}\PYG{l+s+s1}{\PYGZsq{}}\PYG{l+s+s1}{data\PYGZus{}all\PYGZus{}6.pkl}\PYG{l+s+s1}{\PYGZsq{}}\PYG{p}{)} 
\PYG{n}{data\PYGZus{}all\PYGZus{}6}\PYG{o}{.}\PYG{n}{head}\PYG{p}{(}\PYG{p}{)}\PYG{o}{.}\PYG{n}{style}\PYG{o}{.}\PYG{n}{set\PYGZus{}table\PYGZus{}attributes}\PYG{p}{(}\PYG{l+s+s1}{\PYGZsq{}}\PYG{l+s+s1}{style=}\PYG{l+s+s1}{\PYGZdq{}}\PYG{l+s+s1}{font\PYGZhy{}size: 12px}\PYG{l+s+s1}{\PYGZdq{}}\PYG{l+s+s1}{\PYGZsq{}}\PYG{p}{)}
\end{sphinxVerbatim}

\end{sphinxuseclass}\end{sphinxVerbatimInput}
\begin{sphinxVerbatimOutput}

\begin{sphinxuseclass}{cell_output}
\begin{sphinxVerbatim}[commandchars=\\\{\}]
\PYGZlt{}pandas.io.formats.style.Styler at 0x2188435c190\PYGZgt{}
\end{sphinxVerbatim}

\end{sphinxuseclass}\end{sphinxVerbatimOutput}

\end{sphinxuseclass}
\sphinxAtStartPar
최근 60 일 평균수익율 정보를 섹터 별, 일 별로 요약한 값을 추가합니다. 이때 apply 대신 Transform 함수가 이용되었습니다. apply 는 그룹의 숫자 만큼 행을 리턴하나, transform 은 그룹핑 하기 전의 행 수 를 리턴합니다. 그 값을 ‘return over sector’ 라는 변수에 저장합니다.

\begin{sphinxuseclass}{cell}\begin{sphinxVerbatimInput}

\begin{sphinxuseclass}{cell_input}
\begin{sphinxVerbatim}[commandchars=\\\{\}]
\PYG{n}{data\PYGZus{}all\PYGZus{}6}\PYG{p}{[}\PYG{l+s+s1}{\PYGZsq{}}\PYG{l+s+s1}{sector\PYGZus{}return}\PYG{l+s+s1}{\PYGZsq{}}\PYG{p}{]} \PYG{o}{=} \PYG{n}{data\PYGZus{}all\PYGZus{}6}\PYG{o}{.}\PYG{n}{groupby}\PYG{p}{(}\PYG{p}{[}\PYG{l+s+s1}{\PYGZsq{}}\PYG{l+s+s1}{sector}\PYG{l+s+s1}{\PYGZsq{}}\PYG{p}{,} \PYG{n}{data\PYGZus{}all\PYGZus{}6}\PYG{o}{.}\PYG{n}{index}\PYG{p}{]}\PYG{p}{)}\PYG{p}{[}\PYG{l+s+s1}{\PYGZsq{}}\PYG{l+s+s1}{return}\PYG{l+s+s1}{\PYGZsq{}}\PYG{p}{]}\PYG{o}{.}\PYG{n}{transform}\PYG{p}{(}\PYG{k}{lambda} \PYG{n}{x}\PYG{p}{:} \PYG{n}{x}\PYG{o}{.}\PYG{n}{mean}\PYG{p}{(}\PYG{p}{)}\PYG{p}{)}
\PYG{n}{data\PYGZus{}all\PYGZus{}6}\PYG{p}{[}\PYG{l+s+s1}{\PYGZsq{}}\PYG{l+s+s1}{return over sector}\PYG{l+s+s1}{\PYGZsq{}}\PYG{p}{]} \PYG{o}{=} \PYG{p}{(}\PYG{n}{data\PYGZus{}all\PYGZus{}6}\PYG{p}{[}\PYG{l+s+s1}{\PYGZsq{}}\PYG{l+s+s1}{return}\PYG{l+s+s1}{\PYGZsq{}}\PYG{p}{]}\PYG{o}{/}\PYG{n}{data\PYGZus{}all\PYGZus{}6}\PYG{p}{[}\PYG{l+s+s1}{\PYGZsq{}}\PYG{l+s+s1}{sector\PYGZus{}return}\PYG{l+s+s1}{\PYGZsq{}}\PYG{p}{]}\PYG{p}{)} \PYG{c+c1}{\PYGZsh{} 섹터의 평균 수익률 대비 종목 수익률의 비율}
\end{sphinxVerbatim}

\end{sphinxuseclass}\end{sphinxVerbatimInput}

\end{sphinxuseclass}
\sphinxAtStartPar
결과를 보니, 섹터를 이용하여 종목을 선정할 때는 섹터 평균 수익율보다 많이 높거나, 많이 낮는 종목을 선정하는 것이 수익율이 좋게 나왔습니다. 섹터 평균 수익율 대비 종목 수익율은 미래 수익율 예측에 도움이 되는 정보입니다.

\begin{sphinxuseclass}{cell}\begin{sphinxVerbatimInput}

\begin{sphinxuseclass}{cell_input}
\begin{sphinxVerbatim}[commandchars=\\\{\}]
\PYG{n}{pd}\PYG{o}{.}\PYG{n}{options}\PYG{o}{.}\PYG{n}{display}\PYG{o}{.}\PYG{n}{float\PYGZus{}format} \PYG{o}{=} \PYG{l+s+s1}{\PYGZsq{}}\PYG{l+s+si}{\PYGZob{}:,.3f\PYGZcb{}}\PYG{l+s+s1}{\PYGZsq{}}\PYG{o}{.}\PYG{n}{format}
\PYG{n}{ranks} \PYG{o}{=} \PYG{n}{pd}\PYG{o}{.}\PYG{n}{qcut}\PYG{p}{(}\PYG{n}{data\PYGZus{}all\PYGZus{}6}\PYG{p}{[}\PYG{l+s+s1}{\PYGZsq{}}\PYG{l+s+s1}{return over sector}\PYG{l+s+s1}{\PYGZsq{}}\PYG{p}{]}\PYG{p}{,} \PYG{n}{q}\PYG{o}{=}\PYG{l+m+mi}{10}\PYG{p}{)}
\PYG{n+nb}{print}\PYG{p}{(}\PYG{n}{data\PYGZus{}all\PYGZus{}6}\PYG{o}{.}\PYG{n}{groupby}\PYG{p}{(}\PYG{n}{ranks}\PYG{p}{)}\PYG{p}{[}\PYG{l+s+s1}{\PYGZsq{}}\PYG{l+s+s1}{max\PYGZus{}close}\PYG{l+s+s1}{\PYGZsq{}}\PYG{p}{]}\PYG{o}{.}\PYG{n}{describe}\PYG{p}{(}\PYG{n}{percentiles}\PYG{o}{=}\PYG{p}{[}\PYG{l+m+mf}{0.01}\PYG{p}{,} \PYG{l+m+mf}{0.99}\PYG{p}{]}\PYG{p}{)}\PYG{p}{)}
\PYG{n}{data\PYGZus{}all\PYGZus{}6}\PYG{o}{.}\PYG{n}{groupby}\PYG{p}{(}\PYG{n}{ranks}\PYG{p}{)}\PYG{p}{[}\PYG{l+s+s1}{\PYGZsq{}}\PYG{l+s+s1}{max\PYGZus{}close}\PYG{l+s+s1}{\PYGZsq{}}\PYG{p}{]}\PYG{o}{.}\PYG{n}{mean}\PYG{p}{(}\PYG{p}{)}\PYG{o}{.}\PYG{n}{plot}\PYG{p}{(}\PYG{n}{figsize}\PYG{o}{=}\PYG{p}{(}\PYG{l+m+mi}{12}\PYG{p}{,}\PYG{l+m+mi}{5}\PYG{p}{)}\PYG{p}{)}
\end{sphinxVerbatim}

\end{sphinxuseclass}\end{sphinxVerbatimInput}
\begin{sphinxVerbatimOutput}

\begin{sphinxuseclass}{cell_output}
\begin{sphinxVerbatim}[commandchars=\\\{\}]
                        count  mean   std   min    1\PYGZpc{}   50\PYGZpc{}   99\PYGZpc{}   max
return over sector                                                     
(0.378, 0.974]     34,292.000 1.042 0.086 0.700 0.920 1.022 1.384 2.968
(0.974, 0.984]     34,291.000 1.034 0.067 0.702 0.945 1.018 1.302 2.171
(0.984, 0.99]      34,292.000 1.030 0.061 0.700 0.949 1.016 1.278 2.330
(0.99, 0.994]      34,291.000 1.028 0.059 0.701 0.952 1.014 1.265 2.269
(0.994, 0.999]     34,291.000 1.027 0.058 0.708 0.954 1.013 1.256 2.729
(0.999, 1.002]     34,291.000 1.029 0.066 0.700 0.949 1.013 1.295 3.701
(1.002, 1.007]     34,292.000 1.027 0.059 0.700 0.951 1.012 1.255 3.027
(1.007, 1.013]     34,291.000 1.028 0.062 0.700 0.949 1.013 1.276 3.380
(1.013, 1.026]     34,291.000 1.031 0.067 0.326 0.944 1.014 1.307 2.412
(1.026, 1.399]     34,292.000 1.042 0.103 0.700 0.910 1.019 1.419 3.703
\end{sphinxVerbatim}

\begin{sphinxVerbatim}[commandchars=\\\{\}]
\PYGZlt{}AxesSubplot:xlabel=\PYGZsq{}return over sector\PYGZsq{}\PYGZgt{}
\end{sphinxVerbatim}

\noindent\sphinxincludegraphics{{4.4.4_Data_Processing_34_2}.png}

\end{sphinxuseclass}\end{sphinxVerbatimOutput}

\end{sphinxuseclass}
\sphinxAtStartPar
한 섹터에 최소 10 개 이상의 종목이 있어야 섹터의 평균 수익율이 의미가 있을 것 같습니다. 10개 이상의 종목이 있는 섹터만을 매수 대상으로 해서 다시 수익율을 계산해봅니다. 같은 결과를 얻었습니다. 섹터의 평균 수익율보다 아주 낮거나 높은 종목의 수익율의 상승이 높았습니다. 그래프 곡선이 더 부드러워졌습니다.

\begin{sphinxuseclass}{cell}\begin{sphinxVerbatimInput}

\begin{sphinxuseclass}{cell_input}
\begin{sphinxVerbatim}[commandchars=\\\{\}]
\PYG{n}{sector\PYGZus{}count} \PYG{o}{=} \PYG{n}{data\PYGZus{}all\PYGZus{}6}\PYG{o}{.}\PYG{n}{groupby}\PYG{p}{(}\PYG{l+s+s1}{\PYGZsq{}}\PYG{l+s+s1}{sector}\PYG{l+s+s1}{\PYGZsq{}}\PYG{p}{)}\PYG{p}{[}\PYG{l+s+s1}{\PYGZsq{}}\PYG{l+s+s1}{code}\PYG{l+s+s1}{\PYGZsq{}}\PYG{p}{]}\PYG{o}{.}\PYG{n}{nunique}\PYG{p}{(}\PYG{p}{)}\PYG{o}{.}\PYG{n}{sort\PYGZus{}values}\PYG{p}{(}\PYG{p}{)}
\PYG{n}{data\PYGZus{}all\PYGZus{}6x} \PYG{o}{=} \PYG{n}{data\PYGZus{}all\PYGZus{}6}\PYG{p}{[}\PYG{n}{data\PYGZus{}all\PYGZus{}6}\PYG{p}{[}\PYG{l+s+s1}{\PYGZsq{}}\PYG{l+s+s1}{sector}\PYG{l+s+s1}{\PYGZsq{}}\PYG{p}{]}\PYG{o}{.}\PYG{n}{isin}\PYG{p}{(}\PYG{n}{sector\PYGZus{}count}\PYG{p}{[}\PYG{n}{sector\PYGZus{}count}\PYG{o}{\PYGZgt{}}\PYG{o}{=}\PYG{l+m+mi}{10}\PYG{p}{]}\PYG{o}{.}\PYG{n}{index}\PYG{p}{)}\PYG{p}{]}
\PYG{n}{ranks} \PYG{o}{=} \PYG{n}{pd}\PYG{o}{.}\PYG{n}{qcut}\PYG{p}{(}\PYG{n}{data\PYGZus{}all\PYGZus{}6x}\PYG{p}{[}\PYG{l+s+s1}{\PYGZsq{}}\PYG{l+s+s1}{return over sector}\PYG{l+s+s1}{\PYGZsq{}}\PYG{p}{]}\PYG{p}{,} \PYG{n}{q}\PYG{o}{=}\PYG{l+m+mi}{10}\PYG{p}{)}
\PYG{n+nb}{print}\PYG{p}{(}\PYG{n}{data\PYGZus{}all\PYGZus{}6x}\PYG{o}{.}\PYG{n}{groupby}\PYG{p}{(}\PYG{n}{ranks}\PYG{p}{)}\PYG{p}{[}\PYG{l+s+s1}{\PYGZsq{}}\PYG{l+s+s1}{max\PYGZus{}close}\PYG{l+s+s1}{\PYGZsq{}}\PYG{p}{]}\PYG{o}{.}\PYG{n}{describe}\PYG{p}{(}\PYG{n}{percentiles}\PYG{o}{=}\PYG{p}{[}\PYG{l+m+mf}{0.01}\PYG{p}{,} \PYG{l+m+mf}{0.99}\PYG{p}{]}\PYG{p}{)}\PYG{p}{)}
\PYG{n}{data\PYGZus{}all\PYGZus{}6x}\PYG{o}{.}\PYG{n}{groupby}\PYG{p}{(}\PYG{n}{ranks}\PYG{p}{)}\PYG{p}{[}\PYG{l+s+s1}{\PYGZsq{}}\PYG{l+s+s1}{max\PYGZus{}close}\PYG{l+s+s1}{\PYGZsq{}}\PYG{p}{]}\PYG{o}{.}\PYG{n}{mean}\PYG{p}{(}\PYG{p}{)}\PYG{o}{.}\PYG{n}{plot}\PYG{p}{(}\PYG{n}{figsize}\PYG{o}{=}\PYG{p}{(}\PYG{l+m+mi}{12}\PYG{p}{,}\PYG{l+m+mi}{5}\PYG{p}{)}\PYG{p}{)}
\end{sphinxVerbatim}

\end{sphinxuseclass}\end{sphinxVerbatimInput}
\begin{sphinxVerbatimOutput}

\begin{sphinxuseclass}{cell_output}
\begin{sphinxVerbatim}[commandchars=\\\{\}]
                        count  mean   std   min    1\PYGZpc{}   50\PYGZpc{}   99\PYGZpc{}   max
return over sector                                                     
(0.688, 0.973]     26,887.000 1.042 0.087 0.700 0.918 1.022 1.388 2.968
(0.973, 0.983]     26,886.000 1.034 0.067 0.702 0.944 1.019 1.299 2.171
(0.983, 0.989]     26,886.000 1.030 0.060 0.704 0.948 1.016 1.267 2.286
(0.989, 0.994]     26,886.000 1.028 0.060 0.700 0.952 1.014 1.269 2.194
(0.994, 0.998]     26,889.000 1.027 0.058 0.819 0.954 1.013 1.255 2.729
(0.998, 1.002]     26,883.000 1.026 0.059 0.700 0.954 1.012 1.257 2.652
(1.002, 1.007]     26,886.000 1.027 0.062 0.700 0.950 1.012 1.257 3.027
(1.007, 1.014]     26,886.000 1.028 0.062 0.700 0.949 1.013 1.285 3.380
(1.014, 1.027]     26,886.000 1.031 0.068 0.700 0.943 1.014 1.315 2.412
(1.027, 1.399]     26,887.000 1.043 0.105 0.700 0.909 1.019 1.423 3.703
\end{sphinxVerbatim}

\begin{sphinxVerbatim}[commandchars=\\\{\}]
\PYGZlt{}AxesSubplot:xlabel=\PYGZsq{}return over sector\PYGZsq{}\PYGZgt{}
\end{sphinxVerbatim}

\noindent\sphinxincludegraphics{{4.4.4_Data_Processing_36_2}.png}

\end{sphinxuseclass}\end{sphinxVerbatimOutput}

\end{sphinxuseclass}






\renewcommand{\indexname}{Index}
\printindex
\end{document}